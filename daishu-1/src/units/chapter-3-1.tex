\subsection{两角和与差的三角函数}\label{subsec:3-1} % 原书的目录结构就是如此(缺少 section),所以忽略此处的告警

\subsubsection{两角和与差的余弦}

研究两角和与差的三角函数的问题,就是研究怎样利用角 $\alpha$ 与 $\beta$ 的三角函数,表示 $\alpha \pm \beta$
的三角函数。关于 $\cos(\alpha + \beta)$ 怎样用$\alpha$,$\beta$的三角函数来表示,我们有下面的重要公式\footnote{
    $C_{\alpha + \beta}$ 是简记符号,用来表示公式 \\
    \indent \hspace{3em} $\cos(\alpha + \beta) = \cos\alpha \cos\beta - \sin\alpha \sin\beta \text{。}$ \\
    下面还有其他一些简记符号。
}:
\begin{gather}
    \cos(\alpha + \beta) = \cos\alpha \cos\beta - \sin\alpha \sin\beta \text{。} \tag{$C_{\alpha + \beta}$}\label{eq:C-a+b}
\end{gather}

\begin{wrapfigure}[16]{r}{7.0cm}
    \centering
    \begin{tikzpicture}[>=Stealth, scale=0.8, every node/.style = {inner sep = 1pt}]
    \pgfmathsetmacro{\r}{3.5}
    \pgfmathsetmacro{\axis}{\r + 1.0}
    \pgfmathsetmacro{\a}{60}
    \pgfmathsetmacro{\b}{45}

    \draw [->] (-\axis,0) -- (\axis,0) node[anchor=north, inner sep = 5pt] {$x$};
    \draw [->] (0,-\axis) -- (0,\axis) node[anchor=east, inner sep = 5pt] {$y$};
    \node at (-0.3,-0.3) {$O$};
    \draw [name path=c] (0, 0) circle(\r);

    \coordinate (O) at (0,0);
    \node [label=60:$P_1$] (P1) at (0:\r) {};
    \node [label=20:$P_2$] (P2) at (\a:\r) {};
    \node [label=120:$P_3$] (P3) at (\a + \b:\r) {};
    \node [label=-60:$P_4$] (P4) at (-\b:\r) {};

    \draw (P2) -- (O) -- (P4);
    \draw[dashed] (P2) -- (P4);
    \draw (P3) -- (O);
    \draw[dashed] (P3) -- (P1);

    \draw (0.8, 0.4) node {$\alpha$} [->] (0.6, 0) arc (0:\a:0.6);
    \draw (0.3, 1.1) node {$\beta$} [->] (\a:0.8) arc (\a:\a+\b:0.8);
    \draw (1.5, 1.4) node[fill=white, inner sep=0pt] {$\alpha + \beta$} [->] (0:1.5) arc (0:\a+\b:1.5);
    \draw (1.2, -0.5) node {$-\beta$} [->] (0:0.8) arc (0:-\b:0.8);
\end{tikzpicture}

    \vspace{-2em}
    \caption{}\label{fig:3-1}
\end{wrapfigure}

\zhengming 在直角坐标系 $xOy$ 内,作单位圆 $O$,并作 $\alpha$,$\beta$ 和 $-\beta$ 角;使
$\alpha$ 角的始边为 $Ox$,交圆 $O$ 于 $P_1$,终边交圆 $O$ 于 $P_2$;
$\beta$ 角的始边为 $OP_2$,终边交圆 $O$ 于 $P_3$;
$-\beta$ 角的始边为 $OP_1$,终边交圆 $O$ 于$P_4$(图\ref{fig:3-1})。
这时,$P_1$,$P_2$,$P_3$,$P_4$ 的坐标分别是:

$P_1(1, \, 0);$

$P_2(\cos\alpha, \, \sin\alpha);$

$P_3(\cos(\alpha + \beta), \, \sin(\alpha + \beta));$

$P_4(\cos(-\beta), \, \sin(-\beta)).$

由 $|P_1P_3| = |P_2P_4|$ 及两点间距离公式,得

\vspace{-1.5em}
\begin{align*}
      & [\cos(\alpha + \beta) - 1]^2 + \sin^2(\alpha + \beta) \\
    = & [\cos(-\beta) - \cos\alpha]^2 + [\sin(-\beta) - \sin\alpha]^2,
\end{align*}

展开,整理得

\begin{minipage}{10cm}
    \begin{align*}
        & 2 - 2\cos(\alpha + \beta) = 2 - 2(\cos\alpha \cos\beta - \sin\alpha \sin\beta) \text{。}\\
        & \therefore \quad \cos(\alpha + \beta) = \cos\alpha \cos\beta - \sin\alpha \sin\beta \text{。}
    \end{align*}
\end{minipage}

上面的公式,对于任意的角 $\alpha$ 和 $\beta$ 都成立。

在上面的公式中,用 $-\beta$ 代替 $\beta$,就得到
$$\cos(\alpha - \beta) = \cos\alpha \cos(-\beta) - \sin\alpha \sin(-\beta) \text{,}$$
即
\vspace{-1.7em}
\begin{gather}
    \cos(\alpha - \beta) = \cos\alpha \cos\beta + \sin\alpha \sin\beta \text{。} \tag{$C_{\alpha - \beta}$}\label{eq:C-a-b}
\end{gather}

\liti 不查表,求 $\cos 105^\circ$ 及 $\cos 15^\circ$ 的值。

\jie $\begin{aligned}[t]
        \cos 105^\circ &= \cos(60^\circ + 45^\circ) \\
                       &= \cos 60^\circ \cos 45^\circ - \sin 60^\circ \sin 45^\circ \\
                       &= \dfrac 1 2 \cdot \dfrac{\sqrt 2}{2} - \dfrac{\sqrt 3}{2} \cdot \dfrac{\sqrt 2}{2} = \dfrac{\sqrt 2 - \sqrt 6}{4};
\end{aligned}$

\newpage % TODO: 为了防止将公式挤得变形,手工分页。

\hspace{2em} $\begin{aligned}[t]
    \cos 15^\circ &= \cos(45^\circ - 30^\circ) \\
        &= \cos 45^\circ \cos 30^\circ + \sin 45^\circ \sin 30^\circ \\
        &= \dfrac{\sqrt 2}{2} \cdot \dfrac{\sqrt 3}{2} + \dfrac{\sqrt 2}{2} \cdot \dfrac{1}{2} = \dfrac{\sqrt 6 + \sqrt 2}{4} .\\
\end{aligned}$

\liti 已知 $\sin\alpha = \dfrac 2 3$,$\alpha \in \left( \dfrac \pi 2, \, \pi \right)$,$\cos\beta = -\dfrac 3 4$,
$\beta \in \left( \pi, \, \dfrac{3\pi}{2} \right)$,求 $\cos(\alpha - \beta)$ 的值。

\jie 由 $\sin\alpha = \dfrac 2 3$,$\alpha \in \left( \dfrac \pi 2, \, \pi \right)$ 得
$$\cos\alpha = -\sqrt{1 - \sin^2 \alpha} = -\sqrt{1 - \left( \dfrac 2 3 \right)^2} = -\dfrac{\sqrt 5}{3};$$
又由 $\cos\beta = -\dfrac 3 4$,$\beta \in \left( \pi, \, \dfrac{3\pi}{2} \right)$,得
$$\sin\beta = -\sqrt{1 - \cos^2 \beta} = -\sqrt{1 - \left( -\dfrac 3 4 \right)^2} = -\dfrac{\sqrt 7}{4} \text{。}$$

$\therefore$ \begin{minipage}[t]{9cm}
    \vspace{-1.7em}
    \begin{align*}
        \cos(\alpha - \beta) &= \cos\alpha \cos\beta + \sin\alpha \sin\beta \\
            &= \left( -\dfrac{\sqrt 5}{3} \right) \cdot \left( -\dfrac 3 4 \right) + \dfrac 2 3 \cdot \left( -\dfrac{\sqrt 7}{4} \right) \\
            &= \dfrac{3\sqrt 5 - 2\sqrt 7}{12} \text{。}
    \end{align*}
\end{minipage}

\jiange
\liti 证明:公式
$$\cos\left( \dfrac \pi 2 - \alpha \right) = \sin\alpha , \jiange$$
$$\sin\left( \dfrac \pi 2 - \alpha \right) = \cos\alpha , \jiange$$
当 $\alpha$ 为任意角时仍然成立。

\zhengming 利用公式 (\ref{eq:C-a-b}) ,可得

\begin{minipage}{20em}
    \begin{align*}
        & \cos\left( \dfrac \pi 2 - \alpha \right) = \cos\dfrac \pi 2 \cos\alpha + \sin\dfrac \pi 2 \sin\alpha,\\
        \because \quad & \cos\dfrac \pi 2 = 0, \quad \sin\dfrac \pi 2 = 1, \\
        \therefore \quad & \cos\left( \dfrac \pi 2 - \alpha \right) = \sin\alpha.
    \end{align*}
\end{minipage}

因为上式中的 $\alpha$ 为任意角,如果把 $\left( \dfrac \pi 2 - \alpha \right)$ 换成 $\alpha$,就得
$$\cos\alpha = \sin\left( \dfrac \pi 2 - \alpha \right), \jiange $$
即
\vspace{-1.2em}$$\sin\left( \dfrac \pi 2 - \alpha \right) = \cos\alpha . \jiange$$

利用上述两式,不难证明下面两式在两边都有意义时成立:
$$\tan\left( \dfrac \pi 2 - \alpha \right) = \cot\alpha ; \jiange$$
$$\cot\left( \dfrac \pi 2 - \alpha \right) = \tan\alpha . \jiange$$

以上四个公式是当 $\alpha$ 为任意角时 $\left( \dfrac \pi 2 - \alpha \right)$ 的诱导公式。如果把其中的 $\alpha$ 换成 $(-\alpha)$,
就可得到当 $\alpha$ 为任意角时 $\left( \dfrac \pi 2 + \alpha \right)$ 的诱导公式:

\begin{center}
    \framebox{
        \renewcommand\arraystretch{2}
        \begin{tabular}{p{13em}p{13em}}
            $\cos\left( \dfrac \pi 2 + \alpha \right) = -\sin\alpha$; & $\sin\left( \dfrac \pi 2 + \alpha \right) = \cos\alpha$;\\
            $\tan\left( \dfrac \pi 2 + \alpha \right) = -\cot\alpha$; & $\cot\left( \dfrac \pi 2 + \alpha \right) = -\tan\alpha$。
    \end{tabular}}
\end{center}

\lianxi
\begin{xiaotis}

\xiaoti{等式 $\cos(\alpha + \beta) = \cos\alpha + \cos\beta$ 成立吗?用 $\alpha = 60^\circ$,$\beta = 30^\circ$ 代入进行检验。}

\xiaoti{不查表,求下列各式的值:}
\begin{xiaoxiaotis}

    \threeInLineXxt{$\cos 75^\circ$;}{$\cos 165^\circ$;}{$\cos\left( -\dfrac{61\pi}{12} \right)$。}

\end{xiaoxiaotis}

\xiaoti{已知 $\sin\alpha = \dfrac{15}{17}$,$\alpha \in \left( \dfrac \pi 2, \, \pi \right)$,求 $\cos\left( \dfrac \pi 3 - \alpha \right)$ 的值。}
\jiange

\xiaoti{已知 $\cos\theta = -\dfrac{5}{13}$,$\theta \in \left( \pi, \, \dfrac 3 2 \pi \right)$,求 $\cos\left( \theta + \dfrac \pi 6 \right)$ 的值。}
\jiange

\xiaoti{把下列各三角函数化成 $0^\circ$ \~{} $45^\circ$ 的角的三角函数:}
\begin{xiaoxiaotis}

    \threeInLineXxt{$\cos 1856^\circ$;}{$\sin(-1190^\circ)$;}{$\cot(-310^\circ)$。}

\end{xiaoxiaotis}

\xiaoti{不查表,求下列各式的值:}
\begin{xiaoxiaotis}

    \xiaoxiaoti{$\cos 80^\circ \cos 20^\circ + \sin 80^\circ \sin 20^\circ$;}

    \xiaoxiaoti{$\cos^2 15^\circ - \sin^2 15^\circ$。}

\end{xiaoxiaotis}

\end{xiaotis}

\subsubsection{两角和与差的正弦}

因为 $\sin(\alpha + \beta) = \cos\left[ \dfrac \pi 2 - (\alpha + \beta) \right]$,\jiange \\
而 $\begin{aligned}[t]
      & \cos\left[ \dfrac \pi 2 - (\alpha + \beta) \right] =  \cos\left[ (\dfrac \pi 2 - \alpha) - \beta) \right] \\
    = & \cos\left( \dfrac \pi 2 - \alpha \right) \cos\beta + \sin\left( \dfrac \pi 2 - \alpha \right) \sin\beta \\
    = & \sin\alpha \cos\beta + \cos\alpha \sin\beta ,
\end{aligned}$ \\
所以
\begin{minipage}[t]{0.9\textwidth}
    \vspace{-1.7em}\begin{gather}
        \sin(\alpha + \beta) = \sin\alpha \cos\beta + \cos\alpha \sin\beta \text{。} \tag{$S_{\alpha + \beta}$}\label{eq:S-a+b}
    \end{gather}
\end{minipage}

把公式 (\ref{eq:S-a+b}) 中的 $\beta$ 换成 $-\beta$,得
$$\sin(\alpha - \beta) = \sin\alpha \cos(-\beta) + \cos\alpha \sin(-\beta),$$
即\hspace{1em}
\begin{minipage}[t]{0.9\textwidth}
    \vspace{-1.7em}\begin{gather}
        \sin(\alpha - \beta) = \sin\alpha \cos\beta - \cos\alpha \sin\beta \text{。} \tag{$S_{\alpha - \beta}$}\label{eq:S-a-b}
    \end{gather}
\end{minipage}

\liti 不查表,求 $\sin 75^\circ$ 的值。

\jie $\begin{aligned}[t]
    \sin 75^\circ &= \sin(45^\circ + 30^\circ) \\
        &= \sin 45^\circ \cos 30^\circ + \cos 45^\circ \sin 30^\circ \\
        &= \dfrac{\sqrt 2}{2} \cdot \dfrac{\sqrt 3}{2} + \dfrac{\sqrt 2}{2} \cdot \dfrac 1 2 = \dfrac{\sqrt 6 + \sqrt 2}{4} \text{。}
\end{aligned}$

\jiange
\liti 已知 $\cos\varphi = \dfrac 3 5$,$\varphi \in \left( 0, \, \dfrac \pi 2 \right)$,求 $\sin\left( \varphi - \dfrac \pi 6 \right)$。

\jiange
\jie $\because \quad \cos\varphi = \dfrac 3 5$,$\varphi \in \left( 0, \, \dfrac \pi 2 \right)$,
\jiange

$\therefore \quad \sin\varphi = \sqrt{1 - \left( \dfrac 3 5 \right)^2} = \dfrac 4 5$。
\jiange

$\therefore \quad \begin{aligned}[t]
    \sin\left( \varphi - \dfrac \pi 6 \right) &= \sin\varphi \cos\dfrac \pi 6 - \cos\varphi \sin\dfrac \pi 6 \\
        &= \dfrac 4 5 \cdot \dfrac{\sqrt 3}{2} - \dfrac 3 5 \cdot \dfrac 1 2 = \dfrac{4\sqrt 3 - 3}{10} \text{。}
\end{aligned}$

\jiange
\liti 求证 $\dfrac{\sin(\alpha + \beta) \sin(\alpha - \beta)}{\sin^2 \alpha \cos^2 \beta} = 1 - \cot^2 \alpha \tan^2 \beta$。

\zhengming $\begin{aligned}[t]
      & \text{左边} \\
    = & \dfrac{(\sin\alpha \cos\beta + \cos\alpha \sin\beta)(\sin\alpha \cos\beta - \cos\alpha \sin\beta)}{\sin^2 \alpha \cos^2 \beta} \\
    = & \dfrac{\sin^2 \alpha \cos^2 \beta - \cos^2 \alpha \sin^2 \beta}{\sin^2 \alpha \cos^2 \beta} = 1 - \dfrac{\cos^2 \alpha \sin^2 \beta}{\sin^2 \alpha \cos^2 \beta} \\
    = & 1 - \cot^2 \alpha \tan^2 \beta = \text{右边。}
\end{aligned}$

$\therefore$ \quad 原式成立。

\liti 求证 $\cos\alpha + \sqrt 3 \sin\alpha = 2\sin\left( \dfrac \pi 6 + \alpha \right)$。

\textbf{证法一:} $\begin{aligned}[t]
    \text{左边} &= 2\left( \dfrac 1 2 \cos\alpha + \dfrac{\sqrt 3}{2} \sin\alpha \right) \\
        &= 2\left( \sin\dfrac \pi 6 \cos\alpha + \cos\dfrac \pi 6 \sin\alpha \right) \\
        &= 2\sin\left( \dfrac \pi 6 + \alpha \right) = \text{右边。}
\end{aligned}$

$\therefore$ \quad 原式成立。

\textbf{证法二:} $\begin{aligned}[t]
    \text{右边} &= 2\left( \sin\dfrac \pi 6 \cos\alpha + \cos \dfrac \pi 6 \sin\alpha \right) \\
        &= 2\left( \dfrac 1 2 \cos\alpha + \dfrac{\sqrt 3}{2} \sin\alpha \right) \\
        &= \cos\alpha + \sqrt 3 \sin\alpha = \text{左边。}
\end{aligned}$

$\therefore$ \quad 原式成立。

\lianxi
\begin{xiaotis}

\xiaoti{等式 $\sin(\alpha + \beta) = \sin\alpha + \sin\beta$ 成立吗?用 $\alpha = 60^\circ$,$\beta = 30^\circ$ 代入进行检验。}

\xiaoti{不查表,求下列各式的值:}
\begin{xiaoxiaotis}

    \threeInLineXxt{$\sin 105^\circ$;}{$\sin 15^\circ$;}{$\sin\left( -\dfrac{5\pi}{12} \right)$。}
    \jiange

\end{xiaoxiaotis}


\xiaoti{已知 $\cos\theta = \dfrac 3 5$,$\theta \in \left( \dfrac \pi 2, \, \pi \right)$,求 $\sin\left( \theta + \dfrac \pi 3 \right)$ 的值。}
\jiange

\xiaoti{已知 $\sin\alpha = \dfrac 2 3$,$\cos\beta = -\dfrac 3 4$,且 $\alpha$,$\beta$ 都是第二象限角,求 $\sin(\alpha - \beta)$ 及 $\cos(\alpha - \beta)$。}
\jiange

\xiaoti{不查表,求下列各式的值:}
\begin{xiaoxiaotis}

    \xiaoxiaoti{$\sin 13^\circ \cos 17^\circ + \cos 13^\circ \sin 17^\circ$;}

    \xiaoxiaoti{$\sin 70^\circ \cos 25^\circ - \sin 20^\circ \sin 25^\circ$。}

\end{xiaoxiaotis}

\xiaoti{求证:}
\begin{xiaoxiaotis}

    \xiaoxiaoti{$\dfrac 1 2 \left( \cos\alpha + \sqrt 3 \sin\alpha \right) = \cos(60^\circ - \alpha)$;}

    \jiange
    \xiaoxiaoti{$\sin\left( \dfrac{5\pi}{6} - \varphi \right) + \sin\left( \dfrac{5\pi}{6} + \varphi \right) = \cos\varphi$。}
    \jiange

\end{xiaoxiaotis}

\end{xiaotis}

\subsubsection{两角和与差的正切}

由 $\tan(\alpha + \beta) = \dfrac{\sin(\alpha + \beta)}{\cos(\alpha + \beta)} = \dfrac{\sin\alpha \cos\beta + \cos\alpha \sin\beta}{\cos\alpha \cos\beta - \sin\alpha \sin\beta}$,
把最后一个分式的分子、分母分别除以 $\cos\alpha \cdot \cos\beta \, (\cos\alpha \neq 0,\, \cos\beta \neq 0)$,得
\begin{gather}
\tan(\alpha + \beta) = \dfrac{\tan\alpha + \tan\beta}{1 - \tan\alpha \tan\beta} \text{。} \tag{$T_{\alpha + \beta}$}\label{eq:T-a+b}
\end{gather}

\jiange
把公式 (\ref{eq:T-a+b}) 中的 $\beta$ 换成 $-\beta$,得
\begin{gather}
\tan(\alpha - \beta) = \dfrac{\tan\alpha - \tan\beta}{1 + \tan\alpha \tan\beta} \text{。} \tag{$T_{\alpha - \beta}$}\label{eq:T-a-b}
\end{gather}

注意:在两角和与差的正切的公式中,$\alpha$,$\beta$ 的取值范围,应该是使 $\tan\alpha$,$\tan\beta$ 及 $\tan(\alpha \pm \beta)$
都存在的那些值,即 $\alpha$,$\beta$ 及 $\alpha \pm \beta$ 都不能取 $\dfrac \pi 2 + n\pi \, (n \in Z)$ 。例如,如果
$\alpha = \dfrac \pi 2$,$\beta = \dfrac \pi 3$,那么求 $\tan(\alpha + \beta)$ 的值,就不能用和角的正切公式,而应该用诱导公式。

\jiange
\liti 已知 $\tan\alpha = \dfrac 1 3$,$\tan\beta = -2$。\jiange
\begin{xiaoxiaotis}

    \xiaoxiaoti{求 $\cot(\alpha - \beta)$;}

    \xiaoxiaoti{求 $\alpha + \beta$ 的值(其中 $0^\circ < \alpha < 90^\circ$,$90^\circ < \beta < 180^\circ$)。}

\end{xiaoxiaotis}

\jie (1)$\because$ \vspace{-1.5em}$$\tan\alpha = \dfrac 1 3 \text{,} \tan\beta = -2 \text{,}$$
而
\vspace{-1.5em}$$
    \cot(\alpha - \beta) = \dfrac{1}{\tan(\alpha - \beta)}, \vspace{1.5em}
$$
其中
\vspace{-2.5em}$$
    \tan(\alpha - \beta) = \dfrac{\tan\alpha - \tan\beta}{1 + \tan\alpha \tan\beta}
        = \dfrac{\dfrac 1 3 + 2}{1 + \dfrac 1 3 \times (-2)} = 7 \text{。}
$$

$\therefore \quad \cot(\alpha - \beta) = \dfrac 1 7$。\jiange

(2)由 $\tan\alpha = \dfrac 1 3$,$\tan\beta = -2$,得

$$
\tan(\alpha + \beta) = \dfrac{\tan\alpha + \tan\beta}{1 - \tan\alpha \tan\beta}
        = \dfrac{\dfrac 1 3 - 2}{1 - \dfrac 1 3 \times (-2)} = -1 \text{;}
$$
又因 $0^\circ < \alpha < 90^\circ$,$90^\circ < \beta < 180^\circ$,所以
$$ 90^\circ < \alpha + \beta < 270^\circ \text{。}$$

在 $90^\circ$ 与 $270^\circ$ 之间,只有 $135^\circ$ 的正切的值为 $-1$,

$\therefore \quad \alpha + \beta = 135^\circ \text{。}$

\jiange
\liti 计算 $\dfrac{1 + \tan 75^\circ}{1 - \tan 75^\circ}$ 的值。
\jiange

分析:因为 $\tan 45^\circ = 1$,所以原式可以看成是 \jiange
$$\dfrac{\tan 45^\circ + \tan 75^\circ}{1 - \tan 45^\circ \tan 75^\circ} \text{。} \jiange $$
这样,我们就可以运用两角和的正切公式,把原式化为
$$\tan(45^\circ + 75^\circ) \text{,}$$
而因 $45^\circ + 75^\circ = 120^\circ$ 是特殊角,所以可以求得原式的值。

\jie $\because \quad \tan 45^\circ = 1$,\jiange

$\therefore \quad \begin{aligned}[t]
    \dfrac{1 + \tan 75^\circ}{1 - \tan 75^\circ} &= \dfrac{\tan 45^\circ + \tan 75^\circ}{1 - \tan 45^\circ \tan 75^\circ} \\
        &= \tan(45^\circ + 75^\circ) \\
        &= \tan 120^\circ = -\sqrt 3 \text{。}
\end{aligned}$

\liti 设 $\tan\alpha$,$\tan\beta$ 是一元二次方程 $ax^2 + bx + c = 0 \, (b \neq 0)$ 的两个根,求 $\cot(\alpha + \beta)$ 的值。

\jie 在一元二次方程 $ax^2 + bx + c = 0$ 中,$a \neq 0$。由一元二次方程根与系数的关系,得
$$\begin{gathered}
    \tan\alpha + \tan\beta = -\dfrac b a \text{,} \\
    \tan\alpha \tan\beta = \dfrac c a \text{。} \jiange \jiange
\end{gathered}$$
 而 \vspace{-1.5em}$$ \cot(\alpha + \beta) = \dfrac{1}{\tan(\alpha + \beta)} = \dfrac{1 - \tan\alpha \tan\beta}{\tan\alpha + \tan\beta} \text{。}$$

由题设 $b \neq 0$,故 $\tan\alpha + \tan\beta \neq 0$,代入,得
$$\cot(\alpha + \beta) = \dfrac{1 - \dfrac c a}{-\dfrac b a} = \dfrac{a - c}{-b} = \dfrac{c - a}{b} \text{。}$$

\lianxi
\begin{xiaotis}

\xiaoti{不查表,求下列各式的值:}
\begin{xiaoxiaotis}

    \threeInLineXxt{$\tan 75^\circ$;}{$\tan 15^\circ$;}{$\cot 105^\circ$。}
    \jiange

\end{xiaoxiaotis}

\xiaoti{不查表,求下列各式的值:}
\begin{xiaoxiaotis}

    \jiange
    \twoInLineXxt[14em]{$\dfrac{\tan 12^\circ + \tan 33^\circ}{1 - \tan 12^\circ \tan 33^\circ}$;}{$\dfrac{1 - \tan 15^\circ}{1 + \tan 15^\circ}$。}
    \jiange

\end{xiaoxiaotis}

\xiaoti{已知 $\tan\alpha = 2$,求 $\tan\left( \alpha - \dfrac \pi 4 \right)$ 的值。\jiange}

\xiaoti{已知 $\tan\alpha = 2$,$\tan\beta = 3$,并且 $\alpha$,$\beta$ 都是锐角,求证:
    $$\alpha + \beta = 135^\circ \text{。}$$
}

\end{xiaotis}