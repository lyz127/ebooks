\subsection{函数 \texorpdfstring{$y = A \sin(\omega x + \varphi)$}{y=A sin(o x+ v)} 的图象}\label{subsec:2-9}

在物理和工程技术的许多问题中,都要遇到形如 $y = A \sin(\omega x + \varphi)$ 的函数(其中,$A$,$\omega$,$\varphi$ 是常数)。
例如;物体作简谐振动时位移 $y$ 与时间 $x$ 的关系,交流电中电流强度 $y$ 与时间 $x$ 的关系等,
都可用这类函数来表示。下面来讨论这类函数的简图的作法。

\vspace{0.5em}
\liti 作函数 $y = 2\sin x$ 及 $y = \dfrac 1 2 \sin x$ 的简图。
\vspace{0.5em}

\jie 函数 $y = 2\sin x$ 及 $y = \dfrac 1 2 \sin x$ 的周期 $T = 2\pi$,我们先来作 $x \in [0, 2\pi]$ 时函数的简图。

列表:

\begin{table}[H]
\renewcommand\arraystretch{2}
\begin{tabular}{|w{c}{8em}|*{5}{w{c}{3em}|}}
    \hline
    $x$ & $0$ & $\dfrac \pi 2$ & $\pi$ & $\dfrac{3\pi}{2}$ & $2\pi$ \\ \hline
    $\sin x$ & $0$ & $1$ & $0$ & $-1$ & $0$ \\ \hline
    $2\sin x$ & $0$ & $2$ & $0$ & $-2$ & $0$ \\ \hline
    $\dfrac 1 2 \sin x$ & $0$ & $\dfrac 1 2$ & $0$ & $-\dfrac 1 2$ & $0$ \\ \hline
\end{tabular}
\end{table}

描点作图(图 \ref{fig:2-23}):

\begin{figure}[htbp]
    \centering
    \begin{tikzpicture}[>=Stealth]
    \draw [->] (-1.0,0) -- (2*pi +1,0) node[anchor=north] {$x$};
    \draw [->] (0,-2.5) -- (0,2.5) node[anchor=east] {$y$};
    \node at (0.3,-0.3) {$O$};

    \draw(0.5*pi, 0.2) -- (0.5*pi,0) node[anchor=north] {$\dfrac \pi 2$};
    \draw(pi, 0.2) -- (pi,0) node[anchor=north east] {$\pi$};
    \foreach \x / \name in {
        1.5*pi / $\dfrac {3\pi} 2$,
        2 * pi / $2\pi$
    } {
        \draw (\x,0) -- (\x,0.2) node[anchor=south] {\name};
    }

    \foreach \y in {-2,-1,1,2} {
        \draw (0.2,\y) -- (0,\y) node[anchor=east] {\y};
    }
    \foreach \y / \name in {
        -0.5 / -$\frac 1 2$,
        0.5 / $\frac 1 2$
    } {
        \draw (0.2,\y) -- (0,\y) node[anchor=east] {\name};
    }

    \draw[dashed, domain=0:2*pi,samples=50] plot (\x, {sin(\x r)});
    \draw (2.3, 0.8) -- (3.1, 1.4) node[anchor=west] {$y = \sin x$};

    \draw[thick, domain=0:2*pi,samples=50] plot (\x, {0.5 * sin(\x r)});
    \draw (4.3, -0.5) -- (2.8, -1.5) node[anchor=east] {$y = \dfrac 1 2 \sin x$};

    \draw[thick, domain=0:2*pi,samples=50] plot (\x, {2 * sin(\x r)}) (3, 2.0) node {$y = 2\sin x$};
\end{tikzpicture}

    \caption{}\label{fig:2-23}
\end{figure}

利用这类函数的周期性,我们可以把上面的简图向左、右扩展,得出 $y = 2\sin x, \, x \in R$ 及 $y = \dfrac 1 2\sin x, \, x \in R$ 的简图(从略)。
\vspace{0.5em}

从图 \ref{fig:2-23} 可以看出,对于同一个 $x$ 值,$y = 2\sin x$ 的图象上点的纵坐标等于 $y = \sin x$
的图象上点的纵坐标的 $2$倍。因此,$y = 2\sin x$ 的图象可以看作是把 $y = \sin x$ 的图象上所有点的
纵坐标伸长到原来的 $2$ 倍(横坐标不变)而得到的。从而,$y = 2\sin x, \, x \in R$ 的值域是 $[-2, 2]$,
最大值是 $2$,最小值是 $-2$。

类似地,$y = \dfrac 1 2\sin x$ 的图象可以看作是把 $y = \sin x$ 的图象上所有点的纵坐标缩短到原来的
$\dfrac 1 2$ \vspace{0.5em} 倍(横坐标不变)而得到的。从而 $y = \dfrac 1 2\sin x, \, x \in R$ 的值域是
$\left[ -\dfrac 1 2 , \dfrac 1 2 \right]$,最大值是 $\dfrac 1 2$,最小值是 $-\dfrac 1 2$。
\vspace{0.5em}

一般地,函数 $y = A\sin x \, (A > 0 \text{且} A \neq 1)$ 的图象可以看作是把 $y = \sin x$ 的图象上
所有点的纵坐标伸长(当 $A > 1$ 时)或缩短(当 $0 < A < 1$ 时)到原来的 $A$ 倍(横坐标不变)而得到的。
$y = A\sin x, \, x \in R$ 的值域是 $[-A, A]$,最大值是 $A$,最小值是 $-A$。

\vspace{0.5em}
\liti 作函数 $y = \sin 2x$及 $y = \sin \dfrac 1 2 x$ 的简图。
\vspace{0.5em}

\jie 函数 $y = \sin 2x$ 的周期 $T = \dfrac{2\pi}{2} = \pi$,我们先来作 $x \in [0, \pi]$ 时函数的简图。

设 $2x = X$,那么 $\sin 2x = \sin X$。
当 $X$ 取 $0$,$\dfrac \pi 2$,$\pi$,$\dfrac{3\pi}{2}$,$2\pi$ 时,\vspace{0.5em} 所对应的五点是函数
$y = \sin X, \, X \in [0, 2\pi]$ 图象上起关键作用的点。这里 $x = \dfrac X 2$,\vspace{0.5em} 所以
当 $x$ 取 $0$,$\dfrac \pi 4$,$\dfrac \pi 2$,$\dfrac{3\pi}{4}$,$\pi$ 时,所对应的五点是函数
$y = \sin 2x, \, x \in [0, \pi]$ 图象上起关键作用的点。

列表:

\begin{table}[H]
\renewcommand\arraystretch{2}
\begin{tabular}{|w{c}{8em}|*{5}{w{c}{3em}|}}
    \hline
    $x$ & $0$ & $\dfrac \pi 4$ & $\dfrac \pi 2$ & $\dfrac{3\pi}{4}$ & $\pi$ \\ \hline
    $2x$ & $0$ & $\dfrac \pi 2$ & $\pi$ & $\dfrac{3\pi}{2}$ & $2\pi$ \\ \hline
    $\sin 2x$ & $0$ & $1$ & $0$ & $-1$ & $0$ \\ \hline
\end{tabular}
\end{table}

函数 $y = \sin \dfrac 1 2 x$ 的周期 $T = \dfrac{\, 2\pi \,}{\dfrac 1 2} = 4\pi$,
我们来作 $x \in [0, 4\pi]$ 时函数的简图。

列表:

\begin{table}[H]
\renewcommand\arraystretch{2}
\begin{tabular}{|w{c}{8em}|*{5}{w{c}{3em}|}}
    \hline
    $x$ & $0$ & $\pi$ & $2\pi$ & $3\pi$ & $4\pi$ \\ \hline
    $\dfrac 1 2 x$ & $0$ & $\dfrac \pi 2$ & $\pi$ & $\dfrac{3\pi}{2}$ & $2\pi$ \\ \hline
    $\sin \dfrac 1 2 x$ & $0$ & $1$ & $0$ & $-1$ & $0$ \\ \hline
\end{tabular}
\end{table}

描点作图(图 \ref{fig:2-24}):

\begin{figure}[htbp]
    \centering
    \begin{tikzpicture}[>=Stealth]
    \draw [->] (-1.0,0) -- (4*pi +1,0) node[anchor=north] {$x$};
    \draw [->] (0,-1.5) -- (0,1.5) node[anchor=east] {$y$};
    \node at (0.3,-0.3) {$O$};

    \foreach \x / \name in {
        0.25 * pi / $\dfrac \pi 4$,
        0.5 * pi / $\dfrac \pi 2$,
        0.75 * pi / $\dfrac{3\pi}{4}$
    } {
        \draw (\x,0.2) -- (\x,0) node[anchor=north] {\name};
    }
    \foreach \x / \name in {
        pi / $\pi$,
        1.5*pi / $\dfrac {3\pi} 2$,
        2 * pi / $2\pi$,
        3 * pi / $3\pi$,
        4 * pi / $4\pi$
    } {
        \draw (\x,0) -- (\x,0.2) node[anchor=south] {\name};
    }

    \foreach \y in {-1,1} {
        \draw (0.2,\y) -- (0,\y) node[anchor=east] {\y};
    }

    \draw[dashed, domain=0:2*pi,samples=50] plot (\x, {sin(\x r)}) (1.5*pi, -1.3) node {$y = \sin x$};
    \draw[thick, domain=0:pi,samples=50] plot (\x, {sin(2*\x r)})  (0.7*pi, -1.3) node {$y = \sin 2x$};
    \draw[thick, domain=0:4*pi,samples=50] plot (\x, {sin(0.5*\x r)})  (3.2*pi, -1.3) node {$y = \sin \dfrac 1 2 x$};
\end{tikzpicture}

    \caption{}\label{fig:2-24}
\end{figure}

利用这类函数的周期性,我们可以把上面的简图向左、右扩展,得出 $y = \sin 2x, \, x \in R$ 及 $y = \sin \dfrac 1 2 x, \, x \in R$ 的简图(从略)。
\vspace{0.5em}

从图 \ref{fig:2-24} 可以看出,在函数 $y = \sin 2x$ 的图象上横坐标为 $\dfrac{x_0}{2} \, (x_0 \in R)$ \vspace{0.5em}
的点的纵坐标同 $y = \sin x$ 上横坐标为 $x_0$ 的点的纵坐标相等(例如,当 $x_0 = \dfrac \pi 2$
时,$\sin \left( 2 \cdot \dfrac{x_0}{2} \right) = \sin \dfrac \pi 2 = 1$,
$\sin x_0 = \sin \dfrac \pi 2 = 1$)。因此,$y = \sin 2x$ 的图象可以看作是把 $y = \sin x$
的图象上所有点的横坐标缩短到原来的 $\dfrac 1 2$ 倍(纵坐标不变)而得到的。

类似地,$y = \sin \dfrac 1 2 x$ 的图象可以看作是把 $y = \sin x$ 的图象上所有点的横坐标伸长到原来的 $2$ 倍(纵坐标不变)而得到。

一般地,函数  $y = \sin \omega x \, (\omega > 0 \text{且} \omega \neq 1)$ 的图象,可以看作是把
$y = \sin x$ 的图象上所有点的横坐标缩短(当 $\omega > 1$ 时)或伸长(当 $0 < \omega < 1$时)
到原来的 $\dfrac 1 \omega$ 倍(纵坐标不变)而得到的。

\liti 作函数 $y = \sin \left( x + \dfrac \pi 3 \right)$ 和 $y = \sin \left( x - \dfrac \pi 4 \right)$ 的简图。
\vspace{0.5em}

\jie 函数 $y = \sin \left( x + \dfrac \pi 3 \right)$ 的周期是 $2\pi$,
我们来作这个函数在长度为一个周期的闭区间上的简图。

设 $x + \dfrac \pi 3 = X$,那么 $\sin \left( x + \dfrac \pi 3 \right) = \sin X$,$x = X - \dfrac \pi 3$。
当 $X$ 取 $0$,$\dfrac \pi 2$,$\pi$,$\dfrac{3\pi}{2}$,$2\pi$ 时,
$x$ 取 $-\dfrac \pi 3$,$\dfrac \pi 6$,$\dfrac{2\pi}{3}$,$\dfrac{7\pi}{6}$,$\dfrac{5\pi}{3}$,
所对应的五点是函数 $y = \sin \left( x + \dfrac \pi 3 \right), \, x \in \left[ -\dfrac \pi 3, \dfrac{5\pi}{3} \right]$
图象上起关键作用的点。
\vspace{0.5em}

列表:

\begin{table}[H]
\renewcommand\arraystretch{2}
\begin{tabular}{|w{c}{8em}|*{5}{w{c}{3em}|}}
    \hline
    $x$ & $-\dfrac \pi 3$ & $\dfrac \pi 6$ & $\dfrac{2\pi}{3}$ & $\dfrac{7\pi}{6}$ & $\dfrac{5\pi}{3}$ \\ \hline
    $x + \dfrac \pi 3$ & $0$ & $\dfrac \pi 2$ & $\pi$ & $\dfrac{3\pi}{2}$ & $2\pi$ \\ \hline
    $\sin \left( x + \dfrac \pi 3 \right)$ & $0$ & $1$ & $0$ & $-1$ & $0$ \\ \hline
\end{tabular}
\end{table}

类似地,对于函数 $y = \sin \left( x - \dfrac \pi 4 \right)$,可以列出下表:

\begin{table}[H]
\renewcommand\arraystretch{2}
\begin{tabular}{|w{c}{8em}|*{5}{w{c}{3em}|}}
    \hline
    $x$ & $\dfrac \pi 4$ & $\dfrac{3\pi}{4}$ & $\dfrac{5\pi}{4}$ & $\dfrac{7\pi}{4}$ & $\dfrac{9\pi}{4}$ \\ \hline
    $x - \dfrac \pi 4$ & $0$ & $\dfrac \pi 2$ & $\pi$ & $\dfrac{3\pi}{2}$ & $2\pi$ \\ \hline
    $\sin \left( x - \dfrac \pi 4 \right)$ & $0$ & $1$ & $0$ & $-1$ & $0$ \\ \hline
\end{tabular}
\end{table}

描点作图(图 \ref{fig:2-25}):

\begin{figure}[htbp]
    \centering
    \begin{tikzpicture}[>=Stealth]
    \draw [->] (-1.8,0) -- (2.25*pi +1,0) node[anchor=north] {$x$};
    \draw [->] (0,-1.5) -- (0,1.5) node[anchor=east] {$y$};
    \node at (-0.3,-0.3) {$O$};

    \foreach \x / \name in {
        -0.33333 * pi / $-\dfrac \pi 3$,
        0.16667 * pi / $\dfrac \pi 6$,
        0.25 * pi / $\dfrac \pi 4$,
        0.5 * pi / $\dfrac \pi 2$,
        0.66667 * pi / $\dfrac{2\pi}{3}$,
        1.16667 * pi / $\dfrac{7\pi}{6}$,
        1.75 * pi / $\dfrac{7\pi}{4}$
    } {
        \draw (\x,0.1) -- (\x,0) node[anchor=north] {\name};
    }
    \foreach \x / \name in {
        0.75 * pi / $\dfrac{3\pi}{4}$,
        pi / $\pi$,
        1.25 * pi / $\dfrac{5\pi}{4}$,
        1.5 * pi / $\dfrac{3\pi}{2}$,
        1.66667 * pi / $\dfrac{5\pi}{3}$,
        2 * pi / $2\pi$,
        2.25 * pi / $\dfrac{9\pi}{4}$
    } {
        \draw (\x,0) -- (\x,0.1) node[anchor=south] {\name};
    }

    \foreach \y in {-1,1} {
        \draw (0.2,\y) -- (0,\y) node[anchor=east] {\y};
    }

    \draw[dashed, domain=0:2*pi,samples=50] plot (\x, {sin(\x r)}) (0.5*pi, 1.3) node {$y = \sin x$};
    \draw[thick, domain=-pi/3:5*pi/3,samples=50] plot (\x, {sin((\x + pi/3) r)})  (1.0*pi, -1.4) node {$y = \sin \left( x + \dfrac \pi 3 \right)$};
    \draw[thick, domain=pi/4:9*pi/4,samples=50] plot (\x, {sin((\x - pi/4) r)})  (2.4*pi, -1.1) node {$y = \sin \left( x - \dfrac \pi 4 \right)$};
\end{tikzpicture}

    \caption{}\label{fig:2-25}
\end{figure}

利用这类函数的周期性,我们可以把所得到的简图向左、右扩展,得出
$y = \sin \left( x + \dfrac \pi 3 \right), \, x \in R$ 及
$y = \sin \left( x - \dfrac \pi 4 \right), \, x \in R$ 的简图(从略)。
\vspace{0.5em}

由图 \ref{fig:2-25} 可以看出,
$y = \sin \left( x + \dfrac \pi 3 \right)$ \vspace{0.5em} 的图象可以看作是把 $y = \sin x$ 的图象上所有的点向左平行移动 $\dfrac \pi 3$ 个单位而得到的,
$y = \sin \left( x - \dfrac \pi 4 \right)$ \vspace{0.5em} 的图象可以看作是把 $y = \sin x$ 的图象上所有的点向右平行移动 $\dfrac \pi 4$ 个单位而得到的。
\vspace{0.5em}

一般地,函数 $y = \sin(x + \varphi), \, (\varphi \neq 0)$ 的图象,可以看作是把 $y = \sin x$
的图象上所有的点向左(当 $\varphi > 0 $ 时)或向右(当 $\varphi < 0$ 时)平行移动 $|\varphi|$ 个单位而得到的。

\vspace{0.5em}
\liti 作函数 $y = 3\sin \left( 2x + \dfrac \pi 3 \right)$ 的简图。
\vspace{0.5em}

\jie 函数 $y = 3\sin \left( 2x + \dfrac \pi 3 \right)$ 的周期 $T = \dfrac{2\pi}{2} = \pi$。
我们来作这个函数在长度为一个周期的闭区间上的简图。

设 $X = 2x + \dfrac \pi 3$,那么 $3\sin(2x + \dfrac \pi 3) = 3\sin X$,
$x = \dfrac{X - \dfrac \pi 3}{2} = \dfrac X 2 - \dfrac \pi 6$。
当 $X$ 取 $0$,$\dfrac \pi 2$,$\pi$,$\dfrac{3\pi}{2}$,$2\pi$ 时,
$x$ 取 $-\dfrac \pi 6$,$\dfrac{\pi}{12}$,$\dfrac{\pi}{3}$,$\dfrac{7\pi}{12}$,$\dfrac{5\pi}{6}$,
所对应的五点是函数 $y = 3\sin \left( 2x + \dfrac \pi 3 \right), \, x \in \left[ -\dfrac \pi 6, \dfrac{5\pi}{6} \right]$
图象上起关键作用的点。

列表:

\begin{table}[H]
\renewcommand\arraystretch{2}
\begin{tabular}{|w{c}{8em}|*{5}{w{c}{3em}|}}
    \hline
    $x$ & $-\dfrac \pi 6$ & $\dfrac{\pi}{12}$ & $\dfrac{\pi}{3}$ & $\dfrac{7\pi}{12}$ & $\dfrac{5\pi}{6}$ \\ \hline
    $2x + \dfrac \pi 3$ & $0$ & $\dfrac \pi 2$ & $\pi$ & $\dfrac{3\pi}{2}$ & $2\pi$ \\ \hline
    $3\sin \left( 2x + \dfrac \pi 3 \right)$ & $0$ & $3$ & $0$ & $-3$ & $0$ \\ \hline
\end{tabular}
\end{table}

描点作图(图 \ref{fig:2-26}):

\begin{figure}[htbp]
    \centering
    \begin{tikzpicture}[>=Stealth]
    \draw [->] (-1.8,0) -- (2*pi +1,0) node[anchor=north] {$x$};
    \draw [->] (0,-3.5) -- (0,3.5) node[anchor=east] {$y$};
    \node at (-0.2,0.2) {$O$};

    \foreach \x / \name in {
        -0.16667 * pi / $-\dfrac \pi 6$,
        0.08333 * pi / $\dfrac{\pi}{12}$,
        0.58333 * pi / $\dfrac{7\pi}{12}$
    } {
        \draw (\x,0.1) -- (\x,0) node[anchor=north] {\name};
    }
    \draw (-0.33333 * pi,0.1) -- (-0.33333 * pi,0) +(-0.2, 0) node[anchor=north] {$-\dfrac \pi 3$};
    \draw (0.33333 * pi,0.1) -- (0.33333 * pi,0) +(-0.2, 0) node[anchor=north] {$\dfrac \pi 3$};

    \foreach \x / \name in {
        0.66667 * pi / $\dfrac{2\pi}{3}$,
        pi / $\pi$,
        1.66667 * pi / $\dfrac{5\pi}{3}$,
        2 * pi / $2\pi$
    } {
        \draw (\x,0) -- (\x,0.1) node[anchor=south] {\name};
    }

    \foreach \y in {-3,-2,-1,2} {
        \draw (0,\y) -- (0.1,\y) node[anchor=west] {$\y$};
    }
    \draw (0,1) -- (0.1,1) +(0, 0.2) node[anchor=west] {$1$};
    \draw (0.1,3) -- (0,3) node[anchor=east] {$3$};

    \draw[dashed, domain=0:2*pi,samples=50] plot (\x, {sin(\x r)}) (0.7*pi, 1.3) node {$y = \sin x$};

    \draw[dashed, domain=-pi/3:5*pi/3,samples=50] plot (\x, {sin((\x + pi/3) r)});
    \draw (1.1*pi, -1) -- (1.2*pi, -1.4) +(-0.3, 0.3) node [anchor=north west] {$y = \sin \left( x + \dfrac \pi 3 \right)$};

    \draw[dashed, domain=-pi/6:5*pi/6,samples=50] plot (\x, {sin((2*\x + pi/3) r)});
    \draw (0.65*pi, -1) -- (0.8*pi, -2.2) +(-0.3, 0.3) node [anchor=north west] {$y = \sin \left( 2x + \dfrac \pi 3 \right)$};

    \draw[thick, domain=-pi/6:5*pi/6,samples=50] plot (\x, {3*sin((2*\x + pi/3) r)});
    \draw (pi/5, 2.8) node [anchor=west] {$y = 3\sin \left( 2x + \dfrac \pi 3 \right)$};

    \draw (0.83333 * pi,0.1) -- (0.83333 * pi,0) +(0.3, 0) node[anchor=north,fill=white,inner sep=1pt] {$\dfrac{5\pi}{6}$};
\end{tikzpicture}

    \caption{}\label{fig:2-26}
\end{figure}


利用这类函数的周期性,我们可以把上面所得到的简图向左、右扩展,得到
$y = 3\sin \left( 2x + \dfrac \pi 3 \right), \, x \in R$ 的简图(从略)。

函数 $y = 3\sin \left( 2x + \dfrac \pi 3 \right)$ 的图象可以看作是用下面的方法得到的:
先把 $y = \sin x$ 的图象上所有的点向左平行移动 $\dfrac \pi 3$ 个单位,得到 $y = \sin \left( x + \dfrac \pi 3 \right)$ 的图象;
再把 $y = \sin \left( x + \dfrac \pi 3 \right)$ 的图象上所有的点的横坐标缩短到原来的 $\dfrac 1 2$倍(纵坐标不变),得到 $y = \sin \left( 2x + \dfrac \pi 3 \right)$ 的图象;
再把 $y = \sin \left( 2x + \dfrac \pi 3 \right)$ 的图象上所有的点的纵坐标伸长到原来的 $3$ 倍(横坐标不变),从而得到 $y = 3\sin \left( 2x + \dfrac \pi 3 \right)$ 的图象。
\vspace{0.5em}

一般地,函数 $y = A \sin(\omega x + \varphi), \, (A > 0, \, \omega > 0), \, x \in R$
的图象可以看作是用下面的方法得到的:
先把 $y = \sin x$ 的图象上所有的点向左($\varphi > 0$)或向右($\varphi < 0$)平行移动 $|\varphi|$ 个单位,
\vspace{0.5em} 再把所得各点的横坐标缩短($\omega > 1$)或伸长($0 < \omega < 1$)到原来的 $\dfrac 1 \omega$ 倍(纵坐标不变),
\vspace{0.5em} 再把所得各点的纵坐标伸长($A > 1$)或缩短($0 < A < 1$)到原来的 $A$ 倍(横坐标不变)。

当函数 $y = A \sin(\omega x + \varphi), \, (A > 0, \, \omega > 0), \, x \in [0, +\infty)$ \vspace{0.5em}
表示一个振动量时,$A$ 就表示这个量振动时离开平衡位置的最大距离,通常把它叫做这个振动的\textbf{振幅};
往复振动一次所需要的时间 $T = \dfrac{2\pi}{\omega}$,它叫做振动的\textbf{周期};
单位时间内往复振动的次数 $f = \dfrac 1 T = \dfrac{\omega}{2\pi}$,它叫做振动的\textbf{频率};\vspace{0.5em}
$\omega x + \varphi$ 叫做 \textbf{相位},$\varphi$ 叫做 \textbf{初相}(即当 $x = 0$ 时的相位)。

\lianxi
\begin{xiaotis}

\xiaoti{作下列函数在长度为一个周期的闭区间上的简图:}
\begin{xiaoxiaotis}

    \renewcommand\arraystretch{1.5}
    \begin{tabular}[t]{*{3}{@{}p{14em}}}
        \xiaoxiaoti {$y = \dfrac 3 2 \sin x$;} & \xiaoxiaoti{$y = \dfrac 1 3 \sin x$;} \\
        \xiaoxiaoti {$y = \sin 4x$;} & \xiaoxiaoti{$y = 2\sin \dfrac 1 3 x$;} \\
        \xiaoxiaoti {$y = \sin \left( x + \dfrac \pi 4 \right)$;} & \xiaoxiaoti{$y = \sin \left( x - \dfrac \pi 2 \right)$;} \\
        \xiaoxiaoti {$y = 4\sin \left( x - \dfrac \pi 3 \right)$;} & \xiaoxiaoti{$y = \sin \left( 2x + \dfrac \pi 6 \right)$;} \\
        \xiaoxiaoti {$y = 5\sin \left( \dfrac 1 2 x + \dfrac \pi 6 \right)$;} & \xiaoxiaoti{$y = \dfrac 1 2 \sin \left( 3x - \dfrac \pi 4 \right)$。} \\
    \end{tabular}

\end{xiaoxiaotis}

\vspace{0.5em}
\xiaoti{函数 $y = \dfrac 1 8 \sin x$的振幅是多少?它的图象与函数 $y = \sin x$ 的图象有什么关系?}
\vspace{0.5em}

\xiaoti{函数 $y = \sin \dfrac 2 3 x$ 的周期是多少?它的图象与函数 $y = \sin x$ 的图象有什么关系?}
\vspace{0.5em}

\xiaoti{函数 $y = \sin \left( x - \dfrac{\pi}{12} \right)$ 的初相是多少?它的图象与函数 $y = \sin x$ 的图象有什么关系?}
\vspace{0.5em}

\end{xiaotis}
