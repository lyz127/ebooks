\xiaojie

一、本章内容包括两角和与差的三角函数的公式,倍角、半角的三角函数的公式,以及三角函数的
积化和差与和差化积公式。这些公式主要用于三角函数式的计算和推导。它们在高等数学、电工学、
力学、机械设计与制造等方面都有广泛的应用,要熟练地掌握。主要公式如下:

\begin{tabular}[t]{@{}p{10em}@{}l}
    两角和与差公式:& $\sin(\alpha \pm \beta) = \sin\alpha \cos\beta \pm \cos\alpha \sin\beta$;\\
    & $\cos(\alpha \pm \beta) = \cos\alpha \cos\beta \mp \sin\alpha \sin\beta$;\\
    & $\tan(\alpha \pm \beta) = \dfrac{\tan\alpha \pm \tan\beta}{1 \mp \tan\alpha \tan\beta}$。
\end{tabular}

\begin{tabular}[t]{@{}p{10em}@{}l}
    倍角公式:& $\sin2\alpha = 2\sin\alpha \cos\alpha$;\\
    & $\cos2\alpha = \cos^2\alpha - \sin^2\alpha = 2\cos^2\alpha - 1 = 1 - 2\sin^2\alpha$;\\
    & $\tan2\alpha = \dfrac{2\tan\alpha}{1 - \tan^2\alpha}$。
\end{tabular}

\renewcommand\arraystretch{2}
\begin{tabular}[t]{@{}p{10em}@{}l}
    半角公式:& $\sin\dfrac{\alpha}{2} = \pm\sqrt{\dfrac{1 - \cos\alpha}{2}}$;\\
    & $\cos\dfrac{\alpha}{2} = \pm\sqrt{\dfrac{1 + \cos\alpha}{2}}$;\\
    & $\tan\dfrac{\alpha}{2} = \pm\sqrt{\dfrac{1 - \cos\alpha}{1 + \cos\alpha}} = \dfrac{1 - \cos\alpha}{\sin\alpha} = \dfrac{\sin\alpha}{1 + \cos\alpha}$。
\end{tabular}

\renewcommand\arraystretch{1.8}
\begin{tabular}[t]{@{}p{10em}@{}l}
    积化和差公式:& $\sin\alpha \cos\beta = \dfrac{1}{2} [\sin(\alpha + \beta) + \sin(\alpha - \beta)]$;\\
    & $\cos\alpha \sin\beta = \dfrac{1}{2} [\sin(\alpha + \beta) - \sin(\alpha - \beta)]$;\\
    & $\cos\alpha \cos\beta = \dfrac{1}{2} [\cos(\alpha + \beta) + \cos(\alpha - \beta)]$;\\
    & $\sin\alpha \sin\beta = -\dfrac{1}{2} [\cos(\alpha + \beta) - \cos(\alpha - \beta)]$。
\end{tabular}

\begin{tabular}[t]{@{}p{10em}@{}l}
    和差化积公式:& $\sin\alpha + \sin\beta = 2\sin\dfrac{\alpha + \beta}{2} \cos\dfrac{\alpha - \beta}{2}$;\\
    & $\sin\alpha - \sin\beta = 2\cos\dfrac{\alpha + \beta}{2} \sin\dfrac{\alpha - \beta}{2}$;\\
    & $\cos\alpha + \cos\beta = 2\cos\dfrac{\alpha + \beta}{2} \cos\dfrac{\alpha - \beta}{2}$;\\
    & $\cos\alpha - \cos\beta =-2\sin\dfrac{\alpha + \beta}{2} \sin\dfrac{\alpha - \beta}{2}$。
\end{tabular}
\renewcommand\arraystretch{1}

此外,还有万能公式:
\begin{gather*}
    \sin\alpha = \dfrac{2\tan\dfrac{\alpha}{2}}{1 + \tan^2\dfrac{\alpha}{2}} ,\quad
    \cos\alpha = \dfrac{1 - \tan^2\dfrac{\alpha}{2}}{1 + \tan^2\dfrac{\alpha}{2}} ,\quad
    \tan\alpha = \dfrac{2\tan\dfrac{\alpha}{2}}{1 - \tan^2\dfrac{\alpha}{2}} \text{,}
\end{gather*}

\jiange
二、上述公式是以两角和的余弦公式为基础推导得出,这些公式的内在联系和推导的线索如下表(见 \pageref{fig:tuidao} 页)。

掌握表中公式的内在联系及其推导的线索,能够帮助我们理解和记忆这些公式,这是学好本章内容的关键。

三、应注意的几个问题:

(1) 凡使公式中某个式子没有意义的角,都不适合公式;

(2)在半角公式(\ref{eq:S-a/2},\ref{eq:C-a/2},\ref{eq:T-a/2})中,根号前的符号由半角所在的象限来决定;

\begin{figure}[H]
    \centering
    \mylabel{fig:tuidao}
    \begin{tikzpicture}[>=Stealth]
    \pgfmathsetmacro{\b}{10}
    \pgfmathsetmacro{\h}{2.5}

    \draw (-0.6, \b+\h-0.3) rectangle (0.6, \b+\h+1);
    \node at (0, \b+\h+0.6) {$S_{\alpha + \beta}$};
    \node at (0, \b+\h) {$C_{\alpha + \beta}$};
    \draw [->] (0, \b+\h-0.3) -- (0, \b+1);
    \draw (-0.6, \b-0.3) rectangle (0.6, \b+1);
    \node at (0, \b+0.6) {$S_{\alpha - \beta}$};
    \node at (0, \b) {$C_{\alpha - \beta}$};
    \coordinate (C1) at (0, \b-0.3);

    \draw [->] (0.6, \b+\h+0.3) -- (1.4, \b+\h+0.3);
    \draw (1.4, \b+\h) rectangle (2.6, \b+\h+0.7);
    \node at (2, \b+\h+0.3) {$T_{\alpha + \beta}$};

    \draw [->] (0.6, \b+0.3) -- (1.4, \b+0.3);
    \draw (1.4, \b) rectangle (2.6, \b+0.7);
    \node at (2, \b+0.3) {$T_{\alpha - \beta}$};
    \draw (0.6, \b-0.2) -- (2.8, \b-0.2) -- (2.8, \b+\h+0.9) -- (0.6, \b+\h+0.9);


    \pgfmathsetmacro{\s}{5.6}
    \draw [->] (2.8, \b+\h/2) -- (\s, \b+\h/2);
    \draw (\s, \b-0.2) rectangle (\s+7, \b+4);
    \coordinate (C2) at (\s+3.5, \b-0.3);
    \node at (\s+3.5, \b+3.3) {$\sin\alpha \cos\beta = \dfrac{1}{2}[\sin(\alpha + \beta) + \sin(\alpha - \beta)]$};
    \node at (\s+3.5, \b+2.3) {$\cos\alpha \sin\beta = \dfrac{1}{2}[\sin(\alpha + \beta) - \sin(\alpha - \beta)]$};
    \node at (\s+3.5, \b+1.3) {$\cos\alpha \cos\beta = \dfrac{1}{2}[\cos(\alpha + \beta) + \cos(\alpha - \beta)]$};
    \node at (\s+3.6, \b+0.3) {$\sin\alpha \sin\beta =-\dfrac{1}{2}[\cos(\alpha + \beta) - \cos(\alpha - \beta)]$};

    %------------------------------------

    \pgfmathsetmacro{\b}{6}

    \draw[->] (C1) -- (0, \b+2.5);
    \draw (-0.8, \b-0.5) rectangle (3.2, \b+2.5);
    \coordinate (C3) at (1, \b-0.6);
    \node at (0, \b+2.0) {$S_{2\alpha}$};
    \node at (0, \b+1.5) {$C_{2\alpha}$};
    \draw [dashed] (-0.7, \b-0.3) rectangle (3, \b+1.2);
    \node at (1.2, \b+0.8) {$\cos 2\alpha = 2\cos^2\alpha - 1$};
    \node at (1.2, \b) {$\cos 2\alpha = 1 - 2\sin^2\alpha$};

    \pgfmathsetmacro{\s}{5}
    \draw[->] (3.2, \b+2) -- (\s, \b+2);
    \draw (\s, \b+1.6) rectangle (\s+2, \b+2.3);
    \node at (\s+1, \b+2) {$\text{万能公式}$};

    \pgfmathsetmacro{\b}{3}
    \draw[->] (C2) -- ++(0, -2.5);
    \draw (\s, \b-0.2) rectangle (\s+7, \b+4);
    \coordinate (C2) at (\s+3.5, \b-0.3);
    \node at (\s+3.5, \b+3.3) {$\sin A + \sin B = 2\sin\dfrac{A + B}{2} \cos\dfrac{A - B}{2}$};
    \node at (\s+3.5, \b+2.3) {$\sin A - \sin B = 2\cos\dfrac{A + B}{2} \sin\dfrac{A - B}{2}$};
    \node at (\s+3.5, \b+1.3) {$\cos A + \cos B = 2\cos\dfrac{A + B}{2} \cos\dfrac{A - B}{2}$};
    \node at (\s+3.6, \b+0.3) {$\cos A - \cos B =-2\sin\dfrac{A + B}{2} \sin\dfrac{A - B}{2}$};

    %------------------------------------

    \draw[->](C3) -- (1, 3.4);
    \draw (-0.9, -0.3) rectangle (3.2, 3.4);
    \draw [dashed] (-0.7, 1.2) rectangle (3, 3.2);
    \node at (1, 2.7) {$\sin^2{\dfrac{\alpha}{2}} = \dfrac{1 - \cos\alpha}{2}$};
    \node at (1, 1.7) {$\cos^2{\dfrac{\alpha}{2}} = \dfrac{1 + \cos\alpha}{2}$};
    \node at (0, 0.7) {$S_{\frac{\alpha}{2}}$};
    \node at (0, 0) {$C_{\frac{\alpha}{2}}$};

    \draw[->] (3.2, 0.5) -- (4.6, 0.5);
    \draw (4.6, 0) rectangle (5.5, 1.0);
    \node at (5, 0.5) {$T_{\frac{\alpha}{2}}$};

    \draw[->] (3.2, 2.2) -- (9.0, 2.2) -- (9.0, 1.0);
    \draw (6.3, 0) rectangle (11.8, 1.0);
    \node at (9, 0.5) {$\tan\dfrac{\alpha}{2} = \dfrac{\sin\alpha}{1 + \cos\alpha} = \dfrac{1 - \cos\alpha}{\sin\alpha}$};
\end{tikzpicture}

\end{figure}

(3)把 $a\sin\alpha + b\cos\alpha$ 化成 $\sqrt{a^2 + b^2} \sin(\alpha + \varphi)$ 时,
其中辅助角 $\varphi$ 在哪个象限,由 $a$,$b$ 的符号确定,$\varphi$ 的值由 $\tan\varphi = \dfrac{b}{a}$ 确定。
