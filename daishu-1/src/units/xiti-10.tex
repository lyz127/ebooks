\xiti
\begin{xiaotis}

\xiaoti{已知 $\sin\alpha = \dfrac{15}{17}$,$\cos\beta = -\dfrac{5}{13}$,并且 $\alpha$,
    $\beta$ 都是第二象限的角,求 $\cos(\alpha + \beta)$ 和 $\cos(\alpha - \beta)$ 的值。}

\xiaoti{在 $\triangle ABC$ 中,已知 $\cos A = \dfrac 4 5$,$\cos B = \dfrac{12}{13}$,求 $\cos C$ 的值。}

\xiaoti{求证:}
\begin{xiaoxiaotis}

    \renewcommand\arraystretch{2}
    \begin{tabular}[t]{*{2}{@{}p{16em}}}
        \xiaoxiaoti {$\sin\left( \dfrac{3\pi}{2} - \alpha \right) = -\cos\alpha$;} & \xiaoxiaoti {$\cos\left( \dfrac{3\pi}{2} - \alpha \right) = -\sin\alpha$;} \\
        \xiaoxiaoti {$\tan\left( \dfrac{3\pi}{2} - \alpha \right) = \cot\alpha$;} & \xiaoxiaoti {$\sin\left( \dfrac{3\pi}{2} + \alpha \right) = -\cos\alpha$;} \\
        \xiaoxiaoti {$\cos\left( \dfrac{3\pi}{2} + \alpha \right) = \sin\alpha$;} & \xiaoxiaoti {$\tan\left( \dfrac{3\pi}{2} + \alpha \right) = -\cot\alpha$。}
    \end{tabular}
    \jiange
\end{xiaoxiaotis}

\xiaoti{化简:}
\begin{xiaoxiaotis}

    \xiaoxiaoti{$\sin(30^\circ + \alpha) - \sin(30^\circ - \alpha)$;} \jiange

    \xiaoxiaoti{$\sin\left( \dfrac \pi 3 + \alpha \right) + \sin\left( \dfrac \pi 3 - \alpha \right)$;} \jiange

    \xiaoxiaoti{$\cos\left( \dfrac \pi 4 + \phi \right) - \cos\left( \dfrac \pi 4 - \phi \right)$;} \jiange

    \xiaoxiaoti{$\cos(60^\circ + \theta) + \cos(60^\circ - \theta)$;}

    \xiaoxiaoti{$\sin 58^\circ \cos 37^\circ - \cos 58^\circ \sin 37^\circ$;}

    \xiaoxiaoti{$\cos 24^\circ \cos 69^\circ - \sin 24^\circ \sin 69^\circ$;}

    \xiaoxiaoti{$\sin 14^\circ \cos 16^\circ + \sin 76^\circ \cos 74^\circ$;}

    \xiaoxiaoti{$\sin 21^\circ \cos 81^\circ - \sin 69^\circ \cos 9^\circ$;}

    \xiaoxiaoti{$\sin(\alpha - \beta) \cos\beta + \cos(\alpha - \beta) \sin\beta$;}

    \xiaoxiaoti{$\cos(\alpha + \beta) \cos\beta + \sin(\alpha + \beta) \sin\beta$;}

    \xiaoxiaoti{$\cos(36^\circ + x) \cos(54^\circ - x) - \sin(36^\circ + x) \sin(54^\circ - x)$;}

    \xiaoxiaoti{$\sin(70^\circ + \alpha) \cos(10^\circ + \alpha) - \cos(70^\circ + \alpha) \sin(170^\circ - \alpha)$。}

\end{xiaoxiaotis}

\xiaoti{求证:}
\begin{xiaoxiaotis}

    \xiaoxiaoti{$\dfrac{\sqrt 3}{3} \sin\alpha - \dfrac 1 2 \cos\alpha = \sin\left( \alpha - \dfrac \pi 6 \right)$;} \jiange

    \xiaoxiaoti{$\cos\theta - \sin\theta = \sqrt 2 \cos\left( \dfrac \pi 4 + \theta \right)$;} \jiange

    \xiaoxiaoti{$\cos(\alpha + \beta) \cos(\alpha - \beta) = \cos^2 \alpha - \sin^2 \beta$;}

    \xiaoxiaoti{$\sin(\alpha + \beta) \sin(\alpha - \beta) = \sin^2 \alpha - \sin^2 \beta$;}

    \xiaoxiaoti{$\sin(\alpha + \beta) \cos(\alpha - \beta) = \sin\alpha \cos\alpha + \sin\beta \cos\beta$。}

\end{xiaoxiaotis}

\jiange \xiaoti{}
\begin{xiaoxiaotis}

    \vspace{-1.5em} \xiaoxiaoti{已知 $\tan x = \dfrac 1 4$,$\tan y = -3$,求 $\tan(x + y)$ 的值;} \jiange

    \xiaoxiaoti{已知 $\tan \alpha = 2k + 1$,$\tan \beta = 2k - 1$,求 $\cot(\alpha -\beta)$ 的值。}

\end{xiaoxiaotis}

\jiange
\xiaoti{已知 $\cos\theta = -\dfrac{12}{13}$,$\theta \in \left( \pi, \, \dfrac{3\pi}{2} \right)$,求
    $\sin\left( \theta - \dfrac \pi 4 \right)$,$\cos\left( \theta - \dfrac \pi 4 \right)$ 和 $\tan\left( \theta - \dfrac \pi 4 \right)$ 的值。}
\jiange

\xiaoti{用 $\cot\alpha$,$\cot\beta$ 表示 $\cot(\alpha \pm \beta)$,并求 $\dfrac{1 - \cot 15^\circ}{1 + \cot 15^\circ}$ 的值。} \jiange

\xiaoti{化简:}
\begin{xiaoxiaotis}

    \renewcommand\arraystretch{2}
    \begin{tabular}[t]{*{2}{@{}p{16em}}}
        \xiaoxiaoti {$\dfrac{\tan 53^\circ - \tan 23^\circ}{1 + \tan 53^\circ \cot 67^\circ}$;} & \xiaoxiaoti {$\dfrac{\tan 2\theta - \tan\theta}{1 + \tan 2\theta \tan\theta}$;} \\
        \xiaoxiaoti {$\dfrac{1 - \tan 15^\circ}{1 + \cot 75^\circ}$;} & \xiaoxiaoti {$\dfrac{1 + \tan\theta}{1 - \tan\theta}$。}
    \end{tabular}
    \jiange
\end{xiaoxiaotis}

\xiaoti{求证:}
\begin{xiaoxiaotis}

    \xiaoxiaoti{$\tan(x + y) \cdot \tan(x - y) = \dfrac{\tan^2 x - \tan^2 y}{1 - \tan^2 x \tan^2 y}$;}

    \jiange
    \xiaoxiaoti{$\cot\left( \dfrac \pi 4 + \theta \right) = \dfrac{1 - \tan\theta}{1 + \tan\theta}$;}
    \jiange

    \xiaoxiaoti{$\dfrac{\tan x + \tan y}{\tan x - \tan y} = \dfrac{\sin(x + y)}{\sin(x - y)}$。}
    \jiange

\end{xiaoxiaotis}

\xiaoti{已知 $\tan\theta = \dfrac 1 2$,$\tan\phi = \dfrac 1 3$,并且 $\theta$,$\phi$ 都是锐角,求证 $\theta + \phi = 45^\circ$。\jiange}

\begin{minipage}{0.7\textwidth}

\xiaoti{已知 $a\sin(\theta + \alpha) = b\sin(\theta + \beta)$,求证:\jiange}
$$\tan\theta = \dfrac{b\sin\beta - a\sin\alpha}{a\cos\alpha - b\cos\beta} \text{。} \jiange$$

\xiaoti{如图,在 $\triangle ABC$ 中,$AD \perp BC$,垂足为 $D$,且 $BD:DC:AD = 2:3:6$,求 $\angle BAC$ 的度数。}

\xiaoti{已知 $\tan\alpha$,$\tan\beta$ 是方程 $x^2 + 6x + 7 = 0$ 的两个根,求证
    $$\sin(\alpha + \beta) = \cos(\alpha + \beta) \text{。}$$
}
\end{minipage}
\begin{minipage}{0.2\textwidth}
    \centering
    \begin{tikzpicture}[>=Stealth, scale=0.8]
    \coordinate [label=90:$A$] (A) at (0,3);
    \coordinate [label=270:$B$] (B) at (-1,0);
    \coordinate [label=270:$D$] (D) at (0,0);
    \coordinate [label=270:$C$] (C) at (1.5,0);

    \draw [thin] (A) -- (B) -- (C) -- (A);
    \draw (A) -- (D);
    \draw [right angle symbol={A}{D}{C}];
\end{tikzpicture}


    (第13题)
\end{minipage}

\end{xiaotis}
