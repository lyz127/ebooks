\subsection{任意角的三角函数}\label{subsec:2-3}

在初中,我们已经接触过正弦、余弦、正切、余切这四种三角函数。它们都是以角为自变量、以比值为函数值的函数。
这四种三角函数的定义,当时是针对 $0^\circ$ \~{} $360^\circ$ 间的角作出的,并对 $0^\circ$ \~{} $180^\circ$
间的角的三角函数作了一些讨论。下面将三角函数的定义推广到任意角的情形。

\begin{figure}[htbp]
    \centering
    \begin{minipage}{8cm}
    \centering
    \begin{tikzpicture}[>=Stealth]
    \draw [->] (-3.5,0) -- (2.0,0) node[anchor=north] {$x$};
    \draw [->] (0,-1) -- (0,3.5) node[anchor=east] {$y$};
    \node at (-0.3,-0.3) {$O$};
    \draw (0,0) -- (130:3) node[anchor=south] {$\alpha$的终边};
    \draw [fill=black] (130:2) circle(1pt) +(0.5, 0.3) node {$P(x, y)$};
    \node at (-0.7,1.1) {$r$};
\end{tikzpicture}

    \caption*{(\Rmnum 2)}
    \end{minipage}
    \qquad
    \begin{minipage}{8cm}
    \centering
    \begin{tikzpicture}[>=Stealth]
    \draw [->] (-2.0,0) -- (3.5,0) node[anchor=north] {$x$};
    \draw [->] (0,-1) -- (0,3.5) node[anchor=east] {$y$};
    \node at (-0.3,-0.3) {$O$};
    \draw (0,0) -- (60:3) node[anchor=south] {$\alpha$的终边};
    \draw [fill=black] (60:2) circle(1pt) +(0.6, -0.3) node {$P(x, y)$};
    \node at (0.4,1.1) {$r$};
\end{tikzpicture}

    \caption*{(\Rmnum 1)}
    \end{minipage}
    \begin{minipage}{8cm}
    \centering
    \begin{tikzpicture}[>=Stealth]
    \draw [->] (-3.5,0) -- (2.0,0) node[anchor=north] {$x$};
    \draw [->] (0,-3.5) -- (0,1) node[anchor=east] {$y$};
    \node at (0.3,-0.3) {$O$};
    \draw (0,0) -- (220:3) node[anchor=north] {$\alpha$的终边};
    \draw [fill=black] (220:2) circle(1pt) +(0.7, -0.2) node {$P(x, y)$};
    \node at (-1.1,-0.6) {$r$};
\end{tikzpicture}

    \caption*{(\Rmnum 3)}
    \end{minipage}
    \qquad
    \begin{minipage}{8cm}
    \centering
    \begin{tikzpicture}[>=Stealth]
    \draw [->] (-2.0,0) -- (3.5,0) node[anchor=north] {$x$};
    \draw [->] (0,-3.5) -- (0,1) node[anchor=east] {$y$};
    \node at (-0.3,-0.3) {$O$};
    \draw (0,0) -- (310:3) node[anchor=north] {$\alpha$的终边};
    \draw [fill=black] (310:2) circle(1pt) node[anchor=west] {$P(x, y)$};
    \node at (0.7,-0.5) {$r$};
\end{tikzpicture}

    \caption*{(\Rmnum 4)}
    \end{minipage}
    \caption{}\label{fig:2-11}
\end{figure}

设 $\alpha$ 是一个任意大小的角。角 $\alpha$ 的终边上任意一点 $P$ 的坐标是 $(x, y)$,
它与原点的距离是 $r \, (r > 0)$(图 \ref{fig:2-11}),那么角 $\alpha$ 的正弦、余弦、正切、余切分别是

\begin{gather*}
    \sin \alpha = \dfrac y r , \quad \cos \alpha = \dfrac x r , \\
    \tan \alpha = \dfrac y x , \quad \cot \alpha = \dfrac x y \text{。}
\end{gather*}

\begin{wrapfigure}[13]{r}{4.5cm}
    \centering
    \begin{tikzpicture}[>=Stealth]
    \foreach \baseline / \name / \formula in
        {6/{正弦}/{$\displaystyle \frac y r$}, 
         4/{余弦}/{$\displaystyle \frac x r$},
         2/{正切}/{$\displaystyle \frac y x$},
         0/{余切}/{$\displaystyle \frac x y$}}
    {
        \draw (0, \baseline) circle [x radius=0.6cm, y radius=0.8cm];
        \node at (0, \baseline + 0.3) {$\alpha$};
        \node at (0, \baseline - 0.3) {$\vdots$};

        \draw (3, \baseline) circle [x radius=0.6cm, y radius=0.8cm];
        \node at (3, \baseline + 0.3) {\formula};
        \node at (3, \baseline - 0.3) {\vdots};

        \node at (1.5, \baseline + 0.5) {\name};
        \draw [->] (1.0, \baseline + 0.1) -- (2.0, \baseline + 0.1);
    }
\end{tikzpicture}

    \vspace{-20pt}
    \caption{}\label{fig:2-12}
\end{wrapfigure}

\vspace{0.5em}
对于确定的 $\alpha$ 角,这四个比值的大小和 $P$ 点在角 $\alpha$ 的终边上的位置没有关系。
当角 $\alpha$ 的终边在 $x$ 轴上时,$\alpha = k \pi$(或 $\alpha = k \cdot 180^\circ$),$k \in Z$,$\cot \alpha = \dfrac x y$ 无意义(因为 $y = 0$);
当角 $\alpha$ 的终边在 $y$ 轴上时,$\alpha = k \pi + \dfrac \pi 2$(或 $\alpha = k \cdot 180^\circ + 90^\circ$),$k \in Z$,$\tan \alpha = \dfrac y x$ 无意义(因为 $x = 0$)。
此外,对于确定的角 $\alpha$,上面四个比值都是一个确定的实数。这就是说,正弦、余弦、正切、余切
分别可看成从一个角的集合到一个比值的集合的映射(图 \ref{fig:2-12}),它们都是以角为自变量,
以比值为函数值的函数,这些函数都叫做\textbf{三角函数}。

有时,我们还要用到下面两个三角函数:

角 $\alpha$ 的\textbf{正割}:\quad $\sec \alpha = \dfrac r x$,
\vspace{0.5em}

角 $\alpha$ 的\textbf{余割}:\quad $\csc \alpha = \dfrac r y$。

由于角的集合与实数集之间可以建立一一对应关系,三角函数可以看成是以实数为自变量的函数。
例如,当采用弧度制来度量角时,对于每一个实数,对应着一个确定的角(其弧度数等于这个实数),
而这个确定的角又对应着它的三角函数值(所取的实数应使相应的三角函数有意义),从而这个实数
就对应着它的三角函数值,即
$$\text{实数} \to \text{角(其弧度数等于这个实数)} \to \text{三角函数值(实数)} $$

当自变量是用弧度制来度量角所得到的实数 $\alpha$ 时,三角函数的定义域如下表:

\begin{table}[h]
\centering
\begin{tabular}{|c|l|}
    \hline
    三角函数 & 定义域 \\ \hline
    $\sin \alpha$ & $\{\alpha \mid \alpha \in R\}$  \\ \hline
    $\cos \alpha$ & $\{\alpha \mid \alpha \in R\}$  \\ \hline
    \rule{0pt}{1.5em}$\tan \alpha$ & $\{\alpha \mid \alpha \in R, \, \alpha \neq k\pi + \dfrac \pi 2, \, k \in Z\}$  \\ \hline
    $\cot \alpha$ & $\{\alpha \mid \alpha \in R, \, \alpha \neq k\pi, \, k \in Z\}$  \\ \hline
    \rule{0pt}{1.5em}$\sec \alpha$ & $\{\alpha \mid \alpha \in R, \, \alpha \neq k\pi + \dfrac \pi 2, \, k \in Z\}$  \\ \hline
    $\csc \alpha$ & $\{\alpha \mid \alpha \in R, \, \alpha \neq k\pi, \, k \in Z\}$  \\ \hline
\end{tabular}
\end{table}

\liti 已知角 $\alpha$ 的终边经过点 $P(2, -3)$,求 $\alpha$ 的六个三角函数值(图 \ref{fig:2-13})。

\begin{wrapfigure}[8]{r}{4.5cm}
    \centering
    \begin{tikzpicture}[>=Stealth]
    \draw [->] (-1.0,0) -- (3.5,0) node[anchor=north] {$x$};
    \draw [->] (0,-3.5) -- (0,1.5) node[anchor=east] {$y$};
    \node at (-0.3,-0.3) {$O$};
    \foreach \x in {1,2,3} {
        \draw (\x,0.2) -- (\x,0) node[anchor=north] {$\x$};
    }
    \foreach \y in {-3,-2,-1,1} {
        \draw (0.2,\y) -- (0,\y) node[anchor=east] {\y};
    }

    \coordinate (O) at (0, 0);
    \coordinate (P) at (2, -3);
    \draw(O) -- ($(O)!1.1!(P)$) node[anchor=north] {$\alpha$的终边};
    \draw [fill=black] (P) circle(1pt) node[anchor=west] {$P(x, y)$};
\end{tikzpicture}

    \vspace{-20pt}
    \caption{}\label{fig:2-13}
\end{wrapfigure}

\jie $\because x = 2, \quad y = -3,$

$\therefore \quad r = \sqrt{2^2 + (-3)^2} = \sqrt{13}.$

\begin{minipage}{7.7cm}
\begin{align*}
    \therefore \quad & \sin \alpha = \dfrac y r = \dfrac{-3}{\sqrt{13}} = -\dfrac{-3\sqrt{13}}{13}, \\
        & \cos \alpha = \dfrac x r = \dfrac{2}{\sqrt{13}} = \dfrac{2\sqrt{13}}{13}, \\
        & \tan \alpha = \dfrac y x = -\dfrac 3 2, \quad \cot \alpha = \dfrac x y = -\dfrac 2 3, \\
        & \sec \alpha = \dfrac r x = \dfrac{\sqrt{13}}{2}, \quad \cot \alpha = \dfrac r y = -\dfrac{\sqrt{13}}{3}, \\
\end{align*}
\end{minipage}

\liti 求下列各角的六个三角函数值:
\begin{xiaoxiaotis}

    \threeInLine[8em]{\xiaoxiaoti{$0$;}}{\xiaoxiaoti{$\pi$;}}{\xiaoxiaoti{$\dfrac{3\pi}{2}$。}}

\end{xiaoxiaotis}

\jie (1) $\because$ 当 $\alpha = 0$ 时,$x = r$,$y = 0$,

$\therefore$
\begin{tabular}[t]{p{8em}p{8em}}
    $\sin 0 = 0$, & $\cos 0 = 1$,\\
    $\tan 0 = 0$, & $\cot 0$ 不存在,\\
    $\sec 0 = 1$, & $\csc 0$ 不存在。
\end{tabular}

(2) $\because$ 当 $\alpha = \pi$ 时,$x = -r$,$y = 0$,

$\therefore$
\begin{tabular}[t]{p{8em}p{8em}}
    $\sin \pi = 0$, & $\cos \pi = -1$,\\
    $\tan \pi = 0$, & $\cot \pi$ 不存在,\\
    $\sec \pi = -1$, & $\csc \pi$ 不存在。
\end{tabular}

(3) $\because$ 当 $\alpha = \dfrac{3\pi}{2}$ 时,$x = 0$,$y = -r$,

$\therefore$
\begin{tabular}[t]{p{8em}p{8em}}
    $\sin \dfrac{3\pi}{2} = -1$, & $\cos \dfrac{3\pi}{2} = 0$,\\
    \rule{0pt}{2em} $\tan \dfrac{3\pi}{2}$ 不存在, & $\cot \dfrac{3\pi}{2} = 0$,\\
    \rule{0pt}{2em} $\sec \dfrac{3\pi}{2}$ 不存在, & $\csc \dfrac{3\pi}{2} = -1$。
\end{tabular}

\vspace{0.5em}
由三角函数的定义和各象限内点的坐标的符号知道:

\vspace{0.5em}
正弦值 $\left( \dfrac y r \right)$ \vspace{0.5em} 与余割值 $\left( \dfrac r y \right)$ 对于第一、二象限的角是正的 $(y > 0, \, r > 0)$,而对于第三、四象限的角是负的 $(y < 0, \, r > 0)$;

\vspace{0.5em}
余弦值 $\left( \dfrac x r \right)$ \vspace{0.5em} 与正割值 $\left( \dfrac r x \right)$ 对于第一、四象限的角是正的 $(x > 0, \, r > 0)$,而对于第二、三象限的角是负的 $(x < 0, \, r > 0)$;

\vspace{0.5em}
正切值 $\left( \dfrac y x \right)$ \vspace{0.5em} 与余切值 $\left( \dfrac x y \right)$ 对于第一、三象限的角是正的($x$,$y$ 同号),而对于第二、四象限的角是负的($x$,$y$ 异号)。

各三角函数值在每个象限的符号,如图 \ref{fig:2-14} 所示。

\begin{figure}[htbp]
    \centering
    \begin{tikzpicture}[>=Stealth]
    \foreach \pos / \flga / \flgb / \flgc / \flgd / \name in
        {
            0 / $+$ / $+$ / $-$ / $-$ / {$\sin \alpha,\, \csc \alpha$},
            4 / $+$ / $-$ / $-$ / $+$ / {$\cos \alpha,\, \sec \alpha$},
            8 / $+$ / $-$ / $+$ / $-$ / {$\tan \alpha,\, \cot \alpha$}
        }
    {
        \draw [->] (\pos-1.5,0) -- (\pos+1.5,0) node[anchor=north] {$x$};
        \draw [->] (\pos,-1.5) -- (\pos,1.5) node[anchor=east] {$y$};
        \node at (\pos-0.3,-0.3) {$O$};

        \node at (\pos+0.7, 0.7) {\flga};
        \node at (\pos-0.7, 0.7) {\flgb};
        \node at (\pos-0.7, -0.7) {\flgc};
        \node at (\pos+0.7, -0.7) {\flgd};
        \node at (\pos, -1.7) {\name};
    }
\end{tikzpicture}

    \caption{}\label{fig:2-14}
\end{figure}

根据三角函数的定义还可以知道,\textbf{终边相同的角的同一三角函数的值相等}。由此得到一组公式(\textbf{公式一}\mylabel{gongshi:1}):

\begin{center}
    \framebox{\begin{tabular}{p{13em}p{16em}}
            $\sin (k \cdot 360^\circ + \alpha) = \sin \alpha,$ & $\cos (k \cdot 360^\circ + \alpha) = \cos \alpha ,$ \\
            $\tan (k \cdot 360^\circ + \alpha) = \tan \alpha,$ & $\cot (k \cdot 360^\circ + \alpha) = \cot \alpha .  \quad(k \in Z)$
    \end{tabular}}
\end{center}

利用 \hyperref[gongshi:1]{公式一} 可以把求任意角的三角函数值的问题,转化为求 $0^\circ$ \~{} $360^\circ$($0$ \~{} $2\pi$)间角的三角函数值的问题。

\liti 确定下列各三角函数值的符号:
\begin{xiaoxiaotis}

    \begin{tabular}[t]{*{4}{@{}p{9em}}}
        \xiaoxiaoti{$\cos 250^\circ$;} & \xiaoxiaoti{$\sin \left(-\dfrac \pi 4 \right)$;}
            & \xiaoxiaoti{$\tan (-672^\circ 10')$;} & \xiaoxiaoti{$\cot \dfrac{11\pi}{3}$。}
    \end{tabular}
    \vspace{0.5em}
    
\end{xiaoxiaotis}

\jie (1)因为 $250^\circ$ 是第三象限的角,所以 $\cos 250^\circ < 0$;

\vspace{0.5em}
(2)因为 $-\dfrac \pi 4$ 是第四象限的角,所以 $\sin \left(-\dfrac \pi 4 \right) < 0$;
\vspace{0.5em}

(3)因为 $\tan (-672^\circ 10') = \tan(-2 \times 360^\circ + 47^\circ 50') = \tan 47^\circ 50'$,
而 $47^\circ 50'$ 是第一象限的角,所以 $\tan (-672^\circ 10') > 0$;

\vspace{0.5em}
(4)因为 $\cot \dfrac{11\pi}{3} = \cot(2\pi + \dfrac 5 3 \pi) = \cot \dfrac 5 3 \pi$,
而 $\dfrac 5 3 \pi$ 是第四象限的角,所以 $\cot \dfrac{11\pi}{3} < 0$。
\vspace{0.5em}

\liti 根据条件 $\sin \theta < 0$ 且 $\tan \theta > 0$,确定 $\theta$ 是第几象限的角。

\jie $\because \quad \sin \theta < 0$,

$\therefore \quad \theta$ 在第三象限或第四象限,或 $\theta$ 的终边在 $y$ 轴的负半轴上;

$\because \quad \tan \theta > 0$,

$\therefore \quad \theta$ 在第一象限或第三象限。

$\because \quad \sin \theta < 0$ 与 $\tan \theta > 0$ 同时成立,

$\therefore \quad \theta$ 在第三象限。

\liti 求下列各三角函数值:
\begin{xiaoxiaotis}
    
    \vspace{0.5em}
    \begin{tabular}[t]{*{3}{@{}p{9em}}}
        \xiaoxiaoti{$\sin 1480^\circ 10'$;} & \xiaoxiaoti{$\cos \dfrac{9\pi}{4}$;}
            & \xiaoxiaoti{$\tan \left(-\dfrac{7\pi}{6} \right)$。}
    \end{tabular}
    \vspace{0.5em}
    
\end{xiaoxiaotis}

\jie (1)$\sin 1480^\circ 10' = \sin(4 \times 360^\circ + 40^\circ 10') = \sin 40^\circ 10' = 0.6451$;

\vspace{0.5em}
(2)$\cos \dfrac{9\pi}{4} = \cos \left( 2\pi + \dfrac \pi 4 \right) = \cos \dfrac \pi 4 = \dfrac{\sqrt{2}}{2}$;
\vspace{0.5em}

(3)$\tan \left(-\dfrac{7\pi}{6} \right) = \tan \left(-2\pi + \dfrac{5\pi}{6} \right) = \tan \dfrac{5\pi}{6} = \tan \left(\pi - \dfrac{\pi}{6} \right) = - \dfrac{\pi}{6} = -\dfrac{\sqrt{3}}{3}$。
\vspace{0.5em}

\lianxi
\begin{xiaotis}

\xiaoti{已知角 $\alpha$ 的终边经过点 $P(-3, -1)$,求 $\alpha$ 的六个三角函数值。}

\xiaoti{填写下表:}

\begin{table}[H]
\renewcommand\arraystretch{1.5}
\hspace{4em}
\begin{tabular}{|c|*{5}{w{c}{4em}|}}
    \hline
    $\alpha$ & $0^\circ$ & $90^\circ$ & $180^\circ$ & $270^\circ$ & $360^\circ$ \\ \hline
    角 $\alpha$ 的弧度数 & \eline{5} \\ \hline
    $\sin \alpha$ & \eline{5} \\ \hline
    $\cos \alpha$ & \eline{5} \\ \hline
    $\tan \alpha$ & \eline{5} \\ \hline
    $\cot \alpha$ & \eline{5} \\ \hline
\end{tabular}
\end{table}
    
\xiaoti{(口答)设 $\alpha$ 是三角形的一个内角,在 $\sin \alpha$,$\cos \alpha$,
    $\tan \alpha$,$\cot \dfrac \alpha 2$ 中,哪些有可能取负值?}

\xiaoti{确定下列各三角函数值的符号:}
\begin{xiaoxiaotis}

    \renewcommand\arraystretch{1.5}
    \begin{tabular}[t]{*{3}{@{}p{9em}}}
        \xiaoxiaoti{$\csc 156^\circ$;} & \xiaoxiaoti{$\cos \dfrac{16}{5} \pi$;} & \xiaoxiaoti{$\sec (-80^\circ)$;} \\
        \xiaoxiaoti{$\cot \left( -\dfrac{17}{8} \pi \right)$;} & \xiaoxiaoti{$\sin \left( -\dfrac{4\pi}{3} \right)$;} & \xiaoxiaoti{$\tan 556^\circ 12'$。}
    \end{tabular}
    \vspace{0.5em}
    
\end{xiaoxiaotis}

\xiaoti{根据下列条件,确定 $\theta$ 是第几象限的角:}
\begin{xiaoxiaotis}

    \begin{tabular}[t]{*{2}{@{}p{16em}}}
        \xiaoxiaoti{$\sin \theta < 0$ 且 $\cos \theta > 0$;} & \xiaoxiaoti{$\sec \theta < 0$ 且 $\cot \theta < 0$;} \\
        \xiaoxiaoti{$\sin \theta$ 与 $\tan \theta$ 同号;} & \xiaoxiaoti{$\cos \theta$ 与 $\tan \theta$ 异号。}
    \end{tabular}
    
\end{xiaoxiaotis}

\xiaoti{求下列各三角函数值:}
\begin{xiaoxiaotis}

    \begin{tabular}[t]{*{4}{@{}p{9em}}}
        \xiaoxiaoti{$\cos 1109^\circ$;} & \xiaoxiaoti{$\tan \dfrac{19\pi}{3}$;} &
        \xiaoxiaoti{$\sin (-1290^\circ)$;} & \xiaoxiaoti{$\cot \left( -\dfrac{29\pi}{4} \right)$。}
    \end{tabular}
    \vspace{0.5em}

\end{xiaoxiaotis}



\end{xiaotis}