{\centering \nonumsubsection{A \hspace{1em} 组}}

\begin{xiaotis}

\xiaoti{写出与下列各角终边相同的角的集合,并且把集合中在 $-2\pi$ ~ $4\pi$ 之间的角写出:}
\begin{xiaoxiaotis}

    \jiange
    \fourInLineXxt{$\dfrac \pi 4$;}{$-\dfrac 2 3 \pi$;}{$\dfrac{12}{5} \pi$;}{$0$。}
    \jiange

\end{xiaoxiaotis}

\xiaoti{在半径等于 $15cm$ 的圆中,一扇形的弧含有 $54^\circ$,求这扇形的周长和面积
    ($\pi$ 取 $3.14$,计算结果保留两个有效数字)。}

\xiaoti{如图,两轮的半径分别为 $R$,$r$($R > r$),$O'E \perp AO$,$\angle EO'O = \alpha$,
    求连接两轮的皮带传动装置的皮带长。}

\begin{figure}[htbp]
    \centering
    \begin{tikzpicture}
    \coordinate [label=180:$O$] (O) at (0,0);
    \draw (O) circle(2.5);
    \coordinate [label=90:$A$] (A) at (82:2.5);
    \coordinate [label=270:$C$] (C) at (-82:2.5);
    \draw [dashed] (O) -- (A);
    \draw [dashed] (O) -- (C);

    \coordinate [label=0:$O'$] (O') at (7,0);
    \path (O') ++(82:2.5) coordinate (X1); % 设点X1,使 OA // O'X1
    \path (O') ++(-82:2.5) coordinate (X2); % 设点X2,使 OC // O'X2
    \coordinate [label=90:$B$] (B) at ($(O')!(A)!(X1)$); % AB 丄 O'X1(也即 BA 丄 OA)
    \coordinate [label=270:$D$] (D) at ($(O')!(C)!(X2)$); % CD 丄 O'X2(也即 DC 丄 OC)
    \node [draw] at (O') [circle through={(B)}] {}; % 以 O' 为圆心,过点 B 绘制圆
    \draw [dashed] (O') -- (B);
    \draw [dashed] (O') -- (D);
    \draw (A) -- (B);
    \draw [right angle symbol={A}{B}{O}];
    \draw [right angle quadrant=2, right angle symbol={O'}{X1}{A}];
    \draw (C) -- (D);
    \draw [right angle symbol={C}{D}{O}];
    \draw [right angle quadrant=2, right angle symbol={O'}{X2}{C}];

    \draw (O) -- (O');
    \coordinate [label=180:$E$] (E) at ($(O)!(O')!(A)$);
    \draw [dashed] (O') -- (E);
    \draw [right angle quadrant=2, right angle symbol={O}{A}{O'}];

    \draw (O') +(-2.5,0) arc (180:172:2.5) (4.2, 0.2) node {$\alpha$}; % 注:这里的 172 度,并不是计算出来的,而是人工观察图片,调整出的值。
    \node at (0.5, 1.7) {$R$};
    \node at (7.4, 0.8) {$r$};
\end{tikzpicture}

    \caption*{(第 3 题)}
\end{figure}

\xiaoti{$0^\circ$ \~{} $360^\circ$ 间的角 $\alpha$ 的正弦、余弦、正切、余切的定义是怎样的?当 $\alpha$ 为锐角时,
    $90^\circ - \alpha$ 的三角函数与 $\alpha$ 的三角函数之间有什么关系?
    $180^\circ - \alpha$ 的三角函数与 $\alpha$ 的三角函数之间有什么关系?}

\xiaoti{$30^\circ$、$45^\circ$、$60^\circ$ 角的正弦、余弦、正切值是怎样求得的?}

\xiaoti{已知直角三角形中的一个锐角为 $\alpha$,那么 $\sin \alpha$、$\cos \alpha$、
    $\tan \alpha$、$\cot \alpha$ 与 $\alpha$ 的对边、邻边、斜边之间有什么关系?}

\xiaoti{写出余弦定理和正弦定理,怎样用它们来解各种类型的三角形?}

\xiaoti{确定下列各三角函数值的符号:}
\begin{xiaoxiaotis}

    \fourInLineXxt{$\sin 4$;}{$\cos 5$;}{$\tan 8$;}{$\cot(-3)$。}

\end{xiaoxiaotis}

\xiaoti{已知 $\cos \varphi = \dfrac 1 4$,求 $\sin \varphi$,$\tan \varphi$。}
\jiange

\xiaoti{已知 $\sin x = 2\cos x$,求角 $x$ 的六个三角函数值。}

\xiaoti{化简:}
\begin{xiaoxiaotis}

    \xiaoxiaoti{$\cos\alpha \cdot \csc\alpha \cdot \sqrt{\sec^2 \alpha - 1}$($\alpha$ 为第四象限的角);}

    \jiange
    \xiaoxiaoti{$\dfrac{\sqrt{1 - \sin^2 \alpha}}{\sqrt{1 - \cos^2 \alpha}} - \sqrt{\csc^2 \alpha - 1}$($\alpha$ 为第二象限的角);}
    \jiange

    \xiaoxiaoti{$\dfrac{1 - \sin^2 \varphi}{1 - \cos^2 \varphi} + 1 - \dfrac{1}{\sin^2 \varphi}$;}
    \jiange

    \xiaoxiaoti{$\dfrac{1}{\cos\alpha \sqrt{1 + \tan^2 \alpha}} + \dfrac{2\tan\alpha}{\sqrt{\dfrac{1}{\cos^2\alpha} - 1}}$;}
    \jiange

    \xiaoxiaoti{$\dfrac{\sqrt{1 - 2\sin 10^\circ \cos 10^\circ}}{\cos 10^\circ - \sqrt{1 - \cos^2 170^\circ}}$。}
    \jiange

\end{xiaoxiaotis}

\jiange
\xiaoti{}
\begin{xiaoxiaotis}

    \vspace{-1.7em} \begin{minipage}{0.9\textwidth}
    \xiaoxiaoti{用 $\cos\alpha$ 来表示 $\sin^4 \alpha - \sin^2 \alpha + \cos^2 \alpha$;}
    \end{minipage}

    \jiange
    \xiaoxiaoti{用 $\sin\alpha$ 来表示 $\dfrac{\sec^2 \alpha - \tan^2 \alpha}{\cos^2 \alpha}$;}
    \jiange

    \xiaoxiaoti{用 $\sec\alpha$ 来表示 $\dfrac{\sin^4 \alpha - \cos^4 \alpha}{\sin^2 \alpha - \cos^2 \alpha} + (1 + \tan^2 \alpha) \cos\alpha$;}
    \jiange

    \xiaoxiaoti{用 $\cot\alpha$ 来表示 $\dfrac{1 + \sin^2 \alpha}{\sin^2 \alpha} + \dfrac{\cos^2 \alpha}{1 - \cos^2 \alpha} + \dfrac{1}{\sec^2 \alpha - 1}$。}
    \jiange

\end{xiaoxiaotis}

\xiaoti{求证下列恒等式:}
\begin{xiaoxiaotis}

    \xiaoxiaoti{$(\sin x + \cos x)(\tan x + \cot x) = \sec x + \csc x$;}

    \jiange
    \xiaoxiaoti{$\dfrac{1 - 2\sin^2 \alpha}{\sin\alpha \cos\alpha} = \cot\alpha - \tan\alpha$;}
    \jiange

    \xiaoxiaoti{$\dfrac{1}{1 + \tan^2 \alpha} + \dfrac{1}{1 + \cot^2 \alpha} = \dfrac{1}{1 + \sin^2 \alpha} + \dfrac{1}{1 + \csc^2 \alpha}$;}
    \jiange

    \xiaoxiaoti{$2(1 - \sin\alpha)(1 + \cos\alpha) = (1 - \sin\alpha + \cos\alpha)^2$;}

    \xiaoxiaoti{$\sin^2 \alpha + \sin^2 \beta - \sin^2 \alpha \cdot \sin^2 \beta + \cos^2 \alpha \cdot \cos^2 \beta = 1$;}

    \xiaoxiaoti{$(1 - \tan^2 A)^2 = (\sec^2 A - 2\tan A)(\sec^2 A + 2\tan A)$。}

\end{xiaoxiaotis}

\xiaoti{已知 $\tan\alpha = 3$,计算:}
\begin{xiaoxiaotis}

    \renewcommand\arraystretch{1.5}
    \begin{tabular}[t]{*{2}{@{}p{14em}}}
        \xiaoxiaoti {$\dfrac{4\sin\alpha - 2\cos\alpha}{5\cos\alpha + 3\sin\alpha}$;} & \xiaoxiaoti {$\dfrac 2 3 \sin^2 \alpha + \dfrac 1 4 \cos^2 \alpha$;} \\
        \xiaoxiaoti {$\sin\alpha \cos\alpha$;} & \xiaoxiaoti {$(\sin\alpha + \cos\alpha)^2$。}
    \end{tabular}

\end{xiaoxiaotis}

\xiaoti{计算:}
\begin{xiaoxiaotis}

    \xiaoxiaoti{$\sin 420^\circ \cos 750^\circ + \sin(-330^\circ) \cos(-600^\circ)$;}

    \xiaoxiaoti{$\tan 675^\circ + \cot 765^\circ - \tan(-300^\circ) + \cot(-690^\circ)$;}

    \jiange
    \xiaoxiaoti{$\sin\dfrac{25\pi}{6} + \cos\dfrac{25\pi}{3} + \tan\left( -\dfrac{25\pi}{4} \right)$;}
    \jiange

    \xiaoxiaoti{$\sin 2 + \cos 3 + \tan 4$。}

\end{xiaoxiaotis}

\jiange
\xiaoti{已知 $\sin(\pi + \alpha) = -\dfrac 1 2$,计算:}
\begin{xiaoxiaotis}

    \fourInLineXxt{$\cos(2\pi - \alpha)$;}{$\sec(5\pi - \alpha)$;}{$\tan(\alpha - 7\pi)$;}{$\cot(3\pi + \alpha)$。}

\end{xiaoxiaotis}

\xiaoti{求下列各三角函数值:}
\begin{xiaoxiaotis}

    \xiaoxiaoti{$\sin 378^\circ 21'$,$\cos 742.5^\circ$,$\tan 1111^\circ$,$\cot 370^\circ 15'$;}

    \jiange
    \xiaoxiaoti{$\sin(-879^\circ)$,$\tan\left( -\dfrac{33\pi}{8} \right)$,$\cos\left( -\dfrac{13}{10} \pi \right)$,$\cot(-1.2\pi)$;}
    \jiange

    \xiaoxiaoti{$\sin 3$,$\cot(-3)$,$\cos(\sin 2)$。}

\end{xiaoxiaotis}

\xiaoti{设 $\pi < x < 2\pi$,填写下表:}

\begin{table}[H]
\renewcommand\arraystretch{2}
\begin{tabular}{|w{c}{5em}|*{6}{w{c}{4em}|}}
    \hline
    $x$ & $\dfrac{7\pi}{6}$ & \eline{3} & $\dfrac{7\pi}{4}$ & \\ \hline
    $\sin x$ & \eline{6} \\ \hline
    $\cos x$ & & $-\dfrac{\sqrt{2}}{2}$ & \eline{3} & $\dfrac{\sqrt 3}{2}$ \\ \hline
    $\tan x$ & \eline{2} & $1$ & \eline{3} \\ \hline
    $\cot x$ & \eline{3} & $-\dfrac{\sqrt 3}{3}$ & \eline{2} \\ \hline
\end{tabular}
\end{table}

\xiaoti{求适合下列条件的 $x$ 的集合:}
\begin{xiaoxiaotis}

    \begin{tabular}[t]{*{2}{@{}p{14em}}}
        \xiaoxiaoti{$\sin x = 0$;} & \xiaoxiaoti{$\cos x = -0.6124$;} \\
        \xiaoxiaoti{$\cos x = 0$;} & \xiaoxiaoti{$\sin x = 0.1011$;} \\
        \xiaoxiaoti{$\tan x = -4$;} & \xiaoxiaoti{$\cot x = 6.754$。}
    \end{tabular}

\end{xiaoxiaotis}

\xiaoti{已知 $\alpha$ 是 $0$ \~{} $2\pi$ 间的一个角,利用单位圆证明:角 $\alpha$
    的正弦的绝对值与角 $\alpha$ 的余弦的绝对值之和不可能小于 $1$。}

\xiaoti{确定下列函数的定义域:}
\begin{xiaoxiaotis}

    \renewcommand\arraystretch{1.5}
    \begin{tabular}[t]{*{2}{@{}p{14em}}}
        \xiaoxiaoti{$y = \dfrac{1}{1 - \tan x}$;} & \xiaoxiaoti{$y = \dfrac{\cot x}{\cos x - \dfrac 1 2}$;} \\
        \xiaoxiaoti{$y = \tan \dfrac x 2$;} & \xiaoxiaoti{$y = 2\cot\left( 2x - \dfrac \pi 3 \right)$。}
    \end{tabular}
    \jiange

\end{xiaoxiaotis}


\xiaoti{下列各式能不能成立?为什么?}
\begin{xiaoxiaotis}

    \renewcommand\arraystretch{1.5}
    \begin{tabular}[t]{*{2}{@{}p{14em}}}
        \xiaoxiaoti{$\cos^2 x = 1.5$;} & \xiaoxiaoti{$\sin x - \cos x = 2.5 $;} \\
        \xiaoxiaoti{$\tan x + \cot x = 2$;} & \xiaoxiaoti{$\sin^3 x = -\dfrac \pi 4$。}
    \end{tabular}
    \jiange

\end{xiaoxiaotis}

\xiaoti{求下列各函数的最大值、最小值,并且求使函数取得最大值、最小值的 $x$ 的集合:}
\begin{xiaoxiaotis}

    \jiange
    \twoInLineXxt[14em]{$y = \sqrt{2} + \dfrac{\sin x}{\pi}$;}{$y = 3 - 2\cos x$。}
    \jiange

\end{xiaoxiaotis}

\xiaoti{己知 $0 \leqslant x \leqslant 2\pi$,当 $x$ 属于哪个区间时,}
\begin{xiaoxiaotis}

    \xiaoxiaoti{角 $x$ 的正弦函数、余弦函数都是增函数?}

    \xiaoxiaoti{角 $x$ 的正弦函数、余弦函数都是减函数?}

    \xiaoxiaoti{角 $x$ 的正弦函数是增函数,而余弦函数是减函数?}

    \xiaoxiaoti{角 $x$ 的正弦函数是减函数,而余弦函数是增函数?}

\end{xiaoxiaotis}

\xiaoti{确定下列函数哪些是偶函数,哪些是奇函数:}
\begin{xiaoxiaotis}

    \begin{tabular}[t]{*{3}{@{}p{13em}}}
        \xiaoxiaoti{$y = \sec x$;} & \xiaoxiaoti{$y = \csc x$;} & \xiaoxiaoti{$y = x^2 + \cos x$;} \\
        \xiaoxiaoti{$y = |2\sin x|$;} & \xiaoxiaoti{$y = \tan x^2$;} & \xiaoxiaoti{$y = x^2 \sin x$。}
    \end{tabular}

\end{xiaoxiaotis}

\xiaoti{作出下列函数在长度为一个周期的闭区间上的简图:}
\begin{xiaoxiaotis}

    \renewcommand\arraystretch{1.5}
    \begin{tabular}[t]{*{2}{@{}p{14em}}}
        \xiaoxiaoti{$y = \dfrac 1 2 \sin \left( 3x - \dfrac \pi 3 \right)$;} & \xiaoxiaoti{$y = -2 \sin \left( x + \dfrac \pi 4 \right)$;} \\
        \xiaoxiaoti{$y = 1 - \sin \left( 2x - \dfrac \pi 5 \right)$;} & \xiaoxiaoti{$y = 3 \sin \left( \dfrac \pi 6 - \dfrac x 3 \right)$。}
    \end{tabular}
    \jiange
\end{xiaoxiaotis}

\jiange
\xiaoti{}
\begin{xiaoxiaotis}

    \vspace{-1.7em} \begin{minipage}{0.9\textwidth}
    \xiaoxiaoti{用描点法作函数 $y = \sin x$在 $\left[ 0, \, \dfrac \pi 2 \right]$ 上的图象;}
    \end{minipage}

    \xiaoxiaoti{根据(1),如何再运用函数 $y = \sin x$ 的性质得到它在 $[0, \, 2\pi]$ 上的图象?}

    \xiaoxiaoti{根据(2),如何通过移动坐标轴得到 $y = \sin(x + \varphi) + k$($\varphi$,$k$ 都是常数)的图象?}

\end{xiaoxiaotis}

\xiaoti{不画图,写出下列各函数的振幅、周期、初相,并说明这些函数的图象可由正弦曲线 $y = \sin x$ 经过怎样的变化得出。}
\begin{xiaoxiaotis}

    \jiange
    \twoInLineXxt[14em]{$y = \sin\left( 5x + \dfrac \pi 6 \right)$;}{$y = 2\sin \dfrac 1 6 x$。}
    \jiange

\end{xiaoxiaotis}

% TODO: wrapfigure 在这里无法正常使用


\begin{minipage}{0.7\textwidth}

\xiaoti{弹簧挂着的小球作上下振动,它在 $t$ 秒时相对于平衡位置(就是静止时的位置)的高度 $h$ 厘米由下列关系决定:
    $$ h = 2\sin\left( t + \dfrac \pi 4 \right) \text{。}$$
    以 $t$ 为横坐标,$h$ 为纵坐标,作出这个函数在长度为一个周期的闭区间上的图象,并且回答下列问题:}
\begin{xiaoxiaotis}

    \xiaoxiaoti{小球在开始振动时(即 $t = 0$ 时)的位置在哪里?}

    \xiaoxiaoti{小球的最高点和最低点与平衡位置的距离分别是多少?}

    \xiaoxiaoti{经过多少时间小球往复振动一次(周期)?}

    \xiaoxiaoti{每秒钟小球能往复振动多少次(频率)?}

\end{xiaoxiaotis}

\end{minipage}
\begin{minipage}{0.2\textwidth}
    \centering
    \begin{tikzpicture}[>=Stealth]
    \coordinate (O) at (0, 0);
    \draw (O) circle(3.5mm);
    \draw (O) ++(40:2.9mm) arc (40:-100:2.9mm);
    \draw (O) ++(20:2.3mm) arc (20:-80:2.3mm);
    \draw (0, 3.5mm) -- (0, 0.6);
    \draw[decoration={aspect=0.3, segment length=1.5mm, amplitude=3mm,coil},decorate] (0,0.6) -- (0, 2.6);
    \draw (0, 2.6) -- (0, 3);
    \fill [pattern = north east lines] (-1,3) rectangle (1,3.2);
    \draw[thick] (-1,3) -- (1,3);

    \draw [<->, thin] (0.6,-0.6) -- (0.6,0.6);
    \draw [fill] (0.6, 0) circle(1pt);
    \draw (1.2, 0.5) node {$h > 0$}
          (1.2, 0) node {$h = 0$}
          (1.2, -0.5) node {$h < 0$};
\end{tikzpicture}


    (第 29 题)
\end{minipage}

\end{xiaotis}
