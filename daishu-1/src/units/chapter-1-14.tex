\subsection{指数方程和对数方程}\label{subsec:1-14}

在指数里含有未知数的方程叫做\textbf{指数方程},在对数符号后面含有未知数的方程叫做\textbf{对数方程}。
在这两类方程中,我们只能解一些特殊的方程。现在举一些例子。

\liti 解方程 $4^x = 2^{x+1}$

\jie 原方程可化为
$$2^{2x} = 2^{x+1} \text{。}$$

同一个底$a$(这里 $a > 0$,且 $a \neq 1$)的幂相等,必须并且只需它们的幂指数相等。因此上式就是
\begin{align*}
    2x &= x + 1 \text{。} \\
    \therefore \qquad x &= 1
\end{align*}

\liti 电视机厂生产的电视机台数,如果每年平均比上一年增长 $10.4\%$,
那么约经过多少年可以增长到原来的 $2$ 倍(结果保留一个有效数字)?

\jie 设经过 $x$ 年可以增长到原来的 $2$ 倍,根据题意,得
$$(1 +10.4\%)^x = 2 \text{。}$$

两边取对数,得

\begin{gather*}
    x \lg 1.104 = \lg 2 \text{。} \\
    \therefore \qquad x = \dfrac {\lg 2}{\lg 1.104} = \dfrac {0.3010}{0.0429} \approx 7 \text{。}
\end{gather*}

答:约经过 $7$ 年。

(注:在例2中,也可以把指数式 $(1 + 10.4\%)^x = 2$ 化为对数式 $\log_{1.104} 2 = x$,
再利用换底公式得到 $x = \dfrac {\lg 2}{\lg 1.104} \approx 7$。)
\vspace{0.5em}

\liti 解方程 $3^{x+1} + 9^x -18 = 0$。

\jie 原方程可化为
$$3 \cdot 3^x + (3^x)^2 - 18 = 0 \text{。}$$

利用换元法,设 $3^x = y$,方程又成为
$$y^2 + 3y - 18 = 0 \text{,}$$

由此解得
$$y_1 = 3, \quad y_2 = -6 \text{。}$$

由 $3^x = 3$,得 $x = 1$;另 $3^x = -6$ 不符合指数函数意义(性质(1)),应舍去。
所以原方程的解是 $x = 1$。

\liti 解方程 $\lg (x^2 + 11x + 8) - \lg (x + 1) = 1$。

\jie 把原方程化为
$$\lg \dfrac{x^2 + 11x + 8}{x + 1} = \lg 10 \text{。}$$

同一个底的对数相等,必须并且只需它们的真数(使对数有意文)相等。因此上式就是
$$\dfrac{x^2 + 11x + 8}{x + 1} = 10 \text{。}$$

解这个方程,得 $x_1 = -2$,$x_2 = 1$。

检验:$x = -2$ 时,$x + 1 = -1$,负数的对数没有意义,所以 $x = -2$ 不是原方程的根;
$x = 1$ 时,原方程的 $\text{左边} = \lg 20 - \lg2 = \lg 10 = 1 = \text{右边}$,
所以 $x = 1$ 是原方程的根。

注意:解对数方程时,必须对求得的根进行检验。因为在利用对数性质进行变形而得到新方程时,
如果未知数的字母的取值范围扩大,可能产生增根。如例4,由原方程变形为
\vspace{0.5em}
$$\lg \dfrac{x^2 + 11x + 8}{x + 1} = \lg 10$$

时,$x$ 的取值范围由 $\left\{ x \,\middle|\, 
\begin{cases}
    x^2 + 11x + 8 > 0, \\
    x + 1 > 0
\end{cases} \right\}$ 变为
$$
\left\{ x \,\middle|\, 
\begin{cases}
    x^2 + 11x + 8 > 0, \\
    x + 1 > 0
\end{cases} \right\}
\cup
\left\{ x \,\middle|\, 
\begin{cases}
    x^2 + 11x + 8 < 0, \\
    x + 1 < 0
\end{cases} \right\} \text{。}
$$

因为 $x = -2 \in \left\{ x \,\middle|\, 
\begin{cases}
    x^2 + 11x + 8 < 0, \\
    x + 1 < 0
\end{cases} \right\}$,所以方程产生了增根。

\liti 求方程 $x + \lg x = 3$ 的近似解。

\jie 在同一坐标系内画出 $y = \lg x$ 及 $y = 3 - x$ 的图象,求得交点的横坐标 $x \approx 2.6$(图\ref{fig:1-31}),
这个 $x$ 值近似地满足 $\lg x = 3 - x$,所以它就是原方程的近似解。

\begin{figure}[htbp]
    \centering
    \begin{tikzpicture}[>=Stealth]
    \draw [->] (-0.5,0) -- (5.5,0) node[anchor=north] {$x$};
    \draw [->] (0,-1.5) -- (0,3.5) node[anchor=east] {$y$};
    \node at (-0.3,-0.3) {$O$};
    \foreach \x in {1,2,...,5} {
        \draw (\x,0.2) -- (\x,0) node[anchor=north] {$\x$};
    }
    \foreach \y in {-1,1,2,3} {
        \draw (0.2,\y) -- (0,\y) node[anchor=east] {\y};
    }
    
    \draw[name path=a1,domain=0.05:5,samples=100] plot (\x, {log10(\x)}) +(-1, +0.3) node {$y = \lg x$};
    \draw[name path=a2,domain=-0.4:3.8] plot (\x, {3 - \x}) +(-2, +3.3) node {$y = 3 - x$};
    \draw [name intersections={of=a1 and a2, by=A}]
        let \p1 = (A)
        in [dash pattern=on 1mm off 0.5mm] (\x1,\y1) -- (\x1,0);
\end{tikzpicture}

    \caption{}\label{fig:1-31}
\end{figure}

\lianxi
\begin{xiaotis}

\xiaoti{解下列指数方程:}
\begin{xiaoxiaotis}

\renewcommand\arraystretch{1.5}
\begin{tabular}[t]{*{2}{@{}p{16em}}}
    \xiaoxiaoti {$2^{x-1} = 8$;} & \xiaoxiaoti {$3^{\frac 1 x} = 9$;} \\
    \xiaoxiaoti {$\left(\dfrac 1 4 \right)^{-x} = 64$;} & \xiaoxiaoti {$5^{(x-1)(x+2)} = 1$。}
\end{tabular}

\end{xiaoxiaotis}

\vspace{0.5em}
\xiaoti{利用常用对数解下列指数方程:}

\begin{xiaoxiaotis}

\begin{tabular}[t]{*{2}{@{}p{16em}}}
    \xiaoxiaoti {$10^x = 300$;} & \xiaoxiaoti {$2^y = 100$;} \\
    \xiaoxiaoti {$3^y = 12$;} & \xiaoxiaoti {$10^{4m} = 5.75$。}
\end{tabular}
    
\end{xiaoxiaotis}
    
\xiaoti{已知镭经过 $100$ 年剩留原来质量的 $95.76\%$,计算它约经过多少年剩留一半(结果保留四个有效数字)。}

\xiaoti{一个生产队去年粮食平均亩产量是 $817$ 斤,从今年起的 $5$ 年内,计划平均每年比上一年提高 $7\%$,约经过几年可以提高到亩产量 $1000$ 斤(结果保留一个有效数字)?}

\xiaoti{解下列对数方程:}

\begin{xiaoxiaotis}
    
    \twoInLine[16em]{\xiaoxiaoti{$2 \lg x + \lg 7 = \lg 14$;}}{\xiaoxiaoti{$\lg x + \lg (x - 3) = 1$;}}

    \xiaoxiaoti{$\lg (x + 6) - \dfrac 1 2 \lg (2x - 3) = 2 - \lg 25$;}

    \vspace{0.5em}
    \xiaoxiaoti{$\dfrac 1 2 (\lg x - \lg 5) = \lg 2 - \dfrac 1 2 \lg (9 - x)$。}
    \vspace{0.5em}
\end{xiaoxiaotis}

\xiaoti{用换元法解方程:}
\begin{xiaoxiaotis}
    
    \xiaoxiaoti{$5^{2x} - 23 \times 5^x - 50 = 0$;}

    \vspace{0.5em}
    \xiaoxiaoti{$\dfrac 1 {12} (\lg x)^2 = \dfrac 1 3 - \dfrac 1 4 \lg x$。}
    \vspace{0.5em}

\end{xiaoxiaotis}

\end{xiaotis}