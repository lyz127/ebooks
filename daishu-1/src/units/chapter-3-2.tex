\subsection{二倍角的正弦、余弦、正切}\label{subsec:3-2}

在公式 (\ref{eq:S-a+b}),(\ref{eq:C-a+b}),(\ref{eq:T-a+b}) 中,当 $\alpha = \beta$ 时,就可以得出相应的二倍角的三角函数公式:
\begin{gather}
    \sin(2\alpha) = 2\sin\alpha \cos\alpha \text{;} \tag{$S_{2\alpha}$}\label{eq:S-2a} \\
    \cos(2\alpha) = \cos^2\alpha - \sin^2\alpha \text{;} \tag{$C_{2\alpha}$}\label{eq:C-2a} \\
    \tan(2\alpha) = \dfrac{2\tan\alpha}{1 - \tan^2\alpha} \text{。} \tag{$T_{2\alpha}$}\label{eq:T-2a}
\end{gather}

\jiange
因为 $\sin^2\alpha + \cos^2\alpha = 1$,所以公式 (\ref{eq:C-2a}) 可以变形为
$$\cos 2\alpha = 2\cos^2\alpha - 1 \text{,}$$
或
\begin{gather}
    \cos 2\alpha = 1 - 2\sin^2\alpha \text{,} \tag{${C_{2\alpha}}'$}\label{eq:C-2a-'}
\end{gather}

有了二倍角的三角函数公式,就可以用单角的三角函数来表示二倍角的三角函数。

\jiange
\liti 已知 $\sin\alpha = \dfrac{5}{13}$,$\alpha \in \left( \dfrac \pi 2, \, \pi \right)$,求 $\sin 2\alpha$,$\cos 2\alpha$,$\tan 2\alpha$ 的值。\jiange

\jie $\because$ \quad $\sin\alpha = \dfrac{5}{13}$,$\alpha \in \left( \dfrac \pi 2, \, \pi \right)$,

\jiange
$\therefore \quad \cos\alpha = -\sqrt{1 - \sin^2\alpha} = -\sqrt{1 - \left( \dfrac{5}{13} \right)^2} = -\dfrac{12}{13}$。
\jiange

$\therefore \quad \begin{gathered}[t]
    \sin 2\alpha = 2\sin\alpha \cos\alpha = 2 \times \dfrac{5}{13} \times \left( -\dfrac{12}{13} \right) = -\dfrac{120}{169}, \\
    \cos 2\alpha = 1 - 2\sin^2\alpha = 1 - 2 \times \left( \dfrac{5}{13} \right)^2 = \dfrac{119}{169}, \\
    \tan 2\alpha = \dfrac{\sin 2\alpha}{\cos 2\alpha} = \dfrac{-120}{169} \left/ \dfrac{119}{169} = -1\dfrac{1}{119} \right. \text{。}
\end{gathered} \jiange$

\liti \quad \begin{minipage}[t]{0.8\textwidth}
    (1)用 $\sin\theta$ 表示 $\sin 3\theta$;

    (2)用 $\cos\theta$ 表示 $\cos 3\theta$;
\end{minipage}

\jie (1) $\begin{aligned}[t]
    \sin 3\theta &= \sin(2\theta + \theta) \\
        &= \sin 2\theta \cos\theta + \cos 2\theta \sin\theta \\
        &= 2\sin\theta \cos^2\theta + (1 - 2\sin^2\theta)\sin\theta \\
        &= 2\sin\theta(1 - \sin^2\theta) + \sin\theta -2\sin^3\theta \\
        &= 3\sin\theta - 4\sin^3\theta \text{。}
\end{aligned}$

$\therefore \quad \sin 3\theta = 3\sin\theta - 4\sin^3\theta \text{。}$

(2) $\begin{aligned}[t]
    \cos 3\theta &= \cos(2\theta + \theta) \\
        &= \cos2\theta \cos\theta - \sin2\theta \sin\theta \\
        &= (2\cos^2\theta - 1) \cos\theta - 2\sin^2\theta \cos\theta \\
        &= 2\cos^3\theta - \cos\theta - 2\cos\theta(1 - \cos^2\theta) \\
        &= 4\cos^3\theta -3\cos\theta \text{。}
\end{aligned}$

$\therefore \quad \cos3\theta = 4\cos^3\theta -3\cos\theta \text{。}$

\liti 求证
$$[\sin\theta(1 + \sin\theta) + \cos\theta(1 + \cos\theta)] \times [\sin\theta(1 - \sin\theta) + \cos\theta(1 - \cos\theta)] = \sin2\theta \text{。}$$

\zhengming $\begin{aligned}[t]
    \text{左边} &= (\sin\theta + \sin^2\theta + \cos^2\theta + \cos\theta) \times (\sin\theta - \sin^2\theta - \cos^2\theta + \cos\theta) \\
        &= (\sin\theta + \cos\theta + 1)(\sin\theta + \cos\theta - 1) \\
        &= (\sin\theta + \cos\theta)^2 - 1 \\
        &= 2\sin\theta \cos\theta \\
        &= \sin2\theta = \text{右边。}
\end{aligned}$

$\therefore$ \quad 原式成立。

\liti 化简 $\sin 50^\circ (1 + \sqrt 3 \tan 10^\circ)$。

\jie $\begin{aligned}[t]
      & \sin 50^\circ (1 + \sqrt 3 \tan 10^\circ) = \sin 50^\circ \left( 1 + \dfrac{\sqrt 3 \sin 10^\circ}{\cos 10^\circ} \right) \\
    = & \sin 50^\circ \cdot \dfrac{2\left( \dfrac 1 2 \cos 10^\circ + \dfrac{\sqrt 3}{2} \sin 10^\circ \right)}{\cos 10^\circ} \\
    = & 2\sin 50^\circ \cdot \dfrac{\sin 30^\circ \cos 10^\circ + \cos 30^\circ \sin 10^\circ}{\cos 10^\circ} \\
    = & 2\cos 40^\circ \cdot \dfrac{\sin 40^\circ}{\cos 10^\circ} \\
    = & \dfrac{\sin 80^\circ}{\cos 10^\circ} = \dfrac{\cos 10^\circ}{\cos 10^\circ} = 1 \text{。} \jiange
\end{aligned}$
\jiange

\liti 把一段半径为 $R$ 的圆木,锯成横截面为矩形的木料,怎样锯法才能使横截面的面积最大?

\begin{wrapfigure}[10]{r}{7.0cm}
    \centering
    \begin{tikzpicture}[>=Stealth, scale=0.8]
    \pgfmathsetmacro{\r}{3}
    \coordinate (O) at (0,0);

    \pgfmathsetmacro{\b}{45}
    \coordinate (S1) at (\b:\r);
    \coordinate (S2) at (90+\b:\r);
    \coordinate (S3) at (180+\b:\r);
    \coordinate (S4) at (270+\b:\r);
    \draw[name path=c] (O) circle(\r);
    \draw[dashed] (S1) -- (S2) -- (S3) -- (S4) -- (S1);
    \draw (S1) -- (S3);

    \pgfmathsetmacro{\a}{15}
    \path[name path=l1] (S1) -- +(180+\a:5.5);
    \path[name path=l2] (S3) -- +(\a:5.5);
    \path [name intersections={of=c and l1, by={t1,t2}}];
    \path [name intersections={of=c and l2, by={t3,t4}}];
    \draw (S1) -- (t2) -- (S3) -- (t4) -- (S1);

    \node[rotate=\b, above] at (O) {2R};
    \draw (S3)+(1.2,0.7) node {$\theta$} (S3)+(\a:1) arc (\a:\b:1);
    \draw (S3)+(\a:3) node[below,rotate=\a] {$2R\cos\theta$};
    \draw (t4)+(90+\a:1) node[above,rotate=90+\a] {$2R\sin\theta$};
\end{tikzpicture}

    %\vspace{-2em}
    \caption{}\label{fig:3-2}
\end{wrapfigure}

解:因为锯得的矩形横截面是圆内接矩形,所以它的对角线是圆的直径,其长度应为 $2R$。
设对角线与一个边的夹角为 $\theta$(图\ref{fig:3-2}),则矩形的长与宽分别为 $2R\cos\theta$,$2R\sin\theta$。
因此,矩形的面积
\begin{align*}
    S &= 2R\cos\theta \cdot 2R\sin\theta \\
      &= 2R^2 \cdot 2\sin\theta \cos\theta \\
      &= 2R^2 \cdot \sin2\theta \text{。}
\end{align*}

因为 $\sin2\theta \leqslant 1$,所以 $S \leqslant 2R^2$。

当 $\sin2\theta$ 取最大值 $1$ 时,$S$ 取最大值 $2R^2$。

所以,当 $2\theta = 90^\circ$,即 $\theta = 45^\circ$ 时,圆内接矩形的面积最大,这时圆内接矩形为内接正方形。

答:以圆木的直径为对角线,锯成横截面为正方形的木料时,截面的面积最大。

\lianxi
\begin{xiaotis}

\xiaoti{不查表,求下列各式的值:}
\begin{xiaoxiaotis}

    \renewcommand\arraystretch{1.8}
    \begin{tabular}[t]{*{2}{@{}p{16em}}}
        \xiaoxiaoti{$2\sin 67^\circ 30' \cos 67^\circ 30'$;} & \xiaoxiaoti{$\cos^2 \dfrac \pi 8 - \sin^2 \dfrac \pi 8$;} \\
        \xiaoxiaoti{$2\cos^2 \dfrac{\pi}{12} - 1$;} & \xiaoxiaoti{$1 - 2\sin^2 75^\circ$;} \\
        \xiaoxiaoti{$\dfrac{2\tan 22.5^\circ}{1 - \tan^2 22.5^\circ}$;} & \xiaoxiaoti{$\sin 15^\circ \cos 15^\circ$;} \\
        \xiaoxiaoti{$1 - 2\sin^2 750^\circ$;} & \xiaoxiaoti{$\dfrac{2\tan 150^\circ}{1 - \tan^2 150^\circ}$。}
    \end{tabular}
    \jiange
\end{xiaoxiaotis}

\xiaoti{化简:}
\begin{xiaoxiaotis}

    \renewcommand\arraystretch{1.8}
    \begin{tabular}[t]{*{2}{@{}p{16em}}}
        \xiaoxiaoti{$(\sin\alpha - \cos\alpha)^2$;} & \xiaoxiaoti{$\sin\dfrac \theta 2 \cos\dfrac \theta 2$;} \\
        \xiaoxiaoti{$\cos^4\varphi - \sin^4\varphi$;} & \xiaoxiaoti{$\dfrac{1}{1 - \tan\theta} - \dfrac{1}{1 + \tan\theta}$。}
    \end{tabular}
    \jiange
\end{xiaoxiaotis}

\xiaoti{已知 $\sin\alpha = 0.8$,$\alpha \in \left( 0, \, \dfrac \pi 2 \right)$,求 $\sin 2\alpha$,$\cos 2\alpha$ 的值。 \jiange}

\xiaoti{已知 $\cos\alpha = -\dfrac{12}{13}$,$\alpha \in \left( \dfrac \pi 2, \, \pi \right)$,求 $\cos 2\alpha$,$\sin 2\alpha$,$\tan 2\alpha$,$\cot 2\alpha$ 的值。 \jiange}

\xiaoti{已知 $\tan\alpha = \dfrac{1}{2}$,求 $\tan 2\alpha$,$\cot 2\alpha$ 的值。 \jiange}

\xiaoti{写出由 $\tan\alpha$ 求 $\tan 3\alpha$ 的公式。}

\xiaoti{证明下列恒等式:}
\begin{xiaoxiaotis}

    \jiange
    \xiaoxiaoti{$\sin^2\theta = \dfrac{1 - \cos2\theta}{2}$;\jiange}

    \xiaoxiaoti{$\cos^2\theta = \dfrac{1 + \cos2\theta}{2}$;\jiange}

    \xiaoxiaoti{$2\sin(\pi + \alpha) \cos(\pi - \alpha) = \sin2\alpha$;}

    \xiaoxiaoti{$\cos^4\dfrac{x}{2} - \sin^4\dfrac{x}{2} = \cos x$;\jiange}

    \xiaoxiaoti{$1 + 2\cos^2\theta - \cos2\theta = 2$;}

    \jiange
    \xiaoxiaoti{$\dfrac{1 - \cos2\alpha}{\sin\alpha} = 2\sin\alpha$;\jiange}

    \xiaoxiaoti{$\dfrac{2\cot\alpha}{\cot^2\alpha - 1} = \tan2\alpha$;\jiange}

    \xiaoxiaoti{$\dfrac{\sin2\theta}{1 - \cos2\theta} = \cot\theta$;\jiange}

    \xiaoxiaoti{$\cot\varphi - \cot2\varphi = \csc2\varphi$;}

    \jiange
    \xiaoxiaoti{$\dfrac{\sin\alpha \cos\alpha}{\sin^2\alpha - \cos^2\alpha} = -\dfrac{1}{2}\tan2\alpha$。\jiange}

\end{xiaoxiaotis}

\end{xiaotis}
