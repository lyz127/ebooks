\subsection{函数的奇偶性}

对于函数 $f(x) = 3x$,有 $f(-x) = -3x$,即 $f(-x) = -f(x)$;
而对于函数 $f(x) = x^2$,有 $f(-x) = (-x)^2 = x^2$,即 $f(-x) = f(x)$。

一般地,对于函数 $f(x)$:

1. 如果对于函数定义域内任意一个 $x$,都有 $f(-x) = -f(x)$,那么函数 $f(x)$ 就叫做\textbf{奇函数};

2. 如果对于函数定义域内任意一个 $x$,都有 $f(-x) = f(x)$,那么函数 $f(x)$ 就叫做\textbf{偶函数}。

\liti 判断下列函数是否具有奇偶性:

\begin{xiaoxiaotis}
    \begin{tabular}[t]{*{2}{@{}p{14em}}} 
        \xiaoxiaoti {$f(x) = x^3$;} & \xiaoxiaoti {$f(x) = 2x^4 + 3x^2$;} \\
        \xiaoxiaoti {$f(x) = x^3 + x^{-\frac 1 3}$;} & \xiaoxiaoti {$f(x) = x + 1$。}
    \end{tabular}
\end{xiaoxiaotis}

\jie (1) $f(-x) = (-x)^3 = -x^3$,即
$$f(-x) = -f(x) ,$$
所以 $f(x) = x^3$ 是奇函数。

(2) $f(-x) = 2(-x)^4 + 3(-x)^2 = 2x^4 + 3x^2$,即
$$f(-x) = f(x) ,$$
所以 $f(x) = 2x^4 + 3x^2$ 是偶函数。

(3) $f(-x) = (-x)^3 + (-x)^{-\frac 1 3} = -x^3 - x^{-\frac 1 3} = -(x^3 + x^{-\frac 1 3})$,即
$$f(-x) = -f(x) ,$$
所以 $f(x) = x^3 + x^{-\frac 1 3}$ 是奇函数。

(4) $f(-x) = -x + 1$,$-x + 1 \neq -f(x)$,而且 $-x + 1 \neq f(x)$,
所以 $f(x) = x + 1$ 既不是奇函数,也不是偶函数。

\liti 已知函数 $f(x)$ 是奇函数,而且在 $(0, +\infty)$ 上是增函数,
$f(x)$ 在 $(-\infty, 0)$ 上是增函数还是减函数?

\jie 设 $x_1 < 0$,$x_2 < 0$,而且 $x_1 < x_2$。
\begin{align}
    &\because \quad f(x) \text{是奇函数} \notag \\
    &\therefore \quad f(-x_1) = -f(x_1), \quad f(-x_2) = -f(x_2) \text{。} \tag{1}\label{eq:1-1}
\end{align}

由假设可知 $-x_1 > 0$,$-x_2 > 0$,而且 $-x_1 > -x_2$。又已知 $f(x)$ 在 $(0, +\infty)$ 上是增函数,于是有
\begin{gather}
    f(-x_1) > f(-x_2) \text{。} \tag{2}\label{eq:1-2}
\end{gather}

把 \eqref{eq:1-1} 代入  \eqref{eq:1-2},得 $-f(x_1) > -f(x_2)$,从而
$$f(x_1) < f(x_2) \text{。}$$
由此可知,函数 $f(x)$ 在 $(-\infty, 0)$ 上是增函数。

关于奇函数、偶函数的图象,有下面的定理。

\begin{theorem}
    奇函数的图象关于原点成中心对称图形;反过来,如果一个函数的
    图象关于原点成中心对称图形,那么这个函数是奇函数。
\end{theorem}

\zhengming 设函数 $f(x)$ 是奇函数,则有 $f(-x) = -f(x)$。如图 \ref{fig:1-20},在 $f(x)$ 的图象
上任取一点 $P(a, f(a))$,那么 $P$ 关于原点的对称点是点 $P'(-a, -f(a))$,即 $P'(-a, f(-a))$。
而点 $P'(-a, f(-a))$ 是 $f(x)$ 的图象上的点。这就是说,函数 $f(x)$ 图象上任意一点关于原点的对称点
都在 $f(x)$ 的图象上,所以 $f(x)$ 的图象关于原点成中心对称图形。

\begin{figure}[htbp]
    \centering
    \begin{minipage}{8cm}
    \centering
    \begin{tikzpicture}[>=Stealth]
    \draw [->] (-3.5,0) -- (3.5,0) node[anchor=north] {$x$};
    \draw [->] (0,-2) -- (0,2) node[anchor=east] {$y$};
    \node at (0.3,-0.3) {$O$};
    \draw [name path=a1] (0,0) .. controls(1,1.5) and (2,1) .. (2,0.8);
    \path [name path=a2] (1,0) -- (1,2);
    \draw [name intersections={of=a1 and a2, by=A}]
        let \p1 = (A)
        in (\x1,\y1) -- (\x1,0) node[anchor=north] {a}
           (-\x1,-\y1) -- (-\x1,0) node[anchor=south] {-a};
    \draw [rotate=180](0,0) .. controls(1,1.5) and (2,1) .. (2,0.8);
    \node at (1.5,1.3) {$P = (a, f(a))$};
    \node at (-1.5,-1.3) {$P = (-a, f(-a))$};
\end{tikzpicture}
    \caption{}\label{fig:1-20}
    \end{minipage}
    \qquad
    \begin{minipage}{8cm}
    \centering
    \begin{tikzpicture}[>=Stealth]
    \draw [->] (-3.5,0) -- (3.5,0) node[anchor=north] {$x$};
    \draw [->] (0,-2) -- (0,2) node[anchor=east] {$y$};
    \node at (0.3,-0.3) {$O$};
    \draw [name path=a1] (0,1) .. controls(0.8,1) and (1.6,0.3) .. (2.5,0.8);
    \draw [rotate around y=180] (0,1) .. controls(0.8,1) and (1.6,0.3) .. (2.5,0.8);
    \path [name path=a2] (1,0) -- (1,2);
    \draw [name intersections={of=a1 and a2, by=A}]
        let \p1 = (A)
        in (\x1,\y1) -- (\x1,0) node[anchor=north] {a}
           (-\x1,\y1) -- (-\x1,0) node[anchor=north] {-a};
    \node at (1.5,1.3) {$P = (a, f(a))$};
    \node at (-1.5,1.3) {$P = (-a, f(-a))$};
\end{tikzpicture}

    \caption{}\label{fig:1-21}
    \end{minipage}
\end{figure}

反过来,如果 $f(x)$ 的图象关于原点成中心对称图形,在 $f(x)$ 的图象上任取一点 $P(a, f(a))$,那么点
$P$ 关于原点的对称点 $P'(-a, -f(a))$ 也在 $f(x)$ 的图象上。因为 $x = -a$ 时,$f(x) = f(-a)$,
而函数值是唯一的(即每个原象只有一个象),即有 $f(x) = f(-a)$,但 $x$ 取值是任意的,于是在 $f(x)$
的整个定义域内都有 $f(-x) = -f(x)$。从而 $f(x)$ 是奇函数。

\begin{theorem}
    偶函数的图象关于 $y$ 轴成轴对称图形;反过来,如果一个函数的
    图象关于 $y$ 轴成轴对称图形,那么这个函数是偶函数。
\end{theorem}

请同学们自己证明(参看图 \ref{fig:1-21})。

\liti 已知函数 $f(x)$ 是偶函数,它在 $y$ 轴右边的图象如图 \ref{fig:1-22}(1)所示,画出 $f(x)$ 在 $y$ 轴左边的图象。

\jie 因为偶函数的图象是关于 $y$ 轴成轴对称图形,所以画法如下:

(1)如图 \ref{fig:1-22}(2),在 $y$ 轴右边的图象上取几个点。例如取点 $A_1$,$A_2$,$A_3$,$A_4$,$A_5$
(这些点一般应包括图象曲线的最低、最高点等“关键”点)。

(2)画出这些点关于 $y$ 轴的对称点。例如点 $A_1$,$A_2$,$A_3$,$A_4$,$A_5$ 的对称点分别
为 $A'_1$,$A'_2$,$A'_3$,$A'_4$,$A'_5$(图\ref{fig:1-22}(3))。

(3)用一条平滑曲线把对称点连续起来。例如用平滑曲线连接点 $A'_1$,$A'_2$,$A'_3$,$A'_4$,$A'_5$ 后,就得到
$f(x)$ 在 $y$ 轴左边的图象(图\ref{fig:1-22}(3))。

\begin{figure}[htbp]
    \centering
    \begin{minipage}{8cm}
    \centering
    \begin{tikzpicture}[>=Stealth]
    \draw [->] (-3.5,0) -- (3.5,0) node[anchor=north] {$x$};
    \draw [->] (0,-2) -- (0,2) node[anchor=east] {$y$};
    \node at (0.3,-0.3) {$O$};
    \draw (0.2,0.3) .. controls(0.8,2) and (1.6,2) .. (2.5,0.6);
\end{tikzpicture}

    \caption*{(1)}
    \end{minipage}
    \qquad
    \begin{minipage}{8cm}
    \centering
    \begin{tikzpicture}[>=Stealth]
    \draw [->] (-3.5,0) -- (3.5,0) node[anchor=north] {$x$};
    \draw [->] (0,-2) -- (0,2) node[anchor=east] {$y$};
    \node at (0.3,-0.3) {$O$};

    \coordinate (A1) at (0.5,0.3);
    \coordinate (A5) at (3.0,0.8);
    \draw [name path=a0] (A1) .. controls(1.2,2) and (2.1,2) .. (A5);

    \path [name path=a2] (1.0,0) -- (1.0,3);
    \path [name path=a3] (1.7,0) -- (1.7,3);
    \path [name path=a4] (2.5,0) -- (2.5,3);
    \fill (A1) circle[radius=2pt] +(-0.2,0.1) node[anchor=south] {$A_1$};
    \fill [name intersections={of=a0 and a2, by=A2}] (A2) circle[radius=2pt] +(-0.1,0.1) node[anchor=south] {$A_2$};
    \fill [name intersections={of=a0 and a3, by=A3}] (A3) circle[radius=2pt] +(0,0.1) node[anchor=south] {$A_3$};
    \fill [name intersections={of=a0 and a4, by=A4}] (A4) circle[radius=2pt] +(0,0.1) node[anchor=south] {$A_4$};
    \fill (A5) circle[radius=2pt] +(0.2,0.1) node[anchor=south] {$A_5$};
\end{tikzpicture}

    \caption*{(2)}
    \end{minipage}
    \qquad
    \begin{minipage}{8cm}
    \centering
    \begin{tikzpicture}[>=Stealth]
    \draw [->] (-3.5,0) -- (3.5,0) node[anchor=north] {$x$};
    \draw [->] (0,-2) -- (0,2) node[anchor=east] {$y$};
    \node at (0.3,-0.3) {$O$};

    \coordinate (A1) at (0.5,0.3);
    \coordinate (A5) at (3.0,0.8);
    \draw [name path=a0] (A1) .. controls(1.2,2) and (2.1,2) .. (A5);
    \draw [rotate around y=180] (0.5,0.3) .. controls(1.2,2) and (2.1,2) .. (3.0,0.8);
    
    \path [name path=a2] (1.0,0) -- (1.0,3);
    \path [name path=a3] (1.7,0) -- (1.7,3);
    \path [name path=a4] (2.5,0) -- (2.5,3);
    \fill let \p1 = (A1) in 
        (\p1) circle[radius=2pt] +(-0.2,0.1) node[anchor=south] {$A_1$}
        (-\x1,\y1) circle[radius=2pt] +(+0.2,0.1) node[anchor=south] {$A'_1$};
    \fill [name intersections={of=a0 and a2, by=A2}] let \p2 = (A2) in
        (\p2) circle[radius=2pt] +(-0.1,0.1) node[anchor=south] {$A_2$}
        (-\x2,\y2) circle[radius=2pt] +(0.1,0.1) node[anchor=south] {$A'_2$};
    \fill [name intersections={of=a0 and a3, by=A3}] let \p3 = (A3) in
        (\p3) circle[radius=2pt] +(0,0.1) node[anchor=south] {$A_3$}
        (-\x3,\y3) circle[radius=2pt] +(0,0.1) node[anchor=south] {$A_3$};
    \fill [name intersections={of=a0 and a4, by=A4}] let \p4 = (A4) in
        (\p4) circle[radius=2pt] +(0,0.1) node[anchor=south] {$A_4$}
        (-\x4,\y4) circle[radius=2pt] +(0,0.1) node[anchor=south] {$A_4$};
    \fill let \p5 = (A5) in
        (\p5) circle[radius=2pt] +(0.2,0.1) node[anchor=south] {$A_5$}
        (-\x5,\y5) circle[radius=2pt] +(-0.2,0.1) node[anchor=south] {$A_5$};
\end{tikzpicture}

    \caption*{(3)}
    \end{minipage}
    \caption{}\label{fig:1-22}
\end{figure}

\lianxi
\begin{xiaotis}

\xiaoti{判断下列函数是否具有奇偶性:}

\begin{xiaoxiaotis}
    \begin{tabular}[t]{*{2}{@{}p{14em}}} 
        \xiaoxiaoti {$f(x) = x^{-2}$;} & \xiaoxiaoti {$f(x) = 3x^{\frac 2 3}$;} \\
        \xiaoxiaoti {$f(x) = 2x + \sqrt[3]{x}$;} & \xiaoxiaoti {$f(x) = x + \dfrac 1 x$;} \\
        \xiaoxiaoti {$f(x) = 2x^{-4} - x^{-2}$。}
    \end{tabular}
\end{xiaoxiaotis}

\xiaoti{证明函数 $y = x^n$ 当 $n$ 为奇数时是奇函数,当 $n$ 为偶数时是偶函数。}

\xiaoti{已知函数 $f(x)$ 是偶函数,而且在 $(-\infty, 0 )$ 上是增函数。$f(x)$ 在 $(0, +\infty)$ 上是增函数还是减函数?}

\begin{figure}[H]
    \centering
    \begin{minipage}{8cm}
    \centering
    \begin{tikzpicture}[>=Stealth]
    \draw [->] (-3.5,0) -- (3.5,0) node[anchor=north] {$x$};
    \draw [->] (0,-2) -- (0,2) node[anchor=east] {$y$};
    \node at (0.3,-0.3) {$O$};
    \draw (0,1.6) .. controls(0.8,1.6) and (1.3,-1.5) .. (2.5,0.3);
\end{tikzpicture}

    \caption*{(第4题)}
    \end{minipage}
    \qquad
    \begin{minipage}{8cm}
    \centering
    \begin{tikzpicture}[>=Stealth]
    \draw [->] (-3.5,0) -- (3.5,0) node[anchor=north] {$x$};
    \draw [->] (0,-2) -- (0,2) node[anchor=east] {$y$};
    \node at (-0.3,-0.3) {$O$};
    \draw (0,0) .. controls(0.8,-2.6) and (1.3,0.5) .. (2.5,-0.6);
\end{tikzpicture}

    \caption*{(第5题)}
    \end{minipage}
\end{figure}

\xiaoti{如图,已知偶函数 $f(x)$ 在 $y$ 轴右边的一部分图象,根据偶函数的性质,画出它在 $y$ 轴左边的图象。}

\xiaoti{如图,已知奇函数 $f(x)$ 在 $y$ 轴右边的一部分图象,根据奇函数的性质,画出它在 $y$ 轴左边的图象。}

\end{xiaotis}
