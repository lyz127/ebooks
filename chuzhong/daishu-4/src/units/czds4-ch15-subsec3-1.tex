\subsubsection{正弦和余弦表}

\liti 查表求 $\sin 37^\circ24'$ 的值。

从正弦表\textbf{左边}的直列里查出 $37^\circ$,
从\textbf{顶端}的横行里查出 $24'$。
$37^\circ$ 所在的行和 $24'$ 所在的列相交处的数字是 $6074$,
它实际表示 $0.6074$(这里整数部分为 $0$,可从左上角第一个数值 $0.5736$ 的整数部分看出),
这就是 $\sin 37^\circ24'$ 的值。

\begin{table}[H]
    \newcommand{\x}{\cdots}
    \centering
    \caption*{正 弦}
    \NewDocumentCommand{\myarrow}{m m m O{1em}} {
        \tikz [overlay, >=Stealth] {
            \begin{scope}[xshift=#1, yshift=#2]
                \draw [->, rotate=#3] (0, 0) -- +(#4, 0);
            \end{scope}
        }
    }
    \begin{tblr}{
        hline{1, 9} = {1pt, solid},
        hline{2, 8} = {solid},
        vlines,
        vline{1, 2, 7, 8, 11} = {1pt, solid},
        columns={c, mode=math},
        column{1}={3em},
        column{7-10}={2em},
    }
        A        & \myarrow{-1.2em}{.3em}{0} 0'     & \x  & 24'    & \x\x  & 60' &     & 1'  & 2'  & 3' \\
        35^\circ \myarrow{.8em}{.8em}{-90} & 0.5736 &     &  \myarrow{0em}{.8em}{-90}[3em] &  &   &   &   &  \myarrow{0em}{.8em}{-90}[3em] &   \\
        \vdots   &        &     &        &             &     &     &     &     \\
        37^\circ & \myarrow{1.2em}{.3em}{0}[2em] &     & 6074   & \x\x  & \x  & \x \myarrow{0.2em}{.3em}{0}[2em] &     & 5   &     \\
        \vdots   &        &     &        &             &     &     &     &     \\
    \end{tblr}
\end{table}

\jie \hspace*{2em} $\sin 37^\circ24'= 0.6074$。


\lianxi
\begin{xiaotis}

\xiaoti{查表求下列正弦的值,然后按照从小到的顺序,用 “$<$” 号把它们连接起来:\\
    $\sin 78^\circ \nsep \sin 11^\circ24' \nsep \sin 40^\circ48' \nsep \sin 8^\circ30'$。
}

\xiaoti{查表求下列正弦的值,然后按照从大到小的顺序,用 “$>$” 号把它们连接起来:\\
    $\sin 54' \nsep \sin 89^\circ12' \nsep \sin 55^\circ6' \nsep \sin 23^\circ18'$。
}

\end{xiaotis}
\lianxijiange


\liti 查表求 $\sin 37^\circ26'$ 的值。

在正弦表顶端的横行里找不到 $26'$,但 $26'$ 在 $24'$ 和 $30'$ 之间而靠近 $24'$,
比 $24'$ 多 $2'$。所以先查出 $\sin 37^\circ24' = 0.6074$,
再在表右边修正值栏标有 $2'$ 的一列和 $37^\circ$ 所在的行相交处查得 $5$,
它实际表示 $0.6074$ 的最后一个数位上的 $5$,即 $0.0005$。
从正弦表中可以看出,当角度在 $0^\circ$ 与 $90^\circ$ 间变化时,
正弦值随着角度的增大(或减小)而增大(或减小),这从 $\sin 37^\circ24' = 0.6074$、
$\sin 37^\circ30' = 0.6088$ 也可看出。所以,把 $0.6074$ 加上 $2'$ 的修正值 $0.0005$,
即得 $\sin 37^\circ26'$ 的值。

\jie \hspace*{2em} \begin{TrigonometricTblr}
    \sin 37^\circ24' &   & = &       0.6074  & \\
           ( + \,2'  & ) &   & (+ \, 0.0005  & ) \\
    \sin 37^\circ26' &   & = &       0.6079  & \juhao
\end{TrigonometricTblr}


\liti 查表求 $\sin 37^\circ23'$ 的值。

$23'$ 在 $18'$ 和 $24'$ 之间而靠近 $24'$,比 $24'$ 少 $1'$。
所以先查出 $\sin 37^\circ24' = 0.6074$,再减去 $1'$ 的修正值 $0.0002$。

\jie \hspace*{2em} \begin{TrigonometricTblr}
    \sin 37^\circ24' &   & = &       0.6074 & \\
            (- \, 1' & ) &   & (- \, 0.0002 & ) \\
    \sin 37^\circ23' &   & = &       0.6072 & \juhao
\end{TrigonometricTblr}


\lianxi

查表求下列正弦的值:

\begin{xiaoxiaotis}
\resetxxt

    \begin{tblr}{columns={18em, colsep=0pt}}
        \xxt{$\sin 48^\circ40'$;} & \xxt{$\sin 86^\circ19'$;} \\
        \xxt{$\sin 21^\circ8'$;}  & \xxt{$\sin 70^\circ57'$。}
    \end{tblr}
\end{xiaoxiaotis}

\lianxijiange


余弦表的用法和正弦表类似,但要主意以下两点区别:

(1) 查余弦值时,“度” 数在表\textbf{右边}的直列里去查,“分” 数在\textbf{底端}的横行里去查。

(2) 查修正值时,要注意角度在 $0^\circ$ 与 $90^\circ$ 间变化时,
余弦值随着角度的增大(或减小)而减小(或增大)。


\liti 查表求 $\cos 43^\circ26'$ 的值。

$26'$ 在 $24'$ 和 $30'$ 之间而靠近 $24'$,比 $24'$ 多 $2'$。
所以先查出 $\cos 43^\circ24'$ 的值,再减去 $2'$ 的修正值 $0.0004$。


\begin{table}[H]
    \newcommand{\x}{\cdots}
    \centering
    \NewDocumentCommand{\myarrow}{m m m O{1em}} {
        \tikz [overlay, >=Stealth] {
            \begin{scope}[xshift=#1, yshift=#2]
                \draw [->, rotate=#3] (0, 0) -- +(#4, 0);
            \end{scope}
        }
    }
    \begin{tblr}{
        hline{4} = {solid},
        hline{5} = {1pt, solid},
        vlines,
        vline{1, 2, 5, 6, 8} = {1pt, solid},
        columns={c, mode=math},
        column{1, 5}={3em},
        column{6-8}={2em},
    }
        &       &        &       & \vdots   &     &     &    \\
        & \x\x  &  7266   & \myarrow{1em}{.3em}{180}[2em] & 43^\circ & \myarrow{-1em}{.3em}{0}[2em]    &  4  &    \\
        &       &  \myarrow{0em}{-0.5em}{90}[2em]      &       & \myarrow{-1.5em}{-0.5em}{90}[2em]  \vdots   &     &  \myarrow{0em}{-0.5em}{90}[2em]    &    \\
        & \x\x  & 24'    & \x\x \myarrow{0.3em}{0.8em}{180}[2em] &   A      & 1'  & 2'  & 3' \\
    \end{tblr}
    \caption*{余 弦}
\end{table}

\jie \hspace*{2em} \begin{TrigonometricTblr}
    \cos 43^\circ24' &   & = &       0.7266 &   \\
            (+ \, 2' & ) &   & (- \, 0.0004 & ) \\
    \cos 43^\circ26' &   & = &       0.7262 & \juhao
\end{TrigonometricTblr}


\lianxi

查表求下列余弦的值:

\begin{xiaoxiaotis}
\resetxxt

    \begin{tblr}{columns={12em, colsep=0pt}}
        \xxt{$\cos 63^\circ$;}    & \xxt{$\cos 27^\circ12'$;} & \xxt{$\cos 85^\circ36'$;} \\
        \xxt{$\cos 3^\circ12'$;}  & \xxt{$\cos 21^\circ44'$;} & \xxt{$\cos 54^\circ23'$;} \\
        \xxt{$\cos 12^\circ31'$;} & \xxt{$\cos 38^\circ39'$。}
    \end{tblr}
\end{xiaoxiaotis}

\lianxijiange


反过来,已知一个锐角的正弦或余弦值,可用 “正弦和余弦表” 查出这个锐角的值。

\liti 已知 $\sin\alpha = 0.2974$,求锐角 $\alpha$。

从正弦表中找出 $0.2974$,由这个数所在的行向\textbf{左}查得 $17^\circ$,
由 $0.2974$ 所在的列向\textbf{上}查得 $18'$,即
$$ 0.2974 = \sin 17^\circ18' \juhao $$

\jie 查表得 $\sin 17^\circ18' = 0.2974$,所以

\hspace*{1.5em} 锐角 $\alpha = 17^\circ18'$。


\liti 已知 $\cos\alpha = 0.7857$,求锐角 $\alpha$。

在余弦表中找不出 $0.7857$,但能找出同它最接近的数 $0.7859$,
由 $0.7859$ 所在的行向\textbf{右}查得 $38^\circ$,
由 $0.7859$ 所在的列向\textbf{下} 查得 $12'$,
即 $0.7859 = \cos 38^\circ12'$。
但是 $\cos\alpha = 0.7857$,比 $0.7859$ 小 $0.0002$,
这说明锐角 $\alpha$ 比 $38^\circ12'$ 要大。
由 $0.7859$ 所在的行向右查得修正值 $0.0002$ 对应的角度是 $1'$,所以

\hspace*{4em} \begin{TrigonometricTblr}
          0.7859 &   & = & \cos 38^\circ12' &   \\
    (- \, 0.0002 & ) &   &         (+ \, 1' & ) \\
          0.7857 &   & = & \cos 38^\circ13' & \juhao
\end{TrigonometricTblr}

\jie 查表得 $\cos 38^\circ12' = 0.7859$,所以

\hspace*{4em} \begin{TrigonometricTblr}
          0.7859 &   & = & \cos 38^\circ12' &   \\
    (- \, 0.0002 & ) &   &         (+ \, 1' & ) \\
          0.7857 &   & = & \cos 38^\circ13' & \juhao
\end{TrigonometricTblr}

即  \qquad 锐角 $\alpha = 38^\circ13'$。


\liti 已知 $\cos\beta = 0.4511$,求锐角 $\beta$。

从余弦表中找出与 $0.4511$ 最接近的数 $0.4509 = \cos 63^\circ12'$。
但是 $\cos\beta = 0.4511$,比 $0.4509$ 大 $0.0002$,
由 $0.4509$ 所在的行向右查修正值只有 $0.0003$ 最接近 $0.0002$,
所对应的角度是 $1'$, 所以

\hspace*{4em} \begin{TrigonometricTblr}
          0.4509 &   & = & \cos 63^\circ12' &   \\
    (+ \, 0.0003 & ) &   &         (- \, 1' & ) \\
          0.4512 &   & = & \cos 63^\circ11' & \juhao
\end{TrigonometricTblr}

这就是说,从表中只能查得与 $0.4511$ 最接近的数 $0.4512 = \cos 63^\circ11'$,
我们就取锐角 $\beta = 63^\circ11'$。

\jie \hspace*{2em} \begin{TrigonometricTblr}
          0.4509 &   & = & \cos 63^\circ12' &   \\
    (+ \, 0.0003 & ) &   &         (- \, 1' & ) \\
          0.4512 &   & = & \cos 63^\circ11' & \juhao
\end{TrigonometricTblr}

$\therefore$ \qquad 锐角 $\beta = 63^\circ11'$。

