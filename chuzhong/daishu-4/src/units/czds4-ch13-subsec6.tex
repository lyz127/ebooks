\subsection{反对数表}\label{subsec:13-6}

知道了一个数的对数,要求这个数,可以利用反对数表。下面是《中学数学用表》里的表十一“反对数表”的一部分。

\begin{table}[H]
    \newcommand{\x}{\cdots}
    \begin{tblr}{
        hline{1, 10} = {1pt, solid},
        hline{2, 9} = {solid},
        vline{1, 2, 12, 21} = {1pt, solid},
        vline{3-11} = {solid},
        columns={colsep=2.5pt, c, mode=math},
    }
        m   & 0    & 1    & 2    & 3    & 4    & 5    & 6    & 7    & 8    & 9    & 1  & 2  & 3  & 4  & 5  & 6  & 7  & 8  & 9  \\
        \x  & \x   & \x   & \x   & \x   & \x   & \x   & \x   & \x   & \x   & \x   & \x & \x & \x & \x & \x & \x & \x & \x & \x \\
        .25 & 1778 & 1782 & 1786 & 1791 & 1795 & 1799 & 1803 & 1807 & 1811 & 1816 & 0  & 1  & 1  & 2  & 2  & 2  & 3  & 3  & 4  \\
        .26 & 1820 & 1824 & 1828 & 1832 & 1837 & 1841 & 1845 & 1849 & 1854 & 1858 & 0  & 1  & 1  & 2  & 2  & 3  & 3  & 3  & 4  \\
        .27 & 1862 & 1866 & 1871 & 1875 & 1879 & 1884 & 1888 & 1892 & 1897 & 1901 & 0  & 1  & 1  & 2  & 2  & 3  & 3  & 3  & 4  \\
        .28 & 1905 & 1910 & 1914 & 1919 & 1923 & 1928 & 1932 & 1936 & 1941 & 1945 & 0  & 1  & 1  & 2  & 2  & 3  & 3  & 4  & 4  \\
        .29 & 1950 & 1954 & 1959 & 1963 & 1968 & 1972 & 1977 & 1982 & 1986 & 1991 & 0  & 1  & 1  & 2  & 2  & 3  & 3  & 4  & 4  \\
        \x  & \x   & \x   & \x   & \x   & \x   & \x   & \x   & \x   & \x   & \x   & \x & \x & \x & \x & \x & \x & \x & \x & \x \\
        m   & 0    & 1    & 2    & 3    & 4    & 5    & 6    & 7    & 8    & 9    & 1  & 2  & 3  & 4  & 5  & 6  & 7  & 8  & 9 \\
    \end{tblr}
\end{table}

下面我们来说明已知对数查反对数表求真数的方法。

1. 对数的首数是 $0$ 时,它的真数大于或等于 $1$,而小于 $10$,即 $1 \leqslant N < 10$。
因此,从表里查出尾数对应的四位数字,在第一位数字后面加上小数点,即得所求的真数。
例如,已知 $\lg{N} = 0.2846$, 求 $N$。
从表里 $m$ 所在的直列里找到 $.28$ ,从 $m$ 所在的横行里找到 $4$,交叉处的数是 $1923$,
再查 $6$ 对应的修正值是 $3$,$1923 + 3 = 1926$。因此,$N = 1.926$。

2. 对数的首数 $n \neq 0$ 时,从反对数表中查出尾数对应的真数。再将它乘以 $10^n$,即得所求的真数。
例如,已知 $\lg{N} = 2.2635$,求 $N$。
由对数的尾数 $0.2635$,查表得尾数的真数 $1.834$。因此,$N = 1.834 \times 10^2 = 183.4$、

\liti[0] 已知 $\lg{x}$,查表求 $x$:
\begin{xiaoxiaotis}

    \hspace*{1.5em} \begin{tblr}{columns={18em, colsep=0pt}}
        \xxt{$\lg{x} = 5.7635$;} & \xxt{$\lg{x} = \overline{2}.1804$;} \\
        \xxt{$\lg{x} = 0.05$;} & \xxt{$\lg{x} = -0.73248$。}
    \end{tblr}

\resetxxt
\jie \begin{tblr}[t]{columns={colsep=0em}}
    \xxt{$x = 5.801 \times 10^5 = 580100$;} \\
    \xxt{$x = 1.515 \times 10^{-2} = 0.01515$;} \\
    \xxt{$x = 1.122$;} \\
    \xxt{$\lg{x} = -0.73248 = \overline{1}.26752 \approx \overline{1}.2675$, \\
        $\therefore$ \quad $x = 1.851 \times 10^{-1} = 0.1851$。
    }
\end{tblr}
\end{xiaoxiaotis}


\lianxi
\begin{xiaotis}

\xiaoti{已知 $\lg{x}$,查表求 $x$:}
\begin{xiaoxiaotis}

    \begin{tblr}{columns={12em, colsep=0pt}}
        \xxt{$\lg{x} = 0.2138$;} & \xxt{$\lg{x} = 0.0300$;} & \xxt{$\lg{x} = 3.8147$;} \\
        \xxt{$\lg{x} = 1.1620$;} & \xxt{$\lg{x} = \overline{1}.3025$;} & \xxt{$\lg{x} = -2.0381$。}
    \end{tblr}
\end{xiaoxiaotis}


\xiaoti{查表求下列各式中的 $x$:}
\begin{xiaoxiaotis}

    \begin{tblr}{columns={12em, colsep=0pt}}
        \xxt{$\lg{753.8} = x$;}  & \xxt{$\lg{x} = 2.7538$;} & \xxt{$\lg{x} = 1.8429$;} \\
        \xxt{$\lg{1.8429} = x$;} & \xxt{$\lg{x} = 0.6573$;} & \xxt{$\lg{0.6573} = x$;} \\
        \xxt{$\lg{0.0453} = x$;} & \xxt{$\lg{x} = \overline{2}.7801$。}
    \end{tblr}
\end{xiaoxiaotis}

\end{xiaotis}
