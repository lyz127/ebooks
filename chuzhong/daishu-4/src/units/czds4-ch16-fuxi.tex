\fuxiti
\begin{xiaotis}

\xiaoti{什么叫做总体?什么叫做总体的一个样本?为什么常常需要用样本的某些持性去估计总体的相应特性?}


\xiaoti{甲、乙两门炮在相同条件下向同一目标各发射 $50$ 发炮弹,炮弹落点情况如下表所示:\\[0.5em]
    \hspace*{4em}
    \begin{tblr}{hlines, vlines, columns={c, colsep=1em}, row{2-3}={mode=math}, column{1}={mode=text}}
        炮弹落点与目标距离 & 40 米 & 30 米 &  20 米 & 10 米 & 0 米 \\
        甲炮发射的炮弹个数 & 0     &  1    &  3    &  7    & 39  \\
        乙炮发射的炮弹个数 & 1     &  3    &  2    &  3    & 41
    \end{tblr} \\[0.5em]
    分别计算两个样本平均数,根据计算结果,估计哪门炮射击的准确性较好。
}


\xiaoti{一个水库养了某种鱼 $10$ 万条,从中捕捞了 $20$ 条,称得它们的体重如下(单位:$500$ 克):\\
    \hspace*{4em}
    \begin{datatblr}{}
        2.3 & 2.1 & 2.2 & 2.1 & 2.2 & 2.6 & 2.5 & 2.4 & 2.3 & 2.4 \\
        2.4 & 2.3 & 2.2 & 2.5 & 2.4 & 2.6 & 2.3 & 2.5 & 2.2 & 2.3
    \end{datatblr} \\
    计算样本平均数,并根据计算结果估计水库里这种鱼的总重量约是多少?
}


\xiaoti{为了考察甲、乙两种小麦的长势,分别从中抽取 $10$ 株苗,测得苗高如下(单位:厘米):\\
    \hspace*{4em}
    \begin{datatblr}{column{1}={mode=text}}
        甲 & 12 & 13 & 14 & 15 & 10 & 16 & 13 & 11 & 15 & 11 \\
        乙 & 11 & 16 & 17 & 14 & 13 & 19 &  6 &  8 & 10 & 16
    \end{datatblr}
}
\begin{xiaoxiaotis}

    \xxt{分别计算两种小麦的平均苗高。}

    \xxt{哪种小麦长得比较整齐?}

\end{xiaoxiaotis}


\xiaoti{两名跳远运动员在 $10$ 次测验比赛中的成绩分别如下(单位:米):\\
    \hspace*{4em}
    \begin{datatblr}{column{1}={mode=text}}
        甲 & 5.85 & 5.93 & 6.07 & 5.91 & 5.99 & 6.13 & 5.98 & 6.05 & 6.00 & 6.19 \\
        乙 & 6.11 & 6.08 & 5.83 & 5.92 & 5.84 & 5.81 & 6.18 & 6.17 & 5.85 & 6.21
    \end{datatblr} \\
    分别计算两个样本方差,并根据计算结果估计那名运动员的成绩比较稳定。
}


\xiaoti{从一种零件中抽取了 $80$ 件,尺寸数据如下(单位:毫米):\\
    \hspace*{4em}
    \begin{datatblr}{colsep=1em}
        362.5 \times 1 & 362.6 \times 2 & 362.7 \times 2 & 362.8 \times 3 \\
        362.9 \times 3 & 363.0 \times 3 & 363.1 \times 5 & 363.2 \times 6 \\
        363.3 \times 8 & 363.4 \times 9 & 363.5 \times 9 & 363.6 \times 7 \\
        363.7 \times 6 & 363.8 \times 4 & 363.9 \times 3 & 364.0 \times 3 \\
        364.1 \times 2 & 364.2 \times 2 & 364.3 \times 1 & 364.4 \times 1
    \end{datatblr}
}
\begin{xiaoxiaotis}

    \xxt{列出样本的频率分布表,绘出频率分布直方图。}

    \renewcommand{\labelxiaoxiaoti}{*(\arabic{cntxiaoxiaoti})}
    \hspace{-0.5em}\xxt{在频率分布表中加填各分点的累积频率,并绘出累积频率分布图。}

\end{xiaoxiaotis}



\xiaoti{}%
\begin{xiaoxiaotis}%
    \xxt[\xxtsep]{设 $\overline{x}$ 是 $x_1$, $x_2$, $\cdots$, $x_n$ 的平均数,
        $\overline{y}$ 是 $3x_1 + 5$, $3x_2 + 5$, $\cdots$, $3x_n + 5$ 的平均数,
        求证 $\overline{y} = 3\overline{x} + 5$;
    }

    \xxt{设 $s_x$ 是 $x_1$, $x_2$, $\cdots$, $x_n$ 的标准差,
        $s_y$ 是 $3x_1 + 5$, $3x_2 + 5$, $\cdots$, $3x_n + 5$ 的标准差,
        求证 $s_y = 3s_x$。
    }

\end{xiaoxiaotis}

\end{xiaotis}

