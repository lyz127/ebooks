\subsection{积、商、幂、方根的对数}\label{subsec:13-2}

我们学过了指数的定义及运算,知道如果 $a > 0$,那么
\begin{gather*}
    a^p \cdot a^q = a^{p + q} \douhao \\
    a^p \div a^q = a^{p - q} \douhao \\
    (a^p)^n = a^{np} \douhao \\
    \sqrt[n]{a^p} = a^{\frac{p}{n}} \juhao
\end{gather*}
根据这些等式和对数的定义,我们可以得到对数的运算性质。

1. 设 $\log_{a}{M} = p$, $\log_{a}{N} = q$,由对数的意义,得
$$ M = a^p \douhao N = a^q \juhao $$

$\therefore$ \quad $M \cdot N = a^p \cdot a^q = a^{p + q}$,

$\therefore$ \quad $\log_{a}{MN} = p + q = \log_{a}{M} + \log_{a}{N}$。

这就是说,\zhongdian{两个正数的积的对数,等于同一底数的这两个数的对数的和,} 即
$$ \bm{\log_{a}{MN} = \log_{a}{M} + \log_{a}{N} \juhao} $$

当因数多于两个时,上述性质仍然成立,例如
$$ \log_{a}{LMN} = \log_{a}{L} + \log_{a}{M} + \log_{a}{N} \juhao$$


\begin{enhancedline}
2. 设 $\log_{a}{M} = p$, $\log_{a}{N} = q$,则
\begin{gather*}
    M = a^p \douhao N = a^q \juhao \\
    \dfrac{M}{N} = \dfrac{a^p}{a^q} = a^{p - q} \douhao
\end{gather*}

$\therefore$ \quad $\log_{a}{\dfrac{M}{N}} = p - q = \log_{a}{M} - \log_{a}{N}$。

这就是说,\zhongdian{两个正数的商的对数,等于同一底数的被除数的对数减去除数的对数的差,} 即
$$ \bm{\log_{a}{\dfrac{M}{N}} = \log_{a}{M} - \log_{a}{N}} \juhao $$
\end{enhancedline}

3. 设 $\log_{a}{M} = p$,则
\begin{gather*}
    M = a^p \douhao \\
    M^n = (a^p)^n = a^{np} \douhao
\end{gather*}

$\therefore$ \quad $\log_a{M^n} = np = n\log_{a}{M} \juhao$

这就是说,\zhongdian{一个正数的幂的对数,等于幂的底数的对数乘以幂指数,} 即
$$ \bm{\log_a{M^n} = n\log_{a}{M}} \juhao $$

\begin{enhancedline}
4. \zhongdian{一个正的算术根的对数,等于被开方数的对数除以根指数,} 即
$$ \bm{\log_{a}{\sqrt[n]{M}} = \dfrac{1}{n}\log_{a}{M}} \juhao $$

同学们可以自己证明这个性质。


\liti 用 $\log_{a}{x}$,$\log_{a}{y}$,$\log_{a}{z}$ 表示下列各式:
\begin{xiaoxiaotis}

    \hspace*{1.5em} \begin{tblr}{columns={18em, colsep=0pt}}
        \xxt{$\log_{a}{\dfrac{xy}{z}}$;} & \xxt{$\log_{a}{x^3y^5}$;} \\
        \xxt{$\log_{a}{\dfrac{\sqrt{x}}{yz}}$;} & \xxt{$\log_{a}{\dfrac{x^2\sqrt{y}}{\sqrt[3]{z}}}$。}
    \end{tblr}

\resetxxt
\jie \xxt{$\begin{aligned}[t]
        \log_{a}{\dfrac{xy}{z}} &= \log_{a}{xy} - \log_{a}{z} \\
            &= \log_{a}{x} + \log_{a}{y} - \log_{a}{z} \fenhao
    \end{aligned}$}

    \hspace*{1.5em} \xxt{$\begin{aligned}[t]
        \log_{a}{x^3y^5} &= \log_{a}{x^3} + \log_{a}{y^5} \\
            &= 3\log_{a}{x} + 5\log_{a}{y} \fenhao
    \end{aligned}$}

    \hspace*{1.5em} \xxt{$\begin{aligned}[t]
        \log_{a}{\dfrac{\sqrt{x}}{yz}} &= \log_{a}{\sqrt{x}} - \log_{a}{yz} \\
            &= \dfrac{1}{2}\log_{a}{x} - (\log_{a}{y} + \log_{a}{z}) \\
            &= \dfrac{1}{2}\log_{a}{x} - \log_{a}{y} - \log_{a}{z} \fenhao
    \end{aligned}$}

    \hspace*{1.5em} \xxt{$\begin{aligned}[t]
        \log_{a}{\dfrac{x^2\sqrt{y}}{\sqrt[3]{z}}} &= \log_{a}{x^2\sqrt{y}} - \log_{a}{\sqrt[3]{z}} \\
            &= \log_{a}{x^2} + \log_{a}{\sqrt{y}} - \log_{a}{\sqrt[3]{z}} \\
            &= 2\log_{a}{x} + \dfrac{1}{2}\log_{a}{y} - \dfrac{1}{3}\log_{a}{z} \juhao
    \end{aligned}$}

\end{xiaoxiaotis}


\liti 计算:
\begin{xiaoxiaotis}

    \hspace*{1.5em} \begin{tblr}{columns={18em, colsep=0pt}}
        \xxt{$\log_{10}{\sqrt[5]{100}}$;} & \xxt{$\log_{2}{(4^7 \cdot 2^5)}$。}
    \end{tblr}

\resetxxt
\jie \begin{tblr}[t]{columns={18em, colsep=0pt}}
    \xxt{$\log_{10}{\sqrt[5]{100}} = \dfrac{1}{5}\log_{10}{100} = \dfrac{2}{5}$;}
        & \xxt{$\begin{aligned}[t]
            \log_{2}{(4^7 \cdot 2^5)} &= \log_{2}{4^7} + \log_{2}{2^5} \\
                &= 7\log_{2}{4} + 5\log_{2}{2} \\
                &= 7 \times 2 + 5 \times 1 = 19 \juhao
        \end{aligned}$}
\end{tblr}
\end{xiaoxiaotis}
\end{enhancedline}

\lianxi
\begin{xiaotis}

\xiaoti{用 $\log_{10}{x}$,$\log_{10}{y}$,$\log_{10}{z}$,$\log_{10}{(x + y)}$,$\log_{10}{(x - y)}$ 表示下列各式:}
\begin{xiaoxiaotis}

    \begin{tblr}{columns={12em, colsep=0pt}}
        \xxt{$\log_{10}{xyz}$;} & \xxt{$\log_{10}{(x + y)z}$;} & \xxt{$\log_{10}{(x^2 - y^2)}$;} \\
        \xxt{$\log_{10}{\dfrac{xy^2}{z}}$;} & \xxt{$\log_{10}{\dfrac{xy}{(x + y)z}}$;} & \xxt{$\log_{10}{\dfrac{x^2 - xy}{100}}$。}
    \end{tblr}
\end{xiaoxiaotis}


\xiaoti{计算:}
\begin{xiaoxiaotis}

    \begin{tblr}{columns={18em, colsep=0pt}}
        \xxt{$\log_{3}{(27 \times 9^2)}$;} & \xxt{$\log_{10}{100^2}$;} \\
        \xxt{$\log_{10}{0.0001^3}$;}       & \xxt{$\log_{7}{\sqrt[3]{49}}$。}
    \end{tblr}
\end{xiaoxiaotis}


\xiaoti{计算:}
\begin{xiaoxiaotis}

    \begin{tblr}{columns={18em, colsep=0pt}}
        \xxt{$\log_{2}{6} - \log_{2}{3}$;} & \xxt{$\log_{10}{5} + \log_{10}{2}$;} \\
        \xxt{$\log_{5}{3} + \log_{5}{\dfrac{1}{3}}$;} & \xxt{$\log_{3}{5} - \log_{3}{15}$。}
    \end{tblr}
\end{xiaoxiaotis}


\xiaoti{下面的式子对不对?为什么?}
\begin{xiaoxiaotis}

    \begin{tblr}{columns={18em, colsep=0pt}}
        \xxt{$\log_{2}{(8 - 2)} = \log_{2}{8} - \log_{2}{2}$;} & \xxt{$\log_{10}{(4 - 2)} = \dfrac{\log_{10}{4}}{\log_{10}{2}}$;} \\
        \xxt{$\dfrac{\log_{2}{4}}{\log_{2}{8}} = \log_{2}{4} - \log_{2}{8}$。}
    \end{tblr}
\end{xiaoxiaotis}

\end{xiaotis}

