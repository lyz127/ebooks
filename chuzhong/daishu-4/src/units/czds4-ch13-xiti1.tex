\xiti
\begin{xiaotis}

\xiaoti{求下列各式中的 $x$。并指出计算 $x$ 时是求幂,求对数,或是求方根:}
\begin{xiaoxiaotis}

    \begin{tblr}{columns={12em, colsep=0pt}}
        \xxt{$3^4 = x$;} & \xxt{$x^3 = 1000$;} & \xxt{$10^x = 0.0001$。}
    \end{tblr}
\end{xiaoxiaotis}


\xiaoti{用对数形式把下列各式中的 $x$ 表示出来:}
\begin{xiaoxiaotis}

    \begin{tblr}{columns={9em, colsep=0pt}}
        \xxt{$4^x = 16$;} & \xxt{$3^x = 1$;} & \xxt{$4^x = 2$;}  & \xxt{$2^x = 0.5$。}
    \end{tblr}
\end{xiaoxiaotis}


\xiaoti{把下列对数式写成指数式,并求出 $x$ 的值:}
\begin{xiaoxiaotis}

    \begin{tblr}{columns={12em, colsep=0pt}, rows={rowsep=0.5em}}
        \xxt{$\log_{2}{32} = x$;} & \xxt{$\log_{5}{625} = x$;} & \xxt{$\log_{10}{1000} = x$;} \\
        \xxt{$\log_{8}{4} = x$;} & \xxt{$\log_{3}{\dfrac{1}{9}} = x$;} & \xxt{$\log_{3}{3} = x$;} \\
        \xxt{$\log_{10}{\dfrac{1}{1000}} = x$;} & \xxt{$\log_{16}{\dfrac{1}{2}} = x$。}
    \end{tblr}
\end{xiaoxiaotis}


\xiaoti{用 $\log_{a}{x}$,$\log_{a}{y}$,$\log_{a}{z}$,$\log_{a}{(x + y)}$,$\log_{a}{(x - y)}$ 表示下列各式:}
\begin{xiaoxiaotis}

    \begin{tblr}{columns={12em, colsep=0pt}, rows={rowsep=0.5em}}
        \xxt{$\log_{a}{\dfrac{\sqrt{x}}{y^2z}}$;}
            & \xxt{$\log_{a}{x\sqrt[\uproot{12}4]{\dfrac{z^3}{y^2}}}$;}
            & \xxt{$\log_{a}{xy^{\frac{1}{2}}z^{-\frac{2}{3}}}$;} \\
        \xxt{$\log_{a}{\dfrac{xy}{x^2 - y^2}}$;}
            & \xxt{$\log_{a}{\left(\dfrac{x + y}{x - y} \cdot y\right)}$;}
            & \xxt{$\log_{a}{\left[\dfrac{y}{x(x - y)}\right]^3}$。}
    \end{tblr}
\end{xiaoxiaotis}


\xiaoti{计算:}
\begin{xiaoxiaotis}

    \begin{tblr}{columns={18em, colsep=0pt}, rows={rowsep=0.5em}}
        \xxt{$\log_{a}{2} + \log_{a}{\dfrac{1}{2}}$;}
            & \xxt{$\log_{3}{18} - \log_{3}{2}$;} \\
        \xxt{$\log_{10}{\dfrac{1}{4}} - \log_{10}{25}$;}
            & \xxt{$2\log_{5}{10} + \log_{5}{0.25}$。}
    \end{tblr}
\end{xiaoxiaotis}


\xiaoti{}%
\begin{xiaoxiaotis}%
    \xxt[\xxtsep]{用 $a = \log_{10}{5}$ 表示 $\log_{10}{2}$,$\log_{10}{20}$;}

    \xxt{用 $a = \log_{10}{2}$ 与 $b = \log_{10}{3}$ 表示 $\log_{10}{4}$,$\log_{10}{5}$,
        $\log_{10}{6}$,$\log_{10}{12}$,$\log_{10}{15}$。
    }

\end{xiaoxiaotis}

\end{xiaotis}
\lianxijiange

