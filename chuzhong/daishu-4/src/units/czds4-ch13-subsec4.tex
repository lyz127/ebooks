\subsection{对数的首数和尾数}\label{subsec:13-4}

我们知道, $1 < 3.408 < 10$, 所以 $0 < \lg{3.408} < 1$,就是说,$\lg{3.408}$ 是一个正的纯小数。

如果知道了 $\lg{3.408} = 0.5325$, 我们可以求 $0.03408$, $340.8$, $34080$ 的对数。

首先用科学记数法表示上面各数,再求出对数:
\begin{align*}
    & \lg{0.03408} = \lg{(3.408 \times 10^{-2})} = \lg{10^{-2}} + \lg{3.408} = -2 + 0.5325 \fenhao \\
    & \lg{340.8} = \lg{(3.408 \times 10^2)} = \lg{10^2} + \lg{3.408} = 2 + 0.5325 \fenhao \\
    & \lg{340800} = \lg{(3.408 \times 10^4)} = \lg{10^4} + \lg{3.408} = 4 + 0.5325 \juhao
\end{align*}

一般地,可以得到:

\zhongdian{1. 所有正数的对数都可以写成一个整数(正整数、零或负整数)加上一个正的纯小数(或者零)的形式。}

整数部分叫做这个对数的\zhongdian{首数},正的纯小数(或者零)部分叫做这个对数的\zhongdian{尾数}。

例如:

在 $\lg{0.03408} = -2 + 0.5325$ 中,首数是 $-2$, 尾数是 $0.5325$;

在 $\lg{340.8} = 2 + 0.5325$ 中,首数是 $2$, 尾数是 $0.5325$;

在 $\lg{34080} = 4 + 0.5325$ 中,首数是 $4$, 尾数是 $0.5325$。

从上例还可以得到:

\zhongdian{2. 只有小数点位置不同的数,它们的对数的尾数都相同。}

求一个正数的对数的首数时,用科学记数法把这个数写成 $a \times 10^n$ 的形式,其中 $1 \leqslant a < 10$,
$n$ 是整数,$n$ 就是这个正数的对数的首数。

对数的首数是正整数或者零的时侯,可以把首数和尾数相加,写成小数的形式。
例如, $\lg{340.8} = 2.5325$, $\lg{3.408} = 0.5325$。
对数的首数是负整数的时候,通常把 “$-$” 号写在这个整数的上面,而把首数和尾数间的 “$+$” 号略去不写。
例如,$\lg{0.03408} = -2 + 0.5325$ 通常写成 $\lg{0.03408} = \overline{2}.5325$。
这里要注意 $\overline{2}.5325$ 等于 $-2 + 0.5325$ (即 $-1.4675$), 不等于 $-2.5325$。

\liti 用科学记数法表示下列各数,并求出各数的对数的首数。

\hspace*{4em} $32.16\nsep 7.8302\nsep 0.0002076$。

\jie \begin{tblr}[t]{}
    $32.16 = 3.216 \times 10^1$,对数的首数是 $1$;\\
    $7.8302 = 7.8302 \times 10^0$,对数的首数是 $0$;\\
    $0.0002076 = 2.076 \times 10^{-4}$,对数的首数是 $-4$。
\end{tblr}


\liti 写出下列各对数的首数和尾数:

\begin{xiaoxiaotis}
    \xxt{$\lg{a} = 0.2350$;}
    \xxt{$\lg{b} = \overline{2}.4087$;}
    \xxt{$\lg{c} = -2.4087$;}

\resetxxt
\jie \begin{tblr}[t]{}
    \xxt{首数是 $0$,尾数是 $0.2350$;} \\
    \xxt{首数是 $-2$,尾数是 $0.4087$;} \\
    \xxt{$\begin{aligned}[t]
            \lg{c} &= -2 - 0.4087 \\
                &= (-2 - 1) + (1 - 0.4087) \\
                &= -3 + 0.5913 = \overline{3}.5913 \juhao
        \end{aligned}$\\
        首数是 $-3$,尾数是 $0.5913$。
    }
\end{tblr}
\end{xiaoxiaotis}


\lianxi
\begin{xiaotis}

\xiaoti{用科学记数法表示下列各数,并求出它们的对数的首数。\\
    \begin{tblr}{columns={8em, $}}
        2570000, & 354.7, & 40.8,    & 5.06, \\
        9,       & 0.84,  & 0.07563, & 0.00002129 \juhao
    \end{tblr}
}

\begin{enhancedline}
\xiaoti{(口答)说出下列各数的对数的首数:\\
    $6720;\quad 3.1416;\quad \dfrac{1}{2};\quad 80;\quad 0.6428;\quad 0.00495$。
}

\xiaoti{(口答)下列各式的值是什么?}
\begin{xiaoxiaotis}

    \begin{tblr}{columns={9em, colsep=0pt}}
        \xxt{$\lg{10}$;} & \xxt{$\lg{10000}$;} & \xxt{$\lg{1}$;} & \xxt{$\lg{10^6}$;} \\
        \xxt{$\lg{10^{-5}}$;} & \xxt{$\lg{0.01}$;} & \xxt{$\lg{0.1}$;} & \xxt{$\lg{0.000001}$。}
    \end{tblr}
\end{xiaoxiaotis}
\end{enhancedline}

\xiaoti{写出下列各对数的首数和尾数:}
\begin{xiaoxiaotis}

    \begin{tblr}{columns={colsep=0pt}, column{1}={12em}}
        \xxt{$\lg{a} = 3.0720$;} & \xxt{$\lg{b} = 0.0129$;} & \xxt{$\lg{c} = \overline{4}.2157$;} \\
        \xxt{$\lg{d} = -4.2157$;} & \xxt{$\lg{0.000432} = \overline{4}.6355$;} & \xxt{$\lg{0.00574} = -2.2411$。}
    \end{tblr}
\end{xiaoxiaotis}

\end{xiaotis}


