\subsection{二次函数 $y = ax^2 + bx + c$ 的图象和性质}\label{subsec:14-11}
\begin{enhancedline}

\liti 在同一坐标系内,画出函数
\begin{gather*}
    y = \dfrac{1}{2}x^2 \douhao \\
    y = \dfrac{1}{2}(x + 3)^2 \douhao \\
    y = \dfrac{1}{2}(x + 3)^2 - 2
\end{gather*}
的图象。

\jie 在 $x$ 的取值范围内列出这几个函数的对应值表:

\begin{table}[H]
    % \hspace*{2em}
    \begin{tblr}{
        hlines, vlines,
        columns={2.5em, mode=math, c, colsep=0pt},
        column{1}={9em},
        rows={rowsep=0.5em},
    }
        x                             & \cdots & -6            & -5 & -4             & -3            & -2             & -1           & 0             & 1            & 2 & 3             & \cdots \\
        y = \dfrac{1}{2}x^2           &        &               &    & \cdots         & 4\dfrac{1}{2} &  2             & \dfrac{1}{2} & 0             & \dfrac{1}{2} & 2 & 4\dfrac{1}{2} & \cdots \\
        y = \dfrac{1}{2}(x + 3)^2     & \cdots & 4\dfrac{1}{2} &  2 & \dfrac{1}{2}   & 0             & \dfrac{1}{2}   & 2            & 4\dfrac{1}{2} & \cdots       &   &               &        \\
        y = \dfrac{1}{2}(x + 3)^2 - 2 & \cdots & 2\dfrac{1}{2} &  0 & -1\dfrac{1}{2} & -2            & -1\dfrac{1}{2} & 0            & 2\dfrac{1}{2} & \cdots       &   &               &        \\
    \end{tblr}
\end{table}

用描点法画出它们的图象(图 \ref{fig:14-25})。

\begin{figure}[htbp]
    \centering
    \begin{tikzpicture}[>=Stealth, scale=0.5,
    every node/.style={fill=white, inner sep=1pt},
]
    \draw [->] (-8, 0) -- (5, 0) node[below=0.2em] {$x$} coordinate(x axis);
    \draw [->] (0, -3) -- (0, 8) node[left=0.2em]  {$y$} coordinate(y axis);
    \draw (0, 0) node [below left=0.3em] {\small $O$};
    \foreach \x in {-6, ..., 4} {
        \draw (\x, 0) -- (\x, 0.2);
    }
    \foreach \y in {-2, ..., 6} {
        \draw (0.2, \y) -- (0, \y);
    }

    \draw[domain=-3.3:3.3,  samples=50] plot (\x, {\x*\x/2}) node [above] {$y = \frac{1}{2}x^2$};
    \draw[domain=0.3:-6.3,  samples=50] plot (\x, {(\x+3)^2/2}) node [above] {$y = \frac{1}{2}(x+3)^2$};
    \draw[domain=0.3:-6.3,  samples=50] plot (\x, {(\x+3)^2/2-2}) (-7, -2) node {$y = \frac{1}{2}(x+3)^2-2$};
    \draw [dashed] (-3, -3) -- (-3, 8);
\end{tikzpicture}


    \caption{}\label{fig:14-25}
\end{figure}

从图中可以看出,把函数 $y = \dfrac{1}{2}x^2$ 的图象向左平移 $3$ 个单位,就得到函数 $y = \dfrac{1}{2}(x + 3)^2$ 的图象;
再把函数 $y = \dfrac{1}{2}(x + 3)^2$ 的图象向下平移 $2$ 个单位,就得到函数 $y = \dfrac{1}{2}(x + 3)^2 - 2$ 的图象。
由于 $\dfrac{1}{2}(x + 3)^2 - 2 = \dfrac{1}{2}x^2 + 3x + \dfrac{5}{2}$,这也就是函数 $y = \dfrac{1}{2}x^2 + 3x + \dfrac{5}{2}$ 的图象。

由此可知,函数 $y = \dfrac{1}{2}x^2 + 3x + \dfrac{5}{2}$ 与函数 $y = \dfrac{1}{2}x^2$ 的图象,形状是一样的,只是位置不同。
容易知道,函数 $y = \dfrac{1}{2}x^2 + 3x + \dfrac{5}{2} = \dfrac{1}{2}(x + 3)^2 - 2$ 当 $x = -3$ 时的值最小,最小值是 $-2$,
因此抛物线  $y = \dfrac{1}{2}x^2 + 3x + \dfrac{5}{2}$ 的顶点是 $(-3,\, -2)$,
对称轴是经过点 $(-3,\, -2)$ 且与 $y$ 轴平行的直线 $x = -3$。\footnote{“直线 $x = -3$” 的意思是:
    这条直线是由横坐标为 $-3$ 的一切点所构成的,它平行于 $y$ 轴。 “直线 $x = h$” 的意思也是这样。
}

一般地,函数 $y = ax^2 + bx + c$ 的图象与函数 $y = ax^2$ 的图象的形状是一样的,只是位置不同。由于
\begin{align*}
    y &= ax^2 + bx + c = a\left(x^2 + \dfrac{b}{a}x + \dfrac{c}{a}\right) \\
      &= a \left[x^2 + 2 \cdot \dfrac{b}{2a}x + \left(\dfrac{b}{2a}\right)^2 - \left(\dfrac{b}{2a}\right)^2 + \dfrac{c}{a}\right] \\
      &= a \left(x + \dfrac{b}{2a}\right)^2 + \dfrac{4ac - b^2}{4a} \douhao
\end{align*}
所以它的图象可以通过平行移动 $y = ax^2$ 的图象得到:
当 $\dfrac{b}{2a} > 0$ 时,向左移动 $\dfrac{b}{2a}$ 个单位;
当 $\dfrac{b}{2a} < 0$ 时,向右移动 $\left|\dfrac{b}{2a}\right|$ 个单位;
当 $\dfrac{4ac - b^2}{4a} > 0$ 时,向上移动 $\dfrac{4ac - b^2}{4a}$ 个单位;
当 $\dfrac{4ac - b^2}{4a} < 0$ 时,向下移动 $\left|\dfrac{4ac - b^2}{4a}\right|$ 个单位。
因比函数 $y = ax^2 + bx + c$ 的图象是一条抛物线,
它的顶点是 $\left(-\dfrac{b}{2a},\; \dfrac{4ac - b^2}{4a}\right)$,
对称轴是平行于 $y$ 轴的直线 $x = -\dfrac{b}{2a}$。
当 $a > 0$ 时,抛物线 $y = ax^2 + bx + c$ 的开口向上,它的顶点是最低点。因此,
当 $a > 0$ 且 $x = -\dfrac{b}{2a}$ 时,函数 $y$ 有最小值,即
$$ y_{_\text{最小值}} = \dfrac{4ac - b^2}{4a} \juhao $$
当 $a < 0$ 时,抛物线 $y = ax^2 + bx + c$ 的开口向下,它的顶点是最高点。因此,
当 $a < 0$ 且 $x = -\dfrac{b}{2a}$ 时,函数 $y$ 有最大值,即
$$ y_{_\text{最大值}} = \dfrac{4ac - b^2}{4a} \juhao $$

综上所述,\zhongdian{二次函数 $\bm{y = ax^2 + bx + c}$ 有下列性质:}

\zhongdian{(1) 抛物线 $\bm{y = ax^2 + bx + c}$ 的顶点是
    $\bm{\left(-\dfrac{b}{2a},\; \dfrac{4ac - b^2}{4a}\right)}$,
    对称轴是直线 $\bm{x = -\dfrac{b}{2a}}$。
}

\zhongdian{(2)
    当 $\bm{a > 0}$ 时,抛物线的开口向上,并且向上无限伸展;
    当 $\bm{a < 0}$ 时,抛物线的开口向下,并且向下无限伸展。
}

\zhongdian{(3)
    当 $\bm{a > 0}$ 时,在对称轴的左侧,$\bm{y}$ 随着 $\bm{x}$ 的增大而减小;在对称轴的右侧,$\bm{y}$ 随着 $\bm{x}$ 的增大而增大;
    函数 $\bm{y}$ 当 $\bm{x = -\dfrac{b}{2a}}$ 时有最小值 $\bm{\dfrac{4ac - b^2}{4a}}$。
    当 $\bm{a < 0}$ 时,在对称轴的左侧,$\bm{y}$ 随着 $\bm{x}$ 的增大而增大;在对称轴的右侧,$\bm{y}$ 随着 $\bm{x}$ 的增大而减小;
    函数 $\bm{y}$ 当 $\bm{x = -\dfrac{b}{2a}}$ 时有最大值 $\bm{\dfrac{4ac - b^2}{4a}}$。
}


\liti 求抛物线 $y = -\dfrac{1}{2}x^2 - 3x - \dfrac{5}{2}$ 的对称轴和顶点坐标,并画图。

\jie 在函数式 $y = -\dfrac{1}{2}x^2 - 3x - \dfrac{5}{2}$ 中,$a = -\dfrac{1}{2}$,$b = -3$,$c = -\dfrac{5}{2}$,所以
$$ -\dfrac{b}{2a} = 3 \nsep \dfrac{4ac - b^2}{4a} = 2 \juhao $$

也可将 $-\dfrac{1}{2}x^2 - 3x - \dfrac{5}{2}$ 配方,得
\begin{align*}
    y &= -\dfrac{1}{2}(x^2 + 6x + 5) \\
      &= -\dfrac{1}{2}[(x + 3)^2 - 4] \\
      &= -\dfrac{1}{2}(x + 3)^2 + 2 \juhao
\end{align*}

因此,抛物线 $y = -\dfrac{1}{2}x^2 - 3x - \dfrac{5}{2}$ 的对称轴是 $x = -3$,顶点坐标是 $(-3,\; 2)$。

在 $x$ 的取值范围内,根据函数的对称性,列出函数的对应值表:
\begin{table}[H]
    \centering
    \begin{tblr}{
        hlines, vlines,
        columns={2em, mode=math, c},
        column{1}={4em},
        rows={rowsep=0.5em},
    }
        x & \cdots & -6             & -5 & -4            & -3 & -2            & -1 & 0              & \cdots \\
        y & \cdots & -2\dfrac{1}{2} & 0  & 1\dfrac{1}{2} &  2 & 1\dfrac{1}{2} & 0  & -2\dfrac{1}{2} & \cdots
    \end{tblr}
\end{table}

\begin{figure}[htbp]
    \centering
    \begin{tikzpicture}[>=Stealth, scale=0.5,
    every node/.style={fill=white, inner sep=1pt},
]
    \draw [->] (-8, 0) -- (2, 0) node[below=0.2em] {$x$} coordinate(x axis);
    \draw [->] (0, -7) -- (0, 4) node[left=0.2em]  {$y$} coordinate(y axis);
    \draw (0, 0) node [below right=0.3em] {\small $O$};
    \foreach \x in {-6, ..., -1} {
        \draw (\x, 0) -- (\x, 0.2);
    }
    \draw (-1, 0) node [below=0.2em, xshift=-0.5em] {\small $-1$};
    \draw (-3, 0) node [above=0.2em, xshift=-0.7em] {\small $-3$};
    \draw (-5, 0) node [above=0.2em, xshift=-0.5em] {\small $-5$};

    \foreach \y in {-5, ..., 2} {
        \draw (0.2, \y) -- (0, \y);
    }
    \draw (0, 2) node [left=0.2em] {\small $2$};


    \draw [dashed] (-3, -5.5) -- (-3, 3) node [above] {\small $x = -3$};
    \draw[domain=1:-7,  samples=50] plot (\x, {-1/2*(\x)^2 - 3*\x - 5/2}) node [below, xshift=2.5em] {\small $y = -\dfrac{1}{2}x^2 - 3x - \dfrac{5}{2}$};
    \foreach \x in {-6, ..., 0} {
        \draw [fill=black] (\x, {-1/2*(\x)^2 - 3*\x - 5/2}) circle(0.1);
    }
\end{tikzpicture}


    \caption{}\label{fig:14-26}
\end{figure}

用描点法画出它的图象(图 \ref{fig:14-26})。



\liti 求二次函数 $y = 2x^2 - 8x + 1$ 的最大值或最小值。

\jie $y = 2x^2 - 8x + 1 = 2(x^2 - 4x + 4) - 8 + 1 = 2(x - 2)^2 - 7$。

因为 $a = 2 > 0$,所以 $y$ 有最小值。当 $x = 2$ 时,得到
$$ y_{_\text{最小值}} = -7 \juhao $$

\end{enhancedline}



\lianxi
\begin{xiaotis}

\xiaoti{用配方法把下列函数化成 $y = a(x + h)^2 + k$ 的形式,并指出它们的图象的开口方向、顶点和对称轴(不画图):}
\begin{xiaoxiaotis}

    \begin{tblr}{columns={18em, colsep=0pt}, row{3}={rowsep=0.5em}}
        \xxt{$y = x^2 - 2x - 3$;}  & \xxt{$y = x^2 + 6x + 10$;} \\
        \xxt{$y = 2x^2 - 3x + 4$;} & \xxt{$y = -2x^2 - 5x + 7$;} \\
        \xxt{$y = 3x^2 + 2x$;}     & \xxt{$y = \dfrac{5}{2}x - 2 - 3x^2$。}
    \end{tblr}
\end{xiaoxiaotis}


\xiaoti{画出下列函数的图象}
\begin{xiaoxiaotis}

    \begin{tblr}{columns={18em, colsep=0pt}, row{2}={rowsep=0.5em}}
        \xxt{$y = -x^2 - 2x$;}      & \xxt{$y = 1 - 3x^2$;} \\
        \xxt{$y = -2x^2 + 8x - 8$;} & \xxt{$y = \dfrac{1}{2}x^2 + 3x + \dfrac{5}{2}$。}
    \end{tblr}
\end{xiaoxiaotis}


\xiaoti{}%
\begin{xiaoxiaotis}%
    \xxt[\xxtsep]{求函数 $y = 2(x - 3)^2$ 在 $x = 1$,$2$,$2.5$,$2.9$,$3$,$3.1$,$3.5$,$4$,$5$ 时的值。这些值中哪一个最小?函数的最小值是什么?}

    \xxt{求函数 $y = 4 - (x + 2)^2$ 在 $x = -5$,$-4$,$-3$,$-2$,$-1$,$0$,$1$ 时的值。这些值中哪一个最大?函数的最大值是什么?}

\end{xiaoxiaotis}


\xiaoti{求下列函数的最大值或最小值:}
\begin{xiaoxiaotis}

    \begin{tblr}{columns={18em, colsep=0pt}}
        \xxt{$y = x^2 - 2x + 4$;} & \xxt{$y = -x^2 + 3x$;} \\
        \xxt{$S = 1 - 2\,t - t^2$;} & \xxt{$u = 2V^2 + 4V - 5$;} \\
        \xxt{$V = -3\,t^2 + 4t$;}   & \xxt{$y = x(8 - x)$;} \\
        \xxt{$h = 100 - 5\,t^2$;}   & \xxt{$y = (x - 2)(2x + 1)$。}
    \end{tblr}
\end{xiaoxiaotis}

\end{xiaotis}

