\xiti
\begin{xiaotis}

\xiaoti{下列各小题中的两个变量哪些成正比例,那些成反比例,哪些既不成正比例也不成反比例?}
\begin{xiaoxiaotis}

    \xxt{三角形的底边不变时,它的面积与这个底边上的高;}

    \xxt{三角形的面积不变时,它的底边与这个底边上的高;}

    \xxt{质量不变时,物体的体积与密度;}

    \xxt{体积不变时,物体的质量与密度;}

    \xxt{某人的体重与年龄;}

    \xxt{除数不变时,被除数与商;}

    \xxt{被除数不变时,除数与商;}

    \xxt{$x + 3$ 与 $x$;}

    \xxt{$xy = 18$ 中的 $y$ 与 $x$;}

    \xxt{$x:y = 18$ 中的 $y$ 与 $x$。}

\end{xiaoxiaotis}


\xiaoti{已知 $y$ 与 $x$ 成正比例,且 $x = 8$ 时 $y = 6$。分别求 $x = 3$ 和 $x = -9$ 时 $y$ 的值。}


\xiaoti{已知 $y$ 与 $x^2$ 成正比例,且 $x = 2$ 时 $y = 16$, 求:}
\begin{xiaoxiaotis}

    \xxt{$x = -4$ 时 $y$ 的值;}

    \xxt{$y = 64$ 时 $x$ 的值。}

\end{xiaoxiaotis}


\begin{enhancedline}
\xiaoti{在同一坐标系内,画出下列函数的图象:
    $$y = \dfrac{3}{4}x \nsep y = -\dfrac{5}{2}x \nsep y = 2.5x \nsep y = -0.6x \juhao $$
}

\xiaoti{已知 $a$ 与 $b^2$ 成反比例,且 $b = 4$ 时 $a = 5$, 求 $b = \dfrac{4}{5}$ 时 $a$ 的值。}

\xiaoti{设矩形的面积是 $24 \; \pflm$, 长是 $x$ 厘米。}
\begin{xiaoxiaotis}

    \xxt{求它的宽 $y$;}

    \xxt{把这个矩形的宽表示成长的函数,写出自变量的取值范围,并画出它的图象。}

\end{xiaoxiaotis}
\end{enhancedline}

\xiaoti{在同一坐标系内,画出下列函数的图象:
    $$ xy = 1 \nsep xy = -1 \nsep xy - 2 = 0 \nsep xy + 2 = 0 \juhao $$
}

\xiaoti{已知 $y = y_1 + y_2$, $y_1$ 与 $x$ 成正比例, $y_2$ 与 $x^2$ 成反比例,
    并且 $x = 2$ 与 $x = 3$ 时,$y$ 的值都等于 $19$。求 $y$ 与 $x$ 之间的函数关系式。
}


\xiaoti{已知 $y = y_1 + y_2$,$y_1$ 与 $x$ 成正比例,$y_2$与 $x$ 成反比例,
    并且 $x = 1$ 时 $y = 4$, $x = 2$ 时 $y = 5$。求 $x = 4$ 时 $y$ 的值。
}

\end{xiaotis}

