\subsection{平面直角坐标系}\label{subsec:14-1}

我们知道,在直线上规定了原点、正方向和单位长度,就构成了数轴。
在数轴上,每一个点的位置都能用一个实数来表示,这个实数叫做这个点\zhongdian{在数轴上的坐标}。
那么,用什么方法表示平面内点的位置呢?

要在一块矩形板上钻一个孔,只要给出孔的中心到板的左边的距离 $30$ 毫米和到下边的距离 $20$ 毫米(图 \ref{fig:14-1}),
孔心 $M$ 的位置就确定了。可见,用两个实数就可以表示平面内点的位置。

\begin{figure}[htbp]
    \centering
    \begin{minipage}{6cm}
    \centering
    \begin{tikzpicture}[>=Stealth,
    every node/.style={fill=white, inner sep=1pt},
]
    \pgfmathsetmacro{\x}{2.1}
    \pgfmathsetmacro{\y}{1.4}
    \draw (0, 0) rectangle (5, 3);
    \draw [dashed] (0, \y) -- (\x, \y) -- (\x, 0);
    \filldraw [fill=black] (\x, \y) circle (0.05);
    \draw (\x, \y) node[right] {$M$};
    \draw [<->] (0, -0.3) to [xianduan] node {$30$} (\x, -0.3);
    \draw [<->] (-0.3, 0) to [xianduan] node [rotate=90] {$20$} (-0.3, \y);
\end{tikzpicture}


    \caption{}\label{fig:14-1}
    \end{minipage}
    \qquad
    \begin{minipage}{8cm}
    \centering
    \begin{tikzpicture}[>=Stealth, scale=0.55]
    \draw [->] (-6,0) node[anchor=east] {$x'$} -- (6,0) node[anchor=north] {$x$};
    \draw [->] (0,-6) node[anchor=east] {$y'$} -- (0,6) node[anchor=east]  {$y$};
    \node at (-0.3,-0.3) {\small $O$};
    \foreach \x in {-5,...,-1} {
        \draw (\x,0.2) -- (\x,0) node[anchor=north] {\small $\x$};
    }
    \foreach \x in {1,...,5} {
        \draw (\x,0.2) -- (\x,0) node[anchor=north] {\small $\x$};
    }
    \foreach \y in {1,...,5} {
        \draw (0.2,\y) -- (0,\y) node[anchor=east] {\small $\y$};
    }
    \foreach \y in {-5,...,-1} {
        \draw (0.2,\y) -- (0,\y) node[anchor=east] {\small $\y$};
    }

    \draw ( 3,  3) node {第一象限};
    \draw (-3,  3) node {第二象限};
    \draw (-3, -3) node {第三象限};
    \draw ( 3, -3) node {第四象限};
\end{tikzpicture}


    \caption{}\label{fig:14-2}
    \end{minipage}
\end{figure}

在平面内画两条互相垂直而且有公共原点 $O$ 的数轴 $x'x$ 和 $y'y$(图 \ref{fig:14-2})。
$x'x$ 通常画成水平的,叫做 \zhongdian{$\bm{x}$ 轴} 或 \zhongdian{横轴},取向右的方向为正方向;
$y'y$     画成铅直的,叫做 \zhongdian{$\bm{y}$ 轴} 或 \zhongdian{纵轴},取向上的方向为正方向。
两条数轴上的单位长度一般取相同的。$x$ 轴和 $y$ 轴统称 \zhongdian{坐标轴},$O$ 叫做\zhongdian{坐标原点}。
这样,在平面内有公共原点而且互相垂直的两条数轴,就构成了\zhongdian{平面直角坐标系},在本书中简称\zhongdian{坐标系}。
建立了坐标系的平面,叫做\zhongdian{坐标平面}。

$x$ 轴和 $y$ 轴把坐标平面分成四个部分 $xOy$, $yOx'$, $x'Oy'$, $y'Ox$,
依次叫做第一象限、第二象限、第三象限和第四象限。
象限以两轴为界限,$x$ 轴、$y$ 轴上的点不在任一象限内。

在平面内建立了直角坐标系以后,对于平面内的任意一点,都有一对有序实数和它对应。例如,对于点 $M$(图 \ref{fig:14-3}),
  经过点 $M$ 画 $x$ 轴的垂线,垂足为 $M_1$,点 $M_1$ 在 $x$ 轴上的坐标是 $3$;
再经过点 $M$ 画 $y$ 轴的垂线,垂足为 $M_2$,点 $M_2$ 在 $y$ 轴上的坐标是 $2$。
这样,点 $M$ 就有一对坐标 $3$,$2$ 和它对应。
我们把 $3$ 叫做点 $M$ 的 \zhongdian{横坐标},$2$ 叫做点 $M$ 的\zhongdian{纵坐标},
合起来叫做点 \zhongdian{$\bm{M}$ 在平面内的坐标},记作 $M(3,\, 2)$,
其中横坐标规定写在纵坐标的前面,中间用逗号隔开。
这就是说,点 $M$ 在平面内的坐标是一对有序实数(叫做一个\zhongdian{有序实数对})。
在图 \ref{fig:14-3} 中,点 $N$ 的坐标是 $(2, 3)$,记作 $N (2,\, 3)$。
从图中我们可以看到,$M$ 与 $N$ 是坐标平面内不同的两个点,
和它们对应的 $(3,\, 2)$ 与 $(2,\, 3)$ 是两对不同的有序实数,
因此它们具有不同的坐标。

想一想,图 \ref{fig:14-3} 中点 $P$ 和点 $Q$ 的坐标各是什么。

\begin{figure}[htbp]
  \centering
  \begin{minipage}{6cm}
  \centering
  \begin{tikzpicture}[>=Stealth, scale=0.7]
    \draw [->] (-2, 0) -- (4, 0) node[anchor=north] {$x$} coordinate(x axis);
    \draw [->] (0, -2) -- (0, 4) node[anchor=east]  {$y$} coordinate(y axis);
    \node at (-0.3, -0.3) {\small $O$};
    \foreach \x in {-1, 2, 3} {
        \draw (\x, 0.2) -- (\x, 0) node[anchor=north] {\small $\x$};
    }
    \foreach \x in {1} {
        \draw (\x, 0) -- (\x, 0.2) node[anchor=south] {\small $\x$};
    }
    \foreach \y in {-1, 2, 3} {
        \draw (0.2, \y) -- (0, \y) node[anchor=east] {\small $\y$};
    }
    \foreach \y in {1} {
        \draw (0, \y) -- (0.2, \y) node[anchor=west] {\small $\y$};
    }

    % 第一种写法:分成两步
    \draw [dashed] (-1, 0) -- (-1, 1) -- (0, 1);
    \filldraw [fill=black] (-1, 1) circle (0.05) node[anchor=east] {\small $Q$};

    % 第二种写法,一句完成点、线的绘制
    \filldraw [dashed, fill=black] (1, 0) -- (1, -1) circle (0.05) node[anchor=west] {\small $P$} -- (0, -1);

    % 第三种写法,在已知点坐标的情况下,利用 x axis 和 y axis 实现绘制
    \coordinate (M) at (3, 2);
    \filldraw [dashed, fill=black] (M |- x axis) node [above right] {\small $M_1$}
            -- (M) circle (0.05) node[anchor=west] {\small $M$}
            -- (M -| y axis) node [below right] {\small $M_2$};
    \coordinate (N) at (2, 3);
    \filldraw [dashed, fill=black] (N |- x axis) -- (N) circle (0.05) node[anchor=west] {\small $N$} -- (N -| y axis);

    % 其中,
    % 第二种写法更适合于固定坐标,即点的坐标在绘制前已经确定。
    % 第三种写法更适合于动态坐标,即点的坐标是在绘制过程中计算出来的。
\end{tikzpicture}


  \caption{}\label{fig:14-3}
  \end{minipage}
  \qquad
  \begin{minipage}{8cm}
  \centering
  \begin{tikzpicture}[>=Stealth, scale=0.7]
    \draw [->] (-4, 0) -- (4, 0) node[anchor=north] {$x$};
    \draw [->] (0, -3) -- (0, 3) node[anchor=east] {$y$};
    \filldraw [fill=black] (0, 0) circle (0.05) node [anchor=south west] {\small $O(0, 0)$};
    \foreach \x in {-2, -1, 1, 2, 3} {
        \draw (\x, 0.2) -- (\x, 0) node[anchor=north] {\small $\x$};
    }
    \foreach \x in {-3} {
        \draw (\x, 0) -- (\x, 0.2) node[anchor=south] {\small $\x$};
    }
    \foreach \y in {-1, 1, 2} {
        \draw (0.2, \y) -- (0, \y) node[anchor=east] {\small $\y$};
    }
    \foreach \y in {-2} {
        \draw (0, \y) -- (0.2, \y) node[anchor=west] {\small $\y$};
    }

    \filldraw [dashed, fill=black] (-3, 0) -- (-3, -2) circle (0.05) node[anchor=north] {\small $A(-3, -2)$} -- (0, -2);
    \filldraw [dashed, fill=black] (3, 0) -- (3, 2) circle (0.05) node[anchor=south] {\small $B(3, 2)$} -- (0, 2);
    \filldraw [dashed, fill=black] (2.5, 0) -- (2.5, -1) circle (0.05) node[anchor=north] {\small $C(2.5, -1)$} -- (0, -1);
    \filldraw [fill=black] (-2, 0) circle (0.05)  +(0.7, 0) node [anchor=south] {\small $D(-2, 0)$};
\end{tikzpicture}


  \caption{}\label{fig:14-4}
  \end{minipage}
\end{figure}

反过来,对于任意一对有序实数,在坐标平面内都有一个确定的点和它对应,这个点在平面内的坐标就是这一对有序实数。
例如,给出有序实数对 $(-3,\, -2)$,我们就可以
经过 $x$ 轴上坐标为 $-3$ 的点画 $x$ 轴的垂线,
经过 $y$ 轴上坐标为 $-2$ 的点画 $y$ 轴的垂线,
这两条线的交点 $A$ 就是和有序实数对 $(-3,\, -2)$ 对应的点(图 \ref{fig:14-4})。
同样,和有序实数对 $(3,\, 2)$, $(2.5,\, -1)$,$(-2,\, 0)$, $(0,\, 0)$ 对应的点分别是 $B$,$C$,$D$,$O$。

从上面我们看到:对于坐标平面内任意一点 $M$,都有一对有序实数 $(x,\, y)$ 和它对应;
反过来,对于任意一对有序实数 $(x,\, y)$,在坐标平面内都有一点 $M$ 和它对应。
因此,坐标平面内所有的点与所有有序实数对之间是一一对应的。


\lianxi
\begin{xiaotis}

\xiaoti{写出图中 $A$,$B$,$C$,$D$,$E$,$F$,$G$,$H$,$O$ 各点的坐标。}

\begin{figure}[htbp]
  \centering
  \begin{tikzpicture}[>=Stealth, scale=0.5,
    every node/.style={fill=white, inner sep=1pt},
]
    \draw [thick, ->] (-7.5, 0) -- (8, 0) node[below=0.1] {$x$};
    \draw [thick, ->] (0, -9) -- (0, 10) node[left=0.1] {$y$};
    \foreach \x in {-6, ..., -1, 1, 2, ..., 6} {
        \draw (\x, -8.2) -- (\x, 9);
    }
    \foreach \y in {-8, ..., -1, 1, 2, ..., 9} {
        \draw (-7, \y) -- (7, \y);
    }

    \foreach \x in {-6, ..., -1, 1, 2, ..., 6} {
        \draw (\x, 0) node[below=0.1] {\small $\x$};
    }

    \foreach \y in {-7, ..., -1, 1, 2, ..., 8} {
        \draw (0, \y) node[right=0.1] {\small $\y$};
    }

    \filldraw [fill=black] ( 3,  5) circle (0.1) node [above=0.1] {\small $A$};
    \filldraw [fill=black] (-6,  5) circle (0.1) node [above=0.1] {\small $B$};
    \filldraw [fill=black] (-2, -5) circle (0.1) node [below=0.1] {\small $C$};
    \filldraw [fill=black] ( 5, -7) circle (0.1) node [above=0.1] {\small $D$};
    \filldraw [fill=black] ( 5,  0) circle (0.1) node [above=0.1] {\small $E$};
    \filldraw [fill=black] ( 0,  7) circle (0.1) node [left =0.1] {\small $F$};
    \filldraw [fill=black] ( 0, -4) circle (0.1) node [left =0.1] {\small $G$};
    \filldraw [fill=black] (-5,  0) circle (0.1) node [above=0.1] {\small $H$};
    \filldraw [fill=black] ( 0,  0) circle (0.1) node [above left=0.1] {\small $O$};
\end{tikzpicture}


  \caption*{(第 1 题)}
\end{figure}


\xiaoti{在直角坐标系中描出下列各点:\\
    $A(3,\, 6)$, $B(-1.5,\, 3.5)$, $C(-4,\, -1)$, $D(2,\, -3)$,
    $E(3,\, 0)$, $F(-2,\, 0)$, $G(0,\, 5)$, $H(0,\, -4)$。
}

\end{xiaotis}
\lianxijiange


\liti 在坐标平面内,

(1) $x$ 轴上的点的纵坐标有什么特点?

(2) $y$ 轴上的点的横坐标有什么特点?

\jie (1) 如图 \ref{fig:14-5}, 在轴上任取一点 $P$。经过点 $P$ 画 $y$ 轴的垂线,垂足是原点 $O(0, 0)$。
点 $O$ 在 $y$ 轴上的坐标是 $0$, 所以点 $P$ 的纵坐标是 $0$。
这就是说,$x$ 轴上的点的纵坐标都是 $0$。

(2) 同理可知,$y$ 轴上的点的横坐标都是 $0$。

\begin{figure}[htbp]
  \centering
  \begin{minipage}{6cm}
  \centering
  \begin{tikzpicture}[>=Stealth, scale=0.6]
    \draw [->] (-3, 0) -- (3, 0) node[anchor=north] {$x$};
    \draw [->] (0, -3) -- (0, 3) node[anchor=east] {$y$};
    \draw (0, 0) node [anchor=north east] {\small $O$};

    \filldraw [fill=black] (2, 0) circle (0.05) node [anchor=south] {\small $P$};
\end{tikzpicture}


  \caption{}\label{fig:14-5}
  \end{minipage}
  \qquad
  \begin{minipage}{8cm}
  \centering
  \begin{tikzpicture}[>=Stealth, scale=0.6]
    \draw [->] (-1, 0) -- (5, 0) node[anchor=north] {$x$};
    \draw [->] (0, -1) -- (0, 5) node[anchor=east] {$y$};
    \draw (0, 0) node [anchor=north east] {\small $O$};
    \foreach \x in {1, 2, 3, 4} {
        \draw (\x, 0.2) -- (\x, 0) node[anchor=north] {\small $\x$};
    }
    \foreach \y in {1, 2, 3, 4} {
        \draw (0.2, \y) -- (0, \y) node[anchor=east] {\small $\y$};
    }

    \draw [very thick] (0, 0) rectangle (4, 4);
    \draw (0, 0) node [anchor=south west] {$A$};
    \draw (4, 0) node [anchor=south west] {$B$};
    \draw (4, 4) node [anchor=south west] {$C$};
    \draw (0, 4) node [anchor=south west] {$D$};
\end{tikzpicture}


  \caption{}\label{fig:14-6}
  \end{minipage}
\end{figure}

\liti 如图 \ref{fig:14-6}, 已知正方形 $ABCD$ 的边长等于 $4$,求四个顶点的坐标。

\jie 四个顶点的坐标分别为

\hspace*{1.5em} $A(0,\, 0)$, $B(4,\, 0)$, $C(4,\, 4)$, $D(0,\, 4)$。


\lianxi
\begin{xiaotis}

\xiaoti{写出例 2 中正方形的各边中点的坐标。}

\xiaoti{已知正方形的边长等于 $4$,对角线的交点在原点,边与坐标轴平行,求它的各顶点的坐标。}

\xiaoti{已知点 $P$ 的坐标是 $(5,\, -3)$,分列写出 $P$ 点关于 $x$ 轴、$y$轴和原点对称的点的坐标。}

\xiaoti{以点 $(3,\, 0)$ 为圆心,以 $5$ 为半径画一圆,写出圆与坐标轴交点的坐标。}

\end{xiaotis}

