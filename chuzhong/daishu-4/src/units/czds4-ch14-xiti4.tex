\xiti
\begin{xiaotis}

\xiaoti{按照下列程序写出 $y$ 与 $x$ 之间的函数关系式:}
\begin{xiaoxiaotis}

    \begin{tblr}[t]{
        columns={colsep=0pt},
        rows={valign=m, rowsep=0.5em}
    }
        \xxt{} & \begin{minipage}{8cm}
                    \begin{tikzpicture}[every node/.style={minimum size=1cm,draw}, >=Stealth]
    \coordinate (a) at (0, 0);
    \foreach \text  [count=\i] in {
        输入$x$, $\times 3$, $-1$,  输出$y$
    } {
        \ifnum \i > 1 \relax
            \draw [->] (N.east) -- +(0.5, 0) coordinate (a);
        \fi
        \draw (a) node [right] (N) {\text};
    }
\end{tikzpicture}
% 使用 \graph 的写法见 czds4-ch14-xiti4-1-2

                  \end{minipage} \\
        \xxt{} & \begin{minipage}{8cm}
                    % \begin{tikzpicture}[every node/.style={minimum size=1cm,draw}, >=Stealth]
%     \coordinate (a) at (0, 0);
%     \foreach \text  [count=\i] in {
%         输入$x$, $-2$, $\times 3$, $+1$,  输出$y$
%     } {
%         \ifnum \i > 1 \relax
%             \draw [->] (N.east) -- +(0.5, 0) coordinate (a);
%         \fi
%         \draw (a) node [right] (N) {\text};
%     }
% \end{tikzpicture}
\begin{tikzpicture}[every node/.style={minimum size=1cm,draw}, >=Stealth]
    \graph [grow right sep=0.5cm] {
        "输入$x$" -> "$-2$" -> "$\times 3$" -> "$+1$" -> "输出$y$"
    };
\end{tikzpicture}
% 不使用 \graph 的写法见 czds4-ch14-xiti4-1-1

                  \end{minipage}
    \end{tblr}
    \jiange
\end{xiaoxiaotis}


\xiaoti{求下列函数中自变量 $x$ 的取值范围:}
\begin{xiaoxiaotis}

    \begin{tblr}{columns={18em, colsep=0pt}, rows={rowsep=0.3em}}
        \xxt{$y = 3x^2 - 5x + \sqrt{3}$;} & \xxt{$y = \dfrac{2x + 1}{x - 2}$;} \\
        \xxt{$y = \dfrac{x + 1}{x^2 - x - 6}$;} & \xxt{$y = \dfrac{3x}{4x^2 - 9}$;} \\
        \xxt{$y = \sqrt{2x - 5}$;} & \xxt{$y = x + \sqrt{x + 2}$;} \\
        \xxt{$y = \dfrac{x + 2}{x^2 + 5x + 6}$。}
    \end{tblr}
\end{xiaoxiaotis}


\xiaoti{已知函数 $y = x^2 - 3x + 4$,填表: \\[1em]
    \begin{tblr}{
        hlines, vlines,
        columns={mode=math, 2em, c},
        column{1}={4em},
        rows={rowsep=0.5em},
    }
        x & -2 & -1 & 0 & 1 & 1\dfrac{1}{2} & 2 & 3 & 4 \\
        y &    &    &   &   &               &   &   &
    \end{tblr} \jiange
}


\xiaoti{已知函数 $y = \dfrac{2x + 1}{x - 2}$,求当 $x = 3$,$-4$,$0$,$-\dfrac{1}{2}$,$\sqrt{2}$ 时
    的函数值。当 $x = a^2 + 3$ 时,$y$ 等于多少?
}

\xiaoti{已知函数 $y = 2x^2 - 5x + 3$, 求当 $x = 0$,$2$ 时的函数值。$x$ 取什么值时函数值为 $0$?}

\xiaoti{一个铜球在 $0$ ℃时的体积是 $1000 \; \lflm$,加热后温度每增加 $1$ ℃,体积增加 $0.051 \; \lflm$。
    用解析式表示体积 $V$ 是温度 $T$ 的函数,并根据列出的解析式计算铜球加热到 $200$ ℃时的体积。
}


\xiaoti{用解析式将等腰三角形的顶角的度 $y$ 表示为底角的度数 $x$ 的函数,并求自变量 $x$ 的取值范围。}

\xiaoti{已知 $x$,$y$ 满足下列等式,用 $x$ 的代数式表示 $y$:}
\begin{xiaoxiaotis}

    \begin{tblr}{columns={18em, colsep=0pt}, rows={rowsep=0.5em}}
        \xxt{$3x + 4y = 12$;} & \xxt{$xy = 15$;} \\
        \xxt{$(x - 2)(y + 3)= -6$;} & \xxt{$x = \dfrac{3y + 2}{4y - 3}$;} \\
        \xxt{$y^2 = 4x \; (y \geqslant 0)$;} & \xxt{$y - \dfrac{2}{3}x = 0$。}
    \end{tblr}
\end{xiaoxiaotis}


\xiaoti{已知函数 $y = ax + b$ ($a$,$b$ 都是常数),并且当 $x = 1$ 时 $y = 7$,
    当 $x = 2$ 时 $y = 16$,确定 $a$,$b$ 的值。
}

\xiaoti{测得某一弹簧的长度 $y$ 与悬挂的重量 $x$ 有下面的一组对应值:\\[1em]
    \begin{tblr}{
        hlines, vlines,
        columns={mode=math, 2em, c},
        column{1}={4em},
    }
        x\text{(千克)} &  0 &  1   &  2 &  3   &  4 &  5   &  6 &   7  &  8 \\
        y\text{(厘米)} & 12 & 12.5 & 13 & 13.5 & 14 & 14.5 & 15 & 15.5 & 16
    \end{tblr} \\[1em]
    假定 $y$ 与 $x$ 之间的函数关系式是 $y = ax + b$($a$,$b$ 都是常数),
    利用表中任意两对对应值来确定 $a$,$b$ 的值,再用表中其他数据来进行检验。
}


\xiaoti{下表是某天一昼夜间温度变化情况的记录:\\[1em]
    \begin{tblr}{
        hlines, vlines,
        columns={mode=math, 1.5em, c},
        column{1}={5em},
    }
        \text{时间(时)} &  0 &  2 &  4 & 6 & 8 & 10 & 12 & 14 & 16  & 18 & 20  & 22 & 24 \\
        \text{温度(℃)} & -2 & -3 & -4 & 0 & 4 &  7 &  9 & 10 & 8.5 &  7 & 3.5 &  1 & -1
    \end{tblr} \\[1em]
    根据这个表,画出反映这一昼夜间温度变化情况的曲线。
}


\xiaoti{下图是某个地区某月中的日平均温度变化的图象,根据这个图象说明:}
\begin{figure}[htbp]
    \centering
    \begin{tikzpicture}[>=Stealth,
    every node/.style={fill=white, inner sep=1pt},
]
    \pgfmathsetmacro{\factorx}{0.25}
    \pgfmathsetmacro{\factory}{0.25}
    \draw [->] (0, 0) -- (8.6, 0) node[below=0.1] {时间(日)};
    \draw [->] (0, 0) -- (0, 3.8) node[left=0.2em] {温度(℃)};
    \draw (0, 0) coordinate(O) node [below left] {\small $0$};
    \foreach \x in {2, 4, ..., 30} {
        \draw (\factorx * \x, 0) -- (\factorx * \x, -0.15) node [below] {\small $\x$};
    }
    \foreach \y [count=\i] in {8, 10, ..., 18} {
        \draw (0, \factory * \y - \factory * 6) -- (-0.15, \factory * \i * 2) node[left] {\small $\y$};
    }

    \draw plot coordinates{
        %            x               y
        (\factorx *  1, \factory *  12.2 - \factory * 6)
        (\factorx *  2, \factory *  13.0 - \factory * 6)
        (\factorx *  3, \factory *  16.0 - \factory * 6)
        (\factorx *  4, \factory *  13.0 - \factory * 6)
        (\factorx *  5, \factory *  13.8 - \factory * 6)
        (\factorx *  6, \factory *   8.8 - \factory * 6)
        (\factorx *  7, \factory *   9.0 - \factory * 6)
        (\factorx *  8, \factory *   9.2 - \factory * 6)
        (\factorx *  9, \factory *  10.0 - \factory * 6)
        (\factorx * 10, \factory *  15.8 - \factory * 6)
        (\factorx * 11, \factory *  15.0 - \factory * 6)
        (\factorx * 12, \factory *  17.0 - \factory * 6)
        (\factorx * 13, \factory *   9.2 - \factory * 6)
        (\factorx * 14, \factory *  10.0 - \factory * 6)
        (\factorx * 15, \factory *  11.5 - \factory * 6)
        (\factorx * 16, \factory *  12.0 - \factory * 6)
        (\factorx * 17, \factory *  12.8 - \factory * 6)
        (\factorx * 18, \factory *  13.8 - \factory * 6)
        (\factorx * 19, \factory *  15.0 - \factory * 6)
        (\factorx * 20, \factory *  16.2 - \factory * 6)
        (\factorx * 21, \factory *  15.8 - \factory * 6)
        (\factorx * 22, \factory *  15.0 - \factory * 6)
        (\factorx * 23, \factory *  11.0 - \factory * 6)
        (\factorx * 24, \factory *  13.0 - \factory * 6)
        (\factorx * 25, \factory *  17.0 - \factory * 6)
        (\factorx * 26, \factory *  15.0 - \factory * 6)
        (\factorx * 27, \factory *  13.0 - \factory * 6)
        (\factorx * 28, \factory *  11.0 - \factory * 6)
        (\factorx * 29, \factory *  14.5 - \factory * 6)
        (\factorx * 30, \factory *  14.0 - \factory * 6)
    };
\end{tikzpicture}


    \caption*{(第 12 题)}
\end{figure}
\begin{xiaoxiaotis}

    \xxt{这个月中最高与最低的日平均温度各是多少;}

    \xxt{这个月中日平均温度变化的幅度是多大。}

\end{xiaoxiaotis}


\xiaoti{画出下列函数的图象:}
\begin{xiaoxiaotis}

    \begin{tblr}{columns={18em, colsep=0pt}}
        \xxt{$y = 4x$;}    & \xxt{$y = 3x + 1$;} \\
        \xxt{$y = -2x^2$;} & \xxt{$y = 2x^2 - 1$。}
    \end{tblr}
\end{xiaoxiaotis}

\end{xiaotis}

