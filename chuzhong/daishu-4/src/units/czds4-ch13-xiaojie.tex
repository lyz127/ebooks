\xiaojie

一、本章主要内容是对数的概念、运算性质,常用对数,以及利用对数进行计算。


二、设 $a \; (a > 0,\; a \neq 1)$ 的 $b$ 次幂等于 $N$,就是 $a^b = N$,
数 $b$ 就叫以 $a$ 为底的 $N$ 的对数,记作 $\log_{a}{N} = b$,
其中 $a$ 是对数的底数(简称底),$N$ 叫做真数。

在对数式 $\log_{a}{N} = b$ 中,$a > 0$,$a \neq 1$,$N > 0$。


\begin{enhancedline}
三、积、商、幂、方根的对数是:
\begin{align*}
    & \log_{a}{(MN)} = \log_{a}{M} + \log_{a}{N} ; \\
    & \log_{a}{\dfrac{M}{N}} = \log_{a}{M} - \log_{a}{N}; \\
    & \log_{a}{M^n} = n\log_{a}{M}; \\
    & \log_{a}{\sqrt[n]{M}} = \dfrac{1}{n} \log_{a}{M} \juhao
\end{align*}
利用它们可以把乘、除运算转化为加、减运算,把乘方、开方运算转化为乘、除运算。
\end{enhancedline}


四、我们通常用的对数是以 $10$ 为底的对数,这种对数叫做常用对数,$N$ 的常用对数可以简写成 $\lg{N}$。
知道一个正数,可以由 “常用对数表” 查出这个数的对数;
知道一个数的对数,可以由 “反对数表” 查出这个数。
利用 “常用对数表”、“反对数表” 及积、商、幂、方根的对数的性质,可以比较简捷地进行乘、除、乘方、开方的计算。

