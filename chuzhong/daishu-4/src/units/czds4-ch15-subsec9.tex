\subsection{正弦定理}\label{subsec:15-9}

\begin{enhancedline}

\begin{wrapfigure}[10]{r}{6.5cm}
    \centering
    \begin{tikzpicture}[>=Stealth,]
    \draw [->] (-1.8, 0) -- (3, 0) node [below] {$x$} coordinate(x axis);
    \draw [->] (0, -0.5) -- (0, 3) node [left]  {$y$} coordinate(y axis);
    \draw (0, 0) coordinate (O) node [below left] {$O$};

    \coordinate (A) at (O);
    \coordinate (B) at (-1.2, 2);
    \coordinate (C) at (2.5, 0);
    \coordinate (E) at (B |- O);

    \draw [thick] (A) node[below right] {$A$}
         -- (B) node [above, xshift=2.7em] {$B(c \cos A,\; c \sin A)$} node [midway, below left] {$c$}
         -- (C) node [below] {$C$} node [midway, above] {$a$}
         -- cycle  node [midway, below] {$b$};
    \draw [dashed] (B) -- (E) node [below] {$E$}
        pic [draw, solid, angle radius=0.5em] {right angle=O--E--B};
\end{tikzpicture}


    \caption{}\label{fig:15-23}
\end{wrapfigure}

如图 \ref{fig:15-23},以 $\triangle ABC$ 的顶点 $A$ 为原点,边 $AC$ 所在的射线为 $x$ 轴的正半轴建立直角坐标系。
由上节我们知道,顶点 $B$ 的坐标是 $B(c\cos{A},\; c\sin{A})$。
容易知道,$AC$ 边上的高 $BE$ 就是点 $B$ 的纵坐标 $c\sin{A}$,
于是 $\triangle ABC$ 的面积
$$ S_{\triangle} = \exdfrac{1}{2} \times AC \times BE = \exdfrac{1}{2}bc \sin{A} \juhao $$

同样可得(参看图 \ref{fig:15-22}):
$$ S_{\triangle} = \exdfrac{1}{2}ca \sin{B} \nsep S_{\triangle} = \exdfrac{1}{2}ab \sin{C} \juhao $$

由此,我们得到关于任意三角形面积的公式:
\begin{center}
    \framebox[24em]{
        $\bm{ S_{\triangle} = \exdfrac{1}{2}bc \sin{A} = \exdfrac{1}{2}ca \sin{B} = \exdfrac{1}{2}ab \sin{C} }$。
    }
\end{center}

也就是说\zhongdian{三角形的面积等于任意两边与它们夹角的正弦的积的一半。}

将等式
$$ \exdfrac{1}{2}bc \sin{A} = \exdfrac{1}{2}ca \sin{B} = \exdfrac{1}{2}ab \sin{C} $$
各除以 $\exdfrac{1}{2}abc$,可得
$$ \dfrac{\sin{A}}{a} = \dfrac{\sin{B}}{b} = \dfrac{\sin{C}}{c} \juhao $$

由此,我们得到关于任意三角形边和角间的关系的另一个重要定理:

\zhongdian{正弦定理 \quad 在一个三角形中,各边和它所对角的正弦的比相等。}
\begin{center}
    \framebox[14em]{
        $\bm{ \dfrac{a}{\sin{A}} = \dfrac{b}{\sin{B}} = \dfrac{c}{\sin{C}} }$。
    }
\end{center}

如果三角形 $ABC$ 中有一个角是直角,例如,$C = 90^\circ$,
这时 $\sin{C} = 1$, 由正弦定理可得 $\sin{A} = \exdfrac{a}{c}$,$\sin{B} = \exdfrac{b}{c}$,
这和 \ref{subsec:15-4} 节直角三角形中边和角间的关系是一致的。

利用正弦定理与三角形内角和定理,可以解决以下两类解斜三角形的问题:

(1) 已知两角和任一边,求其他两边和一角;

(2) 已知两边和其中一边的对角,求另一边的对角( 而进一步求出其他的边和角)。


\liti 在 $\triangle ABC$ 中,已知 $c = 10$,$A = 45^\circ$,$C = 30^\circ$,
求 $b$ 和 $S_{\triangle}$ (结果保留两个有效数字)。

\jie (1) $\because$ \quad $\dfrac{b}{\sin{B}} = \dfrac{c}{\sin{C}}$,$B = 180^\circ - (A + C)$,

$\therefore$ \quad $\begin{aligned}[t]
    b &= \dfrac{c \sin{[180^\circ - (A + C)]}}{\sin{C}} \\
      &= \dfrac{10\sin{[180^\circ - (45^\circ + 30^\circ)]}}{\sin{30^\circ}} \\
      &= \dfrac{10\sin{75^\circ}}{\exdfrac{1}{2}} \\
      &\approx 19.3 \approx 19 \juhao
\end{aligned}$

(2) $\begin{aligned}[t]
    S_{\triangle} &= \exdfrac{1}{2}bc \sin{A} \\
                  &\approx \exdfrac{1}{2} \times 19.3 \times 10 \times \sin{45^\circ} \\
                  &\approx 68 \juhao
\end{aligned}$


\liti 在 $\triangle ABC$ 中,已知 $a = 20$,$b = 28$,$A = 40^\circ$,
求 $B$ (精确到 $1^\circ$) 和 $c$ (保留两个有效数字)。

\jie (1) $\begin{aligned}[t]
    \sin{B} &= \dfrac{b \sin{A}}{a} = \dfrac{28 \sin{40^\circ}}{20} \\
            &\approx 0.8999 \juhao
\end{aligned}$

查表可得锐角 $B_1 = 64^\circ$。

又因 $\sin(180^\circ - B_1) = \sin{B_1}$,钝角 $B_2 = 180^\circ - B_1$ 也符合题设条件,可得
$$ B_2 = 180^\circ - 64^\circ = 116^\circ \juhao $$

(2) 当 $B_1 = 64^\circ$ 时,

$\begin{aligned}
    c_1 &= \dfrac{a \sin{C}}{\sin{A}} = \dfrac{a \sin{[180^\circ - (A + B_1)]}}{\sin{A}} \\
        &= \dfrac{20 \cdot \sin{[180^\circ - (40^\circ + 64^\circ)]}}{\sin{40^\circ}} \\
        &= \dfrac{20 \cdot \sin{76^\circ}}{\sin{40^\circ}} \\
        &\approx 30 \fenhao
\end{aligned}$

当 $B_2 = 116^\circ$ 时,

$\begin{aligned}
    c_2 &= \dfrac{a \sin{[180^\circ - (A + B_2)]}}{\sin{A}} \\
        &= \dfrac{20 \cdot \sin{[180^\circ - (40^\circ + 116^\circ)]}}{\sin{40^\circ}} \\
        &= \dfrac{20 \cdot \sin{24^\circ}}{\sin{40^\circ}} \\
        &\approx 13 \juhao
\end{aligned}$


想一想,例 2 用余弦定理怎样求解。

\begin{wrapfigure}[10]{r}{6.5cm}
    \centering
    \begin{tikzpicture}[>=Stealth,]
    \pgfmathsetmacro{\factor}{0.1}
    \pgfmathsetmacro{\jiaoa}{40}
    \pgfmathsetmacro{\ac}{28 * \factor}
    \pgfmathsetmacro{\a}{20 * \factor}

    \coordinate (A) at (0, 0);
    \coordinate (C) at (\jiaoa:\ac);
    \coordinate (Q) at ($(A)!1.3!(C)$);
    \coordinate (P) at (4, 0);

    \draw [thick] (A) -- (Q) node [right] {$Q$};
    \node [above] at (C) {$C$};
    \node [above] at ($(A)!0.5!(C)$) {$b$};

    \draw [thick, name path=ap] (A) node [left] {$A$} -- (P) node [below] {$P$};
    \path [name path=cb] (C) + (180:\a) arc (180:360:\a);
    \draw [name intersections={of=ap and cb, by={B2, B1}}];
    \draw (C) -- (B1) node [below right] {$B_1$} node [midway, left]  {$a$};
    \draw (C) -- (B2) node [below right, xshift=-0.4em] {$B_2$} node [midway, right] {$a$};
    \begin{scope}[every node/.style={fill=white, inner sep=1pt, outer sep=3pt},]
        \draw [<->] ([yshift=-1.0em] A) to [xianduan={above=1.0em}] node {$c_2$} ([yshift=-1.0em] B2);
        \draw [<->] ([yshift=-2.0em] A) to [xianduan={above=2.0em}] node {$c_1$} ([yshift=-2.0em] B1);
    \end{scope}

    \pgfmathanglebetweenpoints{\pgfpointanchor{B2}{base}}{\pgfpointanchor{C}{base}}
    \pgfmathsetmacro{\jiaocbp}{\pgfmathresult}
    \draw [dashed] (B2) arc (180+\jiaocbp:360-\jiaocbp:\a);
    \draw (B2) arc (180+\jiaocbp:180+\jiaocbp-20:\a)
          (B1) arc (360-\jiaocbp:360-\jiaocbp+20:\a);
\end{tikzpicture}


    \caption{}\label{fig:15-24}
\end{wrapfigure}

在图 \ref{fig:15-24} 中,我们画 $\angle PAQ = 40^\circ$,在 $AQ$ 上取 $AC = b = 28$ mm,
以 $C$ 为圆心,$a = 20$ mm 为半径画弧。可以看到,所画的弧与 $AP$ 相交于两点 $B_1$、$B_2$,
因而可以作出两个三角形,$\triangle AB_1C$ 和 $\triangle AB_2C$ 都符合题设条件,这就表示例 2 有两解。


\liti 在 $\triangle ABC$ 中,已知 $a = 60$, $b = 50$, $A = 38^\circ$,
求 $B$ (精确到 $1^\circ$) 和 $c$ (保留两个有效数字)。

\jie (1) 已知 $b < a$,所以 $B < A$, $B$ 也为锐角。

$\sin{B} = \dfrac{b \sin{A}}{a} = \dfrac{50 \sin{38^\circ}}{60} \approx 0.5131$。

查表可得锐角 $B = 31^\circ$。

(2) $\begin{aligned}[t]
    c &= \dfrac{a \sin{C}}{\sin{A}} = \dfrac{a \sin{[180^\circ - (A + B)]}}{\sin{A}} = \dfrac{a \sin{(A + B)}}{\sin{A}} \\
      &= \dfrac{60 \sin{(38^\circ + 31^\circ)}}{\sin{38^\circ}} = \dfrac{60 \sin{69^\circ}}{\sin{38^\circ}} \approx 91 \juhao
\end{aligned}$


\liti 在 $\triangle ABC$ 中,已知 $a = 18$, $b = 20$, $A = 150^\circ$,
解这个三角形。

\jie 这里 $A$ 为钝角,$a$ 应是最大边,但 $b > a$,所以本题无解。

对于 例 3 和例 4,照图 \ref{fig:15-24} 的方法,试画三角形.结果怎样?

一般地,已知两边和其中一边的对角解三角形,有两解、一解、无解三种情况。
图 \ref{fig:15-25} 说明了在 $\triangle ABC$ 中,已知 $a$, $b$ 和 $A$ 时解三角形的各种情况。

(1) $A$ 为锐角:

\begin{figure}[H]%[htbp]
    \centering
    \begin{tikzpicture}
    \begin{scope}
        \pgfmathsetmacro{\r}{0.6}
        \coordinate (A) at (0, 0);
        \coordinate (C) at (1.4, 1);
        \coordinate (B) at (2, 0);
        \draw [thick] (B)
            -- (A) node [below] {$A$}
            -- (C) node [above] {$C$} node [midway, above] {$b$}
            -- ($(C) + (300:\r)$) node [pos=0.3, right] {$a$};
        \draw pic [draw] {angle=B--A--C};
        \draw [densely dashed] (C) + (260:\r) arc (260:340:\r);

        \node at (1, -1) {$a < b\sin A$};
        \node at (1, -1.5) {无解};
    \end{scope}

    \begin{scope}[xshift=3.5cm]
        \pgfmathsetmacro{\r}{1}
        \coordinate (A) at (0, 0);
        \coordinate (C) at (1.4, 1);
        \coordinate (B) at (1.4, 0);
        \draw [thick] (B)
            -- (A) node [below] {$A$}
            -- (C) node [above] {$C$} node [midway, above] {$b$}
            -- (B) node [below] {$B$} node [midway, right] {$a$};
        \draw [thick] (B) -- ($(B)!-0.3!(A)$);
        \draw pic [draw] {angle=B--A--C};
        \draw [densely dashed] (C) + (235:\r) arc (235:305:\r);
        \draw pic [draw, angle radius=0.5em] {right angle=A--B--C};

        \node at (1, -1) {$a = b\sin A$};
        \node at (1, -1.5) {一解};
    \end{scope}

    \begin{scope}[xshift=7cm]
        \pgfmathsetmacro{\a}{1.3}
        \coordinate (A) at (0, 0);
        \coordinate (C) at (1.4, 1);
        \coordinate (P) at (3, 0);

        \draw [thick] (A) -- (C) node [above] {$C$} node [midway, above] {$b$};
        \draw [thick, name path=ap] (A) node [left] {$A$} -- (P);
        \path [name path=cb] (C) + (180:\a) arc (180:360:\a);
        \draw [name intersections={of=ap and cb, by={B2, B1}}];
        \draw (C) -- (B1) node [below, xshift= 0.4em] {$B_1$} node [pos=0.7, left]  {$a$};
        \draw (C) -- (B2) node [below, xshift=-0.4em] {$B_2$} node [pos=0.7, right] {$a$};

        \pgfmathanglebetweenpoints{\pgfpointanchor{B2}{base}}{\pgfpointanchor{C}{base}}
        \pgfmathsetmacro{\jiaocbp}{\pgfmathresult}
        \draw [dashed] (B2) arc (180+\jiaocbp:360-\jiaocbp:\a);
        \draw (B2) arc (180+\jiaocbp:180+\jiaocbp-10:\a)
              (B1) arc (360-\jiaocbp:360-\jiaocbp+10:\a);

        \node at (1, -1) {$b\sin A < a < b$};
        \node at (1, -1.5) {两解};
    \end{scope}


    \begin{scope}[xshift=11cm]
        \pgfmathsetmacro{\a}{1.9}
        \coordinate (A) at (0, 0);
        \coordinate (C) at (1.4, 1);
        \coordinate (P) at (4, 0);

        \path [name path=ap] (A) -- (P);
        \path [name path=cb] (C) + (270:\a) arc (270:360:\a);
        \path [name intersections={of=ap and cb, by={B}}];

        \draw [thick] (B)
            -- (A) node [below] {$A$}
            -- (C) node [above] {$C$} node [midway, above] {$b$}
            -- (B) node [below] {$B$} node [midway, right] {$a$};
        \draw (B) -- (P);
        \draw pic [draw] {angle=B--A--C};
        \draw [densely dashed] (C) + (310:\a) arc (310:340:\a);
        \draw [densely dashed] (C) + (200:\a) arc (200:230:\a);

        \node at (1.5, -1) {$a \geqslant b$};
        \node at (1.5, -1.5) {一解};
    \end{scope}
\end{tikzpicture}


\end{figure}


(2) $A$ 为直角或钝角:

\begin{figure}[H]%[htbp]
    \centering
    \begin{tikzpicture}
    \begin{scope}
        \pgfmathsetmacro{\r}{1.3}
        \coordinate (A) at (0, 0);
        \coordinate (C) at (-1, 1.5);
        \coordinate (B) at (2, 0);
        \draw [thick] (B)
            -- (A) node [below] {$A$}
            -- (C) node [above] {$C$} node [pos=0.6, below] {$b$}
            -- ($(C) + (335:\r)$) node [midway, above] {$a$};
        \draw pic [draw, angle radius=0.8em] {angle=B--A--C};
        \draw [densely dashed] (C) + (290:\r) arc (290:350:\r);

        \node at (1, -1) {$a \leqslant b$};
        \node at (1, -1.5) {无解};
    \end{scope}

    \begin{scope}[xshift=5cm]
        \pgfmathsetmacro{\a}{3}
        \coordinate (A) at (0, 0);
        \coordinate (C) at (-1, 1.5);
        \coordinate (P) at (2.5, 0);

        \path [name path=ap] (A) -- (P);
        \path [name path=cb] (C) + (270:\a) arc (270:360:\a);
        \path [name intersections={of=ap and cb, by={B}}];

        \draw [thick] (B)
            -- (A) node [below] {$A$}
            -- (C) node [above] {$C$} node [midway, below] {$b$}
            -- (B) node [below] {$B$} node [midway, above] {$a$};
        \draw (B) -- (P);
        \draw pic [draw, angle radius=0.8em] {angle=B--A--C};
        \draw [densely dashed] (C) + (310:\a) arc (310:340:\a);

        \node at (1, -1) {$a > b$};
        \node at (1, -1.5) {一解};
    \end{scope}
\end{tikzpicture}


    \caption{}\label{fig:15-25}
\end{figure}


\end{enhancedline}


\lianxi
\begin{xiaotis}

\xiaoti{在三角形 $ABC$ 中:}
\begin{xiaoxiaotis}

    \xxt{已知 $c = \sqrt{3}$, $A = 45^\circ$, $B = 60^\circ$, 求 $b$, $S_{\triangle}$;}

    \xxt{已知 $b = 12$, $A = 30^\circ$, $B = 120^\circ$, 求 $a$。}

    \hspace*{1.5em} (结果保留两个有效数字)

\end{xiaoxiaotis}


\xiaoti{根据下列条件解三角形(如果有解,角度精确到 $1^\circ$, 边长保留两个有效数字):}
\begin{xiaoxiaotis}

    \xxt{$a = 15$, $b = 10$, $A = 60^\circ$;}

    \xxt{$b = 40$, $c = 20$, $C = 25^\circ$;}

    \xxt{$b = 11$, $a = 25$, $B = 30^\circ$。}

\end{xiaoxiaotis}

\end{xiaotis}

