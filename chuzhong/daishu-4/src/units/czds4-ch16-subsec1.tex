% 原书的目录结构就是如此(缺少 section)
% 忽略这里的报错: Difference (2) between bookmark levels is greater (hyperref)	than one, level fixed.
\subsection{总体和样本}\label{subsec:16-1}

看下面的例子:

1. 有关部门要了解某个地区初中三年级学生的体重,以掌握这些学生的身体发育情况。
于是,这个地区所有初三学生的体重,就是所要考察的对象。
但由于这个地区的初三学生较多,逐一考察比较困难,
于是只能抽查其中一部分(比如说,$200$ 名)学生的体重,
然后根据这一部分学生的体重去估计所有学生的体重情况。

2. 要了解一块玉米地里所有单株玉米的产量情况(例如,了解它们的平均产量)。
这样,这块地里所有单株玉米的产量就是所要考察的对象。
但是,由于这块地里的玉米株数很多,无法一一加以考察,
只能从中抽取一部分(比如说,抽取 $100$ 株)单株玉米,
然后用这一部分单株玉米的产量,去估计这块地里所有单株玉米的产量。

3. 要考察某批炮弹的杀伤半径(例如,考察杀伤半径的范围)。
这种考察是带有破坏性的,因为测量一个就得发射一个。
这样,即使炮弹的个数不是很多,也不可能一一进行试验,
而是从中抽取一部分(比如说,抽取 $10$ 发)炮弹来进行试验,
然后用这一部分炮弹的杀伤半径,去估计这批炮弹中所有炮弹的杀伤半径。

我们所要考察的对象的全体叫做\zhongdian{总体},
其中每一个考察对象叫做\zhongdian{个体},
从总体中抽取的一部分个体叫做总体的一个\zhongdian{样本},
样本中个体的数目叫做\zhongdian{样本的容量}。

在第一个例子中,所指地区初三学生体重的全体是总体,每个学生的体重是个体,
从中抽取的 $200$ 名学生的体重是总体的一个样本,样本的容量是 $200$。

在第二个例子中,玉米地里单株玉米产量的全体是总体,每个单株玉米的产量是个体,
从中抽取的 $100$ 个单株玉米的产量是总体的一个样本,样本的容量是 $100$。

在第三个例子中,炮弹杀伤半径的全体是总体,每发炮弹的杀伤半径是个体,
从中抽取的 $10$ 发炮弹的杀伤半径是总体的一个样本,样本的容量是 $10$。

我们看到,在统计里所谈到的考察对象,是一种数量指标。
如在例 1 中,我们考察的对象是所指地区各初三学生的体重,而不是笼统指这些学生本身。

总体中包含的个体数往往很多(如第一个例子和第二个例子),
有时虽然总体中包含的个体数不是很多,但考察时带有破坏性(如第三个例子),
因此,我们通常是从总体中抽取一个样本,然后根据样本的特性去估计总体的相应特性。


\lianxi
\begin{xiaotis}

\xiaoti{说明在以下问题中,总休、个体、样本、样本的容量各指什么。}
\begin{xiaoxiaotis}

    \xxt{为了考察某地初中毕业生数学升学考试的情况,从中抽查了 $200$ 名考生的成绩。}

    \xxt{为了了解一批灯泡的使用寿命,从中抽取了 $10$ 个进行试验检查。}

    \xxt{为了考察某公园一年中每天进园的人数,在其中的 $30$ 天里对进园的人数进行了统计。}

\end{xiaoxiaotis}

\xiaoti{举一个在实际生活中通过样本研究总体的例子。}

\end{xiaotis}

