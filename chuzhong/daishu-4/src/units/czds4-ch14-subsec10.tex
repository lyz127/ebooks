\subsection{二次函数 $y = ax^2$ 的图象和性质}\label{subsec:14-10}

我们先研究特殊的二次函数的图象和性质,然后再研究一般的二次函数的图象和性质。

\liti 画出函数 $y = x^2$ 与 $y = -x^2$ 的图象。

\jie 在 $x$ 的取值范围内列出函数的对应值表:
\begin{table}[H]
    \hspace*{2em}
    \begin{tblr}{
        hlines, vlines,
        columns={3em, mode=math, c, colsep=0pt},
        column{1}={5em},
        rows={rowsep=0.5em},
    }
        x     & \cdots & -2 & -1\dfrac{1}{2} & -1 & -\dfrac{1}{2} & 0 & \dfrac{1}{2} & 1 & 1\dfrac{1}{2} & 2 & \cdots \\
        y=x^2 & \cdots &  4 &  2\dfrac{1}{4} &  1 &  \dfrac{1}{4} & 0 & \dfrac{1}{4} & 1 & 2\dfrac{1}{4} & 4 & \cdots
    \end{tblr}
\end{table}

用表里各组对应值作为点的坐标,进行描点,然后用平滑曲线把它们顺次连结起来,就得到函数 $y = x^2$ 的图象(图 \ref{fig:14-22})。

\begin{figure}[htbp]
    \centering
    \begin{minipage}[b]{7cm}
    \centering
    \begin{tikzpicture}[>=Stealth, scale=0.5,
    every node/.style={fill=white, inner sep=1pt},
]
    \draw [->] (-5, 0) -- (5, 0) node[below=0.2em] {$x$} coordinate(x axis);
    \draw [->] (0, -2) -- (0, 7) node[left=0.2em]  {$y$} coordinate(y axis);
    \draw (0, 0) node [below left=0.3em] {\small $O$};
    \foreach \x in {-4, ..., 4} {
        \draw (\x, 0) -- (\x, 0.2);
    }
    \foreach \y in {-1, ..., 6} {
        \draw (0.2, \y) -- (0, \y);
    }

    \draw[domain=-2.2:2.2,  samples=50] plot (\x, {\x*\x}) node [above] {$y = x^2$};
    \foreach \x in {-2, -1.5, ..., 2} {
        \draw [fill=black] (\x, \x*\x) circle(0.1);
    }
\end{tikzpicture}


    \caption{}\label{fig:14-22}
    \end{minipage}
    \qquad
    \begin{minipage}[b]{7cm}
    \centering
    \begin{tikzpicture}[>=Stealth, scale=0.5,
    every node/.style={fill=white, inner sep=1pt},
]
    \draw [->] (-5, 0) -- (5, 0) node[below=0.2em] {$x$} coordinate(x axis);
    \draw [->] (0, -6) -- (0, 3) node[left=0.2em]  {$y$} coordinate(y axis);
    \draw (0, 0) node [above left=0.3em] {\small $O$};
    \foreach \x in {-4, ..., 4} {
        \draw (\x, 0) -- (\x, 0.2);
    }
    \foreach \y in {-5, ..., 2} {
        \draw (0.2, \y) -- (0, \y);
    }

    \draw[domain=-2.2:2.2,  samples=50] plot (\x, {-\x*\x}) node [below] {$y = -x^2$};
    \foreach \x in {-2, -1.5, ..., 2} {
        \draw [fill=black] (\x, -\x*\x) circle(0.1);
    }
\end{tikzpicture}


    \caption{}\label{fig:14-23}
    \end{minipage}
\end{figure}

用同样的方法,可以画出函数 $y = -x^2$ 的图象(图 \ref{fig:14-23})。


\begin{enhancedline}
\liti 画出函数 $y = \dfrac{1}{2}x^2$ 与 $y = 2x^2$ 的图象。

\jie 先画函数 $y = \dfrac{1}{2}x^2$ 的图象。在 $x$ 的取值范围内列出函数的对应值表:
\begin{table}[H]
    \hspace*{2em}
    \begin{tblr}{
        hlines, vlines,
        columns={3em, mode=math, c, colsep=0pt},
        column{1}={5em},
        rows={rowsep=0.5em},
    }
        x & \cdots & -4 & -3            & -2 & -1           & 0 & 1            & 2 & 3             & 4 & \cdots \\
        y & \cdots &  8 & 4\dfrac{1}{2} &  2 & \dfrac{1}{2} & 0 & \dfrac{1}{2} & 2 & 4\dfrac{1}{2} & 8 & \cdots
    \end{tblr}
\end{table}

用表里各组对应值作为点的坐标,进行描点,然后用平滑的曲线把它们顺次连结起来,就得到函数 $y = \dfrac{1}{2}x^2$ 的图象(图 \ref{fig:14-24})。

\begin{wrapfigure}[13]{r}{6cm}
    \centering
    \begin{tikzpicture}[>=Stealth, scale=0.5,
    every node/.style={fill=white, inner sep=1pt},
]
    \draw [->] (-5, 0) -- (5, 0) node[below=0.2em] {$x$} coordinate(x axis);
    \draw [->] (0, -2) -- (0, 10) node[left=0.2em]  {$y$} coordinate(y axis);
    \draw (0, 0) node [below left=0.3em] {\small $O$};
    \foreach \x in {-4, ..., 4} {
        \draw (\x, 0) -- (\x, 0.2);
    }
    \foreach \y in {-1, ..., 9} {
        \draw (0.2, \y) -- (0, \y);
    }

    \draw[domain=-4.1:4.1,  samples=50] plot (\x, {\x*\x/2}) (3, 4) node [right] {$y = \frac{1}{2}x^2$};
    \draw[domain=-2.1:2.1,  samples=50] plot (\x, {2*\x*\x}) node [above] {$y = 2x^2$};
\end{tikzpicture}


    \caption{}\label{fig:14-24}
\end{wrapfigure}

用同样的方法,可以画出函数 $y = 2x^2$ 的图象。我们把它画在图 \ref{fig:14-24} 所示的坐标系内。


函数 $y = ax^2$ 的图象形如物体抛射时所经过的路线,我们把它叫做\zhongdian{抛物线}。
这条抛物线关于 $y$ 轴对称,$y$ 轴叫做抛物线的\zhongdian{对称轴}。
对称轴与抛物线的交点叫做抛物线的\zhongdian{顶点}, 这条抛物线的顶点是原点。

从图 \ref{fig:14-22},\ref{fig:14-23}, \ref{fig:14-24} 可以看出,\zhongdian{二次函数 $\bm{y = ax^2}$ 有下列性质}:


\zhongdian{(1) \; 抛物线 $\bm{y = ax^2}$ 的顶点是原点,对称轴是 $\bm{y}$ 轴。}

\zhongdian{(2) \; 当 $\bm{a > 0}$ 时,抛物线 $\bm{y = ax^2}$ 在 $\bm{x}$ 轴的上方(顶点在 $\bm{x}$ 轴上),它的开口向上,并且向上无限伸展;
                  当 $\bm{a < 0}$ 时,抛物线 $\bm{y = ax^2}$ 在 $\bm{x}$ 轴的下方(顶点在 $\bm{x}$ 轴上),它的开口向下,并且向下无限伸展。
}

\zhongdian{(3) \; 当 $\bm{a > 0}$ 时,
        在对称轴的左侧,$\bm{y}$ 随着 $\bm{x}$ 的增大而减小;
        在对称轴的右侧,$\bm{y}$ 随着 $\bm{x}$ 的增大而增大;
        函数 $\bm{y}$ 当 $\bm{x = 0}$ 时的的值最小。
    当 $\bm{a < 0}$ 时,
        在对称轴的左侧,$\bm{y}$ 随着 $\bm{x}$ 的增大而增大;
        在对称轴的右侧,$\bm{y}$ 随着 $\bm{x}$ 的增大而减小;
        函数 $\bm{y}$ 当 $\bm{x = 0}$ 时的的值最大。
}

\end{enhancedline}



\lianxi
\begin{xiaotis}

\xiaoti{在同一坐标系内,画出下列函数的图象,并比较它们的位置关系:}
\begin{xiaoxiaotis}

    \begin{tblr}{columns={18em, colsep=0pt}, rows={rowsep=0.5em}}
        \xxt{$y = \dfrac{2}{3}x^2$;} & \xxt{$y = -\dfrac{2}{3}x^2$。}
    \end{tblr}
\end{xiaoxiaotis}


\xiaoti{圆面积公式为 $A = \pi r^2$,其中 $r$ 为圆的半径,$A$ 为圆的面积,$\pi$ 取 $3.14$。}
\begin{xiaoxiaotis}

    \xxt{求 $r = 3$, $5$, $2.5$(厘米)时圆的面积;}

    \xxt{画出函数 $A = \pi r^2 \; (0 < r \leqslant 8)$ 的图象;}

    \xxt{根据图象,求面积 $A = 20$,$40$,$60$($\pflm$)时圆的半径。}

\end{xiaoxiaotis}


\end{xiaotis}
