\xiaojie

一、本章内容主要是三角函数的初步概念及解三角形的方法。


二、设角 $\alpha$ 的顶点在原点 $O$,始边与 $x$ 轴的正半轴重合,
终边上任一点 $P$ 的坐标是 $(x,\; y)$, $OP = r$。角 $\alpha$ 的三角函数是:
\begin{center}
    \begin{tblr}{columns={colsep=0pt}, rows={rowsep=0.5em}}
        $\alpha$ 的正弦 $\sin\alpha = \exdfrac{y}{r}$, & $\alpha$ 的余弦 $\cos\alpha = \exdfrac{x}{r}$, \\
        $\alpha$ 的正切 $\tan\alpha = \exdfrac{y}{x}$, & $\alpha$ 的余切 $\cot\alpha = \exdfrac{x}{y}$。
    \end{tblr}
\end{center}

三角函数间有下列关系:

(1) \begin{tblr}[t]{columns={mode=math, colsep=0pt},}
    \sin(90^\circ - \alpha) = \cos\alpha \douhao \\
    \cos(90^\circ - \alpha) = \sin\alpha \douhao \\
    \tan(90^\circ - \alpha) = \cot\alpha \douhao \\
    \cot(90^\circ - \alpha) = \tan\alpha \juhao \\
\end{tblr}

(2) \begin{tblr}[t]{columns={mode=math, colsep=0pt},}
    \sin(180^\circ - \alpha) = \sin\alpha \douhao \\
    \cos(180^\circ - \alpha) = -\cos\alpha \douhao \\
    \tan(180^\circ - \alpha) = -\tan\alpha \douhao \\
    \cot(180^\circ - \alpha) = -\cot\alpha \juhao \\
\end{tblr}


三、特殊角的三角函数值:

\begin{table}[H]
    \centering
    \newcommand{\tbhead}{\diagboxthree[width=8em, height=4em, trim=l]{三角函数}{三角函数值}{$\alpha$}}
    \begin{tblr}{
        hlines,vlines,
        columns={c, mode=math},
        column{2-5}={5em, colsep=0pt},
        cell{1}{1}={mode=text},
        rows={m, rowsep=0.5em},
    }
        \tbhead     & 30^\circ            & 40^\circ             & 60^\circ             & 90^\circ\\
        \sin\alpha  & \exdfrac{1}{2}      & \dfrac{\sqrt{2}}{2}  & \dfrac{\sqrt{3}}{2}  & 1 \\
        \cos\alpha  & \dfrac{\sqrt{3}}{2} & \dfrac{\sqrt{2}}{2}  & \exdfrac{1}{2}       & 0 \\
        \tan\alpha  & \dfrac{\sqrt{3}}{3} & 1                    & \sqrt{3}             & \text{不存在} \\
        \cot\alpha  & \sqrt{3}            & 1                    & \dfrac{\sqrt{3}}{3}  & 0 \\
    \end{tblr}
\end{table}

锐角的三角函数值可以查三角函数表求得。
求钝角的三角函数值,可以根据角 $(180^\circ - \alpha)$ 与角 $\alpha$ 间的三角函数关系式,
把它转化为求锐角的三角函数值。

反过来,如果已知锐角或钝角的三角函数值,可以求得这个锐角或钝角的值。



四、设直角三角形的一个锐角是 $\alpha$,
\begin{center}
    \begin{tblr}{columns={mode=math}, column{1}={leftsep=0pt}, rows={rowsep=0.5em}}
        \sin \alpha = \dfrac{\alpha \text{的对边}}{\text{斜边}} \douhao          & \cos \alpha = \dfrac{\alpha \text{的邻边}}{\text{斜边}} \douhao \\
        \tan \alpha = \dfrac{\alpha \text{的对边}}{\alpha \text{的邻边}} \douhao & \cot \alpha = \dfrac{\alpha \text{的邻边}}{\alpha \text{的对边}} \juhao
    \end{tblr}
\end{center}

利用这些关系式以及勾股定理和两锐角互余的关系,可以解直角三角形,
关键在于适当选用恰含一个未量的关系式。


五、任意三角形 $ABC$ 中,边与角间有如下的关系:

(1) 余弦定理
\begin{gather*}
    a^2 = b^2 + c^2 - 2bc \cos A \douhao \\
    b^2 = c^2 + a^2 - 2ca \cos B \douhao \\
    c^2 = a^2 + b^2 - 2ab \cos C \fenhao
\end{gather*}

(2) 正弦定理
$$ \dfrac{a}{\sin{A}} = \dfrac{b}{\sin{B}} = \dfrac{c}{\sin{C}} \juhao $$


解任意三角形的问题可以分为下列四种类型:

(1) 已知三边;

(2) 已知两边和它们的夹角;

(3) 已知两角和一边;

(4) 已知两边和其中一边的对角。

一般地,
(1), (2) 两种类型可用余弦定理和三角形内角和定理来解;
(3), (4) 两种类型可用正弦定理和三角形内角和定理来解。
第 (1),(2),(3) 三种类型的问题,或者只有一解,或者无解。
第 (4) 种类型的问题,可有两解、一解或无解。

根据三角形中已知的边与角(至少包含一条边)的大小,可以计算出未知的边与角的大小。
这样从数量上进一步阐明了三角形中各元素间的关系,使我们对三角形的认识又深入了一步。

