\subsection{常用对数}\label{subsec:13-3}

由于我们通常用的是十进制记数法,所以我们通常用的对数是以 $10$ 为底的对数,这种对数叫做\zhongdian{常用对数}。
在表示常用对数的时候,通常把底数 $10$ 略去,把 “$\log$” 简写成“$\lg$”,
也就是 $N$ 的常用对数 $\log_{10}{N}$ 可以简写成 $\lg{N}$。例如,$\log_{10}{5}$ 写成 $\lg{5}$。
如不作特殊说明,一般所说的对数都是指常用对数。例如说 $100$ 的对数是 $2$,就是说 $100$ 的常用对数是 $2$。

我们知道:
\begin{center}
    \begin{tblr}{columns={mode=math, 7em}, column{1}={rightsep=5em}}
        \cdots \cdots \cdots & \cdots \cdots \cdots  \\
        10^3 = 1000,         & \lg{1000} = 3, \\
        10^2 = 100,          & \lg{100} = 2, \\
        10^1 = 10,           & \lg{10} = 1, \\
        10^0 = 1,            & \lg{1} = 0,
    \end{tblr}

    \begin{tblr}{columns={mode=math, 7em}, column{1}={rightsep=5em}}
        10^{-1} = 0.1,       & \lg{0.1} = -1, \\
        10^{-2} = 0.01,      & \lg{0.01} = -2, \\
        10^{-3} = 0.001,     & \lg{0.001} = -3, \\
        \cdots \cdots \cdots & \cdots \cdots \cdots
    \end{tblr}
\end{center}

可以看出,$10$ 的整数次幂的对数是一个整数,并且真数较大的时候,它的对数也较大。

一个正数,如果不是 $10$ 的整数次幂,它的对数是一个小数。
例如,$\lg{72}$ 是 $1$ 与 $2$ 之间的一个小数,
$\lg{0.0072}$ 是 $-3$ 与 $-2$ 之间的一个小数。

一个正数的对数(具有一定精确度的近似值),可以利用“对数表”求得。

