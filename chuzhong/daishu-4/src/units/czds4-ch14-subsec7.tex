\subsection{一次函数}\label{subsec:14-7}

看下面的例子:

(1) 汽车离开 $A$ 站 $4$ 公里后,以 $40$ 公里/时的平均速度前进了 $t$ 时,
那么汽车离开 $A$ 站的距离 $s$(公里)与时间 $t$(时)之间的函数关系式是
$$ s = 40\,t + 4 \juhao $$

(2) 拖拉机开始工作时,油箱中有油 $40$ 千克。如果每小时耗油 $6$ 千克,
那么油箱中的余油量 $Q$(千克)与它工作的时间 $t$(时)之间的函数关系式是
$$ Q = 40 - 6\,t = -6\,t + 40 \juhao $$

这些函数关系式都具有 $y = kx + b \; (k \neq 0)$ 的形式。

函数 $y = kx + b$ 叫做 $x$ 的\zhongdian{一次函数},这里 $x$ 是自变量,$k$,$b$ 都是常数,且 $k \neq 0$。

如果 $b = 0$, 一次函数 $y = kx + b$ 就成为 $y = kx$,这就是正比例函数。
所以正比例函数是一次函数的特殊情形。


\liti[0] 汽车从 $A$ 站经 $B$ 站后以匀速 $v_0$ 公里/分\;开往 $C$ 站。
已知离开 $B$ 站 $9$ 分时,汽车离 $A$ 站 $10$ 公里,又行驶一刻钟,离 $A$ 站 $20$ 公里。
如果再行驶半小时,汽车离 $A$ 站多少公里?

\jie 设 $A$,$B$ 两站之间的距离为 $s_0$ 公里。离开 $B$ 站 $t$ 分时,汽车离 $A$ 站 $s$ 公里。于是由物理学可知
$$ s = v_0\,t + s_0 \juhao $$

把 $t = 9$,$s = 10$ 及 $t = 24$,$s = 20$ 分别代入上式,得
$$\begin{cases}
    10 = 9v_0 + s_0 \douhao \\
    20 = 24v_0 + s_0 \juhao
\end{cases}$$

\begin{enhancedline}
这是一个以 $v_0$,$s_0$ 为未知数的二元一次方程组。解这个方程组,得 $v_0 = \dfrac{2}{3}$,$s_0 = 4$。

\fengeSuoyi{s = \dfrac{2}{3}\,t + 4 \juhao}

当 $t = 54$ 时, $s = \dfrac{2}{3} \times 54 + 4 = 40$ (公里)。

答:离开 $B$ 站 $54$ 分时,汽车离 $A$ 站 $40$ 公里。


在上例中,我们先确定 $s$ 与 $t$ 的函数关系式为 $s = v_0\,t + s_0$ 的形式,其中 $v_0$,$s_0$ 是未知的系数,
然后根据条件“$t = 9$ 时 $s = 10$,$t = 24$ 时 $s = 20$”,求出未知系数 $v_0 = \dfrac{2}{3}$,$s_0 = 4$。
象这样,先设某些未知的系数,然后根据所给的条件来确定这些未知系数的方法是\zhongdian{待定系数法。}
待定系数法是数学中常用的一种方法,我们在以前曾经多次用到。
\end{enhancedline}


\lianxi

已知 $y - 3$ 与 $x$ 成正比例,且 $x = 2$ 时 $y = 7$。
\begin{xiaoxiaotis}

    \xxt{写出 $y$ 与 $x$ 之间的函数关系式;}

    \xxt{计算 $x = 4$ 时 $y$ 的值;}

    \xxt{计算 $y = 4$ 时 $x$ 的值。}

\end{xiaoxiaotis}

