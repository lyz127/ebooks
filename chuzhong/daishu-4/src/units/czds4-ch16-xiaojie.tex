\xiaojie

一、这一章我们介绍了统计的一些初步知识。统计方法的特点是:
从所要考察的总体中抽取一个样本,通过研究样本对总体作出估计。
样本容量越大,这种估计也就越精确。


二、总体平均数是表示总体的平均大小(或平均水平)的特征数,通常用样本平均数去估计它。
样本平均数是指样本中各数据的和与样本容量的比值,当样本数据校大时,
利用公式 $\overline{x} = \overline{x'} + a$ 计算样本平均数比较简便。


三、总体方差是表示总体的波动大小的特征数,通常用样本方差去估计它,
用比较两个样本方差去近似地比较两个相应的总体的波动大小。
对干一个容量为 $n$、平均数为 $\overline{x}$ 的样本来说,样本方差
$$ s^2 = \exdfrac{1}{n} \sum_{i=1}^n (x_i - \overline{x})^2 \juhao $$
样本方差的计算较烦,在没有电子计算器的情况下,一般根据简化公式列表进行计算。


四、总体分布反映了总体中各部分个体在总体中所占的比例大小,通常用样本的频率分布去估计它。
频率分布反映了样本数据在各个小范围内所占的比例大小。
要锝到一个样本的频率分布情况,可按下列步驟进行:
计算样本中最大值与最小值的差,决定组距与组数,
决定分点,列频率分布表,绘频率分布直方图。

