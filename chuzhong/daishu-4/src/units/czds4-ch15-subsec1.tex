\subsection{三角函数}\label{subsec:15-1}
\begin{enhancedline}

修建扬水站时,沿着与水平面成 $\alpha$ 角的斜坡铺设水管。
如图 \ref{fig:15-1},当水管从坡底 $O$ 处向上铺设到 $P$ 处,$P$ 离水平面的高为 $MP$,
继续向上铺设到 $P'$ 处,$P'$ 离水平面的高为 $M'P'$,铺设的水管越长,管口离水平面也越高。
容易看出,$\triangle OPM \xiangsi \triangle OP'M'$,因此 $\dfrac{MP}{OP} = \dfrac{M'P'}{OP'}$。
这就是说,当角 $\alpha$ 的大小一定时,管口离水平面的高与管长的比是一个定值。

\begin{figure}[htbp]
    \centering
    \begin{minipage}[b]{7cm}
    \centering
    \begin{tikzpicture}[>=Stealth]
    \tikzset{
        shuiguan/.pic={
            \draw (0, 0.1) rectangle (0.6, 0.2);
            \draw (0.6, 0.1) -- (0.75, 0) -- (0.75, 0.3) -- (0.6, 0.2);
        }
    }

    \pgfmathsetmacro{\jiao}{30}
    \coordinate (O) at (0, 0);
    \coordinate (A) at (\jiao:5);
    \coordinate (B) at (0:5);

    \begin{scope}
        \clip (O) rectangle (A); %删除左正角 (ground) 多余的斜线
        \draw [ground, name path=gnd] (0, 0) -- (A);
    \end{scope}

    \draw [thick] (A) -- (O) node [below left] {$O$} -- (B);
    \draw [->] (1, 0) arc (0:\jiao:1) node [midway, right] {$\alpha$};
    \foreach \x in {0, ..., 3} {
        \draw [rotate=\jiao, transform shape] (0.75 * \x, 0) pic {shuiguan};
    }

    \coordinate (P)  at (\jiao:3);
    \coordinate (Pp) at (\jiao:4);
    \draw (P)  node [above, xshift=0.3em] {$P$}  to[chuizu={direction=left}] (P  |- B) node [below] {$M$};
    \draw (Pp) node [above] {$P'$} to[chuizu={direction=left}] (Pp |- B) node [below] {$M'$};
\end{tikzpicture}


    \caption{}\label{fig:15-1}
    \end{minipage}
    \qquad
    \begin{minipage}[b]{7cm}
    \centering
    \begin{tikzpicture}[>=Stealth]
    \draw [->] (-0.5, 0) -- (5, 0) node[below=0.2em] {$x$} coordinate(x axis);
    \draw [->] (0, -0.5) -- (0, 3) node[left=0.2em]  {$y$} coordinate(y axis);
    \draw (0, 0) node [below left=0.3em] {\small $O$};

    \pgfmathsetmacro{\jiao}{30}
    \coordinate (A) at (\jiao:5);
    \coordinate (P)  at (\jiao:3);
    \coordinate (Pp) at (\jiao:4);

    \draw (0, 0) -- (A);
    \draw [->] (1, 0) arc (0:\jiao:1) node [midway, right] {$\alpha$};
    \draw [densely dashed] (P)  node [above, xshift=-1.5em] {$P(x,y)$}    -- (P  |- x axis) node [below] {$M$} node [midway, right] {$y$};
    \draw [densely dashed] (Pp) node [above, xshift=-1.5em] {$P'(x',y')$} -- (Pp |- x axis) node [below] {$M'$};
    \draw (P)  to [chuizu={skipline=true}] (P  |- x axis);  %\draw (P  |- x axis) +(0, 0.2) -- +(0.2, 0.2) -- +(0.2, 0);
    \draw (Pp) to [chuizu={skipline=true}] (Pp |- x axis);  %\draw (Pp |- x axis) +(0, 0.2) -- +(0.2, 0.2) -- +(0.2, 0);

    \draw (1.5, 0) node [below] {$x$};
    \draw (\jiao:1.5) node [above] {$r$};
\end{tikzpicture}


    \caption{}\label{fig:15-2}
    \end{minipage}
\end{figure}

想一想:当角 $\alpha$ 是 $30^\circ$ 时,上面所说的比值是多少。

上面的实例启发我们进一步研究与角有关的一些比。

设有一个角 $\alpha$,我们以它的顶点作为原点,以它的始边作为 $x$ 轴的正半轴 $Ox$,建立直角坐标系(图 \ref{fig:15-2})。
在角 $\alpha$ 的终边上任取一点 $P(x,\; y)$,它和原点 $O(0,\; 0)$ 的距离是
$r = \sqrt{x^2 + y^2}$ ($r$ 总是正的)。可以得到比值
$$ \dfrac{y}{r}\nsep  \dfrac{x}{r}\nsep \dfrac{y}{x}\nsep \dfrac{x}{y} \juhao $$
设 $P'(x',\; y')$ 是角 $\alpha$ 的终边上另一点,$P'$ 到 $O$ 的距离为
$r' = \sqrt{x'^{\,2} + y'^{\,2}}$。又可以得到比值
$$ \dfrac{y'}{r'}\nsep  \dfrac{x'}{r'}\nsep \dfrac{y'}{x'}\nsep \dfrac{x'}{y'} \juhao $$

过 $P$ 和 $P'$ 分别画 $x$ 轴的垂线 $MP$ 和 $M'P'$, 那么 $\triangle OPM \xiangsi \triangle OP'M'$,
且 $x$ 和 $x'$、$y$ 和 $y'$ 的符号相同,所以
$$ \dfrac{y}{r} = \dfrac{y'}{r'} \nsep \dfrac{x}{r} = \dfrac{x'}{r'} \nsep \dfrac{y}{x} = \dfrac{y'}{x'} \nsep \dfrac{x}{y} = \dfrac{x'}{y'} \juhao $$
由此可知,对于确定的角 $\alpha$,这四个比值都是由角 $\alpha$ 的大小唯一确定的,
与点 $P$ 在角 $\alpha$ 的终边上的位置无关,
所以这四个比值都是自变量 $\alpha$ 的函数.我们把

$\dfrac{y}{r}$ 叫做角 $\alpha$ 的\zhongdian{正弦},记作 $\sin\alpha$, 即 $\sin\alpha = \dfrac{y}{r}$;

$\dfrac{x}{r}$ 叫做角 $\alpha$ 的\zhongdian{余弦},记作 $\cos\alpha$, 即 $\cos\alpha = \dfrac{x}{r}$;

$\dfrac{y}{x}$ 叫做角 $\alpha$ 的\zhongdian{正切},记作 $\tan\alpha$, 即 $\tan\alpha = \dfrac{y}{x}$;\footnote{
    录注:原书中 $\tan\alpha$ 写作 $\text{tg}\,\alpha$,$\cot\alpha$ 写作 $\text{ctg}\,\alpha$。录入过程中,统一按 $\tan\alpha$、$\cot\alpha$ 的形式录入。
}

$\dfrac{x}{y}$ 叫做角 $\alpha$ 的\zhongdian{余切},记作 $\cot\alpha$,\footnotemark 即 $\cot\alpha = \dfrac{x}{y}$;
\footnotetext{“$\text{tg}\,\alpha$” 也可以记作“$\tan\alpha$”,“$\text{ctg}\,\alpha$” 也可以记作“$\cot\alpha$”。}


角 $\alpha$ 的正弦 $\sin\alpha$,角 $\alpha$ 的余弦 $\cos\alpha$,
角 $\alpha$ 的正切 $\tan\alpha$,角 $\alpha$ 的余切 $\cot\alpha$
都叫做角 $\alpha$ 的\zhongdian{三角函数}。

\zhuyi “$\sin\alpha$” 是一个完整的符号,它表示角 $\alpha$ 的正弦,不能理解成 $\sin \cdot \alpha$。
其他三角函数的符号也是这样。


\liti 已知角 $\alpha$ 的终边经过点 $P(3,\; 4)$, 求角 $\alpha$ 的四个三角函数值(图 \ref{fig:15-3})。

\begin{wrapfigure}[9]{r}{5cm}
    \centering
    \begin{tikzpicture}[>=Stealth, scale=0.7]
    \draw [->] (-0.5, 0) -- (5, 0) node[below=0.2em] {$x$} coordinate(x axis);
    \draw [->] (0, -0.5) -- (0, 6) node[left=0.2em]  {$y$} coordinate(y axis);
    \draw (0, 0) coordinate (O) node [below left=0.3em] {\small $O$};
    \foreach \x in {1, ..., 4} {
        \draw (\x, 0) -- (\x, 0.2);
    }

    \foreach \y in {1, ..., 5} {
        \draw (0.2, \y) -- (0, \y);
    }

    \pgfmathsetmacro{\jiao}{53.13}
    \coordinate (P)  at (3, 4);
    \draw (O) -- ($(O) ! 1.3 ! (P)$);
    \draw [->] (1.5, 0) arc (0:\jiao:1.5) node [midway, above right] {$\alpha$};
    \draw [densely dashed] (P)  node [right] {$P(3,4)$}    -- (P  |- x axis);
\end{tikzpicture}


    \caption{}\label{fig:15-3}
\end{wrapfigure}


\jie $\because \quad x = 3\nsep y = 4$,

$\therefore$ \quad $r = \sqrt{x^2 + y^2} = \sqrt{3^2 + 4^2} = 5$。

$\therefore$ \quad \begin{tblr}[t]{columns={colsep=0pt, mode=math}, rows={rowsep=0.5em}}
    \sin\alpha = \dfrac{y}{r} = \dfrac{4}{5} \douhao \\
    \cos\alpha = \dfrac{x}{r} = \dfrac{3}{5} \douhao \\
    \tan\alpha = \dfrac{y}{x} = \dfrac{4}{3} \douhao \\
    \cot\alpha = \dfrac{x}{y} = \dfrac{3}{4} \juhao \\
\end{tblr}


\liti 求证:
\begin{xiaoxiaotis}

    \xxt{$\tan\alpha = \dfrac{\sin\alpha}{\cos\alpha}$;} \quad
    \xxt{$\sin^2\alpha + \cos^2\alpha = 1$。}\footnote{$\sin^2\alpha$ 表示 $(\sin\alpha)^2$,其他三角函数的幂也这样表示。}

\resetxxt
\zhengming \xxt{根据三角函数的定义,}
$$ \dfrac{\sin\alpha}{\cos\alpha} = \dfrac{\phantom{.}\dfrac{y}{r}\phantom{.}}{\dfrac{x}{r}} = \dfrac{y}{x} = \tan\alpha \douhao $$

\fenge{即}{\centering $\tan\alpha = \dfrac{\sin\alpha}{\cos\alpha}$;}

\xxt{$\sin^2\alpha + \cos^2\alpha = \left(\dfrac{y}{r}\right)^2 + \left(\dfrac{x}{r}\right)^2 = \dfrac{y^2 + x^2}{r^2}$,}

\fengeYinwei{y^2 + x^2 = r^2 \douhao }

\fengeSuoyi{\sin^2\alpha + \cos^2\alpha = \dfrac{r^2}{r^2} = 1 \juhao}

\end{xiaoxiaotis}


\lianxi
\begin{xiaotis}

\xiaoti{(口答)$\sin\beta$ 是角 $\beta$ 的哪种三角数,表示怎样的比?$\tan\beta$ 呢? $\cos\beta$ 呢?$\cot\beta$ 呢?}

\xiaoti{已知角 $\alpha$ 的终边分别经过下列各点,求角 $\alpha$ 的四个三角函数值:}
\begin{xiaoxiaotis}

    \begin{tblr}{columns={9em, colsep=0pt}}
        \xxt{$(4,\; 3)$;} & \xxt{$(5,\; 12)$;} & \xxt{$(2,\; 2)$;} & \xxt{$(2,\; 3)$。}
    \end{tblr}
\end{xiaoxiaotis}

\end{xiaotis}

\end{enhancedline}
