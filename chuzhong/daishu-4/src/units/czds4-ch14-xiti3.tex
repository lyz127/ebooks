\xiti
\begin{xiaotis}

\xiaoti{描出横坐标 $x$ 分别等于 $-4$,$-3$,$-2$,$-1$,$0$,$1$,$2$,$3$,$4$,
    纵坐标 $y$ 由方程 $y = x^2$ 决定的各点,并用光滑曲线把各点依次连结起来。
}

\xiaoti{如图,$OA = 8$,$OB = 6$,求点 $A$,$B$ 的坐标。}

\begin{figure}[htbp]
    \centering
    \begin{minipage}{7cm}
    \centering
    \begin{tikzpicture}[>=Stealth,
    every node/.style={fill=white, inner sep=1pt},
]
    \draw [->] (-1.5, 0) -- (2.5, 0) node[below=0.2em] {$x$};
    \draw [->] (0, -.5) -- (0, 2.5) node[left=0.2em] {$y$};
    \draw (0, 0) coordinate(O) node [below left] {\small $O$};

    \coordinate [label=above:\small $A$] (A) at (45:2);
    \coordinate [label=above:\small $B$] (B) at (120:1.5);

    \draw (A) -- (O) --(B);
    \draw [<->] (1, 0) arc (0:45:1) node [midway, above right] {$45^\circ$};
    \draw [<->] (.8, 0) arc (0:120:.8) node [pos=.75, above=0.2] {$120^\circ$};
\end{tikzpicture}


    \caption*{(第 2 题)}
    \end{minipage}
    \qquad
    \begin{minipage}{7cm}
    \centering
    \begin{tikzpicture}[>=Stealth, scale=0.5,
    every label/.style={fill=white, inner sep=1pt},
]
    \draw [->] (-6, 0) -- (5, 0) node[below=0.2em] {$x$};
    \draw [->] (0, -2) -- (0, 5.5) node[left=0.2em] {$y$};
    \draw (0, 0) node [below left] {\small $O$};
    \foreach \x in {-4,...,3} {
        \draw (\x, 0) -- (\x, 0.2);
    }
    \foreach \y in {1,...,3} {
        \draw (-0.2, \y) -- (0, \y);
    }

    \coordinate [label=left:\small $A$] (A) at (-1, 3);
    \coordinate [label=left:\small $B$] (B) at (3, 0);
    \coordinate [label=left:\small $C$] (C) at (-4, 0);
    \pgfmathsetmacro{\r}{0.7}
    \pgfmathsetmacro{\R}{1.2}

    \filldraw [fill=black] (A) circle (0.05);
    \filldraw [fill=black] (B) circle (0.05);
    \filldraw [fill=black] (C) circle (0.05);

    \draw [thick] (A) circle (\r);
    \draw [thick] (B) circle (\r);
    \draw [thick] (C) circle (\r);

    \path (A) +(30:\R)  coordinate(pa);
    \path (C) +(150:\R) coordinate(pc);
    \path (B) +(270:\R) coordinate(pb);

    \draw [rounded corners] (pa)
        arc [radius=\R, start angle=30,  end angle=150]
        -- (pc)
        arc [radius=\R, start angle=150, end angle=270]
        -- (pb)
        arc [radius=\R, start angle=270, end angle=390]
        -- (pa);
\end{tikzpicture}


    \caption*{(第 9 题)}
    \end{minipage}
\end{figure}


\xiaoti{}%
\begin{xiaoxiaotis}%
    \xxt[\xxtsep]{点 $P(x,\; y)$ 在第一象限内,$x$,$y$ 应取什么符号?}

    \xxt{点 $Q(x,\; y)$ 在第三象限内,$x$,$y$ 应取什么符号?}

\end{xiaoxiaotis}


\xiaoti{在第一象限内两条坐标轴夹角平分线上的点,它们的横坐标与纵坐标之间有什么关系?在第二象限内呢?}

\xiaoti{一个菱形每边的长是 $5$ ,一条对角线的长是 $6$,取两条对角线所在的直线作为坐标轴,求四个顶点的坐标(有两种情况〕。}

\xiaoti{写出点 $P(a,\; b)$ 关于坐标轴及原点的对称点的坐标。}

\xiaoti{已知数轴上两点 $A$,$B$ 的坐标 $x_{_A}$,$x_{_B}$ 取下列各值,求 $A$,$B$ 间的距离。}
\begin{xiaoxiaotis}

    \begin{tblr}{columns={18em, colsep=0pt}}
        \xxt{$x_{_A} = 8\nsep x_{_B} = 6$;}  & \xxt{$x_{_A} = 2\nsep x_{_B} = -1$;} \\
        \xxt{$x_{_A} = -3\nsep x_{_B} = 0$;} & \xxt{$x_{_A} = 0\nsep x_{_B} = -8$。}
    \end{tblr}
\end{xiaoxiaotis}


\xiaoti{在下列各题中,分别以 $A$,$B$,$C$ 三点为顶点画三角形,求三角形各边的长,
    并判别所画出的三角形中哪些是等腰三角形,哪些是等边三角形,哪些是直角三角形。
}
\begin{xiaoxiaotis}

    \begin{tblr}{column{1}={colsep=0pt}, column{2-4}={mode=math}, column{2}={leftsep=0pt}}
        \xxt{} & A(-3,\; 0), & B(3,\;  0), & C(  0,\; 3\sqrt{3})\fenhao \\
        \xxt{} & A(-4,\; 3), & B(2,\; -5), & C(  0,\; 6)\fenhao \\
        \xxt{} & A( 5,\; 1), & B(2,\; -2), & C(2.5,\; 0.5)\fenhao \\
        \xxt{} & A( 3,\; 0), & B(6,\;  4), & C( -1,\; 3)\juhao
    \end{tblr}

    %\xxt{\threeInLine[5em]{$A(-3,\; 0)$,}{$B(3,\; 0)$,}{$C(0,\; 3\sqrt{3})$;}}

    %\xxt{\threeInLine[5em]{$A(-4,\; 0)$,}{$B(6,\; 4)$,}{$C(-1,\; 3)$;}}

    %\xxt{\threeInLine[5em]{$A(3,\; 0)$,}{$B(6,\; 4)$,}{$C(-1,\; 3)$;}}

    %\xxt{\threeInLine[5em]{$A(3,\; 0)$,}{$B(6,\; 4)$,}{$C(-1,\; 3)$。}}

\end{xiaoxiaotis}

\xiaoti{在制造如图所示的零件时,需要知道三个孔心间的距离。已知孔心坐标为
    $A(-10,\; 30)$,$B(30,\; 0)$,$C(-40,\; 0)$,求孔心间的距离。
}

\end{xiaotis}

