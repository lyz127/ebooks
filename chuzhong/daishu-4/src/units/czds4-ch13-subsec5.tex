\subsection{对数表}\label{subsec:13-5}

一个数的对数的尾数可以在对数尾数表(简称对数表)里查到。下面是《中学数学用表》里的表十“常用对数表” 的一部分。
其中标有 $N$ 的直列和横行的数是真数,其余的数是对数的尾数(一般是近似值),但是略去了小数点,尾数部分的最后一栏是修正值。

\begin{table}[H]
    \newcommand{\x}{\cdots}
    \begin{tblr}{
        hline{1, 9} = {1pt, solid},
        hline{2, 8} = {solid},
        vline{1, 2, 12, 21} = {1pt, solid},
        vline{3-11} = {solid},
        columns={colsep=2.5pt, c, mode=math},
    }
        N   & 0    & 1    & 2    & 3    & 4    & 5    & 6    & 7    & 8    & 9    & 1  & 2  & 3  & 4  & 5  & 6  & 7  & 8  & 9  \\
        \x  & \x   & \x   & \x   & \x   & \x   & \x   & \x   & \x   & \x   & \x   & \x & \x & \x & \x & \x & \x & \x & \x & \x \\
        5.0 & 6990 & 6998 & 7007 & 7016 & 7024 & 7033 & 7042 & 7050 & 7059 & 7067 & 1  & 2  & 3  & 3  & 4  & 5  & 6  & 7  & 8  \\
        5.1 & 7076 & 7084 & 7093 & 7101 & 7110 & 7118 & 7126 & 7135 & 7143 & 7152 & 1  & 2  & 3  & 3  & 4  & 5  & 6  & 7  & 8  \\
        5.2 & 7160 & 7168 & 7177 & 7185 & 7193 & 7202 & 7210 & 7218 & 7226 & 7235 & 1  & 2  & 2  & 3  & 4  & 5  & 6  & 7  & 7  \\
        5.3 & 7243 & 7251 & 7259 & 7267 & 7275 & 7284 & 7292 & 7300 & 7308 & 7316 & 1  & 2  & 2  & 3  & 4  & 5  & 6  & 6  & 7  \\
        \x  & \x   & \x   & \x   & \x   & \x   & \x   & \x   & \x   & \x   & \x   & \x & \x & \x & \x & \x & \x & \x & \x & \x \\
        N   & 0    & 1    & 2    & 3    & 4    & 5    & 6    & 7    & 8    & 9    & 1  & 2  & 3  & 4  & 5  & 6  & 7  & 8  & 9 \\
    \end{tblr}
\end{table}

下面我们来说明查表求对数的方法,

1. 当 $1 \leqslant N < 10$ 时,对数的首数为零,由表中查出的对数的尾数就是 $N$ 的对数。

(1) 如果 $N$ 有三个有效数字,可由表中直接查出 $N$ 的对数。
例如,查表求 $\lg{5.08}$, 可以从 $N$ 所在的直列里找出 $5.0$,从 $N$ 所在的横行里找出 $8$,
$5.0$ 所在的行与 $8$ 所在的列交叉处的数 $7059$ 表示的数是 $0.7059$,就是所求的数,即 $\lg{5.08} = 0.7059$。
又如,查表可以求得 $\lg{5.3} = 0.7243$(把 $5.3$ 看成 $5.30$ 再查表),
$\lg{5} = 0.6990$(把 $5$ 看成 $5.00$ 再查表)。

(2) 表中右边顶上一横行是真数的第四个有效数字。如果真数有四个有效数字,就要用到第四个有效数字所对应的修正值。
例如,查表求 $\lg{5.263}$。真数 $5.263$ 的第四个有效数字 $3$ 对应的修正值 $2$ 表示 $0.0002$,
因而 $\lg{5.263} = 0.7210 + 0.0002 = 0.7212$。

(3) 如果 $N$ 有五个或者更多个有效数字,可以先用四舍五入的方法把它改写成四个有效数字,再按照 (2) 中所说的方法求对数。
例如求 $\lg{5.3217}$,先把它改写成 $\lg{5.322}$, 再查表,得 $\lg{5.3217} = 0.7261$。


2. 当真数 $N < 1$ 或 $N \geqslant 10$ 时,首先用科学记数法把 $N$ 写成 $a \times 10^n$ 的形式,其中 $1 \leqslant a < 10$, $n$ 是整数。
这时,$N$ 的对数的首数是 $n$,尾数是 $\lg{a}$。 $\lg{a}$ 可以用 1. 中所说的方法求得。例如,
\begin{align*}
    \lg{0.0053217} &= \lg{(5.3217 \times 10^{-3})} = -3 + 0.7261 \\
                   &= \overline{3}.7261 \douhao \\
       \lg{5321.7} &= \lg{(5.3217 \times 10^3)} \\
                   &= 3 + 0.7261 = 3.7261 \juhao
\end{align*}

\liti[0] 查表求下列各数的对数:

\hspace*{2em} $3.65\nsep  804.7\nsep  0.26\nsep  0.00450327$。

\jie \begin{tblr}[t]{columns={$}}
    \lg{3.65} = 0.5623 \douhao \\
    \lg{804.7} = \lg{(8.047 \times 10^2)} = 2.9057 \douhao \\
    \lg{0.26} = \lg{(2.6 \times 10^{-1})} = \overline{1}.4150 \douhao \\
    \lg{0.00450327} = \lg{(4.503 \times 10^{-3})} = \overline{3}.6535 \juhao
\end{tblr}


\lianxi

查表求下列各数的对数:

$4.27$\nsep  $8.4$\nsep  $6$\nsep  $3.207$\nsep  $6.359$\nsep  $0.0037$\nsep

$384275$\nsep  $0.000129413$\nsep  $0.032$\nsep  $170$\nsep  $0.8459$\nsep  $34.06$。



