\subsection{函数的表示法}\label{subsec:14-4}

表示函数的方法,最常用的有以下三种。

\begin{enhancedline}

\zhongdian{1. 解析法} \quad 就是用等式来表示一个变量是另一个变量的函数。
这个等式叫做函数的\zhongdian{解析表达式}(或\zhongdian{函数关系式}),简称\zhongdian{解析式},
例如 $s = 60\,t$,$A = \pi r^2$,$y = \sqrt{x - 2}$,$V = \dfrac{4}{3}\pi r^3$,
$s = \dfrac{1}{1 - u^2}$,$z = \dfrac{1}{\sqrt{t^2 - 5}}$,等等。

\zhongdian{2. 列表法} \quad 就是列出表格来表示一个变量是另一个变量的函数。
如第 14.3 节例 (3) 中的水库存水量与水深的对照表,以及平方表、平方根表、对数表等数学用表,
都是用列表法来表示函数的。

\zhongdian{3. 图象法} \quad 把自变量 $x$ 的一个值和函数 $y$ 的对应值分别作为点的横坐标和纵坐标,
可以在直角坐标系内描出一个点,所有这些点的集合,叫做这个函数的\zhongdian{图象}。
图象法就是用图象来表示一个变量是另一个变量的函数。如第 14.3 节例 (4) 中的气温随时间的变化图。

知道函数的解析式,要画函数的图象,一般分为\zhongdian{列表、描点、连线}三个步骤,
即先列出自变量和函数的一些对应值,用这些对应值为坐标,描出图象上的一些点,然后用一条或几条平滑曲线(包括直线),
按照自变量由小到大的顺序,把所描的点连结起来。这种画函数图象的方法叫做\zhongdian{描点法}。
显然,用描点法所画的图象一般是近似的、部分的,要使画出的图象更精确,需要描出图象上的更多的点。

\liti[0] 画出函数 $y = \dfrac{1}{8}x^3$ 的图象。

\jie 1. 列表。 在 $x$ 的取值范围内取一些值,算出 $y$ 的对应值,列成下表:

\begin{table}[H]
    \centering
    \begin{tblr}{
        hlines, vlines,
        columns={mode=math, c},
        column{7,9,11}={1.5em},
        row{2}={rowsep=0.5em},
    }
        x                   & \cdots & -4 & -3    & -2 & -1    & 0 & 1    & 2 & 3    & 4 & \cdots \\
        y = \dfrac{1}{8}x^3 & \cdots & -8 & -3.38 & -1 & -0.13 & 0 & 0.13 & 1 & 3.38 & 8 & \cdots \\
    \end{tblr}
\end{table}

2. 描点。根据表里这些对应值,在坐标系内描点。

3. 连线。用平滑曲线,按自变量由小到大的顺序,把所描的点连结起来,就是函数 $y = \dfrac{1}{8}x^3$ 的图象(图 \ref{fig:14-12})。
\end{enhancedline}

\begin{figure}[htbp]
    \centering
    \begin{minipage}[b]{5cm}
    \centering
    \begin{tikzpicture}[>=Stealth, scale=0.4]
    \draw [->] (-4.5, 0) -- (5, 0) node[anchor=north] {$x$};
    \draw [->] (0,-9.0) -- (0, 9)  node[anchor=east]  {$y$};
    \draw (0, 0) node [below right] {$O$};
    \foreach \x in {-4, ..., 4} {
        \draw (\x, 0.2) -- (\x, 0);
    }
    \foreach \y in {-8, ..., 8} {
        \draw (0.2, \y) -- (0, \y);
    }

    \draw[domain=-4.1:4.1,samples=50] plot (\x, {(\x^3)/8});
    \foreach \x in {-4, ..., 4} {
        \draw [fill=black] (\x, {(\x^3)/8}) circle(0.1);
    }
\end{tikzpicture}


    \caption{}\label{fig:14-12}
    \end{minipage}
    \qquad
    \begin{minipage}[b]{9cm}
    \centering
    \begin{tikzpicture}[>=Stealth,
    every node/.style={fill=white, inner sep=1pt},
]
    \pgfmathsetmacro{\factorx}{0.1}
    \pgfmathsetmacro{\factory}{0.005}
    \draw [->] (0, 0) -- (5.2, 0) node[below=0.1] {水深(米)};
    \draw [->] (0, 0) -- (0, 3.8) node[left=0.2em] {库容(万($\lfm$))};
    \draw (0, 0) coordinate(O) node [below left] {\small $0$};
    \foreach \x in {5, 10, ..., 40} {
        \draw (\factorx * \x, 0) -- (\factorx * \x, -0.15) node [below] {\small $\x$};
    }
    \foreach \y in {100, 200, ..., 600} {
        \draw (0, \factory * \y) -- (-0.15, \factory * \y) node[left] {\small $\y$};
    }
    \foreach \y in {50, 150, ..., 550} {
        \draw (0, \factory * \y) -- (-0.15, \factory * \y);
    }

    \draw plot [smooth] coordinates{
        %            x               y
        (            0,              0)
        (\factorx *  5, \factory *  25)
        (\factorx * 10, \factory *  50)
        (\factorx * 15, \factory * 100)
        (\factorx * 20, \factory * 175)
        (\factorx * 25, \factory * 275)
        (\factorx * 30, \factory * 450)
        (\factorx * 32, \factory * 550)
    };
\end{tikzpicture}


    \caption*{(第 2 题)}
    \end{minipage}
\end{figure}


\lianxi
\begin{xiaotis}

\xiaoti{用解析式表示下列函数:}
\begin{xiaoxiaotis}

    \xxt{如果每升高 $1$ 千米,气温就下降 $6$ ℃,求气温降低数 $T$(℃) 与高度增加数 $h$(千米) 之间的函数关系式;}

    \xxt{某工厂现有煤 $1500$ 吨,求这些煤能用的天数 $y$ 与这家工厂每天平均用煤的吨数 $x$ 之间的函数关系式。}

\end{xiaoxiaotis}

\xiaoti{根据某水库的水深 - 库容曲线图,填写下表:\\[1em]
    \begin{tblr}[]{
        hlines, vlines,
        column{1}={c}, column{2-6}={4em, c}
    }
        水深(米)          & 5 & 10 & 15 & 20 & 25 \\
        库容(万($\lfm$))  &   &    &    &    &
    \end{tblr} \jiange
}

\xiaoti{画出下列函数的图象:}
\begin{xiaoxiaotis}

    \begin{tblr}{columns={18em, colsep=0pt}}
        \xxt{$y = x$;} & \xxt{$y = x + 1$。}
    \end{tblr}
\end{xiaoxiaotis}

\end{xiaotis}

