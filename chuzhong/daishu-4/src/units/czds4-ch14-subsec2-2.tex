\subsubsection{平面内任意两点间的距离}

\begin{wrapfigure}[8]{r}{6cm}
    \centering
    \begin{tikzpicture}[>=Stealth, scale=0.6]
    \draw [->] (-4, 0) -- (4, 0) node[anchor=north] {$x$};
    \draw [->] (0, -3) -- (0, 4) node[anchor=east] {$y$};
    \draw (0, 0) node [anchor=north east] {$O$};

    \pgfmathsetmacro{\xa}{-3}
    \pgfmathsetmacro{\ya}{-1.5}
    \pgfmathsetmacro{\xb}{1.5}
    \pgfmathsetmacro{\yb}{3}

    \draw (\xa, 0)   coordinate (ma) node [above] {$M_1$};
    \draw (\xa, \ya) coordinate (pa) node [left] {$P_1$};
    \draw (0,   \ya) coordinate (na) node [below left] {$N_1$};
    \draw (\xb, \ya) coordinate (q)  node [right] {$Q$};
    \draw (\xb, 0)   coordinate (mb) node [above right] {$M_2$};
    \draw (\xb, \yb) coordinate (pb) node [right] {$P_2$};
    \draw (0,   \yb) coordinate (nb) node [left] {$N_2$};

    \draw (pa) -- (pb);
    \draw [dashed] (ma) -- (pa) -- (na) -- (q) -- (mb) -- (pb) -- (nb);
\end{tikzpicture}


    \caption{}\label{fig:14-8}
\end{wrapfigure}

设 $P_1(x_1,\; y_1)$,$P_2(x_2,\; y_2)$ 是坐标平面内的任意两点(图 \ref{fig:14-8}),
  从点 $P_1$,$P_2$ 分别画 $x$ 轴的垂线 $P_1M_1$,$P_2M_2$,与 $x$ 轴交于点 $M_1(x_1,\; 0)$,$M_2(x_2,\; 0)$。
再从点 $P_1$,$P_2$ 分别画 $y$ 轴的垂线 $P_1N_1$,$P_2N_2$,与 $y$ 轴交于点 $N_1(0,\; y_1)$,$N_2(0,\; y_2)$。
直线 $P_1N_1$ 与 $P_2M_2$ 相交于点 $Q$。

因为 $\triangle P_1QP_2$ 是直角三角形,根据勾股定理,得

\begin{tblr}{columns={mode=math}, column{1}={10em}}
                & P_1P_2^2 = P_1Q^2 +QP_2^2 \juhao \\
    \because    & P_1Q = M_1M_2 = |x_2 - x_1| \douhao \\
                & QP_2 = N_1N_2 = |y_2 - y_1| \douhao \\
    \therefore  & \begin{aligned}[t]
                        P_1P_2^2 &= |x_2 - x_1|^2 + |y_2 - y_1|^2 \\
                                 &= (x_2 - x_1)^2 + (y_2 - y_1)^2 \juhao
                   \end{aligned}
\end{tblr}

由此得到 $P_1(x_1,\; y_1)$,$P_2(x_2,\; y_2)$ 两点间的距离公式:
\begin{center}
    \framebox{ \quad $\bm{P_1P_2 = \sqrt{(x_2 - x_1)^2 + (y_2 - y_1)^2}$}。\quad }
\end{center}


\liti 求两点 $P_1(-3,\; 5)$,$P_2(1,\; 2)$ 间的距离。

\jie $x_1 = -3$,$y_1 = 5$; $x_2 = 1$,$y_2 = 2$。

代入两点间的距离公式,得

\qquad $\begin{aligned}[t]
    P_1P_2 & = \sqrt{[1 - (-3)]^2 + (2 - 5)^2} \\
           &= \sqrt{4^2 + (-3)^2} \\
           &= 5 \juhao
\end{aligned}$


\liti 在图 \ref{fig:14-9} 给出的零件图上(如果没有特别注明,本书中零件图上的尺寸单位都是毫米),
选择如图 \ref{fig:14-10} 所示的坐标系,分别求孔心 $A$,$B$ 及 $B$,$C$ 间的距离(精确到 $0.01$ 毫米)。


\begin{figure}[htbp]
    \centering
    \begin{minipage}[b]{7cm}
    \centering
    \begin{tikzpicture}[>=Stealth, scale=0.8,
    every node/.style={fill=white, inner sep=1pt},
]
    \pgfmathsetmacro{\rxa}{-1}
    \pgfmathsetmacro{\rya}{-2.7}
    \pgfmathsetmacro{\rxb}{6}
    \pgfmathsetmacro{\ryb}{3.8}
    \draw [thick] (\rxa, \rya) rectangle (\rxb, \ryb);

    \pgfmathsetmacro{\xa}{0}
    \pgfmathsetmacro{\ya}{0}
    \pgfmathsetmacro{\xb}{2.8}
    \pgfmathsetmacro{\yb}{2.6}
    \pgfmathsetmacro{\xc}{4.8}
    \pgfmathsetmacro{\yc}{-1.5}
    \pgfmathsetmacro{\r}{0.7}

    \draw [thick] (\xa, \ya) node [below right] {$A$} circle (\r);
    \draw [thick] (\xb, \yb) node [below right] {$B$} circle (\r);
    \draw [thick] (\xc, \yc) node [below right] {$C$} circle (\r);

    \draw [densely dash dot] (-1.8, \ya) -- (\xa + 1, \ya);
    \draw [densely dash dot] (-1.8, \yb) -- (\xb + 1, \yb);
    \draw [densely dash dot] (-1.8, \yc) -- (\xc + 1, \yc);

    \draw [densely dash dot] (\xa, -3.5) -- (\xa,  \ya + 1);
    \draw [densely dash dot] (\xb, -3.5) -- (\xb,  \yb + 1);
    \draw [densely dash dot] (\xc, -3.5) -- (\xc,  \yc + 1);

    \draw [<->] (-1.4, \yc) to [xianduan] node [rotate=90] {$15$} (-1.4, \ya);
    \draw [<->] (-1.4, \ya) to [xianduan] node [rotate=90] {$26$} (-1.4, \yb);

    \draw [<->] (\xa, -3) to [xianduan] node {$28$} (\xb, -3);
    \draw [<->] (\xb, -3) to [xianduan] node {$20$} (\xc, -3);
\end{tikzpicture}


    \caption{}\label{fig:14-9}
    \end{minipage}
    \qquad
    \begin{minipage}[b]{7.5cm}
    \centering
    \begin{tikzpicture}[>=Stealth, scale=0.8,
    every node/.style={fill=white, inner sep=1pt},
]
    \pgfmathsetmacro{\rxa}{-1}
    \pgfmathsetmacro{\rya}{-2.7}
    \pgfmathsetmacro{\rxb}{6}
    \pgfmathsetmacro{\ryb}{3.8}
    \draw [thick] (\rxa, \rya) rectangle (\rxb, \ryb);

    \pgfmathsetmacro{\xa}{0}
    \pgfmathsetmacro{\ya}{0}
    \pgfmathsetmacro{\xb}{2.8}
    \pgfmathsetmacro{\yb}{2.6}
    \pgfmathsetmacro{\xc}{4.8}
    \pgfmathsetmacro{\yc}{-1.5}
    \pgfmathsetmacro{\r}{0.7}

    \draw [thick] (\xa, \ya) node [below right] {$A$} circle (\r);
    \draw [thick] (\xb, \yb) node [below right] {$B$} circle (\r);
    \draw [thick] (\xc, \yc) node [below right] {$C$} circle (\r);

    \filldraw [fill=black] (\xb, \yb) circle (0.05);
    \filldraw [fill=black] (\xc, \yc) circle (0.05);

    \draw [->] (-2, 0) -- (7, 0) node[below=0.2em] {$x$};
    \draw [->] (0, -3.5) -- (0, 4.5) node[left=0.2em] {$y$};
\end{tikzpicture}


    \caption{}\label{fig:14-10}
    \end{minipage}
\end{figure}

\jie 孔心的坐标是
$$ A(0,\; 0) \nsep B(28,\; 26) \nsep C(48,\; -15) \juhao $$
将点的坐标代入两点间的距离公式,得
\begin{align*}
    & AB = \sqrt{(28 - 0)^2 + (26 - 0)^2} = \sqrt{1460} \approx 38.21 \douhao \\
    & BC = \sqrt{(48 - 28)^2 + (-15 - 26)^2} = \sqrt{2081} \approx 45.62 \douhao
\end{align*}
即孔心 $A$,$B$ 间的距离约是 $38.21$ 毫米, $B$,$C$ 间的距离约是 $45.62$ 毫米。


\lianxi
\begin{xiaotis}

\jiange
\begin{minipage}{9cm}

\xiaoti{求下列两点间的距离:}
\begin{xiaoxiaotis}

    \xxt{$P_1(-1,\; 0)$,$P_2(2,\; 0)$;}

    \xxt{$P_1(0,\; 6)$,$P_2(0,\; -2)$;}

    \xxt{$A(-2,\; 0)$,$B(-4,\; 3)$;}

    \xxt{$A(2,\; -5)$,$C(2,\; 3)$;}

    \xxt{$M(-3,\; 8)$,$N(-1,\; -2)$;}

    \xxt{$O(0,\; 0)$,$P(2,\; -3)$;}

\end{xiaoxiaotis}


\xiaoti{如图,已知零件图上孔心的坐标为 $A(-20,\; 50)$,$B(40,\; 0)$,$C(-40,\; 0)$,
    求每两孔中心间的距离(精确到 $0.01$ 毫米。)
}
\end{minipage}
\begin{minipage}{6cm}
    \begin{figure}[H]
        \centering
        \begin{tikzpicture}[>=Stealth, scale=0.6,
    every node/.style={inner sep=1pt},
]
    \draw [->] (-4, 0) -- (4.5, 0) node[below=0.2em] {$x$};
    \draw [->] (0, -2) -- (0, 4.5) node[left=0.2em] {$y$};
    \draw (0, 0) node [above left] {\small $O$};

    \coordinate (A) at (-1, 2.5);
    \coordinate (B) at (2, 0);
    \coordinate (C) at (-2, 0);
    \pgfmathsetmacro{\r}{0.7}
    \pgfmathsetmacro{\R}{1.5}

    \filldraw [fill=black] (A) circle (0.05);
    \filldraw [fill=black] (B) circle (0.05);
    \filldraw [fill=black] (C) circle (0.05);

    \draw [thick] (A) node [below right] {\small $A$} circle (\r);
    \draw [thick] (B) node [below right] {\small $B$} circle (\r);
    \draw [thick] (C) node [below right] {\small $C$} circle (\r);

    \path (A) +(30:\R)  coordinate(pa);
    \path (C) +(150:\R) coordinate(pc);
    \path (B) +(270:\R) coordinate(pb);

    \draw [rounded corners] (pa)
        arc [radius=\R, start angle=30,  end angle=150]
        to [bend left] (pc)
        arc [radius=\R, start angle=150, end angle=270]
        to [bend left] (pb)
        arc [radius=\R, start angle=270, end angle=390]
        to [bend left] (pa);
\end{tikzpicture}


        \caption*{(第 2 题)}
    \end{figure}
\end{minipage}

\jiange
\xiaoti{甲船在某港口东 $50$ 海里、北 $30$ 海里处,乙船在同一港口东 $17$ 海里、南 $26$ 海里处。
    选择坐标系求甲、乙两船间的距离。
}

\end{xiaotis}

