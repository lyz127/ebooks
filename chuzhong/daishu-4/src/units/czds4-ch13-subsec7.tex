\subsection{利用对数进行计算}\label{subsec:13-7}

利用积、商、幂、方根的对数的运算性质,可以比较简捷地进行乘、除、乘方、开方的计算。

在利用对数进行计算的时候,因为我们所用的表中只有四个有效数字,所以计算得到的结果最多只能保留四个有效数字。

\liti 计算 $17.95 \times 0.08304$。

\jie 设 $x = 17.95 \times 0.08304$,则
$$\lg{x} = \lg{17.95} + \lg{0.08304} \juhao $$
\begin{center}
\vspace*{-1em}
\begin{tblr}[t]{
    columns={r, mode=math, colsep=0em},
    rows={rowsep=0em},
    hline{3}={1}{solid},
    % hline{3}={2}{text=$(+$},
}
    \lg{17.95} = 1.2541 \\
    \lg{0.08304} = \overline{2}.9193 \tikz [overlay] {\draw (1em, -0.5em) node {$(+$};} \\
    \lg{x} = 0.1734
\end{tblr}
\end{center}

$\therefore$
\vspace*{-1.5em} $$ x = 1.490 \juhao $$

这里正尾数和正尾数相加得正尾数,并且进位得正整数 $1$, 所以数是 $1 + (-2) + 1 = 0$。
在利用对数进行计算时,要特别注意关于负首数的计算问题。

\begin{enhancedline}
\liti 计算 $\dfrac{0.1674}{4785}$。

\jie 设 $x = \dfrac{0.1674}{4785}$,则
$$\lg{x} = \lg{0.1674} - \lg{4785} \juhao $$
\end{enhancedline}
\begin{center}
\vspace*{-0.5em}
\begin{tblr}[t]{
    columns={r, mode=math, colsep=0em},
    rows={rowsep=0em},
    hline{3}={1}{solid},
    % hline{3}={2}{text=$(+$},
}
    \lg{0.1674} = \overline{1}.2238 \\
    \lg{4785} = 3.6799 \tikz [overlay] {\draw (1em, -0.5em) node {$(-$};} \\
    \lg{x} = \overline{5}.5439
\end{tblr}
\end{center}

$\therefore$
\vspace*{-1.5em} $$ x = 0.00003498 \juhao $$

这里尾数 $0.2238$ 借 $1$ 减去 $0.6799$ 得正尾数 $0.5439$,首数是 $-1 - 1 - 3 = -5$。


\liti 计算 $0.5084^4$。

\jie 设 $x = 0.5084^4$,则 $\lg{x} = 4\lg{0.5084}$。
$$\lg{0.5084} = \overline{1}.7062 $$
\begin{center}
\vspace*{-0.5em}
\begin{tblr}[t]{
    columns={r, mode=math, colsep=0em},
    rows={rowsep=0em},
    hline{3}={solid},
}
           & \overline{1}.7062 \\
    \times & 4 \\
           & \overline{2}.8248
\end{tblr}
\end{center}
$$ \lg{x} = \overline{2}.8248 \juhao $$

$\therefore$
\vspace*{-1.5em} $$ x = 0.06680 \juhao $$

这里正尾数 $0.7062$ 乘以 $4$ 得正尾数 $0.8248$ 和正首数 $2$,首数是 $(-1) \times 4 + 2 = -2$。


\liti 计算 $\sqrt[5]{0.02781}$。

\begin{enhancedline}
\jie 设 $x = \sqrt[5]{0.02781}$,则 $\lg{x} = \dfrac{1}{5}\lg{0.02781}$。
$$\lg{0.02781} = \overline{2}.4442 = -5 + 3.4442 \juhao $$
\end{enhancedline}
\begin{center}
\vspace*{-1em}
\begin{tblr}[t]{
    columns={r, mode=math, colsep=0em},
    column{1}={rightsep=5pt},
    column{2}={leftsep=5pt},
    rows={rowsep=0em},
    hline{2}={2}{solid},
    vline{2}={1}{solid}
}
    5 & -5 + 3.4442 \\
      & -1 + 0.6888
\end{tblr}
\end{center}
$$ \lg{x} = \overline{1}.6888 \juhao $$

$\therefore$
\vspace*{-1.5em} $$ x = 0.4884 \juhao $$

这里 $\overline{2}.4442$ 的首数 $-2$ 不能被 $5$ 整除,我们把 $-2$ 化成 $-5 + 3$,
其中 $-5$ 是一个能被 $5$ 整除的负数,$3$ 是小于5 的正数。
$-5$ 除以 $5$ 得首数 $-1$, $3$ 与原来的正尾数 $0.4442$ 合并后除以 $5$ 得正尾数 $0.6888$。


\begin{enhancedline}
\liti 计算 $\dfrac{5.16^3}{2.78 \times \sqrt{0.637}}$。

\jie 设 $x = \dfrac{5.16^3}{2.78 \times \sqrt{0.637}}$,则
\end{enhancedline}

\begin{gather*}
    \lg{x} = 3\lg{5.16} - \left(\lg{2.78} + \dfrac{1}{2}\lg{0.637}\right) \juhao \\
    \lg{2.78} = 0.4440 \douhao \\
    \dfrac{1}{2}\lg{0.637} = \dfrac{1}{2} \times \overline{1}.8041 = \overline{1}.9021 \douhao \\
    3\lg{5.16} = 3 \times 0.7126 = 2.1378 \juhao
\end{gather*}

\begin{center}
    \vspace*{-1em}
    \begin{tblr}[]{column{1}={rightsep=5em}}
        \begin{tblr}[t]{
            columns={r, mode=math, colsep=0em},
            rows={rowsep=0em},
            hline{3}={1}{solid},
        }
            0.4440 \\
            \overline{1}.9021 \tikz [overlay] {\draw (1em, -0.5em) node {$(+$};} \\
            0.3461
        \end{tblr} &
        \begin{tblr}[t]{
            columns={r, mode=math, colsep=0em},
            rows={rowsep=0em},
            hline{3}={1}{solid},
        }
            2.1378 \\
            0.3461 \tikz [overlay] {\draw (1em, -0.5em) node {$(-$};} \\
            1.7917
        \end{tblr}
    \end{tblr}
\end{center}
$$ \lg{x} = 1.7917 \juhao $$

$\therefore$
\vspace*{-1.5em} $$ x = 61.90 \juhao $$


\liti 计算 $2.31^3 \times \sqrt[5]{72} + 1.2^2$。

分析:这是含有乘法与加法的混合计算题,可以根据对数的运算性质,
先计算 $2.31^3 \times \sqrt[5]{72}$,再将结果与 $1.2^2$ 相加。

\jie 设 $x = 2.31^3 \times \sqrt[5]{72}$,则
$$\lg{x} = 3\lg{2.31} + \dfrac{1}{5}\lg{72} \juhao $$
\begin{center}
\vspace*{-1em}
\begin{tblr}[t]{
    columns={r, mode=math, colsep=0em},
    % rows={rowsep=0em},
    row{1}={abovesep=0.5em},
    row{2}={belowsep=0.5em},
    hline{3}={1}{solid},
}
    3\lg{2.31} = 3 \times 0.3636 = 1.0908 \\
    \dfrac{1}{5}\lg{72} = \dfrac{1}{5} \times 1.8573 = 0.3715 \tikz [overlay] {\draw (1em, -1.2em) node {$(+$};} \\
    \lg{x} = 1.4623
\end{tblr}
\end{center}

$\therefore$ \quad $x = 28.99$。

又 $1.2^2 = 1.44$,

$\therefore$ \quad $2.31^3 \times \sqrt[5]{72} + 1.2^2 = 28.99 + 1.44 = 30.43$。


\liti 某农场的粮食产量经过五年增长了 $65\%$。求每年比上一年平均增长的百分数。

分析:设原来的产量是 $1$, 每年比上一年平均增长的百分数是 $x$,那么
经过一年的产量是 $1 + x$,
经过二年的产量是 $(1 + x) + (1 + x) \cdot x = (1 + x)^2$,
经过三年的产量是 $(1 + x)^2 + (1 + x)^2 \cdot x = (1 + x)^3$,……
经过五年的产量是 $(1 + x)^5$。

\jie 设原来的产量是 $1$, 每年比上一年平均增长的百分数是 $x$,则
\begin{gather*}
    (1 + x)^5 = 1 + 65\% \juhao \\
    1 + x = \sqrt[5]{1.65} \douhao \\
    \lg{(1 + x)} = \dfrac{1}{5}\lg{1.65} = \dfrac{1}{5} \times 0.2175 = 0.0435 \douhao
\end{gather*}

$\therefore$
\vspace*{-1.5em} $$ 1 + x = 1.105 \juhao $$

$\therefore$
\vspace*{-1.5em} $$ x = 0.105 = 10.5\% \juhao $$

答:每年比上一年平均增长 $10.5\%$。


\lianxi
\begin{xiaotis}

\xiaoti{计算:}
\begin{xiaoxiaotis}

    \begin{tblr}{columns={12em, colsep=0pt}}
        \xxt{$\overline{1}.5483 + \overline{2}.8712$;} & \xxt{$\overline{2}.7125 - \overline{1}.9418$;} & \xxt{$\overline{4}.5082 \times 3$;} \\
        \xxt{$\overline{3}.6479 \times 5$;} & \xxt{$\overline{2}.2418 \div 2$;} & \xxt{$\overline{1}.1535 \div 3$。}
    \end{tblr}
\end{xiaoxiaotis}


\xiaoti{利用对数查表计算:}
\begin{xiaoxiaotis}

    \begin{tblr}{columns={12em, colsep=0pt}, rows={rowsep=0.5em}}
        \xxt{$15.4 \times 2.47$;} & \xxt{$0.064 \times 1.87$;} & \xxt{$4.254 \div 0.1658$;} \\
        \xxt{$7.58^3$;} & \xxt{$\sqrt[3]{13}$;} & \xxt{$\dfrac{65.36}{2.72 \times 435^2}$;} \\
        \xxt{$\dfrac{\sqrt{62.3} \times 47.6^2}{86.08}$;} &  \SetCell[c=2]{l}\xxt{$(0.235)^7 \div \sqrt[5]{0.2771}$。}
    \end{tblr}
\end{xiaoxiaotis}


\xiaoti{长方体的体积是 $45.8\;\text{cm}^3$,长是 $3.5$ cm, 宽是 $2.4$ cm,求它的高。}

\xiaoti{已知圆的面积等于 $0.6567\;\text{m}^2$,求它的周长。}

\xiaoti{某种产品的产量平均每年比上一年增长 $12.5\%$,求经过 $6$ 年产量比原来增长的百分数。}

\end{xiaotis}

