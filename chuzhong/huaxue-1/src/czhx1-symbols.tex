%----------------------------------
% (重)定义一些数学符号
%  一是为了使用/记忆上的方便,二是如果以后有变动,只需要修改一处。
\newcommand{\celsius}{\ensuremath{^\circ\hspace{-0.09em}\mathrm{C}}} % 摄氏度
\newcommand*\xiangsi{%                 % 相似 (如果不在意有些不同,可以使用 \backsim )
    \mathrel{\text{%
            \tikz \draw[baseline] (-.25em,1.15ex) .. controls (-.55em,1.15ex) and (-.51em,.23ex) .. (-.275em,.23ex) .. controls (0,.23ex) and (0,1.15ex) .. (.275em,1.15ex) .. controls (.51em,1.15ex) and (.55em,.23ex) .. (.25em,.23ex);%
}}}


%特殊符号
\newcommand{\nsep}{,\quad}% 连续的数字之间的分隔符
\newcommand{\douhao}{\mathord{\text{,}}}%(中文的)逗号
\newcommand{\juhao}{\mathord{\text{。}}}%(中文的)句号
\newcommand{\fenhao}{\mathord{\text{;}}}%(中文的)分号
\newcommand{\dao}{\mathord{\text{~}}}%(中文的)波浪号

% “正” 字的五步
\newcommand{\za}{\hbox{\lower-0.8ex\hbox{\scalebox{0.9}[1]{一}}}}
\newcommand{\zb}{丅}
\newcommand{\zc}{\hbox{丅{\kern-.5em\lower0.1ex\hbox{\scalebox{0.4}[1]{一}}}}}
\newcommand{\zd}{\hbox{丅{\kern-.5em\lower0.1ex\hbox{\scalebox{0.4}[1]{一}}}{\kern-1.2em\lower0.1ex\hbox{\scalebox{1}[0.6]{丨}}}}}
\newcommand{\ze}{正}


%方程
\newcommand{\zuobian}{\text{左边}}
\newcommand{\youbian}{\text{右边}}

%------------------ 物理单位
% 长度单位
\newcommand{\qianmi}{\mathord{\text{千米}}}%千米
\newcommand{\mi}{\mathord{\text{米}}}%米
\newcommand{\fenmi}{\mathord{\text{分米}}}%分米
\newcommand{\limi}{\mathord{\text{厘米}}}%厘米
\newcommand{\haomi}{\mathord{\text{毫米}}}%毫米
\newcommand{\weimi}{\mathord{\text{微米}}}%微米

% 面积单位
\newcommand{\pfqm}{\mathord{\text{千米}^2}}%平方千米
\newcommand{\pfm}{\mathord{\text{米}^2}}%平方米
\newcommand{\pflm}{\mathord{\text{厘米}^2}}%平方厘米
\newcommand{\pfhm}{\mathord{\text{毫米}^2}}%平方毫米

% 体积单位
\newcommand{\lfm}{\mathord{\text{米}^3}}%立方米
\newcommand{\lffm}{\mathord{\text{分米}^3}}%立方分米
\newcommand{\lflm}{\mathord{\text{厘米}^3}}%立方厘米
\newcommand{\lfhm}{\mathord{\text{毫米}^3}}%立方毫米
\newcommand{\haosheng}{\mathord{\text{毫升}}}%毫升

% 质量单位
\newcommand{\haoke}{\mathord{\text{毫克}}}%毫克
\newcommand{\ke}{\mathord{\text{克}}}%克
\newcommand{\qianke}{\mathord{\text{千克}}}%千克

\newcommand{\qkmlfm}{\mathord{\text{千克}/\text{米}^3}}%千克每立方米
\newcommand{\kmlflm}{\mathord{\text{克}/\text{厘米}^3}}%克每立方厘米
\newcommand{\kms}{\mathord{\text{克}/\text{升}}}%克每升

% 力的单位
\newcommand{\niudun}{\mathord{\text{牛顿}}}%牛顿
\newcommand{\ndmqk}{\mathord{\text{牛顿/千克}}}%牛顿每千克
\newcommand{\ndmpfm}{\mathord{\text{牛顿}/\text{米}^2}}%牛顿每平方米
\newcommand{\pasika}{\mathord{\text{帕斯卡}}}%帕斯卡

% 功的单位
\newcommand{\niudunmi}{\mathord{\text{牛顿·米}}}%牛顿·米
\newcommand{\jiaoer}{\mathord{\text{焦耳}}}%焦耳
\newcommand{\jemm}{\mathord{\text{焦耳/秒}}}%焦耳每秒
\newcommand{\wate}{\mathord{\text{瓦特}}}%瓦特
\newcommand{\qianwa}{\mathord{\text{千瓦}}}%千瓦
\newcommand{\mali}{\mathord{\text{马力}}}%马力

% 时间单位
\newcommand{\miao}{\mathord{\text{秒}}}%秒
\newcommand{\xiaoshi}{\mathord{\text{小时}}}%小时

% 速度单位
\newcommand{\qmmm}{\mathord{\text{千米/秒}}}%千米/秒
\newcommand{\mmm}{\mathord{\text{米/秒}}}%米/秒
\newcommand{\lmmm}{\mathord{\text{厘米/秒}}}%厘米/秒
\newcommand{\mmf}{\mathord{\text{米/分}}}%米/分

% 热量单位
\newcommand{\qianka}{\mathord{\text{千卡}}}%千卡
\newcommand{\ka}{\mathord{\text{卡}}}%卡
\newcommand{\kmkssd}{\mathord{\text{卡/(克·℃)}}}%卡每克摄氏度
\newcommand{\qkmqkssd}{\mathord{\text{千卡/(千克·℃)}}}%千卡每千克摄氏度

% 电的单位
\newcommand{\kulun}{\mathord{\text{库仑}}}%库仑
\newcommand{\anpei}{\mathord{\text{安培}}}%安培
\newcommand{\fute}{\mathord{\text{伏特}}}%伏特
\newcommand{\oumu}{\mathord{\text{欧姆}}}%欧姆
\newcommand{\qianwashi}{\mathord{\text{千瓦时}}}%千瓦时
\newcommand{\du}{\mathord{\text{度}}}%度
