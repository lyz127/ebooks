\section{根据化学方程式的计算}\label{sec:2-8}

化学方程式表示出了化学反应前后反应物和生成物的质和量的关系,因此,
在工农业生产和科学实验里,我们可以根据化学方程式来进行一系列的计算。
如计算用一定数量的原料(反应物),可以生产多少产品(生成物),
或者要制备一定数量的产品,需用多少原料。
这样就便于计划生产,厉行节约,避免浪费。

\liti 实验室里用加热分解氯酸钾的方法制取氧气。现完全分解 $5.8$ 克氯酸钾,能制得多少克氧气?

\jie (1) 写出这个反应的化学方程式
\begin{fangchengshi}
    \ce{ 2KClO3 $\xlongequal[\Delta]{\ce{MnO2}}$ 2KCl + 3O2 ^ }
\end{fangchengshi}

(2) 求出已知物质和待求物质之间的质量比
\begin{fangchengshi}
    \ce{ 2KClO3 $\xlongequal[\Delta]{\ce{MnO2}}$ 2KCl + 3O2 ^ }
\end{fangchengshi}

\hspace*{10em}\begin{tblr}{columns={mode=math, c}, column{2}={2em}}
    2 (39 + 35.5 + 16 \times 3) &  & 3(16 \times 2) \\
    =245                        &  & = 96
\end{tblr}

(3) 列比例式,求出未知数

已知 $245$ 份质量的 \ce{KClO3} 可以制得 $96$ 份质量的 \ce{O2},
设 $5.8$ 克 \ce{KClO3} 可以制得 $x$ 克 \ce{O2}, 因此,可以列比例式
\begin{align*}
    245:96 &= 5.8:x \\
        x  &= \dfrac{96 \times 5.8}{245} \\
           &= 2.3 \; (\ke)
\end{align*}

(4) 简明地写出答案

答:加热分解 $5.8$ 克氯酸钾可以制得 $2.3$ 克氧气。


根据化学方程式的计算,可以按照下面例题所列格式书写。

\liti 某冶金工厂用氢气还原三氧化钨制取钨。现在要制取 $50$ 千克钨.需要多少千克的三氧化钨?

\jie 设需用三氧化钨 $x$ 千克

\begin{tblr}{
    columns={mode=math, c},
}
    \SetCell[c=2]{c} \hspace*{3em} \ce{ WO3 + 3H2 $\xlongequal{\Delta}$ W + 3H2O } \\
    184 + 16 \times 3  & 184 \\
    =232               & \\
    x \qianke          & 50 \qianke \\
    \SetCell[c=2]{c} 232:184 = x:50 \\
    \SetCell[c=2]{c} \begin{aligned}
        x &= \dfrac{232 \times 50}{184} \\
          &= 63 \; (\qianke)
    \end{aligned}
\end{tblr}

答:需用 $63$ 千克氧化钨。


\begin{xiti}

\xiaoti{实验室用 37.7 克锌跟足量的盐酸反应,可制得氢气多少克,氯化锌多少克?}

\xiaoti{有人用氢气还原氧化铜制得 5 克铜,求有多少克氢气参加了反应,这些氢气在标准状况下占多大体积?\\
    (氢气的密度是 $0.09 \; \kms$)
}

\xiaoti{制取五氧化二磷 100 克,需燃烧多少克磷?同时还要消耗多少克氧气?这些氧气在标准状况下占多大体积?}

\xiaoti{在两个试管里各加入 3 克锌,然后在一个试管里加入足量的稀硫酸,在另一个试管里入足量的盐酸,比较反应后放出氢气的质量。}

\end{xiti}

