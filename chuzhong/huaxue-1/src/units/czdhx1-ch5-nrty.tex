\xiaojie

一、电解质和非电解质

凡是在水溶液里或熔化的状态下能够导电的化合物叫做电解质,
在上述情况下都不能导电的化合物叫做非电解质。
酸、碱、盐都是电解质,蔗糖、酒精等是非电解质。


二、本章所学的化合物有下列各种类型:

1. 酸 \quad 电离时所生成的阳离子全部是氢离子的化合物叫做酸。
\begin{fangchengshi}
    \ce{ \text{酸} = H+ + \text{酸根离子} }
\end{fangchengshi}

桉含氧与否,酸可分为无氧酸和含氧酸。

按电离时能生成的氢离子数目,酸可分为一元酸、二元酸和三元酸等。

2. 碱 \quad 电离时所生成的阴离子全部是氢氧根离子的化合物叫做碱。
\begin{fangchengshi}
    \ce{ \text{碱} = \text{金属离子} + OH- }
\end{fangchengshi}

3. 盐 \quad 由金属离子和酸根离子组成的化合物叫做盐。
\begin{fangchengshi}
    \ce{ \text{盐} = \text{金属离子} + \text{酸根离子} }
\end{fangchengshi}

按组成不同,盐可分为正盐、酸式盐和碱式盐。

4. 氧化物 \quad 按能跟酸、碱起反应的性质,氧化物可分为酸性氧化物、碱性氧化物和两性氧化物。


三、由两种化合物互相交换成分,生成另外两种化合物的反应,叫做复分解反应。

本章和以前各章已学过的化学反应主要有下列各种类型:

\begin{tblr}{rowsep=0pt}
    1. 化合反应 & \ce{A + B = AB} \\
    2. 分解反应 & \ce{AB = A + B} \\
    3. 置换反应 & \ce{A + BC = AC + B} \\
    4. 复分解反应 & \ce{AB + CD = AD + CB}
\end{tblr}


四、金属活动性顺序
\begin{center}
    \begin{tblr}{columns={c}}
        \ce{K} & \ce{Ca} & \ce{Na} & \ce{Mg} & \ce{Al} & \ce{Zn} & \ce{Fe}  & \ce{Sn} & \ce{Pb} & \ce{(H)} & \ce{Cu} &  \ce{Hg} & \ce{Ag} & \ce{Pt} & \ce{Au} \\
       \SetCell[c=15]{c} \tikz[overlay, >=Stealth] \draw [->] (-4, 0.5) -- (8.5, 0.5); 金属活动性由强逐渐减弱
    \end{tblr}
\end{center}

在金属活动性顺序里,金属的位置越靠前,在水溶液中就越容易失去电子变成离子,它的活动性就越强。

在金属活动性顺序里,排在氢前面的金属能置换出酸里的氢,只有排在前面的金属才能够把排在后面的金属从它们的盐溶液里置换出来。


五、复分解反应能否发生,要考虑是否有沉淀、气体和水生成。


六、酸碱度

溶液的酸碱性可以用酸碱指示剂来鉴别。

溶液的酸碱度可以用 pH 值来表示,
$\text{pH 值} = 7$ 为中性,
$\text{pH 值} < 7$ 为酸性,
$\text{pH 值} > 7$ 为碱性。


七、常用化学肥料分以下几类:

1. 氮肥, 如氨水、硫铵、硝铵、尿素等等。

2. 磷肥, 如过磷酸钙、重过磷酸钙等等。

3. 钾肥, 如氯化钾、硫酸钾等等。

4. 微量元素肥料(含硼、锌、锰、铜等元素的肥料)。

5. 复合肥料(含两种或两种以上营养元素的肥料)。


八、酸、碱、盐、氧化物的性质和相互关系:

主要内容参阅图 \ref{fig:5-5}。

