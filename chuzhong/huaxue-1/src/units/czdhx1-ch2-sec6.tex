\section{化合价}\label{sec:2-6}

我们已经知道,在钠跟氯起反应生成氯化钠、氢跟氯起反应生成氯化氢这两个反应里,反应物原子个数比都是 $1:1$;
同样不难理解,在镁跟氯反应生成氯化镁、氧跟氢反应生成水这两个反应里,反应物原子个数比都是 $1:2$。
如果不是这个数目比,就不能使离子化合物中的阴阳离子和共价化合物分子中原子的最外电子层成为稳定结构,
也就不能形成稳定的化合物。因此元素之间相互化合时,其原子个数比都有确定的数值。我们把%
\zhongdian{一种元素一定数目的原子跟其它元素一定数目的原子化合的性质,叫做这种元素的化合价。}
化合价有正价和负价。

\zhongdian{在离子化合物里,元素化合价的数值,就是这种元素的一个原子得失电子的数目。}
失去电子的原子带正电,这种元素的化合价是正价;
得到电子的原子带负电,这种元素的化合价是负价。
例如,
在氯化钠这个化合物里,一个钠原子失去一个电子,即钠为 $+1$ 价,一个氯原子得到一个电子,即氯为 $-1$ 价。
在氯化镁这个化合物里,一个镁原子失去两个电子,  镁为 $+2$ 价;而 1 个氯原子还是得到 1 个电子,氯仍为 $-1$ 价。
\zhongdian{在共价化合物里,元素化合价的数值,就是这种元素的一个原子跟其它元素的原子形成的共用电子对的数目。}
化合价的正负由电子对的偏移来决定。电子对偏向哪种原子,哪种原子就为负价;电子对偏离哪种原子,哪种原子就为正价。
例如,
在氯化氢这个化合物里,电子对偏向氯原子,氯为 $-1$ 价,电子对偏离氢原子,氢为 $+1$ 价。
在水这个化合物里,氧为 $-2$ 价,氢为 $+1$ 价。显然,
\zhongdian{不论在离子化合物还是在共价化合物里,正负化合价的代数和都等于零。}

在化合物里,金属元素通常显正价,非金属元素通常显负价。
但在非金属氧化物里,氧显负价,另一非金属元素显正价。
氧和氢在它们各自的化合物里,氧通常显 $-2$ 价,氢通常显 $+1$ 价。

许多元素的化合价不是固定不变的。在不同条件下,同一原子既可失去电子(或电子偏离),
也可得到电子(或电子偏近);而且失去电子的数目也可以不同,因此元素就显示出可变化合价来。
例如,在不同条件下,铁可显 $+2$ 价或 $+3$ 价,硫可显 $-2$ 价、$+4$ 价或 $+6$ 价。

常见元素的化合价见表 \ref{tab:2-3} 。



\begin{table}[htbp]
    \centering
    \caption{常见元素的化合价表}\label{tab:2-3}
    \begin{tblr}{
        hline{1,13}={1.5pt, solid},
        hline{2}={solid},
        vlines,
        vline{1,7}={1.5pt, solid},
        vline{4}={1}{-}{},
        vline{4}={2}{-}{},
        columns={c},
        column{3,6}={mode=math},
        row{1}={m, mode=text},
    }
        {元素\\名称} & {元素\\符号} & 常见的化合价 & {元素\\名称} & {元素\\符号} & 常见的化合价 \\
        钾 & K  & +1                    & 氢  & H  & +1 \\
        钠 & Na & +1                    & 氟  & F  & -1 \\
        银 & Ag & +1                    & 氯  & Cl & -1,\; +1,\; +5,\; +7 \\
        钙 & Ca & +2                    & 溴  & Br & -1 \\
        镁 & Mg & +2                    & 碘  & I  & -1 \\
        钡 & Ba & +2                    & 氧  & O  & -2 \\
        锌 & Zn & +2                    & 硫  & S  & -2,\; +4,\; +6 \\
        铜 & Cu & +1,\; +2              & 碳  & C  & +2,\; +4 \\
        铁 & Fe & +2,\; +3              & 硅  & Si & +4 \\
        铝 & Al & +3                    & 氮  & N  & -3,\; +2,\; +4,\; +5 \\
        锰 & Mn & +2,\; +4,\; +6,\; +7  & 磷  & P  & -3,\; +3,\; +5 \\
    \end{tblr}
\end{table}


元素的化合价是元素的原子在形成化合物时表现出来的一种性质,因此,在单质分子里,元素的化合价为零。

\liti 标出下列物质中各元素的化合价。

\qquad 氯化锌 \quad 金属锌 \quad 氢氧化钠

\jie 先写出各物质的分子式。

\qquad \ce{ZnCl2} \quad \ce{Zn} \quad \ce{NaOH}

然后在各分子式元素符号的上方标出相应的化合价。

\qquad $\overset{+2}{\ce{Zn}} \overset{-1}{\ce{Cl2}}$
    \quad $\overset{0}{\ce{Zn}}$
    \quad $\overset{+1}{\ce{Na}} \overset{-2}{\ce{O}} \overset{+1}{\ce{H}}$


想一想为什么在氯化锌中锌的化合价为 $+2$,而在金属锌中锌的化合价为零。

\liti 计算 \ce{Ca(OH)2} 中各元素化合价的代数和。

\jie 先标出 \ce{Ca(OH)2} 中各元素的化合价。
$$ \overset{+2}{\ce{Ca}} ( \overset{-2}{\ce{O}} \overset{+1}{\ce{H}} )_2 $$

然后计算各元素化合价的代数和,它等于:

钙的化合价 $\times$ 钙原子数 $+$ 氧的化合价 $\times$ 氧原子数 $+$ 氢的化合价 $\times$ 氢原子数

即 $\begin{aligned}[t]
    2 \times 1 + (-2) \times 2 + 1 \times 2 &= 4 - 4 \\
                                            &= 0
\end{aligned}$

答: \ce{Ca(OH)2} 中各元素化合价的代数和为零。


\begin{xiti}

\xiaoti{标出下列物质中各元素的化合价:\\
    \ce{SiO2}, \ce{HBr},  \ce{AgI}, \ce{Fe2O3}, \ce{Cu}。
}

\xiaoti{下面的说法有没有错误?说明理由。}
\begin{xiaoxiaotis}

    \xxt{在 \ce{H2} 中 \ce{H} 的化合价是 $+1$。}

    \xxt{在 \ce{H2O} 中 \ce{O} 的化合价是 $-2$。}

    \xxt{\ce{Fe} 有可变价 $+2$ 和 $+3$ 价,因此在 \ce{Fe2O3} 中 \ce{Fe} 既可以是 $+2$ 价,又可以是 $+3$ 价。}

\end{xiaoxiaotis}


\xiaoti{在下列化合物中 \ce{Cl} 为 $+5$ 价,\ce{Mn} 为 $+7$ 价,\ce{N} 为 $-3$ 价,
    \ce{S} 为 $+6$ 价,计算这些化合物中各元素化合价的代数和,
    并解释为什么化合物里正负化合价的代数和为零。
    $$ \ce{KClO3} \nsep \ce{KMnO4} \nsep \ce{(NH4)2SO4} \juhao $$
}

\end{xiti}
