\section{配制一定浓度的溶液}\label{sec:xssy-sy6}

\begin{shiyanmudi}
    1. 学会配制一定质量百分比浓度的溶液; 2. 学会配制一定体积比浓度的溶液。
\end{shiyanmudi}


\begin{shiyanyongpin}
    托盘天平、烧杯、玻璃棒、药匙、量筒(100 毫升、10 毫升)。

    氯化钠、浓盐酸(密度是 $1.19 \; \kmlflm$)。
\end{shiyanyongpin}


\begin{shiyanbuzhou}
    1. 氯化钠溶液的配制

    (1) 计算 \quad 配制 50 克 $5\%$ 的氯化钠溶液,需要氯化钠 \xhx 克和水 \xhx 克。

    (2) 称量 \quad 按托盘天平的使用要求进行称量操作,称量所需的氯化钠,倒入烧杯里。

    (3) 溶解 \quad 水在 4 ℃ 时的密度为 $1 \; \kmlflm$。用量筒量取 \xhx 毫升水,即可近似地看作是 \xhx 克水。
    把量取好的水倒入装有氯化钠的烧杯里,用玻璃棒搅拌,使氯化钠溶解。这样可以制得 50 克 $5\%$ 的氯化钠溶液。

    2. 稀盐酸的配制

    (1) 计算 \quad 用 5 毫升浓盐酸配制 $1:4$ 的稀盐酸,需加水 \xhx 毫升。

    (2) 量取 \quad 用 10 毫升的量筒量取 5 毫升浓盐酸,把浓盐酸倒入烧杯里。

    (3) 溶解 \quad 用量筒量取所需加的水,把水倒入盛有盐酸的烧杯里,边加边搅拌,使盐酸和水混和均匀。
\end{shiyanbuzhou}


\begin{wentihetaolun}
    1. 在实验步骤 1 中,为什么可以不直接用天平称量水的质量,而换算成体积,用量筒量取?

    2. 在实验步骤 2 中,为什么不选用 100 毫升的量筒量取 5 毫升浓盐酸?
\end{wentihetaolun}

