\section{制取硫酸铜晶体}\label{sec:xssy-xzsy2}

\begin{shiyanmudi}
    1. 学习通过化学反应制取硫酸铜晶体的方法; 2. 巩固对酸的性质的认识。
\end{shiyanmudi}


\begin{shiyanyongpin}
    铁架台(带铁圈)、蒸发皿、坩埚钳、玻璃棒、烧杯、量筒、酒精灯、漏斗、滤纸、剪刀、药匙。

    氧化铜、稀硫酸($1:4$)。
\end{shiyanyongpin}


\begin{shiyanbuzhou}
    1. 用量筒量取 15 毫升稀硫酸倒入一个蒸发皿里,把蒸发皿放在铁架台的铁圈上,
    用酒精灯加热到将近沸腾(但不要使稀硫酸沸腾)。然后注意保持这个温度,
    一边用玻璃棒进行搅拌,一边慢慢地撒入氧化铜粉末,直到氧化铜不再溶解为止。
    记录观察到的现象,并写出反应的化学方程式。

    2. 装置好漏斗,趁热\footnote{硫酸铜在水中的溶解度(克)\\[.5em]
        \begin{tblr}{hlines, vlines,columns={c,colsep=1em}}
                       & 0 ℃ & 20 ℃ & 40 ℃ & 60 ℃ & 80 ℃ \\
            \ce{CuSO4} & 14.3 & 20.7 & 28.5 & 40  & 55 \\
        \end{tblr}
    }
    过滤(过滤时,不溶性物质可以留在蒸发皿里,不必转移到滤纸上)。将滤液收集在烧杯里。

    3. 使滤液逐渐冷却,仔细观察晶体生成时所发生的现象,并记录晶体的颜色和形状。
    如果滤液放置一段时间后,没有晶体生成,可以把滤液倒入洗净的蒸发皿里蒸发几分钟,
    再放置冷却,就会有晶体生成。
\end{shiyanbuzhou}


\begin{wentihetaolun}
    1. 为什么要趁热过滤? 等液体冷后再过滤会出现什么现象?

    2. 如果过滤速度太慢会发生什么情况? 在实验时应怎样做好过滤前的准备工作?
\end{wentihetaolun}

