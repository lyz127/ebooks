\section{常见的酸}\label{sec:5-3}

我们已经知道,酸是一类电解质,它们在电离时生成的阳离子全部是氢离子。
我们已经知道的硫酸、盐酸、碳酸、醋酸都属于酸。现在我们来学习几种常见的、重要的酸。


\subsection{盐酸(\ce{HCl})}

盐酸是氯化氢气体的水溶液。

\begin{shiyan}
    观察纯净的浓盐酸和工业品浓盐酸的颜色、状态以及它们在空气里形成的白雾。
    用手轻轻地在瓶口扇动,小心地闻盐酸的气味。
\end{shiyan}

纯净的浓盐酸是没有颜色的液体,有刺激性气味。
工业品的浓盐酸常因含有杂质而带黄色。
常用的浓盐酸约含 $37\%$ 的氯化氢,密度是 $1.19 \; \kmlflm$。
浓盐酸在空气里会生成白雾,这是因为从浓盐酸挥发出来的氯化氢气体跟空气里的水蒸气接触,
形成盐酸小液滴的缘故。盐酸有酸味,有腐蚀性。

\begin{shiyan}
    把紫色石蕊试液和无色酚酞试液分别加入两个盛有稀盐酸的试管里,观察溶液的颜色有什么变化。
\end{shiyan}

\begin{shiyan}
    把紫色石蕊试液和无色酚酞试液分别加入两个盛有氢氧化钠稀溶液的试管里,观察溶液颜色有什么变化。
\end{shiyan}

石蕊试液遇盐酸变成红色,酚酞试液遇盐酸不变色。
石蕊试液遇碱溶液变蓝色,酚酞试液碱溶滚变红色。
象石蕊和酚酞这能跟酸或碱的溶液起作用而显示不同颜色的物质,叫做\zhongdian{酸碱指示剂},
通常也简称\zhongdian{指示剂}。

盐酸还能跟多种金属、金属氧化物、金属氢氧化物等物质起反应,下面简单介绍盐酸的其它化学性质。


\subsubsection{盐酸跟金属的反应}

\begin{shiyan}
    把锌粒和铁屑分别放入盛有稀盐酸的两个试管里,观察发生的变化。试验产生的气体是不是氢气。
\end{shiyan}

盐酸跟锌、铁起置换反应,分别生成氢气、氯化锌或氯化亚铁。
\begin{fangchengshi}
    \ce{ Zn + 2HCl = \underset{\text{氯化锌}}{\ce{ZnCl2}} + H2 ^ } \\
    \ce{ Fe + 2HCl = \underset{\text{氯化亚铁}}{\ce{FeCl2}} + H2 ^ }
\end{fangchengshi}


\subsubsection{盐酸跟金属氧化物的反应}

\begin{shiyan}
    把一根生锈的铁钉放入盛有稀盐酸的试管里,过一会儿取出,用水洗净,观察铁钉表面的变化。
\end{shiyan}

从实验看出,铁钉表面的锈已被除去,这是因为盐酸跟铁锈(主要成分是 \ce{Fe2O3 . H2O} ) 起反应,
生成可溶性的氯化铁的缘故。
\begin{fangchengshi}
    \ce{ Fe2O3 + 6HCl = \underset{\text{氯化铁}}{\ce{2FeCl3}} + 3H2O }
\end{fangchengshi}

由于盐酸跟金属氧化物起反应后生成可溶性物质,金属制品在电镀、焊接等操作前可以用盐酸来清除表面上的锈。


\subsubsection{盐酸跟碱的反应}

\begin{shiyan}
    在盛有少量氢氧化铜的试管里,加入适量的盐酸,观察发生的变化。
\end{shiyan}

从实验看出,盐酸跟不溶于水的氢氧化铜起反应,生成能溶于水的氯化铜。
\begin{fangchengshi}
    \ce{ Cu(OH)2 + 2HCl = \underset{\text{氯化铜}}{\ce{CuCl2}} + 2H2O }
\end{fangchengshi}



\subsubsection{盐酸跟硝酸银的反应}

\begin{shiyan}
    在盛有少量稀盐酸的试管里,滴入几滴硝酸银溶液和几滴稀硝酸\footnote{加入几滴稀硝酸是为了防止某些杂质的干扰。},观察发生的现象。
\end{shiyan}

从实验看出,盐酸跟硝酸银起反应,生成不溶于硝酸的氯化银凝乳状白色沉淀。
\begin{fangchengshi}
    \ce{ HCl + \underset{\text{硝酸银}}{\ce{AgNO3}} = \underset{\text{氯化银}}{\ce{AgCl}} v + HNO3 }
\end{fangchengshi}

这个反应可以用于检验盐酸,也可用于检验氯化钠等可溶性氯化物,因为反应后都生成不溶于硝酸的氯化银。
\begin{fangchengshi}
    \ce{ NaCl + AgNO3 = AgCl v + NaNO3 }
\end{fangchengshi}

在以上两个反应里,参加反应的两种化合物互相交换成分,生成另外两种化合物。
象这类\zhongdian{由两种化合物互相交换成分,生成另外两种化合物的反应,叫做复分解反应。}

盐酸是一种重要的化工产品。除用于金属表面的除锈外,还用于制造氯化锌等氯化物,
以及某些药剂和试剂。人的胃液里含有少量的盐酸,可以帮助消化。



\subsection{硫酸(\ce{H2SO4})}

\begin{shiyan}
    观察浓硫酸的颜色和状态。用玻璃棒蘸浓硫酸在纸上写字,过一会儿,观察纸上有什么变化。
    用火柴梗蘸一点浓硫酸,放置一会儿,观察火柴梗有什么变化。
\end{shiyan}

纯净的浓硫酸是没有颜色、粘稠、油状的液体,不容易挥发。
常用浓硫酸的浓度是 $98\%$,密度是 $1.84 \; \kmlflm$。

浓硫酸有吸水性,跟空气接触,能够吸收空气里的水分,所以,它常用作某些气体的干燥剂。
浓硫酸也能够夺取纸张、木材、衣服、皮肤(它们都是由含碳、氢、氧等元素的化合物组成的)里的
水分\footnote{严格地说,浓硫酸能将这些物质中的氢、氧元素按水的组成比脱去,这种作用通常叫做脱水作用。},
使它们碳化。上面的实验里,纸和火柴梗的颜色变黑也就是发生了碳化的缘故。
硫酸对皮肤或衣服有很大的腐蚀性,如果不慎在皮肤或衣服上沾上硫酸,应立即用布拭去,再用水冲洗。

浓硫酸很容易溶解于水,同时放出大量的热。如果把水倒进浓硫酸里,水的比重较小,浮在硫酸上面,
溶解时放出的热会使水立刻沸腾,使硫酸液滴向四周飞溅。为了防止发生事故,
\zhongdian{稀释浓硫酸时,一定要把浓硫酸沿着器壁慢慢地注入水里,并不断搅动,
使产生的热量迅速地扩散。切不可把水倒进浓硫酸里。}


\begin{shiyan}
    把浓硫酸沿着烧杯壁缓慢地注入盛有水的烧杯里,用玻璃棒不断搅动,用手接触烧杯外壁,注意溶液温度有什么变化。
\end{shiyan}

稀硫酸也可使紫色的石蕊试液变红,无色的酚酞试液遇稀硫酸不变色。现将稀硫酸的其它化学性质简单介绍于下:

\subsubsection{稀硫酸跟金属的反应}

\begin{shiyan}
    在盛有稀硫酸的试管里,轻轻放入锌粒,观察发生的现象。
\end{shiyan}

稀硫酸跟锌起反应放出氢气,同时生成能溶于水的硫酸锌。
\begin{fangchengshi}
    \ce{ Zn + H2SO4 = ZnSO4 + H2 ^ }
\end{fangchengshi}


\subsubsection{稀硫酸跟金属氧化物的反应}

\begin{shiyan}
    在盛有稀硫酸的试管里,加入一个生锈的铁钉,稍加热,观察发生的变化。
\end{shiyan}

可以看到,铁钉上的锈逐渐消失。
\begin{fangchengshi}
    \ce{ Fe2O3 + 3H2SO4 = \underset{\text{硫酸铁}}{\ce{Fe2(SO4)3}} + 3H2O }
\end{fangchengshi}


\subsubsection{稀硫酸跟碱的反应}

\begin{shiyan}
    在盛有少量氢氧化铜的试管里,加入少量稀硫酸,现察发生的变化。
\end{shiyan}

可以看到,稀硫酸跟不溶于水的氢氧化铜起反应,生成能溶于水的硫酸铜。
\begin{fangchengshi}
    \ce{ Cu(OH)2 + H2SO4 = CuSO4 + 2H2O }
\end{fangchengshi}


\subsubsection{稀硫酸跟氯化钡的反应}

\begin{shiyan}
    在盛有少量稀硫酸的试管里,注入几滴氯化钡溶液和几滴稀硝酸\footnote{加入几滴稀硝酸是为了防止某些杂质的干扰。}, 观察发生的现象。
\end{shiyan}

从实验看出,稀硫酸跟氯化钡起反应,生成不溶于硝酸的白色硫酸钡沉淀。
\begin{fangchengshi}
    \ce{ H2SO4 + \underset{\text{氯化钡}}{\ce{BaCl2}} = \underset{\text{硫酸钡}}{\ce{BaSO4}} v + 2HCl }
\end{fangchengshi}

这个反应可用于检验硫酸,也可用于检验硫酸钠等可溶性的含硫酸根的盐,因为反应都生成不溶于硝酸的硫酸钡。
\begin{fangchengshi}
    \ce{ Na2SO4 + BaCl2 = BaSO4 v + 2NaCl }
\end{fangchengshi}

硫酸是一种非常重要的化工原料,广泛应用于生产化肥、农药、火药、染料以及冶炼有色金属、精炼石油、金属去锈等方面。



\subsection{硝酸(\ce{HNO3})}

\begin{shiyan}
    观察硝酸的状态和顔色。观察稀硝酸对石蕊试液和酚酞试液的颜色变化。
\end{shiyan}

纯净的硝酸是一种无色的液体,具有刺激性的气味。跟盐酸相似,在空气里也能挥发出 \ce{HNO3} 气体,
跟空气里的水蒸气结合成硝酸小液滴,形成白雾。
硝酸也能强烈地腐蚀皮肤和衣服,使用硝酸的时候,要特别小心。
硝酸溶液也能使紫色的石蕊试液变成红色,无色的酚酞试液遇硝酸溶液不变色。

硝酸的氧化性很强,它跟金属起反应的时候,一般生成水而不生成氢气。

硝酸也能跟金属氧化物象氧化锌、氧化镁等以及碱类象氢氧化锌、氢氧化镁等起反应,生成水和硝酸锌、硝酸镁等化合物。

硝酸是一种重要的化工原料,广泛应用于生产化肥、染料、火药等方面。

除上述三种酸以外,磷酸(\ce{H3PO4})也是一种常见的重要的酸。
磷酸是一种无色的晶体,容易吸收水分,易溶于水。
通常用的磷酸是一种糖浆状的水溶液,含有 $70—80\%$ 的磷酸。

磷酸对石蕊或酚酞试液的作用以及跟某些金属、金属氧化物、碱所起的反应都同盐酸、硫酸等相似。

磷酸的主要用途是制造磷肥。


\begin{xiti}

\xiaoti{对于化合、分解、置换、复分解等四种反应类型,各举出两个例子,写出它们的化学方程式。}

\xiaoti{1 克的锌和 1 克的铁分别跟足量的稀盐酸起反应,各生成多少克氢气?它们的体积各是多少升(在标准状况下)?}

\xiaoti{150 毫升密度为 $1.028 \; \kmlflm$ 的盐酸($6\%$ \ce{HCl})跟足量的硝酸银溶液起反应,计算生成的氯化银的质量。}

\xiaoti{在两个烧杯里分别盛 50 克浓硫酸和 50 克浓盐酸。在空气中放置一定时间后,二者的质量有什么变化?为什么?}

\xiaoti{现有浓度为 $98\%$,密度为 $1.84 \; \kmlflm$ 的浓硫酸,要配制 500 克 $20\%$ 硫酸,需用浓硫酸和水各多少毫升?怎样操作?}

\xiaoti{写出下列物质间反应的化学方程式:}
\begin{xiaoxiaotis}

    \xxt{铝跟硫酸,}
    \xxt{硝酸跟氢氧化钙,}
    \xxt{氧化铜跟硫酸。}

\end{xiaoxiaotis}


\xiaoti{有一块已部分氧化的锌片 42 克,跟足量稀硫酸反应完全后,生成氢气 $1.2$ 克,求锌片中锌的百分含量。}

\xiaoti{取 2 克含有杂质氯化钠的硫酸钠,溶于水后,滴加氯化钡溶液到沉淀不再生成为止,生成 $3.2$ 克硫酸钡,求硫酸钠含杂质的百分比。}

\end{xiti}

