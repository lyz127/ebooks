\section{元素 元素符号}\label{sec:1-6}

\subsection{元素}

我们知道,氧分子是由氧原子构成的,二氧化碳分子是由氧原子和碳原子构成的。
无论氧分子中的氧原子还是二氧化碳分子中的氧原子,核电荷数都是 $8$, 都有 $8$ 个质子。
在化学上,我们把\zhongdian{具有相同的核电荷数(即质子数)的同一类原子总称为元素。}
氧元素就是所有氧原子的总称,碳元素就是所有碳原子的总称。
人们把氨水、碳酸氢铵、尿素等化学肥料叫做氮肥,就是因为这些化学肥料的成分里,都含有作物所需要的氮元素的缘故。

在自然界里,物质的种类非常多,有几百万种以上。但是,组成这些物质的元素并不多。
到目前为止,已经知道的元素有 $107$ 种,其中包括十几种人造元素。
这些元素,好象是多种 “积木” 搭成各式各样的造型那样,组成了几百万种以上形形色色的物质。

我们研究某一物质,通常指的是纯净物。
在纯净物里,有的是由同种元素组成的,如氧气是由氧元素组成的,铁是由铁元素组成的。
象这种\zhongdian{由同种元素组成的纯净物叫做单质。}
有的单质由分子构成,如氧气、氮气、氢气等等,有的单质由原子构成, 如铁、镁、铝、铜等等。
根据单质的不同性质,单质一般可分为非金属和金属两大类。
例如,氧气、氮气、硫、磷等等都是非金属单质(组成非金属单质的元素叫非金属元素);
铁、铝、铜等等都是金属单质(组成金属单质的元素叫金属元素)。
非金属没有金属光泽,一般不能导电、传热,通常是固体或气体(溴是液体);
金属的性质就不同,具有特殊的金属光泽,容易导电、传热,有可塑性、延展性,常温下是固体(汞是液体)。
但非金属和金属之间没有绝对的界限。例如,用作半导体材料的硅和锗,既有金属性质,又有非金属性质。

有些物质的组成比较复杂。例如,氧化镁是由氧和镁两种不同的元素组成的;
氯酸钾是由钾、氯和氧三种不同的元素组成的;
碳酸氢铵是由氮、碳、氢和氧四种不同的元素组成的。
象这种\zhongdian{由不同种元素组成的纯净物叫做化合物。}

在各种化合物里,有些是由两种元素组成的,其中一种是氧元素,这种化合物叫做\zhongdian{氧化物}。
如氧化镁、二氧化碳等等,都是氧化物。氧化镁是金属氧化物,二氧化碳是非金属氧化物。

元素一般都有两种存在的形态。
一种是以单质  的形态存在的,叫做元素的\zhongdian{游离态};
一种是以化合物的形态存在的,叫做元素的\zhongdian{化合态}。
例如,氧气里的氧元素就是游离态的,二氧化碳、四氧化三铁里的氧元素就是化合态的。

\begin{figure}[htbp]
    \centering
    \begin{tikzpicture}
    \pgfmathsetmacro{\R}{3}
    \coordinate (O) at (0, 0);
    \foreach \per/\text [remember=\start as \laststart (initially -30)] in {
        1.20/其它,
        0.76/氢,   2.00/镁,    2.47/钾,
        2.74/钠,   3.45/钙,    4.75/铁,
        7.73/铝,    26.30/硅,  48.60/氧
    } {
        \pgfmathsetmacro{\start}{\laststart + \per * 3.6}
        \draw (O) -- (\laststart:\R) arc (\laststart:\start:\R) -- (O);
    }

    \draw (-1, -1) node {氧 $48.60\%$};
    \draw (-0.6,  1.5) node {硅 $26.30\%$};
    \draw (2, 2)       -- +(1.4, 0) node [right] {铝 $7.73\%$};
    \draw (2.4, 1.3)   -- +(1.0, 0) node [right] {铁 $4.75\%$};
    \draw (2.8, 0.6)   -- +(0.6, 0) node [right] {钙 $3.45\%$};
    \draw (2.8, -0.1)  -- +(0.6, 0.2) node [right] {钠 $2.74\%$};
    \draw (2.8, -0.5)  -- +(0.6, 0.0) node [right] {钾 $2.47\%$};
    \draw (2.7, -1.0)  -- +(0.7, 0.0) node [right] {镁 $2.00\%$};
    \draw (2.65, -1.2) -- +(0.75, -0.2) node [right] {氢 $0.76\%$};
    \draw (2.55, -1.3) -- +(0.55, -0.5) node [right] {其它 $1.20\%$};
\end{tikzpicture}


    \caption{地壳里所含各种元素的质量百分比}\label{fig:1-12}
\end{figure}

各种元素在地壳里的含量相差很大。从图 \ref{fig:1-12} 可以看到,
地壳主要是由氧、硅、铝、铁、钙、钠、钾、镁、氢等等元素组成的。
含量最多的元素是氧,其次是硅。氧几乎占地壳的一半。氧在自然界里起着重要的作用。
但是,如果以为含量少的那些元素在自然界里起着次要的作用,那就错了。
例如, 碳、氢和氮对动植物有着非常重要的作用,但是这三种元素在地壳里的质量百分比却较小:
碳是 $0.087\%$, 氢是 $0.76\%$, 氮是 $0.03\%$。


\subsection{元素符号}

在化学上,采用不同的符号表示各种元素。例如,
用 “O” 表示氧元素, 用 “C” 表示碳元素, 用 “S” 表示硫元素, 用 “Fe” 表示铁元素等等。
这种符号叫做\zhongdian{元素符号}。

\begin{table}[htbp]
    \centering
    \caption{一些常见元素的名称、符号、原子量(近似值)}\label{tab:1-2}
    \begin{tblr}{
        hline{1,11}={1.5pt, solid},
        hline{2}={solid},
        vlines,
        vline{1,10}={1.5pt, solid},
        vline{4,7}={1}{-}{},
        vline{4,7}={2}{-}{},
        row{1}={guard, m},
        colspec={
            ccQ[si={table-format=3.1},c]
            ccQ[si={table-format=3.1},c]
            ccQ[si={table-format=3.1},c]
        },
    }
        {元素\\名称} & {元素\\符号} & 原子量 & {元素\\名称} & {元素\\符号} & 原子量 & {元素\\名称} & {元素\\符号} & 原子量 \\
        氢  &  H  &  1   &  碘 &  I  & 127   &  锌 & Zn &  65 \\
        氮  &  N  & 14   &  钠 &  Na &  23   &  银 & Ag & 108 \\
        氧  &  O  & 16   &  镁 &  Mg &  24   &  锡 & Sn & 119 \\
        氯  &  Cl & 35.5 &  铝 &  Al &  27   &  锑 & Sb & 122 \\
        溴  &  Br & 80   &  钾 &  K  &  39   &  钡 & Ba & 137 \\
        碳  &  C  & 12   &  钙 &  Ca &  40   &  钨 & W  & 184 \\
        硅  &  Si & 28   &  锰 &  Mn &  55   &  金 & Au & 197 \\
        磷  &  P  & 31   &  铁 &  Fe &  56   &  汞 & Hg & 201 \\
        硫  &  S  & 32   &  铜 &  Cu &  63.5 &  铅 & Pb & 207 \\
    \end{tblr}
\end{table}


在国际上,元素符号是统一采用该元素的拉丁文名称的第一个大写字母来表示的,
如果几种元素符号的第一个字母相同时,可再附加一个小写字母来区别。
例如, “Cu” 代表铜元素, “Ca” 代表钙元素等等。

书写元素符号时应该注意,第二个字母必须小写,以免混淆。
例如, “Co” 表示钴原子,如果写成 “CO”,便表示一氧化碳分子了。

元素符号表示一种元素,还表示这种元素的一个原子。


一些常见元素的名称\footnotemark 、符号和一般化学计算采用的原子量(近似值)见表 \ref{tab:1-2} 。
\footnotetext{
    表示每种元素的名称都有一个专用的汉字。
    气态非金属元素的名称都有 “气” 字头,
    液态非金属元素的名称都有 “氵” 旁,
    固态非金属元素的名称都 “石” 字旁,
    金属元素的名称都有 “钅” 旁(汞除外)。
}

\begin{xiti}

\xiaoti{下列的说法有没有错误?把错误的说法加以改正。}
\begin{xiaoxiaotis}

    \xxt{二氧化硫分子里含有两个氧元素和一个硫元素。}

    \xxt{二氧化碳是由氧气和碳两种单质组成的。}

    \xxt{凡是含有氧元素的化合物都叫做氧化物。}

\end{xiaoxiaotis}


\xiaoti{指出下列物质里,哪些是单质,哪些是化合物,哪些是混和物?为什么?}
\begin{xiaoxiaotis}

    \threeInLineXxt{空气,}{氧化镁,}{氯酸钾,}

    \threeInLineXxt{氧气,}{硫,}{铁。}

\end{xiaoxiaotis}


\xiaoti{指出下列物质里,各元素以游离态存在,还是以化合态存在,并指明哪些物质是氧化物。}
\begin{xiaoxiaotis}

    \threeInLineXxt{氧气,}{二氧化硫,}{铁粉,}

    \twoInLineXxt{硫粉,}{氧化汞。}

\end{xiaoxiaotis}


\xiaoti{填空:\\
    \begin{tblr}{hlines, vlines,
        columns={c, colsep=0.8em},
        column{1}={4em},
     }
        元素名称 & 氮 & 钾 &    & 氢 &    &    & 碳 & 铁  &   & 钠 &    & 硅 \\
        元素符号 &    &    & P  &    &  O & Al &    &    & Cl &   & Ca  &   \\
        {原子量\\[-0.5em](近似值)} &
    \end{tblr}
}


\end{xiti}


