\section{核外电子排布的初步知识}\label{sec:2-4}

电子是质量很小的带负电的微粒,它在原子核外作高速的运动。
在含有多个电子的原子里,电子的能量并不相同。
能量低的,通常在离核近的区域运动。
能量高的,通常在离核远的区域运动。
为了便于说明问题,通常就用电子层来表明运动着的电子离核远近的不同。
把能最低、离核最近的叫第一层,能量稍高、离核稍远的叫第二层,由里往外依次类推,叫三、四、五、六、七层。
也可以把它们依次叫 K 、L、M 、N、O、P、Q 层。
这样,电子就可以看作是在能量不同的电子层上运动的。

核外电子的分层运动,又叫核外电子的分层排布。
经科学研究证明的核电荷数从 $1$ 到 $20$ 的元素和 $6$ 个惰性气体元素原子的电子层排布情况见表 \ref{tab:2-1} 、\ref{tab:2-2}。

从表 \ref{tab:2-1} 、\ref{tab:2-2} 可以看出,核外电子的分层排布是有一定规律的。

首先,各电子层最多容纳的电子数目是 $2n^2$。即
K 层($n = 1$)为 $2 \times 1^2 =  2$ 个;
L 层($n = 2$)为 $2 \times 2^2 =  8$ 个;
M 层($n = 3$)为 $2 \times 3^2 = 18$ 个等。

其次,最外层电子数目不超过 $8$ 个( K 层为最外层时不超过 $2$ 个)。


\begin{table}[htbp]
    \centering
    \caption{部分元素原子的电子层排布}\label{tab:2-1}
    \begin{tblr}{
        hline{1,23}={1.5pt, solid},
        hline{2,3}={solid},
        vline{1,8}={1.5pt, solid},
        vline{2,3,4}={solid},
        columns={colsep=1.0em},
        row{1,2}={guard, m},
        colspec={Q[si={table-format=2},c]cccccc},
    }
        \SetCell[r=2]{c} 核电荷数
            & \SetCell[r=2]{c} 元素名称
            & \SetCell[r=2]{c} 元素符号
            & \SetCell[c=4]{c} 各电子层的电子数 & & & \\
           &    &    & K & L & M & N \\
        1  & 氢 & H  & 1 &   &   &   \\
        2  & 氦 & He & 2 &   &   &   \\
        3  & 锂 & Li & 2 & 1 &   &   \\
        4  & 铍 & Be & 2 & 2 &   &   \\
        5  & 硼 & B  & 2 & 3 &   &   \\
        6  & 碳 & C  & 2 & 4 &   &   \\
        7  & 氮 & N  & 2 & 5 &   &   \\
        8  & 氧 & O  & 2 & 6 &   &   \\
        9  & 氟 & F  & 2 & 7 &   &   \\
        10 & 氖 & Ne & 2 & 8 &   &   \\
        11 & 钠 & Na & 2 & 8 & 1 &   \\
        12 & 镁 & Mg & 2 & 8 & 2 &   \\
        13 & 铝 & Al & 2 & 8 & 3 &   \\
        14 & 硅 & Si & 2 & 8 & 4 &   \\
        15 & 磷 & P  & 2 & 8 & 5 &   \\
        16 & 硫 & S  & 2 & 8 & 6 &   \\
        17 & 氯 & Cl & 2 & 8 & 7 &   \\
        18 & 氩 & Ar & 2 & 8 & 8 &   \\
        19 & 钾 & K  & 2 & 8 & 8 & 1 \\
        20 & 钙 & Ca & 2 & 8 & 8 & 2 \\
    \end{tblr}
\end{table}

第三,次外层电子数目不超过 $18$ 个,倒数第三层电子数目不超过 $32$ 个。


\begin{table}[htbp]
    \centering
    \caption{惰性气体元素原子的电子层排布}\label{tab:2-2}
    \begin{tblr}{
        hlines,
        hline{1,9}={1.5pt, solid},
        vlines,
        vline{1,10}={1.5pt, solid},
        columns={colsep=1.0em},
        row{1,2}={guard, m},
        colspec={Q[si={table-format=2},c]cccccccc},
    }
        \SetCell[r=2]{c} 核电荷数
            & \SetCell[r=2]{c} 元素名称
            & \SetCell[r=2]{c} 元素符号
            & \SetCell[c=6]{c} 各电子层的电子数 & & & & & \\
           &    &    & K & L & M  & N  & O  & P \\
        2  & 氦 & He & 2 &   &    &    &    &   \\
        10 & 氖 & Ne & 2 & 8 &    &    &    &   \\
        18 & 氩 & Ar & 2 & 8 &  8 &    &    &   \\
        36 & 氪 & Kr & 2 & 8 & 18 &  8 &    &   \\
        54 & 氙 & Xe & 2 & 8 & 18 & 18 &  8 &   \\
        86 & 氡 & Rn & 2 & 8 & 18 & 32 & 18 & 8 \\
    \end{tblr}
\end{table}

科学研究还发现另一条规律,就是核外电子总是尽先排布在能量最低的电子层里,然后再由里往外,
依次排布在能量逐步升高的电子层里,即排满了 K 层才排 L 层,排满了 L 层才排 M 层。

以上几点是互相联系,不能孤立地理解的。
例如,当 M 层不是最外层时,最多可以排布 $18$ 个电子,而当它是最外层时,则最多可以排布 $8$ 个电子。
又如,当 O 层为次外层时,就不是最多排 $2 \times 5^2 = 50$ 个电子,而是最多排布 $18$ 个电子。
另外,这几点只是人们对核外电了排布的粗浅认识,还有一些复杂的情况,将在高中化学里进一步学习。

知道原子的核电荷数和电子层排布以后,我们可以画出原子结构示意图。
图 \ref{fig:2-12} 是几种元素的原子结构示意图。 \tc{+1} 表示原子核及核内有 $1$ 个质子,
弧线表示电子层,弧线上面的数字表示该层的电子数。

\begin{figure}[htbp]
    \centering
    \begin{tikzpicture}
    \draw (0, 0) pic {atom} node [below=1.2cm] {氢原子};
    \draw (3, 0) pic {atom={number=8}} node [below=1.2cm, xshift=0.5cm] {氧原子};
    \draw (6, 0) pic {atom={number=11}} node [below=1.2cm, xshift=0.8cm] {钠原子};
\end{tikzpicture}


    \caption{几种元素原子结构示意图}\label{fig:2-12}
\end{figure}


从表 \ref{tab:2-1} 和表 \ref{tab:2-2} 还可以看出,惰性气体元素、金属元素和非金属元素的原子最外层的电子数目都各有特点:

1. 惰性气体元素,原子的最外层都有 $8$ 个电子(氦是 $2$ 个)。
惰性气体元素的化学性质比较稳定,一般不跟其它物质发生化学反应。
通常认为这种最外层有 $8$ 个电子(最外层是 K 层有 $2$ 个电子)的结构,是一种稳定结构。
这里所说的稳定是相对的,而不是绝对的。

2. 金属元素,象钠、钾、镁、铝等,它们原子的最外层电子的数目一般少于 $4$ 个。

3. 非金属元素,象氟、氯、硫、磷等,它们原子的最外层电子的数目一般多于 $4$ 个。

在化学反应中,金属元素的原子比较容易失去最外层电子而使次外层变成最外层,通常达到 $8$ 个电子的稳定结构,
非金属元素的原子比较容易获得电子,也使最外层通常达到 $8$ 个电子的稳定结构。
所以,元素的性质,特别是化学性质,跟它的原子的最外层电子数目关系非常密切。


\begin{xiti}

\xiaoti{画出氦、碳、氮、镁、磷、钾六种元素的原子结构示意图,并根据原子结构的特征说明它们各属于金属、非金属还是惰性气体元素。}

\xiaoti{某元素的原子核含有 $12$ 个质子,求核电荷数和最外层电子数.并说明它容易得电子,还是容易失电子。}

\xiaoti{表里有五个项目和五列,在每一列的空缺项目里,填入正确的答案。\\
    \begin{tblr}{hlines, vlines, columns={c}}
        元素名称 & 元素符号 & 核内质子数 & 核外电子总数 & 原子结构示意图 \\
        氟       &         &           &             &               \\
                &          &    18     &             &               \\
                &          &           &    20       &               \\
                &          &           &             &  \tikz {\draw (0, 0) pic {atom={number=16}};}  \\
                & \ce{Cl}  &           &             &               \\
    \end{tblr}
}

\end{xiti}


