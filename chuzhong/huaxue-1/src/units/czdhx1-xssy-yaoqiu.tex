\nonumsection{学生实验的要求}\label{sec:xssy-yaoqiu}

学生必须亲自动手做实验。为了保证学生实验课能够顺利地进行, 提高实验教学质量, 实验时必须注意下面几个问题:

1. 上实验课前,要复习课文里的有关内容,阅读实验说明,理解实验目的,明了实验步骤和注意事项。

对于实验习题,要预先经过研究,然后提出解决的方案和需用的仪器和药品。

2. 做实验以前,要检查实验用品是不是齐全。桌上的实验用品要摆得整齐。

3. 做实验的时候,必须按照实验说明的步骤和方法进行,必须遵从教师的指导。

4. 要注意安全。要遵守实验操作规程,特别是实验说明里有关预防发生事故的规定。要谨慎、妥善地处理腐蚀性物质和易燃、易爆、有毒的物质。

5. 要保持实验室里的安静,自觉地遵守纪律。要爱护公共财物和仪器设备。要注意节约药品。

6. 做实验的时候,要认真地和耐心细致地观察与实验目的要求有关的现象,分析现象发生的原因。
对于实验的内容、观察到的现象和得出的结论,都要实事求是地随时作记录。

7. 做完实验后,要认真地写出实验报告。实验报告可以参照下面的格式:

\begin{table}[H]
    \centering
    {\Large 化学实验报告}

    班级 \xhx[5em] 姓名 \xhx[5em] 日期  \xhx[5em] \\[1em]

    \begin{minipage}{12cm}
        实验名称:

        实验目的:

        \vspace*{1em}\begin{tblr}{
            vlines,
            hline{1}={1.5pt,solid},
            hline{2}={solid},
            vline{1,4}={1.5pt,solid},
        }
            实验的内容和装置图 & 观察到的现象 & 结论、解释和化学方程式 \\
            & &
        \end{tblr}

    \end{minipage}

\end{table}

8. 做完实验,拆开实验装置,把仪器里没有用的物质倒在废液缸里,把有用的物质倒在指定的容器里。
然后把仪器洗涤干净放回原处,把实验桌收拾干净。

