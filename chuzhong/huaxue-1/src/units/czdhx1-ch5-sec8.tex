\section{氧化物}\label{sec:5-8}

我们已经知道,氧和另一种元素组成的化合物叫做氧化物。
根据化学性质的不同,氧化物主要可以分成酸性氧化物、碱性氧化物和两性氧化物。


\subsection{酸性氧化物}

\zhongdian{凡能跟碱起反应,生成盐和水的氧化物,叫做酸性氧化物。}例如:
\begin{fangchengshi}
    \ce{ 2NaOH + CO2 = Na2CO3 + H2O } \\[-.8em]
    \ce{ Ca(OH)2 + SO3 = CaSO4 + H2O }
\end{fangchengshi}

上面反应中的二氧化碳、三氧化硫就是酸性氧化物。

非金属氧化物大多数是酸性氧化物。

酸性氧化物大都能够直接跟水化合而生成酸。例如:
\begin{fangchengshi}
    \ce{ CO2 + H2O = H2CO3 } \\[-.8em]
    \ce{ SO3 + H2O = H2SO4 }
\end{fangchengshi}

有些酸性氧化物就不能跟水化合,如二氧化硅(\ce{SiO2})\footnote{沙的主要成分就二氧化硅。}。

酸性氧化物可以由非金属跟氧气直接化合,或由含氧酸盐加热分解制得。例如:
\begin{fangchengshi}
    \ce{ S + O2 $\xlongequal{\text{点燃}}$ SO2 } \\[-.8em]
    \ce{ MgCO3 $\xlongequal{\Delta}$ MgO + CO2 ^ }
\end{fangchengshi}

酸性氧化物还可以由酸分解而制得。例如:
\begin{fangchengshi}
    \ce{ H2SO4 $\xlongequal{\Delta}$ H2O + SO3 ^ }
\end{fangchengshi}

酸性氧化物也叫做\zhongdian{酸酐},酸酐就是酸失去水以后的生成物。
三氧化硫是硫酸失去水以后的生成物,也叫硫酐。



\subsection{碱性氧化物}

\zhongdian{凡能跟酸起反应,生成盐和水的氧化物,叫做碱性氧化物。}例如:
\begin{fangchengshi}
    \ce{ 2HCl + CaO = CaCl2 + H2O } \\[-.8em]
    \ce{ H2SO4 + CuO = CuSO4 + H2O }
\end{fangchengshi}

上面反应中的氧化钙、氧化铜就是碱性氧化物。

金氧化物大多数是碱性氧化物。

有些碱性氧化物能够跟水化合生成碱。例如:
\begin{fangchengshi}
    \ce{ CaO + H2O = Ca(OH)2 } \\[-.8em]
    \ce{ Na2O + H2O = 2NaOH }
\end{fangchengshi}

大多数碱性氧化物(如氧化铜、氧化铁等等)不能直接跟水化合。

碱性氧化物跟酸性氧化物起反应,生成含氧酸的盐。例如:
\begin{fangchengshi}
    \ce{ CaO + SiO2 $\xlongequal{\Delta}$ \underset{\text{偏硅酸钙}\footnotemark}{\ce{CaSiO3}} }
\end{fangchengshi}
\footnotetext{硅酸(\ce{H4SiO4})失去一分子水,得到偏硅酸(\ce{H2SiO3})。\ce{CaSiO3} 是偏硅酸的钙盐。}


这个反应在炼铁的高炉和玻璃熔炉中都得到应用。

碱性氧化物可以由碱类加热分解、金属跟氧气直接化合或由盐类加热分解制得。例如:

\begin{fangchengshi}
    \ce{ Cu(OH)2 $\xlongequal{\Delta}$ CuO + H2O } \\[-.8em]
    \ce{ 2Mg + O2 $\xlongequal{\text{点燃}}$ 2MgO } \\[-.8em]
    \ce{ CaCO3 $\xlongequal{\text{高温}}$ CaO + CO2 ^ }
\end{fangchengshi}


\subsection{两性氧化物}

少数氧化物既能跟酸起反应生成盐和水,又能跟碱起反应生成盐和水,
这类氧化物叫做\zhongdian{两性氧化物}。例如:
\begin{fangchengshi}
    \ce{ ZnO + 2HCl = ZnCl2 + H2O } \\[-.8em]
    \ce{ ZnO + 2NaOH = \underset{\text{锌酸钠}}{\ce{Na2ZnO2}} + H2O } \\[-.8em]
    \ce{ Al2O3 + 6HCl = 2AlCl3 + 3H2O } \\[-.8em]
    \ce{ Al2O3 + 2NaOH = \underset{\text{偏铝酸钠}\footnotemark}{\ce{2NaAlO2}} + H2O }
\end{fangchengshi}
\footnotetext{铝酸(\ce{H3AlO3})失去一分子水,得到偏铝酸(\ce{HAlO2})。 \ce{NaAlO2} 是偏铝酸的钠盐。}



\begin{xiti}

\xiaoti{写出下列氧化物跟水起反应的化学方程式: \\
    \ce{K2O}, \ce{SO3}, \ce{BaO}, \ce{CO2}, \ce{CaO}。
}

\xiaoti{如果某一种氧化物不溶于水,怎样确定它是哪一类氧化物?}

\xiaoti{写出下列变化的化学方程式:}
\begin{xiaoxiaotis}

    \xxt{在硫酸铜溶液里加入氢氧化钠溶液,生成浅蓝色沉淀,过滤,将这种沉淀收集在坩埚中加热,生成黑色粉末。}

    \xxt{在氯化铁溶液里加入氢氧化钠溶液,生成红褐色沉淀,过滤,将这种沉淀收集在坩埚中加热,生成红色粉末。}

\end{xiaoxiaotis}

\xiaoti{写出下列各对物质间的反应的化学方程式:}
\begin{xiaoxiaotis}

    \xxt{氧化铁跟硫酸,}

    \xxt{氢氧化钾跟二氧化碳,}

    \xxt{氧化钙跟硫酐,}

    \xxt{氧化锌跟氢氧化钠,}

    \xxt{氧化铜跟盐酸。}

\end{xiaoxiaotis}

\end{xiti}

