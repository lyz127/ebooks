\fuxiti
\begin{xiaotis}

\xiaoti{下列的说法有没有错误,如有错误,应怎样改正?}
\begin{xiaoxiaotis}

    \xxt{凡能导电的物质都是电解质。}

    \xxt{由于电流通过氯化钠溶液时产生 \ce{Na+} 和 \ce{Cl-}, 所以氯化钠溶液能够导电。}

    \xxt{pH 值等于 7 的溶液呈中性,溶液中既没有 \ce{H+} 也没有 \ce{OH-}。}

\end{xiaoxiaotis}


\xiaoti{写出下面一系列物质变化的化学方程式,并注明反应类型。}
\begin{xiaoxiaotis}

    \xxt{\ce{ Zn -> ZnSO4 -> Zn(OH)2 -> ZnO -> Zn }}

    \xxt{\ce{ C -> CO2 -> CaCO3 -> CaO -> Ca(OH)2 -> CaCl2 }}

\end{xiaoxiaotis}

\xiaoti{有 4 包白色粉末,它们分别是碳酸钠、氯化钠、硫酸锌、硝酸铵,怎样鉴别它们?}

\xiaoti{有两个质量相等的锌片,使其中一片跟足量的稀硫酸起反应,另一片先煅烧成氧化锌,然后也跟足量的稀硫酸起反应。
    用两种方法制得的硫酸锌的质量相等吗?不用计算能回答吗?说明理由。
}

\xiaoti{某工厂利用废铁屑跟废硫酸起反应,制取硫酸亚铁。现有废硫酸 9.8 吨(含纯硫酸 $20\%$)
    跟足量的废铁屑起反应,可生产 \ce{FeSO4.7H2O} 多少吨?
}

\xiaoti{把 100 克铁棒放在硫酸铜溶液里,过一会儿取出,洗净、干燥,棒的质量增加到 103 克,问析出了多少克铜?\\
    (提示:铁棒的质量变化,既要考虑到铜的析出,使铁棒质量增加;也要考虑到铁跟硫酸铜的反应,使铁棒质量减少。)
}

\xiaoti{有一种物质不知道它是氯化钠(\ce{NaCl}) 还是氯化钾(\ce{KCl}), 通过实验分析之后,
    发现它含氯 $47.59\%$,这究竟是哪一种氯化物?
}

\xiaoti{把 50 克 $20\%$ 的氢氧化钠溶液,注入盛有 50 克 $20\%$ 的盐酸的烧杯里,然后滴入紫色石蕊试液,
    石蕊的颜色会有什么变化?为什么?
}

\xiaoti{有一种不知浓度的食盐水,称出 5 克这种溶液,注入烧杯,滴入硝酸银溶液到不能继续产生沉淀为止。
    把得到的氯化银沉淀充分干燥后称量,质量为 $0.1$ 克。求这种食盐水的百分比浓度。
}

\xiaoti{回答下列问题:}
\begin{xiaoxiaotis}

    \xxt{贮存或施用硫铵、硝铵或过磷酸钙时,为什么不能跟消石灰接触?}

    \xxt{草木灰为什么不宜在露天堆放?} % 为了减少页数,这里不换行。
    \xxt{为什么过磷酸钙最好跟农家肥料混和施用?}

\end{xiaoxiaotis}

\end{xiaotis}

