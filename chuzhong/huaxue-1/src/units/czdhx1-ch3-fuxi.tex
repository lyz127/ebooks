\fuxiti
\begin{xiaotis}

\xiaoti{回答下列问题:}
\begin{xiaoxiaotis}

    \xxt{为什么防毒面具里要用活性炭?}

    \xxt{有人说炉子上只要放一壶水就能防止煤气中毒,你认为这种说法对吗?为什么?}

\end{xiaoxiaotis}


\xiaoti{画出实验室收集氢气、氧气和二氧化碳的装置图。说明为什么要这样收集。}


\xiaoti{用什么化学方法可以鉴别下列各组物质?[其中(1)题要求用三种方法]}
\begin{xiaoxiaotis}

    \xxt{\ce{CO2} 和 \ce{CO},}
    \xxt{\ce{H2} 和 \ce{CO},}
    \xxt{\ce{H2} 和 \ce{CO2}。}

\end{xiaoxiaotis}


\xiaoti{用一氧化碳还原氧化锌可制得锌。}
\begin{fangchengshi}
    \ce{ ZnO + CO = Zn + CO2 }
\end{fangchengshi}

\begin{xiaoxiaotis}
    \xxt{用 12.2 克氧化锌跟 5.2 克一氧化碳反应制取锌,能制得锌多少克?
        生成二氧化碳多少克? 原料中有没有剩余的?如有,是什么?剩余多少克?
    }

    \xxt{把生成的二氧化碳通入足量澄清的石灰水里,能生成沉淀多少克?}

\end{xiaoxiaotis}


\xiaoti{下列反应是不是氧化-还原反应?如果是,指出各元素化合价的变化情况,并指出氧化剂和还原剂。}
\begin{xiaoxiaotis}

    \xxt{\ce{ 2HgO = 2Hg + O2 ^ }}

    \xxt{\ce{ 2Mg + O2 = 2MgO }}

    \xxt{\ce{ Fe + 2HCl = FeCl2 + H2 ^ }}

    \xxt{\ce{ H2 + Cl2 = 2HCl }}

    \xxt{\ce{ CaCl2 + Na2CO3 = CaCO3 v + 2NaCl }}

\end{xiaoxiaotis}


\end{xiaotis}

