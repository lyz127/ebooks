\newpage % 为了将 “图 5-5” 和相关说明放在一页,方便阅读。所以这里增加 \newpage 强制换页。
\section{单质、氧化物、酸、碱和盐的相互关系}\label{sec:5-9}

通过对酸、碱、盐、氧化物的性质和反应的学习,我们进一步认识到,这几类化合物是互相联系的和具有内部规律的。
现在把单质、氧化物、酸、碱、盐各类物质的互相联系和在一定条件下互相转变的基本规律,用图 \ref{fig:5-5} 来简单地表示。

\begin{figure}[htbp]
    \centering
    \begin{tikzpicture}[
    >=Stealth, scale=0.7, transform shape,
    element/.style={draw, shape=circle, double, align=center, minimum width=1.6cm, fill=white},
    box/.style={draw, shape=rectangle, minimum width=1.2cm},
    round/.style={rounded corners=4pt},
]
    \pgfmathsetmacro{\x}{7}
    \pgfmathsetmacro{\y}{3.5}
    \coordinate (E11) at (0,  0);
    \coordinate (E12) at (\x, 0);
    \coordinate (E21) at (0,  \y);
    \coordinate (E22) at (\x, \y);
    \coordinate (E31) at (0,  2*\y);
    \coordinate (E32) at (\x, 2*\y);
    \coordinate (E41) at (0,  3*\y);
    \coordinate (E42) at (\x, 3*\y);

    \draw [->,round] ($(E11)+(0,-0.5)$) -- ++(\x/2,0)  -- ++(0, 0.5);
    \draw [->,round] ($(E12)+(0,-0.5)$) -- ++(-\x/2,0) -- ++(0, 0.5)  coordinate(N);
    \draw (N) node[box,above]{两种新盐};

    \coordinate (X) at ($(E11)!0.6!(E22)$);
    \draw [->,round] (E11) -- (X) -- ++(0.9,0);
    \draw [->,rounded corners=8pt] (E22) -- (X) -- ++(0.9,0) coordinate(N);
    \draw (N) node[box,right]{酸和盐};

    \coordinate (X) at ($(E12)!0.6!(E21)$);
    \draw [->,round] (E12) -- (X) -- ++(-0.9,0);
    \draw [->,rounded corners=8pt] (E21) -- (X) -- ++(-0.9,0) coordinate(N);
    \draw (N) node[box,left]{碱和盐};

    \draw [->,round] ($(E21)+(0,-0.3)$) -- ++(\x/2,0)  -- ++(0, 0.5);
    \draw [->,round] ($(E22)+(0,-0.3)$) -- ++(-\x/2,0) -- ++(0, 0.5)  coordinate(N);
    \draw (N) node[box,above,minimum width=1.8cm]{盐和水};

    \coordinate (X) at ($(E21)!0.4!(E32)$);
    \draw [->,round] (E21) -- (X) -- ++(0,-0.5);
    \draw [->,round] (E32) -- (X) -- ++(0,-0.5);

    \coordinate (X) at ($(E22)!0.4!(E31)$);
    \draw [->,round] (E22) -- (X) -- ++(0,-0.5);
    \draw [->,round] (E31) -- (X) -- ++(0,-0.5);

    \coordinate (X) at ($(E31)!0.5!(E32)$);
    \draw [->,round] (E31) -- (X) -- ++(0, 1.5);
    \draw [->,round] (E32) -- (X) -- ++(0, 1.5)  coordinate(N);
    \draw (N) node[box,above,minimum width=1.8cm]{盐};

    \coordinate (X) at ($(E41)!0.5!(E42)$);
    \draw [->,round] (E41) -- (X) -- ++(0, -1.4);
    \draw [->,round] (E42) -- (X) -- ++(0, -1.4);

    \draw (E11) node (N11) [element]{盐};
    \draw (E12) node (N12) [element]{盐};
    \draw (E21) node (N21) [element]{碱};
    \draw (E22) node (N22) [element]{酸};
    \draw (E31) node (N31) [element]{碱性\\[-0.5em] 氧化物};
    \draw (E32) node (N32) [element]{酸性\\[-0.5em] 氧化物};
    \draw (E41) node (N41) [element, name=N41]{金属};
    \draw (E42) node (N42) [element]{非金属};

    \draw (N22) --  ++(1.5, 0) coordinate (X1);
    \draw (N41.north) -- ++(0, 0.5) -- ++(\x,0) -- ++(1.5, 0) coordinate (X2);
    \coordinate (X) at ($(X1)!0.3!(X2)$);
    \draw [->,round] (X1) -- (X) -- ++(0.9,0);
    \draw [->,round] (X2) -- (X) -- ++(0.9,0) coordinate (N);
    \draw (N) node[box,right]{盐和氢气};

    \draw (N11) -- ++(-1.5, 0) coordinate (X1);
    \draw (N41) -- ++(-1.5, 0) coordinate (X2);
    \coordinate (X) at ($(X1)!0.57!(X2)$);
    \draw [->,round] (X1) -- (X) -- ++(-0.9,0);
    \draw [->,round] (X2) -- (X) -- ++(-0.9,0) coordinate (N);
    \draw (N) node[box,left]{盐和金属};

    \draw [->] (N41.south) -- (N31.north);
    \draw [->] (N42.south) -- (N32.north);
    \draw [->] (N31.south) -- (N21.north);
    \draw [->] (N32.south) -- (N22.north);
    \draw [->] (N21.south) -- (N11.north);
    \draw [->] (N22.south) -- (N12.north);
\end{tikzpicture}


    \caption{各类物质的相互关系}\label{fig:5-5}
\end{figure}

由图 \ref{fig:5-5} 我们可以看出:

1. 各类物质的互相转变的关系

从纵的方面可以看出由单质到盐的转变关系。

现在以钙为例来说明由金属到盐的转变关系:
\begin{fangchengshi}
    \ce{ Ca -> CaO -> Ca(OH)2 -> CaSO4 }
\end{fangchengshi}

以碳为例来说明由非金属到盐的转变关系:
\begin{fangchengshi}
    \ce{ C -> CO2 -> H2CO3 -> CaCO3 }
\end{fangchengshi}

从横的方面可以看出金属跟非金属、碱跟酸等等的变化关系。例如:
\begin{fangchengshi}
    \ce{ Fe + S \xlongequal{\Delta} FeS } \\[-.8em]
    \ce{ Ca(OH)2 + H2SO4 = CaSO4 + 2H2O }
\end{fangchengshi}

2. 各类物质的主要化学性质

例如,可以看出,碱的主要化学性质是,碱能跟酸性氧化物起反应生成盐和水,
碱能跟酸起反应生成盐和水,碱能跟盐起反应生成另一种碱和另一种盐。

3. 制取某类物质的可能方法

例如,制取碱的可能方法就有:(1) 用碱性氧化物跟水起反应, (2) 盐跟碱起反应。

但是,事物本身是复杂的,看问题要从各方面去看,不能只从单方面看。
在工业生产中,要制取某种物质,除了首先考虑反应进行的可能性外,
还要根据原料、成本、设备等条件来选择最适当的方法。例如,
工业上制取氢氧化钠就不采用氧化钠跟水的反应(原料少、成本高),
而主要是采用电解食盐水的方法(高中化学课里要讲到),
少量的生产有时也采用盐(如碳酸钠)跟碱(如氢氧化钙)反应的方法。

\taolun 举出 5 种制备硫酸锌的方法。


\begin{xiti}

\xiaoti{完成下列化学方程式:}
\begin{xiaoxiaotis}

    \newcommand{\wenhao}{\hspace*{1em} ? \hspace*{1em}}
    \xxt{\ce{ MgO + \wenhao = Mg(NO3)2 + \wenhao}}

    \xxt{\ce{ Zn + \wenhao = Cu + \wenhao }}

    \xxt{\ce{ \wenhao + H3PO4 = Ca3(PO4)2 + \wenhao }}

    \xxt{\ce{ CuSO4 + \wenhao = Cu(OH)2 + \wenhao }}

    \xxt{\ce{ \wenhao + H2SO4 = H2 + \wenhao }}

\end{xiaoxiaotis}

\xiaoti{在氧化镁、三氧化硫、盐酸、熟石灰等四种物质里,哪两种物质放在一起会发生反应?写出反应的化学方程式。}

\xiaoti{举出 5 种制备氯化镁的方法。}

\xiaoti{要制取 8 千克 \ce{CuSO4.5H2O}, 需用 \ce{CuO} 和 $20\%$ \ce{H2SO4} 溶液各多少千克? \\
    (提示:先计算 8 千克 \ce{CuSO4.5H2O} 中所含 \ce{CuSO4} 的质量,再计算需用 \ce{CuO} 和 $20\%$ \ce{H2SO4} 溶液的质量。)
}

\end{xiti}

