\section{化合价和分子式}\label{sec:2-7}

我们知道,分子式是用元素符号表示物质分子组成的式子,
而化合价则反映了形成某种物质的不同元素原子间的个数关系,因此它们之间有密切的联系。
根据化合物中各元素正负化合价的代数和为零的原则,我们可以从分子式计算元素的化合价,
检查分子式的正误,以及应用化合价写出已知物质的分子式。

\liti 已知氧为 $-2$ 价,计算五氧化二磷里磷的化合价。

\jie 写出五氧化二磷的分子式 \ce{P2O5},根据上述原则:

$\text{磷的化合价} \times \text{磷原子数} + \text{氧的化合价} \times \text{氧原子数} = 0$

即 \quad $\text{磷的化合价} \times 2 + (-2) \times 5 = 0$

$\text{磷的化合价} = \dfrac{(-2) \times 5}{2} = +5$

答:在五氧化二磷里,磷是 $+5$ 价。


\liti 已知铝为 $+3$ 价,氧为 $-2$ 价,写出氧化铝的分子式。

\jie (1) 写出组成化合物的两种元素的符号,正价的写在左边,负价的写在右边。
$$ \ce{AlO} $$

(2) 求两元素的正、负化合价的绝对值的最小公倍数。
$$ 3 \times 2 = 6 $$

(3) 求各元素的原子数。
\begin{align*}
    & \dfrac{\text{最小公倍数}}{\text{正价数(或负价数)}} = \text{原子数} \\
    & \ce{Al}: \exdfrac{6}{3} = 2 \qquad \ce{O}: \exdfrac{6}{2} = 3
\end{align*}

(4) 把原子数写在各元素符号右下方,即得分子式。
$$ \ce{Al2O3} $$


(5) 检查分子式,当 $\text{正价总数} + \text{负价总数} = 0$ 时,分子式才算正确。
$$ (+3) \times 2 + (-2) \times 3 = +6 - 6 = 0 $$

答:氧化铝的分子式是 \ce{Al2O3}。

应该注意的是,只有确实知道有某种化合物存在,才能根据元素的化合价写出它的分子式。
切不可应用化合价任意写出实际上不存在的物质的分子式。


\begin{xiti}

\xiaoti{已知下列元素在氧化物中的化合价,写出它们的氧化物的分子式。\\
    $\overset{+2}{\ce{Ba}} \nsep
        \overset{+4}{\ce{S}} \nsep
        \overset{+2}{\ce{C}} \nsep
        \overset{+5}{\ce{N}} \nsep
        \overset{+2}{\ce{Mg}} \nsep
        \overset{+2}{\ce{Ca}}
    $。
}


\xiaoti{下列化合物中氧为 $-2$ 价,氢为 $-1$ 价,判断各化合物里其它元素的化合价。\\
    \ce{SO3} \nsep \ce{Na2O} \nsep \ce{CaCl2} \nsep \ce{AgCl} \nsep \ce{WO3}。
}

\xiaoti{指出下列七种化合物中错误的分子式,并加以改正。 \\
    水 \; \ce{HO} \nsep
    氧化钡 \; \ce{Ba2O5} \nsep
    氧化钙 \; \ce{CaO2} \nsep
    氯化铝 \; \ce{AlCl2} \nsep \\
    氯化氢 \; \ce{HCl} \nsep
    氯化镁 \; \ce{MgCl} \nsep
    氯化钾 \; \ce{KCl2}。
}


\xiaoti{标出下列物质中氯元素的化合价,并根据氯元素化合价的高低将它们排列一个次序。\\
    \ce{NaClO4}(高氯酸钠)\nsep
    \ce{NaCl} \nsep
    \ce{Cl2} \nsep
    \ce{KClO3} \nsep
    \ce{HClO}(次氯酸)。
}

\xiaoti{求出下列化合物中所含根的化合价。 \\
    \ce{KClO3} \nsep
    \ce{KMnO4} \nsep
    \ce{K2MnO4} \nsep
    \ce{Ca(OH)2} \nsep
    \ce{NH4NO3} \nsep
    \ce{(NH4)2SO4}。
}


\xiaoti{将下表中正价元素和负价元素(或根)所组成的化合物的分子式填在相应的空格内。\\
    \begin{tblr}{
        hline{1,2,6}=solid,
        vlines,
        columns={c},
        column{2-8}={3em},
    }
        \diagbox[height=3.5\line, trim=l]{负价\\[-.5em]元素\\[-.5em](或根)}{正价\\[-.5em]元素}
            & $\overset{+1}{\ce{H}}$
            & $\overset{+1}{\ce{Na}}$
            & $\overset{+2}{\ce{Mg}}$
            & $\overset{+3}{\ce{Al}}$
            & $\overset{+2}{\ce{Fe}}$
            & $\overset{+3}{\ce{Fe}}$
            & $\overset{+2}{\ce{Cu}}$ \\
        $\overset{-1}{\ce{Cl}}$ \\
        $\overset{-2}{\ce{O}}$ \\
        $\overset{-1}{\ce{(OH)}}$ \\
        $\overset{-2}{\ce{(SO4)}}$
    \end{tblr}
}

\end{xiti}

