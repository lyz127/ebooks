\xiaojie

一、空气和氧气

空气是一种重要的天然资源。空气是混和物,它的成分主要是氮气和氧气,还含有少量的惰性气体、二氧化碳和水蒸气等等。

氧气是一种化学性质比较活泼的气体,它能跟碳、硫、磷、铁、乙炔等许多物质发生化学反应,并放出热量。

我们常见的燃烧指的是可燃物跟空气里的氧气发生的一种发热发光的剧烈的氧化反应。

实验室里,常用给氯酸钾加热的方法来制取氧气。用二氧化锰作催化剂,可以加速这个反应的进行。
\begin{fangchengshi}
    \ce{ 2KClO3
         $\xlongequal[\Delta]{\ce{MnO2}}$
         2KCl + 3O2 ^
    }
\end{fangchengshi}

此外,还可以用给高锰酸钾加热的方法来制取氧气。
\begin{fangchengshi}
    \ce{ 2KMnO4 \xlongequal{\Delta} K2MnO4 + MnO2 + O2 ^}
\end{fangchengshi}

生成的氧气可以用排水法收集。


二、物理变化和化学变化

1. 物理变化:一种物质的分子没有变成其它物质的分子。如蒸发、凝固、扩散、破碎等等。

2. 化学变化:一种物质的分子变成其它物质的分子,但原子只是重新组合,没有变成别的原子。如燃烧、化合、分解等等。



三、化合反应和分解反应

1. 化合反应:由两种或两种以上的物质生成另一种物质的反应。

物质跟氧发生的化学反应,叫做氧化反应。

2. 分解反应:由一种物质生成两种或两种以上其它物质的反应。


四、物质的构成

分子、原子都是构成物质的微粒。有些物质是由分子构成的,有些物质是由原子直接构成的。

分子是保持物质化学性质的一种微粒。

原子是化学变化中的最小微粒。

原子具有复杂的结构,它的组成如下:

\vspace*{3em}
$$
    \text{原子} \smash[t]{\left\{ \begin{aligned}
        & \text{原子核} \smash{\left\{ \begin{aligned}
            &\text{质子:1 个质子带 1 个单位正电荷} \\
            &\text{中子:不带电}
        \end{aligned} \right.} \\[1.5em]
        &\text{电子:1 个电子带 1 个单位负电荷}
    \end{aligned} \right.}
$$

$$ \text{核电荷数} = \text{质子数} = \text{电子数} $$

原子具有一定的质量。原子的质量是采取不同原子的相对质量来表示的。
国际上是以一种碳原子的质量的 $1/12$ 作为标准,其它原子的质量跟它相比较所得的数值,
就是该种原子的原子量。一个分子中各原子的原子量的总和就是分子量。



五、物质的简单分类

元素是具有相同核电荷数(即质子数)的同一类原子\footnote{指质子数相同而中子数不同的多种原子,这些原子叫做该元素的同位素。}的总称。
由同种元素组成的纯净物叫做单质。由不同种元素组成的纯净物叫做化合物。

物质的简单分类可归纳如下:

\vspace*{4em}
$$
    \text{物质} \smash[t]{\left\{\begin{aligned}
        & \text{纯净物} \smash{\left\{ \begin{aligned}
            &\text{单质} \smash{\left\{ \begin{aligned}
                & \text{金属(如铁、铝)}\\
                & \text{非金属(如氧气、硫)}
            \end{aligned} \right.} \\[1.5em]
            &\text{化合物(如二氧化碳、氯酸钾)}
        \end{aligned} \right.} \\[2.0em]
        &\text{混和物(如空气)}
    \end{aligned} \right.}
$$



六、质量守恒定律

参加化学反应的各物质的质量总和,等于反应后生成的各物质的质量总和,这个规律叫做质量守恒定律。



七、化学用语

元素符号、分子式和化学方程式是用来表示元素、物质分子组成和物质的化学反应的化学用语,它们是学习化学的重要工具。

书写化学方程式要注意两个原则:一是必须以事实为根据,不能随便臆造;二是要遵循质量守恒定律。



八、化学计算

1. 根据分子式的计算

根据分子式,可以算出物质的分子量,还可以算出组成物质的各元素的质量比以及物质中某一元素的百分含量。


2. 根据化学方程式的计算

根据化学方程式可以计算出反应物、生成物各物质之间的质量比。


