\subsection{一元一次方程和它的解法}\label{subsec:3-3}

\begin{enhancedline}
看下面的方程:
$$ 4 + x = 7\nsep  3x + 5 = 7 - 2x\nsep  \dfrac{y - 2}{6} = \dfrac{y}{3} + 1 \juhao $$
这些方程,只含有一个未知数,并且未知数的次数是一次。这样的方程,叫做\zhongdian{一元一次方程}。

现在,我们运用方程同解原理来解一元一次方程。

\liti 解方程 $x - 7 = 5$。

\jie 根据\nameref{dl:tongjie-1},在上面方程的两边都加上 7,得
$$ x - 7 + 7 = 5 + 7 \juhao $$

合并同类项,得
$$ x = 12 \juhao $$

为了检验解方程时的计算有没有错误,可以把求得的解代替原方程中的未知数,检查方程左右两边的值是不是相等。

把 $x = 12$ 代入原方程,
$$ \zuobian = 12 - 7 = 5 \nsep \youbian  = 5\nsep  \zuobian = \youbian \douhao $$
所以 $x = 12$ 是原方程的解。

\liti 解方程 $7x = 6x - 4$。

\jie 根据\nameref{dl:tongjie-1},在上面方程的两边都减去 $6x$,得
$$ 7x - 6x = -4 \juhao $$

合并同类项,得
$$ x = -4 \juhao $$

检验:把 $x = -4$ 代入原方程,
\begin{gather*}
    \zuobian = 7 \times (-4) = -28, \\
    \youbian = 6 \times (-4) - 4 = -28, \\
    \zuobian = \youbian,
\end{gather*}
所以 $x = -4$ 是原方程的解。

在例 1、例 2 中,我们根据\nameref{dl:tongjie-1},分别作了下面的变形(图 \ref{fig:3-2}):
\begin{figure}[htbp]
    \centering
    \begin{tikzpicture}[>=Stealth,
    boxed/.style={draw, rectangle, inner sep=0.5em, minimum width=2.2em},
]
    \pgfmathsetmacro{\h}{1.2}
    \begin{scope}
        \node at (0.5, 2) {例 1};
        \draw (-0.3, \h) node {$x$} (0.5, \h) node[boxed] {$-7$} (1.5, \h) node {$=5$};
        \draw [->] (0.5, 0.9) -- (0.5, 0.6)  -| (1.5, 0.3);
        \draw (0, 0) node {$x = 5$} (1.5, 0) node[boxed] {$+7$};
    \end{scope}

    \begin{scope}[xshift=4cm]
        \node at (1.5, 2) {例 2};
        \draw (0, \h) node {$7x$} (1.5, \h) node{$=$} (2.6, \h) node[boxed, minimum width=3em] {$6x$} (3.6, \h) node {$-4$};
        \draw [->] (2.5, 0.9) -- (2.5, 0.6)  -| (1.2, 0.3);
        \draw (0, 0) node {$7x$} (1.0, 0) node[boxed] {$-6x$} (2.2, 0) node {$=$} (3.6, 0) node {$-4$};
    \end{scope}
\end{tikzpicture}

    \caption{}\label{fig:3-2}
\end{figure}
从变形前后的两个方程可以看出,这种变形,相当于把方程中的某一项改变符号后,从方程的一边移到另一边。
我们把这种变形叫做\zhongdian{移项}。必须牢记移项要变号。

在解方程时,我们常常利用移项,把方程中含有未知数的项移到等号的一边,
把不含未知数的项移到等号的另一边。象例 1 、例 2 中的两个方程可以这样来解:

(1) $x - 7 = 5$。

把 $-7$ 从方程的左边移到右边,得
$$ x = 5 + 7 \juhao $$

合并同类项,得
$$ x = 12\juhao $$

(2) $7x = 6x - 4$。

把 $6x$ 从方程的右边移到左边,得
$$ 7x - 6x = -4\juhao $$

合并同类项,得
$$ x = -4\juhao $$


\lianxi
\begin{xiaotis}

\xiaoti{根据\nameref{dl:tongjie-1}} 解下列方程,并写出检验:
\begin{xiaoxiaotis}

    \twoInLineXxt[18em]{$x + 6 = 7$;}{$4x = 3x - 2$。}

\end{xiaoxiaotis}

\xiaoti{下面的移项对不对?如果不对,错在哪里?应当怎样改正?}
\begin{xiaoxiaotis}

    \xxt{从 $7 + x = 13$,得到 $x = 13 + 7$;}

    \xxt{从 $5x = 4x + 8$,得到 $5x - 4x = 8$。}

\end{xiaoxiaotis}

\xiaoti{用移项解下列方程,并写出检验:}
\begin{xiaoxiaotis}

    \begin{tblr}{columns={18em, colsep=0pt}}
        \xxt{$x + 12 = 34$;} & \xxt{$x - 15 = 74$;} \\
        \xxt{$3x = 2x + 5$;} & \xxt{$7x - 3 = 6x$。}
    \end{tblr}

\end{xiaoxiaotis}

\end{xiaotis}
\lianxijiange

\liti 解下列方程:

\twoInLine{(1)$-5x = 70$;}{(2)$\dfrac{3}{5}x - 8 = 1$。}

\jie (1) $-5x = 70$。

根据\nameref{dl:tongjie-2},在上面方程的两边都除以 $-5$,得
$$ \dfrac{-5x}{-5} = \dfrac{70}{-5} \douhao $$
即
$$ x = -14 \juhao $$

检验:$\zuobian = -5 \times (-14) = 70$,$\youbian = 70$,
$$ \zuobian = \youbian \douhao $$
所以 $x = -14$ 是原方程的解。

(2)$\dfrac{3}{5}x - 8 = 1$。

移项,得
$$ \dfrac{3}{5}x = 1 + 8 \juhao $$

合并同类项,得
$$ \dfrac{3}{5}x = 9 \juhao $$

方程的两边都除以 $\dfrac{3}{5}$ \Big( 即乘以 $\dfrac{5}{3}$ \Big),得
$$ \dfrac{3}{5}x \cdot \dfrac{5}{3} = 9 \cdot \dfrac{5}{3} \douhao $$
即
$$ x = 15 \juhao $$

检验:$\zuobian = \dfrac{3}{5} \times 15 - 8 = 1$,$\youbian = 1$,
$$ \zuobian = \youbian \douhao $$
所以 $x = 15$ 是原方程的解。

在例 3 中,我们见到的方程 $-5x = 70$ 和 $\dfrac{3}{5}x = 9$,
都是形如 $ax = b$ (这里 $a$、 $b$ 是已知数,且 $a \neq 0$)的方程,
这的方叫做\zhongdian{最简方程}。利用\nameref{dl:tongjie-2},
在方程两边都除以未知数的系数,就能得出这类方程的解是 $x = \dfrac{b}{a}$。

\lianxi

解下列方程,并写出检验:
\begin{xiaotis}

    \begin{tblr}{columns={18em, colsep=0pt}}
        \xiaoti{$15x = 45$。}               & \xiaoti{$-2x = 30$。} \\
        \xiaoti{$-18x = -3$。}              & \xiaoti{$3.5x = 7$。} \\
        \xiaoti{$9x = 0$。}                 & \xiaoti{$32 = 8x$。}
    \end{tblr}

    \vspace*{-1em}
    \begin{tblr}{columns={18em, colsep=0pt}, stretch=1.5}
        \xiaoti{$\dfrac{x}{5} = 3$。}       & \xiaoti{$6x = 16 - 2x$。} \\
        \xiaoti{$\dfrac{2}{5}x - 4 = 12$。} & \xiaoti{$4 - \dfrac{3}{7}y = 13$。} \\
        \xiaoti{$13 = \dfrac{t}{2} + 3$。}  & \xiaoti{$\dfrac{1}{2} = \dfrac{1}{3} + 2x$。}
    \end{tblr}

\end{xiaotis}
\lianxijiange

\liti 解方程 $5x + 2 = 7x - 8$。

\jie 移项,得
$$ 2 + 8 = 7x - 5x \juhao$$

合并同类项,得
$$ 10 = 2x\douhao $$
即
$$ 2x = 10 \juhao $$

两边都除以 2,得
$$ x = 5 \juhao $$

(自己用口算检验。)

\liti 解方程 $2(x - 2) - 3(4x - 1) = 9(1 - x)$。

\jie 去括号,得
$$ 2x - 4 - 12x + 3 = 9 - 9x \juhao $$

移项,得
$$ 2x - 12x + 9x = 9 + 4 - 3 \juhao $$

合并同类项,得
$$ -x = 10 \juhao $$

两边都除以 $-1$,得
$$x = -10 \juhao $$

(自己检验,最好也用口算。)

\lianxi

解下列方程,并用口算检验:
\begin{xiaotis}

\xiaoti{$2x + 5 = 25 - 8x$。}

\xiaoti{$\dfrac{x}{2} - 7 = 5 + x$。}

\xiaoti{$5(x + 2) = 2(2x + 7)$。}

\xiaoti{$3(2y + 1) = 2(1 + y) + 3(y + 3)$。}

\end{xiaotis}

\lianxijiange

\liti  解方程 $\dfrac{5y - 1}{6} = \dfrac{7}{3}$。

解这个方程,可以先根据\nameref{dl:tongjie-2},在方程两边都乘以各分母的最小公倍数
(本题中 6 与 3 的最小公倍数是 6),把方程里各分母都去掉。

\jie 去分母,得
$$ \dfrac{5y - 1}{6} \times 6 = \dfrac{7}{3} \times 6 \douhao $$
即
$$ 5y - 1 = 14 \juhao$$

移项,得
$$ 5y = 14 + 1 \juhao $$

合并同类项,得
$$ 5y = 15 \juhao $$

两边都除以 5, 得
$$ y = 3 \juhao $$

\liti 解方程 $\dfrac{2x - 1}{3} - \dfrac{10x + 1}{6} = \dfrac{2x + 1}{4} - 1$。

本题中各分母 3、6、4 的最小公倍数是 12 。

\jie 去分母,得
$$ 4(2x - 1) - 2(10x + 1) = 3(2x + 1) - 12 \juhao $$

去括号,得
$$ 8x - 4 - 20x - 2 = 6x + 3 - 12 \juhao $$

移项,得
$$ 8x - 20x - 6x = 3 - 12 + 4 + 2 \juhao $$

合并同类项,得
$$ -18x = -3 \juhao $$

两边都除以 $-18$,得
$$ x = \dfrac{1}{6} \juhao $$

\lianxi

解下列方程:
\begin{xiaotis}

    \begin{tblr}{columns={18em, colsep=0pt}}
        \xiaoti{$\dfrac{7x - 5}{4} = \dfrac{3}{8}$。}   & \xiaoti{$\dfrac{3 - x}{2} = \dfrac{x - 4}{3}$。} \\
        \xiaoti{$\dfrac{2x - 1}{6} - \dfrac{5x + 1}{8} = 1$。}  & \xiaoti{$\dfrac{2}{7} (3x + 7) = 2 - 1.5x$。} \\
    \end{tblr}

    \xiaoti{$\dfrac{30}{100}x + \dfrac{70}{100} (200 - x) = 200 \times \dfrac{54}{100}$。}

\end{xiaotis}
\lianxijiange


解一元一次方程的一般步骤是:

\jiange
\framebox{\begin{minipage}{0.93\textwidth}
    \zhongdian{1. 去分母;}

    \zhongdian{2. 去括号;}

    \zhongdian{3. 移项;}

    \zhongdian{4. 合并同类项,化成最简方程 $ax = b \; (a \neq 0)$ 的形式;}

    \zhongdian{5. 方程两边都除以未知数的系数,得出方程的解 $x = \dfrac{b}{a}$。}
\end{minipage}}\jiange


解方程时,上面的有些步骤也可能用不到。要根据方程的形式灵活安排解题步骤。
熟练后,解方程的步驟(包括检验) 可以简化。

\liti 解方程 $\dfrac{x}{0.7} - \dfrac{1.7 - 2x}{0.3} = 1$。

这个方程的分母含有小数,可以先把分母化成整数再解。

\jie 原方程可以化成
$$ \dfrac{10x}{7} - \dfrac{17 - 20x}{3} = 1 \juhao $$

去分母,得
$$ 30x - 7(17 - 20x) = 21 \juhao $$

去括号、移项与合并同类项,得
$$ 170x  = 140 \juhao $$

两边都除以 170,得
$$ x = \dfrac{14}{17} \juhao $$

\liti 在梯形面积公式 $S = \dfrac{1}{2}(a + b)h$ 中,
已知 $S = 120$, $b = 18$, $h = 8$,求 $a$。

\jie  把 $S = 120$, $b = 18$, $h = 8$ 代入公式中,得
$$ 120 = \dfrac{1}{2} \cdot (a + 18) \cdot 8 \juhao $$

解这个方程:
\begin{gather*}
    30 = a + 18 \juhao \\
    a = 12 \juhao
\end{gather*}


\lianxi
\begin{xiaotis}

\xiaoti{解方程:}
\begin{xiaoxiaotis}

    \xxt{$2.4 - \dfrac{y - 4}{2.5} = \dfrac{3}{5}y$;}

    \xxt{$\dfrac{x - 2}{0.2} - \dfrac{x + 1}{0.5} = 3$。}

\end{xiaoxiaotis}

\xiaoti{在梯形面积公式 $S = \dfrac{1}{2}(a + b)h$ 中,}
\begin{xiaoxiaotis}

    \xxt{$S = 30$, $a = 6$, $h = 4$,求 $b$;}

    \xxt{$S = 60$, $a = 8$, $b = 12$,求 $h$。}

\end{xiaoxiaotis}
\end{xiaotis}

\end{enhancedline}

