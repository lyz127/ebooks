\subsection{有理数乘法法则}\label{subsec:1-10}

我们来看下列问题:

问题一 \quad 水池的水位平均每小时上升 3 厘米,2 小时上升了多少厘米?

我们知道,这个问题可以用乘法来解答:
\begin{align}
    3 \times 2 = 6 (\limi) \juhao  \tag{1}
\end{align}

水位上升了 6 厘米(图 \ref{fig:1-18})。

\begin{figure}[htbp]
    \centering
    \begin{minipage}{8cm}
    \centering
    \begin{tikzpicture}[>=Stealth, scale=0.9]
    \draw (0, 5.2) -- (0, 0) -- (4, 0);
    \fill [pattern = dots] (0, 0) rectangle (4, 5);
    \draw [dashed] (0, 3.5) -- (4, 3.5) node [right] {原水面位置};
    \draw (0, 5) -- (4, 5) node [right, align=center] {2 小时后\\水面位置};
    \draw[decorate,decoration={brace,mirror,amplitude=0.2cm}] (-0.1, 5) -- (-0.1, 3.5)
        node [pos=0.5, left=1em] {6 厘米};
    \draw [fill=white] (2.1, 3.6) pic {arrow={.2}{.4}{.4}{.2}};
\end{tikzpicture}

    \caption{}\label{fig:1-18}
    \end{minipage}
    \qquad
    \begin{minipage}{8cm}
    \centering
    \begin{tikzpicture}[>=Stealth, scale=0.9]
    \draw (0, 5.2) -- (0, 0) -- (4, 0);
    \fill [pattern = dots] (0, 0) rectangle (4, 2);
    \draw [dashed] (0, 3.5) -- (4, 3.5) node [right] {原水面位置};
    \draw (0, 2) -- (4, 2) node [right, align=center] {2 小时后\\水面位置};
    \draw[decorate,decoration={brace,mirror,amplitude=0.2cm}] (-0.1, 3.5) -- (-0.1, 2)
        node [pos=0.5, left=1em] {6 厘米};
    \draw [fill=white] (2.1, 3.4) pic [rotate=180] {arrow={.2}{.4}{.4}{.2}};
\end{tikzpicture}

    \caption{}\label{fig:1-19}
    \end{minipage}
\end{figure}

问题二 \quad 水池的水位平均每小时下降 3 厘米,2 小时下降了多少厘米?

显然,结果是水位下降了 6 厘米(图 \ref{fig:1-19} )。

如果我们象前面讲过的那样,上升的量用正数表示,下降的量用负数表示,
仍用乘法来解答这个问题,那么算式就应该写成:
\begin{align}
    (-3) \times 2 = -6 (\limi) \juhao  \tag{2}
\end{align}

把它和 (1) 式对比,可以看出,当把一个因数 “3” 换成它的相反数 “$-3$” 时,
所得的积是原来的积 “6” 的相反数 “$-6$” 。这就启发我们规定:

把一个因数换成它的相反数,所得的积是原来的积的相反数。

按照这个规定,我们来计算:
$$ 3 \times (-2) = ? $$

把它和 (1) 式对比,这里把一个因数 “2” 换成了它的相反数 “$-2$”,由上面的规定,
所得的积是原来的积 “6” 的相反数 “$-6$”, 即
\begin{align}
    3 \times (-2) = -6 \juhao  \tag{3}
\end{align}

最后, 我们来计算:
$$ (-3) \times (-2) = ? $$

把它和 (2) 式对比,这里把一个因数 “2” 换成了它的相反数 “$-2$” ,
所得的积是原来的积 “$-6$” 的相反数 “6”,即
\begin{align}
    (-3) \times (-2) = 6 \juhao  \tag{4}
\end{align}

看上面 (1) ~ (4) 式,积的符号与因数的符号有什么关系?积的绝对值与因数的绝对值有什么关系?

此外,我们将一个因数换成零时,所得的积也是零。 如 $(-3) \times 0 = 0$。

综合上面各种情况,得到有理数乘法的法则:\jiange

\framebox{\begin{minipage}{0.93\textwidth}
    \zhongdian{两数相乘,同号得正,异号得负,并把绝对值相乘。}

    \zhongdian{任何数同零相乘,都得零。}
\end{minipage}}


\lianxi

(口答)\begin{tblr}[t]{columns={8em, l, $$}}
    (+6) \times (+9),  &  (+6) \times (-9),  &  (-6) \times (-9),  &  (-6) \times (+9), \\
    (-6) \times (+1),  &  (+6) \times (+1),  &  (-9) \times (-1),  &  (+9) \times (-1) \juhao
\end{tblr}

\lianxijiange

\zhuyi 一个数同 $+1$ 相乘,得原数;一个数同 $-1$ 相乘,得原数的相反数。

\lianxi
\begin{xiaotis}
\begin{enhancedline}

\xiaoti{计算:\\
    \begin{tblr}[t]{columns={9em, l, $$}}
        (+18) \times (-5),  &  (-37) \times (-3),  &  (-25) \times (+4.8),  &  (-0.1) \times (-1.5), \\
        \left(+\dfrac{4}{7}\right) \times \left(-\dfrac{1}{2}\right),
            &  \left(-\dfrac{5}{12}\right) \times \left(-\dfrac{8}{15}\right),
            &  \left(+1\dfrac{1}{2}\right) \times \left(-\dfrac{2}{3}\right),
            &  (-8) \times 0 \juhao
    \end{tblr}
}


\xiaoti{温度每升高 1 ℃ , 温度计内水银柱就升高 2 毫米。温度升高 12 ℃ 时,水银柱升高多少? 温度升高 $-15$ ℃ 时呢?}
\end{enhancedline}

\end{xiaotis}

想一想,下列各式的积是正的还是负的?
\begin{align*}
    & (-2) \times (+3) \times (+4) \times (+5) ; \\
    & (-2) \times (-3) \times (+4) \times (+5) ; \\
    & (-2) \times (-3) \times (-4) \times (+5) ; \\
    & (-2) \times (-3) \times (-4) \times (-5) \juhao
\end{align*}

你能从中找出规律吗?

\zhongdian{几个不等于零的有理数相乘,积的符号由负因数的个数决定。
    当负因数有奇数个时,积为负;
    当负因数有偶数个时,积为正。
}

\begin{enhancedline}
\liti 计算 $(-3) \times \left(+\dfrac{5}{6}\right) \times \left(-1\dfrac{4}{5}\right) \times \left(-\dfrac{1}{4}\right)$。

\jie $\begin{aligned}[t]
        & (-3) \times \left(+\dfrac{5}{6}\right) \times \left(-1\dfrac{4}{5}\right) \times \left(-\dfrac{1}{4}\right) \\
    ={} & -3 \times \dfrac{5}{6} \times \dfrac{9}{5} \times \dfrac{1}{4} = -1\dfrac{1}{8} \juhao
\end{aligned}$

\zhuyi 几个不等于零的有理数相乘,首先确定积的符号,然后把绝对值相乘。
\end{enhancedline}

想一想,怎样计算
$$ (+7.8) \times (-8.1) \times 0 \times (-19.6) ? $$

\zhongdian{几个有理数相乘,有一个因数为零,积就为零。}


\liti 计算:

(1) \; $8 + 5 \times (-4)$;

(2) \; $(-3) \times (-7) -9 \times (-6)$。

\jie (1) \; $8 + 5 \times (-4) = 8 + (-20) = -12$;

(2) \; $(-3) \times (-7) -9 \times (-6) = 21 - (-54) = 75$。

\zhuyi 含加减乘除的算式中,没有括号指明运算顺序时,要先算乘除,后算加减。


\lianxi

计算:

\begin{xiaotis}
\setcounter{cntxiaoti}{0}

\xiaoti{$(-5) \times (+8) \times (-7) \times (-0.25)$}。

\xiaoti{$(-6) - (-3) \times \dfrac{1}{3}$}。

\xiaoti{$(-1) \times (-8) + 3 \times (-2)$}。

\xiaoti{$1 + 0 \times (-1) - (-1) \times (-1) - (-1) \times 0 \times (-1)$}。

\xiaoti{$3 \times 5 \times 7 - (-3) \times (-5) \times (-7) - (-3) \times (-5) \times 7 + 3  \times (-5) \times 7$}。

\end{xiaotis}




