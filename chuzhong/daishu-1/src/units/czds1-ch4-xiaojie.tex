\xiaojie

一、本章主要内容是不等式和它的性质、不等式的解集和不等式同解原理、一元一次不等式和它的解法。

二、现实世界中的同类量(如长度与长度、时间与时间)之间,有相等关系,也有不等关系。
相等关系用等式来表示,不等关系用不等式来表示。两个可以比较大小的量 $a$ 与 $b$ 之间,
在 $a < b$, $a = b$, $a > b$ 三个式子之中,必定有一个成立并且只有一个成立。

三、不等式与方程的同解原理,以及一元一次不等式与一元一次方程的解法步驟和解的情况,可以对比如下:\\

\begin{tblr}{
    column{1}={2em},
    column{2}={16em},
    column{3}={16em},
    row{1}={c},
    cell{2,5}{1}={1em, c},
    row{5}={m},
    hlines, vlines,
}
    & 方程 & 不等式 \\
    \SetCell[r=3]{} 同解原理
        & 两边都加上(或都减去)同一个数或同一个整式,所得的方程与原方程同解。
        & 两边都加上(或都减去)同一个数或同一个整式,所得的不等式与原不等式同解。 \\
    & \SetCell[r=2]{} 两边都乘以(或都除以)同一个不等于零的数,所得的方程与原方程同解。
        & 两边都乘以(或都除以)同一个正数,所得的不等式与原不等式同解。 \\
    & & 两边都乘以(或都除以)同一个负数,并且把不等号改变方向,所得的不等式与原不等式同解。\\
    解法步骤
        & {解一元一次方程:\\
            1. 去分母;\\
            2. 去括号;\\
            3. 移项; \\
            4. 合并同类项;\\
            5. 方程两边都除以未知数的系数。\vspace*{5em}}
        & {解一元一次不等式: \\
            1. 去分母; \\
            2. 去括号; \\
            3. 移项; \\
            4. 合并同类项; \\
            5. 不等式两边都除以未知数的系数。 \\
            在上面的步骤 1 和步骤 5 中,如果乘数或除数是负数,要把不等号改变方向。} \\
    解的情况
        & 一元一次方程只有一个解。
        & 一元一次不等式的解集含有无限多个数。
\end{tblr}


