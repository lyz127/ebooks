\subsection{有理数的混合运算}\label{subsec:1-14}

一个算式里含有加、减、乘、除、乘方等几种运算时,要接照下面的顺序进行演算:

\framebox{\begin{minipage}{0.93\textwidth}
    \zhongdian{先算乘方,再算乘除,最后算加减。如果有括号,就先算括号里面的。}
\end{minipage}}\jiange

\begin{enhancedline}

\liti 计算 $-1\dfrac{1}{2} + \dfrac{1}{3} + \dfrac{5}{6} - 1\dfrac{1}{4}$。

\jie $\begin{aligned}[t]
        & -1\dfrac{1}{2} + \dfrac{1}{3} + \dfrac{5}{6} - 1\dfrac{1}{4} \\
    ={} & -1 - \dfrac{1}{2} + \dfrac{1}{3} + \dfrac{5}{6} - 1 - \dfrac{1}{4} \\
    ={} & -1 -1 + \dfrac{-6 + 4 + 10 - 3}{12} \\
    ={} & -2 + \dfrac{5}{12} \\
    ={} & -1\dfrac{7}{12} \juhao
\end{aligned}$

\lianxi
计算:
\begin{xiaotis}

\xiaoti{$1.6 + 5.9 - 25.8 + 12.8 - 7.4$。}

\xiaoti{$-5\dfrac{1}{2} + 8\dfrac{2}{3} - 12\dfrac{5}{6}$。}

\end{xiaotis}
\lianxijiange

\liti 计算 $2\dfrac{1}{5} \times \left(\dfrac{1}{3} - \dfrac{1}{2}\right) \times \dfrac{3}{11} \div 1\dfrac{1}{4}$。

\jie $\begin{aligned}[t]
        & 2\dfrac{1}{5} \times \left(\dfrac{1}{3} - \dfrac{1}{2}\right) \times \dfrac{3}{11} \div 1\dfrac{1}{4} \\
    ={} & \dfrac{11}{5} \times \left(-\dfrac{1}{6}\right) \times \dfrac{3}{11} \times \dfrac{4}{5} \\
    ={} & -\dfrac{2}{25} \juhao
\end{aligned}$


\lianxi
计算:
\begin{xiaotis}

\xiaoti{$-2.5 \times (-4.8) \times 0.09 \div (-0.27)$。}

\xiaoti{$2\dfrac{1}{4} \times \left(-\dfrac{6}{7}\right) \div \left(\dfrac{1}{2} - 2\right)$。}

\end{xiaotis}
\lianxijiange
\end{enhancedline}


\liti 计算 $-10 + 8 \div (-2)^2 - (-4) \times (-3)$。

\jie $\begin{aligned}[t]
        & -10 + 8 \div (-2)^2 - (-4) \times (-3) \\
    ={} & -10 + 8 \div 4 - 12 \\
    ={} & -10 + 2 -12 = -20 \juhao
\end{aligned}$


\lianxi
计算:
\begin{xiaotis}

\xiaoti{$-9 + 5 \times (-6) - (-4)^2 \div (-8)$。}

\xiaoti{$2 \times (-3)^3 - 4 \times (-3) + 15$。}

\end{xiaotis}

