\subsection{数轴}\label{subsec:1-2}

生活中,常常在一条直线上画出刻度,用这些刻度来表示量的大小。
例如,利用温度计上的刻度来表示温度的高低:零上一个刻度,表示 1 ℃;零下两个刻度,表示 $-2$ ℃;……。
又如,用直尺上的刻度表示长度的大小,用秤杆上的金星表示重量的大小,等等。

同样,我们可以在一条直线上画出点,用这些点表示正数和负数,方法如下。

如图 \ref{fig:1-5},画一条直线(一般画水平的直线),在这条直线上任取一点 $O$ 作为\zhongdian{原点},用这点表示零。
规定这条直线的一个方向为正方向(一般取从左到右的方向),那么相反的方向就是负方向。
再任意取一条线段的长度作为单位长度。

\begin{figure}[htbp]
    \centering
    \begin{tikzpicture}[>=Stealth]
    \draw [->] (-7,0) -- (7.2,0);
    \foreach \x in {-7,...,+6} {
        \draw (\x,0.4) -- (\x,0);
        \foreach \tmp in {1,...,9} {
            \draw (\x+\tmp/10, 0.2) -- (\x+\tmp/10, 0);
        }
        \draw (\x+0.5, 0.3) -- (\x+0.5, 0);
    }

    \foreach \x in {-6,...,0} {
        \node at (\x, -0.3) {$\x$};
    }

    \foreach \x in {1,...,6} {
        \node at (\x, -0.3) {$+\x$};
    }


    \foreach \pos/\text in {-4/B, -1.5/D, 0/O, 2.4/C, 5/A} {
        \filldraw [fill=black] (\pos, 0) circle (0.05);
        \node at (\pos, 0.6) {$\text$};
    }

    \draw (1, -0.8) -- (1, -1.2)
          (2, -0.8) -- (2, -1.2)
          (2, -1) -- (1, -1)
          node [left] {单位长度};
\end{tikzpicture}

    \caption{}\label{fig:1-5}
\end{figure}

象这样规定了原点、正方向和单位长度的直线叫做\zhongdian{数轴}。

\begin{enhancedline}
于是,$+5$ 就可用数轴上原点右边 5 个单位的 $A$ 点表示,
$-4$ 可用原点左边 4 个单位的 $B$ 点表示,
$+2.4$ 可用原点右边 2.4 个单位的 $C$ 点表示,
$-1\dfrac{1}{2}$ 可用原点左边 $1\dfrac{1}{2}$ 个单位的 $D$ 点表示,等等。

这样,所有的有理数,都可以用数轴上的点表示。

\liti[0] 在数轴上记出下列各数:

\hspace{2em} $+1$, $-5$, $-2.5$, $+4\dfrac{1}{2}$, $0$。
\end{enhancedline}

\jie
\begin{figure}[htbp]
    \centering
    \begin{tikzpicture}[>=Stealth]
    \draw [->] (-7,0) -- (7.2,0);
    \foreach \x in {-6,...,6} {
        \draw (\x,0.3) -- (\x,0);
    }

    \foreach \x in {-6.5,...,5.5} {
        \draw (\x,0.2) -- (\x,0);
    }

    \foreach \pos/\text in {-5/-5, -2.5/-2.5, 0/0, 1/+1, 4.5/+4\frac{1}{2}} {
        \filldraw [fill=black] (\pos, 0) circle (0.05) node [below] {$\text$};
    }
\end{tikzpicture}

    \caption{}\label{fig:1-6}
\end{figure}


\lianxi
\begin{xiaotis}

\xiaoti{(口答)下面数轴上的 $A$ 点表示什么?$B$、$C$、$D$、$E$ 各点呢?}
\begin{figure}[htbp]
    \centering
    \begin{tikzpicture}[>=Stealth]
    \draw [->] (-7,0) -- (7.2,0);
    \foreach \x in {-6,...,6} {
        \draw (\x,0.3) -- (\x,0);
    }

    \foreach \x in {-5.5,...,5.5} {
        \draw (\x,0.2) -- (\x,0);
    }

    \foreach \x in {-5,...,5} {
        \node at (\x, -0.3) {$\x$};
    }

    \foreach \pos/\text in {-4/B, -2.5/D, -1/E, 2/A, 4.5/C} {
        \filldraw [fill=black] (\pos, 0) circle (0.05) +(0, 0.3) node [above] {$\text$};
    }
\end{tikzpicture}

    \caption{}\label{fig:1-7}
\end{figure}

\xiaoti{画一条数轴, 并在数轴上记出下列数:}
$$ +6\douhao 1.5\douhao -6\douhao 2\frac{1}{2}\douhao 0\douhao 0.5\douhao -2\frac{1}{2}\juhao $$

\end{xiaotis}
