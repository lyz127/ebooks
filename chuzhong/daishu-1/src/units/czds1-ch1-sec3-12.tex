\subsection{有理数除法法则}\label{subsec:1-12}

和小学学过的除法意义相同,有理数除法是有理数乘法的逆运算,
有理数除法就是已知两个因数的积与其中的一个因数,求另一个因数的运算。

我们看下面几种情况的结果。

从 \quad $(+3) \times (+2) = +6$,可以得到

\hspace*{2em} $(+6) \div (+2) = +3$。

从 \quad $(+3) \times (-2) = -6$,可以得到

\hspace*{2em} $(-6) \div (-2) = +3$。

从 \quad $(-3) \times (+2) = -6$,可以得到

\hspace*{2em} $(-6) \div (+2) = -3$。

从 \quad $(-3) \times (-2) = +6$,可以得到

\hspace*{2em} $(+6) \div (-2) = -3$。

此外,从 $0 \times (-2) = 0$,可以得到

\hspace*{4em} $0 \div (-2) = 0$。

综合以上各种情况,得到有理数除法的法则:\jiange

\framebox{\begin{minipage}{0.93\textwidth}
    \zhongdian{两数相除,同号得正,异号得负,并把绝对值相除。}

    \zhongdian{零除以任何一个不等于零的数都得零。}
\end{minipage}}\jiange

\zhongdian{\zhuyi 零不能作除数。}


\begin{enhancedline}

\lianxi
\begin{xiaotis}

\xiaoti{(口答)\begin{tblr}[t]{columns={10em, l, $$}}
    (-18) \div (+6),  &  (-63) \div (-7),  &  (+36) \div (-3), \\
    (+32) \div (-8),  &  (-54) \div (-9),  &  0 \div (-8) \juhao
\end{tblr}}

\xiaoti{(口答)\begin{tblr}[t]{columns={10em, l, $$}}
    (+84) \div (-7),    &  (-96) \div (-16),  &  (-6.5) \div (+0.13), \\
    (+8) \div (-0.02),  &  \left(-\dfrac{3}{5}\right) \div \left(-\dfrac{2}{5}\right),  &  \left(-\dfrac{7}{8}\right) \div \left(+\dfrac{3}{4}\right) \juhao
\end{tblr}}

\end{xiaotis}


把 $\dfrac{3}{4}$ 的分子分母颠倒,就得到一个数 $\dfrac{4}{3}$。这就是 $1$ 除以 $\dfrac{3}{4}$ 的商。
一般地,$1$ 除以一个数的商,叫做这个数的\zhongdian{倒数}。

例如,$\dfrac{3}{4}$ 的倒数是 $\dfrac{4}{3}$, $\dfrac{4}{3}$ 的倒数是 $\dfrac{3}{4}$。
又如,$-\dfrac{3}{4}$ 的倒数是 $-\dfrac{4}{3}$, $2$ 的倒数是 $\dfrac{1}{2}$,$-2$ 的倒数是 $-\dfrac{1}{2}$。

\zhongdian{\zhuyi 零没有倒数。}(为什么?)

\lianxi

(口答)说出下列各数的倒数:

\hspace*{2em} $\dfrac{5}{6}$,\quad  $-\dfrac{4}{7}$,\quad $0.2$,\quad $\dfrac{1}{3}$,\quad $-5$,\quad $1\dfrac{1}{3}$,\quad $-1$。

\lianxijiange

\zhongdian{一个数除以另一个数,等于被除数乘以除数的倒数。}

把除数变为它的倒数,除法就可以转化为乘法。

\liti 计算 $-3.5 \div \dfrac{7}{8} \times \left(-\dfrac{3}{4}\right)$。

\jie $-3.5 \div \dfrac{7}{8} \times \left(-\dfrac{3}{4}\right) = \dfrac{7}{2} \times \dfrac{8}{7} \times \dfrac{3}{4} = 3$。

\liti 计算:

(1) \; $8 + 32 \div (-4)$;

(2) \; $-9 \cdot (-2) - 15 \div (-3)$。

\jie (1) \; $8 + 32 \div (-4) = 8 + (-8) = 0$;

(2) \; $-9 \cdot (-2) - 15 \div (-3) = 18 - (-5) = 23$。


\lianxi

计算:

\begin{xiaotis}
\setcounter{cntxiaoti}{0}

\xiaoti{$-0.25 \div \left(-\dfrac{2}{3}\right) \times \left(-1\dfrac{3}{5}\right)$。}

\xiaoti{$14 + 56 \div (-7)$。}

\xiaoti{$-81 \div 3 + 27 \div (-9)$。}

\end{xiaotis}

\end{enhancedline}
