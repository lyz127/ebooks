\subsection{平方表和立方表}\label{subsec:1-16}

求 $0.2468^2$、$(-51.97)^3$ 这样的平方数或立方数,直接计算,比较麻烦;
查平方表或立方表,就能很快得出结果。

下面是四位平方表的一部分,能查到(最多)有四个有效数字的数的平方,查得的结果有四个有效数字。

\begin{table}[H]
\begin{tblr}{vlines,
    hline{1, 7} = {1pt, solid},
    hline{2} = {solid},
    vline{1, 21} = {1pt, solid},
    vline{12} = {1}{-}{},
    vline{12} = {2}{-}{},
    columns={colsep=2.5pt, c, $$},
}
    N   & 0     & 1     & 2     & 3     & 4     & 5     & 6     & 7     & 8     & 9     & 1 & 2  & 3  & 4  & 5  & 6  & 7  & 8  & 9 \\
    2.0 & 4.000 & 4.040 & 4.080 & 4.121 & 4.162 & 4.203 & 4.244 & 4.285 & 4.326 & 4.368 & 4 & 8  & 12 & 16 & 20 & 25 & 29 & 33 & 37 \\
    2.1 & 4.410 & 4.452 & 4.494 & 4.537 & 4.580 & 4.623 & 4.666 & 4.709 & 4.752 & 4.796 & 4 & 9  & 13 & 17 & 21 & 26 & 30 & 34 & 39 \\
    2.2 & 4.840 & 4.884 & 4.928 & 4.973 & 5.018 & 5.063 & 5.108 & 5.153 & 5.198 & 5.244 & 4 & 9  & 13 & 18 & 22 & 27 & 32 & 35 & 40 \\
    2.3 & 5.290 & 5.336 & 5.382 & 5.429 & 5.476 & 5.523 & 5.570 & 5.617 & 5.664 & 5.712 & 5 & 9  & 14 & 19 & 23 & 28 & 33 & 38 & 42 \\
    2.4 & 5.760 & 5.808 & 5.856 & 5.905 & 5.954 & 6.003 & 6.052 & 6.101 & 6.150 & 6.200 & 5 & 10 & 15 & 20 & 24 & 29 & 34 & 39 & 44 \\
\end{tblr}
\end{table}

\liti 查表求 $2.46^2$。

查 2.4 所在横行和 6 所在的直列,交叉处得 6.052 。

\begin{table}[H]
\begin{tblr}{vlines,
    hline{1, 3} = {1pt, solid},
    hline{2} = {solid},
    vline{1, 5} = {1pt, solid},
    columns={colsep=2.5pt, c},
    column{2, 4} = {l, 10em},
    row{2} = {valign=b},
}
    N   &     & 6     &  \\
    2.4
        & \hspace*{2em}\begin{tikzpicture}[>=Stealth]
            \draw [->] (0, 0) -- (0.5, 0);
           \end{tikzpicture}
        & {\begin{tikzpicture}[>=Stealth]
            \draw [->] (0, 0) -- (0, -0.4);
           \end{tikzpicture} \\[-0.5em]
           6.052}
        &  \\
\end{tblr}
\end{table}

\jie $2.46^2 = 6.052$。

查表所得的结果,虽然大都是近似值,一般仍用等号。


\lianxi
(口答)查表求 $2.29^2$,$2.15^2$,$2.07^2$,$2.3^2$。
\lianxijiange

\liti 查表求 $2.468^2$。

查得 $2.46^2 = 6.052$,再查这一横行和修正值表中 8 所在的直列,在交叉处得 39,
它表示 6.052 最后两位上应加 39, 即 6.052 上 0.039,得 6.091 。

\begin{table}[H]
\begin{tblr}{vlines,
    hline{1, 3} = {1pt, solid},
    hline{2} = {solid},
    vline{1, 7} = {1pt, solid},
    columns={colsep=2.5pt, c},
    column{2, 4, 6} = {l, 6em},
    row{2} = {valign=b},
}
    N   &     & 6     &  & 8  & \\
    2.4
        & \hspace*{2em}\begin{tikzpicture}[>=Stealth]
            \draw [->] (0, 0) -- (0.5, 0);
           \end{tikzpicture}
        & {\begin{tikzpicture}[>=Stealth]
            \draw [->] (0, 0) -- (0, -0.4);
           \end{tikzpicture} \\[-0.5em]
           6.052}
        &
        &
        {\begin{tikzpicture}[>=Stealth]
            \draw [->] (0, 0) -- (0, -0.4);
           \end{tikzpicture} \\[-0.5em]
           39}
        &\\
\end{tblr}
\end{table}

\jie $2.468^2 = 6.091$。

\lianxi
(口答)查表求 $2.291^2$,$2.157^2$,$2.073^2$,$2.307^2$。
\lianxijiange


\liti 查表求 $246.8^2$,$0.2468^2$。

表中查不到这两个平方数,只能查到 $2.468^2 = 6.091$。

从习题五第 9 题,我们看到,底数的小数点每向右(或向左)移动一位,
平方数的小数点相应地向右(或向左)移动两位。

因为 246.8 是把 2.468 的小数点右移两位而得,
所以 $246.8^2$ 应是把 $2.468^2$ 的小数点右移四位而得,
即把 6.091 的小数点右移四位,得
$$ 246.8^2 = 60910 \juhao $$

又因 0.2468 是把 2.468 的小数点左移一位而得,
所以 $0.2468^2$ 应是把 $2.468^2$ 的小数点左移两位而得,
即把 6.091 的小数点左移两位,得
$$ 0.2468^2 = 0.06091 \juhao $$

\begin{table}[H]
\centering
\begin{tblr}{
    columns={c, colsep=2pt},
    column{1, 5} = {r},
    column{3, 7} = {l},
    column{4} = {6em},
    rows={rowsep=0pt},
}
    $2.468^2$   & $\xlongequal{\quad\quad}$ & $6.091$      &   & $2.468^2$     & $\xlongequal{\quad\quad}$ & $6.091$ \\
    $\downarrow$\hspace*{1.5em}  &    & \hspace*{1em}$\downarrow$    &   & $\downarrow$\hspace*{1.5em}  &     & \hspace*{1em}$\downarrow$  \\
    {底数的小数\\点右移两位} &   & {结果的小数\\点右移四位}        &  & {底数的小数\\点左移一位} &   & {结果的小数\\点左移两位} \\
    $\downarrow$\hspace*{1.5em}  &    & \hspace*{1em}$\downarrow$    &   & $\downarrow$\hspace*{1.5em}  &     & \hspace*{1em}$\downarrow$  \\
    $246.8^2$   & $\xlongequal{\quad\quad}$ & $60910$         &   & $0.2468^2$     & $\xlongequal{\quad\quad}$ & $0.06091$ \\
\end{tblr}
\end{table}

\jie $246.8^2 = 60910$,$0.2468^2 = 0.06091$。


\lianxi
(口答)查表求 $22.91^2$,$0.2157^2$,$207.3^2$,$0.02307^2$。
\lianxijiange


下面是四位立方表的一部分

\begin{table}[H]
\begin{tblr}{vlines,
    hline{1, 7} = {1pt, solid},
    hline{2} = {solid},
    vline{1, 17} = {1pt, solid},
    vline{12} = {1}{-}{},
    vline{12} = {2}{-}{},
    columns={colsep=2.5pt, c, $$},
}
    N   & 0     & 1     & 2     & 3     & 4     & 5     & 6     & 7     & 8     & 9     & 1 & 2 & 3 & 4 & 5 \\
    5.0 & 125.0 & 125.8 & 126.5 & 127.3 & 128.0 & 128.8 & 129.6 & 130.3 & 131.1 & 131.9 & 1 & 2 & 2 & 3 & 4 \\
    5.1 & 132.7 & 133.4 & 134.2 & 135.0 & 135.8 & 136.6 & 137.4 & 138.2 & 139.0 & 139.8 & 1 & 2 & 2 & 3 & 4 \\
    5.2 & 140.6 & 141.4 & 142.2 & 143.1 & 143.9 & 144.7 & 145.5 & 146.4 & 147.2 & 148.0 & 1 & 2 & 2 & 3 & 4 \\
    5.3 & 148.9 & 149.7 & 150.6 & 151.4 & 152.3 & 153.1 & 154.0 & 154.9 & 155.7 & 156.6 & 1 & 2 & 3 & 3 & 4 \\
    5.4 & 157.5 & 158.3 & 159.2 & 160.1 & 161.0 & 161.9 & 162.8 & 163.7 & 164.6 & 165.5 & 1 & 2 & 3 & 4 & 4 \\
\end{tblr}
\end{table}

\liti 查表求 $5.19^3$。

\jie $5.19^3 = 139.8$。

\lianxi
(口答)查表求 $5.37^3$,$5.06^3$,$5.21^3$,$5.4^3$。
\lianxijiange


\liti 查表求 $5.263^3$,$5.268^3$,$5.194^3$,$5.197^3$。

查 $5.263^3$ 时, 是先查得 $5.26^3 = 145.5$,
再查这一横行和修正值表中 3 所在的直列,在交叉处得 2,表示 145.5 最后一位上的 2,即 0.2。
因此 $5.263^3 = 145.5 + 0.2 = 145.7$。
查 $5.268^3$ 时,是先查得 $5.27^3 = 146.4$,
再查这一横行和修正值表中 2 所在的直列,在交叉处得 2。
因此 $5.268^3 = 146.4 - 0.2 = 146.2$。

\jie \begin{tblr}[t]{columns={l, $$},}
    5.263^3 = 145.7,  & 5.268^3 = 146.2, \\
    5.194^3 = 140.1,  & 5.197^3 = 140.4 \juhao
\end{tblr}

\lianxi
(口答)查表求 $5.373^3$,$5.069^3$,$5.215^3$,$5.398^3$。
\lianxijiange


\liti 查表求 $0.5197^3$, $0.05197^3$, $519.7^3$, $(-51.97)^3$。

先查得 $5.197^3 = 140.4$。从习题五第 10 题,我们看到,
底数的小数点每向右(或向左)移动一位,立方数的小数点相应地向右(或向左)移动三位。

\begin{table}[H]
\centering
\begin{tblr}{
    columns={c, colsep=2pt},
    column{1, 5} = {r},
    column{3, 7} = {l},
    column{4} = {6em},
    rows={rowsep=0pt},
}
    $5.197^3$   & $\xlongequal{\quad\quad}$ & $140.4$      &   & $5.197^3$     & $\xlongequal{\quad\quad}$ & $140.4$ \\
    $\downarrow$\hspace*{1.5em}  &    & \hspace*{1em}$\downarrow$    &   & $\downarrow$\hspace*{1.5em}  &     & \hspace*{1em}$\downarrow$  \\
    {底数的小数\\点左移一位} &   & {结果的小数\\点左移三位}        &  & {底数的小数\\点右移两位} &   & {结果的小数\\点右移六位} \\
    $\downarrow$\hspace*{1.5em}  &    & \hspace*{1em}$\downarrow$    &   & $\downarrow$\hspace*{1.5em}  &     & \hspace*{1em}$\downarrow$  \\
    $0.5197^3$   & $\xlongequal{\quad\quad}$ & $0.1404$         &   & $519.7^3$     & $\xlongequal{\quad\quad}$ & $140400000$ \\
\end{tblr}
\end{table}

\jie $\begin{aligned}[t]
    & 0.5197^3 = 0.1404, \\
    & 0.05197^3 = 0.0001404, \\
    & 519.7^3 = 140400000, \\
    & (-51.97)^3 = -51.97^3 = -140400 \juhao
\end{aligned}$

\lianxi
(口答)查表求 $53.73^3$,$0.05069^3$,$521.5^3$,$(-0.5398)^3$。
\lianxijiange


求多于四个有效数字的数的平方或立方,如求 $0.24683^2$,$(-51.968)^3$,
可以先把底数四舍五入到四个有效数字,再查表求得有四个有效数字的平方数或立方数。例如,
\begin{gather*}
    0.24683^2 \approx 0.2468^2 = 0.06091; \\
    (-51.968)^3 \approx (-51.97)^3 = -140400 \juhao
\end{gather*}


\begin{enhancedline}

\liti 球的体积公式是
$$ \text{球体积} = \dfrac{4}{3} \times \pi \times \text{半径}^3 \juhao $$

查表计算半径是 $0.89\text{m}$ \footnote{m 表示米,相应地,$\text{m}^2$,$\text{m}^3$ 分别表示平方米,立方米。}
的球的体积(结果保留两个有效数字,$\pi$ 取 3.14)。

\jie $\begin{aligned}[t]
    \dfrac{4}{3} \times 3.14 \times 0.89^3 &= \dfrac{4}{3} \times 3.14 \times 0.7050 \\
                                           &\approx 3.0 \juhao
\end{aligned}$

答:半径是 $0.89\text{m}$ 的球的体积约是 $3.0 \text{m}^3$。


\lianxi
\begin{xiaotis}

\xiaoti{计算 $3.14 \times 0.95^2$ (结果保留两个有效数字)。}

\xiaoti{$\text{球体积} = \dfrac{1}{6} \times \pi \times \text{直径}^3$。
    查表计算直径是 $2.35\text{m}$ 的球的体积(结果保留两个有效数字,$\pi$ 取 $3.14$)。
}

\end{xiaotis}

\end{enhancedline}

