\xiti

\begin{enhancedline}
\begin{xiaotis}
\setcounter{cntxiaoti}{0}

\xiaoti{计算:\begin{tblr}[t]{columns={12em, l, $$}}
    (-8) \times (-7),   &  (+12) \times (-5), \\
    (-36) \times (-1),  &  (-25) \times (_16) \juhao
\end{tblr}}


\xiaoti{把图中输入的每一个数,各乘以 $-3$,得到输出的一个数。}

\begin{figure}[htbp]
    \centering
    \begin{minipage}{6.5cm}
    \centering
    \begin{tikzpicture}[>=Stealth]
    %\foreach \pos/\text in {0/-5, 0.5/-1, 1/\hphantom{+}0, 1.5/\hphantom{+}3, 2/\hphantom{+}6, 2.5/\hphantom{+}8} {
    %    \draw (0, \pos) node {$\text$} (0.4, \pos)[->] --(1, \pos);
    \foreach \pos/\text in {0/-5, 0.5/-1, 1/0, 1.5/3, 2/6, 2.5/8} {
        \ifnum \text < 0
            \draw (-0.14, \pos) node {$\text$} (0.3, \pos)[->] --(1, \pos);
        \else
            \draw (0, \pos) node {$\text$} (0.4, \pos)[->] --(1, \pos);
        \fi
        \draw (1.1, \pos) -- (1.5, \pos);
        \draw [->] (1.6, \pos) -- (3, \pos);
    }
    \node at (1.5, 3) {$\times(-3)$};
    \node at (3.4, 2.5) {$-24$};

    \draw (1.6, 1.5) ellipse
        [x radius=1.1,y radius=2];

    \draw[decorate,decoration={brace,mirror,amplitude=0.2cm}] (-0.5, 2.5) -- (-0.5, 0)
        node [pos=0.5, left=1em, align=center] {输\\入};
    \draw[decorate,decoration={brace,mirror,amplitude=0.2cm}] (3.8, 0) -- (3.8, 2.5)
        node [pos=0.5, right=1em, align=center] {输\\出};
\end{tikzpicture}

    \caption*{(第 2 题)}
    \end{minipage}
    \qquad
    \begin{minipage}{9cm}
    \centering
    \begin{tblr}{hlines, vlines,
        columns={1.5em, c, $$},
        %column{5-8}={1em},
    }
        \times & -3 & -2 & -1 & 0 & 1 & 2 & 3 \\
        3      &    &    &    &   &   &   & 9 \\
        2      &    &    &    &   &   & 4 &   \\
        1      &    &    &    &   & 1 &   &   \\
        0      &    &    &    &   &   &   &   \\
        -1     &    &    &    &   &   &   &   \\
        -2     &    &    &    &   &   &   &   \\
        -3     &    &    &    &   &   &   &   \\
    \end{tblr}
    \caption*{(第 3 题)}
    \end{minipage}
\end{figure}


\xiaoti{在表中的各个小方格里,填写所在横行的第一个数与所在直列的第一个数的积。}

\xiaoti{计算:}
\begin{xiaoxiaotis}

    \begin{tblr}{columns={14em, l, colsep=0pt}}
        \xxt{$2.9 \times (-0.4)$;}      & \xxt{$(-30.5) \times 0.2$;} \\
        \xxt{$(+100) \times (-0.001)$;} & \xxt{$(-4.8) \times (-1.25)$;} \\
        \xxt{$(-7.6) \times 0.03$;}     & \xxt{$(-4.5) \times (-0.32)$。}
    \end{tblr}

\end{xiaoxiaotis}


\xiaoti{计算:}
\begin{xiaoxiaotis}

    \begin{tblr}{columns={14em, l, colsep=0pt}}
        \xxt{$\dfrac{1}{4} \times \left(-\dfrac{8}{9}\right)$;}  &  \xxt{$\left(-\dfrac{5}{6}\right) \times \left(-\dfrac{3}{10}\right)$;} \\
        \xxt{$\left(-2\dfrac{4}{15}\right) \times 25$;}          &  \xxt{$(-0.3) \times \left(-1\dfrac{3}{7}\right)$。}
    \end{tblr}

\end{xiaoxiaotis}


\xiaoti{$(-1) \times (-5) = ?$ \qquad $-(-5) = ?$\\
    $(-1) \times (-5)$ 与 $-(-5)$ 是不是相等?
}

\xiaoti{计算:}
\begin{xiaoxiaotis}

    \begin{tblr}{columns={l, colsep=0pt}}
        \xxt{$(-2)(+3)(-4)$;}      & \xxt{$(-6)(-5)(-7)$;} \\
        \xxt{$0.1 \times (-0.001) \times (-1)$;} & \xxt{$(-100) \times (-1) \times (-3) \times (-0.5)$;} \\
        \xxt{$(-17) \times (-49) \times 0 \times (-8) \times (+37)$。}
    \end{tblr}

\end{xiaoxiaotis}


\xiaoti{计算:}
\begin{xiaoxiaotis}

    \begin{tblr}{columns={14em, l, colsep=0pt}}
        \xxt{$-9 \times (-6) - 18$;}      & \xxt{$5 + 23 \times (-2)$;} \\
        \xxt{$-12 \times 4 - (-8) \times 6$;} & \xxt{$8 \cdot (-9) - 7 \cdot (-15)$;} \\
        \xxt{$\left(-\dfrac{2}{3}\right) \times \dfrac{1}{2} + \dfrac{1}{3} \times (-4)$。}
    \end{tblr}

\end{xiaoxiaotis}


\xiaoti{高度每增加 1 千来,气温大约降低 6 ℃ 。现在地面气温是 19 ℃, 那么 5 千米高空的气温是多少?}
% 原书中第 9 题配有插图,由于比较难画,而没有图也不影响做题,所以我没有画它。

\xiaoti{用字母写出加法交换律、加法结合律、乘法交换律、乘法结合律和分配律。}

\xiaoti{计算:}
\begin{xiaoxiaotis}

    \begin{tblr}{columns={l, colsep=0pt}}
        \xxt{$\left(-4\dfrac{1}{20}\right) (+1.25) (-8)$;}                    & \xxt{$(-10) (-8.24) (-0.1)$;} \\
        \xxt{$\left(-\dfrac{5}{6}\right) (+2.4) \left(+\dfrac{3}{5}\right)$;} & \xxt{$\left(\dfrac{7}{9} - \dfrac{5}{6} + \dfrac{3}{4} - \dfrac{7}{18}\right) \times 36$;} \\
        \xxt{$-\dfrac{3}{4} \times \left(8 - 1\dfrac{1}{3} - 0.04\right)$;}   & \xxt{$71\dfrac{15}{16} \times (-8)$。}
    \end{tblr}

\end{xiaoxiaotis}


\xiaoti{计算:}
\begin{xiaoxiaotis}

    \begin{tblr}{columns={14em, l, colsep=0pt}}
        \xxt{$-91 \div 13$;}               & \xxt{$-56 \div (-14)$;} \\
        \xxt{$(-42) \div 0.6$;}            & \xxt{$-25.6 \div (-0.064)$;} \\
    \end{tblr} % 为了实现分页,手工将表格分拆成两个。

    \begin{tblr}{columns={14em, l, colsep=0pt}}
        \xxt{$16 \div (-3)$;}              & \xxt{$1 \div \left(-\dfrac{2}{3}\right)$;} \\
        \xxt{$\dfrac{4}{5} \div (-1)$;}    & \xxt{$-3\dfrac{1}{7} \div \dfrac{11}{12}$;} \\
        \xxt{$-0.25 \div \dfrac{3}{8}$;}   & \xxt{$-\dfrac{1}{4} \div (-1.5)$。}
    \end{tblr}

\end{xiaoxiaotis}


\xiaoti{把图中第一个圈里的每一个数,各除以 $(-5)$,得到第二个圈里的一个数。}

\begin{figure}[htbp]
    \centering
    \begin{tikzpicture}[>=Stealth]
    \foreach \pos/\text in {0/-20, 0.5/-5, 1/\hphantom{+}0, 1.5/+1, 2/+4, 2.5/+8, 3/+10} {
        \draw (0, \pos) node {$\text$} (0.4, \pos)[->] -> (3.5, \pos);
    }
    \node at (1.8, 3.3) {$\div(-5)$};
    \node at (4.0, 1.5) {$-\frac{1}{5}$};

    \draw (0, 1.5) ellipse [x radius=0.9, y radius=2];
    \draw (3.7, 1.5) ellipse [x radius=0.9, y radius=2];
\end{tikzpicture}

    \caption*{(第 13 题)}
\end{figure}


\xiaoti{填写下表: \\
    \begin{tblr}{hlines, vlines,
        columns={3em, c, $$},
        column{1} = {6em, mode=text},
        rows = {rowsep+ = 5pt},
    }
        原来的数  & \dfrac{7}{8} & -\dfrac{4}{5} & -2.5 & \dfrac{1}{6} & -2 & -3\dfrac{1}{3} & 1 \\
        它的倒数  &              &               &      &              &    &                &
    \end{tblr}
}


\xiaoti{计算:}
\begin{xiaoxiaotis}

    \xxt{$\left(-\dfrac{3}{4}\right) \times \left(-1\dfrac{1}{2}\right) \div \left(-2\dfrac{1}{4}\right)$;}

    \xxt{$-6 \div (-0.25) \times \dfrac{11}{14}$。}

\end{xiaoxiaotis}


\xiaoti{计算:}
\begin{xiaoxiaotis}

    \begin{tblr}{columns={l, colsep=0pt}}
        \xxt{$-8 + 4 \div (-2)$;}        & \xxt{$6 - (-12) \div (-3)$;} \\
        \xxt{$3 \cdot (-4) + (-28) \div 7$;}            & \xxt{$(-7) (-5) - 90 \div (-15)$;} \\
        \xxt{$(-48) \div 8 - (-25) (-6)$;}              & \xxt{$42 \times \left(-\dfrac{2}{3}\right) + \left(-\dfrac{3}{4}\right) \div (-0.25)$。}
    \end{tblr}

\end{xiaoxiaotis}


\xiaoti{农业中学气象小组每天在早晨 8 时测量室外温度(摄氏),记录如下:\\
    \hspace*{2em}温度单位:℃  \makebox[28em][r]{1982年11月} \\
    \begin{tblr}{hlines, vlines,
            hline{1, 7} = {1pt, solid},
            hline{3, 5} = {1}{-}{},
            hline{3, 5} = {2}{-}{},
            vline{1, 13} = {1pt, solid, abovepos = 1, belowpos = 1},
            columns={colsep+ = 0.2em, c, $$},
            column{1, 12} = {mode=text},
        }
            日期  & 1   & 2   & 3   & 4   & 5   & 6    & 7  & 8   & 9    & 10   & 上旬平均 \\
            温度  & 6   & 6.5 & 7   & 4   & 2.5 & 3    & 1  & 1.5 & -2   & -3   & \\
            日期  & 11  & 12  & 13  & 14  & 15  & 16   & 17 & 18  & 19   & 20   & 中旬平均 \\
            温度  & -1  & 0   & 1.5 & 0.5 & -2  & -3.5 & -4 & -1  & 2    & 1    & \\
            日期  & 21  & 22  & 23  & 24  & 25  & 26   & 27 & 28  & 29   & 30   & 下旬平均 \\
            温度  & 1.5 & 0.5 & 0   & -3  & -5  & -2   & -1 & -4  & -5.5 & -7.5 & \\
    \end{tblr} \\
    \makebox[36em][r]{月平均 \underline{\hspace*{4em}}} \\
    计算早晨 8 时上旬、中旬、下旬平均温度和月平均温度。
}


\end{xiaotis}
\end{enhancedline}

