% 原书的目录结构就是如此(缺少 section)
% 忽略这里的报错: Difference (2) between bookmark levels is greater (hyperref)	than one, level fixed.
\subsection{方程}\label{subsec:3-1}

在小学里,我们学过方程的初步知识。 这一章将在复习小学所学内容的基础上,进一步学习方程的知识。

看下面的式子:

\hspace*{2em}\begin{tblr}{columns={8em}}
    $1 + 2 = 3$, & $a + b = b + a$, \\
    $S = ab$,    & $4 + x = 7$。
\end{tblr}

象这种表示相等关系的式子,叫做\zhongdian{等式}。等式中,等号左右两边的式子,分别是等式的左边与右边。

关于等式,有下面的性质:等式的两边都加上,或都减去,或都乘以,或都除以(除数不能为零)同一个数,所得到的仍是等式。

我们来研究上面的最后一个等式
$$ 4 + x = 7 \douhao $$
这里,字母 $x$ 表示未知数。 我们来看一看,$x$ 应该取什么值时,等式成立(图 \ref{fig:3-1})。

\begin{figure}[htbp]
    \centering
    \begin{tikzpicture}[>=Stealth]
    \draw (-0.8, 0.6) rectangle (0.8, 1.4);
    \node at (0, 1) {$x = ?$};
    \draw [->] (0, 0.6) -- (0, 0.2);
    \node at (0, 0) {$4 + x = 7$};
\end{tikzpicture}

    \caption{}\label{fig:3-1}
\end{figure}

含有未知数的等式叫做\zhongdian{方程}。 $4 + x = 7$ 是一个方程,下面等式
$$ 5 - 2x = 1 \nsep y^2 + 2 = 4y - 1 \nsep x - 2y = 6 $$
中的 $x$,$y$ 都表示未知数,它们也都是方程。

\liti 根据下列条件列出方程:

(1)某数乘以 2 再减去 3,得 5;

(2)某数加上 2 再乘以 3,得 12。

\jie (1) 设某数是 $x$,那么,“$x$ 乘以 2 再减去 3,得 5”,就可以表示成方程
$$ 2x - 3 = 5 ; $$

(2)设某数是 $x$,那么,“$x$ 加上 2 再乘以 3,得 12”,就可以表示成方程
$$ (x + 2) \times 3 = 12 \douhao $$
\fenge{即}{$$ 3(x + 2) = 12 \juhao $$}

\lianxi
\begin{xiaotis}

根据下列条件列出方程:

\xiaoti{某数减去 5, 得 4 。}

\xiaoti{某数的 3 倍与 2 的和等于 8。}

\end{xiaotis}
\lianxijiange

我们知道,使方程左右两边的值相等的未知数的值,叫做方程的\zhongdian{解}。
例如在方程 $4 + x = 7$ 里, 当未知数 $x$ 的值是 3 时,方程左右两边的值相等,
因此 $x = 3$ 是方程 $4 + 4 = 7$ 的解。
只含有一个未知数的方程的解,也叫做\zhongdian{根}。
例如,也可以说 $x = 3$ 是方程 $4 + x = 7$ 的根。

求方程的解的过程,叫做\zhongdian{解方程}。

\liti 检验下列各数是不是方程 $2x - 3 = 5$ 的解:

\twoInLine{(1)$x = 6$;}{(2)$x = 4$。}

\jie (1) 把 $x = 6$ 代入方程,
$$ \zuobian = 2 \times 6 - 3 = 9 \nsep \youbian = 5 \juhao $$

\fenge{$\because$}{$$ \zuobian \neq \youbian \douhao $$}

$\therefore$ \quad $x = 6$ 不是方程 $2x - 3 = 5$ 的解。

(2) 把 $x = 4$ 代入方程,
$$ \zuobian = 2 \times 4 - 3 = 5 \nsep \youbian = 5 \juhao $$

\fenge{$\because$}{$$ \zuobian = \youbian \douhao $$}

$\therefore$ \quad $x = 4$ 是方程 $2x - 3 = 5$ 的解。

说明:“$\neq$” 是表示不相等的符号,读作“不等于”。

\lianxi
\begin{xiaotis}

\xiaoti{检验下列各数是不是方程 $6(x + 3) = 30$ 的解:}
\begin{xiaoxiaotis}

    \twoInLineXxt{$x = 5$;}{$x = 2$。}

\end{xiaoxiaotis}

\xiaoti{检验下列各数是不是方程 $3x - 1 = 2x + 1$ 的解:}
\begin{xiaoxiaotis}

    \twoInLineXxt{$x = 4$;}{$x = 2$。}

\end{xiaoxiaotis}

\end{xiaotis}

