\subsection{有理数减法法则}\label{subsec:1-8}

和小学学过的减法意义相同,有理数减法是有理数加法的逆运算。
有理数减法就是已知两个数的和与其中的一个加数,求另一个加数的运算。

我们看下面的例子:

(1) \quad $(+3) - (+5)$。

这个算式就是求 $+5$ 与什么数相加得 $+3$。

$\because \; (+5) + (-2) = +3$,

$\therefore \; (+3) - (+5) = -2$。

从有理数的加法,我们知道

\hspace*{2em} $(+3) + (-5) = -2$。

$\therefore \; (+3) - (+5) = (+3) + (-5)$。

(2) \quad $(+3) - (-5)$。

这个算式就是求 $-5$ 与什么数相加得 $+3$。

$\because \; (-5) + (+8) = +3$,

$\therefore \; (+3) - (-5) = +8$。

从有理数的加法,我们知道

\hspace*{2em} $(+3) + (+5) = +8$。

$\therefore \; (+3) - (-5) = (+3) + (+5)$。

\begin{wrapfigure}{r}{4cm}
    \centering
    \begin{tikzpicture}
        \draw (0,0) pic {thermometer={7}};
    \end{tikzpicture}
    \caption{}\label{fig:1-17}
\end{wrapfigure}

综合上面的情况,得到有理数减法的法则:

\framebox{
    \zhongdian{减去一个数, 等于加上这个数的相反数。}
}

这样,在进行有理数减法运算时,把减数的符号改变后,就可以按有理数加法的法则进行运算了。

\liti 计算:

(1) \quad $(-3) - (-5)$;

(2) \quad $0 - (-7)$。

\jie (1) \quad $(-3) - (-5) = (-3) + (+5) = 2$;

(2) \quad $0 - (-7) = 0 + (+7) = 7$。



\liti 零上 7 ℃ 比零上 3 ℃ 高多少? 零上 7 ℃ 比零下 3 ℃ 高多少? (图 \ref{fig:1-17})

\jie $(+7) - (+3) = (+7) + (-3) = 4$;

$(+7) - (-3) = (+7) + (+3) = 10$。

答:零上 7 ℃ 比零上 3 ℃ 高 4 ℃; 零上 7 ℃ 比零下 3 ℃ 高 10 ℃。


\lianxi
\begin{xiaotis}

\xiaoti{(口答)\begin{tblr}[t]{columns={8em, l, $$}}
    (+8) - (+5),  &  (+8) - (-5),  &  (+6) - (+9),  & (+6) - (-9), \\
    (-6) - (+4),  &  (-6) - (-4),  &  (-7) - (+8),  &  (-7) - (-8)\juhao
\end{tblr}}


\xiaoti{(口答)\begin{tblr}[t]{columns={8em, l, $$}}
    (+9) - (+4),  &  (+9) - (-4),  &  (+4) - (+9),  &  (+4) - (-9), \\
    (-9) - (+4),  &  (-9) - (-4),  &  (-4) - (+9),  &  (-4) - (-9)\juhao
\end{tblr}}


\xiaoti{(口答)\begin{tblr}[t]{columns={8em, l, $$}}
    (+5) - (-5),  &  (-5) - (-5),  &  (+5) - (+5),  &  (+5) - (+5), \\
    0 - (-5),     &  (-5) - (0),   &  0 - (+5),     &  (+5) - 0\juhao
\end{tblr}}

\end{xiaotis}

