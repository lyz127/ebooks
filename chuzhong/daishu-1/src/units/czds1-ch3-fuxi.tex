\fuxiti

\begin{enhancedline}
\begin{xiaotis}

\xiaoti{用小学里学过的方法解下列方程:}
\begin{xiaoxiaotis}

    \begin{tblr}{columns={18em, colsep=0pt}}
        \xxt{$120 + x = 150$;} & \xxt{$30 - x = 16$;} \\
        \xxt{$0.5x = 15.5$;}   & \xxt{$x \div \dfrac{2}{3} = \dfrac{7}{8}$。}
    \end{tblr}

\end{xiaoxiaotis}

\xiaoti{什么样的两个方程叫做同解方程?举例说明。}

\xiaoti{方程的两个同解原理是什么?举例说明。}

\xiaoti{下列各题的解法对不对?如果不对,错在哪里?}
\begin{xiaoxiaotis}

    \begin{tblr}{columns={18em, colsep=0pt}}
        \xxt{$2x + 1 = 4x + 1 , $ \\
                \hspace*{.5em} $\begin{aligned}[t]
                    2x + 4x &= 0, \\
                    6x &= 0, \\
                    x &= 0 \juhao
                \end{aligned}$}
            & \xxt{$\dfrac{x}{2} = x + 6, $ \\[.5em]
                \hspace*{.5em} $\begin{aligned}[t]
                    \dfrac{x}{2} - x &= 6, \\
                    -\dfrac{x}{2} &= 6, \\
                    x &= 12 \juhao
                \end{aligned}$} \\
        \xxt{$\dfrac{x + 1}{2} = \dfrac{3x - 1}{2} - 1 , $ \\
                \hspace*{.5em} $\begin{aligned}[t]
                    x + 1 &= 3x - 1 - 1, \\
                    2x &= 3, \\
                    x &= \dfrac{3}{2} \juhao
                \end{aligned}$}
            & \xxt{$\dfrac{2x + 1}{3} - \dfrac{x + 1}{6} = 2 $ \\
                \hspace*{.5em} $\begin{aligned}[t]
                    4x + 2 - x + 1 &= 12, \\
                    3x &= 9, \\
                    x &= 3 \juhao
                \end{aligned}$}
    \end{tblr}

\end{xiaoxiaotis}

解下列方程(第 5 ~ 8 题)

\xiaoti{}%
\begin{xiaoxiaotis}%
    \xxt[\xxtsep]{$\dfrac{4}{3} - 8x = 3 - \dfrac{11}{2}x$;}

    \xxt{$0.5x - 0.7 = 6.5 - 1.3x$;}

    \xxt{$\dfrac{1}{6}(3x - 6) = \dfrac{2}{5}x - 3$;}

    \xxt{$\dfrac{1}{3}(1 - 2x) = \dfrac{2}{7}(3x + 1)$。}

\end{xiaoxiaotis}

\xiaoti{}%
\begin{xiaoxiaotis}%
    \xxt[\xxtsep]{$3(8x - 1) - 2(5x + 1) = 6(2x + 3) + 5(5x - 2)$;}

    \xxt{$3(x - 7) - 2[9 - 4(2 - x)] = 22$;}

    \xxt{$\dfrac{x + 4}{5} - x + 5 = \dfrac{x + 3}{3} - \dfrac{x - 2}{2}$;}

    \xxt{$\dfrac{1}{2}(y + 1) + \dfrac{1}{3}(y + 2) = 3 - \dfrac{1}{4}(y + 3)$。}

\end{xiaoxiaotis}

\begin{withstar}
\xiaoti{}%
\begin{xiaoxiaotis}%
    \xxt[\xxtsep]{$\dfrac{3}{4} \left[\dfrac{4}{3}\left(\dfrac{1}{2}x - \dfrac{1}{4}\right) - 8\right] = \dfrac{3}{2}x + 1$;}

    \xxt{$x - \dfrac{1}{3}\left[x - \dfrac{1}{3}(x - 9)\right] = \dfrac{1}{9}(x - 9)$;}

    \xxt{$3\{2x - 1 - [3(2x - 1) + 3]\} = 5$;}

    \xxt{$4(x - 2) - [5(1 - 2x) - 4(5x - 1)] = 0$。}

\end{xiaoxiaotis}

\xiaoti{}%
\begin{xiaoxiaotis}%
    \xxt[\xxtsep]{$\dfrac{4x - 1.5}{0.5} - \dfrac{5x - 0.8}{0.2} = \dfrac{1.2 - x}{0.1}$;}

    \xxt{$\dfrac{5x + \dfrac{7}{3}}{2} = \dfrac{6x - \dfrac{5}{2}}{3} - \dfrac{x + 4}{6}$。}

\end{xiaoxiaotis}
\end{withstar}

\xiaoti{$x$ 是什么数时,单项式 $3a^2b^{2x + 1}$ 与 $\dfrac{1}{4}a^2b^{3x - 1}$ 是同类项?}

列出一元一次方程解下列应用题( 第10 ~ 23 题):

\xiaoti{两数的和为 25, 其中一个数比另一个数的 2 倍大 4, 求这两个数。}

\xiaoti{三个连续奇数的和为 69, 求这三个数。}

\xiaoti{下图(图中的长度单位为毫米)表示某种零件的锻坯、如果选用直径为 70 毫米的圆钢锻造,求应截取圆钢的长度(精确到 1 毫米)。}

\begin{figure}[htbp]
    \centering
    \begin{tikzpicture}[>=Stealth,
    every node/.style={fill=white, inner sep=1pt},
]
    \pgfmathsetmacro{\factor}{0.02}
    \pgfmathsetmacro{\a}{\factor * 290}
    \pgfmathsetmacro{\b}{\factor * 410}
    \pgfmathsetmacro{\ra}{\factor * 40 /2}
    \pgfmathsetmacro{\rb}{\factor * 60 /2}

    \draw [ultra thick] (0, -\ra) rectangle (\a, \ra);
    \draw [ultra thick] (\a, -\rb) rectangle (\b, \rb);
    \draw [loosely dash dot] (-0.2, 0) -- (0.2 + \b, 0);
    \draw [<->] (0.4+\b, -\rb) to [xianduan={above=0.4cm}] node [rotate=90] {$\phi \, 60$} (0.4+\b, \rb);
    \draw [<->] (0, -0.2-\rb) to [xianduan={above=0.4cm}] node {$290$} (\a, -0.2-\rb);
    \draw [<->] (0, -0.6-\rb) to [xianduan={above=0.6cm}] node {$410$} (\b, -0.6-\rb);

    \draw (0, \ra) -- (-0.5, \ra)
          (0, -\ra) -- (-0.5, -\ra);
    \draw [->] (-0.3, 0.4 + \ra) -- (-0.3, \ra);
    \draw [->] (-0.3, -0.8 - \ra) -- (-0.3, -\ra) node [at end, rotate=90, left=1em] {$\phi\, 40$} ;
    \draw (-0.3, \ra) -- (-0.3, -\ra);
\end{tikzpicture}

    \caption*{(第 12 题)}
\end{figure}


\xiaoti{( 我国古代问题) \footnotemark 好马每天走 240 里, 劣马每天走 150 里。 劣马先走 12 天, 好马几天可以追上劣马?}
\footnotetext{这道题选自元朝朱世杰所著的《算学启蒙》(1299 年)。原题是:
    “良马日行二百四十里,驽马日行一百五十里,驽马先行一十二日,问良马几何日追及之。答曰:二十日。”
}

\xiaoti{运动场的跑道一圈长 400 米。 甲练习骑自行车,平均每分骑 490 米;乙练习跑步,平均每分跑 250 米。
    两人从同一处同时同向出发,经过多少分两人首次相遇?
}

\xiaoti{要加工 200 个零件。 甲先单独加工了 5 小时, 然后又与乙一起加工了 4 小时,完成了任务。
    已知甲每小时比乙多加工 2 个零件,求甲、乙每小时各加工多少个零件。
}

\xiaoti{用 50 ppm 的某种农药溶液喷雾,可以防治稻瘟病。现将 $2\%$ 的这种农药乳油稀释成 50 ppm 的溶液,应加水多少倍?}

\xiaoti{一架敌机侵犯我领空, 我机立即起飞迎击。 在两机相距 50 千米时,敌机扭转机头,以 15 千米每分的速度逃跑,
    我机以 22 千米每分的速度追击。当我机追至距敌机 1 千米时,与敌机展开了激战,只用半分就击落了敌机。
    敌机从逃跑到被我机歼灭时只有几分?
}

\xiaoti{有一架飞机, 最多能在空中连续飞行 4 小时。 飞出时的速度是 950 千米每小时, 返回时的速度是 850 千米每小时,
    这架飞机最远飞出多少千米就应返回(答案只保留整数部分)?
}


\xiaoti{一个拖拉机队耕一片地, 第一天耕了这片地的 $\dfrac{1}{3}$, 第二天耕了剩下的一片地的 $\dfrac{1}{2}$,
    这时还剩下 38 公顷没有耕。 这片地一共有多少公顷?
}

\begin{withstar}
\xiaoti{两个长方形的长与宽的比都是 $2 : 1$ , 大长方形的宽比小长方形的宽多 3 厘米,
    大长方形的周长是小长方形的周长的 2 倍, 求两个长方形的面积。
}

\xiaoti{(我国古代问题)\footnotemark 用绳子量井深: 把绳三折来量, 井外余绳 4 尺;
    把绳四折来量, 井外余绳 1 尺。 求井深和绳长各是多少。
}
\footnotetext{这道题选自明朝程大位所著的《算法统宗》(1592 年) 卷七。原题是:
    “假如井不知深, 先将绳三折入井, 绳长四尺, 后将绳四折入井, 亦长一尺。
    向井深及绳长各若干。 答曰:井深八尺, 绳长三丈六尺。”
}

\xiaoti{一块金与银的合金的重量是 250 克,放在水中称减轻了 16 克。已知金在水中称重量减轻 $\dfrac{1}{19}$,
    银在水中称重量减轻 $\dfrac{1}{10}$, 求这块合金中金、银各占多少克。
}

\xiaoti{一次考试出了 25 道题。 回答每道题时, 只需要在所附的 4 种答案中选定一种。
    答对一题给 4 分, 不答或答错一题扣 1 分。 如果一个学生得 90 分, 他答对了多少道题? 如果得 60 分呢?
}

\end{withstar}
\end{xiaotis}
\end{enhancedline}

