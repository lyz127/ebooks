\subsection{添括号}\label{subsec:2-6}

从去括号知道:
\begin{align*}
    & a + (b - c) = a + b - c; \\
    & a - (b - c) = a - b + c\juhao
\end{align*}

反过来,上面的两个式子可以写成:
\begin{align*}
    & a + b - c = a + (b - c); \\
    & a - b + c = a - (b - c)\juhao
\end{align*}

可以看出,添括号时有如下法则:

\zhongdian{括号前面是 “$+$” 号,括到括号里的各项都不变;
    括号前面是 “$-$” 号,括到括号里的各项都变号。}

我们可以根据需要,按着添括号的法则,把一个多项式或者它的一部分括在括号里,
而不改变这个多项式的值。

\liti 不改变多项式 $3a - 2b + c$ 的值,按照下列要求添括号:

(1)把它放在前而带有 “$+$” 号的括号里;

(2)把它放在前面带有 “$-$” 号的括号里。

\jie (1)$3a - 2b + c = + (3a - 2b + c)$;

(2)$3a - 2b + c = -(-3a + 2b - c)$。


\liti 不改变多项式 $x^3 - 5x^2 - 4x + 9$ 的值,按照下列要求添括号:

(1)后两项放在前而带有 “$+$” 号的括号里;

(2)后两项放在前面带有 “$-$” 号的括号里。

\jie (1)$x^3 - 5x^2 - 4x + 9 = x^3 - 5x^2 + (-4x + 9)$;

(2)$x^3 - 5x^2 - 4x + 9 = x^3 - 5x^2 - (4x - 9)$。


\lianxi
\begin{xiaotis}

\xiaoti{在等号右边的括号内,填上适当的项:}
\begin{xiaoxiaotis}

    \xxt{$a + b + c - d = a + \ewkh[6em]$;}

    \xxt{$a - b + c - d = a - \ewkh[6em] $;}

    \xxt{$a - b - c - d = a - b + \ewkh[4.3em] $;}

    \xxt{$a + b - c + d = a + b - \ewkh[4.3em] $。}

\end{xiaoxiaotis}

\xiaoti{}%
\begin{xiaoxiaotis}%
    \xxt[\xxtsep]{不改变 $m^2 + mn - 5m -5n$ 的值,把前两项放在前面带有 “$+$” 号的括号里,后两项放在前而带有 “$-$” 号的括号里;}

    \xxt{不改变 $-3ax + 4ay + 3bx - 4by$ 的值,把前两项放在前面带有 “$-$” 号的括号里,后两项放在前而带有 “$+$” 号的括号里。}

\end{xiaoxiaotis}

\end{xiaotis}

