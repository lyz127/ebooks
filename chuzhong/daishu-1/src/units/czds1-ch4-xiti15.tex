\xiti

\begin{enhancedline}
\begin{xiaotis}

\xiaoti{按照下列条件,写出仍能成立的不等式:}
\begin{xiaoxiaotis}

    \xxt{$5 > -4$,两边都加上 8;}

    \xxt{$1 < 3$,两边都减去 4;}

    \xxt{$-3 < -2$,两边都乘以 2;}

    \xxt{$-14 < 20$,两边都除以 2;}

    \xxt{$-4 < -1$,两边都乘以 $-3$;}

    \xxt{$-8 < -4$,两边都除以 $-4$。}

\end{xiaoxiaotis}

\xiaoti{已知 $a < b$,用不等号连结下列各题中的两式:}
\begin{xiaoxiaotis}

    \begin{tblr}{columns={18em, colsep=0pt}}
        \xxt{$a + 1$ 与 $b + 1$;} & \xxt{$a - 3$ 与 $b - 3$;} \\
        \xxt{$-3a$ 与 $-3b$;} & \xxt{$\dfrac{a}{4}$ 与 $\dfrac{b}{4}$;} \\
        \xxt{$-\dfrac{a}{7}$ 与 $-\dfrac{b}{7}$;} & \xxt{$a - b$ 与 $0$。}
    \end{tblr}

\end{xiaoxiaotis}

\xiaoti{根据下列数量关系,列出不等式:}
\begin{xiaoxiaotis}

    \xxt{$x$ 的 $\dfrac{2}{3}$ 减去 5 小于 1;}

    \xxt{$x$ 与 6 的和不小于 9;}

    \xxt{8 与 $y$ 的 2 倍的和是正数;}

    \xxt{$a$ 的 3 倍与 7 的差是负数。}

\end{xiaoxiaotis}

\xiaoti{在数轴上表示不等式的解集:}
\begin{xiaoxiaotis}

    \begin{tblr}{columns={18em, colsep=0pt}}
        \xxt{$x > 3$;}         & \xxt{$x \geqslant -2$;} \\
        \xxt{$x \leqslant 4$;} & \xxt{$x < 0$。}
    \end{tblr}

\end{xiaoxiaotis}

\xiaoti{为什么下列各题中的两个不等式是同解不等式?}
\begin{xiaoxiaotis}

    \xxt{$\dfrac{1}{2} > 2x$ 与 $1 > 4x$;}

    \xxt{$4x - 2 \geqslant 6$ 与 $4x \geqslant 8$;}

    \xxt{$3.14x < 0$ 与 $x < 0$;}

    \xxt{$4 \leqslant -\dfrac{5}{17}x$ 与 $-68 \geqslant 5x$;}

    \xxt{$-\dfrac{22}{7}x < 0$ 与 $x > 0$;}

    \xxt{$-\dfrac{x}{4} > -\dfrac{2x}{3}$ 与 $3x < 8x$。}

\end{xiaoxiaotis}

\xiaoti{说明下列不等式变形的根据是不等式的哪一条同解原理:}
\begin{xiaoxiaotis}

    \xxt{如果 $x + 2 > 7$,那么 $x > 7 - 2$;}

    \xxt{如果 $3x > 1 - 2x$,那么 $3x - 2x > 1$;}

    \xxt{如果 $2x < -5$,那么 $x < -\dfrac{5}{2}$;}

    \xxt{如果 $-\dfrac{x}{2} < 3$,那么 $x > -6$。}

\end{xiaoxiaotis}


\xiaoti{用小于号 “$<$” 或大于号 “$>$” 填空,使所得的不等式与原不等式是同解不等式,并说出根据的是不等式的哪一条同解原理。}
\begin{xiaoxiaotis}

    \xxt{如果 $-a < 5$,那么 $a \xhx -5$;}

    \xxt{如果 $3a > 6$,那么 $a \xhx 2$。}

\end{xiaoxiaotis}


\xiaoti{解下列不等式,并把它们的解集在数轴上表示出来:}
\begin{xiaoxiaotis}

    \begin{tblr}{columns={18em, colsep=0pt}}
        \xxt{$5x > -10$;}                            & \xxt{$-3x < -12$;} \\
        \xxt{$\dfrac{x}{2} \geqslant 3$;}            & \xxt{$-\dfrac{3x}{5} < -3$;} \\
        \xxt{$8x - 1 \geqslant 6x + 5$;}             & \xxt{$3x - 5 < 1 + 5x$;} \\
        \xxt{$3(2x + 5) > 2(4x + 3)$;}               & \xxt{$10 - 4(x - 3) \leqslant 2(x - 1)$;} \\
        \xxt{$\dfrac{x - 3}{2} > \dfrac{x + 6}{5}$;} & \xxt{$\dfrac{2(4x - 3)}{3} \leqslant \dfrac{5(5x + 12)}{6}$。}
    \end{tblr}

\end{xiaoxiaotis}

\xiaoti{解下列不等式:}
\begin{xiaoxiaotis}

    \begin{tblr}{columns={18em, colsep=0pt}}
        \xxt{$\dfrac{x + 5}{2} - 1 < \dfrac{3x + 2}{2}$;}       & \xxt{$\dfrac{y + 1}{3} - \dfrac{y - 1}{2} \geqslant \dfrac{y - 1}{6}$;} \\
        \xxt{$2 + \dfrac{3(x + 1)}{8} > 3 - \dfrac{x - 1}{4}$;} & \xxt{$\dfrac{3x - 2}{3} - \dfrac{9 - 2x}{3} \leqslant \dfrac{5x + 1}{2}$。}
    \end{tblr}

\end{xiaoxiaotis}

\xiaoti{不求出下列各题中的两数的积,分别说出这些积是大于 0, 小于 0,还是等于 0:}
\begin{xiaoxiaotis}

    \begin{tblr}{columns={18em, colsep=0pt}}
        \xxt{3 与 2;} & \xxt{$-\dfrac{1}{5}$ 与 $-\dfrac{1}{2}$;} \\
        \xxt{$-0.4$ 与 0.7;} & \xxt{1.5 与 $-6$。}
    \end{tblr}

\end{xiaoxiaotis}

\xiaoti{用小于号 “$<$” 或大于号 “$>$” 填空:}
\begin{xiaoxiaotis}

    \xxt{当 $a > 0$, $b > 0$ 时,$ab \xhx 0$;}

    \xxt{当 $a < 0$, $b > 0$ 时,$ab \xhx 0$;}

    \xxt{当 $a < 0$, $b < 0$ 时,$ab \xhx 0$;}

    \xxt{当 $a > 0$, $b < 0$ 时,$ab \xhx 0$。}

\end{xiaoxiaotis}

\xiaoti{$x$ 取什么值时,代数式 $4x + 8$ 的值:}
\begin{xiaoxiaotis}

    \threeInLineXxt{是正数?}{是负数?}{是 0?}

\end{xiaoxiaotis}

\end{xiaotis}
\end{enhancedline}

