\xiaojie

一、本章主要内容是有理数的有关概念及其运算。

二、“数学是从人的\textbf{需要}中产生的” 。
正数和负数的概念是实际生活中大量存在的相反意义的量的反映,它们构成了数学中的一对矛盾。

三、有理数包括正整数、零、负整数、正分数、负分数。有理数可以用数轴上的点表示出来。

四、有理数加法的法则:两数相加,同号的取原来的符号,并把绝对值相加;
异号的取绝对值较大的加数的符号,并用较大的绝对值减去较小的绝对值。

有理数乘法的法则:两数相乘,同号得正,异号得负,并把绝对值相乘。

减去一个数,等于加上这个数的相反数。

除以一个数,等于乘以这个数的倒数。

五、有理数的运算律有:

\jiange
\begin{tblr}{colspec={Q[l,7em] Q[l,$]}, colsep=0pt, rowsep=0pt}
    加法交换律 &  a + b = b + a; \\
    加法结合律 & (a + b) + c = a + (b + c); \\
    乘法交换律 & ab = ba; \\
    乘法结合律 & (ab)c = a(bc); \\
    分配律    & a(b + c) = ab + ac \,\juhao
\end{tblr}\jiange

六、求几个相同因数的积的运算是乘方,即
$$ \underbrace{a \cdot a \cdot \cdots \cdot a}_\text{$n$ 个} = a^n \juhao $$

平方数和立方数可以从表中查得。

