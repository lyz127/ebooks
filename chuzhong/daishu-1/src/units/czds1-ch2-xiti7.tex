\xiti

\begin{xiaotis}

\xiaoti{填写下表:}

% 绘制立方体的命令
% 共有六个参数:
%   1. 长
%   2. 长 的 字母
%   3. 宽
%   4. 宽 的 字母
%   5. 高
%   6. 高 的 字母
\newcommand{\changfangti}[6]{
    \begin{tikzpicture}
        \pgfmathsetmacro{\x}{#1}
        \pgfmathsetmacro{\z}{#3}
        \pgfmathsetmacro{\y}{#5}
        \draw (0,0,0) -- (\x,0,0) -- (\x,\y,0) -- (0,\y,0) -- cycle;
        \draw (\x,0,0) -- (\x,\y,0) -- (\x,\y,-\z) -- (\x,0,-\z) -- cycle;
        \draw (0,\y,0) -- (0,\y,-\z) -- (\x,\y,-\z) -- (\x,\y,0) -- cycle;
        \draw [dashed] (0,0,0) -- (0,0,-\z) -- (0,\y,-\z);
        \draw [dashed] (0,0,-\z) -- (\x,0,-\z);

        \node at (\x/2,   0, 0.4)        {$#2$};
        \node at (\x+0.2, 0, -\z/2)      {$#4$};
        \node at (\x+0.1, \y/2, -\z-0.2) {$#6$};
    \end{tikzpicture}
}

\def\lend{6em}
\begin{longtblr}[theme=nocaption]{
    hlines, vlines,
    colspec={cclcc},
    cells={valign=m},
    rowhead = 1,
}
    名称 & 图形 &  文字表示的公式 & 字母的意义 & {字母表示\\的公式} \\
    长方体
        & \begin{minipage}[c]{3cm}
            \changfangti{2}{a}{1.2}{b}{0.3}{c}
          \end{minipage}
        & $\text{体积} = \text{长} \times \text{宽} \times \text{高}$
        & \begin{minipage}[c][6em]{\lend}
            $V$ —— 体积 \\
            $a$ —— 长 \\
            $b$ —— 宽 \\
            $c$ —— 高
          \end{minipage}
        &  \\
    正方体
        & \begin{minipage}[c]{2cm}
            \changfangti{1}{a}{1}{a}{1}{a}
          \end{minipage}
        & $\text{体积} = \text{棱长}^3$
        & \begin{minipage}[c][6em]{\lend}
            $V$ —— 体积 \\
            $a$ —— 棱长
          \end{minipage}
        &  \\
    圆柱
        & \begin{minipage}[c]{2cm}
            % 使用 tikz-3dplot 绘制圆柱
\tdplotsetmaincoords{50}{0}
\begin{tikzpicture}[tdplot_main_coords, >=Stealth, scale=0.7]
    \pgfmathsetmacro{\r}{1}
    \pgfmathsetmacro{\h}{3}

    \draw plot[smooth,variable=\t,domain=\tdplotmainphi:\tdplotmainphi-180]
            ({\r*cos(\t)},{\r*sin(\t)},\h)
        -- plot[smooth,variable=\t,domain=\tdplotmainphi-180:\tdplotmainphi]
            ({\r*cos(\t)},{\r*sin(\t)},0) -- cycle;
    \draw plot[smooth,variable=\t,domain=0:360] ({\r*cos(\t)},{\r*sin(\t)},\h);
    \draw [dashed] plot[smooth,variable=\t,domain=0:360] ({\r*cos(\t)},{\r*sin(\t)},0);

    \coordinate (O) at (0, 0, \h);      % 上底圆心
    \coordinate (A) at ({\r*cos(-50}, {\r*sin(-50)}, \h);
    \coordinate (B) at ({\r*cos(40}, {\r*sin(40)}, \h);

    \coordinate (O') at (0, 0, 0);      % 下底圆心
    \coordinate (A') at ({\r*cos(-50}, {\r*sin(-50)}, 0);
    \coordinate (B') at ({\r*cos(40}, {\r*sin(40)}, 0);
    \coordinate (C') at ({\r*cos(180}, {\r*sin(180)}, 0);

    \draw [dash dot] ($(O)!1.5!(A)$) -- ($(O)!-1.5!(A)$);
    \draw [dash dot] ($(O)!1.5!(B)$) -- ($(O)!-1.5!(B)$);

    \draw [dash dot] ($(O')!1.5!(A')$) -- ($(O')!-1.5!(A')$);
    \draw [dash dot] ($(O')!1.5!(B')$) -- ($(O')!-1.5!(B')$);
    \draw (O) -- (O') node [midway, right] {$h$};
    \draw [->] (O') -- (C') node [pos=0.6, above, inner sep=0.1em] {$r$};
\end{tikzpicture}

          \end{minipage}
        & $\text{体积} = \text{底面积} \times \text{高}$
        & \begin{minipage}[c][6em]{\lend}
            $V$ —— 体积 \\
            $r$ —— 底半径 \\
            $h$ —— 高
          \end{minipage}
        &  \\
    圆锥
        & \begin{minipage}[c]{2cm}
            % 使用 tikz-3dplot 绘制圆锥
\tdplotsetmaincoords{50}{0}
\begin{tikzpicture}[tdplot_main_coords, >=Stealth, scale=0.7]
    \pgfmathsetmacro{\r}{1}
    \pgfmathsetmacro{\h}{3}

    \coordinate (O) at (0, 0, \h);      % 锥顶坐标
    \coordinate (O') at (0, 0, 0);      % 下底圆心
    \coordinate (A') at ({\r*cos(-50}, {\r*sin(-50)}, 0);
    \coordinate (B') at ({\r*cos(40}, {\r*sin(40)}, 0);
    \coordinate (C') at ({\r*cos(180}, {\r*sin(180)}, 0);

    \draw [dashed] plot[smooth,variable=\t,domain=\tdplotmainphi+30:\tdplotmainphi+150] ({\r*cos(\t)},{\r*sin(\t)},0);
    \draw plot[smooth,variable=\t,domain=\tdplotmainphi:\tdplotmainphi-180]
            ({\r*cos(\t)},{\r*sin(\t)}, 0)
        -- plot[smooth,variable=\t,domain=\tdplotmainphi-180:\tdplotmainphi]
            (0,0, \h) -- cycle;

    \draw [dash dot] ($(O')!1.5!(A')$) -- ($(O')!-1.5!(A')$);
    \draw [dash dot] ($(O')!1.5!(B')$) -- ($(O')!-1.5!(B')$);
    \draw (O) -- (O') node [midway, right, inner sep=0.1em] {$h$};
    \draw [->] (O') -- (C') node [midway, above, inner sep=0.1em] {$r$};
\end{tikzpicture}

          \end{minipage}
        & $\text{体积} = \dfrac{1}{3} \times \text{底面积} \times \text{高}$
        & \begin{minipage}[c][6em]{\lend}
            $V$ —— 体积 \\
            $r$ —— 底半径 \\
            $h$ —— 高
          \end{minipage}
        &  \\
    球
        & \begin{minipage}[c]{2cm}
            % 纯 2D 手工绘制球
\begin{tikzpicture}[>=Stealth, scale=0.8]
    \pgfmathsetmacro{\r}{1}
    \coordinate (O) at (0, 0);      % 球心

    \draw (O) circle (\r);
    \draw (-\r,0) arc [start angle=180, end angle=360, x radius=\r, y radius=\r/2];
    \draw [dashed] (\r,0) arc [start angle=0, end angle=180, x radius=\r, y radius=\r/2];

    % 这里"绘制"一小段 arc 是为了获取坐标
    \path (-\r,0) arc [start angle=180, end angle=210, x radius=\r, y radius=\r/2] coordinate (A);
    \fill[fill=black] (O) circle (1pt);
    \draw[->] (O) -- node[above]{$r$} (A);
\end{tikzpicture}

          \end{minipage}
        & $\text{体积} = \dfrac{4}{3} \pi \times \text{球半径}^3$
        & \begin{minipage}[c][6em]{\lend}
            $V$ —— 体积 \\
            $r$ —— 球半径
          \end{minipage}
        &  \\
\end{longtblr}

\xiaoti{练习本每本定价 9 分, 铅笔每支定价 6 分。}
\begin{xiaoxiaotis}

    \xxt{买 5 本练习本和 4 支铅笔共需多少钱?}

    \xxt{买 2 本练习本和 $y$ 支铅笔共需多少钱?}

    \xxt{买 $x$ 本练习本和 $3$ 支铅笔共需多少钱?}

    \xxt{买 $x$ 本练习本和 $y$ 支铅笔共需多少钱?}

\end{xiaoxiaotis}


\xiaoti{用代数式表示:}
\begin{xiaoxiaotis}

    \xxt{ $a$ 与 $b$ 的和乘以 $6$ 的积;}

    \xxt{$x$ 与 $y$ 的积的 $2$ 倍;}

    \xxt{$a$,$b$ 两数的积与 $1$ 的和;}

    \xxt{$a$ 与 $5$ 的差除以 $b$ 的商。}

\end{xiaoxiaotis}


\xiaoti{用代数式表示:}
\begin{xiaoxiaotis}

\begin{enhancedline}
    \xxt{$x$ 的 $1\dfrac{1}{2}$ 倍与 $7$ 的和;}

    \xxt{$y$ 的 $b$ 倍的 $\dfrac{2}{3}$;}

    \xxt{$x$ 如的相反数与 $-2$ 的差;}
\end{enhancedline}

    \xxt{$a$ 与 $b$ 的和除以 $c$ 的商;}

    \xxt{比 $x$ 与 $y$ 的积大 $13$ 的数;}

    \xxt{比 $a$ 的 $160\%$ 少 $108$ 的数;}

    \xxt{$a$ 除 $b$ 的商与 $c$ 的倒数的差;}

    \xxt{$x$ 平方的 $3$ 倍与 $y$ 的 $25\%$ 的和;}

    \xxt{$m$,$n$ 两数的立方差;}

    \xxt{$m$,$n$ 两数差的立方。}
\end{xiaoxiaotis}


\xiaoti{用代数式表示:}
\begin{xiaoxiaotis}

    \xxt{比 $a$,$b$ 两数的和的 $2$ 倍大 $c$ 的数;}

    \xxt{$a$,$b$,$c$ 三数和的平方;}

    \xxt{ 比 $a$,$b$ 两数的立方差的 $3$ 倍小 $c$ 的数;}

    \xxt{$a$,$b$,$c$ 三数的立方和减去 $a$,$b$,$c$ 三数的积。}

\end{xiaoxiaotis}


\xiaoti{设字母 $x$ 表示某数,用代数式表示:}
\begin{xiaoxiaotis}

    \xxt{某数的平方的 2 倍与 13 的和;}

    \xxt{$-3$ 的绝对值与某数的差的 3 倍;}

    \xxt{某数与这个数的相反数的差;}

    \xxt{某数的立方与 3 的差除以这个数的商。}

\end{xiaoxiaotis}

\xiaoti{设甲数为$x$,乙数为 $y$,用代数式表示:}
\begin{xiaoxiaotis}

    \xxt{甲乙两数乘积的 3 倍;}

    \xxt{甲乙两数和的平方与甲乙两数差的平方的积;}

    \xxt{甲数的 2 倍与乙数除以 3 的差;}

    \xxt{甲乙两数的平方和与甲乙两数乘积的和。}
\end{xiaoxiaotis}


\xiaoti{设甲数为 $x$,用代数式表示乙数:}
\begin{xiaoxiaotis}

    \xxt{甲数比乙数少 10 ;}

    \xxt{甲乙两数的差为 $-15$ ;}

    \xxt{甲数的 3 倍比乙数多 6 ;}

    \xxt{甲数的 2 倍比乙数少 9 。}
\end{xiaoxiaotis}

\xiaoti{用代数式表示下列图中阴影部分的面积:}

\begin{figure}[htbp]
    \centering
    \begin{minipage}{7cm}
    \centering
    \begin{tikzpicture}[>=Stealth,
    every node/.style={fill=white, inner sep=1pt},
]
    \pgfmathsetmacro{\r}{0.8}
    \coordinate (O) at (1.5, 2);
    \coordinate (A) at ($(O) + (-\r, 0)$);
    \coordinate (B) at ($(O) + (\r, 0)$);
    \path (A) arc [start angle=180, end angle=210, x radius=\r, y radius=\r] coordinate (C);

    \draw [fill=gray!50]  (0, 0) -- (0, 2)
        -- (A) arc [start angle=180, end angle=360, x radius=\r, y radius=\r] -- (B) -- (3, 2)
        -- (3, 0) --cycle;
    \draw [dashed] (A) -- (B);
    \draw [dashed] (1.5, 2.2) -- (1.5, -0.2);
    \draw [->] (O) -- (C) node [midway, rotate=30] {$r$};

    \draw [<->] (0, -0.4) to[xianduan={above=0.2}] node [midway] {$a$} (3, -0.4);
    \draw [<->] (3.3, 0) to[xianduan={above=0.2}]  node [midway, rotate=90] {$b$} (3.3, 2);
\end{tikzpicture}

    \caption*{(1)}
    \end{minipage}
    \qquad
    \begin{minipage}{7cm}
    \centering
    \begin{tikzpicture}[>=Stealth,
    every node/.style={fill=white, inner sep=1pt},
]
    \pgfmathsetmacro{\x}{0.5}
    \draw [fill=gray!50] (\x, 0)
        -- (\x, \x)     -- (0, \x)      -- (0, 3*\x)
        -- (\x, 3*\x)   -- (\x, 4*\x)   -- (5*\x, 4*\x)
        -- (5*\x, 3*\x) -- (6*\x, 3*\x) -- (6*\x, \x)
        -- (5*\x, \x)   -- (5*\x, 0)    -- cycle;
    \draw [dashed] (3*\x, -0.2) --  +(0, 4*\x+0.4);
    \draw [dashed] (-0.2, 2*\x) --  +(6*\x+0.4, 0);

    \draw [<->] (-0.3, 0) to [xianduan={below=0.8cm}] node [pos=0.8, left=0.5em, rotate=90] {$x$} (-0.3, \x);
    \draw (0, -0.2) to [xianduan] node {$x$} (\x, -0.2);
    \draw [->] (-0.5, -0.2) -- (0, -0.2);
    \draw [->] (1, -0.2) -- (\x, -0.2);
    \draw [<->] (0, -0.5) to [xianduan={above=1.2cm}] node {$6x$} (6*\x, -0.5);
    \draw [<->] (0.3+6*\x, 0) to [xianduan={above=0.8cm}] node [midway, rotate=90] {$4x$} (0.3+6*\x, 4*\x);
\end{tikzpicture}

    \caption*{(2)}
    \end{minipage}
    \caption*{第(9)题}
\end{figure}



\xiaoti{李庄乡有棉田 $m$ 公顷, 计划每公顷加施化肥 $a$ 千克, 有稻田 $n$ 公顷, 计划每公顷加施化肥 $b$ 千克, 用代数式表示共需化肥的千克数。}

\xiaoti{某县棉花丰收, $m$ 公顷地每公顷产皮棉 $a$ 千克, $n$ 公顷地每公顷产皮棉 $b$千克。用代数式表示平均每公顷产量。}

\xiaoti{某拖拉机厂 8 月份生产手扶拖拉机 $S$ 台, 9 份的产量比 8 月份的2  倍少 5 台, 用代数式表示9 月份的产量。}

\xiaoti{用拖拉机耕地 120 公顷, 原计划每天耕 $x$ 公顷, 需要几天耕完? 如果每天多耕 5 公顷, 需要几天耕完? 比原计划提前几天耕完?}

\xiaoti{一个生产小组要制造 $a$ 个零件, 原计划每天制造 $b$ 个, 要多少天完成? 如果每天比原计划多制造 $d$ 个, 可以提前几天完成?}

\xiaoti{一个工了原有工人 $a$ 人, 今年增加了一些科技人员, 人数是原来工人人数的 $6\%$, 现在这个工厂有多少工人?}

\xiaoti{解放前, 贫农李大爷全家辛苦劳动一年, 收获粮食 $m$ 千克,其中 $85\%$ 交纳地租。用代数式表示李大爷剩下粮食的千克数。}

\xiaoti{有浓度为 $20\%$ 的盐水 $n$ 千克。 含纯盐多少千克? 含水多少千克?}

\xiaoti{某汽车修配厂 1980 年装配的汽车数比 1970 年增加 5 倍, 已知这个厂 1970 年装配的汽车是 $Q$ 辆, 用代数式表示 1980 年装配的汽车数。}

\xiaoti{顺次大 1 的整数, 如 14 、15 、16 , 叫做连续整数。 三个连续整数里,
    (1)中间的一个是 $m$, 用代数式表示其他两个;
    (2) 最大的一个是 $n$,用代数式表示其他两个。
}

\xiaoti{一列慢车从甲站开往乙站, 每小时走 56 千米; 同时, 一列快车从乙站开往甲站, 每小时走 72 千米。 $t$ 小时后两车相遇。 用代数式表示甲乙两站间的路程。}

\xiaoti{有一片稻田需要灌水,单独用甲抽水机 $a$ 小时可灌完, 单独用乙抽水机 $b$ 小时可灌完。 用代数式表示:}
\begin{xiaoxiaotis}

    \xxt{单独用甲抽水机, 1 小时能完成任务的几分之几;}

    \xxt{单独用乙抽水机, 1 小时能完成任务的几分之几;}

    \xxt{同时开动甲乙两抽水机, 1 小时能完成任务的几分之几。}

\end{xiaoxiaotis}


\xiaoti{开挖一条渠道, 甲施工队单独挖 $a$ 天可以完成。 甲施工队挖了 3 天, 余下的由其他施工队完成。 用代数式表示余下的任务。}

\end{xiaotis}

