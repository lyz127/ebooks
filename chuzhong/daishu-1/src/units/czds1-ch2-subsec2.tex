\subsection{代数式的值}\label{subsec:2-2}

\begin{enhancedline}

我们知道, 底是 $a$ 厘米, 高是 $h$ 厘米的三角形的面积, 可用代数式
$$ \dfrac{1}{2}ah \text{(平方厘米)} $$
来表示。现在,我们根据这个代数式来计算下面三个三角形的面积(图 \ref{fig:2-3})。

\begin{figure}[htbp]
    \centering
    \begin{tikzpicture}
    \tikzset{
        pics/sanjiao/.style n args={5}{
        code = {
            \coordinate (A) at (0, 0);
            \coordinate (B) at (#1, 0);
            \coordinate (C) at (1.8, #2);
            \coordinate (D) at (1.8, 0);
            \draw (A) -- (B) node [midway, below] {#3} -- (C) -- cycle;
            \draw (C) -- (D) node [pos=0.7, fill=white, inner sep=1pt]  {#4};
            \node at (1.5, -1) {(#5)};
        }}
    }
    \draw (0,0) pic {sanjiao={3}{2}{3厘米}{2厘米}{1}};
    \draw (4,0) pic {sanjiao={3.5}{2.5}{3{\large $\frac{1}{2}$} 厘米}{2{\large $\frac{1}{4}$} 厘米}{2}};
    \draw (8,0) pic {sanjiao={3.8}{1.9}{3.8厘米}{1.9厘米}{3}};
\end{tikzpicture}

    \caption{}\label{fig:2-3}
\end{figure}

在 (1) 中,$a = 3$,$h = 2$,
$$ \dfrac{1}{2}ah = \dfrac{1}{2} \times 3 \times 2 = 3 ; $$

在 (2) 中,$a = 3\dfrac{1}{2}$,$h = 2\dfrac{1}{4}$,
$$ \dfrac{1}{2}ah = \dfrac{1}{2} \times 3\dfrac{1}{2} \times 2\dfrac{1}{4} = 3\dfrac{15}{16} ; $$

在 (3) 中,$a = 3.8$,$h = 1.9$,
$$ \dfrac{1}{2}ah = \dfrac{1}{2} \times 3.8 \times 1.9 = 3.61 \juhao $$

所以,三个三角形的面积分别是 3 平方厘米, $3\dfrac{15}{16}$ 平方厘米和 3.61 平方厘米。

用数值代替代数式里的字母, 计算后所得的结果,叫做\zhongdian{代数式的值}。

从上面的例子中可以看到, 改变字母 $a$,$h$ 所取的值时, 代数式 $\dfrac{1}{2}ah$ 的值也就随着改变。
因此, 代数式的值是由代数式里字母所取的值确定的。



代数式里的字母, 虽然可以取各种不同的数值, 但是这些数值不应当使代数式和它所表示的实际数量失掉意义。
例如,上例中的代数式 $\dfrac{1}{2}ah$, 由于它所表示的是三角形的面积, 所以 $a$ 和 $h$ 都不能取负数或者零。
又如, 在代数式 $\dfrac{12}{x}$ 里, 因为零不能做除数, 所以 $x$ 不能取零。

\begin{wrapfigure}[10]{r}{8cm}
    \centering
    \begin{tikzpicture}[>=Stealth]
    \tikzset{
        seagull/.pic={
            \draw [rounded corners=6pt] (0, 0)
                arc [start angle=250, end angle=360, radius=1.2]
                arc [start angle=220, end angle=140, radius=2]
                arc [start angle=0, end angle=110, radius=1.2]
                .. controls (-1,3) and (-1,1) .. (0,0);
        }
    }
    \draw (0, 0) pic {seagull};
    \draw (6, 4.8) pic [rotate=180] {seagull};

    \draw   (0.6, 5.2) node {$x$}
            (5.5, 5.2) node {$-2x+5$};
    %\draw [->] (0.8, 4.5) node[left] {$4$}  -- (5.2, 4.4) node [right] {$-3$};
    \foreach \x/\y/\h in {4/-3/4, 0/\hphantom{0}5/3, -5/15/2, \vdots/\hphantom{0}\vdots/1} {
        \draw [->] (0.8, \h) node[left] {$\x$}  -- (5.2, \h) node [right] {$\y$};
    }
\end{tikzpicture}

    \caption{}\label{fig:2-4}
\end{wrapfigure}

\liti 根据下面所给的值, 求代数式 $-2x + 5$ 的值:

\threeInLine[6em]{(1)$x=4$;}{(2)$x = 0$;}{(3)$x = -5$。}


\jie (1)当 $x = 4$ 时,

\hspace*{2em} $-2x + 5 = -2 \times 4 + 5 = -3$;

(2)当 $x = 0$ 时,

\hspace*{2em} $-2x + 5 = -2 \times 0 + 5 = 5$;

(3)当 $x = -5$ 时,

\hspace*{2em} $-2x + 5 = -2 \times (-5) + 5 = 15$。

从 例 1 可以看出,当我们给 $x$ 一个值, 代数式 $-2x + 5$ 就取得一个确定的值(图 \ref{fig:2-4})。



\lianxi

填表:

\begin{table}[H]
\begin{tblr}{
    hlines, vlines,
    cells={5em, c, $$},
}
    x       & -2  &  -1 & 0   & 1 & 2 \\
    4x - 5  & -13 &     &     &   &   \\
    x^2 + 2 &     &     & 2   &   &
\end{tblr}
\end{table}

\lianxijiange

\liti 求 $a = -2$ 时,代数式 $2a^3 - \dfrac{1}{2}a^2 + 3$ 的值。

\jie 当  $a = -2$ 时,

\hspace*{2em} $\begin{aligned}
    2a^3 - \dfrac{1}{2}a^2 + 3 &= 2 \times (-2)^3 - \dfrac{1}{2} \times (-2)^2 + 3 \\
        &= -16 - 2 + 3 \\
        &= -15 \juhao
\end{aligned}$

\liti 当 $x = \dfrac{1}{2}$,$y = -2$  时,求下列代数式的值:

\twoInLine{(1)$2x^2 - y + 3$;}{(2)$\dfrac{4x - 2y}{xy}$。}

\jie (1)当 $x = \dfrac{1}{2}$,$y = -2$  时,

\hspace*{2em} $\begin{aligned}
    2x^2 - y + 3 &= 2 \times \left(\dfrac{1}{2}\right)^2 - (-2) + 3 \\
        &= 5\dfrac{1}{2};
\end{aligned}$

(2)当 $x = \dfrac{1}{2}$,$y = -2$  时,

\hspace*{2em} $\begin{aligned}
    \dfrac{4x - 2y}{xy} &= \dfrac{4 \times \dfrac{1}{2} - 2 \times (-2)}{\dfrac{1}{2} \times (-2)} \\
        &= \dfrac{2 + 4}{-1} \\
        &= -6 \juhao
\end{aligned}$


\liti 当 (1)$x = -5$,$y = 3$ 时;(2)$x = 5$,$y = -3$ 时,求代数式
$$ |x| + |y| - 2|x| \cdot |y| $$
的值。

\jie (1)$x = -5$,$y = 3$ 时,

\hspace*{2em} $\begin{aligned}
    |x| + |y| - 2|x| \cdot |y| &= |-5| + |3| - 2 |-5| \cdot |3| \\
        &= 5 + 3 - 2 \times 5 \times 3 \\
        &= -22;
\end{aligned}$

(2)$x = -5$,$y = 3$ 时;

\hspace*{2em} $\begin{aligned}
    |x| + |y| - 2|x| \cdot |y| &= |5| + |-3| - 2 |5| \cdot |-3| \\
        &= 5 + 3 - 2 \times 5 \times 3 \\
        &= -22 \juhao
\end{aligned}$


\lianxi
\begin{xiaotis}

\xiaoti{求 $x = -2$ 时, 代数式 $x^3 - 3x^2 + 2x + 7$ 的值。}

\xiaoti{当 $x = -3$,$y = 4$ 时,求代数式 $x^2 + 3xy - y^2 -5$ 的值。}

\xiaoti{求 $a = 2$,$b = -1$,  $c = -1\dfrac{1}{2}$ 时,下列各代数式的值:}
\begin{xiaoxiaotis}

    \twoInLineXxt{$a^2 - b^2 + 2bc$;}{$\dfrac{2c}{a + b}$。}

\end{xiaoxiaotis}


\xiaoti{当  $x = \dfrac{1}{2}$,$y = -2$ 时,求下列各代数式的值:}
\begin{xiaoxiaotis}

    \twoInLineXxt{$|x| + 3|y|$;}{$|x + 3y|$。}

\end{xiaoxiaotis}

\end{xiaotis}
\lianxijiange

\begin{wrapfigure}[5]{r}{5cm}
    \centering
    \begin{tikzpicture}[>=Stealth,
    every node/.style={fill=white, inner sep=1pt},
]
    \draw [ground] (0, 0) -- (1, 0) -- (2, -1) -- (2.8, -1) -- (3.8, 0) -- (4.8, 0);
    \draw [<->] (1, 0.3) to [xianduan={below=0.3cm}] node {$a$} (3.8, 0.3);
    \draw [<->] (2, -1.5) to [xianduan={above=0.5cm}] node {$b$} (2.8, -1.5);
    \draw (2, -1) -- (0, -1);
    \draw [<->] (0.3, 0) to [xianduan] node {$h$} (0.3, -1);
\end{tikzpicture}

    \caption{}\label{fig:2-5}
\end{wrapfigure}

\liti 图 \ref{fig:2-5} 中的渠道横断面是梯形, 用代数式表示它的面积, 并计算当
$a = 2.8$, $b = 0.8$, $h = 1$ (单位:米) 时, 渠道横断面的面积。

\jie 因为渠道的横断面是梯形, 它的两底分别是 $a$, $b$, 高是 $h$, 所以, 表示渠道横断面面积的代数式是
$$ \dfrac{1}{2} (a + b) h \juhao $$

当 $a = 2.8$, $b = 0.8$, $h = 1$ 时,

\hspace*{2em} $\begin{aligned}
    \dfrac{1}{2} (a + b) h & = \dfrac{1}{2} (2.8 + 0.8) \times 1 \\
        &= \dfrac{1}{2} \times 3.6 \times 1 \\
        &= 1.8 \juhao
\end{aligned}$

答:当 $a = 2.8$, $b = 0.8$, $h = 1$ (单位:米) 时, 渠道横断面的面积 1.8 平方米。

\liti 工人师傅为了便于计算, 常把圆柱形钢管堆成如图 \ref{fig:2-6}(1) 的形状, 下面一层比上面一层多放一根。
只要数出顶层的根数 $a$, 底层的根数 $b$ 和层数 $n$,就可用公式  $\dfrac{n(a + b)}{2}$ 算出这堆钢管的根数。
计算当 $n = 6$,$a = 4$,$b = 9$ 时, 这堆钢管的根数。

\begin{figure}[htbp]
    \centering
    \begin{tikzpicture}
    \tikzset{
        yuanzhu/.pic = {
            \pgfmathsetmacro{\angle}{45}
            \pgfmathsetmacro{\len}{2}
            \pgfmathsetmacro{\r}{0.2}

            \coordinate (O) at (0, 0);
            \coordinate (O') at ({\len * cos(\angle)},  {\len * sin(\angle)});
            \coordinate (dA) at ({\r * cos(90 + \angle)}, {\r * sin(90 + \angle)});
            \coordinate (dB) at ({\r * cos(270 + \angle)}, {\r * sin(270 + \angle)});

            \draw [fill=white] (O) circle (\r);
            \draw [fill=white]
                ($(O) + (dA)$) -- ($(O') + (dA)$)
                arc [start angle=90 + \angle, end angle=-90 + \angle, radius=\r]
                -- ($(O) + (dB)$)
                arc [start angle=-90 + \angle, end angle=90 + \angle, radius=\r];
        }
    }

    \begin{scope}
        \foreach \y in {1,...,6}
            \foreach \x [parse=true] in {1,...,10-\y}
                \draw (0.2*\y+0.4*\x,0.33*\y) pic{yuanzhu};

    \draw[decorate,decoration={brace,mirror,amplitude=0.2cm}] (4.4, 3.7) -- (3.0, 3.7)
        node [pos=0.5, above=0.4em, align=center] {$a$};
    \draw[decorate,decoration={brace,mirror,amplitude=0.2cm}] (0.4, 0) -- (4, 0)
        node [pos=0.5, below=0.3em, align=center] {$b$};
    \draw[decorate,decoration={brace,mirror,amplitude=0.2cm}] (1.4, 2.2) -- (0.3, 0.4)
        node [pos=0.5, above left=0.3em, align=center] {$n$};
    \node at (2, -0.8) {(1)};
    \end{scope}

    \begin{scope}[xshift=5cm]
        \foreach \y in {1,...,6}
            \foreach \x [parse=true] in {1,...,10-\y}
                \draw (0.2*\y+0.4*\x,0.33*\y) circle(0.2);
        \foreach \y in {1,...,6}
            \foreach \x [parse=true] in {1,...,3+\y}
                \draw [densely dotted] (4-0.2*\y+0.4*\x,0.33*\y) circle(0.2);
        \node at (3, -0.8) {(2)};
    \end{scope}
\end{tikzpicture}

    \caption{}\label{fig:2-6}
\end{figure}

\jie 当 $n = 6$,$a = 4$,$b = 9$ 时,

\hspace*{2em} $\dfrac{n(a + b)}{2} = \dfrac{6 \times (4 + 9)}{2} = 39$。

答: 当 $n = 6$,$a = 4$,$b = 9$ 时,有 39 根钢管。


\lianxi
\begin{xiaotis}

\xiaoti{如图,堤坝的横截面是梯形,用代数式表示它的面积,
    并计算 $a = 2$,$b = 13$,$h = 3$ (单位: 米) 叶,堤坝的横截面的面积。
}

\begin{figure}[htbp]
    \centering
    \begin{tikzpicture}[>=Stealth,
    every node/.style={fill=white, inner sep=1pt},
]
    \draw (0, 0) -- (6.5, 0) -- (5, 1.5) -- (4, 1.5) --cycle;
    \draw [<->] (4, 1.7) to [xianduan] node {$a$} (5, 1.7);
    \draw [<->] (0, -0.2) to [xianduan] node {$b$} (6.5, -0.2);
    \draw [<->] (6.7, 0) to [xianduan={above=1.7cm}] node {$h$} (6.7, 1.5);
\end{tikzpicture}

    \caption*{(第 1 题)}
\end{figure}

\begin{figure}[htbp]
    \centering
    \begin{tikzpicture}
    \tikzset{
        yuanzhu/.pic = {
            \pgfmathsetmacro{\angle}{45}
            \pgfmathsetmacro{\len}{2}
            \pgfmathsetmacro{\r}{0.2}

            \coordinate (O) at (0, 0);
            \coordinate (O') at ({\len * cos(\angle)},  {\len * sin(\angle)});
            \coordinate (dA) at ({\r * cos(90 + \angle)}, {\r * sin(90 + \angle)});
            \coordinate (dB) at ({\r * cos(270 + \angle)}, {\r * sin(270 + \angle)});

            \draw [fill=white] (O) circle (\r);
            \fill[fill=black] (O) circle (1pt);
            \draw [fill=white]
                ($(O) + (dA)$) -- ($(O') + (dA)$)
                arc [start angle=90 + \angle, end angle=-90 + \angle, radius=\r]
                -- ($(O) + (dB)$)
                arc [start angle=-90 + \angle, end angle=90 + \angle, radius=\r];
        }
    }

    \begin{scope}
        \pgfmathsetmacro{\angle}{45}
        \pgfmathsetmacro{\len}{2}
        \pgfmathsetmacro{\y}{8}
        % \draw (0, 0) -- (0.2*\y, 0.33*\y);
        % \draw (0.4, 0) -- (-0.2*\y+0.4, 0.33*\y);
        \path [name path=a1, xshift=0.25cm]  (0.2*\y, 0.33*\y) -- ($(0.2*\y, 0.33*\y)!1.1!(0, 0)$);
        \path [name path=a2, xshift=-0.25cm] (-0.2*\y+0.4, 0.33*\y) -- ($(-0.2*\y+0.4, 0.33*\y)!1.1!(0.4, 0)$);
        \path [name intersections={of=a1 and a2, by=BMa}];      % Bottom Middle point a
        \coordinate (TRa) at (0.25 + 0.2*\y, 0.33*\y);          % Top Right point a
        \coordinate (TLa) at (-0.25 -0.2*\y+0.4, 0.33*\y);      % Top Left point a
        \coordinate (BMb) at ($(BMa) + ({\len * cos(\angle)},  {\len * sin(\angle)})$);
        \coordinate (TRb) at ($(TRa) + (({\len * cos(\angle)}, {\len * sin(\angle)})$);
        \coordinate (TLb) at ($(TLa) + ({\len * cos(\angle)},  {\len * sin(\angle)})$);

        \pgfmathsetmacro{\delta}{0.1}
        \coordinate (BMea) at ($(BMa) + (0, -0.1)$);            % Bottom Middle extend point a
        \coordinate (BMeb) at ($(BMb) + (0, -0.1)$);
        \coordinate (BMec) at ($(BMb) + ({\delta * cos(\angle)},  {\delta * sin(\angle)})$);
        \coordinate (TLea) at ($(TLa) + (-0.1, 0)$);
        \coordinate (TLeb) at ($(TLb) + (-0.1, 0)$);
        \coordinate (TLec) at ($(TLeb) + ({\delta * cos(\angle)},  {\delta * sin(\angle)})$);
        \coordinate (TRea) at ($(TRa) + (0.1, 0)$);
        \coordinate (TReb) at ($(TRb) + (0.1, 0)$);
        \coordinate (TRec) at ($(TReb) + ({\delta * cos(\angle)},  {\delta * sin(\angle)})$);

        \coordinate (TMPLa) at ($(BMea) + (-3, 0)$);            % Temp Left point a
        \coordinate (TMPRa) at ($(BMea) + (3, 0)$);
        \coordinate (BLea) at ($(TMPLa)!(TLea)!(TMPRa)$);       % Bottom Left extend point a
        \coordinate (BRea) at ($(TMPLa)!(TRea)!(TMPRa)$);       % Bottom Right extend point a

        \coordinate (TMPLc) at ($(BMec) + (-3, 0)$);
        \coordinate (TMPRc) at ($(BMec) + (3, 0)$);
        \coordinate (BLec) at ($(TMPLc)!(TLec)!(TMPRc)$);
        \coordinate (BRec) at ($(TMPLc)!(TRec)!(TMPRc)$);

        \draw (BLea) -- (BLec) -- (TLec) -- (TLea) -- cycle;
        \draw ($(BLea)!0.5!(TLea)$) -- ($(BLec)!0.5!(TLec)$);
        \draw ($(BLea)!0.5!(BLec)$) -- ($(TLea)!0.5!(TLec)$);

        \draw ($(BLea)!0.5!(BLec)$) -- ($(BRea)!0.5!(BRec)$);

        \draw [pattern={mylines[angle=-45, distance={4pt}]}]
            (BMa) -- (BMb) -- (TLb) -- (TLa) -- cycle;
        \draw [pattern={mylines[angle=-45, distance={4pt}]}]
            (BMb) -- (TLb) -- (TRb) -- cycle;

        \foreach \y in {1,...,6}
            \foreach \x [parse=true] in {1,...,\y}
                \draw (-0.2*\y+0.4*\x,0.33*\y) pic{yuanzhu};

        \draw [fill=white] (BMa) -- (BMb) -- (TRb) -- (TRa) -- cycle;



        \draw [fill=white] (BMa) -- (BMea) -- (TLea) -- (TLa) -- cycle;
        \draw [fill=white] (BMa) -- (BMea) -- (TRea) -- (TRa) -- cycle;
        \draw [fill=white] (TRa) -- (TRb) -- (TReb) -- (TRea) -- cycle;
        \draw [fill=white] (BMea) -- (TRea) -- (TReb) -- (BMeb) -- cycle;
        \draw (TReb) -- (TLeb) -- (TLea);
        \draw [fill=white] (TLeb) -- (TLec) -- (TRec) -- (TReb) -- cycle;

        \draw (BRea) -- (BRec) -- (TRec) -- (TRea) -- cycle;
        \draw ($(BRea)!0.5!(TRea)$) -- ($(BRec)!0.5!(TRec)$);
        \draw ($(BRea)!0.5!(BRec)$) -- ($(TRea)!0.5!(TRec)$);

        \draw  (TLea) -- (BLea) -- (BRea) -- (TRea);

        \node at (0.2, -0.5) {(1)};
    \end{scope}

    \begin{scope}[xshift=7cm]
        \foreach \y in {1,...,6}
            \foreach \x [parse=true] in {1,...,\y}
                \draw (-0.2*\y+0.4*\x,0.33*\y) circle(0.2);

        \foreach \y in {1,...,6}
            \foreach \x [parse=true] in {1,...,7-\y}
                \draw [densely dotted] (0.1+0.2*\y+0.4*\x,0.33*\y) circle(0.2);

        \pgfmathsetmacro{\y}{8}
        %\draw (0, 0) -- (0.2*\y, 0.33*\y);
        %\draw (0.4, 0) -- (-0.2*\y+0.4, 0.33*\y);

        \path [name path=a1, xshift=-0.25cm] (-0.2*\y+0.4, 0.33*\y) -- ($(-0.2*\y+0.4, 0.33*\y)!1.1!(0.4, 0)$);
        \path [name path=a2, xshift=0.25cm]  (0.2*\y, 0.33*\y) -- ($(0.2*\y, 0.33*\y)!1.1!(0, 0)$);
        \path [name path=a3, xshift=0.25cm]
            (0.1+0.2*6+0.4*1,0.33*6)  coordinate (tmpa)
            (0.1+0.2*1+0.4*6,0.33*1)  coordinate (tmpb)
            ($(tmpa)!1.3!(tmpb)$) -- ($(tmpb)!1.3!(tmpa)$);
        \path [name intersections={of=a2 and a3, by=C}];
        \path [name path=top] ($(C) + (-4, 0)$) -- ($(C) + (1, 0)$);
        \path [name intersections={of=a1 and a2, by=B}];
        \path [name path=bottom] ($(B) + (-1, 0)$) -- ($(B) + (4, 0)$);
        \path [name intersections={of=a1 and top, by=A}];
        \path [name intersections={of=a3 and bottom, by=D}];

        \draw (A) -- (B) -- (C) -- (D);

        % \fill[fill=red] (C) circle (1pt);
        \node at (1, -0.5) {(2)};
    \end{scope}
\end{tikzpicture}

    \caption*{(第 2 题)}
\end{figure}

\xiaoti{有一种放铅笔的 $V$ 形槽如图所示, 第一层放 1 支, 第二层放 2 支, 依次每层增放1 支。
    只要数一数顶层的支数 $n$, 就可用公式 $\dfrac{n (n + 1)}{2}$ 算出槽内铅笔的支数.
    分别计算当 $n = 6$,$n = 11$ 时, 槽内铅笔的支数。
}


\end{xiaotis}

\end{enhancedline}

