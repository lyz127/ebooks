\subsection{近似数和有效数字}\label{subsec:1-15}

我们看:

(1) 初一(2) 班有 48 个同学,

(2) 东风厂有 126 台机床, \\
这里的 48 、126 是与实际完全符合的准确数;

(3) 月球离地球的距离约是 38 万公里,

(4) 李为民的身高约是 1.57 米,\\
这里的 38 万、1.57 是与实际接近的近似数。

月球离地球的距离约是 38 万公里,是经过四舍五入得来的,这表示月球离地球的距离大于或等于 37.5 万公里而小于38.5 万公里(图 \ref{fig:1-21})。

\begin{figure}[htbp]
    \centering
    \begin{tikzpicture}
    \draw (0,0) -- (6,0);

    \foreach \x in {0.2, 0.4, ..., 6} {
        \draw (\x, 0.2) -- (\x, 0);
    }

    \foreach \x in {0, 2, ..., 6} {
        \draw (\x, 0.3) -- (\x, 0);
    }

    \foreach \x/\text in {1/37, 3/38, 5/39} {
        \draw (\x, 0) -- (\x, 0.4) node [above] {$\text$};
    }

    \filldraw [pattern=north west lines, thick] (2, 0) rectangle (4, -0.3);

    \draw [|-|] (1.2, -1.3) -- (3.2, -1.3) node [pos=0.5, above] {万公里};
\end{tikzpicture}

    \caption{}\label{fig:1-21}
\end{figure}


李为民的身高约是 1.57 米,表示李为民的身高大于或等于 1.565 米而小于 1.575 米(图 \ref{fig:1-22})。

\begin{figure}[htbp]
    \centering
    \begin{tikzpicture}
    % 绘制左侧的小图
    \def\x{0}
    \def\factor{0.5}
    \draw (\x, -0.2) -- (\x, 1.3);
    \foreach \y in {0.1, 0.2, ..., 1.9} {
        \draw (-0.1*\factor+\x, \y*\factor) -- (\x, \y*\factor);
    }
    \foreach \y in {0.5, 1, 1.5} {
        \draw (-0.2*\factor+\x, \y*\factor) -- (\x, \y*\factor);
    }
    \foreach \y in {0, 1, 2} {
        \draw (-0.3*\factor+\x, \y*\factor) -- (\x, \y*\factor);
    }
    \foreach \y/\text in {0/1.56, 1/1.57, 2/1.58} {
        \draw (-0.3*\factor+\x, \y*\factor) node [left] {\tiny $\text$};
    }
    \filldraw [pattern=north east lines, thick] (\x, 0.5*\factor) rectangle (0.2*\factor+\x, 1.5*\factor);

    \draw [|-|] (1.5, 0) -- (1.5, 1*\factor) node [pos=0.5, left] {\tiny 0.01米};


    % 绘制右侧的大图
    \def\x{4}
    \def\factor{1.5}
    \draw (\x, 0) -- (\x, 3);
    \foreach \y in {0.1, 0.2, ..., 1.9} {
        \draw (-0.1*\factor+\x, \y*\factor) -- (\x, \y*\factor);
    }
    \foreach \y in {0.5, 1, 1.5} {
        \draw (-0.2*\factor+\x, \y*\factor) -- (\x, \y*\factor);
    }
    \foreach \y in {1} {
        \draw (-0.3*\factor+\x, \y*\factor) -- (\x, \y*\factor);
    }
    \foreach \y/\text in {0.5/1.565, 1/1.57, 1.5/1.575} {
        \draw (-0.3*\factor+\x, \y*\factor) node [left] {\tiny $\text$};
    }
    \filldraw [pattern=north east lines, thick] (\x, 0.5*\factor) rectangle (0.2*\factor+\x, 1.5*\factor);

    % 绘制 圆圈 及 左右两图之间的连线
    \draw (\x, 1.5) circle(1.5);
    \draw (\x, 1.5) + (160:1.5) -- (0.1, 0.5);
    \draw (\x, 1.5) + (190:1.5) -- (0.1, 0.5);
\end{tikzpicture}

    \caption{}\label{fig:1-22}
\end{figure}

我们说,上面的近似数 38 万,精到万位;近似数 1.57, 精确到百分位(或精确到 0.01)。
一般地,一个近似数,四舍五入到哪一位,就说这个近似数精确到哪一位。

这时,从左边第一个不是零的数字起,到这一位数字止,所有的数字,都叫做这个数的\zhongdian{有效数字}。
如上面的近似数 38 万有两个有效数字 3 、8 ; 近似数 1.57 有三个有效数字 1 、5 、7 。

\liti 下列由四舍五入得到的近似数,各精确到哪一位,各有几个有效数字?
\begin{xiaoxiaotis}

    \begin{tblr}{columns={10em, l, colsep=0pt}}
        \xxt{$10$ 亿;} & \xxt{$507$ 万;} & \xxt{$43.8$;} \\
        \xxt{$0.002$;} & \xxt{$0.03086$;} & \xxt{$2.4$ 万。}
    \end{tblr}

\end{xiaoxiaotis}

\jie (1) 10 亿,精确到亿位,有两个有效数字 1、0;

(2) 507 万, 精确到万位, 有三个有效数字 5、0、7;

(3) 43.8 , 精确到十分位(即精确到 0.1),有三个有效数字 4、3、8;

(4) 0.002, 精确到千分位(即精确到 0.001), 有一个有效数字 2;

(5) 0.03086,精确到十万分位(即精确到 0.00001),有四个有效数字 3、0、8、6;

(6) 2.4 万,精确到千位,有两个有效数字 2 、4。


\lianxi

(口答)圆周率 $\pi = 3.14159265\cdots\cdots$。取近似值 3.14,是精确到哪一位,有几个有效数字?
取近似 3.142 呢? 取近似值 3.1416 呢?

\lianxijiange

\begin{wrapfigure}{r}{5cm}
    \begin{minipage}{2cm}
        \centering
        \begin{tikzpicture}[scale=0.3]
    \draw (0, 0) -- (0, 10);
    \foreach \y in {1, ..., 9} {
        \draw (-0.3, \y) -- (0, \y);
    }
    \foreach \y in {0, 5, 10} {
        \draw (-0.5, \y) -- (0, \y);
    }
    \foreach \y/\text in {0/1.55, 4/1.59, 5/1.60, 6/1.61, 10/1.65} {
        \draw (-0.3, \y) node [left] {\tiny $\text$};
    }
    \filldraw [pattern=north east lines, thick] (0, 4.5) rectangle (0.5, 5.5);

    \foreach \y in {4.5, 5.5} {
        \draw (-0.3, \y) -- (0, \y);
    }
    \foreach \y/\text in {4.5/1.595, 5.5/1.605} {
        \draw (0.5, \y) node [right] {\tiny $\text$};
    }
\end{tikzpicture}

        \caption{}\label{fig:1-23}
    \end{minipage}
    \qquad
    \begin{minipage}{2cm}
        \centering
        \begin{tikzpicture}[scale=0.3]
    \draw (0, 0) -- (0, 20);
    \foreach \y in {1, ..., 19} {
        \draw (-0.3, \y) -- (0, \y);
    }
    \foreach \y in {0, 10, 20} {
        \draw (-0.5, \y) -- (0, \y);
    }
    \foreach \y/\text in {0/1.5, 10/1.6, 20/1.7} {
        \draw (-0.3, \y) node [left] {\tiny $\text$};
    }
    \filldraw [pattern=north east lines, thick] (0, 5) rectangle (0.5, 15);

    \foreach \y/\text in {5/1.55, 15/1.65} {
        \draw (0.5, \y) node [right] {\tiny $\text$};
    }
\end{tikzpicture}

        \caption{}\label{fig:1-24}
    \end{minipage}
\end{wrapfigure}


\liti 用四舍五入法,按要求对下列各数取近似值:

(1) 0.85149 (精确到千分位);

(2) 47.6 (精确到个位);

(3) 0.02076 (保留三个有效数字);

(4) 1.5972 (精确到 0.01)。

\jie (1) $0.85149 \approx 0.851$;

(2) $47.6 \approx 48$;

(3) $0.02076 \approx 0.0208$;

(4) $1.5972 \approx 1.60$。

\zhuyi 上面的 (4) 中,由四舍五入得来的 1.60, 跟 1.6 不一样,不能把最后一个 0 随便去掉。
例如,王大明身高约 1.60 米, 是说他的身高大于或等于 1.595 米而小于 1.605 米,精确到 0.01 米(图 \ref{fig:1-23});
而张小玲身高约 1.6 米,是说她的身高大于或等于 1.55 米而小于 1.65 米,精确到 0.1 米(图 \ref{fig:1-24})。


\lianxi

用四舍五入法,对下列各数按括号中的要求取近似值:

1. 56.32(保留三个有效数字)。

2. 0.6648 (精确到 0.01)。

3. 0.7096(精确到千分位)。

