\begin{tikzpicture}
    \tikzset{
        rongye/.style={
            left color=gray!50,
            right color=gray!30,
            middle color=white,
            fill opacity=0.6,
        },
        shaobei/.pic={
            \pgfmathsetmacro{\h}{#1}
            \path[rounded corners=2pt, rongye]
                (0.1, -1.95 + \h) -- (0.1, -1.95)
                -- (1.9, -1.95) -- (1.9, -1.95 + \h) -- cycle;
            % 下一行是为了消除填充色上部的圆角
            \path[rongye] (0.1, -1.95 + \h) rectangle (1.9, -1.95+\h-0.05);

            \draw [rounded corners=2pt]
                    (0, 0) -- (0.05, -0.1) -- (0.05, -2)
                -- (1.95, -2) -- (1.95, -0.1) -- (2.0, 0);
            \draw [rounded corners=2pt] (0, 0) -- (0.1, -0.1) -- (0.1, -1.95)
                -- (1.9, -1.95) -- (1.9, -0.1) -- (2.0, 0);
            \draw (0, 0) -- (2, 0);
        }
    }

    \begin{scope}
        \draw (0, 0) pic {shaobei = 0.4};
        \foreach \x/\y in {
            0.2/-1.85, 0.3/-1.8, 0.5/-1.75,
            0.8/-1.85, 1.0/-1.7, 1.2/-1.85,
            1.4/-1.75, 1.6/-1.8, 1.8/-1.75
        } {
            \draw (\x, \y) circle(0.03);
        }
        \node at (1, -2.4) {$16\%$ 氨水};
        \node at (1, -2.8) {30 千克};
    \end{scope}

    \node at (4, -0.6) {加水 $x$ 千克后};
    \draw [fill=white] (3.5, -1.1) pic [rotate=-90] {arrow={.2}{.6}{.4}{.2}};
    \draw (3, -2.5) circle(0.03) node [right] {示意含氨量};

    \begin{scope}[xshift=6cm]
        \draw (0, 0) pic {shaobei = 1.5};
        \foreach \x/\y in {
            0.9/-1.6, 1.5/-1.6, 0.4/-1.3,
            1.3/-1.25, 0.8/-1.0, 1.6/-1.0,
            0.6/-0.8, 1.0/-0.7, 1.8/-0.8
        } {
            \draw (\x, \y) circle(0.03);
        }
        \node at (1, -2.4) {$0.15\%$ 氨水};
        \node at (1, -2.8) {$(30 + x)$ 千克};
    \end{scope}
\end{tikzpicture}
