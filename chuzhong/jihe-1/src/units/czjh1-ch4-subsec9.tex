\subsection{梯形}\label{subsec:czjh1-4-9}

前面,我们研究的平行四边形是两组对边分别平行的特殊四边形。
现在,我们研究只有一组对边平行的另一种特殊四边形。

一组对边平行而另一组对边不平行的四边形叫做\zhongdian{梯形}(图 \ref{fig:czjh1-4-36})。
平行的两边叫做\zhongdian{梯形的底}(通常把较短的底叫做上底,较长的底叫做下底),
不平行的两边叫做\zhongdian{梯形的腰},两底的距离叫做\zhongdian{梯形的高}。

\begin{figure}[htbp]
    \centering
    \begin{minipage}[b]{4.5cm}
        \centering
        \begin{tikzpicture}
	\tkzDefPoints{0/0/B, 3/0/C, 1.2/1.5/A, 2.5/1.5/D, 1.6/0/F, 1.6/1.5/E}
    \tkzDrawPolygon(A,B,C,D)
	\tkzDrawSegments(E,F)
    \tkzMarkRightAngle(E,F,C)
    \tkzLabelPoints[left](A,B)
    \tkzLabelPoints[right](D,C)
    \tkzLabelPoints[above](E)
    \tkzLabelPoints[below](F)
\end{tikzpicture}


        \caption{}\label{fig:czjh1-4-36}
    \end{minipage}
    \qquad
    \begin{minipage}[b]{4.5cm}
        \centering
        \begin{tikzpicture}
	\tkzDefPoints{0/0/B, 3/0/C, 1.2/1.5/A, 3/1.5/D}
    \tkzDrawPolygon(A,B,C,D)
    \tkzMarkRightAngle(D,C,B)
    \tkzLabelPoints[left](A,B)
    \tkzLabelPoints[right](D,C)
\end{tikzpicture}


        \caption{}\label{fig:czjh1-4-37}
    \end{minipage}
    \qquad
    \begin{minipage}[b]{4.5cm}
        \centering
        \begin{tikzpicture}
	\tkzDefPoints{0/0/B, 3/0/C, 0.8/1.5/A, 2.2/1.5/D}
    \tkzDrawPolygon(A,B,C,D)
    \tkzMarkSegments(A,B  C,D)
    \tkzLabelPoints[left](A,B)
    \tkzLabelPoints[right](D,C)
\end{tikzpicture}


        \caption{}\label{fig:czjh1-4-38}
    \end{minipage}
\end{figure}

一腰垂直于底的梯形叫做\zhongdian{直角梯形}(图 \ref{fig:czjh1-4-37}),
两腰相等的梯形叫做\zhongdian{等腰梯形}(图 \ref{fig:czjh1-4-38})。

直角梯形和等腰梯形都是特殊的梯形。

下面研究等腰梯形的性质:

\begin{dingli}[等腰梯形性质定理]
    等腰梯形在同一底上的两个角相等。
\end{dingli}

已知: 如图 \ref{fig:czjh1-4-39}, 在梯形 $ABCD$ 中,$AD \pingxing BC$, $AB = DC$。

求证: $\angle B = \angle C$。

分析: 我们学过 “等腰三角形两底角相等”。 如果能将等腰梯形在同一底上的两个角,
转化成等腰三角形的两个底角,定理就容易证明了。

\zhengming 过点 $D$ 作 $DE \pingxing AB$, 交 $BC$ 于点 $E$, 得到等腰三角形 $DEC$。

$\because$ \quad $AD \pingxing BC$, $DE \pingxing AB$,

$\therefore$ \quad $AB = DE$ (夹在两平行线间的平行线段相等)。

$\because$ \quad $AB = DC$,

$\therefore$ \quad $DE = DC$。

$\therefore$ \quad $\angle 1 = \angle C$。

$\because$ \quad $\angle 1 = \angle B$,

$\therefore$ \quad $\angle B = \angle C$。

反过来,如果梯形在同一底上的两个角相等,两腰是否相等呢?

\begin{figure}[htbp]
    \centering
    \begin{minipage}[b]{4.5cm}
        \centering
        \begin{tikzpicture}
	\tkzDefPoints{0/0/B, 3/0/C, 0.8/1.5/A, 2.2/1.5/D}
    \tkzDrawPolygon(A,B,C,D)
    \tkzLabelPoints[left](A,B)
    \tkzLabelPoints[right](D,C)

    \tkzDefLine[parallel=through D](A,B)  \tkzGetPoint{e}
    \tkzInterLL(D,e)(B,C)  \tkzGetPoint{E}
    \tkzDrawSegment[dashed](D,E)
    \extkzLabelAngel[0.3](C,E,D){$1$}
    \tkzLabelPoints[below](E)
\end{tikzpicture}


        \caption{}\label{fig:czjh1-4-39}
    \end{minipage}
    \qquad
    \begin{minipage}[b]{4.5cm}
        \centering
        \begin{tikzpicture}
	\tkzDefPoints{0/0/B, 3/0/C, 0.8/1.5/A, 2.2/1.5/D}
    \tkzDrawPolygon(A,B,C,D)
    \tkzLabelPoints[left](A,B)
    \tkzLabelPoints[right](D,C)

    \tkzInterLL(B,A)(C,D)  \tkzGetPoint{E}
    \tkzDefLine[bisector](B,E,C)  \tkzGetPoint{f}
    \tkzInterLL(E,f)(B,C)  \tkzGetPoint{F}
    \tkzInterLL(E,f)(A,D)  \tkzGetPoint{G}
    \tkzDrawSegments[dashed](A,E  D,E)
    \tkzDrawSegments(E,F)
    \tkzMarkRightAngle(A,G,E)
    \tkzMarkRightAngle(B,F,E)
    \tkzLabelPoints[above](E)
    \tkzLabelPoints[below](F)
\end{tikzpicture}


        \caption{}\label{fig:czjh1-4-40}
    \end{minipage}
\end{figure}

如图 \ref{fig:czjh1-4-40}, 在梯形 $ABCD$ 中,$\angle B = \angle C$, 如果延长 $BA$、$CD$ 交于点 $E$,
那么由 $\angle B = \angle C$, $AD \pingxing BC$ 可以推出 $\triangle BCE$ 与
$\triangle ADE$ 都是等腰三角形, 得 $\angle E$ 的平分线 $EF$ 垂直平分 $AD$ 和 $BC$,
所以梯形 $ABCD$ 是以 $EF$ 为轴的对称图形, 因此 $AB = DC$, 由此得到:

\begin{dingli}[等腰梯形判定定理]
    在同一底上的两个角相等的梯形是等腰梯形。
\end{dingli}

由定理的推导过程可知, 等腰梯形是轴对称图形,经过两底中点的直线是它的对称轴。


\liti[0] 已知梯形的两底和两腰,作梯形。

已知:线段 $a$、$b$、$c$、$d$, 其中 $a > b$ (图\ref{fig:czjh1-4-41})。

求作:梯形 $ABCD$, 使 $AB \pingxing DC$, $AB = a$, $DC = b$, $DA = c$, $CB = d$。

\begin{figure}[htbp]
    \centering
    \begin{tikzpicture}
    \pgfmathsetmacro{\a}{4}
    \pgfmathsetmacro{\b}{2}
    \pgfmathsetmacro{\c}{3}
    \pgfmathsetmacro{\d}{2.5}

    \begin{scope}
        \tkzDefPoints{0/0/d1, \d/0/d2, 0/0.8/c1, \c/0.8/c2, 0/1.6/b1, \b/1.6/b2, 0/2.4/a1, \a/2.4/a2}
        \tkzDrawSegments[xianduan={below=0pt}](a1,a2  b1,b2  c1,c2  d1,d2)
        \tkzLabelSegment[above](a1,a2){$a$}
        \tkzLabelSegment[above](b1,b2){$b$}
        \tkzLabelSegment[above](c1,c2){$c$}
        \tkzLabelSegment[above](d1,d2){$d$}
    \end{scope}

    \begin{scope}[xshift=6cm]
        % 1
        \pgfmathsetmacro{\ae}{\a - \b}
        \tkzDefPoints{0/0/A, \ae/0/E}
        \tkzInterCC[R](A,\c)(E,\d)  \tkzGetFirstPoint{D}
        \tkzDrawSegments(A,D  A,E)
        \tkzDrawSegments[dashed](D,E)
        \tkzLabelSegment[left](A,D){$c$}
        \tkzLabelPoints[above](D)
        \tkzLabelPoints[left](A)
        \tkzLabelPoints[below](E)

        % 2
        \tkzDefPoints{\a/0/B}
        \tkzDrawSegments(E,B)
        \tkzLabelSegment[above right](A,B){$a$}
        \tkzLabelPoints[right](B)

        % 3
        \tkzInterCC[R](B,\d)(D,\b)  \tkzGetSecondPoint{C}
        \tkzCompasss(B,C  D,C)
        \tkzLabelPoints[above right](C)

        % 4
        \tkzDrawSegments(B,C  D,C)
        \tkzLabelSegment[right](B,C){$d$}
        \tkzLabelSegment[above](D,C){$b$}
    \end{scope}
\end{tikzpicture}


    \caption{}\label{fig:czjh1-4-41}
\end{figure}

分析:假定梯形 $ABCD$ 已经作出。 作 $DE \pingxing CB$ 交 $AB$ 于点 $E$,
可得平行四边形 $EBCD$, 于是 $DE = CB = d$, $EB = DC = b$, $AE = a - b$。
又 $DA = c$, 已知三边可以作出 $\triangle AED$, 再作平行四边形 $DEBC$,
就可以得到所求的梯形, 问题就解决了。

\zuofa 1. 作 $\triangle AED$, 使 $AE = a - b$, $DA = c$, $DE = d$。

2. 延长 $AE$ 到点 $B$, 使 $EB = b$。

3. 分别以点 $B$、$D$ 为圆心, $d$、$b$ 为半径作弧, 两弧相交于点 $C$。

4. 连结 $CB$、$CD$。

四边形 $ABCD$ 就是所求的梯形。

\zhengming 根据作法,$AB = (a - b) + b = a$, $CD = b$, $DA = c$, $CB = d$。

又因四边形 $EBCD$ 是平行四边形(对边相等), $AB \pingxing CD$。

所以,四边形 $ABCD$ 是所求的梯形。


讨论: 三条线段 $(a - b)$、$c$、$d$ 符合三角形三边关系定理时,作图题才有解。


\begin{lianxi}

\xiaoti{(口答) 一个四边形有一组对边平行但不相等,它是梯形吗?为什么?}

\xiaoti{求证:
    (1) \zhongdian{等腰梯形的对角线相等;}
    (2) 对角线相等的梯形是等腰梯形。
}

\xiaoti{求证:等腰梯形上底的中点与下底两端点距离相等。}

\end{lianxi}

