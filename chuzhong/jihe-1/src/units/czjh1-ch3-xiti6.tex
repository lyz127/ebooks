\xiti
\begin{xiaotis}

\xiaoti{已知 $\triangle ABC$。 画出它所有的外角,如果 $\angle ABC = 28^\circ$,$\angle BCA = 52^\circ$,
    $\angle CAB = 100^\circ$,求 $\triangle ABC$ 各外角的度数。
}

\begin{figure}[htbp]
    \centering
    \begin{tikzpicture}
    \tkzDefPoints{0/0/B, 6/0/C}
    \tkzDefPointBy[rotation=center B angle  28](C)  \tkzGetPoint{a1}
    \tkzDefPointBy[rotation=center C angle -52](B)  \tkzGetPoint{a2}
    \tkzInterLL(B,a1)(C,a2)  \tkzGetPoint{A}

    \tkzDrawPolygon(A,B,C)
    \tkzLabelPoints[above](A)
    \tkzLabelPoints[below](B,C)
\end{tikzpicture}


    \caption*{(第 1 题)}
\end{figure}


\xiaoti{在下面的每个三角形中,过顶点 $A$ 画出中线、角平分线和高。}

\begin{figure}[htbp]
    \centering
    \begin{tikzpicture}
    \begin{scope}
        \tkzDefPoints{0/0/B, 2/0/C, 2/3/A}

        \tkzDrawPolygon(A,B,C)
        \tkzMarkRightAngle(A,C,B)
        \tkzLabelPoints[above](A)
        \tkzLabelPoints[below](B,C)
    \end{scope}

    \begin{scope}[xshift=3cm]
        \tkzDefPoints{0/0/B, 1.5/0/C, 2.5/2.8/A}

        \tkzDrawPolygon(A,B,C)
        \tkzLabelPoints[above](A)
        \tkzLabelPoints[below](B,C)
    \end{scope}

    \begin{scope}[xshift=6cm]
        \tkzDefPoints{0/0/B, 3.2/0/C, 2.5/2.3/A}

        \tkzDrawPolygon(A,B,C)
        \tkzLabelPoints[above](A)
        \tkzLabelPoints[below](B,C)
    \end{scope}
\end{tikzpicture}


    \caption*{(第 2 题)}
\end{figure}



\xiaoti{已知 $\triangle ABC$ 的周长是 12 cm, $c + a = 2b$,
    $c - a = 2\;\limi$, $a$、$b$、$c$ 各等于多少?
}

\xiaoti{两根木棒的长分别是 7 cm 和 10 cm ,要选择第三根木棒,将它们钉成一个三角架,第三根木棒的长有什么限制?}

\xiaoti{}%
\begin{xiaoxiaotis}%
    \xxt[\xxtsep]{已知等腰三角形的一边等于 $5$, 一边等于 $6$, 求它的周长;}

    \xxt{已知等腰三角形的一边等于 $4$, 一边等于 $9$, 求它的周长。}

\end{xiaoxiaotis}


\xiaoti{已知等腰三角形的周长是 16 cm, 腰比底边长 2 cm, 求这个等腰三角形各边的长。}

\xiaoti{根据下列条件,求 $\triangle ABC$ 中 $\angle C$ 的大小:}
\begin{xiaoxiaotis}

    \xxt{$\angle A = 65^\circ 40'$, $\angle B = 36^\circ 25'$;}

    \xxt{$\angle A = 35^\circ$, $\angle B = \angle C$;}

    \xxt{$\angle B = \angle C = 2 \angle A$;}

    \xxt{$\angle A = 105^\circ$, $\angle B - \angle C = 15^\circ$。}

\end{xiaoxiaotis}

\xiaoti{一块模板如图,按规定 $AB$、$CD$ 的延长线应交成 $85^\circ$ 角,因交点不在板上,
    不便测量,工人师傅连结 $AC$, 测得 $\angle BAC = 32^\circ$, $\angle DCA = 65^\circ$,
    这时就可以知道,$AB$、$CD$ 的延长线相交所成的角是不是符合规定。为什么?
}


\begin{figure}[htbp]
    \centering
    \begin{minipage}[b]{7cm}
        \centering
        \begin{tikzpicture}
    \tkzDefPoints{0/0/E, 3.5/0/F,  0/1.5/A, 3.5/1.5/C}
    \tkzDefPointBy[rotation=center A angle  32](C)  \tkzGetPoint{b}
    \tkzDefPointBy[rotation=center C angle -65](A)  \tkzGetPoint{d}
    \tkzInterLL(A,b)(C,d)  \tkzGetPoint{O}
    \tkzDefPointOnLine[pos=0.7](A,O)  \tkzGetPoint{B}
    \tkzDefPointOnLine[pos=0.6](C,O)  \tkzGetPoint{D}
    \tkzDefLine[mediator,K=0.5](B,D)  \tkzGetFirstPoint{M}

    \tkzDrawSegments(B,A  A,E  E,F  F,C  C,D)
    \tkzDrawSegments[dashed](A,C)
    \tkzDrawArc(M,B)(D)
    \tkzLabelPoints[above](B,D)
    \tkzLabelPoints[left](A,E)
    \tkzLabelPoints[right](C,F)
\end{tikzpicture}


        \caption*{(第 8 题)}
    \end{minipage}
    \qquad
    \begin{minipage}[b]{7cm}
        \centering
        \begin{tikzpicture}
    \tkzDefPoints{0/0/B, 3.5/0/C, 2.5/2.5/A}
    \tkzDefPointOnLine[pos=0.4](A,B)  \tkzGetPoint{D}
    \tkzDefPointOnLine[pos=0.6](A,C)  \tkzGetPoint{E}
    \tkzInterLL(B,E)(C,D)  \tkzGetPoint{F}

    \tkzDrawPolygon(A,B,C)
    \tkzDrawSegments(B,E  C,D)
    \tkzLabelPoints[above](A,F)
    \tkzLabelPoints[below](B,C)
    \tkzLabelPoints[left](D)
    \tkzLabelPoints[right](E)
\end{tikzpicture}


        \caption*{(第 9 题)}
    \end{minipage}
\end{figure}


\xiaoti{已知:如图, $D$ 是 $AB$ 上一点, $E$ 是 $AC$ 上一点, $BE$ 和 $CD$ 相交于点 $F$。\\
    求证:(1) $\angle BDC = \angle A + \angle ACD$; \\
    (2) $\angle BFC = \angle ABF + \angle A + \angle ACD$。
}

\xiaoti{在括号内填写理由:\\
    已知: $P$ 是 $\triangle ABC$ 内一点。 \\
    求证: $\angle BPC > \angle BAC$。 \\
    \zhengming 连结 $AP$,并延长到点 $D$。 \\
    $\because$ \quad \begin{zmtblr}[t]{}
        $\angle BPD > \angle BAD$ (\hspace*{2cm}), \\
        $\angle DPC > \angle DAC$ (\hspace*{2cm}), \\
    \end{zmtblr} \\
    $\therefore$ \quad $\angle BPD + \angle DPC > \angle BAD + \angle DAC$ (\hspace*{2cm})。 \\
    即 \quad $\angle BPC > \angle BAC$。
}

\begin{figure}[htbp]
    \centering
    \begin{minipage}[b]{4.5cm}
        \centering
        \begin{tikzpicture}
    \tkzDefPoints{0/0/B, 3.5/0/C, 0.8/2.5/A, 1.5/1/P}
    \tkzDefPointOnLine[pos=1.3](A,P)  \tkzGetPoint{D}

    \tkzDrawPolygon(A,B,C)
    \tkzDrawSegments(B,P  C,P)
    \tkzDrawSegments[dashed](A,D)
    \tkzLabelPoints[above](A,P)
    \tkzLabelPoints[below](B,C,D)
\end{tikzpicture}


        \caption*{(第 10 题)}
    \end{minipage}
    \qquad
    \begin{minipage}[b]{4.5cm}
        \centering
        \begin{tikzpicture}
    \tkzDefPoints{0/0/B, 3.5/0/C}
    \tkzDefTriangle[two angles=40 and 70](B,C)  \tkzGetPoint{A}
    \tkzDefTriangle[two angles=-40 and -70, swap](A,C)  \tkzGetPoint{D}

    \tkzDrawPolygon(A,B,C)
    \tkzDrawSegments(A,D)
    \tkzMarkAngles[size=0.3](C,B,A  D,A,C)
    \tkzLabelPoints[above](A)
    \tkzLabelPoints[below](B,C,D)
\end{tikzpicture}

        \caption*{(第 11 题)}
    \end{minipage}
    \qquad
    \begin{minipage}[b]{4.5cm}
        \centering
        \begin{tikzpicture}
    \tkzDefPoints{0/0/B, 3/0/C}

    % \tkzDefPointBy[rotation=center B angle  66](C)  \tkzGetPoint{a1}
    % \tkzDefPointBy[rotation=center C angle -54](B)  \tkzGetPoint{a2}
    % \tkzInterLL(B,a1)(C,a2)  \tkzGetPoint{A}
    \tkzDefTriangle[two angles=66 and 54](B,C)  \tkzGetPoint{A}

    \tkzDefLine[bisector](B,A,C)  \tkzGetPoint{d}
    \tkzInterLL(A,d)(B,C)         \tkzGetPoint{D}

    \tkzDrawPolygon(A,B,C)
    \tkzDrawSegments(A,D)
    \tkzMarkAngles[size=0.5](B,A,D)
    \tkzMarkAngles[size=0.6](D,A,C)
    \tkzLabelPoints[above](A)
    \tkzLabelPoints[below](B,C,D)
\end{tikzpicture}

        \caption*{(第 13 题)}
    \end{minipage}
\end{figure}


\xiaoti{完成下面的证明: \\
    已知:如图, $\angle DAC = \angle B$。 \\
    求证: $\angle ADC = \angle BAC$。 \\
    \zhengming $\because$ \quad \begin{zmtblr}[t]{}
        $\angle ADC = \angle B + \angle BAD$ (\hspace*{2cm}), \\
        $\angle B = \angle DAC$ (\hspace*{2cm}), \\
    \end{zmtblr} \\
    $\therefore$ \quad $\angle ADC = \ewkh[1cm] + \angle BAD$ (\hspace*{2cm}), \\
    即 \quad $\angle ADC = \ewkh[1cm]$。
}

\begin{enhancedline}
\xiaoti{适合下列条件的 $\triangle ABC$ 是锐角三角形、直角三角形、还是钝角三角形?}
\begin{xiaoxiaotis}

    \xxt{$\angle A = \angle B = \angle C$;}

    \xxt{$\angle A + \angle B = \angle C$;}

    \xxt{$\angle A = \angle B = 30^\circ$;}

    \xxt{$\angle A = \exdfrac{1}{2} \angle B = \exdfrac{1}{3} \angle C$。}

\end{xiaoxiaotis}


\xiaoti{在 $\triangle ABC$ 中, 已知 $AD$ 是角平分线,$\angle B = 66^\circ$,
    $\angle C = 54^\circ$。 求 $\angle ADB$  和 $\angle ADC$ 的度数。
}

\xiaoti{在 $\triangle ABC$ 中, 已知 $\angle ABC = 66^\circ$, $\angle ACB = 54^\circ$,
    $BE$ 是 $AC$ 上的高, $CF$ 是 $AB$ 上的高, $H$ 是 $BE$ 和 $CF$ 的交点。
    求 $\angle ABE$、$\angle ACF$ 和 $\angle BHC$ 的度数。
}
\end{enhancedline}


\end{xiaotis}

