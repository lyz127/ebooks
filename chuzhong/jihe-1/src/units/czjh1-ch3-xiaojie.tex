\xiaojie

一、本章主要内容是有关三角形的一些知识, 重点是三角形全等的判定和两种特殊三角形——等腰三
角形、直角三角形的性质。


二、三角形的边、角有下面一些主要关系:

1. 两边的和大于第三边,两边的差小于第三边。

2. 内角的和等于 $180^\circ$。 一个外角等于和它不相邻的两个内角的和,大于其中任何一个。

三角形可以按角或按边分类如下:

\begin{figure}[H]
    \centering
    \begin{tikzpicture}
    \pgfmathsetmacro{\R}{2}
    \begin{scope}
        \coordinate (O) at(0,0);
        \draw (O) circle [radius=\R];
        \draw (O) -- (90:\R);
        \draw (O) -- (210:\R);
        \draw (O) -- (330:\R);

        \draw (-0.8,  0.8) node [align=center] {锐角\\[-0.5em]三角形};
        \draw ( 0.8,  0.8) node [align=center] {钝角\\[-0.5em]三角形};
        \draw (   0, -0.8) node [align=center] {直角\\[-0.5em]三角形};

        \draw (0, 2.5) node {按角分};
        \draw (0, -2.5) node {三角形集合};
    \end{scope}

    \begin{scope}[xshift=6cm]
        \coordinate (O) at(0,0);
        \draw (O) circle [radius=\R];
        \draw (90:\R) -- (270:\R);

        \pgfmathsetmacro{\r}{0.8}
        \coordinate (O') at(-\R/2,0);
        \draw (O') circle [radius=\r];

        \draw (-0.8, 1.3) node [align=right]  {等腰\\[-0.5em]三角形};
        \draw (\R/2, 0)   node [align=center] {不等边\\[-0.5em]三角形};
        \draw (O')        node [align=center] {等边\\[-0.5em]三角形};

        \draw (0, 2.5) node {按边分};
        \draw (0, -2.5) node {三角形集合};
    \end{scope}
\end{tikzpicture}


\end{figure}

三、判定两个三角形全等有下面一些公理和定理:

一般三角形:$SAS$, $ASA$, $AAS$, $SSS$。

直角三角形:除 $SAS$, $ASA$, $AAS$, $SSS$ 外,还有 $HL$。

三角形全等的判定公理和定理,是进一步研究平面几何问题的基础。
例如,利用三角形全等可以证明线段、角相等。


四、等腰三角形和直角三角形都是特殊三角形,除具有一般三角形的性质外,还有一些重要性质。

等腰三角形:

1. 底角相等。

2. 顶角的平分线、底边上的高、中线互相重合。

直角三角形:

1. 两个锐角互余。

2. 斜边上的中线等于斜边的一半。

3. 如果一个锐角等于 $30^\circ$, 那么它所对的直角边等于斜边的一半,而且逆命题也成立。


五、尺规作图与工具作图不同,只允许使用直尺(无刻度)和圆规。
一些尺规作图的作法,都是由基本作图组成的。
“边边边” 定理是基本作图的主要根据。


六、关于某直线对称,是对两个图形说的,它表示两个图形之间的对称关系;
轴对称图形是对一个图形说的,它表示某个图形的特性。
这两个概念有联系,也有区别。

线段是轴对称图形。线段的垂直平分线是它的对称轴。
线段的垂直平分线上的点和这条线段的两端距离相等。 而且逆命题也成立。

角是轴对称图形。角的平分线所在的直线是它的对称轴。
在角的平分线上的点到这个角的两边的距离相等。 而且逆命题也成立。

原命题是真命题, 它的逆命题不一定也是真命题。
如果原命题是真命题,逆命也是真命题,那么它们组成一对互逆定理。

