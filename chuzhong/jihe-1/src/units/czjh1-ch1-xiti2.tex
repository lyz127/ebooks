\xiti
\begin{xiaotis}

\xiaoti{分别用三个大写字母表示图中的 $\angle 1$、$\angle 2$、$\angle 3$、
    $\angle 4$、$\angle 5$、$\angle 6$、$\angle 7$、$\angle 8$。
}

\begin{figure}[htbp]
    \centering
    \begin{minipage}[b]{7cm}
        \centering
        \begin{tikzpicture}
	\tkzDefPoints{0/0/A, 5/0.3/B, 0/2/C, 4.5/1.5/D, 2/2.6/E, 3.0/-1/F}
	\tkzInterLL(A,B)(E,F)   \tkzGetPoint{H}
	\tkzInterLL(C,D)(E,F)   \tkzGetPoint{G}

	\tkzDrawSegments(A,B  C,D  E,F)

	\tkzMarkAngle[size=0.4](E,G,C)  \tkzLabelAngle[pos=0.6](E,G,C){$1$}
	\tkzMarkAngle[size=0.3](D,G,E)  \tkzLabelAngle[pos=0.5](D,G,E){$2$}
	\tkzMarkAngle[size=0.5](C,G,F)  \tkzLabelAngle[pos=0.8](C,G,F){$3$}
	\tkzMarkAngle[size=0.4](F,G,D)  \tkzLabelAngle[pos=0.6](F,G,D){$4$}

	\tkzMarkAngle[size=0.4](E,H,A)  \tkzLabelAngle[pos=0.6](E,H,A){$5$}
	\tkzMarkAngle[size=0.3](B,H,E)  \tkzLabelAngle[pos=0.5](B,H,E){$6$}
	\tkzMarkAngle[size=0.5](A,H,F)  \tkzLabelAngle[pos=0.8](A,H,F){$7$}
	\tkzMarkAngle[size=0.4](F,H,B)  \tkzLabelAngle[pos=0.6](F,H,B){$8$}

	\tkzLabelPoints[below](A, B, C, D)
	\tkzLabelPoints[right](E)
	\tkzLabelPoints[left](F)
	\tkzLabelPoints[below, xshift=-0.4em](G, H)
\end{tikzpicture}


        \caption*{(第 1 题)}
    \end{minipage}
    \qquad
    \begin{minipage}[b]{7cm}
        \centering
        \begin{tikzpicture}
	\pgfmathsetmacro{\r}{2}
	\tkzDefPoints{-\r/0/A, 0/0/O, \r/0/D}
	\tkzDefPoint(45:\r){C}
	\tkzDefPoint(120:\r){B}

	\tkzDrawSegments(O,A  O,B  O,C  O,D)
	\tkzLabelPoints[below](A, O, D)
	\tkzLabelPoints[right](B, C)
\end{tikzpicture}


        \caption*{(第 2 题)}
    \end{minipage}
\end{figure}

\xiaoti{如图,$AOD$ 是直线,图中小于 $180^\circ$ 的角有几个?是哪几个?}

\begin{enhancedline}

\xiaoti{$\exdfrac{1}{4}$ 平角等于多少度? $\exdfrac{1}{6}$ 周角呢?}

\xiaoti{在括号内填上适当的分数:}
\begin{xiaoxiaotis}

    \begin{tblr}{columns={18em, colsep=0pt}}
        \xxt{$15^\circ = \text{\ewkh 平角}$;} & \xxt{$60^\circ = \text{\ewkh 平角}$;} \\
        \xxt{$45^\circ = \text{\ewkh 周角}$;} & \xxt{$135^\circ = \text{\ewkh 平角}$。}
    \end{tblr}

\end{xiaoxiaotis}
\end{enhancedline}

\xiaoti{用度、分、秒表示:(1) $4.56^\circ$; (2) $64.23^\circ$。}

\xiaoti{用度表示:(1)$30^\circ 45'$; (2)$25^\circ 12' 18''$。}

\xiaoti{计算:}
\begin{xiaoxiaotis}

    \xxt{$23^\circ 35' 36'' + 66^\circ 24' 24''$;}

    \xxt{$180^\circ - 132^\circ 46' 50''$;}

    \xxt{$15^\circ 27' 38'' \times 3$;}

    \xxt{$49^\circ 28' 52'' \div 4$。}

\end{xiaoxiaotis}


\xiaoti{根据图形,在括号内填上适当的角:}
\begin{xiaoxiaotis}

    \xxt{$\angle AOC = \ewkh[3em] + \ewkh[3em]$;}

    \xxt{$\angle AOD - \angle BOD = \ewkh[3em]$;}

    \xxt{$\angle BOC = \ewkh[3em] - \angle COD$。}

\end{xiaoxiaotis}

\begin{figure}[htbp]
    \centering
    \begin{minipage}[b]{7cm}
        \centering
        \begin{tikzpicture}
	\pgfmathsetmacro{\r}{2}
	\tkzDefPoints{0/0/O, \r/0/A}
	\tkzDefPoint(30:\r){B}
	\tkzDefPoint(70:\r){C}
	\tkzDefPoint(130:\r){D}

	\tkzDrawSegments(O,A  O,B  O,C  O,D)
	\tkzLabelPoints[below](A, O)
	\tkzLabelPoints[right](B, C, D)
\end{tikzpicture}


        \caption*{(第 8 题)}
    \end{minipage}
    \qquad
    \begin{minipage}[b]{7cm}
        \centering
        \begin{tikzpicture}
	\pgfmathsetmacro{\r}{2}
	\tkzDefPoints{0/0/O, \r/0/A}
	\tkzDefPoint(45:\r){B}
	\tkzDefPoint(70:\r){C}
	\tkzDefPoint(110:\r){D}

	\tkzDrawSegments(O,A  O,B  O,C  O,D)
	\tkzLabelPoints[below](A, O)
	\tkzLabelPoints[right](B, C, D)
\end{tikzpicture}


        \caption*{(第 9 题)}
    \end{minipage}
\end{figure}

\xiaoti{根据图形填空:}
\begin{xiaoxiaotis}

    \xxt{\begin{tblr}[t]{colsep=0pt, rowsep=0pt}
        因为 \quad  $\angle AOB = \angle COD$, \\
        所以 \quad  $\angle AOB + \angle BOC = \angle COD + \ewkh[3em]$, \\
        即 \quad $\angle AOC = \ewkh[3em]$;
    \end{tblr}}

    \xxt{\begin{tblr}[t]{colsep=0pt, rowsep=0pt}
        因为 \quad  $\angle AOC = \angle BOD$, \\
        所以 \quad  $\angle AOC - \angle BOC = \angle BOD - \ewkh[3em]$, \\
        即 \quad $\angle AOB = \ewkh[3em]$。
    \end{tblr}}

\end{xiaoxiaotis}


\xiaoti{如图,已知 $AOB$ 是直线, $\angle AOC = 73^\circ$, $\angle BOD = 58^\circ$,求 $\angle COD$ 的大小。}

\begin{figure}[htbp]
    \centering
    \begin{minipage}[b]{5cm}
        \centering
        \begin{tikzpicture}
	\pgfmathsetmacro{\r}{2}
	\tkzDefPoints{0/0/O, -\r/0/A, \r/0/B}
	\tkzDefPoint(58:\r){D}
	\tkzDefPoint(107:\r){C}

	\tkzDrawSegments(O,A  O,B  O,C  O,D)
	\tkzLabelPoints[below](A, O, B)
	\tkzLabelPoints[left](C)
	\tkzLabelPoints[right](D)
\end{tikzpicture}


        \caption*{(第 10 题)}
    \end{minipage}
    \qquad
    \begin{minipage}[b]{5cm}
        \centering
        \begin{tikzpicture}
	\pgfmathsetmacro{\r}{2}
	\tkzDefPoints{0/0/O, \r/0/B}
	\tkzDefPoint(165:\r){A}
	\tkzDefPoint(90:\r){D}
	\tkzDefPoint(75:\r){C}

	\tkzDrawSegments(O,A  O,B  O,C  O,D)
	\tkzMarkRightAngle(B,O,D)
	\tkzMarkRightAngle[size=0.4](A,O,C)
	\tkzLabelPoints[below](A, O, B)
	\tkzLabelPoints[right](C)
	\tkzLabelPoints[left](D)
\end{tikzpicture}


        \caption*{(第 11 题)}
    \end{minipage}
    \qquad
    \begin{minipage}[b]{4cm}
        \centering
        \begin{tikzpicture}
	\pgfmathsetmacro{\r}{2}
	\tkzDefPoints{0/0/O}
	\tkzDefPoint(15:\r){A}
	\tkzDefPoint(105:\r){B}
	\tkzDefPoint(-10:\r){C}
	\tkzDefPoint(-100:\r){D}

	\tkzDrawSegments(O,A  O,B  O,C  O,D)
	\tkzMarkRightAngle(A,O,B)
	\tkzMarkRightAngle(C,O,D)
	\tkzLabelPoints[left](O)
	\tkzLabelPoints[below](C)
	\tkzLabelPoints[right](A, B, D)
\end{tikzpicture}


        \caption*{(第 20 题)}
    \end{minipage}
\end{figure}

\xiaoti{如图,已知 $\angle AOB = 165^\circ$, $\angle AOC = \angle BOD = 90^\circ$,求 $\angle COD$ 的大小。}

\xiaoti{用三角板画一个 $90^\circ$ 的角,再把它三等分。}


\xiaoti{用直尺和量角器画 $\angle AOB = 100^\circ$, 再把 $\angle AOB$ 分成 5 等分。}

\xiaoti{用直尺和量角器把一个周角分成 9 等分。}

\xiaoti{已知三个锐角 $\angle 1$、$\angle 2$、 $\angle 3$ $(\angle 2 > \angle 3)$。
    用直尺和量角器画一个角,使它等于:
}
\begin{xiaoxiaotis}

    \begin{tblr}{columns={18em, colsep=0pt}}
        \xxt{$\angle 1 + \angle 2$;} & \xxt{$\angle 2 - \angle 3$;} \\
        \xxt{$\angle 1 + \angle 2 - \angle 3$;} & \xxt{$2 \angle 2 - \angle 3$。}
    \end{tblr}

\end{xiaoxiaotis}


\xiaoti{用直尺和量角器画 $\angle AOB = 70^\circ$, 再画它的角平分线。}


\xiaoti{求下列各角的余角和补角的大小:}
\begin{xiaoxiaotis}

    \begin{tblr}{columns={18em, colsep=0pt}}
        \xxt{$76^\circ 45'$ 的角;} & \xxt{$14^\circ 20' 30''$ 的角;} \\
        \xxt{$n^\circ$ 的角 \; $(0 < n < 90)$。}
    \end{tblr}

\end{xiaoxiaotis}


\xiaoti{一个角等于它的余角的 3 倍,求这个角。}

\xiaoti{已知一个锐角,画它的余角与补角。}

\xiaoti{如图,已知 $\angle AOB = \angle COD = Rt \angle$,
    $\angle AOD$ 与 $\angle BOC$ 是否相等,为什么?
}

\end{xiaotis}

