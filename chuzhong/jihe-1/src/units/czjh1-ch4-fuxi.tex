\fuxiti
\begin{xiaotis}
\begin{enhancedline}

\xiaoti{求证: 四边形两条对角线的和大于周长的一半而小于周长。}

\xiaoti{一个四边形的内角能都是锐角吗?能都是直角吗?能都是钝角吗? 为什么?}

\xiaoti{四边形四个内角的比是 $1:2:3:4$, 求各角的度数。}

\xiaoti{已知: $\pxsbx ABCD$ 中, $\angle BAC > \angle CAD$。求证:$\angle BDC > \angle ADB$。}

\xiaoti{已知: $\pxsbx ABCD$ 中, $E$、$F$ 分别是 $AD$ 和 $BC$ 的中点, $AF$ 与 $BE$ 交于点 $G$,
    $CE$ 和 $DF$ 交于点 $H$。 求证: $EF$ 与 $HG$ 互相平分。
}

\xiaoti{已知: $\pxsbx ABCD$ 的对角线相交于点 $O$, $EF$ 经过点 $O$,与 $AB$ 交于点 $E$,
    与 $CD$ 交于点 $F$, $G$、$H$ 分别是 $AO$ 和 $CO$ 的中点。 求证: 四边形 $EHFG$ 是平行四边形。
}

\xiaoti{求证:
    (1) 等腰三角形底边上任一点与两腰的距离的和等于腰上的高;
    (2) 等腰三角形底边延长线上任一点与两腰的距离的差等于腰上的高。
}

\xiaoti{求证: 如果平行四边形四个内角的平分线能够围成一个四边形,那么这个四边形是矩形。}

\xiaoti{已知: 矩形的对角线长 $d$, 一边长 $s$ ($d > s$)。  求作: 矩形。}

\xiaoti{已知: 菱形周长为 $p$, 一条对角线长 $d$ ($d < \exdfrac{1}{2} p$)。求作: 菱形。}

\xiaoti{已知: 菱形的周长等于它的高的 $8$ 倍。 求它的各角。}

\xiaoti{从菱形两条对角线的交点分别向各边引垂线。 求证: 连结各垂足的四边形是矩形。}

\xiaoti{菱形周长为 $20\;\limi$, 两邻角的比为 $1:2$, 求较短的对角线长。}

\xiaoti{菱形周长为 $10\;\limi$, 一条对角线长为 $2.5\;\limi$, 求菱形各角的度数。}

\xiaoti{在已知锐角三角形 $ABC$ 的外面作正方形 $ABDE$ 和正方形 $ACFG$。
    求证:(1) $BG = CE$; (2) $BG \perp CE$。
}

\xiaoti{在正方形 $ABCD$ 的边 $BC$ 的延长线上取一点 $E$, 使 $CE = AC$,连接 $AE$,
    交 $CD$ 于 $F$。 求 $\angle AFC$ 的度数。
}

\xiaoti{已知: $E$ 是正方形 $ABCD$ 内一点, $EA = AB = BE$。
    求证: $\angle ECD = \angle EDC = 15^\circ$。
}


\xiaoti{已知: $\triangle ABC$ 中, $\angle ACB = Rt \angle$, 四边形 $ACDE$ 和 $CBFG$
    是在 $\triangle ABC$ 外的正方形, $\triangle ABC$ 的高 $CH$ 的反向延长线交 $DG$ 于点 $M$。
    求证:(1)$DG = AB$; (2)$CM = \exdfrac{1}{2} DG$。
}

\xiaoti{已知: $O$ 是 $\pxsbx ABCD$ 的对称中心,$EF$ 和 $GH$ 经过点 $O$,
    $EF$ 分别交 $AB$.$CD$ 于点 $E$、$F$,
    $GH$ 分别交 $AD$、$BC$ 于点 $G$、$H$。
    求证: 四边形 $EHFG$ 是平行四边形。
}

\xiaoti{已知: $\triangle BEC$ 和 $\triangle DFA$ 是 $\pxsbx ABCD$ 外的等边三角形。
    求证:$\triangle BEC$ 和 $\triangle DFA$ 是中心对称图形。
}

\xiaoti{}%
\begin{xiaoxiaotis}%
    \xxt[\xxtsep]{已知: 四边形 $ABCD$ 中, $AB = DC$, $AC = BD$, $AD \neq BC$。
        求证: 四边形 $ABCD$ 是等腰梯形;
    }

    \xxt{如果 (1) 的题中没有 $AD \neq BC$. 那么四边形一定是等腰梯形吗?为什么?}

\end{xiaoxiaotis}


\begin{figure}[htbp]
    \centering
    \begin{minipage}[b]{7cm}
        \centering
        \begin{tikzpicture}
    \tkzDefPoints{0/0/A, 1.5/0/B, 0.5/2/D, 2.0/2/C}
    \tkzDefTriangle[two angles=60 and 60](C,B)  \tkzGetPoint{E}
    \tkzDefTriangle[two angles=60 and 60](A,D)  \tkzGetPoint{F}

    \tkzDrawPolygon(A,B,C,D)
    \tkzDrawSegments(A,F  D,F  B,E  C,E)
    \tkzDrawSegments[dashed](B,D  E,F)
    \tkzLabelPoints[below](A,B)
    \tkzLabelPoints[above](D,C)
    \tkzLabelPoints[left](F)
    \tkzLabelPoints[right](E)
\end{tikzpicture}


        \caption*{(第 20 题)}
    \end{minipage}
    \qquad
    \begin{minipage}[b]{7cm}
        \centering
        \begin{tikzpicture}
    \tkzDefPoints{0/0/C, 5/0/D, 0/2/A, 5/2/B}
    \foreach \x in {1,...,4} {
        \tkzDefPointOnLine[pos=0.22*\x](A,B)  \tkzGetPoint{A_\x}
        \tkzDefPointOnLine[pos=0.22*\x](D,C)  \tkzGetPoint{B_\x}
    }

    \tkzDrawSegments(B,A  A,D  D,C)
    \foreach \x in {1,...,4} {
        \pgfmathsetmacro{\y}{int(5-\x)}
        \tkzDrawSegments[dashed](A_\x,B_\y)
        \tkzLabelPoints[above](A_\x)
        \tkzLabelPoints[below](B_\y)
    }
    \tkzLabelPoints[left](A,C)
    \tkzLabelPoints[right](B,D)
\end{tikzpicture}


        \caption*{(第 23 题)}
    \end{minipage}
\end{figure}


\xiaoti{画梯形 $ABCD$, 使底 $AD = 2\;\limi$, 底 $BC = 4\;\limi$, $\angle B = 45^\circ$, $\angle C = 60^\circ$。}

\xiaoti{如图, 在 $AB$ 的两旁作 $\angle ADC = \angle DAB$, 以 $A$ 和 $D$ 为起点,
    分别在 $AB$ 和 $DC$ 上截取 $4$ 条线段,所有线段的长相等,那么相应分点的连线把 $AD$ 五等分。为什么?
}

\xiaoti{已知: $\triangle ABC$ 中,$\angle A = 90^\circ$,
    $D$、$E$、$F$ 分别是 $BC$、$CA$、$AB$ 边的中点。
    求证: $AD = FE$。
}

\xiaoti{已知: $E$ 和 $F$ 分别是 $\pxsbx ABCD$ 的边 $AD$ 和 $BC$ 上的点,并且 $AE = BE$,
    $G$ 是 $AF$ 和 $BE$ 的交点, $H$ 是 $CE$ 和 $DF$ 的交点。
    求证:$GH \pingxing BC$, $GH = \exdfrac{1}{2} BC$。
}

\xiaoti{已知: $AD$ 是 $\triangle ABC$ 的中线, $E$ 是 $AD$ 的中点,
    $F$ 是 $BE$ 的延长线与 $AC$ 的交点。 求证: $AF = \exdfrac{1}{2} FC$。
}

\xiaoti{已知等腰梯形的中位线长 $6\;\limi$, 腰长 $5\;\limi$, 求它的周长。}

\xiaoti{从 $\pxsbx ABCD$ 的顶点 $A$、$B$、$C$、$D$ 向形外的任意直线 $MN$ 引垂线
    $AA'$、$BB'$、$CC'$、$DD'$, 垂足是 $A'$、$B'$、$C'$、$D'$。
    求证: $AA' + CC' = BB' + DD'$。
}

\xiaoti{求证: 梯形对角线中点的连线平行于底,并且等于两底差的一半。}

\xiaoti{在 $\triangle ABC$ 中, 如果 $AB = 30\;\limi$, $BC = 24\;\limi$, $CA = 27\;\limi$,
    $AE = EF = FB$, $EG \pingxing FD \pingxing BC$, $FM \pingxing EN \pingxing AC$。
    求阴影部分三个三角形周长的和。
}


\begin{figure}[htbp]
    \centering
    \begin{tikzpicture}
    \tkzDefPoints{0/0/A, 3/0/B}
    \tkzInterCC[R](A,2.7)(B,2.4)  \tkzGetFirstPoint{C}
    \tkzDefPointOnLine[pos=1/3](A,B)  \tkzGetPoint{E}
    \tkzDefPointOnLine[pos=2/3](A,B)  \tkzGetPoint{F}
    \tkzDefPointOnLine[pos=1/3](B,C)  \tkzGetPoint{M}
    \tkzDefPointOnLine[pos=2/3](B,C)  \tkzGetPoint{N}
    \tkzDefPointOnLine[pos=1/3](C,A)  \tkzGetPoint{D}
    \tkzDefPointOnLine[pos=2/3](C,A)  \tkzGetPoint{G}
    \tkzInterLL(E,N)(D,F)  \tkzGetPoint{O}

    \tkzDrawPolygon(A,B,C)
    % \tkzDrawPolygon(G,E,N,D,F,M)
    \tkzDrawPolygon[pattern={mylines[angle=140, distance={3pt}]}](G,E,O)
    \tkzDrawPolygon[pattern={mylines[angle=140, distance={3pt}]}](F,M,O)
    \tkzDrawPolygon[pattern={mylines[angle=140, distance={3pt}]}](N,D,O)
    \tkzLabelPoints[left](A,G,D)
    \tkzLabelPoints[right](B,M,N)
    \tkzLabelPoints[below](E,F)
    \tkzLabelPoints[above](C)
\end{tikzpicture}


    \caption*{(第 30 题)}
\end{figure}

\end{enhancedline}
\end{xiaotis}

