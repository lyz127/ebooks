\xiaojie

1. 本章的主要内容是点、直线、射线、线段及角的概念、性质和画法;线段、角的比较、度量、和与差。

2. 几何图形的概念,是从实际物体中抽象出来的,它们反映了物体在形状、大小和位置关系方面的一些本质属性。

3. 线段和射线都是直线的一部分,线段有两个端点,射线有一个端点,直线没有端点。两点决定一条直线。两点之间线段最短。

点、直线、射线、线段是组成各种图形的基本图形。这些图形以及它们的性质,是平面几何的基础。

4. 角是由具有公共端点的两条射线组成的。
两条边成一直线的角是平角,平角的 2 倍是周角,平角的一半是直角。
小于直角的角是锐角,大于直角而小于平角的角是钝角。
两个角的和等于 1 直角时,这两个角互为余角;
两个角的和等于 1 平角时,这两个角互为补角。

同角或等角的余角相等;同角或等角的补角相等。

5. 定义是说明名词含义的语句,使各名词互不相混。

图形的某些性质是人们经过长期实践证实是正确的,我们把它当作公理,作为推出其他图形性质的根据。

在本章中讲了两条公理:

经过两点有一条直线,并且只有一条直线;

在所有连结两点的线中,线段最短。

