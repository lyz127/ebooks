\subsection{全等三角形}\label{subsec:czjh1-3-4}

把一块样板按在纸板上,画下图形,照图形裁下来的纸板就和样板完全一样,把样板和裁得的纸板放在一起能够完全重合。
从同一张底片冲洗出来的两张照片上的图形,放在一起也能够完全重合。

能够完全重合的两个图形叫做\zhongdian{全等形},两个全等三角形重合时,
互相重合的顶点叫做\zhongdian{对应顶点},
互相重合的边叫做\zhongdian{对应边},
互相重合的角叫做\zhongdian{对应角}。

\begin{figure}[htbp]
    \centering
    \begin{tikzpicture}
    % 两个 scope 的区别,仅仅在于各点的名称不同。
    % 所以将绘制代码抽取出来(复用)
    \def\drawtriangle{
        \tkzDefPoints{0/0/B, 3.5/0/C, 2.8/2/A}
        \tkzDrawPolygon(A,B,C)
        \tkzMarkSegment[mark=|](B,C)
        \tkzMarkSegment[mark=||](A,C)
        \tkzMarkSegment[mark=|||](A,B)
        \tkzMarkAngle[arc=l, size=0.3](B,A,C)
        \tkzMarkAngle[arc=ll, size=0.3](C,B,A)
        \tkzMarkAngle[arc=lll, size=0.3](A,C,B)
    }

    \begin{scope}
        \drawtriangle
        \tkzLabelPoints[above](A)
        \tkzLabelPoints[below](B,C)
    \end{scope}

    \begin{scope}[xshift=5cm]
        \drawtriangle
        \tkzLabelPoint[above](A){$A'$}
        \tkzLabelPoint[below](B){$B'$}
        \tkzLabelPoint[below](C){$C'$}
    \end{scope}
\end{tikzpicture}


    \caption{}\label{fig:czjh1-3-14}
\end{figure}

例如,图 \ref{fig:czjh1-3-14} 中的两个三角形能够完全重合,就是全等三角形,
“全等” 用符号 “$\quandeng$” 来表示, 读作 “全等于” 。
图 \ref{fig:czjh1-3-14} 中的 $\triangle ABC$ 和 $\triangle A'B'C'$ 全等,
记作 “$\triangle ABC \quandeng \triangle A'B'C'$”。
其中 $A$ 和 $A'$、$B$ 和 $B'$、 $C$ 和 $C'$ 是对应顶点,
$BC$ 和 $B'C'$、 $CA$ 和 $C'A'$、 $AB$和 $A'B'$ 是对应边,
$\angle A$ 和 $\angle A'$、$\angle B$ 和 $\angle B'$、$\angle C$ 和 $\angle C'$ 是对应角。

我们知道,能够重合的两条线段是相等的线段,能够重合的两个角是相等的角,
所以\zhongdian{全等三角形的对应边相等,对应角相等。}
例如,图 \ref{fig:czjh1-3-14} 中, $\triangle ABC \quandeng \triangle A'B'C'$,
那么 $BC = B'C'$, $CA = C'A'$, $AB = A'B'$,
$\angle A = \angle A'$, $\angle B = \angle B'$, $\angle C = \angle C'$。

记两个全等三角形时,我们通常把表示对应顶点的字母写在对应的位置上。
例如,图 \ref{fig:czjh1-3-15} 中两个三角形全等, 点 $A$、$A'$, $B$、$B'$, $C$、$C'$ 是对应点,
记作 “$\triangle ABC \quandeng \triangle A'B'C'$”,
而不记作 “$\triangle ABC \quandeng \triangle A'C'B'$”
或 “$\triangle ABC \quandeng \triangle B'A'C'$” 等。

\begin{figure}[htbp]
    \centering
    \begin{tikzpicture}
    \tkzDefPoints{0/0/B, 3.5/0/C, 2.8/2/A}
    \tkzDefPoints{4.25/0/M, 4.25/1/N}
    \tkzDefPointBy[reflection = over M--N](A)  \tkzGetPoint{A'}
    \tkzDefPointBy[reflection = over M--N](B)  \tkzGetPoint{B'}
    \tkzDefPointBy[reflection = over M--N](C)  \tkzGetPoint{C'}

    % 绘制左侧的三角形
    \tkzDrawPolygon(A,B,C)
    \tkzMarkSegment[mark=|](B,C)
    \tkzMarkSegment[mark=||](A,C)
    \tkzMarkSegment[mark=|||](A,B)
    \tkzMarkAngle[arc=l, size=0.3](B,A,C)
    \tkzMarkAngle[arc=ll, size=0.3](C,B,A)
    \tkzMarkAngle[arc=lll, size=0.3](A,C,B)
    \tkzLabelPoints[above](A)
    \tkzLabelPoints[below](B,C)

    % 绘制右侧的三角形
    \tkzDrawPolygon(A',B',C')
    \tkzMarkSegment[mark=|](B',C')
    \tkzMarkSegment[mark=||](A',C')
    \tkzMarkSegment[mark=|||](A',B')
    \tkzMarkAngle[arc=l, size=0.3](C',A',B')
    \tkzMarkAngle[arc=ll, size=0.3](A',B',C')
    \tkzMarkAngle[arc=lll, size=0.3](B',C',A')
    \tkzLabelPoints[above](A')
    \tkzLabelPoints[below](B',C')
\end{tikzpicture}


    \caption{}\label{fig:czjh1-3-15}
\end{figure}


\begin{lianxi}

\xiaoti{(口答) 如图, $\triangle AOC \quandeng \triangle BOD$,
    $\angle A$ 和 $\angle B$, $\angle C$ 和 $\angle D$ 是对应角,
    说出对应边和另外一组对应角。
}

\begin{figure}[htbp]
    \centering
    \begin{minipage}[b]{4.5cm}
        \centering
        \begin{tikzpicture}
    \tkzDefPoints{0/0/A, 0.7/1.8/C, 1.7/1/O}
    \tkzDefPointOnLine[pos=2](A,O)  \tkzGetPoint{B}
    \tkzDefPointOnLine[pos=2](C,O)  \tkzGetPoint{D}

    \tkzDrawPolygon(A,O,C)
    \tkzDrawPolygon(B,O,D)
    \tkzMarkAngles[arc=l,  size=0.4](B,A,C)
    \tkzMarkAngles[arc=ll, size=0.4](A,C,D)
    \tkzMarkAngles[arc=l,  size=0.4](A,B,D)
    \tkzMarkAngles[arc=ll, size=0.4](B,D,C)
    \tkzLabelPoints[above](C,B)
    \tkzLabelPoints[below](A,O,D)
\end{tikzpicture}


        \caption*{(第 1 题)}
    \end{minipage}
    \qquad
    \begin{minipage}[b]{5cm}
        \centering
        \begin{tikzpicture}
    \tkzDefPoints{0/0/A, 3/0/B, -1/1.5/D, 2/1.5/C}

    \tkzDrawPolygon(A,B,C,D)
    \tkzDrawSegments(A,C)
    \tkzMarkSegments[mark=|](A,B  C,D)
    \tkzMarkSegments[mark=||](A,D  B,C)
    \tkzLabelPoints[above](C,D)
    \tkzLabelPoints[below](A,B)
\end{tikzpicture}


        \caption*{(第 2 题)}
    \end{minipage}
    \qquad
    \begin{minipage}[b]{4.5cm}
        \centering
        \begin{tikzpicture}
    \tkzDefPoints{0/0/A, 3.5/0/D, 1.75/1/O}
    \tkzDefPointOnLine[pos=1.4](A,O)  \tkzGetPoint{B}
    \tkzDefPointOnLine[pos=1.4](D,O)  \tkzGetPoint{C}

    \tkzDrawPolygon(A,O,C)
    \tkzDrawPolygon(B,O,D)
    \tkzLabelPoints[above](C,B)
    \tkzLabelPoints[below](A,O,D)
\end{tikzpicture}


        \caption*{(第 3 题)}
    \end{minipage}
\end{figure}

\xiaoti{(口答)如图,$\triangle ABC \quandeng \triangle CDA$,
    $AB$ 和 $CD$, $BC$ 和 $DA$ 是对应边,说出对应角和另外一组对应边。
    由对应边找对应角,由对应角找对应边有什么规律?
}

\xiaoti{(口答)如图,$\triangle OCA \quandeng \triangle OBD$,
    $C$ 和 $B$, $A$ 和 $D$ 是对应顶点,说出这两个三角形中相等的边和角。
}

\end{lianxi}
