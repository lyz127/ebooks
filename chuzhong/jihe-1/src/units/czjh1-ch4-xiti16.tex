\xiti
\begin{xiaotis}

\xiaoti{作一条线段,再把它 7 等分。}

\xiaoti{求证:直角梯形的两个直角顶点到对腰中点的距离相等。}

\xiaoti{已知:$M$、$N$ 分别是 $\pxsbx ABCD$ 中边 $AB$、$CD$ 的中点, $CM$ 交 $BD$ 于点 $E$,
    $AN$ 交 $BD$ 于点 $F$。 求证:$BE = EF = FD$。
}

\xiaoti{已知: $\triangle ABC$ 中,$D$、$E$、$F$ 分别是 $BC$、$CA$、$AB$ 边的中点。求证:}
\begin{xiaoxiaotis}

    \xxt{$\angle FDE = \angle A$;}

    \xxt{四边形 $AFDE$ 的周长等于 $AB + AC$。}

\end{xiaoxiaotis}


\xiaoti{求证: 三角形的一条中位线与第三边上的中线互相平分。}

\xiaoti{已知: 四边形 $ABCD$ 中, $E$ 是 $AB$ 的中点, $F$ 是 $CD$ 的中点,
    $G$ 是 $AC$ 的中点, $H$ 是 $BD$ 的中点,并且 $E$、$F$、$G$、$H$ 不在同一条直线上。
    求证: $EF$ 和 $GH$ 互相平分。
}

\xiaoti{求证: 顺次连结矩形四边中点所得的四边形是菱形。}

\xiaoti{求证: 顺次连结菱形四边中点所得的四边形是矩形。}

\xiaoti{如图,已知: $AA' \pingxing EE'$, $AB = BC = CD = DE$,
    $A'B' = B'C' = C'D' = D'E'$, $AA' = 28\;\haomi$, $EE' = 36\;\haomi$,
    求 $BB'$、$CC'$、$DD'$ 的长。
}

\begin{figure}[htbp]
    \centering
    \begin{minipage}[b]{4.9cm}
        \centering
        \begin{tikzpicture}
    \tkzDefPoints{0/0/E, 3.6/0/E', 0.6/2.5/A, 3.4/2.5/A'}
    \tkzDefPointOnLine[pos=0.25](A,E)  \tkzGetPoint{B}
    \foreach \n [count=\i] in {B,C,D} {
        \pgfmathsetmacro{\p}{0.25*\i}
        \tkzDefPointOnLine[pos=\p](A,E)  \tkzGetPoint{\n}
        \tkzDefPointOnLine[pos=\p](A',E')  \tkzGetPoint{\n'}
    }
    \tkzDrawPolygon(A,E,E',A')
    \tkzDrawSegments(B,B' C,C' D,D')
    \tkzLabelPoints[left](A,B,C,D,E)
    \tkzLabelPoints[right](A',B',C',D',E')
\end{tikzpicture}


        \caption*{(第 9 题)}
    \end{minipage}
    \qquad
    \begin{minipage}[b]{4.2cm}
        \centering
        \begin{tikzpicture}
    \tkzDefPoints{0/0/B, 3/0/C, 0.8/2/A, 2.6/2/D}
    \tkzDefMidPoint(A,B)  \tkzGetPoint{E}
    \tkzDefMidPoint(C,D)  \tkzGetPoint{F}
    \tkzInterLL(B,D)(E,F)  \tkzGetPoint{G}
    \tkzInterLL(A,C)(E,F)  \tkzGetPoint{H}
    \tkzDrawPolygon(A,B,C,D)
    \tkzDrawSegments(A,C  B,D  E,F)
    \tkzLabelPoints[left](A,B,E)
    \tkzLabelPoints[right](C,D,F)
    \tkzLabelPoints[below](G,H)
\end{tikzpicture}


        \caption*{(第 12 题)}
    \end{minipage}
    \qquad
    \begin{minipage}[b]{5.2cm}
        \centering
        \begin{tikzpicture}
    \tkzDefPoints{0/0/B_0, 4/0/C_0, 3/3/A_0}
    \tkzDrawPolygon(A_0,B_0,C_0)
    \foreach \n [count=\i] in {1,...,4} {
        \pgfmathsetmacro{\p}{int(\i-1)}
        \tkzDefMidPoint(A_\p,B_\p)  \tkzGetPoint{A_\i}
        \tkzDefMidPoint(B_\p,C_\p)  \tkzGetPoint{B_\i}
        \tkzDefMidPoint(C_\p,A_\p)  \tkzGetPoint{C_\i}
        \tkzDrawPolygon(A_\i,B_\i,C_\i)
    }
    \tkzLabelPoint[above](A_0){$A$}
    \tkzLabelPoint[left](B_0){$B$}
    \tkzLabelPoint[right](C_0){$C$}
\end{tikzpicture}


        \caption*{(第 13 题)}
    \end{minipage}
\end{figure}

\xiaoti{已知: 梯形 $ABCD$ 中, $AD \pingxing BC$, $AB = AD + BC$, $M$ 为 $CD$ 的中点。
    求证: $AM$、$BM$ 分别平分 $\angle DAB$ 和 $\angle CBA$。
}

\xiaoti{在等腰梯形中,已知一角是 $45^\circ$, 高是 $h$ 米, 中位线长 $m$ 米,求两底的长。}

\begin{enhancedline}

\xiaoti{已知:如图,在梯形 $ABCD$ 中。$AD \pingxing BC$, $E$、$F$ 分别是 $AB$、$DC$ 的中点。\\
    求证: $GH = \exdfrac{1}{2} (BC - AD)$。
}

\xiaoti{已知 $\triangle ABC$ 的三边长 $a$、$b$、$c$,三条中位线组成一个新三角形,
    新三角形的三条中位线又组成一个三角形,以此类推,求第四次组成的三角形的边长。
}
\end{enhancedline}

\end{xiaotis}

