\subsection{等腰三角形的判定}\label{subsec:czjh1-3-9}

我们已经知道,等腰三角形有两个角(底角)相等,现在来证明有两个角相等的三角形一定是等腰三角形。

\begin{dingli}[等腰三角形的判定定理]
    如果一个三角形有两个角相等,那么这两个角所对的边也相等。
\end{dingli} (简写成 “\zhongdian{等角对等边}”。)

\begin{wrapfigure}[8]{r}{5cm}
    \centering
    \begin{tikzpicture}
    \tkzDefPoints{0/0/B,  2.4/0/C,  1.2/3/A,  1.2/0/D}

    \tkzDrawPolygon(A,B,C)
    \tkzDrawSegment[dashed](A,D)
    \tkzMarkRightAngle(A,D,C)
    \extkzLabelAngel[0.5](B,A,D){$1$}
    \extkzLabelAngel[0.6](D,A,C){$2$}
    \tkzLabelPoints[above](A)
    \tkzLabelPoints[below](B,C,D)
\end{tikzpicture}


    \caption{}\label{fig:czjh1-3-35}
\end{wrapfigure}

已知: $\triangle ABC$ 中,$\angle B = \angle C$ (图 \ref{fig:czjh1-3-35})。

求证: $AB = AC$。

\zhengming 作 $\angle BAC$ 的平分线 $AD$。

在 $\triangle BAD$ 和 $\triangle CAD$ 中,

\hspace{2em} $\begin{cases}
    \angle 1 = \angle 2 \quad \text{(角平分线的定义),} \\
    \angle B = \angle C \quad \text{(已知),} \\
    AD = AD \quad \text{(公共边),} \\
\end{cases}$

$\therefore$ \quad $\triangle BAD \quandeng \triangle CAD$ ($AAS$)。

$\therefore$ \quad $AB = AC$ (全等三角形的对应边相等)。

\begin{tuilun}[推论 1]
    三个角都相等的三角形是等边三角形。
\end{tuilun}

\begin{tuilun}[推论 2]
    有一个角等于 $60^\circ$ 的等腰三角形是等边三角形。
\end{tuilun}


\begin{wrapfigure}[8]{r}{5cm}
    \centering
    \begin{tikzpicture}
    \tkzDefPoints{0/0/B,  2.4/0/C,  1.2/3/A}
    \tkzDefPointOnLine[pos=1.4](B,A)  \tkzGetPoint{E}
    \tkzDefLine[bisector](C,A,E)  \tkzGetPoint{D}

    \tkzDrawPolygon(A,B,C)
    \tkzDrawSegments(A,D  A,E)
    \extkzLabelAngel[0.3](D,A,E){$1$}
    \extkzLabelAngel[0.4](C,A,D){$2$}
    \tkzMarkSegments(A,B  A,C)
    \tkzLabelPoints[left](A,E)
    \tkzLabelPoints[below](B,C)
    \tkzLabelPoints[right](D)
\end{tikzpicture}


    \caption{}\label{fig:czjh1-3-36}
\end{wrapfigure}


\liti 求证:如果三角形一个外角的平分线平行于三角形的一边,那么这个三角形是等腰三角形。

已知: $\angle 1 = \angle 2$, $AD \pingxing BC$ (图 \ref{fig:czjh1-3-36})。

求证: $AB = AC$。

分析:要证明 $AB = AC$,可先证明 $\angle B = \angle C$。 因为已知 $\angle 1 = \angle 2$,
所以可以设法找出 $\angle B$、 $\angle C$ 与 $\angle 1$、 $\angle 2$ 的关系。

\zhengming $\because$ \quad $AD \pingxing BC$ (已知),

$\therefore$ \quad \begin{zmtblr}[t]{}
    $\angle 1 = \angle B$ (两直线平行,同位角相等), \\
    $\angle 2 = \angle C$ (两直线平行,内错角相等)。 \\
\end{zmtblr}

$\because$ \quad $\angle 1 = \angle 2$ (已知)

$\therefore$ \quad $\angle B = \angle C$ (等量代换)。

$\therefore$ \quad $AB = AC$ (等角对等边)。



% \begin{wrapfigure}[8]{r}{5cm}
%     \centering
%     \begin{tikzpicture}
    \tkzDefPoints{0/0/A,  0/2/B,  0/3.0/N}
    \tkzDefPointBy[rotation=center A angle 42](N)  \tkzGetPoint{c1}
    \tkzDefPointBy[rotation=center B angle 84](N)  \tkzGetPoint{c2}
    \tkzInterLL(A,c1)(B,c2)  \tkzGetPoint{C}

    \tkzDrawPolygon(A,B,C)
    \tkzDrawLine[add=0.3 and 0](A,N)
    \extkzLabelAngel[0.6](N,A,C){$42^\circ$}
    \extkzLabelAngel[0.4](N,B,C){$84^\circ$}
    \tkzLabelPoints[left](C)
    \tkzLabelPoints[right](A,B,N)
\end{tikzpicture}


%     \caption{}\label{fig:czjh1-3-37}
% \end{wrapfigure}

\liti 上午 8 时,一条船从 $A$ 处出发以每小时 15 海里的速度向正北航行, 10 时到达 $B$处。
从 $A$、$B$ 望灯塔 $C$, 测得 $\angle NAC = 42^\circ$, $\angle NBC = 84^\circ$。
求从 $B$ 处到灯塔 $C$ 的距离(图 \ref{fig:czjh1-3-37})。

\jie $\because$ \quad $\angle NBC = \angle A + \angle C$ (三角形的一个外角等于不相邻的两个内角的和),

$\therefore$ \quad $\angle C = 84^\circ - 42^\circ = 42^\circ$。

$\therefore$ \quad  $BA = BC$ (等角对等边)。

$\because$ \quad $AB = 15 \times (10 - 8) = 30$,

$\therefore$ \quad $BC = 30 \text{(海里)}$。

答 \quad $B$ 到灯塔的距离是30 海里。


\begin{figure}[htbp]
    \centering
    \begin{minipage}[b]{7cm}
        \centering
        \begin{tikzpicture}
    \tkzDefPoints{0/0/A,  0/2/B,  0/3.0/N}
    \tkzDefPointBy[rotation=center A angle 42](N)  \tkzGetPoint{c1}
    \tkzDefPointBy[rotation=center B angle 84](N)  \tkzGetPoint{c2}
    \tkzInterLL(A,c1)(B,c2)  \tkzGetPoint{C}

    \tkzDrawPolygon(A,B,C)
    \tkzDrawLine[add=0.3 and 0](A,N)
    \extkzLabelAngel[0.6](N,A,C){$42^\circ$}
    \extkzLabelAngel[0.4](N,B,C){$84^\circ$}
    \tkzLabelPoints[left](C)
    \tkzLabelPoints[right](A,B,N)
\end{tikzpicture}


        \caption{}\label{fig:czjh1-3-37}
    \end{minipage}
    \qquad
    \begin{minipage}[b]{7cm}
        \centering
        \begin{tikzpicture}
    \tkzDefPoints{0/0/B,  3.0/0/C,  2.5/2/A}
    \tkzFindAngle(C,B,A)  \tkzGetAngle{b}
    \tkzDefPointBy[rotation=center C angle -\b](B)  \tkzGetPoint{d}
    \tkzInterLL(C,d)(A,B)  \tkzGetPoint{D}

    \tkzDrawPolygon(A,B,C)
    \tkzDrawSegment[dashed](C,D)
    \tkzMarkAngles[size=0.4](C,B,A  D,C,B)
    \tkzLabelPoints[above](A)
    \tkzLabelPoints[below](B,C)
    \tkzLabelPoints[left](D)
\end{tikzpicture}


        \caption{}\label{fig:czjh1-3-38}
    \end{minipage}
\end{figure}

% \begin{wrapfigure}[8]{r}{5cm}
%     \centering
%     \begin{tikzpicture}
    \tkzDefPoints{0/0/B,  3.0/0/C,  2.5/2/A}
    \tkzFindAngle(C,B,A)  \tkzGetAngle{b}
    \tkzDefPointBy[rotation=center C angle -\b](B)  \tkzGetPoint{d}
    \tkzInterLL(C,d)(A,B)  \tkzGetPoint{D}

    \tkzDrawPolygon(A,B,C)
    \tkzDrawSegment[dashed](C,D)
    \tkzMarkAngles[size=0.4](C,B,A  D,C,B)
    \tkzLabelPoints[above](A)
    \tkzLabelPoints[below](B,C)
    \tkzLabelPoints[left](D)
\end{tikzpicture}


%     \caption{}\label{fig:czjh1-3-38}
% \end{wrapfigure}

\liti \begin{xingzhi}
    在一个三角形中,如果两个角不等,那么它们所对的边也不等,大角所对的边较大
\end{xingzhi} (简写成 “\zhongdian{大角对大边}”)。

已知: $\triangle ABC$ 中, $\angle ACB > \angle B$ (图 \ref{fig:czjh1-3-38})。

求证: $AB > AC$。

\zhengming 在较大的 $\angle ACB$ 内作 $\angle BCD = \angle B$, $CD$ 交 $AB$ 于 $D$。

$\therefore$ \quad $BD = DC$ (等角对等边)。

在 $\triangle ADC$ 中,

\qquad $AD + DC > AC$  (三角形两边的和大于第三边),

$\therefore$ \quad $AD + BD > AC$ (等量代换)。

即 \quad $AB > AC$。



\begin{lianxi}

\xiaoti{}%
\begin{xiaoxiaotis}%
    \xxt[\xxtsep]{如图甲,已知 $\angle A = 36^\circ$, $\angle DBC = 36^\circ$,
        $\angle C = 72^\circ$。计算 $\angle 1$ 和 $\angle 2$ 的度数,并说明图中有哪些等腰三角形。
    }

    \xxt{如图乙,$CD$ 是等腰直角三角形斜边上的高。找出图中的等腰直角三角形。}

\end{xiaoxiaotis}

\begin{figure}[htbp]
    \centering
    \begin{minipage}[b]{7cm}
        \centering
        \begin{tikzpicture}
    \tkzDefPoints{0/0/B,  3/0/C}
    \tkzDefTriangle[golden](B,C)  \tkzGetPoint{A}
    \tkzDefPointBy[rotation=center B angle 36](C)  \tkzGetPoint{d}
    \tkzInterLL(A,C)(B,d)  \tkzGetPoint{D}

    \tkzDrawPolygon(A,B,C)
    \tkzDrawSegment(B,D)
    \extkzLabelAngel[0.4](B,D,C){$1$}
    \extkzLabelAngel[0.5](D,B,A){$2$}
    \extkzLabelAngel[0.8](B,A,C){$36^\circ$}
    \extkzLabelAngel[0.8](C,B,D){$36^\circ$}
    \extkzLabelAngel[0.5](D,C,B){$72^\circ$}
    \tkzLabelPoints[above](A)
    \tkzLabelPoints[below](B,C)
    \tkzLabelPoints[right](D)
\end{tikzpicture}


        \caption*{甲}
    \end{minipage}
    \qquad
    \begin{minipage}[b]{7cm}
        \centering
        \begin{tikzpicture}
    \tkzDefPoints{0/0/A,  4/0/B}
    \tkzDefTriangle[isosceles right](A,B)  \tkzGetPoint{C}
    \tkzDefLine[altitude](A,C,B)  \tkzGetPoint{D}

    \tkzDrawPolygon(A,B,C)
    \tkzDrawSegment(C,D)
    \tkzMarkRightAngle(A,C,B)
    \tkzMarkRightAngle(B,D,C)
    \tkzLabelPoints[above](C)
    \tkzLabelPoints[below](A,B,D)
\end{tikzpicture}


        \caption*{乙}
    \end{minipage}
    \caption*{(第 1 题)}
\end{figure}


\xiaoti{(口答)在 $\triangle ABC$ 中, $\angle A = 70^\circ$, $\angle B = 50^\circ$。比较三边的大小。}

\xiaoti{(口答)直角三角形的斜边大于直角边,即斜线段大于垂线段。为什么?}

\xiaoti{(口答)在 $\triangle ABC$ 中,如果 $AB$ 最大,那么 $\angle A$、 $\angle B$ 一定是锐角。为什么?}

\end{lianxi}


