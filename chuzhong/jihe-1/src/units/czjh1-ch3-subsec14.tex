\subsection{逆命题、逆定理}\label{subsec:czjh1-3-14}

在等腰三角形一节里,我们学过这样两个命题:
(1) 如果一个三角形是等腰三角形,那么它的两个底角相等;
(2) 如果一个三角形有两个角相等,那么,它是一个等腰三角形。
其中第一个命题的题设 “等腰三角形” 正好是第二个命题的结论,
第一个命题的结论 “两个底角相等” 正好是第二个命题的题设。

在两个命题中,如果第一个命题的题设是第二个命题的结论,
而第一个命题的结论又是第二个命题的题设,
那么这两个命题叫做\zhongdian{互逆命题}。
如果把其中一个叫做\zhongdian{原命题},那么另一个叫做它的\zhongdian{逆命题}。

如果一个定理的逆命题经过证明是真命题,那么它也是一个定理,
这两个定理叫做\zhongdian{互逆定理},其中一个叫做另一个的\zhongdian{逆定理}。

例如,上面提到的等腰三角形的两个定理是互逆定哩,
直角三角形性质定理 2 的两个推论也是互逆定理。

虽然每个命题都有逆命题,但要注意,因为一个真命题的逆命题不一定也是真命题,
所以并不是所有的定理都有逆定例。例如,
“对顶角相等” 的逆命题是假命题,所以 “对顶角相等” 这个定理没有逆定理。


\begin{lianxi}

\xiaoti{(口答) 说出下列命题的题设和结论,再说出它们的逆命题:}
\begin{xiaoxiaotis}

    \xxt{两条直平行,内错角相等;}

    \xxt{直角三角形的两个锐角互为余角;}

    \xxt{直角三角形的一个角是 $30^\circ$, 它所对的边等于斜边的一半;}

    \xxt{等边三角形的每个角都等于 $60^\circ$。}

\end{xiaoxiaotis}


\xiaoti{举例说明,下列定理没有逆定理:}
\begin{xiaoxiaotis}

    \xxt{如果 $a = b$, 那么 $a^2 = b^2$;}

    \xxt{如果两个角都是直角,那么这两个角相等;}

    \xxt{如果三角形有一个角是钝角,那么它的另外两个角是锐角。}

\end{xiaoxiaotis}

\end{lianxi}

