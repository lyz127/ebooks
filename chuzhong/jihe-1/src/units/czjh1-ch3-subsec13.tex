\subsection{直角三角形全等的判定}\label{subsec:czjh1-3-13}

要判定两个直角三角形全等,除了可以应用一般三角形全等的判定公理和判定定理外,还有下面的判定定理。

\begin{dingli}[斜边、直角边定理]
    有斜边和一条直角边对应相等的两个直角三角形全等
\end{dingli}(可以简写成“\zhongdian{斜边、直角边}” 或 “$\bm{HL}$”)。


已知:在 $\triangle ABC$ 和 $\triangle A'B'C'$  中,
$\angle ACB = \angle A'C'B' = Rt \angle$,
$AB = A'B'$, $AC = A'C'$ (图 \ref{fig:czjh1-3-51})。

求证; $Rt \triangle ABC \quandeng Rt \triangle A'B'C'$。

\zhengming 把 $\triangle ABC$ 和 $\triangle A'B'C'$ 拼在一起,
使相等的直角边 $AC$ 和 $A'C'$ 重合,
并且使点 $B$ 和 $B'$  在 $A'C'$  的两旁。

$\because$ \quad \begin{zmtblr}[t]{}
    $\angle A'C'B' = Rt \angle$, \\
    $\angle A'C'B = \angle ACB = Rt \angle$,
\end{zmtblr}

$\therefore$ \quad $\angle B'C'B = 2 Rt \angle$。

$\therefore$ \quad $B'$、$C'$、$B$ 的同一条直线上。

在 $\triangle A'B'B$ 中,

$\because$ \quad $A'B' = AB = A'B$,

$\therefore$ \quad $\angle B' = \angle B$ (等边对等角)。

在 $\triangle ABC$ 和 $\triangle A'B'C'$ 中,

$\angle ACB = \angle A'C'B'$, $\angle B = \angle B'$, $AB = A'B'$,

$\therefore$ \quad $\triangle ABC \quandeng \triangle A'B'C'$ ($AAS$)。

\begin{figure}[htbp]
    \centering
    \begin{minipage}[b]{8cm}
        \centering
        \begin{tikzpicture}
    \begin{scope}
        \tkzDefPoints{0/0/B, 2/0/C, 2/3/A}
        \tkzDrawPolygon(A,B,C)
        \tkzMarkRightAngle(A,C,B)
        \tkzMarkSegment(A,B)
        \tkzMarkSegment[mark=||](A,C)
        \tkzLabelPoints[above](A)
        \tkzLabelPoints[below](B,C)
    \end{scope}


    \begin{scope}[xshift=3.0cm]
        \tkzDefPoints{0/0/B', 2/0/C, 2/3/A, 4/0/B}
        \tkzDrawPolygon(A,B,C)
        \tkzDrawPolygon(A,B',C)
        \tkzMarkRightAngle(A,C,B)
        \tkzMarkRightAngle(A,C,B')
        \tkzMarkSegments(A,B  A,B')
        \tkzMarkSegment[mark=||](A,C)
        \tkzLabelPoint[above,xshift=1em](A){$A'(A)$}
        \tkzLabelPoints[below](B,B')
        \tkzLabelPoint[below,xshift=1em](C){$C'(C)$}
    \end{scope}
\end{tikzpicture}


        \caption{}\label{fig:czjh1-3-51}
    \end{minipage}
    \qquad
    \begin{minipage}[b]{6cm}
        \centering
        \begin{tikzpicture}
    \tkzDefPoints{0/0/B, 2.6/0/C, 1.3/3/A}
    \tkzDefMidPoint(B,C)  \tkzGetPoint{D}
    \tkzDefLine[altitude](A,D,B)  \tkzGetPoint{F}
    \tkzDefLine[altitude](A,D,C)  \tkzGetPoint{E}
    \tkzDrawPolygon(A,B,C)
    \tkzDrawSegments(D,E  D,F)
    \tkzMarkRightAngle[size=0.2](D,F,B)
    \tkzMarkRightAngle[size=0.2](D,E,C)
    \tkzLabelPoints[above](A)
    \tkzLabelPoints[left](F)
    \tkzLabelPoints[right](E)
    \tkzLabelPoints[below](B,D,C)
\end{tikzpicture}


        \caption{}\label{fig:czjh1-3-52}
    \end{minipage}
\end{figure}


这个定理的条件,实际上是已知两边和其中一边的对角对应相等,前面我们已经讲过,
具备这样的条件的两个三角形不一定全等。但是,如果这个角是个直角,那么这两个三角形就会全等了。

\liti 已知: $\triangle ABC$ 中, $D$ 是 $BC$ 的中点, $DF \perp AB$,
$DE \perp AC$, 垂足分别是 $F$、$E$, $DF = DE$ (图 \ref{fig:czjh1-3-52})。

求证: $AB = AC$。

\zhengming 在 $Rt \triangle DBF$ 和 $Rt \triangle DCE$ 中,

\qquad $DB = DC$, $DF = DE$,

$\therefore$ \quad $Rt \triangle DBF \quandeng Rt \triangle DCE$ ($HL$)。

$\therefore$ \quad $\angle B = \angle C$。

$\therefore$ \quad $AB = AC$ (等角对等边)。



\liti 已知斜边和一条直角边,求作直角三角形。

已知:线段 $c$ 和 $a$ (图 \ref{fig:czjh1-3-53}).

求作:$Rt \triangle ABC$,使它的斜边 $AB = c$,一条直角边 $BC = a$。

\zuofa 1. 作线段 $BC = a$。

2. 过点 $C$ 作直线  $CD \perp BC$。

3. 以点 $B$ 为圆心,$c$ 为半径作弧,交 $CD$ 于点 $A$。

4. 连接 $AB$。

$\triangle ABC$ 就是所求的直角三角形。

\zhengming $\because$ \quad $CD \perp BC$,

$\therefore$ \quad $\angle C = Rt \angle$。

又 $AB = c$, $BC = a$,

所以 $\triangle ABC$ 为所求的三角形。

\textbf{讨论:} 因为 $Rt \triangle$ 中,斜边一定要大于直角边,所以  $c \leqslant a$ 时,此题无解。

\begin{figure}[htbp]
    \centering
    \begin{minipage}[b]{7cm}
        \centering
        \begin{tikzpicture}
    \pgfmathsetmacro{\a}{2.5}
    \pgfmathsetmacro{\c}{3}

    \tkzDefPoints{0/0/c1, \c/0/c2, 0/0.8/a1, \a/0.8/a2}
    \tkzDrawSegments[xianduan={below=0pt}](c1,c2)
    \tkzLabelSegment[above](c1,c2){$c$}
    \tkzDrawSegments[xianduan={below=0pt}](a1,a2)
    \tkzLabelSegment[above](a1,a2){$a$}

    \begin{scope}[yshift=-3cm]
        % 1
        \tkzDefPoints{0/0/B, \a/0/C}
        \tkzDrawSegment(B,C)
        \tkzLabelPoints[left](B)
        \tkzLabelPoints[right](C)
        \tkzLabelSegment[below](B,C){$a$}

        % 2
        \tkzDefLine[perpendicular=through C,normed](B,C)  \tkzGetPoint{d}
        \tkzDefPointOnLine[pos=2.5](C,d)  \tkzGetPoint{D}
        \tkzDrawSegment(C,D)
        \tkzLabelPoints[right](D)

        % 3
        \tkzInterLC[R,near](D,C)(B,\c)  \tkzGetFirstPoint{A}
        \tkzCompass(B,A)
        \tkzLabelPoints[right](A)

        % 4
        \tkzDrawSegments(A,B)
        \tkzLabelSegment[above](B,A){$c$}
    \end{scope}
\end{tikzpicture}


        \caption{}\label{fig:czjh1-3-53}
    \end{minipage}
    \qquad
    \begin{minipage}[b]{7cm}
        \centering
        \begin{tikzpicture}
    \pgfmathsetmacro{\a}{2.0}
    \pgfmathsetmacro{\c}{1.2}
    \tkzDefPoints{0/0/A, \c/0/E, \c/\a/C}
    \tkzDefShiftPoint[E](0.6,0){F}
    \tkzDefShiftPoint[F](\c,0){B}
    \tkzDefShiftPoint[F](0,-\a){D}

    \tkzDrawSegments(E,C  C,A  A,B  B,D  D,F)
    \tkzMarkRightAngle(A,E,C)
    \tkzMarkRightAngle(B,F,D)
    \tkzLabelPoints[above](C,F)
    \tkzLabelPoints[below](E,D)
    \tkzLabelPoints[left](A)
    \tkzLabelPoints[right](B)
\end{tikzpicture}


        \caption*{(第 2 题)}
    \end{minipage}
\end{figure}


\begin{lianxi}

\xiaoti{}%
\begin{xiaoxiaotis}%
    \xxt[\xxtsep]{两条直角边对应相等的两个直角三角形是否全等?为什么?}

    \xxt{两个锐角对应相等的两个直角三角形是否全等?为什么?}

\end{xiaoxiaotis}


\xiaoti{已知:如图, $CE \perp AB$, $DF \perp AB$, $AC \pingxing AB$,且 $AC = DB$。 \\
    求证: $CE = DF$。
}

\xiaoti{求证:有两条高相等的三角形是等腰三角形。}

\end{lianxi}


