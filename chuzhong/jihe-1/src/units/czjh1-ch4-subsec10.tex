\subsection{平行线等分线段}\label{subsec:czjh1-4-10}

为了进一步研究梯形的性质,我们来证明下面的定理:

\begin{dingli}[平行线等分线段定理]
    如果一组平行线在一条直线上截得的线段相等,那么在其他直线上截得的线段也相等。
\end{dingli}

我们仅对三条平行线的情形进行证明。

\begin{wrapfigure}[8]{r}{5cm}
    \centering
    \begin{tikzpicture}
    \tkzDefPoints{-0.5/0.8/e, 3/0.8/f,  -0.5/1.9/c, 2.8/1.9/d, -0.5/3.0/a, 2.5/3.0/b}
	\tkzDefPoints{0/0/H, 0.5/3.5/G, 3/0/L, 1.2/3.5/K}
    \tkzInterLL(G,H)(a,b)  \tkzGetPoint{A}
    \tkzInterLL(G,H)(c,d)  \tkzGetPoint{C}
    \tkzInterLL(G,H)(e,f)  \tkzGetPoint{E}
    \tkzInterLL(K,L)(a,b)  \tkzGetPoint{B}
    \tkzInterLL(K,L)(c,d)  \tkzGetPoint{D}
    \tkzInterLL(K,L)(e,f)  \tkzGetPoint{F}
    \tkzDefLine[parallel=through B](G,H)  \tkzGetPoint{m}
    \tkzDefLine[parallel=through D](G,H)  \tkzGetPoint{n}
    \tkzInterLL(B,m)(C,D)  \tkzGetPoint{M}
    \tkzInterLL(D,n)(E,F)  \tkzGetPoint{N}

    \tkzDrawSegments(a,b c,d e,f G,H K,L)
    \tkzDrawSegments[dashed](B,M  D,N)
    \tkzMarkAngles[size=0.3](M,B,L  N,D,L)
    \tkzMarkAngles[size=0.2](B,D,C  D,F,E)
    \tkzLabelAngle[pos=0.5](M,B,L){\small $1$}
    \tkzLabelAngle[pos=0.5](N,D,L){\small $2$}
    \tkzLabelAngle[pos=0.35](B,D,C){\small $3$}
    \tkzLabelAngle[pos=0.35](D,F,E){\small $4$}

    \tkzLabelPoints[below left](A,C,E)
    \tkzLabelPoints[below right,xshift=.3em](B,D,F)
    \tkzLabelPoints[left](G)
    \tkzLabelPoints[right](K)
    \tkzLabelPoints[below](H,L,M,N)
\end{tikzpicture}


    \caption{}\label{fig:czjh1-4-42}
\end{wrapfigure}


已知: 如图 \ref{fig:czjh1-4-42}, $AB \pingxing CD \pingxing EF$, $AC = CE$。

求证: $BD = DF$。

\zhengming  过点 $B$、$D$ 分别作 $GH$ 的平行线 $BM$、$DN$,
分别交 $CD$、$EF$ 于点 $M$、$N$,得 $\pxsbx ACMB$、$\pxsbx CEND$。

$\therefore$ \quad $BM = AC$, $DN = CE$, $BM \pingxing DN$。

$\because$ \quad $AC = CE$,

$\therefore$ \quad $BM = DN$。

又 $\because$ \quad $BM \pingxing DN$, $MD \pingxing NF$,

$\therefore$ \quad $\angle 1 = \angle 2$, $\angle 3 = \angle 4$。

$\therefore$ \quad $\triangle BMD \quandeng \triangle DNF$。

$\therefore$ \quad $BD = DF$。

从这个定理, 可以推出下面推论:

\begin{tuilun}[推论1]
    经过梯形一个腰的中点与底平行的直线,必平分另一腰。
\end{tuilun}

如图 \ref{fig:czjh1-4-43}, 在梯形 $ABCD$ 中, 如果 $E$ 是腰 $AB$ 的中点,
$EF \pingxing AD$, 交 $DC$ 于点 $F$, 那么 $DF = FC$。

\begin{figure}[htbp]
    \centering
    \begin{minipage}[b]{7cm}
        \centering
        \begin{tikzpicture}
    \tkzDefPoints{0/0/B, 4/0/C, 1.5/2/A, 3.5/2/D}
    \tkzDefMidPoint(A,B)  \tkzGetPoint{E}
    \tkzDefLine[parallel=through E](B,C)  \tkzGetPoint{f}
    \tkzInterLL(E,f)(C,D)  \tkzGetPoint{F}
    \tkzDrawPolygon(A,B,C,D)
    \tkzDrawSegments(E,F)
    \tkzLabelPoints[left](A,B,E)
    \tkzLabelPoints[right](C,D,F)
\end{tikzpicture}


        \caption{}\label{fig:czjh1-4-43}
    \end{minipage}
    \qquad
    \begin{minipage}[b]{7cm}
        \centering
        \begin{tikzpicture}
    \tkzDefPoints{0/0/B, 3/0/C, 1.8/2/A, -0.5/2/M, 3/2/N}
    \tkzDefMidPoint(A,B)  \tkzGetPoint{E}
    \tkzDefLine[parallel=through E](B,C)  \tkzGetPoint{f}
    \tkzInterLL(E,f)(A,C)  \tkzGetPoint{F}
    \tkzDrawPolygon(A,B,C)
    \tkzDrawSegments(E,F)
    \tkzDrawSegments[dashed](M,N)
    \tkzLabelPoints[above](A)
    \tkzLabelPoints[left](B,E,M)
    \tkzLabelPoints[right](C,F,N)
\end{tikzpicture}


        \caption{}\label{fig:czjh1-4-44}
    \end{minipage}
\end{figure}

如图 \ref{fig:czjh1-4-44}, $E$ 是 $\triangle ABC$ 的 $AB$ 边的中点, $EF \pingxing BC$,交 $AC$ 于点 $F$,
过顶点 $A$ 引直线 $MN \pingxing BC$。由上面的定理可知:$AF = FC$,由此得到下面的推论:

\begin{tuilun}[推论2]
    经过三角形一边的中点与另一边平行的直线必平分第三边。
\end{tuilun}

应用平行线等分线段定理,我们可以任意等分一条线段。

\liti[0] 已知:线段 $AB$ (图 \ref{fig:czjh1-4-45})。

求作: 线段 $AB$ 的五等分点。

\zuofa 1. 作射线 $AC$。

2. 在射线 $AC$ 上以任意长顺次截取 $AD = DE = EF = FG = GH$。

3. 连结 $HB$。

4. 过点 $G$、$F$、$E$、$D$ 分别作 $HB$ 的平行线 $GL$、$FK$、$EJ$、$DI$,
分别交 $AB$ 于点 $L$、$K$、$J$、$I$。

$I$、$J$、$K$、$L$ 就是所求的五等分点。

\zhengming  过点 $A$ 作 $MN \pingxing HB$。

$\because$ \quad \begin{zmtblr}[t]{}
    $MN \pingxing DI \pingxing EJ \pingxing FK \pingxing GL \pingxing HB$, \\
    $AD = DE = EF = FG = GH$,
\end{zmtblr}

$\therefore$ \quad $AI = IJ = JK = KL = LB$ (平行线等分线段定理)。

\begin{figure}[htbp]
    \centering
    \begin{minipage}[b]{7cm}
        \centering
        \begin{tikzpicture}
    \tkzDefPoints{0/0/A, 5/0/B}
    \tkzDrawSegment(A,B)
    \tkzLabelPoints[left](A)
    \tkzLabelPoints[below](B)

    % 1
    \tkzDefPoints{5.2/3/C}
    \tkzDrawSegment[dashed](A,C)
    \tkzLabelPoints[above](C)

    % 2
    \foreach \n [count=\i] in {D,E,F,G,H} {
        \pgfmathsetmacro{\pos}{0.15*\i}
        \tkzDefPointOnLine[pos=\pos](A,C)  \tkzGetPoint{\n}
        \tkzLabelPoints[above](\n)
    }

    % 3
    \tkzDrawSegment[dashed](B,H)

    % 4
    \foreach \n/\m in {G/L, F/K, E/J, D/I} {
        \tkzDefLine[parallel=through \n](B,H)  \tkzGetPoint{tmp}
        \tkzInterLL(\n,tmp)(A,B)  \tkzGetPoint{\m}
        \tkzDrawSegment[dashed](\n,\m)
        \tkzLabelPoints[below](\m)
    }

    %
    \tkzDefLine[parallel=through A,normed](B,H)  \tkzGetPoint{M}
    \tkzDefPointOnLine[pos=2](M,A)  \tkzGetPoint{N}
    \tkzDrawSegment[dashed](M,N)
    \tkzLabelPoints[left](M,N)
\end{tikzpicture}


        \caption{}\label{fig:czjh1-4-45}
    \end{minipage}
    \qquad
    \begin{minipage}[b]{7cm}
        \centering
        \begin{tikzpicture}
    \tkzDefPoints{0/2/l11, 4/2/l12, 0/0/l21, 4/0/l22}
    \tkzDrawSegments(l11,l12  l21,l22)
    \tkzLabelLine[pos=1,right](l11,l12){$l_1$}
    \tkzLabelLine[pos=1,right](l21,l22){$l_2$}

    \tkzDefPointOnLine[pos=0.1](l11,l12)  \tkzGetPoint{A}
    \tkzDefPointOnLine[pos=0.8](l11,l12)  \tkzGetPoint{D}
    \tkzDefPointOnLine[pos=0.9](l21,l22)  \tkzGetPoint{B}
    \tkzDefMidPoint(A,B)  \tkzGetPoint{C}
    \tkzInterLL(D,C)(l21,l22)  \tkzGetPoint{E}
    \tkzDrawSegments(A,B  E,D)
    \tkzDrawPoint(C)
    \tkzLabelPoints[above](A,D)
    \tkzLabelPoints[below](B,E)
    \tkzLabelPoints[right,xshift=.5em](C)
\end{tikzpicture}


        \caption*{(第 3 题)}
    \end{minipage}
\end{figure}

\begin{lianxi}

\xiaoti{练习本上的横格是平行且等距的, 用横将 $10\;\limi$ 长的线绳五等分、七等分、九等分,
    分别用不同色笔在绳上画上等分点。
}

\xiaoti{画出一条线段, 用直尺、圆规将它分成三等分。}

\xiaoti{如图,已知:$l_1 \pingxing l_2$, $AC = CB$。
    求证:$DC = CE$。
}

\end{lianxi}


