\subsection{平行线的性质}\label{subsec:czjh1-2-7}

上一节我们学习了平行线的判定,它是由同位角相等,或内错角相等,或同旁内角互补来判定两条直线是平行线。

这一节要学习的平行线的性质和平行线的判定正好相反,它是已知两条直线平行,再以此为基础,推出图形具有某种性质。

平行线有下面的性质:

\begin{xingzhi}
    两条平行线被第三条直线所截,同位角相等。
\end{xingzhi}

这句话简单说成\begin{xingzhi}%
    两直线平行,同位角相等。
\end{xingzhi}

如图 \ref{fig:czjh1-2-19} , 如果 $AB \pingxing CD$, 那么, $\angle 1 = \angle 2$, 这是因为,
假如 $\angle 1 \neq \angle 2$,我们就可以经过点 $O$ 画一条直线 $A'B'$, 使 $\angle EOB' = \angle 2$。
根据同位角相等,两直线平行,可得 $A'B' \pingxing CD$。这样,经过点 $O$ 就有两条直线 $AB$、$A'B'$ 都与 $CD$ 平行。
这与平行公理不相符合。

\begin{figure}[htbp]
    \centering
    \begin{minipage}[b]{7cm}
        \centering
        \begin{tikzpicture}
    \tkzDefPoints{0/0/C, 4/0/D, 0/1.5/A, 4/1.5/B, 3/2.5/E, 1/-1/F}
    \tkzInterLL(A,B)(E,F)  \tkzGetPoint{O}
    \tkzInterLL(C,D)(E,F)  \tkzGetPoint{P}
    \tkzDefPoints{0.3/2/A'}
    \tkzDefPointOnLine[pos=1.8](A',O)  \tkzGetPoint{B'}

    \tkzDrawSegments(A,B  C,D  E,F)
    \tkzDrawSegments[dashed](A',B')
    \tkzMarkAngles[size=0.3](B,O,E  D,P,E)
    \tkzLabelAngle[pos=0.5](B,O,E){$1$}
    \tkzLabelAngle[pos=0.5](D,P,E){$2$}
    \tkzLabelPoints[left](A)
    \tkzLabelPoints[above left](A')
    \tkzLabelPoints[below](C,D,B')
    \tkzLabelPoints[right](B,E,F)
    \tkzLabelPoints[below left, xshift=-0.3em](O)
\end{tikzpicture}


        \caption{}\label{fig:czjh1-2-19}
    \end{minipage}
    \qquad
    \begin{minipage}[b]{7cm}
        \centering
        \begin{tikzpicture}
    \tkzDefPoints{0/0/C, 4/0/D, 0/1.5/A, 4/1.5/B, 1/2.5/E, 3/-1/F}
    \tkzInterLL(A,B)(E,F)  \tkzGetPoint{O}
    \tkzInterLL(C,D)(E,F)  \tkzGetPoint{P}

    \tkzDrawSegments(A,B  C,D  E,F)
    \tkzMarkAngles[size=0.3](B,O,E  D,P,E  A,O,F)
    \tkzMarkAngles[size=0.4](F,O,B)
    \tkzLabelAngle[pos=0.5](B,O,E){$1$}
    \tkzLabelAngle[pos=0.5](D,P,E){$2$}
    \tkzLabelAngle[pos=0.5](A,O,F){$3$}
    \tkzLabelAngle[pos=0.6](F,O,B){$4$}
    \tkzLabelSegment[pos=1.0, right](A,B){$a$}
    \tkzLabelSegment[pos=1.0, right](C,D){$b$}
    \tkzLabelSegment[pos=0.0, right](E,F){$c$}
\end{tikzpicture}


        \caption{}\label{fig:czjh1-2-20}
    \end{minipage}
\end{figure}



根据平行线的这个性质,从两直线平行可以推出同位角相等,在 \ref{subsec:czjh1-2-3} 节的\hyperref[liti:czjh1-2-3]{例题}里,我们曾由同位角相等推出了内错角相等。
把这两条结合起来,可以写出如下推理:如图 \ref{fig:czjh1-2-20},

$\because$ \quad $a \pingxing b$(已知)

$\therefore$ \quad $\angle 1 = \angle 2$(两直线平行,同位角相等)。

又 $\because$ \quad $\angle 1 = \angle 3$(对顶角相等)。

$\therefore$ \quad $\angle 3 = \angle 2$ (等量代换)。

由此,我们得到平行线的另一个性质:

\begin{xingzhi}
    两条平行线被第三条直线所截,内错角相等。
\end{xingzhi}

这句话可以简单说成:\begin{xingzhi}
    两直线平行,内错角相等。
\end{xingzhi}

$\because$ \quad $a \pingxing b$ (已知),

$\therefore$ \quad $\angle 1 = \angle 2$ (两直线平行,同位角相等)。

$\because$ \quad $c$ 是直线,

$\therefore$ \quad $\angle 1 + \angle 4 = 180^\circ$(邻补角定义)。

$\therefore$ \quad $\angle 2 + \angle 4 = 180^\circ$(等量代换)。

由此,我们得到平行线的又一个性质:

\begin{xingzhi}
    两条平行线被第三条直线所截,同旁内角互补。
\end{xingzhi}

这句话可以简单说成:\begin{xingzhi}
    两直线平行,同旁内角互补。
\end{xingzhi}

下面,我们来看图 \ref{fig:czjh1-2-21}, 其中 $CD \pingxing OB$、 $EF \pingxing OA$。
我们利用平行线的性质来研究 $\angle 1$ 与 $\angle O$ 有什么关系。

$\because$ \quad $CD \pingxing OB$ (已知),

$\therefore$ \quad $\angle 1 = \angle 2$ (两直线平行,内错角相等);

$\because$ \quad $EF \pingxing OA$ (已知),

$\therefore$ \quad $\angle O = \angle 2$ (两直线平行,同位角相等);

$\therefore$ \quad $\angle 1 = \angle O$ (等量代换)。

由此,我们得到了两边分别平行的两个角 $\angle 1 = \angle O$。


\begin{wrapfigure}[8]{r}{6cm}
    \centering
    \begin{tikzpicture}
    \tkzDefPoints{0/0/O, 4/0/B, 0/1.5/C, 4/1.5/D, 1.5/2.5/A, 3.5/2.5/E, 2/0/F}
    \tkzInterLL(C,D)(E,F)  \tkzGetPoint{P}

    \tkzDrawSegments(O,B  O,A  C,D)
    \tkzDrawLine[add=0 and 0.2](E,F)
    \tkzMarkAngles[size=0.3](C,P,F  B,F,E  D,P,E)
    \tkzMarkAngles[size=0.4](F,P,D  E,P,C)
    \tkzLabelAngle[pos=0.5](C,P,F){$1$}
    \tkzLabelAngle[pos=0.5](B,F,E){$2$}
    \tkzLabelAngle[pos=0.5](D,P,E){$4$}
    \tkzLabelAngle[pos=0.6](F,P,D){$3$}
    \tkzLabelAngle[pos=0.6](E,P,C){$5$}

    \tkzLabelPoints[below](O,B,C,D)
    \tkzLabelPoints[below right](F)
    \tkzLabelPoints[right](A,E)
\end{tikzpicture}


    \caption{}\label{fig:czjh1-2-21}
\end{wrapfigure}

在图 \ref{fig:czjh1-2-21} 中, 除了 $\angle 1$ 与 $\angle O$ 的两边分别平行以外,
$\angle 3$、$\angle 4$、$\angle 5$ 的两边也是与 $\angle O$ 的两边分别平行的。
我们再来研究它们与 $\angle O$ 有什么关系。


$\because$ \quad $\angle 1 = \angle O$ (已证),

又 $\because$ \quad $\angle 4 = \angle 1$ (对顶角相等),

$\therefore$ \quad $\angle 4 = \angle O$(等量代换)。

又 $\because$ \quad \begin{zmtblr}[t]{}
    $\angle 3 + \angle 1 = 180^\circ$, \\
    $\angle 5 + \angle 1 = 180^\circ$ (邻补角定义), \\
\end{zmtblr}

$\therefore$ \quad \begin{zmtblr}[t]{}
    $\angle 3 + \angle O = 180^\circ$, \\
    $\angle 5 + \angle O = 180^\circ$ (等量代换)。
\end{zmtblr}

$\angle 1$、$\angle 3$、$\angle 4$、$\angle 5$ 的两边都分别与 $\angle O$ 的两边行,
但其中 $\angle 1$、$\angle 4$ 与 $\angle O$ 相等, $\angle 3$、$\angle 5$ 与 $\angle O$ 互补。
由此我们得到如下的性质:

\begin{xingzhi}
    如果一个角的两边分别平行于另一个角的两边,那么这两个角相等或互补。
\end{xingzhi}


\begin{lianxi}

口答下列各题:

\xiaoti{如图,一条公路,两次拐弯后,和原来的方向平行。第一次拐的角 $\angle B$ 是 $142^\circ$,
    第二次拐的角 $\angle C$ 是多少度?为什么?
}

\begin{figure}[htbp]
    \centering
    \begin{minipage}[b]{4cm}
        \centering
        \begin{tikzpicture}
    \tkzDefPoints{0/0/A, 1/0/B}
    \tkzDefPointOnCircle[R = center B angle 38  radius 1.5] \tkzGetPoint{C}
    \tkzDefLine[parallel=through C](A,B) \tkzGetPoint{D}
    \tkzDrawSegments[dashed](A,B  B,C  C,D)
    \tkzMarkAngles[size=0.3](C,B,A  B,C,D)
    \tkzLabelPoints[below](B)
    \tkzLabelPoints[above](C)

    \begin{scope}[yshift=0.5cm]
        \tkzDefPoints{0/0/A, 1/0/B}
        \tkzDefPointOnCircle[R = center B angle 38  radius 1.5] \tkzGetPoint{C}
        \tkzDefLine[parallel=through C](A,B) \tkzGetPoint{D}
        \tkzDrawSegments[thick](A,B  B,C  C,D)
    \end{scope}
    \begin{scope}[yshift=-0.5cm]
        \tkzDefPoints{0/0/A, 1/0/B}
        \tkzDefPointOnCircle[R = center B angle 38  radius 1.5] \tkzGetPoint{C}
        \tkzDefLine[parallel=through C](A,B) \tkzGetPoint{D}
        \tkzDrawSegments[thick](A,B  B,C  C,D)
    \end{scope}
\end{tikzpicture}


        \caption*{(第 1 题)}
    \end{minipage}
    \qquad
    \begin{minipage}[b]{4.5cm}
        \centering
        \begin{tikzpicture}
    \tkzDefPoints{0/0/A, 2/0/O, 3/0/E}
    \tkzDefPointOnCircle[R = center A angle -110  radius 1.5] \tkzGetPoint{B}
    \tkzDefPointOnCircle[R = center O angle 70    radius 1]   \tkzGetPoint{C}
    \tkzDefPointOnCircle[R = center O angle -110  radius 1.5] \tkzGetPoint{D}

    \tkzDrawSegments(A,B  C,D  A,E)
    \tkzMarkAngles[size=0.3](B,A,E  C,O,A  D,O,E)
    \tkzMarkAngles[size=0.4](A,O,D)
    \tkzLabelAngle[pos=0.5](B,A,E){$1$}
    \tkzLabelAngle[pos=0.5](C,O,A){$2$}
    \tkzLabelAngle[pos=0.5](D,O,E){$3$}
    \tkzLabelAngle[pos=0.6](A,O,D){$4$}

    \tkzLabelPoints[below](B,D,E)
    \tkzLabelPoints[above](A)
    \tkzLabelPoints[right](C)
\end{tikzpicture}


        \caption*{(第 2 题)}
    \end{minipage}
    \qquad
    \begin{minipage}[b]{4.5cm}
        \centering
        \begin{tikzpicture}
    \tkzDefPoints{0/0/B, 3/0/C}
    \tkzDefPointOnCircle[R = center B angle 60   radius 1.5] \tkzGetPoint{a1}
    \tkzDefPointOnCircle[R = center C angle 140  radius 1.5] \tkzGetPoint{a2}
    \tkzInterLL(B,a1)(C,a2)  \tkzGetPoint{A}
    \tkzDefPointOnLine[pos=0.6](A,B)  \tkzGetPoint{D}
    \tkzDefPointOnLine[pos=0.6](A,C)  \tkzGetPoint{E}

    \tkzDrawPolygon(A,B,C)
    \tkzDrawSegment(D,E)
    \tkzLabelPoints[above](A)
    \tkzLabelPoints[left](B,D)
    \tkzLabelPoints[right](C)
    \tkzLabelPoints[right, yshift=0.3em](E)
\end{tikzpicture}

        \caption*{(第 3 题)}
    \end{minipage}
\end{figure}


\xiaoti{如图,已知平行线,$AB$、$CD$ 被直线所截。}
\begin{xiaoxiaotis}

    \xxt{从 $\angle 1 = 110^\circ$,可以知道 $\angle 2$ 是多少度?为什么?}

    \xxt{从 $\angle 1 = 110^\circ$,可以知道 $\angle 3$ 是多少度?为什么?}

    \xxt{从 $\angle 1 = 110^\circ$,可以知道 $\angle 4$ 是多少度?为什么?}

\end{xiaoxiaotis}


\xiaoti{已知 $D$ 是 $AB$ 上一点, $E$ 是 $AC$ 上一点。 $\angle ADE = 60^\circ$。
    $\angle B = 60^\circ$, $\angle AED = 40^\circ$
}
\begin{xiaoxiaotis}

    \xxt{$DE$ 和 $BC$ 平行吗?为什么?}
    \xxt{$\angle C$ 是多少度?为什么?}
\end{xiaoxiaotis}

\end{lianxi}
