\subsection{直角三角形的性质}\label{subsec:czjh1-3-12}
\begin{enhancedline}

直角三角形也是一种常见的特殊三角形,它除了有一般三角形的性质外,还有一些特殊性质。

因为直角三角形有一个角是直角,根据三角形内角和定理可推出下面的定理:

\begin{dingli}[定理1]
    在直角三角形中,两个锐角互余。
\end{dingli}

下面,我们再来研究直角三角形另外一些性质。

在图 \ref{fig:czjh1-3-48} 中, 从 $Rt \triangle$\footnote{符号 “$Rt \triangle$” 表示直角三角形。}$ABC$ 的直角顶点 $C$,
作射线 $CD$, 交 $AB$ 于 $D$, 使 $\angle ACD = \angle A$。

\begin{wrapfigure}[6]{r}{7cm}
    \centering
    \begin{tikzpicture}
    \tkzDefPoints{0/0/B, 2/0/C, 2/3/A}
    \tkzDefMidPoint(A,B)  \tkzGetPoint{D}
    \tkzDrawPolygon(A,B,C)
    \tkzDrawSegment(C,D)
    \tkzMarkAngle[size=0.6](A,C,D)
    \extkzLabelAngel[0.4](D,C,B){$1$}
    \tkzLabelPoints[right](A)
    \tkzLabelPoints[left](D)
    \tkzLabelPoints[below](B,C)
\end{tikzpicture}


    \caption{}\label{fig:czjh1-3-48}
\end{wrapfigure}

我们看 $CD$ 和斜边 $AB$ 的大小有什么关系。

$\because$ \quad $\angle ACD = \angle A$,

$\therefore$ \quad $AD = CD$(等角对等边),

又 $\because$ \quad \begin{zmtblr}[t]{}
    $\angle B + \angle A = 90^\circ$(直角三角形两锐角互余),\\
    $\angle 1 + \angle ACD = 90^\circ$,
\end{zmtblr}


$\therefore$ \quad $\angle B = \angle 1$。

$\therefore$ \quad $BD = CD$。

于是得 \quad $AD = BD = CD$。

即 \quad $CD$ 是斜边 $AB$ 上的中线, 并且 $CD = \exdfrac{1}{2} AB$。

由此,我们得到下面的定理:

\begin{dingli}[定理2]
    在直角三角形中,斜边上的中线等于斜边的一半。
\end{dingli}

注意:从本节开始,在证明过程中,括号内的理由,可以只写主要公理和定理,已知、定义等一般不再注明。

由定理2 可知,在图 \ref{fig:czjh1-3-48} 中, $CD = BD$。
如果 $\angle A = 30^\circ$,那么,$\angle B= 60^\circ$, $\triangle DBC$ 是等边三角形,
所以,$BC = BD = \exdfrac{1}{2} AB$。 由此,得到下面的推论:

\begin{tuilun}[推论1]
    在直角三角形中,如果一个锐角等于 $30^\circ$, 那么它所对的直角边等于斜边的一半。
\end{tuilun}

反过来,如果在 $Rt \triangle ABC$ 中, $\angle C = Rt\angle$,$BC = \exdfrac{1}{2} AB$,
那么 $\triangle BCD$ 是等边三角形,$\angle B = 60^\circ$。所以 $\angle A = 30^\circ$。
于是,又有下面的推论:

\begin{tuilun}[推论2]
    在直角三角形中,如果一条直角边等于斜边的一半,那么这条直角边所对的角等于 $30^\circ$。
\end{tuilun}


\begin{lianxi}

\xiaoti{说出直角三角形有哪些重要性质。}

\xiaoti{在 $Rt \triangle ABC$ 中,$\angle C = Rt\angle$, $CD$ 是高。找出图中相等的角。}

\begin{figure}[htbp]
    \centering
    \begin{minipage}[b]{7cm}
        \centering
        \begin{tikzpicture}
    \tkzDefPoints{0/0/A, 4/0/B}
    \tkzDefTriangle[two angles=30 and 60](A,B)  \tkzGetPoint{C}
    \tkzDefLine[altitude](A,C,B)  \tkzGetPoint{D}
    \tkzDrawPolygon(A,B,C)
    \tkzDrawSegment(C,D)
    \tkzMarkRightAngle(A,C,B)
    \tkzMarkRightAngle(B,D,C)
    \tkzLabelPoints[left](A)
    \tkzLabelPoints[right](B)
    \tkzLabelPoints[above](C)
    \tkzLabelPoints[below](D)
\end{tikzpicture}


        \caption*{(第 2 题)}
    \end{minipage}
    \qquad
    \begin{minipage}[b]{7cm}
        \centering
        \begin{tikzpicture}[scale=0.5]
    \tkzDefPoints{0/0/A, 10/0/B, 5/0/D, 2.5/0/E}
    \tkzDefLine[perpendicular=through E](A,B)  \tkzGetPoint{x}
    \tkzCalcLength(A,D)  \tkzGetLength{ad}
    \tkzInterLC[R](E,x)(D,\ad)  \tkzGetFirstPoint{C}
    \tkzDrawPolygon(A,B,C)
    \tkzDrawSegments(C,D  C,E)
    \tkzMarkRightAngle[size=0.4](A,C,B)
    \tkzMarkRightAngle[size=0.4](C,E,A)
    \tkzLabelPoints[left](A)
    \tkzLabelPoints[right](B)
    \tkzLabelPoints[above](C)
    \tkzLabelPoints[below](D,E)
\end{tikzpicture}


        \caption*{(第 3 题)}
    \end{minipage}
\end{figure}


\xiaoti{在 $Rt \triangle ABC$, $CD$ 是斜边上的中线,$CE$ 是高。已知 $AB = 10\;\limi$,
    $DE = 2.5\;\limi$, 求 $CD$ 的长和 $\angle DCE$ 的度数。
}

\end{lianxi}

\liti 图 \ref{fig:czjh1-3-49} 是屋架的设计图的一部分,其中 $AB = 7.4$ 米,$D$ 是 $AB$ 的中点,
并且 $DE$、$BC$ 都垂直于 $AC$。 如果 $\angle BAC = 30^\circ$,
$DE$、$DC$ 和 $BC$ 的长各是多少米?

\begin{wrapfigure}[6]{r}{7cm}
    \centering
    \begin{tikzpicture}[scale=0.5]
    \tkzDefPoints{0/0/A, 1/0/c}
    \tkzDefPoint(30:7.4){B}
    \tkzDefMidPoint(A,B)  \tkzGetPoint{D}
    \tkzDefLine[altitude](A,B,c)  \tkzGetPoint{C}
    \tkzDefLine[altitude](A,D,c)  \tkzGetPoint{E}
    \tkzDrawPolygon(A,B,C)
    \tkzDrawSegments(C,D  D,E)
    \tkzLabelPoints[below](A,C,E)
    \tkzLabelPoints[above](B,D)
\end{tikzpicture}


    \caption{}\label{fig:czjh1-3-49}
\end{wrapfigure}


\jie 在 $\triangle ABC$ 中,

$\because$ \quad $BC \perp AC$, $\angle ACB = Rt \angle$, $\angle BAC = 30^\circ$,

$\therefore$ \quad $BC = \exdfrac{1}{2} AB$(在直角三角形中,$30^\circ$ 角所对的边等于斜边的一半)。

$\therefore$ \quad $BC = \exdfrac{1}{2} \times 7.4 = 3.7$(米)。

又 $\because$ \quad $D$ 是 $AB$ 中点, $CD$ 是中线,

$\therefore$ \quad $DC = \exdfrac{1}{2} AB$ (在直角三角形中,斜边上的中线等于斜边的一半)。

$\therefore$ \quad $DC = 3.7$ (米)。

在 $\triangle AED$ 中,同理可求得

\qquad $DE = \exdfrac{1}{2} AD = \exdfrac{1}{4} AB =  1.75$ (米)。

答: $DE$、 $DC$、$BC$ 的长分别是 $1.75$ 米、$3.7$ 米、$3.7$ 米。


\liti 已知: 在 $\triangle ABC$ 中, $\angle ACB = Rt \angle$, $AB = 2 AC$。
$CD$、 $CE$ 分别是中线和高(图 \ref{fig:czjh1-3-50})。

求证;$\angle ACE = \angle ECD = \angle DCB$。

\zhengming 在 $\triangle ABC$ 中,

$\because$ \quad $\angle ACB = Rt \angle$, $AB = 2 AC$,

$\therefore$ \quad $\angle B = 30^\circ$
 (在直角三角形中,如果一条直角边等于斜边的一半,那么这条直角边所对的角等于 $30^\circ$)。

又 $\because$ \quad $CD$ 是中线,

$\therefore$ \quad $CD = BD$(在直角三角形中,斜边上的中线等于斜边的一半),

$\therefore$ \quad $\angle DCB = \angle B = 30^\circ$ (等边对等角)。

在 $\triangle ACD$ 中,

$\because$ \quad $AC = \exdfrac{1}{2} AB = CD$, $CE$ 是高,

$\therefore$ \quad $\angle ACE = \angle DCE$ (等腰三角形底边上的高与顶角平分线重合)。

$\because$ \quad $\angle ACE = \exdfrac{1}{2} (90^\circ - \angle DCB) = \exdfrac{1}{2} (90^\circ - 30^\circ) = 30^\circ$,

$\therefore$ \quad $\angle ACE = \angle ECD = \angle DCB$。

\begin{figure}[htbp]
    \centering
    \begin{minipage}[b]{7cm}
        \centering
        \begin{tikzpicture}
    \tkzDefPoints{0/0/A, 4/0/B}
    \tkzDefTriangle[two angles=60 and 30](A,B)  \tkzGetPoint{C}
    \tkzDefMidPoint(A,B)  \tkzGetPoint{D}
    \tkzDefLine[altitude](A,C,B)  \tkzGetPoint{E}
    \tkzDrawPolygon(A,B,C)
    \tkzDrawSegments(C,D  C,E)
    \tkzMarkRightAngle(A,C,B)
    \tkzMarkRightAngle(C,E,A)
    \tkzLabelPoints[left](A)
    \tkzLabelPoints[right](B)
    \tkzLabelPoints[above](C)
    \tkzLabelPoints[below](D,E)
\end{tikzpicture}


        \caption{}\label{fig:czjh1-3-50}
    \end{minipage}
    \qquad
    \begin{minipage}[b]{7cm}
        \centering
        \begin{tikzpicture}
    \tkzDefPoints{0/0/A, 4/0/B}
    \tkzDefTriangle[two angles=30 and 60](A,B)  \tkzGetPoint{C}
    \tkzDefLine[altitude](A,C,B)  \tkzGetPoint{D}
    \tkzDrawPolygon(A,B,C)
    \tkzDrawSegment(C,D)
    \tkzMarkRightAngle(A,C,B)
    \tkzMarkRightAngle(B,D,C)
    \tkzLabelPoints[left](A)
    \tkzLabelPoints[right](B)
    \tkzLabelPoints[above](C)
    \tkzLabelPoints[below](D)
\end{tikzpicture}


        \caption*{(第 1 题)}
    \end{minipage}
\end{figure}


\begin{lianxi}

\xiaoti{(口答) 已知 $\triangle ABC$ 中,$\angle ACB = 90^\circ$, $CD$ 是高,
    $\angle A = 30^\circ$, $AB = 4 \;\limi$,
    依次求 $BC$、$\angle BCD$、$BD$、$AD$。
}


\xiaoti{一个人从山下沿 $30^\circ$ 的坡路登上山顶, 共走了 $500$ 米, 求这座山的高度。}

\end{lianxi}

\end{enhancedline}

