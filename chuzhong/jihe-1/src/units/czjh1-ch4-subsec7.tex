\subsection{正方形}\label{subsec:czjh1-4-7}

有一组邻边相等并且有一个角是直角的平行四边形叫做\zhongdian{正方形}。

由正方形的定义可以得知,正方形既是有一组邻边相等的矩形,又是有一个角是直角的菱形。
所以,正方形同时具有矩形和菱形的所有性质。这些图形的包含关系如图 \ref{fig:czjh1-4-27}。

\begin{figure}[htbp]
    \centering
    \begin{minipage}[b]{7cm}
        \centering
        \begin{tikzpicture}
    \draw (-.9,0) circle [x radius = 2cm, y radius = .8cm];
    \draw ( .9,0) circle [x radius = 2cm, y radius = .8cm];
    \draw (   0,0) node {正方形集合};
    \draw (-2.0,0) node {矩形集合};
    \draw ( 2.0,0) node {菱形集合};
\end{tikzpicture}


        \caption{}\label{fig:czjh1-4-27}
    \end{minipage}
    \qquad
    \begin{minipage}[b]{7cm}
        \centering
        \begin{tikzpicture}
    \pgfmathsetmacro{\a}{3.5}
    \tkzDefPoints{0/0/B,  \a/0/C, 0/\a/A, \a/\a/D}
    \tkzDefPointOnLine[pos=0.3](A,B)  \tkzGetPoint{A'}
    \tkzDefPointOnLine[pos=0.3](B,C)  \tkzGetPoint{B'}
    \tkzDefPointOnLine[pos=0.3](C,D)  \tkzGetPoint{C'}
    \tkzDefPointOnLine[pos=0.3](D,A)  \tkzGetPoint{D'}
    \tkzDrawPolygon(A,B,C,D)
    \tkzDrawPolygon(A',B',C',D')
    \extkzLabelAngel[0.4](D',A',A){$1$}
    \extkzLabelAngel[0.8](A,D',A'){$2$}
    \extkzLabelAngel[0.8](B,A',B'){$3$}
    \tkzLabelPoints[left](A,B,A')
    \tkzLabelPoints[right](C,D,C')
    \tkzLabelPoints[above](D')
    \tkzLabelPoints[below](B')
\end{tikzpicture}


        \caption{}\label{fig:czjh1-4-28}
    \end{minipage}
\end{figure}

容易知道,正方形有下面性质:

\begin{dingli}[正方形性质定理1]
    正方形的四个角都是直角,四条边都相等。
\end{dingli}

\begin{dingli}[正方形性质定理2]
    正方形的两条对角线相等,并且互相垂直平分,每条对角线平分一组对角。
\end{dingli}

反过来,如果一个图形既是矩形又是菱形,那么根据定义就可以断定它是正方形。


\liti[0] 已知:如图 \ref{fig:czjh1-4-28},四边形 $ABCD$ 是正方形,$AA' = BB' = CC' = DD'$。

求证:四边形 $A'B'C'D'$ 是正方形。

\zhengming  $\because$ \quad 四边形 $ABCD$ 是正方形,

$\therefore$ \quad $AB = BC = CD = DA$ (正方形的四条边都相等)。

又 $\because$ \quad $AA' = BB' = CC' = DD'$,

$\therefore$ \quad $D'A = A'B = B'C = C'D$。

$\because$ \quad $\angle A = \angle B = \angle C = \angle D = 90^\circ$ (正方形的四个角都是直角),

$\therefore$ \quad $\triangle AA'D' \quandeng \triangle BB'A' \quandeng \triangle CC'B' \quandeng \triangle DD'C'$。

$\therefore$ \quad $D'A' = A'B' = B'C' = C'D'$。

$\therefore$ \quad 四边形 $A'B'C'D'$ 是菱形(四边都相等的四边形是菱形)。

又 $\because$ \quad $\angle 2 = \angle 3$, $\angle 1 + \angle 2 = 90^\circ$,

$\therefore$ \quad $\angle 1 + \angle 3 = 90^\circ$。

$\because$ \quad $\angle D'A'B' = 180^\circ - \angle 1 - \angle 3 = 180^\circ - 90^\circ = 90^\circ$,

$\therefore$ \quad 四边形 $A'B'C'D'$ 是正方形(有一个角是直角的菱形是正方形)。


\begin{lianxi}

\xiaoti{两条对角线互相垂直平分且相等的四边形是哪种四边形?为什么?}

\xiaoti{正方形的一条对角线和一边所成的角是多少度的角?为什么?}

\xiaoti{如果一个菱形的两条对角线相等,那么它一定是正方形。为什么?}

\xiaoti{如果一个矩形的两条对角线互相垂直,那么它一定是正方形。为什么?}

\end{lianxi}

