\fuxiti
\begin{enhancedline}
\begin{xiaotis}

\xiaoti{当 $x$ 取什么数值时,下列分式的值为正?}
\begin{xiaoxiaotis}

    \begin{tblr}{columns={18em, colsep=0pt}}
        \xxt{$\dfrac{2}{2x-3}$;} & \xxt{$\dfrac{1}{3-x}$。}
    \end{tblr}

\end{xiaoxiaotis}


\xiaoti{计算:}
\begin{xiaoxiaotis}

    \begin{tblr}{columns={18em, colsep=0pt}, rows={rowsep+=.25em}}
        \SetCell[c=2]{} \xxt{$\dfrac{a^2+7a+10}{a^2-a+1} \cdot \dfrac{a^3+1}{a^2+4a+4} \div \dfrac{a^2+6a+5}{a+2}$;} \\
        \xxt{$\dfrac{2x+4}{x^2-4x+4} \div \dfrac{x^3+8}{2x-4} \cdot (x^2-4)$;} \\
        \xxt{$\dfrac{a^3}{a-1} - a^2 - a - 1$;} & \xxt{$\dfrac{1}{a-1} - \dfrac{1}{a+1} - \dfrac{2}{a^2+1}$;} \\
        \SetCell[c=2]{} \xxt{$\dfrac{a}{(a-b)(a-c)} + \dfrac{b}{(b-c)(b-a)} + \dfrac{c}{(c-a)(c-b)}$;} \\
        \xxt{$\dfrac{a^2+5a+4}{a^2+a-6} \div \dfrac{a^2+3a-4}{a^2-6a+8} - 1$;} & \xxt{$\dfrac{x-15}{2x^2+6x} - \dfrac{9-3x^2}{x^2-9} - 3$;} \\
        \xxt{$\left(\dfrac{e + \dfrac{1}{e}}{2}\right)^2 - \left(\dfrac{e -\dfrac{1}{e}}{2}\right)^2$;} & \xxt{$\dfrac{a^n+b^n}{a^n-b^n} - \dfrac{4a^nb^n}{a^{2n} - b^{2n}}$。}
    \end{tblr}

\end{xiaoxiaotis}


\xiaoti{$x$ 是什么数时,$\dfrac{1}{x^2-1} + \dfrac{2}{x+1} - \dfrac{1}{x-1}$ 的值是零?}

\xiaoti{用带余除法把下列分式化成整式与真分式(分子、分母都是同一个字母的整式,并且分子的次数低于分母的次数)的和的形式:}
\begin{xiaoxiaotis}

    \begin{tblr}{columns={18em, colsep=0pt}}
        \xxt{$\dfrac{x^3+2x^2-3x+5}{x^2-x-3}$;} & \xxt{$\dfrac{x^3+x^2-1}{x+2}$。}
    \end{tblr}

\end{xiaoxiaotis}

\xiaoti{解下列方程:}
\begin{xiaoxiaotis}

    \xxt{$\dfrac{7}{x^2+x} + \dfrac{3}{x^2-x} = \dfrac{6}{x^2-1}$;}

    \xxt{$\dfrac{x^2-8x+16}{x^2-4x+4} + \left(1 + \dfrac{2}{x-2}\right)^2 = \dfrac{2x}{x-2}$;}

    \xxt{$\dfrac{1-x}{1+x+x^2} + \dfrac{6}{1-x^3} = \dfrac{1}{1-x}$;}

    \xxt{$\dfrac{x-4}{x-5} - \dfrac{x-5}{x-6} = \dfrac{x-7}{x-8} - \dfrac{x-8}{x-9}$ \\
        (提示:先分别计算两边)。
    }

\end{xiaoxiaotis}

\xiaoti{解下列方程组:}
\begin{xiaoxiaotis}

    \begin{tblr}{columns={18em, colsep=0pt}, rows={rowsep+=.25em}}
        \xxt{$\begin{cases}
                \dfrac{x-1}{x+5} = \dfrac{y-4}{y+2} \douhao \\[1em]
                \dfrac{x+3}{x} = \dfrac{y-6}{y} \fenhao
            \end{cases}$} & \xxt{$\begin{cases}
                \dfrac{35}{x+y} + \dfrac{20}{x-y} = 3 \douhao \\[1em]
                \dfrac{28}{x+y} + \dfrac{25}{x-y} = 3 \fenhao
            \end{cases}$} \\
        \xxt{$\begin{cases}
                \dfrac{2}{x+4} + \dfrac{y}{2} = 5 \douhao \\[1em]
                \dfrac{3}{x+4} - \dfrac{y}{3} = 1 \fenhao
            \end{cases}$} & \xxt{$\begin{cases}
                \dfrac{x+y}{3} - \dfrac{3}{x-y} = -\dfrac{1}{6} \douhao \\[1em]
                \dfrac{x+y}{2} + \dfrac{2}{x-y} = 3 \juhao
            \end{cases}$}
    \end{tblr}

\end{xiaoxiaotis}

\xiaoti{}%
\begin{xiaoxiaotis}%
    \xxt[\xxtsep]{已知 $H = \dfrac{d^2Sn}{12000} \quad (n \neq 0, d \neq 0)$,求 $S$;}

    \xxt{已知 $\dfrac{U - V}{R} = \dfrac{V}{S} \quad (R + S \neq 0)$,求 $V$;}

    \xxt{已知 $e = \dfrac{m - a}{n - a} \quad (e \neq 1)$,求 $a$。}

\end{xiaoxiaotis}

\xiaoti{由公式 $\dfrac{1}{R} = \dfrac{1}{r_1} + \dfrac{1}{r_2}$,推出 $R = \dfrac{r_2}{1 + \dfrac{r_2}{r_1}} \quad (r_1 + r_2 \neq 0)$。}

\xiaoti{用代数式表示图中阴影部分的面积。已知这个面积是 $S$,求用 $S$,$R$,$r$ 的代数式表示 $a$。}

\begin{figure}[htbp]
    \centering
    \begin{tikzpicture}[>=Stealth,
    every node/.style={fill=white, inner sep=1pt},
]
    \pgfmathsetmacro{\r}{1}
    \pgfmathsetmacro{\R}{1.6}
    \pgfmathsetmacro{\a}{4}

    \draw [thick, pattern={mylines[angle=45, distance={5pt}]}]
        (0, \R) arc [start angle=90, end angle=270, radius=\R]
        -- +(\a, 0)
        arc [start angle=270, end angle=450, radius=\R]
        -- cycle;
    \draw [thick, fill=white]
        (0, \r) arc [start angle=90, end angle=270, radius=\r]
        -- +(\a, 0)
        arc [start angle=270, end angle=450, radius=\r]
        -- cycle;

    \draw [dash dot] (-1.5*\R, 0) -- (\a + 1.5*\R, 0);
    \draw [dash dot] (0, 1.2*\R) -- (0, -1.2*\R);
    \draw [dash dot] (\a, 1.2*\R) -- (\a, -1.2*\R);

    \draw [->] (0, 0) -- (200:\r) node [midway] {$r$};
    \draw [->] (\a, 0) -- +(320:\R) node [pos=0.4] {$R$};
    \draw [<->] (0, -0.3-\R) to [xianduan={above=0.3cm}] node {$a$} (\a, -0.3-\R);
\end{tikzpicture}

    \caption*{(第 9 题)}
\end{figure}


列出分式方程(组)解下列应用题;

\xiaoti{某煤矿现在平均每天比原计划多采 330 吨媒,已知现在采 33000 吨煤所需的时间和
    原计划采 23100 吨煤的时间相同。问现在平均每天采煤多少吨。
}

\xiaoti{我军某部由驻地到距离 30 千米的地方去执行任务,由于情况发生了变化,行军速度必须是原计划速度的 $1.5$ 倍,
    才能按要求提前 2 小时到达。求急行军的速度。
}

\xiaoti{一台电子收报机,它的译电效率相当于人工译电效率 75 倍,译电 3000 个字比人工少用 2 小时 28 分。
    这台收报机与人工每分各译电多少字?
}

\xiaoti{甲乙二人合打一份稿件,4 小时后,甲另有任务,余下部分由乙单独又用了 6 小时才完成。
    已知甲打 6 小时的稿件,乙要打 7 小时 30 分。问甲乙单独完成各需多少小时。
}

\xiaoti{从甲站到乙站共有 80 千米,其中开始的 20 千米是平路,然后是 30 千米的上坡路,余下的又是平路。
    火车从甲站出发,经过 50 分,到达甲乙两站的中点,再经过 45 分到达乙站,求火车在平路上和上坡路上的速度。
}

\end{xiaotis}
\end{enhancedline}
