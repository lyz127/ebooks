\xiaojie
\begin{enhancedline}

一、本章主要内容是分式的概念、基本性质和运算,以及有关分式方程的一些初步知识。

二、形如 $\dfrac{A}{B}$ 的式子叫分式,其中 $B$ 里含有字母。$\dfrac{A}{B}$ 表示 $A \div B$ 所得的商,
因此,$B$ 的值不能为零,这是分式概念中的一个要点。

三、分式的基本性质是
$$\dfrac{A}{B} = \dfrac{AM}{BM} \quad (M \neq 0) \juhao $$
它是分式运算的重要依据。

四、可以对比分数学习分式,分式的约分、通分以及四则运算等都与分数类似。
还要注意分式与整式的联系,整式运算是分式运算的基础。

五、根据需要,一个公式有时要变换成不同的形式。公式变形往往就是解含有字母已知数的方程。
解含有字母已知数的方程和解数字已知数的方程相同,只是将数的运算换成式的运算,但要注意,
当出现分式时,字母取值不能使分母的值为零。

六、解分式方程的一般步骤是:

1. 在方程的两边都乘以最简公分母,约去分母,化成整式方程;

2. 解这个整式方程;

3. 把整式方程的根代入最简公分母,看结果是不是零,使最简公分母为零的根是原方程的增根,应舍去。

\end{enhancedline}

