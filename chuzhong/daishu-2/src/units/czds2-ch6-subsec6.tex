\subsection{多项式的乘法}\label{subsec:6-6}

现在我们来研究多项式的乘法。我们来计算
$$ (a + b) (m + n) \juhao $$

这是多项式乘以多项式。先把 $(m + n)$ 看成一个单项式,运用单项式与多项式相乘的法则,得
$$ (a + b) (m + n) = a(m + n) + b(m + n) \douhao $$
再运用单项式与多项式相乘的法则,就得
\begin{align*}
    (a + b) (m + n) &= a(m + n) + b(m + n) \\
                    &= am + an + bm + bn \douhao
\end{align*}
即
\begin{figure}[H]
    \begin{minipage}{7cm}
        \begin{align*}
                & \tikz [overlay, >=Stealth] {
                    \draw [->]  (.8em, .6em) arc [start angle=120, end angle=60, radius=3em];
                    \draw [->]  (.6em, .6em) arc [start angle=120, end angle=60, radius=5em];
                  }
                  (a +
                    \tikz [overlay, >=Stealth] {
                        \draw [->]  (.2em, -.2em) arc [start angle=240, end angle=300, radius=1.5em];
                        \draw [->]  (.1em, -.24em) arc [start angle=240, end angle=300, radius=3.5em];
                    }
                  b) (m + n) \\[1em]
            ={} & am + an + bm + bn \juhao
        \end{align*}
    \end{minipage}
    \begin{minipage}{7cm}
        \centering
        \begin{tikzpicture}[>=Stealth,
    every node/.style={fill=white, inner sep=1pt},
]
    \pgfmathsetmacro{\a}{2.5}
    \pgfmathsetmacro{\b}{1.5}
    \pgfmathsetmacro{\m}{2}
    \pgfmathsetmacro{\n}{1}

    \draw (0, 0) rectangle (\a + \b, \m + \n);
    \draw (0, \m) -- (\a + \b, \m);
    \draw (\a, 0) -- (\a, \m + \n);

    \node at (\a/2, \m/2) {$am$};
    \node at (\a + \b/2, \m/2) {$bm$};
    \node at (\a/2, \m + \n/2) {$an$};
    \node at (\a + \b/2, \m + \n/2) {$bn$};

    \draw [<->] (0, -0.3) to [xianduan={above=0.3cm}] node {$a$} (\a, -0.3);
    \draw [<->] (\a, -0.3) to [xianduan={above=0.3cm}] node {$b$} (\a+\b, -0.3);
    \draw [<->] (\a+\b+0.3, 0) to [xianduan={above=0.3cm}] node [rotate=90] {$m$} (\a+\b+0.3, \m);
    \draw [<->] (\a+\b+0.3, \m) to [xianduan={above=0.3cm}] node [rotate=90] {$n$} (\a+\b+0.3, \m+\n);
\end{tikzpicture}


        \caption{}\label{fig:6-2}
    \end{minipage}
\end{figure}

这个结果也可以从图 \ref{fig:6-2} 明显地反映出来。

一般地,\zhongdian{多项式与多项式相乘,先用一个多项式的每一项乘以另一个多项式的每一项,再把所得的积相加。}

\liti 计算:
\begin{xiaoxiaotis}

    \xxt{$(x + 2y) (5a + 3b)$;}

    \xxt{$(2x - 3) (x + 4)$;}

    \xxt{$(3x + y) (x - 2y)$。}

\resetxxt
\jie \xxt{$\begin{aligned}[t]
        & (x + 2y) (5a + 3b) \\
    ={} & x \cdot 5a + x \cdot 3b + 2y \cdot 5a + 2y \cdot 3b \\
    ={} & 5ax + 3bx + 10ay + 6by \fenhao
\end{aligned}$}

\xxt{$\begin{aligned}[t]
        & (2x - 3) (x + 4) \\
    ={} & 2x^2 + 8x - 3x - 12 \\
    ={} & 2x^2 + 5x - 12 \fenhao
\end{aligned}$}

\xxt{$\begin{aligned}[t]
        & (3x + y) (x - 2y) \\
    ={} & 3x^2 - 6xy + xy - 2y^2 \\
    ={} & 3x^2 - 5xy - 2y^2 \juhao
\end{aligned}$}

\end{xiaoxiaotis}


\begin{enhancedline}
\liti 计算:
\begin{xiaoxiaotis}

    \xxt{$\left(-\dfrac{a}{2} + 3b^2\right) (a^2 - 2b)$;}

    \xxt{$(x + y) (x^2 -xy + y^2)$。}

\resetxxt
\jie \xxt{$\begin{aligned}[t]
        & \left(-\dfrac{a}{2} + 3b^2\right) (a^2 - 2b) \\
    ={} & -\dfrac{a^3}{2} + ab + 3a^2b^2 - 6b^3 \fenhao
\end{aligned}$;}

\xxt{$\begin{aligned}[t]
        & (x + y) (x^2 -xy + y^2) \\
    ={} & x^3 - x^2y + xy^2 + x^2y - xy^2 + y^3 \\
    ={} & x^3 + y^3 \juhao
\end{aligned}$}

\end{xiaoxiaotis}
\end{enhancedline}

\lianxi
\begin{xiaotis}

\xiaoti{(口答)计算:}
\begin{xiaoxiaotis}

    \begin{tblr}{columns={18em, colsep=0pt}}
        \xxt{$(m + n) (u + v)$;} & \xxt{$(x + y) (a - b)$;} \\
        \xxt{$(p - q) (r + s)$;} & \xxt{$(a - b) (c - d)$。}
    \end{tblr}

\end{xiaoxiaotis}

\xiaoti{计算:}
\begin{xiaoxiaotis}

    \begin{tblr}{columns={18em, colsep=0pt}}
        \xxt{$(2n + 6) (n - 3)$;} & \xxt{$(2x + 3) (3x - 1)$;} \\
        \xxt{$(2a - 3b) (a + 5b)$;} & \xxt{$(3x - 2y) (3x + 2y)$;} \\
        \xxt{$(2a + 3) \left(\dfrac{3}{2}b - 5\right)$;} & \xxt{$(2x + 5) (2x + 5)$。}
    \end{tblr}

\end{xiaoxiaotis}

\xiaoti{计算:}
\begin{xiaoxiaotis}

    \begin{tblr}{columns={18em, colsep=0pt}, column{2}={20em}}
        \xxt{$(x + 1) (x^2 - 2x + 3)$;} & \xxt{$(x - 1) (x^2 + x + 1)$;} \\
        \xxt{$(4x - 3) (5x^2 - 4x + 7)$;} & \xxt{$(3x + 2) (3x - 2) (x^2 - 1)$;} \\
        \xxt{$(3a - 2) (a - 1) + (a + 1) (a + 2)$;} & \xxt{$(2x^2 - 1)(x - 4) - (x^2 + 3)(2x - 5)$。}
    \end{tblr}

\end{xiaoxiaotis}

\end{xiaotis}
\lianxijiange

\liti 计算:
\begin{xiaoxiaotis}

    \xxt{$(x + 2)(x + 5)$;}

    \xxt{$(y + 2)(y - 5)$。}

\resetxxt
\jie \xxt{$\begin{aligned}[t]
        & (x + 2)(x + 5) \\
    ={} & x^2 + 5x + 2x + 10 \\
    ={} & x^2 + 7x + 10 \fenhao
\end{aligned}$}

\xxt{$\begin{aligned}[t]
        & (y + 2)(y - 5) \\
    ={} & y^2 - 5y + 2y - 10 \\
    ={} & y^2 - 3y - 10 \juhao
\end{aligned}$}

\end{xiaoxiaotis}

从例 3 可以看出, 含有一个相同字母的两个一次二项式相乘, 得到的积是同一个字母的二次多项式。
从例 3 还可以看出, 如果因式中一次项的系数都是 1 , 那么积的二次项系数也是 1 。
这时, 由于积的一次项是由两个因式中的常数项分别乘以两个因式中的一次项后, 合并同类项得到的,
因此, 积的一次项系数等于两个因式中常数项的和。 积的常数项等于两个因式中常数项的积。
因此, 如果用 $a$,$b$ 分别表示两个因式中的常数项, 那么,
\begin{center}
    \framebox{\quad $(x + a)(x + b) = x^2 + (a + b)x + ab$。\;}
\end{center}

\liti 解下列方程:
\begin{xiaoxiaotis}

    \xxt{$(x + 3)(x - 4) = x^2 - 16$;}

    \xxt{$3x(x + 2) + (x + 1)(x - 1) = 4(x^2 + 8)$。}

\resetxxt
\jie \xxt{$\begin{aligned}[t]
    (x + 3)(x - 4) &= x^2 - 16, \\
    x^2 - x - 12 &= x^2 - 16, \\
    -x &= -4, \\
    x &= 4 \juhao
\end{aligned}$}

\xxt{$\begin{aligned}[t]
    3x(x + 2) + (x + 1)(x - 1) &= 4(x^2 + 8), \\
    3x^2 + 6x + x^2 - 1 &= 4x^2 + 32, \\
    6x &= 33, \\
    x &= 5\dfrac{1}{2} \juhao
\end{aligned}$}

\end{xiaoxiaotis}

\lianxi
\begin{xiaotis}

\xiaoti{计算:}
\begin{xiaoxiaotis}

    \begin{tblr}{columns={18em, colsep=0pt}}
        \xxt{$(x + 1)(x + 4)$;} & \xxt{$(m - 2)(m + 3)$;} \\
        \xxt{$(y + 4)(y - 5)$;} & \xxt{$(x - 3)(x - 5)$;} \\
        \xxt{$\left(y - \dfrac{1}{2}\right)\left(y + \dfrac{1}{3}\right)$;} & \xxt{$(7x + 8)(6x - 5)$;} \\
        \xxt{$\left(\dfrac{1}{2}x + 4\right)\left(6x - \dfrac{3}{4}\right)$;} & \xxt{$(y^2 + y + 1)(y + 2)$;} \\
        \xxt{$(x + 2)(x + 3) - x(x + 1) - 8$;} & \xxt{$(3y - 1)(2y - 3) + (6y - 5)(y - 4)$。}
    \end{tblr}

\end{xiaoxiaotis}

\xiaoti{解下列方程:}
\begin{xiaoxiaotis}

    \xxt{$(2x + 3)(x - 4) - (x + 2)(x - 3) = x^2 + 6$;}

    \xxt{$2x(3x - 5) - (2x - 3)(3x + 4) = 3(x + 4)$。}

\end{xiaoxiaotis}

\end{xiaotis}

