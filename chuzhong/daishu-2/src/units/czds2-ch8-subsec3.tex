\subsection{约分}\label{subsec:8-3}
\begin{enhancedline}

与分数的约分类似,根据分式的基本性质,把一个分式的分子与分母的公因式约去,叫做分式的\zhongdian{约分}。
分子与分母没有公因式的分式叫做\zhongdian{最简分式}。
约分时,通常要把分子与分母所有的公因式都约去,使所得结果成为最简分式或整式。

例如,分式 $\dfrac{6ab^2}{8b^3}$ 的分子与分母中都有因式 $b$, 这因式的最低次幂是 $b^2$,
分子和分母的系数的最大公约数是 2, 因此可以约去 $2b^2$,即
$$ \dfrac{6ab^2}{8b^3} = \dfrac{3a \cdot 2b^2}{4b \cdot 2b^2} = \dfrac{3a}{4b} \juhao $$

又如,分式 $\dfrac{x^3 - 2x^2y}{x^3y - 2xy^2}$ 的分子与分母分别是
$$ x^3 - 2x^2y = x^2(x - 2y)\nsep x^3y - 2xy^2 = xy(x - 2y) \douhao $$
它们的公因式是 $x(x - 2y)$,把 $x(x - 2y)$ 约去,得
$$ \dfrac{x^3 - 2x^2y}{x^3y - 2xy^2} = \dfrac{x^2(x - 2y)}{xy(x - 2y)} = \dfrac{x}{y} \juhao $$

从上面的例子可以看出:把一个分式约分,如果分子、分母都是几个因式的积的形式,就约去分子、分母中相同因式的最低次幂,
当分子、分母的系数是整数时,还要约去它们的最大公约数;如果分子、分母是多项式,一般先进行因式分解,再约分。

\liti 约分:
\begin{xiaoxiaotis}

    \begin{tblr}{columns={18em, colsep=0pt}}
        \xxt{$\dfrac{-32a^2b^3c}{24b^2cd}$;} & \xxt{$\dfrac{-35(x - y)^2}{-45(x - y)}$。}
    \end{tblr}

\resetxxt
\jie \xxt{$\dfrac{-32a^2b^3c}{24b^2cd} = -\dfrac{8b^2c \cdot 4a^2b}{8b^2c \cdot 3d} = -\dfrac{4a^2b}{3d}$;}

\xxt{$\dfrac{-35(x - y)^2}{-45(x - y)} = \dfrac{5(x - y) \cdot 7(x - y)}{5(x - y) \cdot 9} = \dfrac{7(x - y)}{9}$。}

\end{xiaoxiaotis}

分子或分母的系数是负数时,一般先把负号提到分式本身的前边去。

\liti 约分:
\begin{xiaoxiaotis}

    \begin{tblr}{columns={18em, colsep=0pt}}
        \xxt{$\dfrac{m^2 - 3m}{9 - m^2}$;} & \xxt{$\dfrac{(2a - a^2)(a^2 + 4a + 3)}{(a^2 + a)(a^2 + a - 6)}$。}
    \end{tblr}

\resetxxt
\jie \begin{tblr}[t]{columns={colsep=0pt}, column{1}={16.2em}}
    \xxt{$\huitui\begin{aligned}[t]
                &\dfrac{m^2 - 3m}{9 - m^2} \\
            ={} & \dfrac{m(m - 3)}{-(m^2 - 9)} \\
            ={} & -\dfrac{m(m - 3)}{(m + 3)(m - 3)} \\
            ={} & -\dfrac{m}{m + 3} \fenhao
    \end{aligned}$} & \xxt{\huitui$\begin{aligned}[t]
            & \dfrac{(2a - a^2)(a^2 + 4a + 3)}{(a^2 + a)(a^2 + a - 6)} \\
        ={} & \dfrac{[a(-a + 2)][(a + 1)(a + 3)]}{[a(a + 1)][(a - 2)(a + 3)]} \\
        ={} & \dfrac{-a(a - 2)(a + 1)(a + 3)}{a(a + 1)(a - 2)(a + 3)} \\
        ={} & -1 \juhao
    \end{aligned}$}
\end{tblr}

\end{xiaoxiaotis}

为了便于看出分子和分母的公因式,可以把所有因式的各项都按照某一字母的降幂排列。
如果因式第一项的系数是负数,再把负号提到括号外面去,如上例中 $(-a + 2)$ 写成 $-(a - 2)$。

\lianxi
\begin{xiaotis}

\xiaoti{(口答)约分:}
\begin{xiaoxiaotis}

    \fourInLineXxt{$\dfrac{6xy}{6x^2}$;}
                  {$\dfrac{x^3}{x^3}$;}%分子的指数看不清楚,约是 3
                  {$\dfrac{-a^3}{a^4}$;}
                  {$\dfrac{a(a + b)}{b(a + b)}$。}

\end{xiaoxiaotis}

\xiaoti{约分:}
\begin{xiaoxiaotis}

    \begin{tblr}{columns={18em, colsep=0pt}}
        \xxt{$\dfrac{4a^2b}{6ab^2}$;} & \xxt{$\dfrac{-4m^3n^2}{2m^3n^6}$;} \\
        \xxt{$\dfrac{3a^2b(m - 1)}{9ab^2(1 - m)}$;} & \xxt{$\dfrac{12a^3(y - x)^2}{27a(x - y)}$。}
    \end{tblr}

\end{xiaoxiaotis}

\xiaoti{约分:}
\begin{xiaoxiaotis}

    \begin{tblr}{columns={18em, colsep=0pt}}
        \xxt{$\dfrac{x}{x^2 - x}$;} & \xxt{$\dfrac{x^2y + xy^2}{2xy}$;} \\
        \xxt{$\dfrac{a^2 + ab}{a^2 + 2ab + b^2}$;} & \xxt{$\dfrac{m^2 - 2m + 1}{1 - m^2}$;} \\
        \xxt{$\dfrac{9 - x^2}{x^2 + 5x + 6}$;} & \xxt{$\dfrac{y - 2 + y^2}{y^2 + 4y + 4}$。}
    \end{tblr}

\end{xiaoxiaotis}

\xiaoti{(口答)下列各式对不对?}
\begin{xiaoxiaotis}

    \begin{tblr}{columns={18em, colsep=0pt}}
        \xxt{$\dfrac{x^6}{x^2} = x^3$;} & \xxt{$\dfrac{a + x}{b + x} = \dfrac{a}{b}$;} \\
        \xxt{$\dfrac{x + y}{x + y} = 0$;} & \xxt{$\dfrac{a^2 + b^2}{a + b} = a + b$;} \\
        \xxt{$\dfrac{-x + y}{x - y} = -1$。}
    \end{tblr}

\end{xiaoxiaotis}

\end{xiaotis}

\end{enhancedline}

