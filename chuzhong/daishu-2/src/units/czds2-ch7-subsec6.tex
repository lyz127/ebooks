\begin{starred}
\subsection{十字相乘法\footnote{本节为选学内容。}}\label{subsec:7-6}
\end{starred}

我们知道
\begin{align*}
        & (a_1x + c_1)(a_2x + c_2) \\
    ={} & a_1a_2x^2 + a_1c_2x + a_2c_1x + c_1c_2 \\
    ={} & a_1a_2x^2 + (a_1c_2 + a_2c_1)x + c_1c_2 \juhao
\end{align*}
反过来,就得到
\begin{align*}
        & a_1a_2x^2 + (a_1c_2 + a_2c_1)x + c_1c_2 \\
    ={} & (a_1x + c_1)(a_2x + c_2) \juhao
\end{align*}

利用这个等式,我们可以用下面的写法,尝试把某些二次三项式如 $ax^2 + bx + c$ 分解因式。
先把 $a$ 分解成 $a = a_1a_2$,把 $c$ 分解成 $c = c_1c_2$,
并把 $a_1$,$a_2$,$c_1$,$c_2$ 排列如下:
\begin{center}
    \tikz \draw (0, 0) pic {cross={a_1}{a_2}{c_1}{c_2}};
\end{center}
这里按斜线交叉相乘的积的和就是 $a_1c_2 + a_2c_1$,如果它正好等于二次三项式 $ax^2 + bx + c$ 中一次项的系数 $b$,
那么 $ax^2 + bx + c$ 就可分解成 $(a_1x + c_1)(a_2x + c_2)$,
其中 $a_1$,$c_1$ 是上图中上面一行的两个数,$a_2$,$c_2$ 是下面一行的两个数。

例如,把二次三项式 $3x^2 + 11x + 10$ 分解因式。我们知道,$3 = 1 \times 3$,$10 = 2 \times 5$,写成
\begin{center}
    \tikz \draw (0, 0) pic {cross={1}{3}{2}{5}};
\end{center}
后,发现 $1 \times 5 + 2 \times 3$ 正好等于 11,所以
$$ 3x^2 + 11x + 10 = (x + 2)(3x + 5) \juhao $$

这种经过画十字交叉线的帮助把二次三项式分解因式的方法,叫做\zhongdian{十字相乘法}。

\zhuyi 因为分解因数及十字相乘都有多种可能情况,所以往往要经过多次尝试,才能确定一个二次三项式能否分解和怎样分解。
比方在上面的二次三项式 $3x^2 + 11x + 10$ 中, 二次项系数 3 可以分解成 1, 3, 或 $-1$,$-3$ 的积,
常数项 10 可以分解成 1,10,或 $-1$,$-10$, 或 2, 5, 或 $-2$, $-5$ 的积,其中十字相乘
\begin{center}
    \tikz \draw (0, 0) pic {cross={1}{3}{2}{5}};
\end{center}
能获得成功,而十字相乘如
\begin{center}
    \tikz \draw (0, 0) pic {cross={1}{3}{5}{2}};
\end{center}
等不能成功。所以用十字相乘法分解因式,往往要经过多次尝试。

\liti[0] 把下列各式分解因式:
\begin{xiaoxiaotis}

    \begin{tblr}{columns={18em, colsep=0pt}}
        \xxt{$2x^2 - 7x + 3$;} & \xxt{$6x^2 - 7x - 5$;} \\
        \xxt{$5x^2 + 6xy - 8y^2$。}
    \end{tblr}

\resetxxt
\jie \huitui\begin{tblr}[t]{columns={18em, colsep=0pt}}
        \xxt{\huitui$\begin{aligned}[t]
                & 2x^2 - 7x + 3 \\
            ={} & (x - 3)(2x - 1) \fenhao
        \end{aligned}$} & \tikz [overlay] \draw (0, -0.7) pic {cross={1}{2}{-3}{-1}}; \\
        \xxt{\huitui$\begin{aligned}[t]
                & 6x^2 - 7x - 5 \\
            ={} & (2x + 1)(3x - 5) \fenhao
        \end{aligned}$} & \tikz [overlay] \draw (0, -0.7) pic {cross={2}{3}{1}{-5}}; \\
        \xxt{\huitui$\begin{aligned}[t]
                & 5x^2 + 6xy - 8y^2 \\
            ={} & (x + 2y)(5x - 4y) \juhao
        \end{aligned}$} & \tikz [overlay] \draw (0, -0.7) pic {cross={1}{5}{2y}{-4y}};
    \end{tblr}

\end{xiaoxiaotis}

\lianxi
\begin{xiaotis}

把下列各式分解因式:

\xiaoti{}%
\begin{xiaoxiaotis}%
    \huitui\begin{tblr}[t]{columns={18em, colsep=0pt}}
        \xxt{$2x^2 + 15x + 7$;} & \xxt{$3a^2 - 8a + 4$;} \\
        \xxt{$8m^2 + 3m - 5$;} & \xxt{$2x^2 - 7x - 15$;} \\
        \xxt{$5x^2 + 7x - 6$;} & \xxt{$6y^2 - 11y - 10$。}
    \end{tblr}

\end{xiaoxiaotis}

\xiaoti{}%
\begin{xiaoxiaotis}%
    \huitui\begin{tblr}[t]{columns={18em, colsep=0pt}}
        \xxt{$6a^2 + 17ab + 12b^2$;} & \xxt{$15x^2 - xy - 6y^2$;} \\
        \xxt{$5a^2b^2 + 23ab - 10$;} & \xxt{$10x^2y^2 - 17abxy + 3a^2b^2$。}
    \end{tblr}

\end{xiaoxiaotis}

\end{xiaotis}


