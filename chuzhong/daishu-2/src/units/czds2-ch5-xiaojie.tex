\xiaojie

一、本章主要内容是二元一次方程组的解法和它的应用, 以及三元一次方程组的解法举例。

\jiange
二、解一次方程组可以通过逐步 “消元”, 变 “多元” 为 “一元” , 如:
\jiange 三元一次方程组 $\xrightarrow{\text{消元}}$ 二元一次方程组 $\xrightarrow{\text{消元}}$ 一元一次方程,
从而实现由 “未知” 到 “知” 的转化。


三、本章介绍了二元一次方程组的两种消元的方法:

(1) 代入法 \quad 把其中一个方程的某一个未知数用含另一个未知数的代数式表示, 然后代入另一个方程,就可以消去这个未知数。

(2) 加减法 \quad 先使两个方程中的某一个未知数的系数的绝对值相等,然后把方程的两边分别相加或相减,就可以消去这个未知数。

对于多元的一次方程组也可以用以上的方法逐步消元。

一般说来,当某个未知数的系数为 1 时,用代入法比较简便; 当两个方程中有一个未知数的系效的绝对值相等或成整数倍时,用加减法比较简便。

四、对于含有多个未知数的问题,利用方程组来解,在列方程时常常比列一元一次方程容易一些。
列方程时,一般地说,选定几个未知数,就要根据问题中的相等关系列出几个方程。
解由这些方程组成的方程组,求出未知数的值,并且根据问题的实际意义,检查求得的值是不是合理,即可得出问题的答案。

