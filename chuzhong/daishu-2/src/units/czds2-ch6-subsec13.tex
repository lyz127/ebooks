\subsection{多项式除以多项式}\label{subsec:6-13}

现在我们可以学习多项式除以多项式了。

两个多项式相除,可以先把这两个多项式都按照同一字母降幂排列,
然后再仿照两个多位数相除的演算方法,用竖式进行演算。例如,我们来计算
$$ (7x + 2 + 6x^2) \div (2x + 1) \douhao $$
仿照 $672 \div 21$,演算如下:

\begin{minipage}{4cm}
    $$
    \begin{array}{r@{\;}*{4}{r@{}}}
        & & & 3 & 2\\
        \cline{3-5}
        21 & \vstretch{1.3}{\big)}\mkern-7.3mu & \;\; 6 & 7 & 2 \\
        & & 6 & 3 \\
        \cline{3-5}
        & & & 4 & 2\\
        & & & 4 & 2 \\
        \cline{4-5}
        & & & & 0
    \end{array}
    $$
\end{minipage}
\begin{minipage}{9cm}
    $$
    \begin{array}{r@{\;}r@{}*{5}{r}l}
        & & & & 3x & + & 2 & \leftarrow \text{商式} \\
        \cline{3-7}
        \text{除式 } \rightarrow 2x + 1 & \vstretch{1.3}{\big)}\mkern-7.3mu & \;\; 6x^2 & + & 7x & + & 2 & \leftarrow \text{被除式} \\
        & & 6x^2 & + & 3x \\
        \cline{3-7}
        & & & & 4x & + & 2\\
        & & & & 4x & + & 2 \\
        \cline{5-7}
        & & & & & & 0 & \leftarrow \text{余式}
    \end{array}
    $$
\end{minipage}

\fengeSuoyi{(7x + 2 + 6x^2) \div (2x + 1) = 3x + 2 \juhao}

演算的步骤是:

1. 用除式的第一项 $2x$ 去除被除式的第一项 $6x^2$,得商式的第一项 $3x$;

2. 用商式的第一项 $3x$ 去乘除式,把积 $6x^2 + 3x$ 写在被除式下面(同类项对齐),
从被除式中减去这个积,得 $4x + 2$ ;

3. 把 $4x + 2$ 当作新的被除式,再按照上面的方法继续演算,直到余式是零
(或余式的次数低于除式的次数)为止。

一般多项式的除法也可按照上面演算步骤进行。

\liti 计算 $(5x^2 + 2x^3 - 1) \div (1 + 2x)$。

\jie
$$
\begin{array}{r@{\;}r@{}*{7}{r}}
    & & & & x^2 & + & 2x & - & 1 \\
    \cline{3-9}
    2x + 1 & \vstretch{1.3}{\big)}\mkern-7.3mu & \;\; 2x^3 & + & 5x^2 &  &  & - & 1 \\
    & & 2x^3 & + & x^2 \\
    \cline{3-9}
    & & & & 4x^2\\
    & & & & 4x^2 & + & 2x \\
    \cline{5-9}
    & & & & & - & 2x & - & 1 \\
    & & & & & - & 2x & - & 1 \\
    \cline{5-9}
    & & & & & & & & 0
\end{array}
$$

$\therefore \quad (5x^2 + 2x^3 - 1) \div (1 + 2x) = x^2 + 2x - 1$。

\zhuyi 按照 $x$ 降幂排列,如果被除式有缺项,要留出空位。也可以采用加零的办法补足缺项,
例如,把 $2x^3 + 5x^2 - 1$ 写成 $2x^3 + 5x^2 + 0 - 1$。

例 1 的余式为零。 如果一个多项式除以另一个多项式的余式为零,
我们就说这个多项式能被另一个多项式\zhongdian{整除}, 这时也可说除式能整除被除式。

整式除法也有不能整除的情况。按照某个字母降幂排列的整式除法,当余式不是零而次数低于除式的次数时,
除法演算就不能继续进行了,这说明除式不能整除被除式。

\liti 计算 $(2x^3 + 9x^2 + 3x + 5) \div (x^2 + 4x - 3)$。

\jie
$$
\begin{array}{r@{\;}r@{}*{7}{r}}
    & & & & & & 2x & + & 1 \\
    \cline{3-9}
    x^2 + 4x - 3 & \vstretch{1.3}{\big)}\mkern-7.3mu & \;\; 2x^3 & + & 9x^2 & + & 3x & + & 5 \\
    & & 2x^3 & + & 8x^2 & - & 6x \\
    \cline{3-9}
    & & & & x^2 & + & 9x & + & 5 \\
    & & & & x^2 & + & 4x & - & 5 \\
    \cline{5-9}
    & & & & & & 5x & + & 8
\end{array}
$$

$\therefore \quad \text{商式} = 2x + 1 \douhao \qquad \text{余式} = 5x + 8$。

我们知道,整数相除,有时不能整除,带有余数,例如,

$$
\begin{array}{r@{\;}*{4}{r@{}}}
    & & & 3 & 7\\
    \cline{3-5}
    21 & \vstretch{1.3}{\big)}\mkern-7.3mu & \; 7 & 8 & 5 \\
    & & 6 & 3 \\
    \cline{3-5}
    & & 1 & 5 & 5 \\
    & & 1 & 4 & 7 \\
    \cline{3-5}
    & & & & 8
\end{array}
$$

在数的带余除法中,有下面的关系:
\begin{center}
\begin{tblr}{columns={colsep=0pt},
    row{1}={$, halign=c},
}
    785 & = & 21 & \times & 37 & + & 8 & \juhao \\[1em]
    \tikz [overlay, >=Stealth] {
        \draw [<->] (1.7em, .8em) -- (1.7em, 2.5em);
        \draw [<->] (5.0em, .8em) -- (5.0em, 2.5em);
        \draw [<->] (8.0em, .8em) -- (8.0em, 2.5em);
        \draw [<->] (10.5em, .8em) -- (10.5em, 2.5em);
    }
    被除数 & & 除数 & & 商数 & & 余数
\end{tblr}
\end{center}

与数的带余除法类似,在上面的例 2 中,也有下面的关系:
\begin{center}
    \begin{tblr}{columns={colsep=0pt},
        row{1}={$},
        row{2}={halign=c},
    }
        (2x^3 + 9x^2 + 3x + 5) & = & (x^2 + 4x - 3) & (2x + 1) & + & (5x + 8) & \juhao \\[1em]
        \tikz [overlay, >=Stealth] {
            \draw [<->] (1.7em, .8em) -- (1.7em, 2.5em);
            \draw [<->] (10.0em, .8em) -- (10.0em, 2.5em);
            \draw [<->] (14.5em, .8em) -- (14.5em, 2.5em);
            \draw [<->] (19.0em, .8em) -- (19.0em, 2.5em);
        }
        被除式 & & 除式 & 商式 & & 余式
    \end{tblr}
\end{center}

一般地,被除式、除式、商式及余式之间有下面的关系:

\begin{center}
    \framebox{\quad \zhongdian{$\bm{\text{被除式} = \text{除式} \times \text{商式} + \text{余式}}$。}\;}
\end{center}


\lianxi
\begin{xiaotis}

\xiaoti{计算:}
\begin{xiaoxiaotis}

    \begin{tblr}{columns={18em, colsep=0pt}, column{2}={20em}}
        \xxt{$(3x^2 + 14x - 5) \div (x + 5)$;} & \xxt{$(6x^2 + 14x + 4) \div (3x + 1)$;} \\
        \xxt{$(8x^2 + 53x - 21) \div (x + 7)$;} & \xxt{$(1 + x^2 + x^4) \div (x^2 + 1 - x)$;} \\
        \xxt{$(x^3 - 3x + x^2 - 8) \div (x - 2)$;} & \xxt{$(3x^3 - 4x - 5x^2 + 3) \div (x^2 - x + 5)$。}
    \end{tblr}

\end{xiaoxiaotis}

\xiaoti{已知除式、商式及余式,求被除式:}
\begin{xiaoxiaotis}

    \xxt{$\text{除式} = 3x - 5$, $\text{商式} = 2x + 7$, $\text{余式} = 10$;}

    \xxt{$\text{除式} = x^2 - 2x + 1$, $\text{商式} = x^2 + 2x - 1$, $\text{余式} = 4x$。}
\end{xiaoxiaotis}

\end{xiaotis}

