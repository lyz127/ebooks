\subsection{分式的基本性质}\label{subsec:8-2}
\begin{enhancedline}

我们已经知道,分数的基本性质是:分数的分子与分母都乘以(或除以)同一个不等于零的数,分数的值不变。例如,
$$ \dfrac{3}{5} = \dfrac{3 \times 4}{5 \times 4}\nsep \dfrac{4}{18} = \dfrac{4}{18} = \dfrac{4 \div 2}{18 \div 2}\juhao $$

分式也有类似的性质,就是:

\zhongdian{分式的分子与分母都乘以(或除以)同一个不等于零的整式,分式的值不变。}
这个性质叫做\zhongdian{分式的基本性质}, 用式子表示是:
\begin{center}
    \setlength{\fboxsep}{.6em}
    \framebox{\quad $\begin{aligned}[t]
        & \dfrac{A}{B} = \dfrac{A \times M}{B \times M} \douhao \dfrac{A}{B} = \dfrac{A \div M}{B \div M} \\
        & \text{(其中 $M$ 是不等于零的整式)。}
    \end{aligned}$\;}
\end{center}

\liti 下列等式的右边是怎样从左边得到的?
\begin{xiaoxiaotis}

    \begin{tblr}{columns={18em, colsep=0pt}}
        \xxt{$\dfrac{a}{2b} = \dfrac{ac}{2bc} \; (c \neq 0)$;} & \xxt{$\dfrac{x^3}{xy} = \dfrac{x^2}{y}$。}
    \end{tblr}

\resetxxt
\jie  \begin{tblr}[t]{columns={18em, colsep=0pt}}
    \xxt{$\begin{aligned}[t]
        &\because \quad c \neq 0 \douhao \\
        & \therefore \quad \dfrac{a}{2b} = \dfrac{a \cdot c}{2b \cdot c} = \dfrac{ac}{2bc} \fenhao
    \end{aligned}$} & \xxt{$\begin{aligned}[t]
        &\because \quad x \neq 0 \douhao \\
        &\therefore \quad \dfrac{x^3}{xy} = \dfrac{x^3 \div x}{xy \div x} = \dfrac{x^2}{y} \juhao
    \end{aligned}$}
\end{tblr}

\end{xiaoxiaotis}

\liti 分别写出下列等式中未知的分子或分母:
\begin{xiaoxiaotis}

    \begin{tblr}{columns={18em, colsep=0pt}}
        \xxt{$\dfrac{a + b}{ab} = \dfrac{?}{a^2b}$;} & \xxt{$\dfrac{x^2 + xy}{x^2} = \dfrac{x + y}{?}$。}
    \end{tblr}

\resetxxt
\jie \xxt{右边的分母 $a^2b$ 等于左边的分母 $ab$ 乘以 $a$,所以
    $$\dfrac{a + b}{ab} = \dfrac{(a + b) \cdot a}{(ab) \cdot a} = \dfrac{a^2 + ab}{a^2b} \douhao $$
    即所求的分子是 $a^2 + ab$;
}

\xxt{右边的分子 $(x + y)$ 等于左边的分子 $x(x + y)$ 除以 $x$,所以
    $$ \dfrac{x^2 + xy}{x^2} = \dfrac{x(x + y) \div x}{x^2 \div x} = \dfrac{x + y}{x} \douhao $$
    即所求的分母是 $x$。
}

\end{xiaoxiaotis}


\liti 不改变分式的值,把下列各式的分子与分母中各项的系数都化为整数:
\begin{xiaoxiaotis}

    \begin{tblr}{columns={18em, colsep=0pt}}
        \xxt{$\dfrac{\dfrac{1}{2}x + \dfrac{1}{3}y}{\dfrac{1}{2}x - \dfrac{1}{3}y}$;}
            & \xxt{$\dfrac{0.3a + 0.7b}{0.2a - b}$。}
    \end{tblr}

\resetxxt
\jie \xxt{$\dfrac{\dfrac{1}{2}x + \dfrac{1}{3}y}{\dfrac{1}{2}x - \dfrac{1}{3}y} = \dfrac{\left(\dfrac{1}{2}x + \dfrac{1}{3}y\right) \times 6}{\left(\dfrac{1}{2}x - \dfrac{1}{3}y\right) \times 6} = \dfrac{3x + 2y}{3x - 2y}$;}

\xxt{$\dfrac{0.3a + 0.7b}{0.2a - b} = \dfrac{(0.3a + 0.7b) \times 10}{(0.2a - b) \times 10} = \dfrac{3a + 7b}{2a - 10b}$。}

\end{xiaoxiaotis}

\lianxi
\begin{xiaotis}

\xiaoti{下列等式的右边是怎样从左边得到的?}
\begin{xiaoxiaotis}

    \begin{tblr}{columns={18em, colsep=0pt}}
        \xxt{$\dfrac{1}{ab} = \dfrac{c}{abc} \; (c \neq 0)$;} & \xxt{$\dfrac{a^2x}{bx} = \dfrac{a^2}{b}$;} \\
        \xxt{$\dfrac{1}{x - 1} = \dfrac{x + 1}{x^2 - 1} \; (x + 1 \neq 0)$;} & \xxt{$\dfrac{(x - y)^2}{x^2 - y^2} = \dfrac{x - y}{x + y}$。}
    \end{tblr}

\end{xiaoxiaotis}

\xiaoti{写出下列等式中未知的分子或分母:}
\begin{xiaoxiaotis}

    \begin{tblr}{columns={18em, colsep=0pt}}
        \xxt{$\dfrac{y}{x} = \dfrac{?}{x^2}$;} & \xxt{$\dfrac{ab}{a^2} = \dfrac{b}{?}$;} \\
        \xxt{$\dfrac{1}{xy} = \dfrac{?}{2xy^2}$;} & \xxt{$\dfrac{a^2 + a}{ac} = \dfrac{?}{c}$。}
    \end{tblr}

\end{xiaoxiaotis}

\xiaoti{不改变分式的值,把下列各式的分子与分母中各项的系数都化为整数:}
\begin{xiaoxiaotis}

    \begin{tblr}{columns={18em, colsep=0pt}}
        \xxt{$\dfrac{0.5x - 0.7y}{0.3x + 0.2y}$;} & \xxt{$\dfrac{a + \dfrac{1}{4}b}{\dfrac{3}{4}a - 2b}$。}
    \end{tblr}

\end{xiaoxiaotis}

\end{xiaotis}
\lianxijiange

根据分式的基本性质,可以得到:
$$ \dfrac{-a}{-b} = \dfrac{a}{b} \nsep \dfrac{-a}{b} = \dfrac{a}{-b} \juhao $$

这就是说,\zhongdian{分子与分母同时改变符号,分式的值不变。}

根据有理数除法法则,我们知道:
$$ \dfrac{-2}{3} = -\dfrac{2}{3} \nsep \dfrac{2}{-3} = -\dfrac{2}{3} \juhao $$

分式也有类似的法则:
$$ \dfrac{-a}{b} = -\dfrac{a}{b} \nsep \dfrac{a}{-b} = -\dfrac{a}{b} \juhao $$

这就是说,\zhongdian{只改变分子(或分母)的符号,分式本身的符号也要改变,分式的值才不变。}

把上面两条符号法则,概括起来就是:

\zhongdian{分子、分母与分式本身的符号,改变其中任何两个,分式的值不变。}


\liti 不改变分式的值,使下列分式的分子与分母都不含 “$-$” 号:
\begin{xiaoxiaotis}

    \threeInLine{\xxt{$\dfrac{-5b}{-6a}$;}}
                {\xxt{$\dfrac{-x}{3y}$;}}
                {\xxt{$\dfrac{2m}{-n}$。}}

\resetxxt
\jie \begin{tblr}[t]{columns={10em, colsep=0pt}}
    \xxt{$\dfrac{-5b}{-6a} = \dfrac{5b}{6a}$;}
        & \xxt{$\dfrac{-x}{3y} = -\dfrac{x}{3y}$;}
        & \xxt{$\dfrac{2m}{-n} = -\dfrac{2m}{n}$。} \\
\end{tblr}

% \jie \xxt{$\dfrac{-5b}{-6a} = \dfrac{5b}{6a}$;}

% \xxt{$\dfrac{-x}{3y} = -\dfrac{x}{3y}$;}

% \xxt{$\dfrac{2m}{-n} = -\dfrac{2m}{n}$。}

\end{xiaoxiaotis}


\liti 不改变分式的值,使下列分式的分子与分母的最高次项的系数是正数:
\begin{xiaoxiaotis}

    \threeInLine{\xxt{$\dfrac{x}{1 - x^2}$;}}
                {\xxt{$\dfrac{-a - 1}{a^2 - 2}$;}}
                {\xxt{$\dfrac{2 - x}{-x^2 + 3}$。}}

\resetxxt
\jie \xxt{$\dfrac{x}{1 - x^2} = \dfrac{x}{-(x^2 - 1)} = -\dfrac{x}{x^2 - 1}$;}

\xxt{$\dfrac{-a - 1}{a^2 - 2} = \dfrac{-(a + 1)}{a^2 - 2} = -\dfrac{a + 1}{a^2 - 2}$;}

\xxt{$\dfrac{2 - x}{-x^2 + 3} = \dfrac{-(x - 2)}{-(x^2 - 3)} = \dfrac{x - 2}{x^2 - 3}$。}

\end{xiaoxiaotis}


\lianxi
\begin{xiaotis}

\xiaoti{不改变分式的值,使下列分式的分子与分母都不含 “$-$” 号:}
\begin{xiaoxiaotis}

    \begin{tblr}{columns={10em, colsep=0pt}}
        \xxt{$\dfrac{-a}{-2b}$;} & \xxt{$\dfrac{-2x}{3y}$;} & \xxt{$\dfrac{3m}{-4n}$;} \\
        \xxt{$-\dfrac{-x}{2y}$;} & \xxt{$-\dfrac{4m}{-5n}$。}
    \end{tblr}

\end{xiaoxiaotis}


\xiaoti{不改变分式的值,使下列分式的分子与分母的最高次项的系数是正数:}
\begin{xiaoxiaotis}

    \threeInLine{\xxt{$\dfrac{1 - a - a^2}{1 + a^2 - a^3}$;}}
                {\xxt{$\dfrac{x - 1}{1 - x^2}$;}}
                {\xxt{$-\dfrac{1 - a^3}{a^2 - a + 1}$。}}

\end{xiaoxiaotis}

\xiaoti{下列各式对不对?如果不对,应怎样改正?}
\begin{xiaoxiaotis}

    \xxt{$\dfrac{-(a - b)}{c} = - \dfrac{a - b}{c}$;}

    \xxt{$\dfrac{-a - b}{c} = - \dfrac{a - b}{c}$;}

    \xxt{$\dfrac{-a + b}{c} = -\dfrac{a + b}{c}$。}

\end{xiaoxiaotis}

\end{xiaotis}

\end{enhancedline}

