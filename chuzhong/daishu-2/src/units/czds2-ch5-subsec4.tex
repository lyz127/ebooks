\subsection{用加减法解二元一次方程组}\label{subsec:5-4}

我们再来学习另一种通过消去未知数来解一次方程组的方法。 例如, 解方程组
\begin{numcases}{}
    x + y = 5 \douhao \tag{1} \\
    x - y = 1 \juhao  \tag{2}
\end{numcases}

在这个方程组的两个方程中, 未知数 $y$ 的系数互为相反数, 如果把这两个方程的两边分别相加,
就可以消去 $y$, 得到一个一元一次方程。
\begin{gather*}
    x \quad + y
        \tikz [overlay] \draw [thick] (-2em, 1.5em) rectangle (.3em, -2.8em);
        \; = 5 \douhao \tag{1} \\
    x \quad - y \; = 1 \juhao  \tag{2}
\end{gather*}

$(1) + (2)$,得
\begin{gather*}
    2x = 6 \juhao \tag{3}
\end{gather*}

由 (3),得 $x = 3$。把 $x = 3$ 代入 (1) 或 (2),得 $y = 2$。经过检验,
$$\begin{cases}
    x = 3 \douhao \\
    y = 2
\end{cases}$$
是原方程组的解。

在上面的方程组的两个方程中, 我们又看到, 未知数 $x$ 的系数相等。
如果把这两个方程的两边分别相减,就可以消去未知数 $x$, 也能得到一个一元一次方程。
\begin{gather*}
    x
        \tikz [overlay] \draw [thick] (-1em, 1.3em) rectangle (.3em, -2.5em);
        \hspace*{.5em} + y = 5 \douhao \tag{1} \\
    x \hspace*{.5em} - y = 1 \juhao  \tag{2}
\end{gather*}

$(1) - (2)$,得
\begin{gather*}
    2y = 4 \juhao \tag{4}
\end{gather*}

由 (4), 得 $y = 2$ . 把 $y = 2$ 代入 (1) 或 (2), 得 $x = 3$。


我们再看几个例子。

\liti 解方程组
\begin{numcases}{}
    3x + 7y = -20 \douhao \tag{1} \\
    3x - 5y = 16  \juhao  \tag{2}
\end{numcases}

分析: 在这两个方程中, 未知数 $x$ 的系数相等,把方程 (1), (2) 的两边分别相减, 就可以消去 $x$ 。

\jie $(1) - (2)$,得
$$ 12y = -36 \douhao $$

\fenge{$\therefore$}{$$ y = -3 \juhao $$}

把 $y = -3$ 代入 (1),得
$$ 3x + 7 \times (-3) = -20 \juhao $$

\begin{enhancedline}
\fenge{$\therefore$}{$$ x = \dfrac{1}{3} \juhao $$}

\fenge{$\therefore$}{
    $$\begin{cases}
        x = \dfrac{1}{3} \douhao \\
        y = -3 \juhao
    \end{cases}$$
}
\end{enhancedline}

\liti 解方程组
\begin{numcases}{}
    9u + 2v = 15 \douhao \tag{1} \\
    3u + 4v = 10 \juhao  \tag{2}
\end{numcases}

分析: 在这两个方程中, 同一个未知数的系数的绝对值都不相等,
如果直接把这两个方程的两边分别相加或相减, 都不能消去任何一个未知数。
但是只要在方程 (1) 的两边都乘以 2 , 就可以使两个方程中的未知数 $v$ 的系数相等,
然后再把两个方程的两边分别相减而消去 $v$。

\jie $(1) \times 2$,得
\begin{gather*}
    18u + 4v = 30 \juhao \tag{3}
\end{gather*}

$(3) - (2)$,得
$$ 15u = 20 \douhao $$

\begin{enhancedline}
\fenge{$\therefore$}{$$ u = 1\dfrac{1}{3} \juhao $$}

把 $u = 1\dfrac{1}{3}$ 代入 (2),得
\begin{align*}
    3 \times 1\dfrac{1}{3} + 4v &= 10,  \hspace*{4em} \\
    4v &= 6,
\end{align*}

\fenge{$\therefore$}{$$ v = 1\dfrac{1}{2} \juhao $$}

\fenge{$\therefore$}{
    $$\begin{cases}
        u = 1\dfrac{1}{3} \douhao \\[1em]
        v = 1\dfrac{1}{2} \juhao
    \end{cases}$$
}

想一想: 能不能先消去未知数 $u$,如果能, 应当怎样做。
\end{enhancedline}


\liti 解方程组
\begin{numcases}{}
    3x + 4y = 16 \douhao \tag{1} \\
    5x - 6y = 33 \juhao  \tag{2}
\end{numcases}

分析: 在方程 (1) 的两边都乘以 3, 方程 (2)  的两边都乘以 2 ,
就可使未知数 $y$ 的系数的绝对值相等, 然后把两个方程的两边分别相加而消去 $y$。

\jie $(1) \times 3$,得
\begin{gather*}
    9x + 12y = 48 \juhao \tag{3}
\end{gather*}

$(2) \times 2$,得
\begin{gather*}
    10x - 12y = 66 \juhao \tag{4}
\end{gather*}

$(3) + (4)$,得
$$ 19 = 114 \douhao $$

\fenge{$\therefore$}{$$ x = 6 \juhao $$}

把 $x = 6$ 代入 (1),得
\begin{align*}
    3 \times 6 + 4y = 16, \\
    4y = -2,
\end{align*}

\fenge{$\therefore$}{$$ y = -\dfrac{1}{2} \juhao $$}

\fenge{$\therefore$}{
    $$\begin{cases}
        x = 6 \douhao \\
        y = -\dfrac{1}{2} \juhao
    \end{cases}$$
}


\jiange
解上面几个例题时, 我们在方程组里一个方程的两边都乘以一个适当的数,
或者分别在两个方程的两边都乘以一个适当的数, 使其中某一个未知数的系数的绝对值相等,
然后把方程两边分别相加或相减, 消去这个未知数, 使解二元一次方程组转化为解一元一次方程。
这种解方程组的方法叫做\zhongdian{加减消元法}, 简称\zhongdian{加减法}。
用加减法解题时, 一般可以先把方程组里的每个方程整理成含未知数的项在方程左边、常数项在方程右边的形式,
然后再进行加减消元。


\liti 解方程组
\begin{numcases}{}
    2(x - 150) = 5(3y + 50) \douhao \tag{1} \\
    10\% \cdot x + 6\% \cdot y = 8.5\% \times 800 \juhao  \tag{2}
\end{numcases}

\jie 把方程 (1),(2) 分别化简,得
\begin{numcases}{}
    2x - 15y = 550 \douhao \tag{3} \\
    5x + 3y = 3400 \juhao  \tag{4}
\end{numcases}

$(3) + (4) \times 5$,得
$$ 27x = 17550 \douhao $$

\fenge{$\therefore$}{$$ x = 650 \juhao $$}

把 $x =650$ 代入 (4),得
\begin{gather*}
    5 \times 650 + 3y = 3400 \douhao  \\
    3y = 150 \douhao
\end{gather*}

\fenge{$\therefore$}{$$ y = 50 \juhao $$}

\fenge{$\therefore$}{
    $$\begin{cases}
        x = 650 \douhao \\
        y = 50 \juhao
    \end{cases}$$
}


\lianxi

用加减法解下列方程组:
\begin{xiaotis}

% 另外一种拼凑的写法:
% \vspace*{1.5em}
% \xiaoti{}
% \begin{xiaoxiaotis}

%     \vspace*{-2.5em}
%     \begin{tblr}{columns={18em, colsep=0pt}}
%         \xxt{$\begin{cases} 3x + y = 8, \\ 2x - y = 7; \end{cases}$}
%             & \xxt{$\begin{cases} 3m + 2n = 16, \\ 3m - n = 1; \end{cases}$} \\
%         \xxt{$\begin{cases} 3p + 7q = 9, \\ 4p - 7q = 5; \end{cases}$}
%             & \xxt{$\begin{cases} x + 2z = 9, \\ 3x -z = -1; \end{cases}$} \\
%         \xxt{$\begin{cases} 5x + 2y = 25, \\ 3x + 4y = 15; \end{cases}$}
%             & \xxt{$\begin{cases} 3x - 7y = 1, \\ 5x - 4y = 17; \end{cases}$} \\
%         \xxt{$\begin{cases} 8s + 9t = 23, \\ 17s - 6t = 74; \end{cases}$}
%             & \xxt{$\begin{cases} 4x - 15y - 17 = 0, \\ 6x - 25y - 23 = 0 \juhao \end{cases}$} \\
%     \end{tblr}

% \end{xiaoxiaotis}

\xiaoti{}%
\begin{xiaoxiaotis}%
    \twoInLine[18em]{\xxt[\xxtsep]{$\begin{cases} 3x + y = 8, \\ 2x - y = 7; \end{cases}$}}{\xxt[\xxtsep]{$\begin{cases} 3m + 2n = 16, \\ 3m - n = 1; \end{cases}$}}

    \begin{tblr}{columns={18em, colsep=0pt}}
        \xxt{$\begin{cases} 3p + 7q = 9, \\ 4p - 7q = 5; \end{cases}$}
            & \xxt{$\begin{cases} x + 2z = 9, \\ 3x -z = -1; \end{cases}$} \\
        \xxt{$\begin{cases} 5x + 2y = 25, \\ 3x + 4y = 15; \end{cases}$}
            & \xxt{$\begin{cases} 3x - 7y = 1, \\ 5x - 4y = 17; \end{cases}$} \\
        \xxt{$\begin{cases} 8s + 9t = 23, \\ 17s - 6t = 74; \end{cases}$}
            & \xxt{$\begin{cases} 4x - 15y - 17 = 0, \\ 6x - 25y - 23 = 0 \juhao \end{cases}$} \\
    \end{tblr}

\end{xiaoxiaotis}

\xiaoti{}%
\begin{xiaoxiaotis}%
    \xxt[\xxtsep]{$\begin{cases} \dfrac{x}{3} + \dfrac{y}{5} = 1, \\ 3(x + y) + 2(x - 3y) = 15; \end{cases}$}

    \xxt{$\begin{cases} x + y = 60, \\  30\% \cdot x + 6\% \cdot y = 10\% \times 60 \juhao \end{cases}$}

\end{xiaoxiaotis}

\end{xiaotis}
