\subsection{可化为$x^2 + (a + b)x + ab$型的二次三项式的因式分解}\label{subsec:7-4}

我们知道,
$$ (x + a)(x + b) = x^2 + (a + b)x + ab \douhao $$
反过来,就得到
\begin{center}
    \setlength{\fboxsep}{.6em}
    \framebox{\quad $x^2 + (a + b)x + ab = (x + a)(x + b)$。\;}
\end{center}

这就是说,对于二次三项式 $x^2 + px + q$,如果能够把常数项 $q$ 分解成两个因数 $a$,$b$ 的积,
并使 $a + b = p$,那么它就可以分解因式,即
$$ x^2 + px + q = x^2 + (a + b)x + ab = (x + a)(x + b) \juhao $$
运用这个公式,可以把某些二次项系数为 1 的二次三项式分解因式。

例如,把二次三项式 $x^2 + 5x + 6$ 分解因式。我们设法把常数项 6 分解成两个因数的积,
使这两个因数的和等于一次项的系数 5。因为 $6 = 2 \times 3$, 并且 $2 + 3 = 5$,所以
$$ x^2 + 5x + 6 = (x + 2)(x + 3) \juhao $$

\liti 把 $x^2 + 3x + 2$ 分解因式。

分析:常数项 2 可以分解成 1,2 或 $-1$,$-2$ 的积,其中只有 $1 + 2$ 等于一次项的系数 3 。

\jie 因为 $2 = 1 \times 2$,并且 $1 + 2 = 3$,所以
$$ x^2 + 3x + 2 = (x + 1)(x + 2) \juhao $$

\liti 把 $x^2 - 7x + 6$ 分解因式。

分析:常数项 6 可以分解成 1,6,或 $-1$,$-6$ 或 2,3 或 $-2$,$-3$ 的积,
其中只有 $(-1) + (-6)$ 等于一次项的系数 $-7$。

\jie 因为 $6 = (-1) \times (-6)$,并且 $(-1) + (-6) = -7$,所以
\begin{align*}
    x^2 - 7x + 6 &= [x + (-1)][x + (-6)] \\
                 &= (x - 1)(x - 6) \juhao
\end{align*}

\liti 把 $x^2 - 4x - 21$ 分解因式。

分析:常数项 $-21$ 可以分解成 $1$,$-21$,或 $-1$,$21$ 或 $3$,$-7$ 或 $-3$,$7$ 的积,
其中只有 $3 + (-7)$ 等于一次项的系数 $-4$。

\jie 因为 $-21 = 3 \times (-7)$,并且 $3 + (-7) = -4$,所以
$$ x^2 - 4x - 21 = (x + 3)[x + (-7)] = (x + 3)(x - 7) \juhao $$


\liti 把 $x^2 + 2x - 15$ 分解因式。

\jie 因为 $-15 = (-3) \times 5$,并且 $(-3) + 5 = 2$,所以
$$ x^2 + 2x - 15 = [x + (-3)](x + 5) = (x - 3)(x + 5) \juhao $$

通过例 1~4 可以看出:
常数项是正数时,应分解成两个同号因数,它们的符号与一次项系数的符号相同;
常数项是负数时,应分解成两个异号因数,其中绝对值较大的因数与一次项系数的符号相同。

\liti 把下列各式分解因式:
\begin{xiaoxiaotis}

    \xxt{$x^4 + 6x^2 + 8$;}

    \xxt{$(a + b)^2 - 4(a + b) + 3$。}

\resetxxt
\jie \begin{tblr}[t]{columns={18em, colsep=0pt}}
    \xxt{\huitui$\begin{aligned}[t]
        & x^4 + 6x^2 + 8 \\
        ={} & (x^2)^2 + 6(x^2) + 8 \\
        ={} & [(x^2) + 2][(x^2) + 4] \\
        ={} & (x^2 + 2)(x^2 + 4) \fenhao
    \end{aligned}$} & \xxt{\huitui$\begin{aligned}[t]
            & (a + b)^2 - 4(a + b) + 3 \\
        ={} & [(a + b) - 1][(a + b) - 3] \\
        ={} & (a + b - 1)(a + b - 3) \juhao
    \end{aligned}$}
\end{tblr}

\end{xiaoxiaotis}

\liti 把 $x^2 - 3xy + 2y^2$ 分解因式。

分析:把 $x^2 - 3xy + 2y^2$ 看成 $x$ 的二次三项式,这时,常数项是 $2y^2$,一次项系数是 $-3y$。
把 $2y^2$ 分解成 $-y$ 与 $-2y$ 的积,$(-y) + (-2y) = -3y$,恰好等于一次项系数。

\jie $\begin{aligned}[t]
        & x^2 - 3xy + 2y^2 \\
    ={} & x^2 - 3yx + 2y^2 \\
    ={} & (x - y)(x - 2y) \juhao
\end{aligned}$

\liti 把 $x^4 - 3x^3 - 28x^2$ 分解因式。

\jie $\begin{aligned}[t]
        & x^4 - 3x^3 - 28x^2 \\
    ={} & x^2(x^2 - 3x - 28) \\
    ={} & x^2(x + 4)(x - 7) \juhao
\end{aligned}$

\lianxi
\begin{xiaotis}

\xiaoti{(口答)把下列各数分解成两个因数的积(要把所有可能的情况都列举出来):}
\begin{xiaoxiaotis}

    \fourInLineXxt{$3$;}{$-5$;}{$12$;}{$-8$。}

\end{xiaoxiaotis}

把下列各式分解因式(第 2~5 题):

\xiaoti{}%
\begin{xiaoxiaotis}%
    \huitui\begin{tblr}[t]{columns={18em, colsep=0pt}}
        \xxt{$x^2 + 4x + 3$;}  & \xxt{$a^2 + 7a + 10$;} \\
        \xxt{$y^2 - 7y + 12$;} & \xxt{$q^2 - 6q + 8$;} \\
        \xxt{$x^2 + x - 20$;}  & \xxt{$m^2 + 7m - 18$;} \\
        \xxt{$p^2 - 5p - 36$;} & \xxt{$t^2 - 2t - 8$。}
    \end{tblr}

\end{xiaoxiaotis}

\xiaoti{}%
\begin{xiaoxiaotis}%
    \huitui\begin{tblr}[t]{columns={18em, colsep=0pt}}
        \xxt{$x^4 - x^2 - 20$;} & \xxt{$a^2x^2 + 7ax - 8$。}
    \end{tblr}

\end{xiaoxiaotis}

\xiaoti{}%
\begin{xiaoxiaotis}%
    \huitui\begin{tblr}[t]{columns={18em, colsep=0pt}}
        \xxt{$a^2 - 9ab + 14b^2$;} & \xxt{$x^2 + 11xy + 18y^2$。}
    \end{tblr}

\end{xiaoxiaotis}

\xiaoti{}%
\begin{xiaoxiaotis}%
    \huitui\begin{tblr}[t]{columns={18em, colsep=0pt}}
        \xxt{$x^2y^2 - 5x^2y - 6x^2$;} & \xxt{$-a^3 - 4a^2 + 12a$。}
    \end{tblr}

\end{xiaoxiaotis}

\end{xiaotis}

