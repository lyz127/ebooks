\subsection{可化为一元一次方程的分式方程}\label{subsec:8-11}
\begin{enhancedline}

我们看下面的问题:把 $\dfrac{1}{5}$ 的分子与分母都加上同一个什么数,能使分数的值变为 $\dfrac{1}{2}$?

设所求的数是 $x$, 那么根据题意可以列出方程
$$ \dfrac{1 + x}{5 + x} = \dfrac{1}{2} \juhao $$

象这样,分母里含有未知数的方程叫做\zhongdian{分式方程}。
以前学过的,分母里不含有未知数的方程叫做\zhongdian{整式方程}。
一元一次方程是最简单的整式方程。

怎样解分式方程呢?如果能把分式方程的分母去掉,使分式方程化成整式方程,就可以利用整式方程的解法求解了。

例如,在分式方程
$$ \dfrac{1 + x}{5 + x} = \dfrac{1}{2} $$
的两边都乘以最简公分母 $2(5 + x)$, 得
$$ 2(5 + x) \cdot \dfrac{1 + x}{5 + x} = \dfrac{1}{2} \cdot 2(5 + x) \juhao $$

约去分母,就得到整式方程
$$ 2(1 + x) = 5 + x \juhao $$

解这个整式方程,得
$$ x = 3 \juhao $$

$x = 3$ 是不是原来分式方程的根呢?把 $x = 3$ 代入原方程检验:
$$ \zuobian = \dfrac{1 + 3}{5 + 3} = \dfrac{4}{8} = \dfrac{1}{2} \nsep  \youbian = \dfrac{1}{2} \juhao $$

左右两边相等,说明 $x = 3$ 是原分式方程的根。

再看另一个分式方程
$$ \dfrac{2}{x + 1} + \dfrac{3}{x - 1} = \dfrac{6}{x^2 - 1} \juhao $$

在方程的两边都乘以最简公分母 $(x + 1)(x - 1)$,得整式方程
$$ 2(x - 1) + 3(x + 1) = 6 \juhao $$

解这个整式方程,得 $x = 1$。

把 $x = 1$ 代入原分式方程检验,结果 $x = 1$ 使分式 $\dfrac{3}{x - 1}$ 和
$\dfrac{6}{x^2 - 1}$ 的分母的值为零,这两个分式没有意义。因此,1 不是原分式方程的根。

实际上,原分式方程无解。

从上面的两个例子可以看出,为了解分式方程,就要在方程两边都乘以同一个含有未知数的整式
(各分式的最简公分母),把分式方程化为整式方程。
这样得到的整式方程有时与原分式方程是同解方程,如第一个例子;
有时与原分式方程不是同解方程,如第二个例子,变形后得到的整式方程产生了一个不适合原分式方程的根。

上面的现象是怎样产生的呢?方程同解原理 2 指出:方程的两边都乘以不等于零的同一个数,所得方程与原方程同解。
在前面解第一个分式方程的时候,方程的两边都乘以 $2(5 + x)$,接着求出 $x = 3$,
而 $2(5 + x) = 16$,所以相当于方程两边都乘以 $16$,$16 \neq 0$,因此所得的整式方程与原分式方程同解。
在解第二个分式方程的时候,方程两边都乘以 $(x + 1)(x - 1)$,接着求出$x = 1$,
相当于方程两边都乘以零,结果使原分式方程中有的分式没有意义了,这样得到的整式方程就与原分式方程不同解了。

在方程变形时,有时可能产生不适合原方程的根,这种根叫做原方程的\zhongdian{增根}。
前面第二个例子中求出的整式方程的根 $x = 1$ 就是原分式方程的增根。
因为解分式方程时可能产生增根,所以解分式方程必须检验。
为了简便,通常把求得的根代入变形时所乘的整式(最简公分母),看它的值是否为零,
使这个整式为零的根是原方程的增根,必须舍去。

综上所述,解分式方程的一般步骤是:

\jiange
\framebox{\begin{minipage}{0.93\textwidth}
    \zhongdian{1. 在方程的两边都乘以最简公分母,约去分母,化成整式方程;}

    \zhongdian{2. 解这个整式方程;}

    \zhongdian{3. 把整式方程的根代入最筒公分母,看结果是不是零。使最简公分母为零的根是原方程的增根,必须舍去。}
\end{minipage}}\jiange

\liti 解方程 $\dfrac{5}{x} = \dfrac{7}{x - 2}$。

\jie 方程两边都乘以 $x(x - 2)$,约去分母,得
$$ 5(x - 2) = 7x \juhao $$

解这个整式方程,得
$$ x = -5 \juhao $$

检验:当 $x = -5$ 时,
$$ x(x - 2) = (-5) \times (-5 - 2) = 35 \neq 0 \douhao $$
所以 $-5$ 是原方程的根。

\liti 解方程 $\dfrac{x-2}{x+2} - \dfrac{16}{x^2-4} = \dfrac{x+2}{x-2}$。

\jie 方程的两边都乘以 $(x+2)(x-2)$,约去分母,得
$$ (x-2)^2 - 16 = (x+2)^2 \juhao $$

解这个整式方程,得
$$ x = -2 \juhao $$

检验:当 $x=-2$ 时,$(x+2)(x-2)=0$, 所以 $-2$ 是增根,原方程无解。


\liti 在公式 $\dfrac{1}{u} + \dfrac{1}{v} = \dfrac{1}{f}$ 中,$u + v \neq 0$,已知 $u$,$v$,求 $f$。

\jie 公式两边都乘以 $uvf$ ,得
$$ vf + uf = uv \douhao $$
即
$$ (u + v)f = uv \juhao $$

因为 $u + v \neq 0$, 方程两边都除以 $u + v$,得
$$ f = \dfrac{uv}{u + v} \juhao $$

\zhuyi 本书中含有字母已知数的分式方程,一律不要求检验。

\lianxi
\begin{xiaotis}

\xiaoti{解下列方程:}
\begin{xiaoxiaotis}

    \begin{tblr}{columns={18em, colsep=0pt}}
        \xxt{$\dfrac{2}{x} = \dfrac{3}{x+1}$;} & \xxt{$\dfrac{2}{x+3} = \dfrac{1}{x-1}$;} \\
        \xxt{$\dfrac{x}{x-3} = 2 + \dfrac{3}{x-3}$;} & \xxt{$\dfrac{1}{x-2} = \dfrac{1-x}{2-x} - 3$;} \\
        \xxt{$\dfrac{x}{2x-5} + \dfrac{5}{5-2x} = 1$;} & \xxt{$\dfrac{7}{x^2+x} + \dfrac{1}{x^2-x} = \dfrac{6}{x^2-1}$。}
    \end{tblr}

\end{xiaoxiaotis}

\xiaoti{在公式 $\dfrac{P_1}{V_2} = \dfrac{P_2}{V_1}$ 中,$P_1 \neq 0$,求出表示 $V_1$ 的公式。}

\xiaoti{在公式 $\dfrac{1}{R} = \dfrac{1}{R_1} + \dfrac{1}{R_2}$ 中,$R \neq R_1$,求出表示 $R_2$ 的公式。}

\end{xiaotis}
\lianxijiange


\liti 解方程组
\begin{numcases}{}
    \dfrac{y+6}{x-2} - 2 = 0 \douhao \tag{1} \\
    \dfrac{x}{x+4} = \dfrac{y+1}{y-3} \juhao \tag{2}
\end{numcases}

\jie (1) 式两边都乘以 $x-2$,化简,得
$$ 2x - y = 10 \juhao $$

(2) 式两边都乘以 $(x+4)(y-3)$,化简,得
$$ x + y = -1 \juhao $$

解方程组
$$\begin{cases}
    2x - y = 10 \douhao \\
    x + y = -1 \douhao
\end{cases}$$
得
$$\begin{cases}
    x = 3 \douhao \\
    y = -4 \juhao
\end{cases}$$

经检验,
$$\begin{cases}
    x = 3 \douhao \\
    y = -4
\end{cases}$$
是原方程组的解。


\liti 解方程组
\begin{numcases}{}
    \dfrac{6}{x} + \dfrac{6}{y} = \dfrac{1}{2} \douhao \tag{1} \\
    \dfrac{8}{x} - \dfrac{3}{y} = \dfrac{3}{10} \juhao \tag{2}
\end{numcases}

\jie 设 $\dfrac{1}{x} = X$,$\dfrac{1}{y} = Y$,则原方程组变为
$$\begin{cases}
    6X + 6Y = \dfrac{1}{2} \douhao \\[1em]
    8X - 3Y = \dfrac{3}{10} \juhao
\end{cases}$$

解这个整式方程组,得
$$\begin{cases}
    X = \dfrac{1}{20} \douhao \\[1em]
    Y = \dfrac{1}{30} \douhao
\end{cases}$$
即
$$\begin{cases}
    \dfrac{1}{x} = \dfrac{1}{20} \douhao  \\[1em]
    \dfrac{1}{y} = \dfrac{1}{30} \juhao
\end{cases}$$

由此得
$$\begin{cases}
    x = 20 \douhao \\
    y = 30 \juhao
\end{cases}$$

经检验,
$$\begin{cases}
    x = 20 \douhao \\
    y = 30
\end{cases}$$
是原方程组的解。

例 5 的解法叫做\zhongdian{换元法},也就是把适当的含有未知数的式子如 $\dfrac{1}{x}$,$\dfrac{1}{y}$ 换成
新的未知数 $X$,$Y$,求出 $X$,$Y$ 之后再求 $x$,$y$。

\lianxi

解列方程组:

\begin{xiaoxiaotis}

    \setcounter{cntxiaoxiaoti}{0}
    \begin{tblr}{columns={18em, colsep=0pt}, rows={rowsep=.5em}}
        \xxt{$\begin{cases}
                \dfrac{x}{y} = \dfrac{3}{4} \douhao \\[1em]
                \dfrac{x-1}{y+2} = \dfrac{1}{2} \fenhao
            \end{cases}$} & \xxt{$\begin{cases}
                \dfrac{4}{x} + \dfrac{5}{y} = 0 \douhao \\[1em]
                \dfrac{x}{x+4} - \dfrac{y+3}{y-3} = 0 \fenhao
            \end{cases}$} \\
        \xxt{$\begin{cases}
            \dfrac{2}{x} + \dfrac{3}{y} = 1 \douhao \\[1em]
            \dfrac{3}{x} - \dfrac{2}{y} = -1 \juhao
        \end{cases}$}
    \end{tblr}

\end{xiaoxiaotis}


\end{enhancedline}

