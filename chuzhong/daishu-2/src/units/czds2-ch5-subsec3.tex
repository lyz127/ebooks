\subsection{用代入法解二元一次方程组}\label{subsec:5-3}

\begin{enhancedline}
求方程组的解的过程,叫做\zhongdian{解方程组}。

下面我们来学习解二元一次方程组的两种常用方法。

我们学过解一元一次方程,如果能够通过二元一次方程组里的两个方程,得到一个只含一个未知数的一元方程(即一元一次方程),
求出这个未知数的值,然后再设法求出另一个未知数的值,问题就解决了。

下面我们就按照这条化 “二元” 为 “一元” 的思路来分析具体问题。例如,解方程组
$$\begin{cases}
    y = 2x, \\
    x + y = 3,
\end{cases}$$
这就是要求出这两个二元一次方程的公共解。
如果这两个二元一次方程有公共解,那么两个方程中同一个未知数就应取相同的值。
因此,第二个方程中的 $y$ 可用第一个方程中表示 $y$ 的代数式来代替:\jiange
\begin{align}
    y = &\quad  2x
        \tikz[overlay, >=Stealth] {
            \draw (-1.4em, 1.3em) rectangle (.3em, -.8em);
            \draw [->] (-.8em, -.8em) [->] -- (-.8em, -2.1em);
        }
        \quad \douhao \\[1em]
    x \; + &\quad y = 3 \juhao
\end{align}
把 (1) 代入 (2), 得 $x + 2x = 3$。
这样,就由两个二元一次方程得到一个一元一次方程,消去了一个未知数。
解这个一元一次方程, 得 $x = 1$, 把 $x = 1$ 代入 (1), 就可以得到 $y = 2$。

要检验所得结果是不是原方程组的解,应把这对数值代入原方程组里的每一个方程进行检验。

经过检验可以知道,由上述步骤得到的一对未知数的值
$$\begin{cases}
    x = 1, \\
    y = 2
\end{cases}$$
是原方程组的解。

我们再看几个例子。

\liti 解方程组
\begin{numcases}{}
    y = 1 - x \douhao  \tag{1} \\
    3x + 2y = 5 \juhao \tag{2}
\end{numcases}

\jie 把 (1) 代入 (2),得
\begin{align*}
    & 3x + 2(1 - x) = 5 , \\
    & 3x + 2 - 2x = 5 ,
\end{align*}

\fenge{$\therefore$}{$$ x = 3 \juhao $$}

把 $x = 3$ 代入 (1),得
$$ y = -2 \juhao $$

\fenge{$\therefore$}{
    $$\begin{cases}
        x = 3, \\
        y = -2 \juhao
    \end{cases}$$
}

检验:把 $x = 3$, $y = -2$ 代入 (1),得
\begin{gather*}
    \zuobian = -2 \douhao \youbian = 1 - 3 = -2 \douhao \\
    \zuobian = \youbian \fenhao
\end{gather*}
再代入 (2),得
\begin{gather*}
    \zuobian = 3 \times 3 + 2 \times (-2) = 5 \douhao \youbian = 5 \douhao \\
    \zuobian = \youbian \douhao
\end{gather*}
所以
$$\begin{cases}
    x = 3, \\
    y = -2
\end{cases}$$
是原方程组的解。

(检验可用口算,不必写出,以下同。)

\liti 解方程组
\begin{numcases}{}
    2x + 5y = -21 \douhao  \tag{1} \\
    x + 3y = 8 \juhao \tag{2}
\end{numcases}

分析:在这个方程组里,方程 (2) 中未知数 $x$ 的系数是 1,为了方便起见,
可以先把方程 (2) 变形,用含 $y$ 的代数式表示 $x$,然后再解。

\jie 由 (2),得
\begin{gather}
    x = 8 - 3y \juhao \tag{3}
\end{gather}

把 (3) 代入 (1),得
\begin{gather*}
    2(8 - 3y) + 5y = -21 , \\
    16 - 6y + 5y = -21 , \\
    -y = -37 ,
\end{gather*}

\fenge{$\therefore$}{$$ y = 37 \juhao $$}

把 $y = 37$ 代入 (3),得
$$ x = 8 - 3 \times 37 , $$

\fenge{$\therefore$}{$$ x = -103 \juhao $$}

\fenge{$\therefore$}{
    $$\begin{cases}
        x = -103 \douhao \\
        y = 37 \juhao
    \end{cases}$$
}


\liti 解方程组
\begin{numcases}{}
    2x - 7y = 8 \douhao  \tag{1} \\
    3x - 8y - 10 = 0 \juhao \tag{2}
\end{numcases}

分析:在这个方程组里,每个方程中各个未知数的系数都不是 1,但可以运用方程同解原理
把其中的一个方程变形,使这个方程中的一个未知数的系数为 1, 然后再解。

\jie 由 (1),得
\begin{align*}
    2x &= 8 + 7y, \\
    x  &= \dfrac{8 + 7y}{2} \juhao \tag{3}
\end{align*}

把 (3) 代入 (2),得
\begin{gather*}
    \dfrac{3(8 + 7y)}{2} - 8y - 10 = 0, \\
    24 + 21y - 16y - 20 = 0, \\
    5y = -4,
\end{gather*}

\fenge{$\therefore$}{$$ y = -\dfrac{4}{5} \juhao $$}

把 $y = -\dfrac{4}{5}$ 代入 (3),得
$$ x = \dfrac{8 + 7 \times \left(-\dfrac{4}{5}\right)}{2} , $$

\fenge{$\therefore$}{$$ x = 1\dfrac{1}{5} \juhao $$}

\fenge{$\therefore$}{
    $$\begin{cases}
        x = 1\dfrac{1}{5} \douhao \\
        y = -\dfrac{4}{5} \juhao
    \end{cases}$$
}


上面几个例题的解题步骤一般是:

1. 将方程组里的一个方程变形,用含有一个未知数的代数式表示另一个未知数;

2. 用这个代数式代替另一个方程中相应的未知数,使解二元一次方程组转化为解一元一次方程,求得一个未知数的值;

3. 把求得的这个未知数的值代入原方程组里的任意一个方程,求得另一个未知数的值,从而得到方程组的解。

这种解方程组的方法叫做\zhongdian{代入消元法}, 简称\zhongdian{代入法}。


\lianxi
\begin{xiaotis}

\xiaoti{用代入法解下列方程组, 并写出检验:}
\begin{xiaoxiaotis}

    \twoInLineXxt[18em]{
        $\begin{cases}
            y = 2x, \\
            7x - 3y = 1;
        \end{cases}$
    }{
        $\begin{cases}
            x + 5z = 6, \\
            3x - 6z = 4 \juhao
        \end{cases}$
    }

\end{xiaoxiaotis}


\xiaoti{用代入法解下列方程组:}
\begin{xiaoxiaotis}

    \begin{tblr}{columns={18em, colsep=0pt}}
        \xxt{$\begin{cases}
                y = 2x - 3, \\
                3x + 2y = 8;
              \end{cases}$}
            & \xxt{$\begin{cases}
                6x - 5y = -1, \\
                x = \dfrac{2}{3};
              \end{cases}$} \\
        \xxt{$\begin{cases}
                2s = 3t, \\
                3s - 2t = 5;
              \end{cases}$}
            & \xxt{$\begin{cases}
                2x - z = 5, \\
                3x + 4z = 2;
              \end{cases}$} \\
       \xxt{$\begin{cases}
                2x + 3y = -1, \\
                4x - 9y = 8;
              \end{cases}$}
            & \xxt{$\begin{cases}
                3m - 4n = 7, \\
                9m - 10n + 25 = 0 \juhao
              \end{cases}$} \\
    \end{tblr}

\end{xiaoxiaotis}

\end{xiaotis}
\end{enhancedline}

