\subsection{三元一次方程组的解法举例}\label{subsec:5-5}

\begin{enhancedline}
含有三个未知数,并且含有未知数的项的次数都是 1 的方程叫做\zhongdian{三元一次方程}。
例如,$x + 3y + 5z = 4$ 就是一个关于与 $x$,$y$,$z$ 的三元一次方程。
任何三元一次方程都有无数个解。

由几个一次方程组成并含有三个未知数的方程组,叫做\zhongdian{三元一次方程组}。例如,方程组
\begin{numcases}{}
    3x + 2y + z = 13 \douhao \tag{1} \\
    x + y + 2z = 7   \douhao \tag{2} \\
    2x + 3y - z = 12         \tag{3}
\end{numcases}
就是由三个一次方程组成的三元一次方程组。

本章中所说的三元一次方程组,都是指由三个一次方程组成的三元一次方程组。下面我们学习三元一次方程组的一种解法。

解三元一次方程组,可以把方程组里的一个方程分别与另外两个方程结合成两组,
利用代入法或加减法消去这两组中的同一个未知数, 得到含另外两个未知数的两个二元一次方程,
解由这两个二元一次方程组成的二元一次方程组, 求得两个未知数的值后, 再求出被消去的那个未知数的值。
这样, 通过把 “三元” 的问题逐步化为 “二元” 、“一元” 的问题, 从而求得了原方程组的解。 我们看几个例子。

\liti 解前面给出的三元一次方程组。

分析: 在这个方程组里,方程 (1) 中未知数 $z$ 的系数为 1, 可以把方程 (1) 分别与 (2), (3) 结合,先消去 $z$。

\jie $(1) + (3)$,得
\begin{gather*}
    5x + 5y = 25 \juhao \tag{4}
\end{gather*}

$(1) \times 2 - (2)$,得
\begin{gather*}
    5x + 3y = 19 \juhao \tag{5}
\end{gather*}

把方程 (4), (5) 组成一个二元一次方程组
\begin{numcases}{}
    5x + 5y = 25 \douhao \tag{4} \\
    5x + 3y = 19 \juhao \tag{5}
\end{numcases}

解这个二元一次方程组,得
$$ \begin{cases} x = 2 \douhao \\ y = 3 \juhao \end{cases} $$

把 $x = 2$,$y = 3$ 代入 (1),得
$$ 3 \times 2 + 2 \times 3 + z = 13 \douhao $$

\fenge{$\therefore$}{$$ z = 1 \juhao $$}

\fenge{$\therefore$}{
    $$\begin{cases}
        x = 2 \douhao \\
        y = 3 \douhao \\
        z = 1 \juhao
    \end{cases}$$
}


要检验所得的结果是不是原方程组的解, 应把这组值代入原方程组里的每一个方程进行检验, 但检验一般不必写出。


\liti 解方程组
\begin{numcases}{}
    3x + 4z      = 7 \douhao \tag{1} \\
    2x + 3y + z  = 9 \douhao \tag{2} \\
    5x - 9y + 7z = 8 \juhao  \tag{3}
\end{numcases}

分析: 在这个方程组里, 方程 (1) 只含两个未知数 $x$,$z$,
所以只要由 (2), (3) 消去 $y$,就可以得到含 $x$,$z$ 的二元一次方程组。因此先消去 $y$ 比较方便。

\jie $(2) \times 3 + (3)$,得
\begin{gather*}
    11x + 10z = 35 \juhao \tag{4}
\end{gather*}

把方程 (1), (4) 组成一个二元一次方程组
\begin{numcases}{}
    3x + 4z   = 7  \douhao \tag{1} \\
    11x + 10z = 35 \juhao  \tag{4}
\end{numcases}

解这个二元一次方程组,得
$$\begin{cases}
    x =  5 \douhao \\
    z = -2 \juhao
\end{cases}$$

把 $x = 5$, $z = -2$ 代入 (2),得
\begin{gather*}
    2 \times 5 + 3y - 2 = 9 \douhao \\
    3y = 1 \douhao
\end{gather*}

\fenge{$\therefore$}{$$ y = \dfrac{1}{3} \juhao $$}

\fenge{$\therefore$}{
    $$\begin{cases}
        x = 5 \douhao \\
        y = \dfrac{1}{3} \douhao \\
        z = -2 \juhao
    \end{cases}$$
}

\lianxi

解下列三元一次方程组:
\begin{xiaoxiaotis}

    \twoInLineXxt[18em]{
        $\begin{cases}
            3x - y + z  = 4  \douhao \\
            2x + 3y - z = 12 \douhao \\
            x  + y  + z = 6  \fenhao
        \end{cases}$
    }{
        $\begin{cases}
            2x + 4y + 3z = 9  \douhao \\
            3x - 2y + 5z = 11 \douhao \\
            5x - 6y + 7z = 13 \fenhao
        \end{cases}$
    }

    \twoInLineXxt[18em]{
        $\begin{cases}
            4x      -  9z = 17 \douhao \\
            3x +  y + 15z = 18 \douhao \\
             x + 2y +  3z =  2 \fenhao
        \end{cases}$
    }{
        $\begin{cases}
            z = x + y  \douhao \\
            2x - 3y + 2z = 5 \douhao \\
            x + 2y - z = 3 \juhao
        \end{cases}$
    }

\end{xiaoxiaotis}
\end{enhancedline}
