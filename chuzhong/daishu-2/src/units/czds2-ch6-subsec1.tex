\subsection{同底数的幂的乘法}\label{subsec:6-1}

我们已经学过整式的加减,现在进一步学习整式的乘除。为此,先研究同底数的幂的乘法。我们来计算
$$ 10^3 \times 10^2,\quad 2^3 \times 2^2 \juhao $$

根据乘方的意义,得
\begin{align*}
    10^3 \times 10^2 &= (10 \times 10 \times 10) \times (10 \times 10) \\
        &= 10 \times 10 \times 10 \times 10 \times 10 \\
        &= 10^5, \\
    2^3 \times 2^2 &= (2 \times 2 \times 2) \times (2 \times 2) \\
        &= 2 \times 2 \times 2 \times 2 \times 2 \\
        &= 2^5 \juhao
\end{align*}

从上面看到,$10^3$ 是 3 个 10 连乘,$10^2$ 是 2 个 10 连乘,所以 $10^3 \times 10^2$ 是 5 个 10 连乘。
同样,$2^3$ 是 3 个 2 连乘,$2^2$ 是 2 个 2 连乘,$2^3 \times 2^2$ 是 5 个 2 连乘。也就是
\begin{align*}
    10^3 \times 10^2 = 10^{3+2} \douhao \\
    2^3  \times 2^2  = 2^{3+2} \juhao
\end{align*}

同样,
\begin{align*}
    a^3 \cdot a^2 &= (aaa) \; (aa) \\
        &= aaaaa \\
        &= a^5 \douhao
\end{align*}
也就是
$$ a^3 \cdot a^2 = a^{3+2} \juhao $$

一般地,如果 $m$,$n$ 都是正整数,那么
\begin{align*}
    a^m \cdot a^n &= (\underbrace{aa \cdots a}_{m \text{个} a}) \; (\underbrace{aa \cdots a}_{n \text{个} a}) \\
        &= \underbrace{aa \cdots a}_{(m+n) \text{个} a} = a^{m+n} \douhao
\end{align*}
即
\begin{center}
    \framebox{\quad $a^m \cdot a^n = a^{m+n}$。\;}
\end{center}
这就是说,\zhongdian{同底数的幂相乘,底数不变,指数相加。}

当三个或三个以上同底数的幂相乘时,也具有这一性质。例如

$a^m \cdot a^n \cdot a^p = a^{m+n+p}$ ($m$,$n$,$p$ 都是正整数。)\footnote{本章所有的幂指数都是正整数。}


\liti 计算:
\begin{xiaoxiaotis}

    \begin{tblr}{columns={18em, colsep=0pt}}
        \xxt{$10^7 \times 10^4$;} & \xxt{$x^2 \cdot x^5$;} \\
        \xxt{$y \cdot y^2 \cdot y^3$;} & \xxt{$-a^2 \cdot a^6$。}
    \end{tblr}

\resetxxt
\jie \xxt{$10^7 \times 10^4 = 10^{7+4} = 10^{11}$;}

\xxt{$x^2 \cdot x^5 = x^{2+5} = x^7$;}

\xxt{$y \cdot y^2 \cdot y^3 = y^{1+2+3} = y^6$;}

\xxt{$-a^2 \cdot a^6 = -(a^2 \cdot a^6) = -a^{2+6} = -a^8$。}

\end{xiaoxiaotis}

\liti 计算:
\begin{xiaoxiaotis}

    \twoInLineXxt[18em]{$x^n \cdot x^2$;}{$y^m \cdot y^{m+1}$。}

\resetxxt
\jie \xxt{$x^n \cdot x^2 = x^{n+2}$;}

\xxt{$y^m \cdot y^{m+1} = y^{m+(m+1)} = y^{2m+1}$。}

\end{xiaoxiaotis}

\liti 把下列各式化成 $(p + q)^n$ 或 $(s - t)^n$ 的形式。
\begin{xiaoxiaotis}

    \xxt{$(p + q)^3 \cdot (p + q)^2$;}

    \xxt{$(s - t)^2 \cdot (s - t) \cdot (s - t)^4$;}

    \xxt{$(p + q)^m \cdot (p + q)^n$。}

分析: 把 $(p + q)$ 或 $(s - t)$ 看作底数 $a$,就可运用同底数的幂相乘的性质来进行计算。

\resetxxt
\jie \xxt{$(p + q)^3 \cdot (p + q)^2 = (p + q)^{3+2} = (p + q)^5$;}

\xxt{$(s - t)^2 \cdot (s - t) \cdot (s - t)^4 = (s - t)^{2+1+4} = (s - t)^7$;}

\xxt{$(p + q)^m \cdot (p + q)^n = (p + q)^{m+n}$。}

\end{xiaoxiaotis}


\lianxi
\begin{xiaotis}

\xiaoti{(口答)计算:}
\begin{xiaoxiaotis}

    \begin{tblr}{columns={12em, colsep=0pt}}
        \xxt{$10^5 \cdot 10^6$;} & \xxt{$s^5 \cdot s^8$;} & \xxt{$a^7 \cdot a^3$;} \\
        \xxt{$y^3 \cdot y^2$;}   & \xxt{$b^5 \cdot b$;}   & \xxt{$x^4 \cdot x^4 \cdot x$;} \\
        \xxt{$a^n \cdot a$;}     & \xxt{$x^n \cdot x^n$。}
    \end{tblr}

\end{xiaoxiaotis}


\xiaoti{计算:}
\begin{xiaoxiaotis}

    \begin{tblr}{columns={12em, colsep=0pt}}
        \xxt{$10 \times 10^8$;}   & \xxt{$a^4 \cdot a^6$;} & \xxt{$x^5 \cdot x^5$;} \\
        \xxt{$y^{12} \cdot y^6$;} & \xxt{$x^{10} \cdot x$;} & \xxt{$-b^3 \cdot b^7$;} \\
        \xxt{$y^4 \cdot y^3 \cdot y^2 \cdot y$;} & \xxt{$x^5 \cdot x^6 \cdot x^3$;} & \xxt{$10^2 \cdot 10^n$;} \\
        \xxt{$a^n \cdot a^{2n}$;} & \xxt{$y^{m+1} \cdot y^{m-1}$;} & \xxt{$(-2)^2 \cdot (-2)^3$;} \\
        \xxt{$y^n \cdot y \cdot y^{n+1}$。}
    \end{tblr}

\end{xiaoxiaotis}

\xiaoti{把下列各式化成 $(x + y)^n$ 的形式:}
\begin{xiaoxiaotis}

    \begin{tblr}{columns={18em, colsep=0pt}}
        \xxt{$(x + y)^2 \cdot (x + y)^2$;} & \xxt{$(x + y)^3 \cdot (x + y)$;} \\
        \xxt{$(x + y)^3 \cdot (x + y) \cdot (x + y)^2$;} & \xxt{$(x + y)^{m+1} \cdot (x + y)^{m+n-1}$。}
    \end{tblr}

\end{xiaoxiaotis}

\xiaoti{下面的计算对不对,为什么?如果不对,应怎样改正?}
\begin{xiaoxiaotis}

    \begin{tblr}{columns={18em, colsep=0pt}}
        \xxt{$b^5 \cdot b^5 = 2b^5$;} & \xxt{$x^5 + x^5 = x^{10}$;} \\
        \xxt{$c \cdot c^3 = c^3$;} & \xxt{$m^3 \cdot m^2 = m^5$。}
    \end{tblr}

\end{xiaoxiaotis}

\end{xiaotis}

