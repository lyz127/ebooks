\begin{tikzpicture}
    % 绘制已知的 正五边形
    \tkzDefPoints{0/0/A, 2/0/B}
    \tkzDefRegPolygon[side,sides=5,name=P](A,B)
    \tkzDrawPolygon[thick](P1,P...,P5)
    \coordinate (C) at (P3); % 后面还会用到这些点,所以将其命名为 C、D、E
    \coordinate (D) at (P4);
    \coordinate (E) at (P5);
    \tkzLabelPoint[left,  yshift=-.3em](A){$A$}
    \tkzLabelPoint[right, yshift=-.3em](B){$B$}
    \tkzLabelPoint[right](C){$C$}
    \tkzLabelPoint[above](D){$D$}
    \tkzLabelPoint[left](E){$E$}

    % 绘制外接圆
    \tkzDefLine[mediator](A,B)   \tkzGetPoints{a}{b}
    \tkzDefLine[mediator](C,D)  \tkzGetPoints{c}{d}
    \tkzInterLL(a,b)(c,d)  \tkzGetPoint{O}
    \tkzDrawCircle[very thick](O,A)
    \tkzLabelPoint[left](O){$O$}

    % 绘制内切圆
    \tkzDefPointBy[projection= onto A--B](O)  \tkzGetPoint{H}
    \tkzDrawCircle[very thick](O,H)

    % 其它
    \tkzDrawSegments[dashed](O,A  O,B  O,C  O,D)
    \extkzLabelAngel[0.5](C,B,O){$1$}
    \extkzLabelAngel[0.5](O,C,B){$2$}
    \extkzLabelAngel[0.7](O,B,A){$3$}
    \extkzLabelAngel[0.7](D,C,O){$4$}
\end{tikzpicture}

