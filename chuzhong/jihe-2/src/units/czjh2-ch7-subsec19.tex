\subsection{圆周长、弧长}\label{subsec:czjh2-7-19}

\begin{enhancedline}

\subsubsection{圆周长}

我们知道,圆周长 $C$ 与半径 $R$ 之间有下面的关系

\begin{center}
    \framebox[8em]{\zhongdian{$\bm{C = 2 \pi R$}。}}
\end{center}

这里 $\pi = 3.14159\cdots$, 这个无限不循环小数叫做\zhongdian{圆周率}(详见\hyperref[sec:czjh2-7-fulu]{附录})。

\subsubsection{弧长}

\begin{wrapfigure}[8]{r}{7.5cm}
    \centering
    \begin{tikzpicture}
    \pgfmathsetmacro{\factor}{.025}

    \tkzDefPoints{0/0/O}
    \foreach \n [count=\i] in {110, 90, 70} {
        \ifnum\i=2\relax
            \def\linestyle{densely dash dot}
        \else
            \def\linestyle{very thick}
        \fi

        \pgfmathsetmacro{\R}{\n*\factor}

        \tkzDefPoint(40:\R){A\i}
        \tkzDefPoint(140:\R){B\i}
        \tkzDrawArc[\linestyle](O,A\i)(B\i)

        \tkzDefLine[perpendicular=through A\i, normed](A\i,O)  \tkzGetPoint{c\i}
        \tkzDefPointOnLine[pos=70*\factor](A\i,c\i)  \tkzGetPoint{C\i}
        \tkzDrawSegment[\linestyle](A\i,C\i)

        \tkzDefLine[perpendicular=through B\i, normed](O,B\i)  \tkzGetPoint{d\i}
        \tkzDefPointOnLine[pos=70*\factor](B\i,d\i)  \tkzGetPoint{D\i}
        \tkzDrawSegment[\linestyle](B\i,D\i)
    }
    \tkzDrawSegments[very thick](C3,C1  D3,D1)

    %
    \tkzLabelPoints[below](O)
    \extkzLabelAngel[0.5](A2,O,B2){$100^\circ$}
    \tkzDrawSegment[-Latex](O,A2)
    \tkzLabelSegment[pos=.4, rotate=45](O,A2){$R\;90$}

    \tkzDefPoint(90:90*\factor){l}
    \tkzLabelPoints[centered, fill=white, inner sep=1pt](l)

    \tkzDrawLines[add=0 and 0.2](O,A1   O,B1)
    \tkzDrawLines[add=0 and 0.5](C3,C1  D3,D1)

    \tkzDrawSegments[sloped, dim={$70$,1em,}](A1,C1)
    \tkzDrawSegments[sloped, dim={$70$,1em,}](D1,B1)
\end{tikzpicture}


    \caption{}\label{fig:czjh2-7-75}
\end{wrapfigure}

因为 $360^\circ$ 的圆心角所对的弧长就是圆周长 $C = 2 \pi R$,
所以 $1^\circ$ 的圆心角所对的弧长是 $\dfrac{2 \pi R}{360}$,即 $\dfrac{\pi R}{180}$。
这样,我们就得到,半径为 $R$ 的圆中, $n^\circ$ 的圆心角所对的弧长 $l$ 的计算公式:
$$ \bm{l = \dfrac{n \pi R}{180}} \juhao $$

\liti 弯制管道时,先按中心线计算“展直长度”,再下料。
试计算图 \ref{fig:czjh2-7-75} 所示管道的展直长度 $L$ (单位:mm)。

\jie 由弧长公式,得

$l = \dfrac{100 \times 90 \times \pi}{180} = 50 \pi \approx 157$ (mm) 。

所要求的展直长度

$L = 2 \times 70 + 157 = 297$ (mm)。

答:管道的展直长度为 297 mm。



\liti 如图 \ref{fig:czjh2-7-76}, 两个皮带轮的中心距离为 2.1 m, 直径分别为 0.65 m 和 0.24 m。
(1) 求皮带长; (2)如果小轮每分转 750 转,求大轮每分转多少转。

\jie (1) 作过切点的半径 $O_1A$、$O_1D$、$O_2B$、$O_2C$, 作 $O_2E \perp O_1A$,垂足为 $E$。

$\because$ \quad $O_1A = \exdfrac{1}{2} \times 0.65 = 0.325$,
    $AE = O_2B = \exdfrac{1}{2} \times 0.24 = 0.12$, $O_1O_2 = 2.1$,

$\therefore$ \quad $O_1E = O_1A - AE = 0.325 - 0.12 = 0.205$。

$\therefore$ \quad $AB = O_2E = \sqrt{O_1O_2^2 - O_1E^2} \approx 2.090$ (m)。

$\because$ \quad $\cos\alpha = \dfrac{O_1E}{O_1O_2} \approx 0.0976$, $\angle\alpha \approx 84.4^\circ$,

$\therefore$ \quad $\angle\alpha_1 = 360^\circ - 2 \angle\alpha = 191.2^\circ$。

$\therefore$ \quad $\yuanhu{AmD}$  的长 $l_1 = \dfrac{191.2 \times \pi \times 0.325}{180} \approx 1.085$ (m)。

$\because$ \quad $\angle \alpha_2 = 360^\circ - \angle \alpha_1 = 168.8^\circ$,

$\therefore$ \quad $\yuanhu{BnC}$ 的长 $l_2 = \dfrac{168.8 \times \pi \times 0.12}{180} \approx 0.354$ (m)。

$\therefore$ \quad 皮带长 $l = l_1 + l_2 + 2 AB = 5.62$ (m)。

(2) 大轮每分的转数为

\qquad $n = \dfrac{0.12 \times 750}{0.325} \approx 277$ (转)。

答:皮带长 5.62 m, 大轮每分转 277 转。

\begin{figure}[htbp]
    \centering
    \begin{minipage}[b]{10.5cm}
        \centering
        \begin{tikzpicture}
    \pgfmathsetmacro{\factor}{.03}
    \pgfmathsetmacro{\R}{65*\factor}
    \pgfmathsetmacro{\r}{24*\factor}
    \pgfmathsetmacro{\Oo}{210*\factor}

    \tkzDefPoints{0/0/O1, \Oo/0/O2}
    \tkzDefCircle[R](O1,\R)  \tkzGetPoint{o1}
    \tkzDefCircle[R](O2,\r)  \tkzGetPoint{o2}
    \tkzDefSimilitudeCenter[ext](O1,o1)(O2,o2)  \tkzGetPoint{J}
    \tkzDefLine[tangent from = J](O1,o1)  \tkzGetPoints{D}{A}
    \tkzDefLine[tangent from = J](O2,o2)  \tkzGetPoints{C}{B}
    \tkzDefLine[altitude](A,O2,O1)  \tkzGetPoint{E}

    \tkzDrawCircle[very thick](O1, o1)
    \tkzDrawCircle[very thick](O2, o2)
    \tkzDrawSegments[thick](A,B  C,D)
    \tkzDrawSegments[dashed](O1,A  O1,D  O2,B  O2,C  O2,E  O1,O2)

    \extkzLabelAngel[0.5](O2,O1,A){$\alpha$}

    % \extkzLabelAngel[0.7](A,O1,D){$\alpha_1$}
    \tkzMarkAngle[size=0.7, latex-latex](A,O1,D)
    \tkzLabelAngle[pos=.7, fill=white, inner sep=1pt](A,O1,D){$\alpha_1$}

    %\extkzLabelAngel[0.4](C,O2,B){$\alpha_2$}
    \tkzMarkAngle[size=0.4, latex-latex](C,O2,B)
    \tkzLabelAngle[pos=.4, fill=white, inner sep=1pt](C,O2,B){$\alpha_2$}

    \tkzLabelPoints[above](A,B)
    \tkzLabelPoints[below](C,D)
    \tkzLabelPoints[left](E)
    \tkzLabelPoints[below right](O1)
    \tkzLabelPoints[above, xshift=-.6em](O2)

    \tkzDefPoint(180:\R){m}
    \tkzDefShiftPoint[O2](0:\r){n}
    \tkzLabelPoints[left](m)
    \tkzLabelPoints[right](n)
\end{tikzpicture}


        \caption{}\label{fig:czjh2-7-76}
    \end{minipage}
    \begin{minipage}[b]{5cm}
        \centering
        \begin{tikzpicture}
    \pgfmathsetmacro{\R}{2}
    \pgfmathsetmacro{\r}{1}

    \tkzDefPoints{0/0/O_1, 0/-\r/O_2}
    \tkzDefPoint(270:\R){A}
    \tkzDefPoint(235:\R){C}
    \tkzInterLC[common=O_1](O_1,C)(O_2,A)  \tkzGetFirstPoint{B}

    \tkzDrawCircle[thick](O_1,A)
    \tkzDrawCircle[thick](O_2,A)
    \tkzDrawPoints(O_2)
    \tkzDrawSegments(O_1,A  O_1,C)
    \tkzLabelPoints[above](O_1)
    \tkzLabelPoints[right](O_2)
    \tkzAutoLabelPoints[center=O_1, centered, dist= .15](A,C)
    \tkzLabelPoints[left, yshift=.3em](B)
\end{tikzpicture}


        \caption*{(第 2 题)}
    \end{minipage}
\end{figure}


\begin{lianxi}

\xiaoti{求半径为 46.0 厘米的 $18^\circ 30'$ 的弧长 $l$ (保留 3 个有效数字)。}

\xiaoti{如图,大圆 $O_1$ 的半径 $O_1A$ 是小圆 $O_2$ 的直径,
    $\yuan\,O_1$ 的半径 $O_1C$ 交 $\yuan\,O_2$ 于点 $B$。
    求证: $\yuanhu{AB}$ 和 $\yuanhu{AC}$ 的长相等。
}

\end{lianxi}

\end{enhancedline}

