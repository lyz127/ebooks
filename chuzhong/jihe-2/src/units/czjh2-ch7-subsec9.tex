\subsection{圆的切线的作法,切线长定理}\label{subsec:czjh2-7-9}

根据切线的判定定理,可以得到经过一个已知点作已知圆的切线的方法。
分已知点在圆上与圆外两种情况说明如下:

(1)已知:$\yuan\,O$ 及 $\yuan\,O$ 上的一点 $P$ (图 \ref{fig:czjh2-7-37})。

求作:经过点 $P$ 的 $\yuan\,O$ 的切线。

\zuofa 1. 连结 $OP$。

2. 经过点 $P$ 作 $BC \perp OP$。

直线 $BC$ 就是所求的切线。

由作法可以知道,经过 $\yuan\,O$ 上的一点 $P$,
可以作出并且只可以作出一条 $\yuan\,O$ 的切线。

\begin{figure}[htbp]
    \centering
    \begin{minipage}[b]{7cm}
        \centering
        \begin{tikzpicture}
    \tkzDefPoints{0/0/O}
    \tkzDefPoint(0:1.5){P}
    \tkzDefLine[perpendicular=through P,normed](O,P)  \tkzGetPoint{b}
    \tkzDefPointOnLine[pos=2.0](P,b)  \tkzGetPoint{B}
    \tkzDefPointOnLine[pos=2.0](B,P)  \tkzGetPoint{C}

    \tkzDrawCircle[thick](O,P)
    \tkzDrawPoint(O)
    \tkzDrawSegments(O,P  B,C)
    \tkzLabelPoints[left](O)
    \tkzLabelPoints[right](B,C,P)
\end{tikzpicture}


        \caption{}\label{fig:czjh2-7-37}
    \end{minipage}
    \qquad
    \begin{minipage}[b]{7cm}
        \centering
        \begin{tikzpicture}
    \tkzDefPoints{0/0/O, 1.1/0/R, 3/0/P}
    \tkzDrawCircle[thick](O,R)
    \tkzLabelPoints[left](O)
    \tkzLabelPoints[below=.2em, xshift=.3em](P)

    % 1
    \tkzDrawSegment(O,P)

    % 2
    \tkzDefMidPoint(O,P)  \tkzGetPoint{C}
    \tkzDrawCircle[thick](C,O)
    \tkzDrawPoint(C)
    \tkzInterCC(O,R)(C,O)  \tkzGetPoints{A}{B}
    \tkzLabelPoints[above, xshift=-.2em](A)
    \tkzLabelPoints[below, xshift=-.2em](B)
    \tkzLabelPoints[below](C)

    % 3
    \tkzDrawLines[add=0.2 and 0.2](P,A  P,B)
    \tkzDrawSegments[dashed](O,A  O,B)
\end{tikzpicture}


        \caption{}\label{fig:czjh2-7-38}
    \end{minipage}
\end{figure}

(2)已知: $\yuan\,O$ 及 $\yuan\,O$ 外的一点 $P$(图 \ref{fig:czjh2-7-38})。

求作:经过点 $P$ 的 $\yuan\,O$ 的切线。

分析: 设 $PA$ 是经过点 $P$ 和 $\yuan\,O$ 相切于点 $A$ 的直线,由切线的性质定理,
可知 $OA \perp AP$, 点 $A$ 必在以 $OP$ 为直径的圆上。

\zuofa 1. 连结 $OP$。

2. 以 $OP$ 为直径作 $\yuan\,C$, $\yuan\,C$ 和 $\yuan\,O$ 相交干两点 $A$、$B$。

3. 作直线 $PA$、$PB$。

直线 $PA$、$PB$ 就是所求的切线。

\zhengming 连结 $OA$、$OB$。

$\because$ \quad $OP$ 是 $\yuan\,C$ 的直径,

$\therefore$ \quad $\angle OAP$、$\angle OBP$ 都是直角。

因此, $PA$、$PB$ 是 $\yuan\,O$ 的切线。

由作法可以知道, 经过 $\yuan\,O$ 外的一点 $P$, 可以作出 $\yuan\,O$ 的两条切线。

在经过圆外一点的切线上, 这一点和切点之间的线段的长叫做这点到圆的\zhongdian{切线长}。

在图 \ref{fig:czjh2-7-38} 中,

$\because$ \quad $OA = OB$, $OP = OP$,

$\therefore$ \quad $Rt \triangle AOP \quandeng Rt \triangle BOP$。

$\therefore$ \quad $PA = PB$, $\angle OPA = \angle OPB$。

由此得到下面的定理:

\begin{dingli}[切线长定理]
    从圆外一点引圆的两条切线,它们的切线长相等,圆心和这一点的连线平分两条切线的夹角。
\end{dingli}



\begin{lianxi}

\xiaoti{作已知圆的切线,使它:}
\begin{xiaoxiaotis}

    \xxt{和一条已知直线平行;}

    \xxt{和一条已知直线垂直。}

\end{xiaoxiaotis}


\xiaoti{已知: $\yuan\,O$ 的半径为 3 厘米, 点 $P$ 和圆心 $O$ 的距离为 6 厘米。}
\begin{xiaoxiaotis}

    \xxt{经在点 $P$ 作 $\yuan\,O$ 的切线;}

    \xxt{求两条切线的夹角及切线长。}
\end{xiaoxiaotis}

\xiaoti{$PA$ 和 $PB$ 是 $\yuan\,O$ 的切线, $A$ 和 $B$ 是切点。
    求证: $OP$ 垂直平分弦 $AB$。
}

\end{lianxi}

