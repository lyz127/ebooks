\subsection{点和圆的位置关系}\label{subsec:czjh2-7-1}

在日常生活中,我们到处都会见到圆形的物体。如各种车轮、茶杯的杯口等都是圆形的。
人们为什么把它们做成圆形的呢?这是因为圆形具有许多有用的性质。
在本章中,我们将详细研究圆的性质及其应用。

如图 \ref{fig:czjh2-7-1}, 线段 $OA$ 绕它固定的一个端点 $O$ 旋转一周,另一个端点 $A$ 所经过的封闭曲线叫做\zhongdian{圆}。
固定的点 $O$ 叫做\zhongdian{圆心}; 线段 $OA$ 叫做\zhongdian{半径}。

\begin{figure}[htbp]
    \centering
    \begin{minipage}[b]{4cm}
        \centering
        \begin{tikzpicture}
    \tkzDefPoints{0/0/O, 1.5/0/A}

    \tkzDrawCircle[thick](O,A)
    \tkzDrawSegment(O,A)
    \tkzDrawPoint(O)
    \tkzLabelSegment[above](O,A){$r$}
    \tkzLabelPoints[left](O)
    \tkzLabelPoints[right](A)
\end{tikzpicture}


        \caption{}\label{fig:czjh2-7-1}
    \end{minipage}
    \qquad
    \begin{minipage}[b]{5cm}
        \centering
        \begin{tikzpicture}
    \tkzDefPoints{0/0/O, 2/0/Q, -0.5/1.0/P}
    \tkzDefPoint(20:1.5){A}

    \tkzDrawCircle[thick](O,A)
    \tkzDrawSegments(O,A  O,P  O,Q)
    \tkzDrawPoints(O,P,Q)
    \tkzLabelSegment[above](O,A){$r$}
    \tkzLabelPoints[left](O,P)
    \tkzLabelPoints[right](Q)
\end{tikzpicture}


        \caption{}\label{fig:czjh2-7-2}
    \end{minipage}
    \qquad
    \begin{minipage}[b]{4.5cm}
        \centering
        \begin{tikzpicture}
    \tkzDefPoints{0/0/O, -1.5/0/A, 1.5/0/B}
    \tkzDefPoint(200:1.5){C}
    \tkzDefPoint(290:1.5){D}

    \tkzDrawCircle[thick](O,A)
    \tkzDrawSegments(A,B  C,D)
    \tkzDrawPoint(O)
    \tkzLabelPoints[left](A,C)
    \tkzLabelPoints[right](B)
    \tkzLabelPoints[below right](D)
    \tkzLabelPoints[below](O)
\end{tikzpicture}


        \caption{}\label{fig:czjh2-7-3}
    \end{minipage}
\end{figure}

从上面的定义可以知道:

(1)圆上各点到定点(圆心 $O$ )的距离都等于定长(半径的长 $r$);

(2)到定点的距离等于定长的点都在圆上。

也就是说,圆是那些到定点的距离等于定长的所有的点组成的图形。

\zhongdian{圆可以看作是到定点的距离等于定长的点的集合。}
定点就是圆心,定长就是半径的长,通常也称为半径。

从画圆的过程中,还可以知道:

圆内各点(如图 \ref{fig:czjh2-7-2} 中的点 $P$)到圆心的距离都小于半径;到圆心的距离小于半径的点都在圆内。
也就是说,\zhongdian{圆的内部可以看作是到圆心的距离小于半径的点的集合。}
圆外各点(如图 \ref{fig:czjh2-7-2} 中的点 $Q$)到圆心的距离都大于半径;到圆心的距离大于半径的点都在圆外。
也就是说,\zhongdian{圆的外部可以看作是到圆心的距离大于半径的点的集合。}

% \begin{figure}[htbp]
%     \centering
%     \begin{minipage}[b]{7cm}
%         \centering
%         \begin{tikzpicture}
    \tkzDefPoints{0/0/O, -1.5/0/A, 1.5/0/B}
    \tkzDefPoint(200:1.5){C}
    \tkzDefPoint(290:1.5){D}

    \tkzDrawCircle[thick](O,A)
    \tkzDrawSegments(A,B  C,D)
    \tkzDrawPoint(O)
    \tkzLabelPoints[left](A,C)
    \tkzLabelPoints[right](B)
    \tkzLabelPoints[below right](D)
    \tkzLabelPoints[below](O)
\end{tikzpicture}


%         \caption{}\label{fig:czjh2-7-3}
%     \end{minipage}
%     \qquad
%     \begin{minipage}[b]{7cm}
%         \centering
%         \begin{tikzpicture}
    \tkzDefPoints{0/0/O}
    \tkzDefPoint(150:1.5){A}
    \tkzDefPoint(95:1.5){B}
    \tkzDefPoint(330:1.5){C}

    \tkzDrawCircle[thick](O,A)
    \tkzDrawSegments[dashed](O,A  O,B  O,C)
    \tkzDrawPoint(O)
    \tkzLabelPoints[left](A)
    \tkzLabelPoints[above](B)
    \tkzLabelPoints[right](C)
    \tkzLabelPoints[below left](O)
\end{tikzpicture}


%         \caption{}\label{fig:czjh2-7-4}
%     \end{minipage}
% \end{figure}


以点 $O$ 为圆心的圆,记作 “$\yuan \, O$”,读作 “圆 $O$”。

连结圆上任意两点的线段(如图 \ref{fig:czjh2-7-3} 中的 $CD$)叫做\zhongdian{弦},
经过圆心的弦(如图 \ref{fig:czjh2-7-3} 中的 $AB$)叫做\zhongdian{直径}。
直径等于半径的 2 倍。


圆上任意两点间的部分叫做\zhongdian{圆弧},简称\zhongdian{弧}。
弧用符号“$\yuanhu{\hspace{1em}}$” 表示。
以 $A$、$B$ 为端点的弧记作 $\yuanhu{AB}$,读作“圆弧 $AB$”,或 “弧 $AB$”。
圆的任定一条直径的两个端点分圆成两条弧,每一条弧都叫做\zhongdian{半圆}。
大于半圆的弧(用三个字母表示,如图 \ref{fig:czjh2-7-4} 中的 $\yuanhu{BAC}$)叫做\zhongdian{优弧};
小于半圆的弧(如图 \ref{fig:czjh2-7-4} 中的 $\yuanhu{BC}$)叫做\zhongdian{劣弧}。

圆心相同、半径不相等的两个圆叫做\zhongdian{同心圆}。
图 \ref{fig:czjh2-7-5} 中的两个圆是以点 $O$ 为圆心的同心圆。

\begin{figure}[htbp]
    \centering
    \begin{minipage}[b]{4cm}
        \centering
        \begin{tikzpicture}
    \tkzDefPoints{0/0/O}
    \tkzDefPoint(150:1.5){A}
    \tkzDefPoint(95:1.5){B}
    \tkzDefPoint(330:1.5){C}

    \tkzDrawCircle[thick](O,A)
    \tkzDrawSegments[dashed](O,A  O,B  O,C)
    \tkzDrawPoint(O)
    \tkzLabelPoints[left](A)
    \tkzLabelPoints[above](B)
    \tkzLabelPoints[right](C)
    \tkzLabelPoints[below left](O)
\end{tikzpicture}


        \caption{}\label{fig:czjh2-7-4}
    \end{minipage}
    \qquad
    \begin{minipage}[b]{4cm}
        \centering
        \begin{tikzpicture}
    \tkzDefPoints{0/0/O}
    \tkzDefPoint(30:1.5){A}
    \tkzDefPoint(160:1.0){B}

    \tkzDrawCircle[thick](O,A)
    \tkzDrawCircle[thick](O,B)
    \tkzDrawSegments(O,A  O,B)
    \tkzDrawPoint(O)
    \tkzLabelSegment[below,yshift=-.3em](O,A){$r_1$}
    \tkzLabelSegment[below](O,B){$r_2$}
    \tkzLabelPoints[below](O)
\end{tikzpicture}


        \caption{}\label{fig:czjh2-7-5}
    \end{minipage}
    \qquad
    \begin{minipage}[b]{5cm}
        \centering
        \begin{tikzpicture}
    \def\drawingcode{
        \tkzDefPoints{0/0/O}
        \tkzDefPoint(35:1.0){A}
        \tkzDrawCircle[thick](O,A)
        \tkzDrawSegment(O,A)
        \tkzDrawPoint(O)
        \tkzLabelSegment[above](O,A){$r$}
    }

    \begin{scope}
        \drawingcode
        \tkzLabelPoint[below](O){$O_1$}
    \end{scope}

    \begin{scope}[xshift=2.5cm]
        \drawingcode
        \tkzLabelPoint[below](O){$O_2$}
    \end{scope}
\end{tikzpicture}


        \caption{}\label{fig:czjh2-7-6}
    \end{minipage}
\end{figure}


能够重合的两个圆叫做\zhongdian{等圆},半径相等的两个圆是等圆。
如图 \ref{fig:czjh2-7-6} 中, $\yuan \, O_1$ 和 $\yuan \, O_2$ 的半径都等于 $r$,所以它们是两个等圆。
反过来,\zhongdian{同圆或等圆的半径相等。}

在同圆或等圆中,能够互相重合的弧叫做\zhongdian{等弧}。


\begin{lianxi}

\xiaoti{设 $AB=3$ 厘米,画图说明具有下列性质的点的集合是怎样的图形:}
\begin{xiaoxiaotis}

    \xxt{和点 $A$ 的距离等于 2 厘米的点的集合;}

    \xxt{和点 $B$ 的距离等于 2 厘米的点的集合;}

    \xxt{和点 $A$、$B$ 的距离都等于 2 厘米的点的集合;}

    \xxt{和点 $A$、$B$ 的距离都小于 2 厘米的点的集合。}

\end{xiaoxiaotis}


\xiaoti{下列各题中的两句话都对吗?如果不对,为什么?}
\begin{xiaoxiaotis}

    \xxt{“直径是弦”、“弦是直径”;}

    \xxt{“半圆是弧”、“弧是半圆”。}

\end{xiaoxiaotis}


\xiaoti{适合下列条件的圆,各画三个:}
\begin{xiaoxiaotis}

    \xxt{以已知点 $O$ 为圆心的圆;}

    \xxt{半径等于 2.5 厘米的圆;}

    \xxt{经过已知点 $A$ 的圈;}

    \xxt{经过已知点 $A$ 和 $B$ 的圆。}

\end{xiaoxiaotis}

\end{lianxi}

