\subsection{圆、扇形、弓形的面积}\label{subsec:czjh2-7-20}

\begin{enhancedline}

\subsubsection{圆面积}

我们知道,圆面积 $S$ 与半径 $R$ 之间有下面的关系(详见\hyperref[sec:czjh2-7-fulu]{附录}):

\begin{center}
    \framebox[8em]{\zhongdian{$\bm{S = \pi R^2$}。}}
\end{center}


\subsubsection{扇形面积}

\begin{wrapfigure}[8]{r}{5cm}
    \centering
    \begin{tikzpicture}
    \pgfmathsetmacro{\R}{1.5}

    \tkzDefPoints{0/0/O}
    \tkzDefPoint(270:\R){A}
    \tkzDefPoint(230:\R){B}
    \tkzDefPoint(310:\R){C}

    \tkzDrawCircle[very thick](O,B)
    \tkzDrawSector[pattern={mylines[angle=55, distance={4pt}]}](O,B)(C)
    \tkzDrawLines[add=0 and 0.3](O,B  O,C)
    \tkzMarkAngle[size=0.7, latex-latex](B,O,C)
    \tkzLabelAngle[pos=.7, fill=white, inner sep=0pt](B,O,C){$n^\circ$}
    \tkzMarkAngle[size=1.8, latex-latex](B,O,C)
    \tkzLabelAngle[pos=1.8, fill=white, inner sep=0pt](B,O,C){$l$}
    \tkzLabelSegment[right](O,C){$R$}

    \tkzLabelPoints[above](O)
    \tkzLabelPoints[above](A)
    \tkzLabelPoints[left=.3em](B)
    \tkzLabelPoints[right=.3em](C)
\end{tikzpicture}


    \caption{}\label{fig:czjh2-7-77}
\end{wrapfigure}


一条弧和经过这条弧的端点的两条半径所组成的图形叫做\zhongdian{扇形}。

在半径为 $R$ 的圆中,因为圆心角是 $360^\circ$ 的扇形面积就是圆面积 $S = \pi R^2$,
所以圆心角是 $1^\circ$ 的扇形面积是 $\dfrac{\pi R^2}{360}$。
这样,在半径为 $R$ 的圆中(图 \ref{fig:czjh2-7-77}), 圆心角为 $n^\circ$ 的扇形面积的计算公式是
$$ S_\text{扇形} = \dfrac{n}{360} \pi R^2 \juhao $$

又因为,扇形的弧长 $l = \dfrac{n \pi R}{180}$, 扇形面积 $\dfrac{n}{360} \pi R^2$
可以写成 $\exdfrac{1}{2} \cdot \dfrac{n \pi R}{180} \cdot R$, 所以,又得到
$$ S_\text{扇形} = \exdfrac{1}{2} l R \juhao $$


\subsubsection{弓形面积}

从图 \ref{fig:czjh2-7-78} 中可以看出,把扇形 $OAmB$ 的面积以及 $\triangle OAB$ 的面积计算出来,
就可以得到弓形 $AmB$ 的面积。

\begin{figure}[htbp]
    \centering
    \begin{minipage}[b]{5cm}
        \begin{tikzpicture}
    \pgfmathsetmacro{\R}{1.5}

    \tkzDefPoints{0/0/O}
    \tkzDefPoint(220:\R){A}
    \tkzDefPoint(320:\R){B}
    \tkzDefPoint(270:\R){m}

    \tkzDrawSegment[very thick](A,B)
    \tkzDrawArc[very thick, pattern={mylines[angle=55, distance={4pt}]}](O,A)(B)
    \tkzDrawArc[dashed](O,B)(A)
    \tkzDrawSegments[dashed](O,A  O,B)

    \tkzLabelPoints[above](O)
    \tkzLabelPoints[below](m)
    \tkzLabelPoints[left=.3em](A)
    \tkzLabelPoints[right=.3em](B)
\end{tikzpicture}


    \end{minipage}
    \begin{minipage}[b]{5cm}
        \begin{tikzpicture}
    \pgfmathsetmacro{\R}{1.5}

    \tkzDefPoints{0/0/O}
    \tkzDefPoint(140:\R){A}
    \tkzDefPoint(40:\R){B}
    \tkzDefPoint(270:\R){m}

    \tkzDrawSegment[very thick](A,B)
    \tkzDrawArc[very thick, pattern={mylines[angle=55, distance={4pt}]}](O,A)(B)
    \tkzDrawArc[dashed](O,B)(A)
    \tkzDrawSegments[dashed](O,A  O,B)

    \tkzLabelPoints[below](O)
    \tkzLabelPoints[below](m)
    \tkzLabelPoints[left=.3em](A)
    \tkzLabelPoints[right=.3em](B)
\end{tikzpicture}


    \end{minipage}
    \begin{minipage}[b]{5cm}
        \begin{tikzpicture}
    \pgfmathsetmacro{\R}{1.5}

    \tkzDefPoints{0/0/O}
    \tkzDefPoint(180:\R){A}
    \tkzDefPoint(0:\R){B}
    \tkzDefPoint(270:\R){m}

    \tkzDrawSegment[very thick](A,B)
    \tkzDrawArc[very thick, pattern={mylines[angle=55, distance={4pt}]}](O,A)(B)
    \tkzDrawArc[dashed](O,B)(A)
    \tkzDrawPoint(O)

    \tkzLabelPoints[above](O)
    \tkzLabelPoints[below](m)
    \tkzLabelPoints[left=.3em](A)
    \tkzLabelPoints[right=.3em](B)
\end{tikzpicture}


    \end{minipage}
    \caption{}\label{fig:czjh2-7-78}
\end{figure}

\liti[0] 水平放着的圆柱形排水管的截面半径是 12 cm,其中水面高为 6 cm,
求截面上有水的弓形的面积(精确到 $1\,\pflm$)。

\begin{wrapfigure}[8]{r}{5cm}
    \centering
    \begin{tikzpicture}
    \pgfmathsetmacro{\R}{1.5}

    \tkzDefPoints{0/0/O}
    \tkzDefPoint(210:\R){A}
    \tkzDefPoint(330:\R){B}
    \tkzDefPoint(270:\R){C}
    \tkzInterLL(A,B)(O,C)  \tkzGetPoint{D}

    \tkzDrawCircle[very thick](O,A)
    \tkzDrawArc[very thick, pattern={mylines[angle=55, distance={4pt}]}](O,A)(B)
    \tkzDrawSegments[very thick](A,B  O,C)
    \tkzDrawSegments[dashed](O,A  O,B)
    \tkzMarkRightAngle[size=.2](B,D,O)

    \tkzLabelPoints[above](O)
    \tkzLabelPoints[left](A)
    \tkzLabelPoints[above right](B)
    \tkzLabelPoints[below](C)
    \tkzLabelPoints[above left](D)

    \tkzDefPointOnLine[pos=1.5](D,B)  \tkzGetPoint{b}
    \tkzDefPointBy[translation=from D to b](C)  \tkzGetPoint{c}
    \tkzDefShiftPoint[b](0,.5){b'}
    \tkzDefShiftPoint[c](0,-.5){c'}

    \tkzLabelSegment[above, rotate=30](O,A){\small 12cm}

    \tkzDrawLines[add=0 and 0.1](D,b  C,c)
    \tkzDrawSegments[-Latex](b',b  c',c)
    \tkzLabelSegment[centered, rotate=90](c,b){\small 6cm}
\end{tikzpicture}


    \caption{}\label{fig:czjh2-7-79}
\end{wrapfigure}


\jie 如图 \ref{fig:czjh2-7-79}, 连结 $OA$、$OB$, 作弦 $AB$ 的垂直平分线 $OD$,
垂足为 $D$, 交 $\yuanhu{AB}$ 于点 $C$。已知半径 $OA = 12$ cm,
$DC = 6$ cm。 那么

$OD = OC - DC = 12 - 6 = 6$ (cm),

$AD = \sqrt{OA^2 - OD^2} = \sqrt{12^2 - 6^2} = 6\sqrt{3}$ (cm),

$\angle AOD = 60^\circ$。

$\begin{aligned}
    S_{\text{弓形}ACB} &= S_{\text{扇形}OACB} - S_{\triangle OAD} \\
        &= \dfrac{120}{360} \times \pi \times 12^2 - \exdfrac{1}{2} \times 12\sqrt{3} \times 6 \\
        &= 48 \pi - 36\sqrt{3} \approx 88 (\pflm) \juhao
\end{aligned}$

答: 截面上有水的弓形的面积约为 $88\,\pflm$。


\begin{lianxi}

\xiaoti{设圆周长为 $C$, 圆面积为 $S$。 求证: $S = \dfrac{C^2}{4 \pi}$。}

\xiaoti{已知扇形的圆心角为 $150^\circ$, 弧长为 $20 \pi$ cm。求扇形的面积。}

\xiaoti{在本节的例中,设水面高为 7 cm, 其他条件不变, 利用三角函数解同样的问题。}

\end{lianxi}

\end{enhancedline}

