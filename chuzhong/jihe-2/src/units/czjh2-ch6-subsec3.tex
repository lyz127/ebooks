\subsection{平行线分线段成比例定理}\label{subsec:czjh2-6-3}
\begin{enhancedline}

\begin{wrapfigure}[9]{r}{4.5cm}
    \centering
    \begin{tikzpicture}
    \tkzDefPoints{0/0/xa1, 3/0/xa2, 0/1/xb1, 3/1/xb2, 0/2/xc1, 3/2/xc2}
    \tkzDefPoints{1.1/3/ya1, 0.5/-0.5/ya2, 1.6/3/yb1, 2.5/-0.5/yb2}

    \tkzInterLL(xc1,xc2)(ya1,ya2)  \tkzGetPoint{A}
    \tkzInterLL(xc1,xc2)(yb1,yb2)  \tkzGetPoint{D}
    \tkzInterLL(xb1,xb2)(ya1,ya2)  \tkzGetPoint{B}
    \tkzInterLL(xb1,xb2)(yb1,yb2)  \tkzGetPoint{E}
    \tkzInterLL(xa1,xa2)(ya1,ya2)  \tkzGetPoint{C}
    \tkzInterLL(xa1,xa2)(yb1,yb2)  \tkzGetPoint{F}

    \tkzDrawSegments(xa1,xa2  xb1,xb2  xc1,xc2)
    \tkzDrawSegments(ya1,ya2  yb1,yb2)
    \tkzLabelSegment[pos=1, right](xc1,xc2){$l_1$}
    \tkzLabelSegment[pos=1, right](xb1,xb2){$l_2$}
    \tkzLabelSegment[pos=1, right](xa1,xa2){$l_3$}
    \tkzLabelPoints[above left](A,B,C)
    \tkzLabelPoints[above right](D,E,F)
\end{tikzpicture}


    \caption{}\label{fig:czjh2-6-5}
\end{wrapfigure}

在四边形一章里,我们学过平行线分线段定理。如图 \ref{fig:czjh2-6-5}, $l_1 \pingxing l_2 \pingxing l_3$,
如果 $AB = BC$,那么 $DE = EF$。

由于 $\dfrac{AB}{BC} = 1$, $\dfrac{DE}{EF} = 1$,我们可得比例:
$$ \dfrac{AB}{BC} = \dfrac{DE}{EF} \juhao $$

这就是说,平行线等分线段时,分得的线段成比例。

下面我们来研究平行线不等分线段的情形。如图 \ref{fig:czjh2-6-6}, $l_1 \pingxing l_2 \pingxing l_3$,
如果 $AB \neq BC$,那么四条线段 $AB$、$BC$、$DE$、$EF$ 是否也有比例关系。

以 $B$ 为起点,在 $BA$ 上顺次截取和 $BC$ 相等的线段,有以下几种可能情形:

(1)如果截取 3 次正好截尽,这些分点和点 $B$ 四等分线段 $AC$(图 \ref{fig:czjh2-6-6} 甲),
这时 $\dfrac{AB}{BC} = 3$。经过分点分别作 $l'$、$l''$ 平行 $l_1$。
根据平行线等分线段定理,$l'$、$l''$ 和 $l_2$ 也等分线段 $DF$,
即 $DE = 3EF$, $\dfrac{DE}{EF} = 3$。这就得到
$$ \dfrac{AB}{BC} = \dfrac{DE}{EF} \juhao $$

\begin{figure}[htbp]
    \centering
    \begin{minipage}[b]{6cm}
        \centering
        \begin{tikzpicture}
    \tkzDefPoints{0/0/xa1, 3/0/xa2, 0/0.6/xb1, 3/0.6/xb2, 0/1.2/xc1, 3/1.2/xc2, 0/1.8/xd1, 3/1.8/xd2, 0/2.4/xe1, 3/2.4/xe2}
    \tkzDefPoints{1.1/3/ya1, 0.5/-0.5/ya2, 1.6/3/yb1, 2.5/-0.5/yb2}

    \tkzInterLL(xe1,xe2)(ya1,ya2)  \tkzGetPoint{A}
    \tkzInterLL(xe1,xe2)(yb1,yb2)  \tkzGetPoint{D}
    \tkzInterLL(xb1,xb2)(ya1,ya2)  \tkzGetPoint{B}
    \tkzInterLL(xb1,xb2)(yb1,yb2)  \tkzGetPoint{E}
    \tkzInterLL(xa1,xa2)(ya1,ya2)  \tkzGetPoint{C}
    \tkzInterLL(xa1,xa2)(yb1,yb2)  \tkzGetPoint{F}

    \tkzDrawSegments(xa1,xa2  xb1,xb2  xc1,xc2  xd1,xd2  xe1,xe2)
    \tkzDrawSegments(ya1,ya2  yb1,yb2)
    \tkzLabelSegment[pos=1, right](xe1,xe2){$l_1$}
    \tkzLabelSegment[pos=1, right](xd1,xd2){$l'$}
    \tkzLabelSegment[pos=1, right](xc1,xc2){$l''$}
    \tkzLabelSegment[pos=1, right](xb1,xb2){$l_2$}
    \tkzLabelSegment[pos=1, right](xa1,xa2){$l_3$}
    \tkzLabelSegment[pos=0, above](ya1,ya2){$a$}
    \tkzLabelSegment[pos=0, above](yb1,yb2){$b$}
    \tkzLabelPoints[above left](A,B,C)
    \tkzLabelPoints[above right](D,E,F)
\end{tikzpicture}


        \caption*{甲}
    \end{minipage}
    \qquad
    \begin{minipage}[b]{6cm}
        \centering
        \begin{tikzpicture}
    \tkzDefPoints{0/0/xa1, 3/0/xa2, 0/0.6/xb1, 3/0.6/xb2, 0/1.2/xc1, 3/1.2/xc2, 0/1.8/xd1, 3/1.8/xd2, 0/2.4/xe1, 3/2.4/xe2}
    \tkzDefPoints{0.5/2.46/xxa1, 2/2.46/xxa2, 0.5/2.52/xxb1, 2/2.52/xxb2, 0.5/2.58/xxc1, 2/2.58/xxc2, 0/2.64/xxd1, 2.5/2.64/xxd2}
    \tkzDefPoints{1.1/3/ya1, 0.5/-0.5/ya2, 1.6/3/yb1, 2.5/-0.5/yb2}

    \tkzInterLL(xxd1,xxd2)(ya1,ya2)  \tkzGetPoint{A}
    \tkzInterLL(xxd1,xxd2)(yb1,yb2)  \tkzGetPoint{D}
    \tkzInterLL(xe1,xe2)(ya1,ya2)  \tkzGetPoint{G}
    \tkzInterLL(xb1,xb2)(ya1,ya2)  \tkzGetPoint{B}
    \tkzInterLL(xb1,xb2)(yb1,yb2)  \tkzGetPoint{E}
    \tkzInterLL(xa1,xa2)(ya1,ya2)  \tkzGetPoint{C}
    \tkzInterLL(xa1,xa2)(yb1,yb2)  \tkzGetPoint{F}

    \tkzDrawSegments(xa1,xa2  xb1,xb2  xc1,xc2  xd1,xd2  xe1,xe2)
    \tkzDrawSegments(xxa1,xxa2  xxb1,xxb2  xxc1,xxc2  xxd1,xxd2)
    \tkzDrawSegments(ya1,ya2  yb1,yb2)
    \tkzLabelSegment[pos=1, right](xxd1,xxd2){$l_1$}
    \tkzLabelSegment[pos=1, right](xd1,xd2){$l'$}
    \tkzLabelSegment[pos=1, right](xc1,xc2){$l''$}
    \tkzLabelSegment[pos=1, right](xb1,xb2){$l_2$}
    \tkzLabelSegment[pos=1, right](xa1,xa2){$l_3$}
    \tkzLabelSegment[pos=0, above](ya1,ya2){$a$}
    \tkzLabelSegment[pos=0, above](yb1,yb2){$b$}
    \tkzLabelPoints[above left](A,B,C)
    \tkzLabelPoints[above right](D,E,F)
    \tkzLabelPoints[below left](G)
\end{tikzpicture}


        \caption*{乙}
    \end{minipage}
    \caption{}\label{fig:czjh2-6-6}
\end{figure}


(2)如果截取三次后还剩余一条小于 $BC$ 的线段 $GA$(图 \ref{fig:czjh2-6-6} 乙),
那么再以 $G$ 为起点,在 $GA$ 上顺次截取等于 $\dfrac{BC}{10}$ 的线段,
截 4 次正好截尽,这时 $\dfrac{AB}{BC} = 3.4$。
运用(1)中那样作平行线的方法,可以得到 $\dfrac{DE}{EF} = 3.4$,
因此也有
$$ \dfrac{AB}{BC} = \dfrac{DE}{EF} \juhao $$

(3)如果 $\dfrac{AB}{BC} = 3.47$,同样也有 $\dfrac{DE}{EF} = 3.47$;
如果 $\dfrac{AB}{BC} = 3.476$,也有 $\dfrac{DE}{EF} = 3.476$;
如果 $\dfrac{AB}{BC} = 3.476\cdots$,也有 $\dfrac{DE}{EF} = 3.476\cdots$。

这样,对于 $\dfrac{AB}{BC}$ 是任何实数,都可得
$$ \dfrac{AB}{BC} = \dfrac{DE}{EF} \juhao $$

利用合比性质,可得
$$ \dfrac{AB}{AC} = \dfrac{DE}{DF} \juhao $$


这样,我们就得到

\begin{dingli}[平行线分线段成比例定理]
    三条平行线截两条直线,所得的对应线段成比例。
\end{dingli}

如图 \ref{fig:czjh2-6-7},在 $\triangle ABC$ 中,已知 $DE \pingxing BC$。
过点 $A$ 作 $MN \pingxing DE$,依据上述定理得:
$$ \dfrac{AD}{DB} = \dfrac{AE}{EC}  \quad \text{或} \quad  \dfrac{AD}{AB} = \dfrac{AE}{AC} \juhao $$

这样,就得到

\begin{tuilun}[推论]
    平行于三角形一边的直线截其他两边,所得的对应线段成比例。
\end{tuilun}

利用比例的性质,从推论还可以得到图 \ref{fig:czjh2-6-7} 中对应线段的各种比例。
例如 $\dfrac{AD}{AE} = \dfrac{DB}{EC} = \dfrac{AB}{AC}$ 等。

\begin{figure}[htbp]
    \centering
    \begin{minipage}[b]{7cm}
        \centering
        \begin{tikzpicture}
    \tkzDefPoints{0/0/B, 3/0/C, 2/1.5/A}
    \tkzDefPoints{0/1.5/M, 3/1.5/N}
    \tkzDefPointOnLine[pos=0.6](A,B)  \tkzGetPoint{D}
    \tkzDefPointOnLine[pos=0.6](A,C)  \tkzGetPoint{E}

    \tkzDrawPolygon(A,B,C)
    \tkzDrawSegment(D,E)
    \tkzDrawSegment[dashed](M,N)
    \tkzLabelPoints[above](A)
    \tkzLabelPoints[left](B,D,M)
    \tkzLabelPoints[right](C,E,N)
\end{tikzpicture}


        \caption{}\label{fig:czjh2-6-7}
    \end{minipage}
    \qquad
    \begin{minipage}[b]{7cm}
        \centering
        \begin{tikzpicture}
    \pgfmathsetmacro{\a}{1.6}
    \pgfmathsetmacro{\b}{2}
    \pgfmathsetmacro{\c}{1.3}

    \begin{scope}[yshift=2cm]
        \tkzDefPoints{0/0/c1, \c/0/c2, 0/0.6/b1, \b/0.6/b2, 0/1.2/a1, \a/1.2/a2}
        % \tkzDrawSegments[xianduan={above=3pt,below=3pt}](a1,a2)
        % \tkzDrawSegments[xianduan={above=3pt,below=3pt}](b1,b2)
        % \tkzDrawSegments[xianduan={above=3pt,below=3pt}](c1,c2)
        % \tkzLabelSegment[above](a1,a2){$a$}
        \foreach \x in {a,b,c} {
            \tkzDrawSegments[xianduan={above=3pt,below=3pt}](\x1,\x2)
            \tkzLabelSegment[above](\x1,\x2){$\x$}
        }
    \end{scope}

    \begin{scope}
        % 1
        \tkzDefPoints{0/0/O, 4/0/M, 3.5/2/N}
        \tkzDrawSegments(O,M  O,N)
        \tkzLabelPoints[left](O)
        \tkzLabelPoints[right](M,N)

        % 2
        \tkzInterLC[R](O,M)(O,\a)  \tkzGetSecondPoint{A}
        \tkzInterLC[R](O,M)(A,\b)  \tkzGetSecondPoint{B}
        \tkzLabelSegment[below](O,A){$a$}
        \tkzLabelSegment[below](A,B){$b$}
        \tkzLabelPoints[below](A,B)

        \tkzInterLC[R](O,N)(O,\c)  \tkzGetSecondPoint{C}
        \tkzLabelSegment[above](O,C){$c$}
        \tkzLabelPoints[above](C)

        % 3
        \tkzDrawSegment(A,C)
        \tkzDefLine[parallel=through B](A,C)  \tkzGetPoint{d}
        \tkzInterLL(O,N)(B,d)  \tkzGetPoint{D}
        \tkzDrawSegment(B,D)
        \tkzLabelSegment[above](C,D){$x$}
        \tkzLabelPoints[above](D)
    \end{scope}
\end{tikzpicture}


        \caption{}\label{fig:czjh2-6-8}
    \end{minipage}
\end{figure}


\liti \zhongdian{作已知线段 $a$、$b$、$c$ 的第四比例项。}

已知:线段 $a$、$b$、$c$。

求作:线段 $x$,使 $a:b = c:x$。

\zuofa 如图 \ref{fig:czjh2-6-8}。

1. 作以点 $O$ 为端点的射线 $OM$ 和 $ON$。

2. 在 $OM$ 上依次截取 $OA = a$, $AB = b$; 在 $ON$ 上截取 $OC = c$。

3. 连结 $AC$。 过点 $B$ 作 $BD \pingxing AC$,交 $ON$ 于点 $D$。

$CD$ 就是所求的线段 $x$。

证明略。


\liti \begin{dingli}
    平行于三角形的一边,并且和其他两边相交的直线,所截得的三角形的三边与原三角形三边对应成比例。
\end{dingli}

已知: $\triangle ABC$ 中, $DE \pingxing BC$, 分别交 $AB$、$AC$ 于 $D$、$E$(图 \ref{fig:czjh2-6-9})。

求证: $\dfrac{AD}{AB} = \dfrac{AE}{AC} = \dfrac{AE}{BC}$。

分析:由本节定理的推论可以直接得到
$$ \dfrac{AD}{AB} = \dfrac{AE}{AC} \juhao $$

$\dfrac{DE}{BC}$ 中的 $DE$ 不在 $\triangle ABC$ 的边 $BC$ 上,我们不能直接利用前面所学的定理。
但从比例 $\dfrac{AD}{AB} = \dfrac{DE}{BC}$ 可以看出,除 $DE$ 以外,其他线段都在 $\triangle ABC$ 的边上,
因此,我们只要将 $DE$ 移到 $BC$ 边上去,得 $CF = DE$,然后再证明 $\dfrac{AD}{AB} = \dfrac{CF}{BC}$ 就可以了。
这只要过点 $D$ 作 $DF \pingxing AC$,交 $BC$ 于点 $F$, $CF$ 就是平移 $DE$ 所得的线段。

\zhengming 过点 $D$ 作 $DE \pingxing AC$,交 $BC$ 于点 $F$。

$\left. \begin{aligned}
    & \left. \begin{aligned}
        DE \pingxing BC \\
        DF \pingxing AC
    \end{aligned} \right\} \tuichu DE = FC \\
    & DF \pingxing AC  \tuichu \dfrac{AD}{AB} = \dfrac{FC}{BC}
\end{aligned} \; \right\} \tuichu \dfrac{AD}{AB} = \dfrac{DE}{BC} \juhao$

$DE \pingxing BC  \tuichu \dfrac{AD}{AB} = \dfrac{AE}{AC}$。

$\therefore$ \quad $\dfrac{AD}{AB} = \dfrac{AE}{AC} = \dfrac{DE}{BC}$。


\begin{figure}[htbp]
    \centering
    \begin{minipage}[b]{7cm}
        \centering
        \begin{tikzpicture}
    \tkzDefPoints{0/0/B, 3/0/C, 2/2/A}
    \tkzDefPointOnLine[pos=0.6](A,B)  \tkzGetPoint{D}
    \tkzDefPointOnLine[pos=0.6](A,C)  \tkzGetPoint{E}
    \tkzDefLine[parallel=through D](A,C)  \tkzGetPoint{f}
    \tkzInterLL(D,f)(B,C)  \tkzGetPoint{F}

    \tkzDrawPolygon(A,B,C)
    \tkzDrawSegment(D,E)
    \tkzDrawSegment[dashed](D,F)
    \tkzLabelPoints[above](A)
    \tkzLabelPoints[left](B,D)
    \tkzLabelPoints[right](C,E)
    \tkzLabelPoints[below](F)
\end{tikzpicture}


        \caption{}\label{fig:czjh2-6-9}
    \end{minipage}
    \qquad
    \begin{minipage}[b]{7cm}
        \centering
        \begin{tikzpicture}[scale=0.3] % 复杂
    % 原题中并没有给出三条平行线的间隔
    % 但 M 点能连接 A、B、C、D,所以将 M 点设置为原点。
    \tkzDefPoints{0/0/M, 1/0/m}
    % AM=3,所以 l_1 的 y 值小于3,这里任意取为 2.8
    % EF = 15, E在M点右上方,同时还得考虑:l_1 和 l_2 之间的间距要小于 l_2 和 l_3 之间的间距(原图中能看出)
    % 所以将 E 点的 x 坐标设置成 8
    \tkzDefPoints{0/2.8/xa1, 1/2.8/xa2, 8/2.8/E}
    \tkzInterLC[R](xa1,xa2)(M,3)    \tkzGetSecondPoint{A}  % AM = 3
    \tkzInterLC[R](xa1,xa2)(M,4.5)  \tkzGetSecondPoint{C}  % CM = 4.5
    \tkzDefPointOnLine[pos=8/3](A,M)  \tkzGetPoint{B}      % AB = AM + BM = 3 + 5 = 8
    \tkzDefLine[parallel=through B](M,m)  \tkzGetPoint{b}
    \tkzInterLC[R](B,b)(E,15)    \tkzGetFirstPoint{F}      % EF= 15
    \tkzInterLL(C,M)(B,b)  \tkzGetPoint{D}
    \tkzInterLL(E,F)(M,m)  \tkzGetPoint{K}

    \tkzDrawLine[add=6 and 9](xa1,xa2) \tkzLabelSegment[pos=12,left](xa1,xa2){$l_1$}
    \tkzDrawLine[add=6 and 9](M,m)     \tkzLabelSegment[pos=12,left](M,m){$l_2$}
    \tkzDrawLine[add=8 and 7](B,b)     \tkzLabelSegment[pos=10,left](B,b){$l_3$}
    \tkzDrawPoints(M,K)
    % \tkzDrawSegments(M,A  M,B  M,C  M,D  E,F)
    \tkzDrawLine[add=0.2 and 0.2](A,B)
    \tkzDrawLine[add=0.2 and 0.2](C,D)
    \tkzDrawLine[add=0.2 and 0.2](E,F)
    \tkzLabelPoints[above](C,E,F,D)
    \tkzLabelPoints[above, xshift=.5em](A)
    \tkzLabelPoints[below right](B)
    \tkzLabelPoints[below](K)
    \tkzLabelPoints[below left](M)
\end{tikzpicture}


        \caption*{(第 1 题)}
    \end{minipage}
\end{figure}


\begin{lianxi}

\xiaoti{已知:如图, $l_1 \pingxing l_2 \pingxing l_3$。
    $AM = 3$ 厘米,$BM = 5$ 厘米,$CM = 4.5$ 厘米,$EF = 15$ 厘米。
    求 $DM$、$EK$、$FK$ 的长。
}

\xiaoti{已知:线段 $a$、$b$。求作: 线段 $a$、$b$ 的第三比例线段
    (即求作 $x$,使 $a:b = b:x$)。
}

\xiaoti{平行于 $\triangle ABC$ 的边 $BC$ 的直线,与另两边 $AB$、$AC$
    (或 $BA$、$CA$)的延长线相交于点 $D$、$E$,画出图形,并说出图中所有的成比例线段。
}

\xiaoti{已知:如图 \ref{fig:czjh2-6-9}, $DE \pingxing BC$, $DF \pingxing AC$。
    判断下列比例是否正确,不对的加以改正:
}
\begin{xiaoxiaotis}

    \begin{tblr}{columns={12em, colsep=0pt}}
        \xxt{$\dfrac{AD}{BD} = \dfrac{DE}{BC}$;}
            & \xxt{$\dfrac{AE}{EC} = \dfrac{BF}{FC}$;}
            & \xxt{$\dfrac{DF}{AC} = \dfrac{DE}{BC}$。}
    \end{tblr}
\end{xiaoxiaotis}

\end{lianxi}

\end{enhancedline}

