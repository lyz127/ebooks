\subsection{直线和圆的位置关系}\label{subsec:czjh2-7-7}

在黑板上画一个圆,把直尺当做一条直线在黑板面上移动。我们可以看到,直线和圆的位置关系有下面三种。

\begin{figure}[htbp]
    \centering
    \begin{minipage}[b]{4.5cm}
        \centering
        \begin{tikzpicture}
    \tkzDefPoints{0/0/O}
    \tkzDefPoint(0:1.5){A}
    \tkzDefPoint(270:1.8){P}
    \tkzDefLine[perpendicular=through P,normed](O,P)  \tkzGetPoint{q}
    \tkzDefPointOnLine[pos=2.0](P,q)  \tkzGetPoint{Q}

    \tkzDrawCircle[thick](O,A)
    \tkzDrawPoint(O)
    \tkzDrawSegment(O,P)
    \tkzDrawLine[add=0 and 1](Q,P)
    \tkzMarkRightAngle[size=.2](O,P,Q)
    \tkzLabelPoints[above](O)
    \tkzLabelPoints[below](P)
    \tkzLabelSegment[pos=1, right](P,Q){$l$}
\end{tikzpicture}


        \caption*{甲}
    \end{minipage}
    \qquad
    \begin{minipage}[b]{4.5cm}
        \centering
        \begin{tikzpicture}
    \tkzDefPoints{0/0/O}
    \tkzDefPoint(0:1.5){A}
    \tkzDefPoint(270:1.5){P}
    \tkzDefLine[perpendicular=through P,normed](O,P)  \tkzGetPoint{q}
    \tkzDefPointOnLine[pos=2.0](P,q)  \tkzGetPoint{Q}

    \tkzDrawCircle[thick](O,A)
    \tkzDrawPoint(O)
    \tkzDrawSegment(O,P)
    \tkzDrawLine[add=0 and 1](Q,P)
    \tkzMarkRightAngle[size=.2](O,P,Q)
    \tkzLabelPoints[above](O)
    \tkzLabelPoints[below](P)
    \tkzLabelSegment[pos=1, right](P,Q){$l$}

    % 通过显示一个无色的 "P",实现三张图片的 “圆点” 在同一水平线。
    \tkzDefPoint(270:1.8){P}
    \tkzLabelPoints[below,transparent](P)
\end{tikzpicture}


        \caption*{乙}
    \end{minipage}
    \qquad
    \begin{minipage}[b]{4.5cm}
        \centering
        \begin{tikzpicture}
    \tkzDefPoints{0/0/O}
    \tkzDefPoint(0:1.5){A}
    \tkzDefPoint(270:1.0){P}
    \tkzDefLine[perpendicular=through P,normed](O,P)  \tkzGetPoint{q}
    \tkzDefPointOnLine[pos=2.0](P,q)  \tkzGetPoint{Q}

    \tkzDrawCircle[thick](O,A)
    \tkzDrawPoint(O)
    \tkzDrawSegment(O,P)
    \tkzDrawLine[add=0 and 1](Q,P)
    \tkzMarkRightAngle[size=.2](O,P,Q)
    \tkzLabelPoints[above](O)
    \tkzLabelPoints[below](P)
    \tkzLabelSegment[pos=1, right](P,Q){$l$}

    % 通过显示一个无色的 "P",实现三张图片的 “圆点” 在同一水平线。
    \tkzDefPoint(270:1.8){P}
    \tkzLabelPoints[below,transparent](P)
\end{tikzpicture}


        \caption*{丙}
    \end{minipage}
    \caption{}\label{fig:czjh2-7-32}
\end{figure}

(1)直线和圆没有公共点时,叫做直线和圆\zhongdian{相离}(图 \ref{fig:czjh2-7-32} 甲)。

(2)直线和圆有唯一公共点时,叫做直线和圆\zhongdian{相切}(图 \ref{fig:czjh2-7-32} 乙)。
这时直线叫做圆的\zhongdian{切线},唯一的公共点叫做\zhongdian{切点}。

(3)直线和圆有两个公共点时,叫做直线和圆\zhongdian{相交}(图 \ref{fig:czjh2-7-32} 丙)。
这时直线叫做圆的\zhongdian{割线}。

根据直线与圆相离、相切、相交的定义,容易看出:

如果 $\yuan\,O$ 的半径为 $r$, 圆心 $O$ 到直线 $l$ 的距离为 $d$,那么


\zhongdian{(1) 直线 $\bm{l}$ 和 $\bm{\yuan\,O}$ 相离 $\dengjiayu \bm{d > r}$;}

\zhongdian{(2) 直线 $\bm{l}$ 和 $\bm{\yuan\,O}$ 相切 $\dengjiayu \bm{d = r}$;}

\zhongdian{(3) 直线 $\bm{l}$ 和 $\bm{\yuan\,O}$ 相交 $\dengjiayu \bm{d < r}$。}


\begin{lianxi}

已知 $Rt \triangle ABC$ 的斜边 $AB = 6$ cm,直角边 $AC = 3$ cm。
圆心为 $C$,半径分别为 $2$ cm, $4$ cm 的两个圆与 $AB$ 有怎样的位置关系?
半径多长时, $AB$ 与圆相切?

\end{lianxi}

