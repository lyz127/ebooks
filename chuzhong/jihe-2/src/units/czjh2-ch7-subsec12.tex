\subsection{和圆有关的比例线段}\label{subsec:czjh2-7-12}

\begin{dingli}[相交弦定理]
    圆内的两条相交弦,被交点分成的两条线段长的积相等。
\end{dingli}

已知:弦 $AB$ 和 $CD$ 相交于 $\yuan\,O$ 内一点 $P$(图 \ref{fig:czjh2-7-44})。

求证:$PA \cdot PB = PC \cdot PD$。

\zhengming 连结 $AC$、$BD$。 由圆周角定理,得

$\begin{aligned}
    \left.\begin{aligned}
        \angle A = \angle D \\
        \angle C = \angle B
    \end{aligned}\right\}  &\tuichu \triangle PAC \xiangsi \triangle PDB \\
                           &\tuichu PA : PD = PC : PB \\
                           &\tuichu PA \cdot PB = PC \cdot PD \juhao
\end{aligned}$


由相交弦定理,可以得出下面的推论:

\begin{tuilun}[推论]
    如果弦与直径垂直相交,那么弦的一半是是它分直径所成的两条线段的比例中项。
\end{tuilun}

\begin{figure}[htbp]
    \centering
    \begin{minipage}[b]{4.5cm}
        \centering
        \begin{tikzpicture}
    \tkzDefPoints{0/0/O}
    \tkzDefPoint(120:1.5){A}
    \tkzDefPoint(350:1.5){B}
    \tkzDefPoint(240:1.5){C}
    \tkzDefPoint(40:1.5){D}
    \tkzInterLL(A,B)(C,D)  \tkzGetPoint{P}

    \tkzDrawCircle[thick](O,A)
    \tkzDrawPoint(O)
    \tkzDrawSegments(A,B  C,D)
    \tkzDrawSegments[dashed](A,C  B,D)
    \tkzLabelPoints[above](O)
    \tkzLabelPoints[above](A)
    \tkzLabelPoints[right](B)
    \tkzLabelPoints[below](C)
    \tkzLabelPoints[right](D)
    \tkzLabelPoints[below=.3em](P)
\end{tikzpicture}


        \caption{}\label{fig:czjh2-7-44}
    \end{minipage}
    \qquad
    \begin{minipage}[b]{4.5cm}
        \centering
        \begin{tikzpicture}
    \tkzDefPoints{0/0/O}
    \tkzDefPoint(180:1.5){A}
    \tkzDefPoint(0:1.5){B}
    \tkzDefPoint(40:1.5){C}
    \tkzDefPoint(-40:1.5){D}
    \tkzInterLL(A,B)(C,D)  \tkzGetPoint{P}

    \tkzDrawCircle[thick](O,A)
    \tkzDrawPoint(O)
    \tkzDrawSegments(A,B  C,D)
    \tkzMarkRightAngle[size=.2](B,P,C)
    \tkzLabelPoints[above](O)
    \tkzLabelPoints[left](A)
    \tkzLabelPoints[right](B)
    \tkzLabelPoints[above, xshift=.3em](C)
    \tkzLabelPoints[below, xshift=.3em](D)
    \tkzLabelPoints[below left](P)
\end{tikzpicture}


        \caption{}\label{fig:czjh2-7-45}
    \end{minipage}
    \qquad
    \begin{minipage}[b]{5cm}
        \centering
        \begin{tikzpicture}
    \pgfmathsetmacro{\a}{2}
    \pgfmathsetmacro{\b}{1.5}

    \begin{scope}[yshift=2.6cm]
        \tkzDefPoints{0/0/b1, \b/0/b2, 0/.7/a1, \a/.7/a2}
        \tkzDrawSegments[xianduan={below=0pt}](b1,b2  a1,a2)
        \tkzLabelSegment[above](a1,a2){$a$}
        \tkzLabelSegment[above](b1,b2){$b$}
    \end{scope}

    % 1
    \tkzDefPoints{0/0/A, \a/0/P}
    \tkzDrawSegment(A,P)
    \tkzLabelSegment[below](A,P){$a$}
    \tkzLabelPoints[left](A)
    \tkzLabelPoints[below](P)

    % 2
    \tkzDefPoints{\a+\b/0/B}
    \tkzDrawSegment(P,B)
    \tkzLabelSegment[below](P,B){$b$}
    \tkzLabelPoints[right](B)

    % 3
    \tkzDefMidPoint(A,B)  \tkzGetPoint{O}
    \tkzDrawPoint(O)
    \tkzDrawArc(O,B)(A)

    % 4
    \tkzDefLine[perpendicular=through P](A,B)  \tkzGetPoint{c}
    \tkzInterLC(P,c)(O,B)  \tkzGetSecondPoint{C}
    \tkzDrawSegment(P,C)
    \tkzLabelSegment[left](P,C){$c$}
    \tkzMarkRightAngle(B,P,C)
    \tkzLabelPoints[above](C)
\end{tikzpicture}


        \caption{}\label{fig:czjh2-7-46}
    \end{minipage}
\end{figure}


如图 \ref{fig:czjh2-7-45}, $CD$ 是弦, $AB$ 是直径,$CD \perp AB$,垂足是 $P$,则
$$ PC^2 = PA \cdot PB \juhao $$


\liti 已知:线段 $a$、$b$。

求作: 线段 $c$ ,使 $c^2 = ab$。

\zuofa 1. 作线段 $AP=a$ (图 \ref{fig:czjh2-7-46})。

2. 延长 $AP$ 到点 $B$,使 $PB = b$。

3. 以 $AB$ 为直径作半圆。

4. 过点 $P$ 作 $PC \perp AB$,交平圆于点 $C$。

$PC$ 就是 $a$、$b$ 的比例中项。

证明略。


\begin{lianxi}

\xiaoti{如图, $AP = 3$ 厘米, $PB = 5$ 厘米, $CP = 2.5 $ 厘米。 求 $CD$。}

\xiaoti{如图, $O$ 是圆心, $OP \perp AB$, $AP = 4$ 厘米, $PD = 2 $ 厘米。 求 $OP$。}

\end{lianxi}

\begin{figure}[htbp]
    \centering
    \begin{minipage}[b]{4cm}
        \centering
        \begin{tikzpicture} % 示意图,与所给数据无关
    \tkzDefPoints{0/0/O}
    \tkzDefPoint(205:1.5){A}
    \tkzDefPoint(335:1.5){B}
    \tkzDefPoint(240:1.5){C}
    \tkzDefPoint(40:1.5){D}
    \tkzInterLL(A,B)(C,D)  \tkzGetPoint{P}

    \tkzDrawCircle[thick](O,A)
    \tkzDrawPoint(O)
    \tkzDrawSegments(A,B  C,D)
    \tkzLabelPoints[above](O)
    \tkzLabelPoints[left](A)
    \tkzLabelPoints[right](B)
    \tkzLabelPoints[below left](C)
    \tkzLabelPoints[above right](D)
    \tkzLabelPoints[below](P)
\end{tikzpicture}


        \caption*{(第 1 题)}
    \end{minipage}
    \qquad
    \begin{minipage}[b]{4.5cm}
        \centering
        \begin{tikzpicture}
    \tkzDefPoints{0/0/O}
    \tkzDefPoint(180:1.5){C}
    \tkzDefPoint(0:1.5){D}
    \tkzDefPointOnLine[pos=2/10](D,C)  \tkzGetPoint{P}  % 由 AP=4, PD=2, 得 CP=8
    \tkzDefLine[perpendicular=through P](C,D)  \tkzGetPoint{a}
    \tkzInterLC(P,a)(O,C)  \tkzGetPoints{B}{A}

    \tkzDrawCircle[thick](O,A)
    \tkzDrawPoint(O)
    \tkzDrawSegments(A,B  C,D)
    \tkzMarkRightAngle[size=.2](D,P,A)
    \tkzLabelPoints[below](O)
    \tkzLabelPoints[above](A)
    \tkzLabelPoints[below](B)
    \tkzLabelPoints[left](C)
    \tkzLabelPoints[right](D)
    \tkzLabelPoints[below left](P)
\end{tikzpicture}


        \caption*{(第 2 题)}
    \end{minipage}
    \qquad
    \begin{minipage}[b]{6.0cm}
        \centering
        \begin{tikzpicture}
    \tkzDefPoints{0/0/O, -3/-0.4/P}
    \tkzDefPoint(-20:1.5){A}
    \tkzDefPoint(80:1.5){C}
    \tkzInterLC[common=A](P,A)(O,A)  \tkzGetFirstPoint{B}
    \tkzInterLC[common=C](P,C)(O,A)  \tkzGetFirstPoint{D}

    \tkzDefMidPoint(O,P)  \tkzGetPoint{Q}
    \tkzInterCC(O,A)(Q,P)  \tkzGetFirstPoint{T}

    \tkzDrawCircle[thick](O,A)
    \tkzDrawPoint(O)
    \tkzDrawLine[add=0 and 0.4](P,T)
    \tkzDrawSegments(P,A  P,C)
    \tkzDrawSegments[dashed](T,A  T,B)
    \extkzLabelAngel[0.8](B,T,P){$1$}

    \tkzLabelPoints[above](O)
    \tkzLabelPoints[right](A)
    \tkzLabelPoints[above, xshift=.5em](B)
    \tkzLabelPoints[above](C)
    \tkzLabelPoints[above left](D)
    \tkzLabelPoints[left](P)
    \tkzLabelPoints[below](T)
\end{tikzpicture}


        \caption{}\label{fig:czjh2-7-47}
    \end{minipage}
\end{figure}


\hspace{1em}

\begin{dingli}[切割线定理]
    从圆外一点引圆的切线和割线,切线长是这点到割线与圆交点的两条线段长的比例中项。
\end{dingli}


已知:点 $P$ 是 $\yuan\,O$ 外一点, $PT$ 是切线, $T$ 是切点,
$PA$ 是割线, 点 $A$、$B$ 是它与 $\yuan\,O$ 的交点(图 \ref{fig:czjh2-7-47})。

求证: $PT^2 = PA \cdot PB$。

\zhengming 连结 $TA$、$TB$。

$\left.\begin{aligned}
    \angle BPT = \angle TPA \\
    \angle 1 = \angle A
\end{aligned}\right\}  \tuichu  \triangle BPT \xiangsi \triangle TPA$

\quad $\tuichu PB : PT = PT : PA  \tuichu PT^2 = PA \cdot PB$。


\begin{tuilun}[推论]
    从圆外一点引圆的两条割线,这一点到每条割线与圆的交点的两条线段长的积相等。
\end{tuilun}

如图 \ref{fig:czjh2-7-47} 中, $PA \cdot PB = PC \cdot PD$。

\liti $\yuan\,O_1$、$\yuan\,O_2$、$\yuan\,O_3$、 … 都经过点 $A$ 和 $B$。
求证: 从线段 $AB$ 的延长线上任意一点向各圆引切线,切点在同一个圆上。

已知:如图 \ref{fig:czjh2-7-48},$\yuan\,O_1$、$\yuan\,O_2$、$\yuan\,O_3$、… 都经过点 $A$ 和 $B$。
点 $P$ 是线段 $AB$ 的延长线上任意一点, 且 $PC$、$PD$、$PE$、… 分别与
$\yuan\,O_1$、$\yuan\,O_2$、$\yuan\,O_3$、 … 相切于点 $C$、$D$、$E$、…。

求证: $C$、$D$、$E$、… 在同一个圆上。

\zhengming $\because$ \quad $PC$ 是 $\yuan\,O_1$ 的切线,$PA$ 是 $\yuan\,O_1$ 的割线,

$\therefore$ \quad $PC^2 = PA \cdot PB$。

同理 $PD^2 = PA \cdot PB$, $PE^2 = PA \cdot PB$, …。

$\therefore$ \quad $PC = PD = PE = \cdots$。

$\therefore$ \quad $C$、$D$、$E$、… 都在以点 $P$ 为圆心, $PC$ 为半径的圆上。

\begin{figure}[htbp]
    \centering
    \begin{minipage}[b]{7cm}
        \centering
        \begin{tikzpicture}
    \tkzDefPoints{0/0.5/B, 0/-0.5/A, -1/0/O_1, 0.5/0/O_2, 1.3/0/O_3}

    \tkzDrawCircle[thick](O_1,A)
    \tkzDrawCircle[thick](O_2,A)
    \tkzDrawCircle[thick](O_3,A)
    \tkzDrawPoints(O_1, O_2, O_3, A, B)
    \tkzLabelPoints[left=.5em](A)
    \tkzLabelPoints[left=.5em](B)
    \tkzLabelPoints[below](O_1, O_2)
    \tkzLabelPoints[below, xshift=.2em](O_3)

    % 过点P作切线
    \tkzDefPointOnLine[pos=2.5](A,B)  \tkzGetPoint{P}
    \tkzDrawLine[add=0 and 0.1](A,P)
    \tkzLabelPoints[left](P)

    \tkzDefMidPoint(O_1,P)  \tkzGetPoint{Q}
    \tkzInterCC(O_1,A)(Q,P)  \tkzGetFirstPoint{C}
    \tkzDrawPoint(C)
    \tkzDrawLine[add=0 and 0.3](P,C)
    \tkzLabelPoints[above](C)

    \tkzDefMidPoint(O_2,P)  \tkzGetPoint{Q}
    \tkzInterCC(O_2,A)(Q,P)  \tkzGetSecondPoint{D}
    \tkzDrawPoint(D)
    \tkzDrawLine[add=0 and 0.3](P,D)
    \tkzLabelPoints[above, xshift=.2em](D)

    \tkzDefMidPoint(O_3,P)  \tkzGetPoint{Q}
    \tkzInterCC(O_3,A)(Q,P)  \tkzGetSecondPoint{E}
    \tkzDrawPoint(E)
    \tkzDrawLine[add=0 and 0.3](P,E)
    \tkzLabelPoints[above](E)
\end{tikzpicture}


        \caption{}\label{fig:czjh2-7-48}
    \end{minipage}
    \qquad
    \begin{minipage}[b]{7cm}
        \centering
        \begin{tikzpicture}
    \tkzDefPoints{0/0/B, 4/0/C, 4/3/A}
    \tkzDefMidPoint(A,C)  \tkzGetPoint{O}
    \tkzInterLC[common=A](B,A)(O,A)   \tkzGetFirstPoint{D}

    \tkzDrawPolygon(A,B,C)
    \tkzDrawCircle[thick](O,A)
    \tkzDrawPoint(O)
    \tkzLabelPoints[above](A)
    \tkzLabelPoints[left](B)
    \tkzLabelPoints[below](C)
    \tkzLabelPoints[above left](D)
    \tkzLabelPoints[right](O)
\end{tikzpicture}


        \caption*{(第 1 题)}
    \end{minipage}
\end{figure}


\begin{lianxi}

\xiaoti{已知:$Rt \triangle ABC$ 的两条直角边 $AC$、$BC$ 的长分别为 3 cm、4 cm。
    以 $AC$ 为直径作圆与斜边 $AB$ 交于点 $D$。 求 $BD$ 的长。
}

\xiaoti{运用 “切割线定理”,作已知线段 $a$、$b$ 的比例中项。}

\end{lianxi}

