\subsection{正多边形作图}\label{subsec:czjh2-7-18}

\begin{enhancedline}

半径为 $R$ 的正多边形的作图问题,实际上是它的外接圆的等分问题。
把圆 $n$ 等分后,顺次连结各分点,就得到正 $n$ 边形。

由于正 $n$ 边形的中心角 $\alpha_n = \dfrac{360^\circ}{n}$, 使用量角器,
作出中心角 $\alpha_n$, 就可以把圆分成 $n$ 等份,从而作出半径为 $R$ 的正 $n$ 边形。

又因为正 $n$ 边形的边长 $a_n = 2 R \sin \dfrac{180^\circ}{n}$,使用正弦函数表,
可以算出半径为 $R$ 的正 $n$ 边形的边长 $a_n$,用刻度尺和圆规也可以把圆分成 $n$ 等份,
再作出半径为 $R$ 的正 $n$ 边形。 《中学数学用表》上还有 “等分圆周表”,
给出了直径为 1 的圆的内接正三角形至内接正一百边形的边长,
即给出了 $2R = 1$ 时, $n$ 为 3 ~\, 100 的 $\alpha_n$ 的值。
实际上,这些值就是 $\sin\dfrac{180^\circ}{n}$ 的值。

用上面的方法作出的正 $n$ 边形,都是近似的,但对于一些特殊的正 $n$ 边形,
还可以使用直尺和圆规来作准确的图形。

(1) 正四、八、… 边形的作法。

\begin{figure}[htbp]
    \centering
    \begin{minipage}[b]{7cm}
        \begin{tikzpicture}
    \pgfmathsetmacro{\R}{1.5}
    \tkzDefPoints{0/0/O}
    \tkzDefPoint(180:\R){A}
    \tkzDefPoint(270:\R){B}
    \tkzDefPoint(0:\R){C}
    \tkzDefPoint(90:\R){D}

    \tkzDrawCircle[very thick](O,A)
    \tkzDrawPolygon[red, thick](A,B,C,D)
    \tkzDrawSegments(A,C  B,D)
    \tkzLabelPoints[above right](O)
    \tkzLabelPoints[left](A)
    \tkzLabelPoints[below](B)
    \tkzLabelPoints[right](C)
    \tkzLabelPoints[above](D)
\end{tikzpicture}


    \end{minipage}
    \begin{minipage}[b]{7cm}
        \begin{tikzpicture}
    \pgfmathsetmacro{\R}{1.5}
    \tkzDefPoints{0/0/O}
    \tkzDefPoint(180:\R){A}
    \tkzDefPoint(270:\R){B}
    \tkzDefPoint(0:\R){C}
    \tkzDefPoint(90:\R){D}
    \tkzDefPoint(225:\R){E}
    \tkzDefPoint(315:\R){F}
    \tkzDefPoint(45:\R){G}
    \tkzDefPoint(135:\R){H}

    \tkzDrawCircle[very thick](O,A)
    \tkzDrawPolygon[red, thick](A,E,B,F,C,G,D,H)
    \tkzDrawSegments(A,C  B,D  E,G  F,H)
    \tkzLabelPoints[above=.4em, xshift=.4em](O)
    \tkzLabelPoints[left](A)
    \tkzLabelPoints[below](B)
    \tkzLabelPoints[right](C)
    \tkzLabelPoints[above](D)
    \tkzLabelPoints[below left](E)
    \tkzLabelPoints[below right](F)
    \tkzLabelPoints[above right](G)
    \tkzLabelPoints[above left](H)
\end{tikzpicture}


    \end{minipage}
    \caption{}\label{fig:czjh2-7-70}
\end{figure}


如图 \ref{fig:czjh2-7-70}, 在 $\yuan\,O$ 中,用直尺和圆规作两条互相垂直的直径,就可以作出正四边形。
再逐次作所成的中心角的平分线,还可以作出正八、十六、… 边形。

(2) 作正六、三、十二、… 边形。

\begin{figure}[htbp]
    \centering
    \begin{minipage}[b]{5cm}
        \begin{tikzpicture}
    \pgfmathsetmacro{\R}{1.5}
    \tkzDefPoints{0/0/O}
    \tkzDefPoint(240:\R){A}
    \tkzDefRegPolygon[center,sides=6,name=P](O,A)
    \foreach \P [count=\i from 2] in {B,C,...,F} {
        \coordinate (\P) at (P\i);
    }

    \tkzDrawCircle[very thick](O,A)
    \tkzDrawPolygon[red, thick](A,...,F)
    \tkzDrawSegments(C,F)
    \tkzDrawArc(F,A)(E)
    \tkzDrawArc(C,D)(B)
    \tkzLabelPoints[above right](O)
    \tkzAutoLabelPoints[center=O, centered, dist= .25](A,B,...,F)
\end{tikzpicture}

% \begin{tikzpicture}
%     \pgfmathsetmacro{\R}{1.5}
%     \tkzDefPoints{0/0/O}
%     \tkzDefPoint(0:\R){C}
%     \tkzDefPoint(60:\R){D}
%     \tkzDefPoint(120:\R){E}
%     \tkzDefPoint(180:\R){F}
%     \tkzDefPoint(240:\R){A}
%     \tkzDefPoint(300:\R){B}

%     \tkzDrawCircle[very thick](O,A)
%     \tkzDrawPolygon[red, thick](A,...,F)
%     \tkzDrawSegments(C,F)
%     \tkzDrawArc(F,A)(E)
%     \tkzDrawArc(C,D)(B)
%     \tkzLabelPoints[above right](O)
%     \tkzLabelPoints[below left](A)
%     \tkzLabelPoints[below right](B)
%     \tkzLabelPoints[right](C)
%     \tkzLabelPoints[above right](D)
%     \tkzLabelPoints[above left](E)
%     \tkzLabelPoints[left](F)
% \end{tikzpicture}


    \end{minipage}
    \begin{minipage}[b]{5cm}
        \begin{tikzpicture}
    \pgfmathsetmacro{\R}{1.5}
    \tkzDefPoints{0/0/O}
    \tkzDefPoint(240:\R){A}
    \tkzDefRegPolygon[center,sides=6,name=P](O,A)
    \foreach \P [count=\i from 2] in {B,C,...,F} {
        \coordinate (\P) at (P\i);
    }

    \tkzDrawCircle[very thick](O,A)
    \tkzDrawPolygon[red, thick](A,C,E)
    \tkzDrawPoint(O)
    \foreach \P in {B, D, F} {
        \tkzDefPointOnLine[pos=0.1](\P, O)  \tkzGetPoint{a}
        \tkzDefPointOnLine[pos=2](a, \P)  \tkzGetPoint{b}
        \tkzDrawSegment(a,b)
    }
    \tkzLabelPoints[below](O)
    \tkzAutoLabelPoints[center=O, centered, dist= .25](A,B,...,F)
\end{tikzpicture}


    \end{minipage}
    \begin{minipage}[b]{5cm}
        \begin{tikzpicture}
    \pgfmathsetmacro{\R}{1.5}
    \tkzDefPoints{0/0/O}
    \tkzDefPoint(0:\R){C}
    \tkzDefRegPolygon[center,sides=12,name=P](O,C)

    \tkzDrawCircle[very thick](O,C)
    \tkzDrawPolygon[red, thick](P1,P...,P12)
    \tkzDrawSegments[dashed](C,P7  P4,P10)
    \foreach \n in {30, 60, 120, 150, 210, 240, 300, 330} {
        \tkzDefPoint(\n:\R){P}
        \tkzDefPointOnLine[pos=0.1](P, O)  \tkzGetPoint{a}
        \tkzDefPointOnLine[pos=2](a, P)  \tkzGetPoint{b}
        \tkzDrawSegment(a,b)
    }
    \tkzLabelPoints[below left](O)
    \tkzLabelPoint[below left](P9){$A$}
    \tkzLabelPoint[below right](P11){$B$}
    \tkzLabelPoints[right](C)
\end{tikzpicture}


    \end{minipage}
    \caption{}\label{fig:czjh2-7-71}
\end{figure}

正六边形的边长 $a_6 = R$, 从图 \ref{fig:czjh2-7-71} 中容易看出,用直尺圆规可以作出正六、三、十二、… 边形。


(3) 作正十、五边形。

\begin{figure}[htbp]
    \centering
    \begin{minipage}[b]{7cm}
        \centering
        \begin{tikzpicture}
    \pgfmathsetmacro{\R}{1.5}
    \tkzDefPoints{0/0/O}
    \tkzDefPoint(0:\R){T1}
    \tkzDrawCircle[very thick](O,T1)

    \tkzDefRegPolygon[center,sides=10,name=P](O,T1)
    \tkzDrawPolygon[thick,red](P1,P...,P10)

    \coordinate (A) at (P8);
    \coordinate (B) at (P9);
    \coordinate (C) at (P10);

    \foreach \n in {-36, 0, ..., 216} {
        \tkzDefPoint(\n:\R){P}
        \tkzDefPointOnLine[pos=0.1](P, O)  \tkzGetPoint{a}
        \tkzDefPointOnLine[pos=2](a, P)  \tkzGetPoint{b}
        \tkzDrawSegment(a,b)
    }


    \tkzDefLine[bisector](O,B,A)  \tkzGetPoint{m}
    \tkzInterLL(B,m)(O,A)  \tkzGetPoint{M}
    \tkzDrawSegments[dashed](O,A  O,B  B,M)

    \tkzMarkAngle[size=.3](A,O,B)
    \tkzDefPoint(270:.3){x}
    \tkzDefShiftPoint[x](-.3,.2){y}
    \tkzDefShiftPoint[y](-.4,0){z}
    \tkzDrawSegment[green](x,y)
    \tkzDrawSegment[green,thick](y,z)
    \tkzLabelSegment[above](y,z){\small $36^\circ$}

    \tkzMarkAngle[size=.3](O,B,M)
    \tkzDefShiftPoint[B](120:.3){x}
    \tkzDefShiftPoint[x](.3,.5){y}
    \tkzDefShiftPoint[y](.4,0){z}
    \tkzDrawSegment[green](x,y)
    \tkzDrawSegment[green,thick](y,z)
    \tkzLabelSegment[above](y,z){\small $36^\circ$}

    \tkzMarkAngle[size=.2](B,A,O)
    \tkzDefShiftPoint[A](45:.2){x}
    \tkzDefShiftPoint[x](-.4,.5){y}
    \tkzDefShiftPoint[y](-.4,0){z}
    \tkzDrawSegment[green](x,y)
    \tkzDrawSegment[green,thick](y,z)
    \tkzLabelSegment[above](y,z){\small $72^\circ$}

    \tkzLabelPoints[above](O)
    \tkzLabelPoints[below](A)
    \tkzLabelPoints[below](B)
    \tkzLabelPoints[below right](C)
    \tkzLabelPoints[xshift=-.5em, yshift=1.2em](M)
\end{tikzpicture}


        \caption{}\label{fig:czjh2-7-72}
    \end{minipage}
    \qquad
    \begin{minipage}[b]{7cm}
        \centering
        \begin{tikzpicture}
    \pgfmathsetmacro{\R}{1.5}
    \pgfmathsetmacro{\halfR}{\R/2}

    \tkzDefPoints{0/0/O}
    \tkzDefPoint(270:\R){A}
    \tkzDrawCircle[very thick](O,A)

    \tkzDefRegPolygon[center,sides=10,name=P](O,A)
    \tkzDrawPolygon[thick,red](P1,P...,P10)
    \foreach \P [count=\i from 2] in {B,C,...,J} {
        \coordinate (\P) at (P\i);
    }
    \tkzAutoLabelPoints[center=O, centered, dist= .2](A,B,...,J)
    \tkzLabelPoints[above](O)

    %------ 黄金分割
    \tkzDrawSegment[dashed](O,A)

    % 1
    \tkzDefPointOnLine[pos=0.5](O,A)  \tkzGetPoint{p1}
    \tkzDefLine[perpendicular=through A](O,A)  \tkzGetPoint{p2}
    \tkzInterLC(A,p2)(A,p1)  \tkzGetFirstPoint{P}
    \tkzDrawSegment[dashed](A,P)
    \tkzLabelPoints[below](P)

    % 2
    \tkzDrawSegment[dashed](O,P)
    \tkzInterLC(O,P)(P,A)  \tkzGetFirstPoint{N}
    \tkzDrawArc[towards](P,A)(N)
    % \tkzDrawPoint(N)

    % 3
    \tkzInterLC(O,A)(O,N)  \tkzGetSecondPoint{M}
    \tkzDrawArc[towards](O,N)(M)
    \tkzDrawPoint(M)
    \tkzLabelPoints[right](M)
\end{tikzpicture}


        \caption{}\label{fig:czjh2-7-73}
    \end{minipage}
\end{figure}

假设正十边形已经作出,如图 \ref{fig:czjh2-7-72}, $AB$ 是 $\yuan\,O$ 内接正十边形的一边,
连结 $OA$、$OB$ 得到顶角 $\alpha_{10} = 36^\circ$,
底角为 $\dfrac{180^\circ - 36^\circ}{2} = 72^\circ$ 的等腰三角形 $OAB$,
作底角的平分线 $BM$ 交 $OA$ 于 $M$, 可得到 $\angle ABM = \angle MBO = \angle AOB = 36^\circ$,
所以 $OM = MB$。 由 $\angle AMB = \angle MAB = 72^\circ$,得 $MB = AB$, 所以 $OM = AB$。
因为 $\triangle OAB \xiangsi \triangle BAM$, 得到 $OA : AB = BA : AM$,即
$$ OA : OM = OM : AM \juhao $$
所以 $M$ 是半径 $OA$ 的黄金分割点, $OM$ 等于正十边形的边长。

把圆的半径 $OA$ 作黄金分割, 如图 \ref{fig:czjh2-7-73}, $OM$ 就等于正十边形的边长。
以 $OM$ 为半径作弧等分圆周,顺次连结各分点,就得到正十边形 $ABC\cdots J$。

图 \ref{fig:czjh2-7-73} 中, 分点 $A$、$C$、$E$、$G$、$I$ 把 $\yuan\,O$ 分成 5 等份。
顺次连结各点,就得正五边形 $ACEGI$。

把圆分成 5 等份,还可以用下面作法:

如图 \ref{fig:czjh2-7-74} 甲,作已知圆 $O$ 的互相垂直的直径 $XY$ 和 $AZ$;取半径 $OX$ 的中点 $M$;
以 $M$ 为圆心, $MA$ 为半径作弧 $\yuanhu{AN}$, 和半径 $OY$ 相交于 $N$;
在 $\yuan\,O$ 上连续截取等弧, 使弦 $AB = BC = CD = DE = AN$;
则 $A$、$B$、$C$、$D$、$E$ 就是所求作的分点。

把圆分成了 5 等份,还可以作出如图 \ref{fig:czjh2-7-74} 乙所示的正五角星形。

\begin{figure}[htbp]
    \centering
    \begin{minipage}[b]{10cm}
        \centering
        \begin{minipage}[b]{5cm}
            \centering
            \begin{tikzpicture}
    \pgfmathsetmacro{\R}{1.5}
    \tkzDefPoints{0/0/O}
    \tkzDefPoint(90:\R){A}
    \tkzDefRegPolygon[center,sides=5,name=P](O,A)
    \foreach \P [count=\i from 2] in {E,D,...,B} {
        \coordinate (\P) at (P\i);
    }
    \tkzDefPoint(180:\R){X}
    \tkzDefPoint(0:\R){Y}
    \tkzDefPoint(270:\R){Z}

    \tkzDrawCircle[very thick](O,A)
    \tkzDrawPolygon[red, thick](A,B,...,E)
    \tkzDrawSegments(A,Z  X,Y)
    \tkzMarkRightAngle[size=.2](A,O,X)

    \tkzLabelPoints[above right](O)
    \tkzAutoLabelPoints[center=O, centered, dist= .25](A,B,...,E,X,Y,Z)

    %
    \tkzDefMidPoint(O,X)  \tkzGetPoint{M}
    \tkzInterLC(O,Y)(M,A)  \tkzGetSecondPoint{N}
    \tkzDrawSegment[dashed](M,A)
    \tkzDrawArc(M,N)(A)
    \tkzLabelPoints[below](M,N)

    % \tkzInterCC(A,N)(O,A)  \tkzGetFirstPoint{B}
    \tkzDrawSegment[dashed](A,N)
    \tkzDrawArc(A,N)(B)
\end{tikzpicture}


            \caption*{甲}
        \end{minipage}
        \begin{minipage}[b]{4.5cm}
            \centering
            \begin{tikzpicture}
    \pgfmathsetmacro{\R}{1.5}
    \tkzDefPoints{0/0/O}
    \tkzDefPoint(90:\R){A}
    \tkzDefRegPolygon[center,sides=5,name=P](O,A)
    \foreach \P [count=\i from 2] in {E,D,...,B} {
        \coordinate (\P) at (P\i);
    }

    \tkzDrawCircle[very thick](O,A)
    \tkzDrawPolygon[red, thick](A,C,E,B,D)

    \tkzDefPoint(270:\R){Z}
    \tkzLabelPoints[white,below](Z)
\end{tikzpicture}


            \caption*{乙}
        \end{minipage}
        \caption{}\label{fig:czjh2-7-74}
    \end{minipage}
    \begin{minipage}[b]{4.5cm}
        \centering
        \begin{tikzpicture}
    \pgfmathsetmacro{\R}{1.5}
    \tkzDefPoints{0/0/O}
    \tkzDefPoint(90:\R){A}
    \tkzDefRegPolygon[center,sides=6,name=P](O,A)
    \foreach \P [count=\i from 2] in {B,C,...,F} {
        \coordinate (\P) at (P\i);
    }
    \tkzLabelPoints[above, xshift=.5em](O)
    \tkzAutoLabelPoints[center=O, centered, dist= .25](A,B,...,F)

    % 1
    \tkzDefPoint(180:\R){M}
    \tkzDefPoint(0:\R){N}
    \tkzDrawCircle[very thick](O,A)
    \tkzDrawSegments(A,D  M,N)
    \tkzLabelPoints[above, xshift=.6em](M)
    \tkzLabelPoints[below, xshift=-.6em](N)

    % 2
    \tkzInterLL(D,B)(M,N)  \tkzGetPoint{O_1}
    \tkzInterLL(D,F)(M,N)  \tkzGetPoint{O_2}
    \tkzDrawSegments(D,B  D,F)
    \tkzLabelPoints[above, xshift=.4em](O_1)
    \tkzLabelPoints[above, xshift=-.4em](O_2)

    % 3
    \tkzCalcLength(O_1,B)    \tkzGetLength{rOB}
    \tkzDrawArc[red, thick, R with nodes](O_1,\rOB)(B,C)  %\tkzDrawArc(O_1,B)(C)
    \tkzDrawArc[red, thick, R with nodes](O_2,\rOB)(E,F)  %\tkzDrawArc(O_2,E)(F)

    \tkzCalcLength(D,B)    \tkzGetLength{rDB}
    \tkzDrawArc[red, thick, R with nodes](A,\rDB)(C,E)
    \tkzDrawArc[red, thick, R with nodes](D,\rDB)(F,B)
\end{tikzpicture}


        \caption*{}
        \caption*{(第 3 题)}
    \end{minipage}
\end{figure}


\begin{lianxi}

\xiaoti{运用量角器作半径 $R = 4$ 厘米的正七边形和正九边形。
    再运用三角函数表计算出它们的边长,进行检验。
}

\xiaoti{按照图 \ref{fig:czjh2-7-74} 甲所示的作法,在半径为 3 厘米的圆中作出如图 \ref{fig:czjh2-7-74} 乙所示的正五角星形。}

\xiaoti{如图画近似椭圆。 先将 $\yuan\,O$ 六等分,分点为 $A$、$B$、$C$、$D$、$E$、$F$,
    作互相垂直的直径 $AD$、$MN$, 连结 $DB$、$DF$ 与 $MN$ 分别相交于 $O_1$、$O_2$,
    分别以 $O_1$、$O_2$ 为圆心,以 $O_1B$ 为半径作 $\yuanhu{BC}$、$\yuanhu{EF}$;
    分别以 $A$、$D$ 为圆心, 以 $DB$ 为半径作 $\yuanhu{CE}$、$\yuanhu{BF}$,
    四条环连接成一个椭圆。
}

\end{lianxi}

\end{enhancedline}
