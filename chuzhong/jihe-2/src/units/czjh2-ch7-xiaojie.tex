\xiaojie

一、本章研究圆的有关知识。 主要内容有:圆的概念和性质,圆与点、圆与直线、圆与圆、
圆与角以及圆与三角形、四边形、正多边形的位置关系,以及它们的应用。
同时介绍了四种命题的关系、轨迹的概念和常见的六种轨迹。


二、圆是到定点的距离等于定长的点的集合。不在同一直线上的三点确定一个圆。

圆是轴对称图形,而且它的任意一条直径所在的直线都是对称轴。
圆也是中心对称图形,并且绕圆心旋转任意大小的角度,都能够与原图形重合。
由圆的对称性,可得出圆的有关性质:垂直于弦的直径必平分弦;
在同圆和等圆中,两个圆心角、圆心角所对的弧、弦、弦心距中任何一对量相等时,其余对应的量也都相等。


三、由圆心到直线的距离与半径的大小关系,能够确定直线与圆的位置关系。
特别是当圆心到直线的距离等于半径时,直线与圆相切。
圆的切线垂直于过切点的半径(逆命题也正确),从圆外一点引圆的两条切线,切线长相等。

由圆心距与半径的大小关系,能够确定圆与圆的位置关系。
两圆相交时,连心线垂直平分公共弦;
两圆相切时,连心线经过切点。
两圆的外(内)公切线长相等。

由角的顶点在圆心、圆上以及一边与圆相切等不同的情形,分别得到圆心角、圆周角、弦切角。
圆周角等于同弧所对的圆心角的一半,弦切角等于它所夹的弧所对的圆周角。

三角形有且只有一个外接圆和一个内切圆。
圆的内接四边形对角互补,外角等于它的内对角。
圆的外切四边形两组对边的和相等。
正多边形必有外接圆和内切圆。
利用正多边形与圆的关系,可以求得圆的周长和面积公式,从而得到弧长和扇形面积公式。


四、从一点引两条直线与圆相交,直线被这一点和交点分成一些比例线段,
有相交弦定理、切割线定理和推论。


五、轨迹是几何中一个很重要的概念。
当图形 $F$ 上的每一个点都符合某个条件 $C$;
符合某个条件 $C$ 的每一个点都在图形 $F$上时,
图形 $F$ 就是符合某个条件 $C$ 的点的轨迹(或集合)。

原命题与它的逆否命题是等价命题。
在证明轨迹问题时,常用证明逆否命题来代替证明原命题。


六、反证法是一种间接证明命题的方法。
当命题不易用直接证法证明时,常用反证法。
用反证法证明时,首先否定命题的结论,由此推出矛盾,从而肯定命题的结论正确。

