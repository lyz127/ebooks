\subsection{四种命题的关系}\label{subsec:czjh2-7-21}

我们知道,判断一件事情的语句叫做命题。一个命题是由题设和结论两部分组成的。
命题有真有假,正确的命题是真命题,错误的命题是假命题。

另外,我们还研究过命题之间的关系。
例如,交换一个命题的题设和结论得到的新命题,与原命题是互逆命题。
下面,我们再进一步研究命题之间的另外几种关系。先看如下两组命题。

1. 设 $a$、$b$ 为实数,

\quad(1)如果 $a=0$, 那么 $ab = 0$;

\quad(2)如果 $ab = 0$, 那么 $a = 0$;

\quad(3)如果 $a \neq 0$, 那么 $ab \neq 0$;

\quad(4)如果 $ab \neq 0$,那么 $a \neq 0$。

2. (1)内接于圆的四边形的对角互补;

\quad(2)对角互补的四边形内接于圆;

\quad(3)不内接于圆的四边形的对角不互补;

\quad(4)对角不互补的四边形不内接于圆。

以上各组中的四个命题的题设和结论之间有下面的相互关系:

在(1)和(2)两个命题中,一个命题的题设和结论分别是另一个命题的结论和题设。
我们已知,这样的两个命题叫做\zhongdian{互逆命题}。
把其中的一个叫做\zhongdian{原命题}时,另一个就叫做它的\zhongdian{逆命题}。

在(1)和(3)两个命题中,一个命题的题设和结论分别是另一个命题的题设的否定和结论的否定。
这样的两个命题叫做\zhongdian{互否命题}。
把其中的一个叫做原命题时,另一个就叫做它的\zhongdian{否命题}。

在(1)和(4)两个命题中,一个命题的题设和结论分别是另一个命题的结论的否定和题设的否定。
这样的两个命题叫做\zhongdian{互为逆否命题}。
把其中的一个叫做原命题时,另一个就叫做它的\zhongdian{逆否命题}。

用 $A$ 和 $B$ 分别表示原命题的题设和结论, 用 $\buji{A}$ 和 $\buji{B}$
分别表示 $A$ 和 $B$ 的否定时,四种命题的形式就是:

原命题 \quad 若 $A$ 成立则 $B$ 就成立,或 “$A \tuichu B$”;

逆命题 \quad 若 $B$ 成立则 $A$ 就成立,或 “$B \tuichu A$”;

否命题 \quad 若 $A$ 不成立则 $B$ 不成立,或 “$\buji{A} \tuichu \buji{B}$”;

逆否命题 \quad 若 $B$ 不成立则 $A$ 不成立,或 “$\buji{B} \tuichu \buji{A}$”。

互逆命题、互否命题、互为逆否命题都是说两个命题之间的关系,
把其中一个命题叫做原命题时,另一个就叫做它的逆命题、否命题、逆否命题。因此,
同一个命题的否命题和逆否命题也是互逆的;
同一个命题的逆命题和逆否命题也是互否的;
同一个命题的逆命题和否命题也是互为逆否的。

四种命题之间的这些形式上的相互关系,如图 \ref{fig:czjh2-7-80} 。

\begin{figure}[htbp]
    \centering
    \begin{tikzpicture}[>=Stealth,
    box/.style = {draw, rectangle, align=center, minimum width=2cm},
]
    \node (YMT)  [box] at (0,3) {原命题   \\ $A \tuichu B$};
    \node (NMT)  [box] at (5,3) {逆命题   \\ $B \tuichu A$};
    \node (FMT)  [box] at (0,0) {否命题   \\ $\buji{A} \tuichu \buji{B}$};
    \node (NFMT) [box] at (5,0) {逆否命题 \\ $\buji{B} \tuichu \buji{A}$};

    \draw [<->] (YMT.east) -- (NMT.west)  node [midway, above] {互逆};
    \draw [<->] (FMT.east) -- (NFMT.west) node [midway, below] {互逆};

    \draw [<->] (YMT.south) -- (FMT.north)
        node [midway, left, align=center] {互\\[-0.5em]否}
    ;
    \draw [<->] (NMT.south) -- (NFMT.north)
        node [midway, right, align=center] {互\\[-0.5em]否}
    ;

    \draw [<->] (YMT.south east) -- (NFMT.north west)
        node [pos=.2, above, rotate=-30] {互为}
        node [pos=.8, above, rotate=-30] {逆否}
    ;
    \draw [<->] (FMT.north east) -- (NMT.south west)
        node [pos=.2, above, rotate=30] {互为}
        node [pos=.8, above, rotate=30] {逆否}
    ;
\end{tikzpicture}


    \caption{}\label{fig:czjh2-7-80}
\end{figure}


现在来研究一个命题的真假与其他三种命题的真假的关系。

我们知道,\zhongdian{原命题正确,它的逆命题不一定同时正确。}
例如,虽然上面第2组中,(1)、(2)同时正确,但是上面第1组中,(1)正确,(2)不正确。
同样,可以看到,第2组中,(1)、(3)同时正确;而第1组中,(1)正确,(3)不正确。
可见,\zhongdian{原命题正确,它的否命题也不一定同时正确。}
因此,除非经过了另外的证明,我们不能够根据某一个证明是正确的命题,
去断定这个命题的逆命题或否命题是否正确。

但是,一个命题的真假和它的逆否命题的真假却有特殊的关系。
上面第1组的(1)和(4)同时正确;第2组的(1)和(4)也同时正确。这里有必然的联系。

如果原命题“若 $A$ 成立,则 $B$ 就成立”正确,那么 $B$ 不成立时,试想 $A$ 成立不成立呢?当然 $A$ 不能成立。
因为,假定 $A$ 成立,那么根据正确的原命题, $B$ 就应成立,这和这里的题设 $B$ 不成立相矛后。
因此,“若 $B$ 不成立,则 $A$ 不成立”。这就证明了:

\zhongdian{原命题正确,那么它的逆否命题一定正确。}

如果有两个命题,从第一个命题正确(或错误)可以得出第二个命题正确(或错误),
从第二个命题正确(或错误)也可以得出第一个命题正确(或错误),
那么这样的两个命题叫做\zhongdian{等价命题}。
如果两个命题互为逆否,那么从其中任何一个命题正确(或错误),
都可以得出另一个命题也正确(或错误)。
因此,\zhongdian{两个互为逆否的命题是等价命题。}
这个关系可以写成
\zhongdian{$$ \text{原命题} \bm{\dengjiayu} \text{逆否命题} \juhao $$}

由于两个互为逆否的命题具有等价关系,当我们证明某个命题有困难时,
可以用它的逆否命题的证明来代替原命题的证明。
例如,我们在前面证明“对角互补的四边形,内接于圆”时,
实际上是证明了它的逆否命题“不内接于圆的四边形的对角不互补”。



\begin{lianxi}

\xiaoti{(口答)下列各命题看作原命题时,它的逆命题、否命题、逆否命题各是什么?哪些正确?那些不正确?}
\begin{xiaoxiaotis}

    \xxt{末位是0的整数,可以被5整除;}

    \xxt{当 $x = 2$ 时,$x^2 - 3x  + 2 = 0$;}

    \xxt{对顶角相等;}

    \xxt{线段的垂直平分线上的点和这条线段两个端点的距离相等;}

    \xxt{到圆心的距离不等于半径的直线不是圆的切线。}
\end{xiaoxiaotis}

\xiaoti{下列各对命题的相互关系怎样?它们是否等价?}
\begin{xiaoxiaotis}

    \xxt{$A \tuichu B$ \quad 和 \quad $\buji{A} \tuichu \buji{B}$;}

    \xxt{$B \tuichu A$ \quad 和 \quad $\buji{A} \tuichu \buji{B}$;}

    \xxt{$\buji{B} \tuichu \buji{A}$ \quad 和 \quad $\buji{A} \tuichu \buji{B}$。}

\end{xiaoxiaotis}

\end{lianxi}

