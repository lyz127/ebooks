\subsection{相似三角形的性质}\label{subsec:czjh2-6-8}
\begin{enhancedline}

两个三角形相似,根据定义可知它们具有对应角相等,对应边成比例这些性质。下面我们来研究其它性质。

已知 $\triangle ABC \xiangsi \triangle A'B'C'$, 相似比为 $k$, $AD$、$A'D'$ 是对应高(图 \ref{fig:czjh2-6-25})。
那么在 $\triangle ABD$ 和 $\triangle A'B'D'$ 中, $\angle B = \angle B'$,
$\angle ADB = \angle A'D'B' = Rt \angle$, 所以

$\triangle ABD \xiangsi \triangle A'B'D'  \tuichu  \dfrac{AD}{A'D'} = \dfrac{AB}{A'B'} = k$。

\begin{figure}[htbp]
    \centering
    \begin{minipage}[b]{7cm}
        \centering
        \begin{tikzpicture}
    \def\drawingcode{
        \tkzDefTriangle[two angles=40 and 65](B,C)  \tkzGetPoint{A}
        \tkzDefLine[altitude](B,A,C)  \tkzGetPoint{D}
        \tkzDrawPolygon(A,B,C)
        \tkzDrawSegment(A,D)
        \tkzMarkRightAngle(A,D,B)
    }

    \begin{scope}
        \tkzDefPoints{0/0/B, 2/0/C}
        \drawingcode
        \tkzLabelPoints[above](A)
        \tkzLabelPoints[below](B,D,C)
    \end{scope}

    \begin{scope}[xshift=3cm]
        \tkzDefPoints{0/0/B, 3/0/C}
        \drawingcode
        \tkzLabelPoint[above](A){$A'$}
        \foreach \x in {B,C,D} {
            \tkzLabelPoint[below](\x){$\x'$}
        }
    \end{scope}
\end{tikzpicture}


        \caption{}\label{fig:czjh2-6-25}
    \end{minipage}
    \qquad
    \begin{minipage}[b]{7cm}
        \centering
        \begin{tikzpicture}
    \begin{scope}[scale=1.5]
        \tkzDefPoints{0/0/B, 2/0/C, 1.5/1/A}
        \tkzDrawPolygon(A,B,C)
        \tkzLabelPoints[above](A)
        \tkzLabelPoints[below](B,C)
    \end{scope}

    \begin{scope}[xshift=4cm]
        \tkzDefPoints{0/0/B', 2/0/C', 1.5/1/A'}
        \tkzDrawPolygon(A',B',C')
        \tkzLabelPoints[above](A')
        \tkzLabelPoints[below](B',C')
    \end{scope}
\end{tikzpicture}


        \caption{}\label{fig:czjh2-6-26}
    \end{minipage}
\end{figure}

类似地,可以推得两个相似三角形对应中线的比,对应角平分线的比也等于相似比。就是

\begin{dingli}
    相似三角形对应高的比,对应中线的比和对应角平分线的比都等于相似比。
\end{dingli}

如图 \ref{fig:czjh2-6-26}, $\triangle ABC \xiangsi \triangle A'B'C'$,相似比是 $k$。

$\triangle ABC \xiangsi \triangle A'B'C'  \tuichu  \dfrac{AB}{A'B'} = \dfrac{BC}{B'C'} = \dfrac{CA}{C'A'} = k  \tuichu \dfrac{AB + BC + CA}{A'B' + B'C' + C'A'} = k$。

由此可得

\begin{dingli}
    相似三角形周长的比等于相似比。
\end{dingli}

如图 \ref{fig:czjh2-6-25}, $\triangle ABC \xiangsi \triangle A'B'C'$,相似比是 $k$,
$AD$、$A'D'$ 分别是两个三角形的高。那么

$\dfrac{S_{\triangle ABC}}{S_{\triangle A'B'C'}} = \dfrac{\exdfrac{1}{2} AD \cdot BC}{\exdfrac{1}{2} A'D' \cdot B'C'} = \dfrac{AD}{A'D'} \cdot \dfrac{BC}{B'C'} = k \cdot k = k^2$。

由此得到下面的定理:

\begin{dingli}[定理]
    相似三角形面积的比等于相似比的平方。
\end{dingli}


\liti 有一块三角形余料 $ABC$,它的边 $BC = 120$ 毫米, 高 $AD = 80$ 毫米(图 \ref{fig:czjh2-6-27}),
要把它加工成正方形零件,使正方形的一边在 $BC$ 上,其余两个顶点分别在 $AB$、$AC$ 上。
加工成的正方形零件的边长为多少毫米?

\begin{wrapfigure}[6]{r}{5cm}
    \centering
    \begin{tikzpicture}
    \pgfmathsetmacro{\factor}{.03}
    \tkzDefPoints{0/0/B, 120*\factor/0/C}
    \tkzDefPointOnLine[pos=0.6](B,C)  \tkzGetPoint{D}
    \tkzDefShiftPoint[D](0, 80*\factor){A}
    \tkzDefShiftPoint[D](0, 48*\factor){E}
    \tkzDefLine[parallel=through E](B,C)  \tkzGetPoint{e}
    \tkzInterLL(E,e)(A,B)  \tkzGetPoint{P}
    \tkzInterLL(E,e)(A,C)  \tkzGetPoint{N}
    \tkzDefPointBy[projection= onto B--C](P)  \tkzGetPoint{Q}
    \tkzDefPointBy[projection= onto B--C](N)  \tkzGetPoint{M}

    \tkzDrawPolygon(A,B,C)
    \tkzDrawPolygon(P,Q,M,N)
    \tkzDrawSegments(A,D)
    \tkzMarkRightAngle(C,D,A)
    \tkzLabelPoints[above](A)
    \tkzLabelPoints[left](B,P)
    \tkzLabelPoints[right](C,N)
    \tkzLabelPoints[below](Q,D,M)
    \tkzLabelPoints[above,xshift=-.5em](E)
\end{tikzpicture}


    \caption{}\label{fig:czjh2-6-27}
\end{wrapfigure}

\jie 设正方形 $PQMN$ 为加工成的正方形。边 $QM$ 在 $BC$ 上,顶点 $P$、$N$ 分别在 $AB$、$AC$ 上,
高 $AD$ 与边 $PN$ 相交于点 $E$。设正方形的边长为 $x$ 毫米,则有

\qquad $PN \pingxing BC  \tuichu \triangle APN \xiangsi \triangle ABC  \tuichu  \dfrac{AE}{AD} = \dfrac{PN}{BC}$。

因此,可以列出方程:

\qquad $\dfrac{80 - x}{80} = \dfrac{x}{120}$。

解得 \quad $x = 48$(毫米)。



\liti 已知:如图 \ref{fig:czjh2-6-28}, $DE \pingxing BC$, $\dfrac{AD}{AB} = \exdfrac{3}{5}$,
$S_{\triangle ABC} = S$。求 $S_{\triangle ADE}$。

$\left.\begin{aligned}
    \text{\jie} DE \pingxing BC \tuichu \triangle ADE \xiangsi \triangle ABC \tuichu \dfrac{S_{\triangle ADE}}{S_{\triangle ABC}} = \dfrac{AD^2}{AB^2} \\
    \dfrac{AD}{AB} = \exdfrac{3}{5}
\end{aligned}\right\}$

\quad $\left.\begin{aligned}
    \tuichu \dfrac{S_{\triangle ADE}}{S_{\triangle ABC}} = \exdfrac{3^2}{5^2}  \tuichu S_{\triangle ADE} = \dfrac{9}{25} S_{\triangle ABC} \\
    S_{\triangle ABC} = S
\end{aligned}\right\}  \tuichu  S_{\triangle ADE} = \dfrac{9}{25} S \juhao$

\begin{figure}[htbp]
    \centering
    \begin{minipage}[b]{7cm}
        \centering
        \begin{tikzpicture}
    \tkzDefPoints{0/0/B, 4/0/C, 2.8/3/A}
    \tkzDefPointOnLine[pos=3/5](A,B)  \tkzGetPoint{D}
    \tkzDefPointOnLine[pos=3/5](A,C)  \tkzGetPoint{E}

    \tkzDrawPolygon(A,B,C)
    \tkzDrawPolygon(D,E)
    \tkzLabelPoints[above](A)
    \tkzLabelPoints[left](B,D)
    \tkzLabelPoints[right](C,E)
\end{tikzpicture}


        \caption{}\label{fig:czjh2-6-28}
    \end{minipage}
    \qquad
    \begin{minipage}[b]{7cm}
        \centering
        \begin{tikzpicture}
    \def\drawingcode{
        \tkzDefTriangle[two angles=50 and 70](B,C)  \tkzGetPoint{A}
        \tkzDefLine[altitude](B,A,C)  \tkzGetPoint{D}
        \tkzDefLine[altitude](A,B,C)  \tkzGetPoint{E}
        \tkzDrawPolygon(A,B,C)
        \tkzDrawSegments(A,D B,E)
        \tkzMarkRightAngle(A,D,B)
        \tkzMarkRightAngle(B,E,C)
    }

    \begin{scope}
        \tkzDefPoints{0/0/B, 3/0/C}
        \drawingcode
        \tkzLabelPoints[above](A)
        \tkzLabelPoints[right](E)
        \tkzLabelPoints[below](B,D,C)
    \end{scope}

    \begin{scope}[xshift=4cm]
        \tkzDefPoints{0/0/B, 2/0/C}
        \drawingcode
        \tkzLabelPoint[above](A){$A'$}
        \tkzLabelPoint[right](E){$E'$}
        \foreach \x in {B,C,D} {
            \tkzLabelPoint[below](\x){$\x'$}
        }
    \end{scope}
\end{tikzpicture}


        \caption{}\label{fig:czjh2-6-29}
    \end{minipage}
\end{figure}

\liti 已知: $\triangle ABC$ 与 $\triangle A'B'C'$ 中
$AD$、$BE$ 是 $\triangle ABC$ 的高,
$A'D'$、$B'E'$ 是 $\triangle A'B'C'$ 的高,
且 $\dfrac{AB}{AD} = \dfrac{A'B'}{A'D'}$, $\angle C = \angle C'$ (图 \ref{fig:czjh2-6-29})。
求证: $\dfrac{AD}{BE} = \dfrac{A'D'}{B'E'}$。

分析:从图形可知,求证的四条成比例线段不分别在两个三角形中,所以不能用直接证两个三角形相似得出。
但我们知道这四条线段分别是 $\triangle ABC$ 和 $\triangle A'B'C'$ 的高,
如果能证明 $\triangle ABC \xiangsi \triangle A'B'C'$,那么比例就成立了。
可是由题设不能直接证得这两个三角形相似,但可证得 $\triangle ABD \xiangsi \triangle A'B'D'$,
得 $\angle ABC = \angle A'B'C'$,从而 $\triangle ABC \xiangsi \triangle A'B'C'$。

$\left.\begin{aligned}
    \text{\zhengming} && \dfrac{AB}{AD} = \dfrac{A'B'}{A'D'} \\
                      && \angle ADB = \angle A'D'B' = 90^\circ
\end{aligned}\right\}  \tuichu Rt \triangle ABD \xiangsi Rt \triangle A'B'D'$

\qquad $\left.\begin{aligned}
    \tuichu \angle ABC = \angle A'B'C' \\
    \angle C = \angle C'
\end{aligned}\right\}  \tuichu \triangle ABC \xiangsi \triangle A'B'C'$

\qquad $\tuichu \dfrac{AB}{A'B'} = \dfrac{BE}{B'E'} = \dfrac{AD}{A'D'}  \tuichu  \dfrac{AD}{BE} = \dfrac{A'D'}{B'E'}$。


\begin{lianxi}

\xiaoti{证明:两个相似三角形的对应中线的比等于相似比。}

\xiaoti{已知:点 $M$、$N$、$P$ 分别是 $\triangle ABC$ 的中线 $AD$、$BE$、$CF$ 的中点。
    求 $\triangle ABC$ 与 $\triangle MNP$ 面积的比。
}

\xiaoti{把一个三角形改成和它相似的三角形。}
\begin{xiaoxiaotis}

    \xxt{如果边长扩大为原来的 100 倍,那么面积扩大为原来的多少倍?}

    \xxt{如果面积扩大为原来的 100 倍,那么边长扩大为原来的多少倍?}

\end{xiaoxiaotis}

\end{lianxi}
\end{enhancedline}

