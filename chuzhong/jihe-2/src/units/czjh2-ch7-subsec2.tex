\subsection{经过三点的圆}\label{subsec:czjh2-7-2}

我们知道,经过一个点 $A$ 作圆很容易,只要以点 $A$ 以外的任意一点为圆心,
以这一点与点 $A$ 的距离为半径就可以作出。这样的圆有无数多个(图 \ref{fig:czjh2-7-7})。
如果要作通过两个点 $A$、$B$ 的圆,那就要找这样一个点作圆心,
使它与点 $A$、$B$ 的距离都相等,这样的点在线段 $AB$ 的垂直平分线上。
因此,以线段 $AB$ 的垂直平分线上任意一点为圆心,以这一点与点 $A$ 或点 $B$ 的距离为半径就可以作出。
这样的圆也有无数多个(图 \ref{fig:czjh2-7-8})。


\begin{figure}[htbp]
    \centering
    \begin{minipage}[b]{4cm}
        \centering
        \begin{tikzpicture}
    \tkzDefPoints{0/0/A, -1/0/O_1, 0.3/-0.5/O_2, 0.8/0.8/O_3}

    \tkzDrawCircle[thick](O_1,A)
    \tkzDrawCircle[thick](O_2,A)
    \tkzDrawCircle[thick](O_3,A)
    \tkzDrawPoints(O_1, O_2, O_3, A)
    \tkzLabelPoints[above right](A)
    \tkzLabelPoints[below](O_1, O_2)
    \tkzLabelPoints[right](O_3)
\end{tikzpicture}


        \caption{}\label{fig:czjh2-7-7}
    \end{minipage}
    \qquad
    \begin{minipage}[b]{6cm}
        \centering
        \begin{tikzpicture}
    \tkzDefPoints{0/0.5/A, 0/-0.5/B, -1/0/O_1, 0.5/0/O_2, 1.3/0/O_3}

    \tkzDrawLine[add=.7 and .8](O_1, O_3)
    \tkzDrawCircle[thick](O_1,A)
    \tkzDrawCircle[thick](O_2,A)
    \tkzDrawCircle[thick](O_3,A)
    \tkzDrawPoints(O_1, O_2, O_3, A, B)
    \tkzLabelPoints[above=.5em](A)
    \tkzLabelPoints[below=.5em](B)
    \tkzLabelPoints[below](O_1, O_2, O_3)
\end{tikzpicture}


        \caption{}\label{fig:czjh2-7-8}
    \end{minipage}
    \qquad
    \begin{minipage}[b]{4.5cm}
        \centering
        \begin{tikzpicture}
    \tkzDefPoints{0/0/B, 3/0/C, 2/2/A}
    \tkzLabelPoints[above](A)
    \tkzLabelPoints[left](B)
    \tkzLabelPoints[right](C)

    % 1
    \tkzDefLine[mediator, K=.5](A,B)  \tkzGetPoints{E}{D}
    \tkzCompasss(A,D  B,D  A,E  B,E)
    \tkzDrawSegments(A,B  D,E)
    \tkzLabelPoints[above right](D,E)

    % 2
    \tkzDefLine[mediator, K=.5](B,C)  \tkzGetPoints{F}{G}
    \tkzInterLL(D,E)(F,G)  \tkzGetPoint{O}
    \tkzCompasss(B,F  C,F  B,G  C,G)
    \tkzDrawSegments(B,C  F,G)
    \tkzDrawPoint(O)
    \tkzLabelPoints[right](F)
    \tkzLabelPoints[below](G)
    \tkzLabelPoints[below left](O)

    % 3
    \tkzDrawCircle[thick](O,B)

    % ex
    \tkzDrawSegment(A,C)
\end{tikzpicture}


        \caption{}\label{fig:czjh2-7-9}
    \end{minipage}
\end{figure}

现在来讨论,经过三个已知点的圆。

\zhongdian{作圆,使它经过不在同一直线上的三个已知点。}

已知:不在同一直线上的三点 $A$、$B$、$C$(图 \ref{fig:czjh2-7-9})。

求作: $\yuan \, O$,使它经过点 $A$、$B$、$C$。

分析:要作一个圆经过三个已知点 $A$、$B$、$C$,就要确定一个点作圆心,使它到这三点的距离相等。
以前我们学过,三角形三边的垂直平分线相交于一点,这个点到三角形三个顶点的距离相等。
因此可以把 $\triangle ABC$ 三边的垂直平分线的交点作为圆心。

\zuofa 1. 连结 $AB$,作线段 $AB$ 的垂直平分线 $DE$。

2. 连结 $BC$,作线段 $BC$ 的垂直平分线 $FG$,交 $DE$ 于点 $O$。

3. 以 $O$ 为圆心, $OB$ 为半径作圆。

$\yuan\,O$ 就是所求作的圆。

\zhengming 因为 $\yuan\,O$ 的半径等于 $OB$,所以点 $B$ 在 $\yuan\,O$ 上, 就是 $\yuan\,O$ 经过点 $B$。

因为 $O$ 在 $AB$ 的垂直平分线上,所以 $OA = OB$,因此 $\yuan\,O$ 经过点 $A$。
同样可证 $\yuan\,O$ 经过点 $C$。

我们知道,过 $A$、$B$ 两点的圆和过 $B$、$C$两点的圆, 它们的圆心分别在 $AB$ 和 $BC$ 的垂直平分线上。
从上面的作法又可以知道:当已知点 $A$、$B$、$C$ 不在同一直线上时,
$\triangle ABC$ 三边的垂直平分线有一个且只有一个交点,
所以经过点 $A$、$B$、$C$ 可以作一个且只可作一个圆,这就得到:

\begin{dingli}[定理]
    不在同一直线上的三个点确定一个圆。
\end{dingli}

当点 $A$、$B$、$C$ 在同一直线上时,不能作一个圆经过这三点。(为什么?)

由定理可知,经过三角形三个顶点可以作一个圆。
经过三角形各顶点的圆叫做\zhongdian{三角形的外接圆},
外接圆的圆心叫做\zhongdian{三角形的外心},
这个三角形叫做这个\zhongdian{圆的内接三角形}。

一般地,如果一个圆经过多边形的各顶点,这个圆叫做\zhongdian{多边形的外接圆},
这个多边形叫做这个\zhongdian{圆的内接多边形}。

图 \ref{fig:czjh2-7-10} 中,四边形 $ABCD$ 是 $\yuan\,O$ 的内接四边形;
$\yuan\,O$ 是四边形 $ABCD$ 的外接圆。

注意:经过任意四点不一定能作一个圆,所以多于三边的多边形不一定有外接圆。

\begin{figure}[htbp]
    \centering
    \begin{minipage}[b]{4cm}
        \centering
        \begin{tikzpicture}
    \tkzDefPoint(0,0){O}
    \tkzDefPoint(215:1.5){A}
    \tkzDefPoint(325:1.5){B}
    \tkzDefPoint(40:1.5){C}
    \tkzDefPoint(100:1.5){D}

    \tkzDrawPolygon(A,B,C,D)
    \tkzDrawCircle[thick](O,A)
    \tkzDrawPoint(O)
    \tkzLabelPoints[left](A)
    \tkzLabelPoints[right](B,C,O)
    \tkzLabelPoints[above](D)
\end{tikzpicture}


        \caption{}\label{fig:czjh2-7-10}
    \end{minipage}
    \qquad
    \begin{minipage}[b]{5cm}
        \centering
        \begin{tikzpicture}
    \tkzDefPoints{0/0/O}
    \tkzDefPoint(135:1.5){A}
    \tkzDefPoint(45:1.5){B}
    \tkzDefShiftPoint[A](0,-.4){E}
    \tkzDefShiftPoint[B](0,-.4){F}
    \tkzDefLine[mediator](A,B)  \tkzGetPoints{c}{d}
    \tkzInterLL(E,F)(c,d)  \tkzGetPoint{C}
    \tkzDefShiftPoint[C](0,-2.5){D}
    \tkzDefShiftPoint[C](-0.4,0){G}
    \tkzDefShiftPoint[G](0,-2.5){H}

    \tkzDrawCircle[thick](O,A)
    \tkzDrawPolygon[fill=white](A,B,F,C,D,H,G,E) % 为了遮挡住 D 点附近的圆弧
    \tkzDrawPolygon[pattern={mylines[angle=45, distance={4pt}]}](A,B,F,C,D,H,G,E)
    \tkzDrawPoint(O)
    \tkzLabelPoints[left](A)
    \tkzLabelPoints[right](B,D)
    \tkzLabelPoints[below right](C)
\end{tikzpicture}


        \caption*{(第 1 题)}
    \end{minipage}
    \qquad
    \begin{minipage}[b]{4.5cm}
        \centering
        \begin{tikzpicture}
    \tkzDefPoints{0/0/O}
    \tkzDefPoint(60:1.5){A}
    \tkzDefPoint(205:1.5){B}
    \tkzDefPoint(335:1.5){C}

    \tkzDrawPolygon(A,B,C)
    \tkzDrawCircle[thick](O,A)
    \tkzDrawPoint(O)
    \tkzLabelPoints[above](A)
    \tkzLabelPoints[left](B)
    \tkzLabelPoints[right](C,O)
\end{tikzpicture}


        \caption*{(第 4 题)}
    \end{minipage}
\end{figure}

\begin{lianxi}

\xiaoti{(口答)如图, $CD$ 所在直线垂直平分线段 $AB$。为什么使用这样的工具可以找到圆形工件的圆心。}

\xiaoti{作边长分别为 2 厘米、 2.5 厘米、3 厘米的三角形,再作出这个三角形的外接圆,量出这个圆的直径(精确到 0.1 厘米)。}

\xiaoti{作一个直角三角形,作它的外接圆;作一个钝角三角形,作它的外接圆。这两个三角形的外心的位置各是怎样的?}

\xiaoti{按图填空:}
\begin{xiaoxiaotis}

    \xxt{$\triangle ABC$ 是 $\yuan\,O$ 的 \xhx[1cm] 接三角形;}

    \xxt{$\yuan\,O$ 是 $\triangle ABC$ 的 \xhx[1cm] 接圆。}

\end{xiaoxiaotis}

\end{lianxi}



