\subsection{圆的内接四边形}\label{subsec:czjh2-7-6}
\begin{enhancedline}

我们知道,圆的内接四边形的四个顶点都在同一个圆上,所以它的四个内角都是圆周角。
这样,我们就可以利用圆周角定理,来研究圆的内接四边形的角。

\begin{wrapfigure}[9]{r}{4.5cm}
    \centering
    \begin{tikzpicture}
    \tkzDefPoints{0/0/O}
    \tkzDefPoint(130:1.5){A}
    \tkzDefPoint(210:1.5){B}
    \tkzDefPoint(330:1.5){C}
    \tkzDefPoint(70:1.5){D}
    \tkzDefPointOnLine[pos=1.4](B,C)  \tkzGetPoint{E}

    \tkzDrawCircle[thick](O,A)
    \tkzDrawPolygon(A,B,C,D)
    \tkzDrawSegments(B,E)
    \tkzDrawPoint(O)
    \tkzLabelPoints[right](O)
    \tkzLabelPoints[left, yshift=.3em](A)
    \tkzLabelPoints[left](B)
    \tkzLabelPoints[below right](C)
    \tkzLabelPoints[above](D)
    \tkzLabelPoints[below](E)
\end{tikzpicture}


    \caption{}\label{fig:czjh2-7-27}
\end{wrapfigure}

如图 \ref{fig:czjh2-7-27}, 四边形 $ABCD$ 是 $\yuan\,O$ 的内接四边形。

$\because$ \quad $\yuanhu{BAD}$ 和 $\yuanhu{BCD}$ 所对的圆心角的和是周角。

$\therefore$ \quad $\angle A + \angle BCD = 180^\circ$。

同理 $\angle B + \angle D = 180^\circ$。

如果延长 $BC$ 到 $E$,那么 $\angle BCD + \angle DCE = 180^\circ$,所以

$\angle A = \angle DCE$。

$\angle A$ 是与 $\angle DCE$ 相邻的内角 $\angle DCB$ 的对角(简称为 $\angle DCE$ 的内对角),
于是我们得到圆的内接四边形的性质定理。

\begin{dingli}[定理]
    圆的内接四边形的对角互补,并且任何一个外角都等于它的内对角。
\end{dingli}


\liti 如图 \ref{fig:czjh2-7-28}, $\yuan\,O_1$ 和 $\yuan\,O_2$ 相交于 $A$、$B$ 两点,
经过点 $A$ 的直线 $CD$ 与 $\yuan\,O_1$ 交于点 $C$,与 $\yuan\,O_2$ 交于点 $D$。
经过点 $B$ 的直线 $EF$, 与 $\yuan\,O_1$ 交于点 $E$,与 $\yuan\,O_2$ 交于点 $F$。

求证: $CE \pingxing DF$。

\zhengming 连结 $AB$。

$\because$ \quad $ABEC$ 是 $\yuan\,O_1$ 的内接四边形,

$\therefore$ \quad $\angle BAD = \angle E$。

又 $\because$ \quad $ADFB$ 是 $\yuan\,O_2$ 的内接四边形,

$\therefore$ \quad $\angle BAD + \angle F = 180^\circ$。

$\therefore$ \quad $\angle E + \angle F = 180^\circ$。

$\therefore$ \quad $CD \pxqdy DF$。


$\therefore$ \quad $AB \cdot AC = AE \cdot AD$。

\begin{figure}[htbp]
    \centering
    \begin{minipage}[b]{7.5cm}
        \centering
        \begin{tikzpicture}
    \tkzDefPoints{0/0/O_1, 3/0/O_2}
    \tkzInterCC[R](O_1,1.5)(O_2,2)  \tkzGetPoints{A}{B}
    \tkzDefPoint(170:1.5){C}
    \tkzDefPoint(220:1.5){E}
    \tkzInterLC[common=A](C,A)(O_2,A)  \tkzGetFirstPoint{D}
    \tkzInterLC[common=B](E,B)(O_2,B)  \tkzGetFirstPoint{F}

    \tkzDrawCircle[thick](O_1,A)
    \tkzDrawCircle[thick](O_2,A)
    \tkzDrawPolygon(C,D,F,E)
    \tkzDrawSegments[dashed](A,B)
    \tkzDrawPoints(O_1, O_2)
    \tkzLabelPoints[right](O_1, O_2)
    \tkzLabelPoints[above, yshift=.3em](A)
    \tkzLabelPoints[below, yshift=-.3em](B)
    \tkzLabelPoints[left](C)
    \tkzLabelPoints[above right](D)
    \tkzLabelPoints[below left](E)
    \tkzLabelPoints[right](F)
\end{tikzpicture}


        \caption{}\label{fig:czjh2-7-28}
    \end{minipage}
    \qquad
    \begin{minipage}[b]{7cm}
        \centering
        \begin{tikzpicture}
    \tkzDefPoints{0/0/O}
    \tkzDefPoint(200:1.5){A}
    \tkzDefPoint(170:1.5){C}
    \tkzDefPoint(190:3.2){P}
    \tkzInterLC[common=A](P,A)(O,A)  \tkzGetFirstPoint{B}
    \tkzInterLC[common=C](P,C)(O,A)  \tkzGetFirstPoint{D}

    \tkzDrawCircle[thick](O,A)
    \tkzDrawPolygon(A,D,B,C)
    \tkzDrawSegments(P,B  P,D)
    \tkzDrawPoint(O)
    \tkzLabelPoints[right](O)
    \tkzLabelPoints[below left](A)
    \tkzLabelPoints[right](B)
    \tkzLabelPoints[above left](C)
    \tkzLabelPoints[above](D)
    \tkzLabelPoints[left](P)
\end{tikzpicture}


        \caption*{(第 2 题)}
    \end{minipage}
\end{figure}


\begin{lianxi}

\xiaoti{求证:圆内接平行四边形是矩形。}

\xiaoti{如图,经过圆外一点 $P$ 的两条直线与 $\yuan\,O$ 相交于 $A$、$B$ 和 $C$、$D$ 四点,
    在图中有几对相似三角形?为什么?
}

\end{lianxi}


圆的内接四边形的性质定理有下面的逆定理:

\begin{dingli}[定理]
    如果一个四边形的一组对角互补,那么这个四边形内接于圆。
\end{dingli}

已知:四边形 $ABCD$ 中, $\angle B + \angle D = 180^\circ$。

求证:四边形 $ABCD$ 内接于圆。

\begin{figure}[htbp]
    \centering
    \begin{minipage}[b]{6cm}
        \centering
        \begin{tikzpicture}
    \tkzDefPoints{0/0/O}
    \tkzDefPoint(110:1.5){A}
    \tkzDefPoint(200:1.5){B}
    \tkzDefPoint(340:1.5){C}
    \tkzDefPoint(45:1.5){D'}
    \tkzDefPointOnLine[pos=1.2](B,D')  \tkzGetPoint{D}

    \tkzDrawCircle[thick](O,A)
    \tkzDrawPolygon(A,B,C,D)
    \tkzDrawSegments[dashed](A,D'  C,D'  B,D)
    \tkzLabelPoints[above left](A)
    \tkzLabelPoints[left](B)
    \tkzLabelPoints[right](C)
    \tkzLabelPoints[right](D)
    \tkzLabelPoints[below, xshift=-.5em, yshift=-.5em](D')
\end{tikzpicture}


        \caption*{甲}
    \end{minipage}
    \qquad
    \begin{minipage}[b]{6cm}
        \centering
        \begin{tikzpicture}
    \tkzDefPoints{0/0/O}
    \tkzDefPoint(110:1.5){A}
    \tkzDefPoint(200:1.5){B}
    \tkzDefPoint(340:1.5){C}
    \tkzDefPoint(45:1.5){D'}
    \tkzDefPointOnLine[pos=0.8](B,D')  \tkzGetPoint{D}

    \tkzDrawCircle[thick](O,A)
    \tkzDrawPolygon(A,B,C,D)
    \tkzDrawSegments[dashed](A,D'  C,D'  B,D')
    \tkzLabelPoints[above left](A)
    \tkzLabelPoints[left](B)
    \tkzLabelPoints[right](C)
    \tkzLabelPoints[left=.3em](D)
    \tkzLabelPoints[above right](D')
\end{tikzpicture}


        \caption*{乙}
    \end{minipage}
    \caption{}\label{fig:czjh2-7-29}
\end{figure}

分析:要证明四边形 $ABCD$ 内接于圆,就是要证明 $A$、$B$、$C$、$D$ 四点在同一个圆上。
因为 $A$、$B$、$C$ 三点不在同一直线上,可以确定一个圆,
所以只要证明第四点 $D$ 也在这个圆上就可以了。
但直接证明点 $D$ 在圆上比较困难。现在我们采用一种间接证明的方法,
就是假设点 $D$ 不在圆上,经过推理论证,得出错误的结论,
这说明假设点 $D$ 不在圆上是错误的,从而证明点 $D$ 在圆上。

\zhengming 经过四边形三个顶点 $A$、$B$、$C$ 作 $\yuan\,O$。

假设点 $D$ 不在圆上,那么只有两种情况:(1)点 $D$ 在圆外; (2)点 $D$ 在圆内。

(1)所设点 $D$ 在圆外(图 \ref{fig:czjh2-7-29} 甲)。连结 $BD$ 交 $\yuan\,O$ 于点 $D'$。连结 $AD'$、$CD'$。

$\because$ \quad $\angle AD'B$、 $BD'C$ 分别是 $\triangle AD'D$、 $CD'D$ 的外角,

$\therefore$ \quad \begin{zmtblr}[t]{}
    $\angle AD'B > \angle ADB$, \\
    $\angle BD'C > \angle BDC$。
\end{zmtblr}

$\therefore$ \quad $\angle AD'B + \angle BD'C > \angle ADB + \angle BDC$,

即 \quad $\angle AD'C > \angle ADC$。

又 $\because$ \quad $\angle ADC + \angle ABC = 180^\circ$,

$\therefore$ \quad $\angle AD'C + \angle ABC > 180^\circ$。

这与圆内接四边形性质定理矛盾。

所以点 $D$ 不能在圆外。

(2) 同 (1) 类似可证明点 $D$ 不能在圆内(图 \ref{fig:czjh2-7-29} 乙)。

$\therefore$ \quad 点 $D$ 在 $\yuan\,O$ 上,

即四边形 $ABCD$ 是 $\yuan\,O$ 的内接四边形。


这个定理的证明,不是直接去证明命题的结论,而是先提出与结论相反(相排斥)的假设,
然后推导出和已经证明的定理或公理、定义、题设等相矛盾的结果,
这样就证明了与结论相反的假设不能成立,从而肯定了原来的结论必定成立,
这种间接证明命题的方法叫做\zhongdian{反证法}。

用反证法证明命题一般有下面三个步骤:

(1)假设命题的结论不成立;

(2)从这个假设出发,经过推理论证,得出矛盾;

(3)由矛盾判定假设不正确,从而肯定命题的结论正确。



\liti 求证:圆的两条相交弦(直径除外)不能互相平分。

已知:如图 \ref{fig:czjh2-7-30},弦 $AB$、$OD$ 相交于点 $P$。

求证:$AB$、$CD$ 不能互相平分。

证明:用反证法。

假设 $AB$ 与 $CD$ 互相平分。

因为 $AB$、$CD$ 不是直径,所以点 $P$ 与 $O$ 不重合,连结 $OP$。

$\because$ \quad $AP = PB$,

$\therefore$ \quad $OP \perp AB$。

同理 $OP \perp CD$。

这就是说过点 $P$ 有两条直线 $AB$、$CD$ 都垂直于 $OP$,
这与过一点只有一条直线与已知直线垂直相矛盾。

所以 $AB$ 和 $CD$ 不能互相平分。

\begin{figure}[htbp]
    \centering
    \begin{minipage}[b]{7cm}
        \centering
        \begin{tikzpicture}
    \tkzDefPoints{0/0/O}
    \tkzDefPoint(190:1.5){A}
    \tkzDefPoint(325:1.5){B}
    \tkzDefPoint(230:1.5){C}
    \tkzDefPoint(10:1.5){D}
    \tkzInterLL(A,B)(C,D)  \tkzGetPoint{P}

    \tkzDrawCircle[thick](O,A)
    \tkzDrawSegments(A,B  C,D)
    \tkzDrawSegments[dashed](O,P)
    \tkzLabelPoints[left](A)
    \tkzLabelPoints[right](B)
    \tkzLabelPoints[below left](C)
    \tkzLabelPoints[right](D)
    \tkzLabelPoints[above](O)
    \tkzLabelPoints[below](P)
\end{tikzpicture}


        \caption{}\label{fig:czjh2-7-30}
    \end{minipage}
    \qquad
    \begin{minipage}[b]{7cm}
        \centering
        \begin{tikzpicture}
    \tkzDefPoints{0/0/A, 2.5/0/B}
    % C 和 D 的角相等
    \tkzDefTriangle[two angles=95 and 25](A,B)  \tkzGetPoint{C}
    \tkzDefTriangle[two angles=40 and 80](A,B)  \tkzGetPoint{D}
    \tkzDefCircle[circum](A,B,C)  \tkzGetPoints{O}{R}

    \tkzDrawPolygon(A,B,C)
    \tkzDrawPolygon(A,B,D)
    \tkzDrawCircle[thick,dashed](O,R)
    \tkzDrawPoint(O)
    \tkzLabelPoints[right](O)
    \tkzLabelPoints[below left](A)
    \tkzLabelPoints[right](B)
    \tkzLabelPoints[above left](C)
    \tkzLabelPoints[above](D)
\end{tikzpicture}


        \caption{}\label{fig:czjh2-7-31}
    \end{minipage}
\end{figure}

\liti \zhongdian{如果两个三角形有一条公共边,这条边所对的角相等,并且在公共边的同侧,
    那么这两个三角形有公共的外接圆。
}

已知: 如图 \ref{fig:czjh2-7-31}, $\angle C$、$\angle D$ 在 $AB$ 同侧, $\angle C = \angle D$。

求证: $\triangle ABC$ 和 $\triangle ABD$ 有公共外接圆。

证明:用反证法。

假设 $\triangle ABC$ 和 $\triangle ABD$ 没有公共外接圆,
即  $A$、$B$、$C$、$D$ 四点不在同一个圆上。

过 $A$、$B$、$C$ 三点作 $\yuan\,O$, 则点 $D$ 不在 $\yuan\,O$ 上。
同 7.5 节例 2 的证明一样可得,
$$ \angle ADB \neq \angle ACB \juhao $$

这与题设相矛盾。

$\therefore$ \quad  $\triangle ABC$ 和 $\triangle ABD$ 有公共外接圆。



\begin{lianxi}

\xiaoti{按照图 \ref{fig:czjh2-7-29} 乙,证明定理。}

\xiaoti{否定下列各结论,并写出由此可能出现的情况:}
\begin{xiaoxiaotis}

    \begin{tblr}{columns={12em, colsep=0pt}}
        \xxt{$a = b$;} & \xxt{$\angle A > 60^\circ$;} & \xxt{$AB \pingxing CD$;} \\
        \xxt{点 $A$ 在 $\yuan\,O$ 上;} &  \xxt{点 $A$ 在直线 $a$ 上。}
    \end{tblr}
\end{xiaoxiaotis}

\xiaoti{已知:如图, $\pxsbx ABCD$ 中,过点 $A$、$B$ 的圆与 $AD$、$BC$ 分别交于点 $E$、$F$。
    求证:$C$、$D$、$E$、$F$ 四点在同一个圆上。
}

\begin{figure}[htbp]
    \centering
    \begin{minipage}[b]{7cm}
        \centering
        \begin{tikzpicture}
    \tkzDefPoints{0/0/O}
    \tkzDefPoint(120:1.5){A}
    \tkzDefPoint(200:1.5){B}
    \tkzDefPoint(340:1.8){C}
    \tkzDefPointBy[translation=from B to C](A)  \tkzGetPoint{D}
    \tkzInterLC[common=A](A,D)(O,A)  \tkzGetFirstPoint{E}
    \tkzInterLC[common=B](B,C)(O,A)  \tkzGetFirstPoint{F}

    \tkzDrawCircle[thick](O,A)
    \tkzDrawPolygon(A,B,C,D)
    \tkzLabelPoints[above](A,E,D)
    \tkzLabelPoints[left](B)
    \tkzLabelPoints[right](C)
    \tkzLabelPoints[below](F)
\end{tikzpicture}


        \caption*{(第 3 题)}
    \end{minipage}
    \qquad
    \begin{minipage}[b]{7cm}
        \centering
        \begin{tikzpicture}
    \tkzDefPoints{0/0/B, 3/0/C, 2.3/2/A}
    \tkzDefLine[altitude](A,B,C)  \tkzGetPoint{E}
    \tkzDefLine[altitude](A,C,B)  \tkzGetPoint{F}

    \tkzDrawPolygon(A,B,C)
    \tkzDrawSegments(B,E  C,F)
    \tkzMarkRightAngle[size=.2](B,E,C)
    \tkzMarkRightAngle[size=.2](B,F,C)
    \tkzLabelPoints[above](A)
    \tkzLabelPoints[left](B)
    \tkzLabelPoints[right](C,E)
    \tkzLabelPoints[above](F)
\end{tikzpicture}


        \caption*{(第 4 题)}
    \end{minipage}
\end{figure}

\xiaoti{已知:如图,$BE$ 和 $CF$ 是 $\triangle ABC$ 的高。
    求证:$F$、$B$、$C$、$E$ 四点在同一个圆上。
}

\end{lianxi}

\end{enhancedline}

