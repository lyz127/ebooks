\xiaojie

一、本章的主要内容是比例的性质、比例线段的有关定理,相似三角形及相似多边形的概念、判定和性质,
以及关于位似图形的一些初步知识和应用。


二、一般比例中的各项都是实数,由于线段的长度是正数,两条线段的比值一定是正数。


三、平行线分线段成比例定理是相似形中的基本定理,它把直线的平行性质和成比例线段联系起来,
平行于三角形一边的直线和三角形的另外两边或两边的延长线相交,
所得的对应线段成比例是平行线分线段成比例定理的重要推论,
它的逆命题是判定相似的重要依据。


四、在证明三角形内、外角平分线性质定理时,添加了平行线来转移比例,
证明比例线段时,常添加这种平行线作为辅助线。


五、全等三角形是相似比为 1 的特殊的相似三角形。
两个三角形相似的判定和性质与三角形全等的判定和性质相类似,后者是前者的特例。
判定两个三角形相似和研究相似三角形时,同样要注意角、边的对应关系。

判定三角形相似的条件有:

1. 一个角对应相等,夹这个角的两边对应成比例;

2. 两个角对应相等;

3. 三边对应成比例;

4. 两个直角三角形的斜边和一条直角边对应成比例。

当相似比等于 1 时, 上面判定三角形相似的定理, 就成为相应的判定三角形全等的定理: “SAS”; \\
“ASA”(或“AAS”);“SSS”;“HL”。

相似三角形的相似比是对应边的比;
对应的中线、高、角平分线及周长的比都等于相似比,
面积的比等于相似比的平方。


六、相似多边形对应对角线的比和周长的比都等于相似比,面积比等于相似比的平方,
以相似多边形三个对应顶点为顶点的对应三角形相似。


*七、两个图形的对应点连线交于一点,并且对应点到这点的距离成比例时,这两个图形就是位似图形。
位似图形一定是相似形,位似图形是具有特殊相对位置的相似形。
位似理论是测绘的基础。

