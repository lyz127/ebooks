%----------------------------------
% 图形相关的项
% 需要 TexLive 2020 版本

% 自定义填充格式,用于实现绘制阴影
% TikZ 提供的 north east lines 等默认 pattern, 或 Lines ,有个问题:使用不同的PDF阅读器,会得到不同的效果。
% 如:
%       VSCode 中 LaTeX WorkShop 看到的是线段,
%       SumatraPDF 看到的是连贯的线条。
% 所以这里采用自定义的格式。
% 下面的代码基本是将 Lines 的代码修改而成。
% Lines实现中,bounding box 是 (-0.5\distance, -0.5\distance) 到 (0.5\distance, 0.5\distance)
% 这里的 TikZ 实现是 (0, 0) 到 (\distance, \distance),按说应该没有区别,但实际效果确有不同。
\pgfdeclarepattern{
  name=mylines,
  parameters={
      \pgfkeysvalueof{/pgf/pattern keys/distance},
      \pgfkeysvalueof{/pgf/pattern keys/angle},
      \pgfkeysvalueof{/pgf/pattern keys/xshift},
      \pgfkeysvalueof{/pgf/pattern keys/yshift},
      \pgfkeysvalueof{/pgf/pattern keys/line width},
  },
  bottom left={%
    \pgfpoint
      {0}%
      {0}},
  top right={%
    \pgfpoint
      {(\pgfkeysvalueof{/pgf/pattern keys/distance})}%
      {(\pgfkeysvalueof{/pgf/pattern keys/distance})}},
  tile size={%
    \pgfpoint
      {\pgfkeysvalueof{/pgf/pattern keys/distance}}%
      {\pgfkeysvalueof{/pgf/pattern keys/distance}}},
  tile transformation={%
    \pgftransformshift{%
      \pgfpoint
        {\pgfkeysvalueof{/pgf/pattern keys/xshift}}%
        {\pgfkeysvalueof{/pgf/pattern keys/yshift}}}%
    \pgftransformrotate{\pgfkeysvalueof{/pgf/pattern keys/angle}}},
  defaults={
    distance/.initial=3pt,
    angle/.initial=0,
    xshift/.initial=0pt,
    yshift/.initial=0pt,
    line width/.initial=\the\pgflinewidth,
  },
  code={%
    \pgfsetlinewidth{\pgfkeysvalueof{/pgf/pattern keys/line width}}%
    \pgfpathmoveto{\pgfpoint{0pt}{0pt}}%
    \pgfpathlineto{\pgfpoint{(\pgfkeysvalueof{/pgf/pattern keys/distance})}{0pt}}%
    \pgfusepath{stroke}%
  },
}


% 绘制温度计(温度范围:-10 ~ 20 ℃)
\tikzset{
    % 参数1: 显示的温度。
    pics/thermometer/.style n args={1}{
      code = {
        % 绘制外框
        \draw [thick] (-1, -0.6) -- (0.8, -0.6) -- (0.8, 6) -- (-1, 6) -- cycle;

        % 绘制玻璃管
        \draw [thick] (135:0.35) arc (135:405:0.35)
            to [out=135, in=270] (0.2, 0.5)
            -- (0.2, 6) -- (-0.2, 6) -- (-0.2, 0.5)
            to [out=270, in=45] (135:0.35);

        % 绘制刻度
        \foreach \y in {1,...,7} {
            \ifnum \y<3
                \def\value{\the\numexpr5*(3-\y)}
            \else
                \def\value{\the\numexpr5*(\y-3)}
            \fi
            \draw (-0.2,0.5+0.7*\y) -- (-0.4,0.5+0.7*\y) node[anchor=east] {$\value$};
            \ifnum \y < 7
                \foreach \tmp in {1,...,4} {
                    \draw (-0.2,0.5+0.7*\y+0.14*\tmp) -- (-0.3,0.5+0.7*\y+0.14*\tmp);
                }
            \fi
        }

        % 绘制水银柱
        \def\h{\the\numexpr0.5+2.1+0.14*#1}
        \draw [black!80, fill] (135:0.3) arc (135:405:0.3)
            to [out=135, in=270] (0.15, 0.4)
            -- (0.15, \h) -- (-0.15, \h) -- (-0.15, 0.4)
            to [out=270, in=45] (135:0.3);
      }
    }
}


% 用于填充的水面的纹路
\tikzset{
    waterwave/.pic={
        \draw [dash=on 17pt off 2pt phase 0pt] (0, -0.01) -- (10, -0.01);
        \draw [dash=on 15pt off 3pt phase 0pt] (0, -0.1)  -- (10, -0.1);
        \draw [dash=on 10pt off 4pt phase 0pt] (0, -0.2)  -- (10, -0.2);
        \draw [dash=on 9pt off 3pt phase 0pt]  (0, -0.3)  -- (10, -0.3);
        \draw [dash=on 6pt off 2pt phase 0pt]  (0, -0.4)  -- (10, -0.4);
        \draw [dash=on 4pt off 2pt phase 0pt]  (0, -0.5)  -- (10, -0.5);
        \draw [dash=on 3pt off 2pt phase 0pt]  (0, -0.6)  -- (10, -0.6);
    }
}

% 路面样式(用于修改 Path)
\tikzset{
    ground/.style={
        postaction={draw,decorate,decoration={border,angle=-45,
                    amplitude=0.3cm,segment length=2mm}}},
}


% 绘制一个箭头。
% 共有四个参数:
%   1. (箭头下部矩形)的底的长度
%   2. (箭头下部矩形)的高的长度
%   3. (箭头上部三角形)的底的长度
%   4. (箭头上部三角形)的高的长度
% 调用的例子:
%   \draw [fill=white] (0, 0) pic {arrow={.2}{.3}{.4}{.2}};
%   \draw [red, thick, pattern=dots] (0, 0) pic [rotate=90] {arrow={.2}{.3}{.4}{.2}};
\tikzset{
    pics/arrow/.style n args={4}{
      code = {
        \filldraw [pic actions] (-#1/2, 0) -- (#1/2, 0) -- (#1/2, #2)
            -- (#3/2, #2) -- (0, #2+#4)
            -- (-#3/2, #2) -- (-#1/2, #2)
            -- cycle;
    }}}


% 绘制立方体 ( 代码改编自 https://tex.stackexchange.com/a/48776/121799 )
% 依赖的库:
%   \usetikzlibrary{3d}
% 共有六个参数:
%   1. X 的最小值
%   2. X 的最大值
%   3. Y 的最小值
%   4. Y 的最大值
%   5. Y 的最小值
%   6. Y 的最大值
% 调用举例:以步进 0.5,使用粗线条,绘制一个 3 * 2 * 2 的立方体。
%   \draw [step=0.5, thick] (0, 0) pic {cubes={0}{3}{0}{2}{0}{2}};
\tikzset{
    pics/cubes/.style n args={6}{
      code = {
        \def\XGridMin{#1}
        \def\XGridMax{#2}
        \def\YGridMin{#3}
        \def\YGridMax{#4}
        \def\ZGridMin{#5}
        \def\ZGridMax{#6}
         %
        \begin{scope}[canvas is xy plane at z=\ZGridMax]
            \draw [pic actions] (\XGridMin,\YGridMin) grid (\XGridMax,\YGridMax);
        \end{scope}
        \begin{scope}[canvas is yz plane at x=\XGridMax]
            \draw [pic actions] (\YGridMin,\ZGridMin) grid (\YGridMax,\ZGridMax);
        \end{scope}
        \begin{scope}[canvas is xz plane at y=\YGridMax]
            \draw [pic actions] (\XGridMin,\ZGridMin) grid (\XGridMax,\ZGridMax);
        \end{scope}
    }
  }
}


% 绘制两端有竖线的线段
\tikzset{
  xianduan/.style={xianduan/.cd, #1, wrapper},
  xianduan/.cd,
    above/.store in=\above,  above=0.5em,
    below/.store in=\below,  below=0.5em,
    wrapper/.style={/tikz/.cd, to path={
        ($(\tikztostart)!\above!90:(\tikztotarget)$)  -- ($(\tikztostart)!\below!-90:(\tikztotarget)$)
        ($(\tikztotarget)!\above!-90:(\tikztostart)$) -- ($(\tikztotarget)!\below!90:(\tikztostart)$)
        (\tikztostart) -- (\tikztotarget) \tikztonodes
    }}
}


% 绘制垂足
\tikzset{
    chuizu/.style={chuizu/.cd, #1, wrapper},
    chuizu/.cd,
        size/.store in=\size,   size=0.5em,
        % (从上至下画线时)垂足在显示的方位:right表示显示在右侧(缺省值),left表示在左侧)
        direction/.is choice,
            direction/right/.code = {\def\chuizuangle{-90}},
            direction/left/.code  = {\def\chuizuangle{90}},
            direction=right,
        wrapper/.style={/tikz/.cd, to path={
            ($(\tikztotarget)!\size!0:(\tikztostart)$)
                -- ($(\tikztotarget)!1.414*\size!0.5*\chuizuangle:(\tikztostart)$)
                -- ($(\tikztotarget)!\size!\chuizuangle:(\tikztostart)$)
            (\tikztostart) -- (\tikztotarget) \tikztonodes
        }}
}


% 在数轴上绘制区间
\tikzset{
  infinity interval/.pic={ % 无限区间
    \begin{scope}[infinity interval options/.cd,#1]
      \pgfkeysgetvalue{/tikz/infinity interval options/start}{\start}
      \pgfkeysgetvalue{/tikz/infinity interval options/stop}{\stop}
      \pgfkeysgetvalue{/tikz/infinity interval options/height}{\height}
      \ifthenelse {\start < \stop} {
        \pgfmathsetmacro{\sign}{1}
      }{
        \pgfmathsetmacro{\sign}{-1}
      }
      \pgfmathsetmacro{\posOne}{\sign * 0.2}
      \pgfmathsetmacro{\posTwo}{\sign * 0.5}

      \draw (\start, 0) .. controls (\start+\posOne, \height) and (\start+\posTwo, \height) .. (\start+\posTwo, \height)
        -- (\stop, \height);
    \end{scope}
  },
  infinity interval options/.is family,
  infinity interval options/.cd,
    start/.initial=0,   % 指定参数 start 的默认值
    stop/.initial=1,    % 指定参数 stop 的默认值
    height/.initial=1   % 指定参数 height 的默认值
}

\tikzset{
  interval/.pic={ % 有限区间
    \begin{scope}[interval options/.cd,#1]
      \pgfkeysgetvalue{/tikz/interval options/start}{\start}
      \pgfkeysgetvalue{/tikz/interval options/stop}{\stop}
      \pgfkeysgetvalue{/tikz/interval options/height}{\height}
      \ifthenelse {\start < \stop} {
        \pgfmathsetmacro{\sign}{1}
      }{
        \pgfmathsetmacro{\sign}{-1}
      }
      \pgfmathsetmacro{\posOne}{\sign * 0.2}
      \pgfmathsetmacro{\posTwo}{\sign * 0.5}

      \draw (\start, 0) .. controls (\start+\posOne, \height) and (\start+\posTwo, \height) .. (\start+\posTwo, \height)
        -- (\stop-\posTwo, \height) .. controls (\stop-\posTwo, \height) and (\stop-\posOne, \height) .. (\stop, 0);
      \end{scope}
  },
  interval options/.is family,
  interval options/.cd,
    start/.initial=0,   % 指定参数 start 的默认值
    stop/.initial=1,    % 指定参数 stop 的默认值
    height/.initial=1   % 指定参数 height 的默认值
}


% 绘制 “十字相乘法”
% 共有四个参数:
%   1. a_1
%   2. a_2
%   3. c_1
%   4. c_2
\tikzset{
  pics/cross/.style n args={4}{
    code = {
      \begin{scope}[every node/.style={minimum width=3em}]
        \node [align=right] (a1) at (0, 1) {$#1$};
        \node [align=right] (a2) at (0, 0) {$#2$};
        \node [align=left] (c1) at (3, 1) {$#3$};
        \node [align=left] (c2) at (3, 0) {$#4$};
        \draw (a2.east) -- (c1.west);
        \draw (a1.east) -- (c2.west);
      \end{scope}
    }
  }
}

