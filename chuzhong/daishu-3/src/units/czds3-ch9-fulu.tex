{
\newcommand\labelfulu{附录 \hspace*{0.5em} 平方根的笔算求法}
\nonumsection[\labelfulu]{\Large \labelfulu}
}

求一个正数的平方根,除查平方根表外,还可以用笔算来求。

把一个正整数开平方,首先要确定平方根有几位整数。

我们知道,两个正数里,较大的一个正数的平方也较大。由平方运算得到,
\begin{center}
    \begin{tblr}{columns={$}, row{4}={c}}
        1^2 = 1,        & 9^2 = 81,         \\
        10^2 = 100,     & 99^2 = 9801,      \\
        100^2 = 10000,  & 999^2 = 998001,   \\
         \cdots\cdots,  & \cdots\cdots\juhao
    \end{tblr}
\end{center}
可以看出,一位数的平方是一位数或者二位数,二位数的平方是三位数或者四位数,
三位数的平方是五位数或者六位数, ……; 反过来,

一位数或者二位数的平方根有一位整数,

三位数或者四位数的平方根有两位整数,

五位数或者六位数的平方根有三位整数,

\hspace*{4em} …… \hspace*{4em} …… 。

因此,把被开方数从右向左每两位为一段用一个撇号“$'$” 分开(最左边的一段可以只有一位),
所分得的段数,就是这个数的平方根的整数的位数。
例如,$1156$ 可以分成 $11'56$ 两段,它的平方根有两位整数;
$85264$ 可以分成 $8'52'64$ 三段,它的平方根有三位整数。

其次,要确定这个正整数的算术平方根的最高位上的数。

在两个正数中,较大的数的平方也较大,反过来,在两个正数中,较大的数的算术平方根也较大。
根据这个性质,可以确定一个数的算术平方根的最高位上的数。
例如 $11'56$ 的左边一段是 $11$, $11$ 在 $3^2$ 与 $4^2$ 之间,
由此可知,$1156$ 在 $30^2$ 与 $40^2$ 之间,所以 $1156$ 的平方根的最高位上的数是 $3$;
$8'52'64$ 的左边第一段是 $8$, $8$ 在 $2^2$ 与 $3^2$ 之间,
由此可知,$8'52'64$ 在 $200^2$ 与 $300^2$ 之间,所以 $85264$ 的算术平方根的最高位上的数是 $2$。

第三,确定正整数的平方根的其他位上的数。下面我们通过例题来研究确定的方法。

例如,求 $\sqrt{11'56}$。

我们已经知道 $\sqrt{11'56}$ 有两位整数,最高位上的数是 $3$,即平方根的十位数是 $3$。
如果用 $a$ 代表这个平方根的个位上的数,那么平方根就可以写成 $30 + a$ 的形式。于是
\begin{align*}
    1156 &= (30 + a)^2 \\
         &= 30^2 + 2 \cdot 30a + a^2 \juhao
\end{align*}
从 $1156$ 减去 $30^2 = 900$,得
\begin{center}
\begin{tblr}{columns={$}, column{2}={r}, rows={rowsep=-4pt}}
        & 1156 & \cdots\cdots & (30 + a)^2  \\
    -)  &  900 & \cdots\cdots & 30^2        \\[.5em]
        \cline{1, 2}
        &  256 & \cdots\cdots & 2 \cdot 30a + a^2
\end{tblr}
\end{center}
就是说, $256 = 2 \times 30a + a^2 = (2 \times 30 + a)a$。
根据这个关系,我们可以求出个位数 $a$。

由于 $a$ 是个位数, $2 \times 30$ 要比 $a$ 大得多,
所以把 $2 \times 30 + a$ 看作近似于 $2 \times 30$,
用 $2 \times 30$ 去试除 $256$, 得到试商 $4$。

要确定 $a$ 的值是否等于 $4$, 只要计算当 $a = 4$ 时,
$(2 \cdot 30 + a)a$ 的值是不是 $256$。
由 $(60 + 4) \cdot 4 = 256$, 所以 $a$ 的值恰是 $4$,
因此 $\sqrt{1156} = 34$。

上面所说的计算过程,可以写成下面的竖式:

% “根号”的尾部。
% 用于和表格中的横线 拼接 成一个 “根号”
% 由于只会在本节中使用,所以命令也只定义在这里。
\newcommand{\ghwei} {\tikz [overlay] {
    \draw (-.15em, .7em) node {$\sqrt{}$};
}}

\begin{center}
\begin{tblr}{columns={$, r, colsep=.25em}, rows={rowsep=0pt},
    hline{2} = {4-5}{.5pt},
    hline{4} = {3-5}{.5pt},
    hline{6} = {1-3}{.5pt},
    hline{7} = {4-5}{.5pt},
    %vline{4} = {2}{text=$\sqrt{1}$},
    vline{5} = {2}{text=$'$},
    vline{4} = {4-6}{.5pt},
}
        &               &        &  3 &  4 \\
        &               & \ghwei & 11 & 56 \\
        &               &        &  9 &    \\
        & 20 \times 3 = &     60 &  2 & 56 \\
    +)  &               &      4 &    &    \\
        &               &     64 &  2 & 56 \\
        &               &        &    &  0
\end{tblr}
\end{center}

这里余数是零,表示开得尽,也就是 $34^2 = 1156$, $1156$ 是一个完全平方数。
一般地,如果一个正数恰好是另一个有理数的平方,这个正数就叫做\zhongdian{完全平方数}。

笔算开平方演算的主要步骤是:

1. 把被开平方的整数,从个位起向左每隔两位为一段,用撇号分开;

2. 根据左边第一段里的数,求得算术平方根的最高位上的数;

3. 从第一段里的数减去求得的最高位上数的平方,在它们的差的右边写上第二段数作为第一个余数;

4. 把求得的初商乘以 $20$ 去试除第一个余数,所得的最大整数作为试商
(如果这个最大整数大于或等于 $10$ , 就用 $9$ 做试商);\footnote{录注:以上面的 $256$ 为例,
    $256 = 2 \times 30a + a^2 = (2 \times 30 + a)a = (20 \times 3 + a) \times a \approx (20 \times 3) \times a$,
    这就是 $20$ 的由来。
}

5. 用初商的 $20$ 倍加上这个试商再乘以试商。
如果所得的积小于或等于余数,这个试商就是算术平方根的第二位数;
如果所得的积大于余数,就把试商逐次减小再试,直到积小于或等于余数为止;

6. 用同样的方法,继续求算术平方根的其他各位上的数。


\liti 求%
\begin{xiaoxiaotis}
    \twoInLineXxt[8em]{$\sqrt{1444}$;}{$\sqrt{85264}$。}

\resetxxt
\jie \begin{tblr}[t]{column{1}={16em}}
    \xxt{} & \xxt{} \\
    \begin{tblr}[t]{columns={$, r, colsep=.25em}, rows={rowsep=0pt},
        hline{2} = {4-5}{.5pt},
        hline{4} = {3-5}{.5pt},
        hline{6} = {1-3}{.5pt},
        hline{7} = {4-5}{.5pt},
        vline{5} = {2}{text=$'$},
        vline{4} = {4-6}{.5pt},
    }
            &               &        &  3 &  8 \\
            &               & \ghwei & 14 & 44 \\
            &               &        &  9 &    \\
            & 20 \times 3 = &     60 &  5 & 44 \\
        +)  &               &      8 &    &    \\
            &               &     68 &  5 & 44 \\
            &               &        &    &  0
    \end{tblr} & \begin{tblr}[t]{columns={$, r, colsep=.25em}, rows={rowsep=0pt},
        hline{2} = {2-5}{.5pt},
        hline{4} = {1-5}{.5pt},
        hline{6} = {2-5}{.5pt},
        hline{8} = {3-5}{.5pt},
        vline{3, 4} = {2}{text=$'$},
        vline{2} = {4-5}{.5pt},
        vline{3} = {6-7}{.5pt},
    }
                              &  2 &  9 &  2 \\
                       \ghwei &  8 & 52 & 64 \\
                              &  4 &    &    \\
                           49 &  4 & 52 &    \\
                              &  4 & 41 &    \\
        \SetCell[c=2]{r}{582} &    & 11 & 64 \\
                              &    & 11 & 64 \\
                              &    &    &  0
    \end{tblr} \\
    $\therefore \quad \sqrt{1444} = 38$; &
        $\therefore \quad \sqrt{85264} = 292$。

\end{tblr}

\end{xiaoxiaotis}

\zhuyi
例 1 第 (1) 小题中 $544$ 除以 $60$ 得试商 $9$, 但是 $69 \times 9$ 的积大于 $544$,所以试商改用8;
例 1 第 (2) 小题中 $452$ 除以 $40$ 得试商 $11$,但是试商只能是一位数,所以改用 $9$。


\liti 求 $\sqrt{10404}$。

\jie \hspace*{2em} \begin{tblr}[t]{columns={$, r, colsep=.25em}, rows={rowsep=0pt},
    hline{2} = {2-5}{.5pt},
    hline{4} = {1-5}{.5pt},
    hline{6} = {3-5}{.5pt},
    vline{3, 4} = {2}{text=$'$},
    vline{3} = {4-5}{.5pt},
    column{1} = {0.5em},% 为了减少 根号尾 和首位数字 1 之间的空白
}
                          &  1 &  0 &  2 \\
                   \ghwei &  1 & 04 & 04 \\
                          &  1 &    &    \\
    \SetCell[c=2]{r}{202} &    &  4 & 04 \\
                          &    &  4 & 04 \\
                          &    &    &  0
\end{tblr}

$\therefore \quad \sqrt{10404} = 102$。


\zhuyi 注在第一步计算中,$1$ 减 $1$, 余 $0$, 写下被开方数的下一段得 $4$,
用 $20$ 除 $4$ 不够商 $1$,在算术平方根的第二位写 $0$,再把被开方数第三段写下来,得 $404$。
用 $200$ 去除 $404$ 得试商 $2$。

正小数的开平方,可以仿照上面的方法来演算,不同的是分段的时候,
要从小数点起向左、向右每隔两位用撇号分开,最后一段如果只有一位,必须添零补成两位。
例如 $0.3249$ 分段后应该是 $0.32'49$,
$232.5625$ 分段后应该是 $2'32.56'25$,
$312.512$ 分段后应该是 $3'12.51'20$。
还要注意所得结果的小数点必须与被开方数的小数点对齐。


\liti 求%
\begin{xiaoxiaotis}
    \twoInLineXxt[8em]{$\sqrt{0.3249}$;}{$\sqrt{232.5625}$。}

\resetxxt
\jie \begin{tblr}[t]{column{1}={16em}}
    \xxt{} & \xxt{} \\
    \begin{tblr}[t]{columns={$, r, colsep=.25em}, rows={rowsep=0pt},
        hline{2} = {2-5}{.5pt},
        hline{4} = {1-5}{.5pt},
        hline{6} = {3-5}{.5pt},
        vline{3} = {1,2}{text=$.$},
        vline{4} = {2}{text=$'$},
        vline{3} = {4-5}{.5pt},
        column{1} = {0.5em},
    }
                              &  0 &  5 &  7 \\
                       \ghwei &  0 & 32 & 49 \\
                              &    & 25 &    \\
        \SetCell[c=2]{r}{107} &    &  7 & 49 \\
                              &    &  7 & 49 \\
                              &    &    &  0
    \end{tblr} & \begin{tblr}[t]{columns={$, r, colsep=.25em}, rows={rowsep=0pt},
        hline{2} = {2-5}{.5pt},
        hline{4} = {1-5}{.5pt},
        hline{6} = {2-5}{.5pt},
        hline{8} = {3-5}{.5pt},
        hline{10}= {3-5}{.5pt},
        vline{4} = {1,2}{text=$.$},
        vline{3,5} = {2}{text=$'$},
        vline{2} = {4-5}{.5pt},
        vline{3} = {6-9}{.5pt},
    }
                              &  1 &  5 &  2 &  5 \\
                       \ghwei &  2 & 32 & 56 & 25 \\
                              &  1 &    &    &    \\
                           25 &  1 & 32 &    &    \\
                              &  1 & 25 &    &    \\
        \SetCell[c=2]{r}{302} &    &  7 & 56 &    \\
                              &    &  6 & 04 &    \\
        \SetCell[c=2]{r}{3045}&    &  1 & 52 & 25 \\
                              &    &  1 & 52 & 25 \\
                              &    &    &    &  0 \\
    \end{tblr} \\
    $\therefore \quad \sqrt{0.3249} = 0.57$; &
        $\therefore \quad \sqrt{232.5625} = 15.25$。

\end{tblr}

\end{xiaoxiaotis}


前面三个例子中所给出的被开方数都是完全平方数。
类似地也可以求出非完全平方数的近似平方根。

\liti 求%
\begin{xiaoxiaotis}
    \hspace*{-1em}\begin{tblr}[t]{}
        \xxt{$\sqrt{12.5}$ 的近似值(精确到 $0.01$);} \\
        \xxt{$\sqrt{2}$ 的近似值(精确到 $0.0001$)。}
    \end{tblr}

\resetxxt
\jie \begin{tblr}[t]{column{1}={16em}}
    \xxt{} & \xxt{} \\
    \begin{tblr}[t]{columns={$, r, colsep=.25em}, rows={rowsep=0pt},
        hline{2} = {2-5}{.5pt},
        hline{4} = {1-5}{.5pt},
        hline{6} = {2-5}{.5pt},
        hline{8} = {3-5}{.5pt},
        hline{10} = {3-5}{.5pt},
        vline{3} = {1,2}{text=$.$},
        vline{4,5} = {2}{text=$'$},
        vline{2} = {4-5}{.5pt},
        vline{3} = {6-9}{.5pt},
    }
                              &  3 &  5 &  3 &  5 \\
                       \ghwei & 12 & 50 & 00 & 00 \\
                              &  9 &    &    &    \\
                           65 &  3 & 50 &    &    \\
                              &  3 & 25 &    &    \\
        \SetCell[c=2]{r}{703} &    & 25 & 00 &    \\
                              &    & 21 & 09 &    \\
        \SetCell[c=2]{r}{7065}&    &  3 & 91 & 00 \\
                              &    &  3 & 53 & 25 \\
                              &    &    & 37 & 75
    \end{tblr} & \begin{tblr}[t]{columns={$, r, colsep=.25em}, rows={rowsep=0pt},
        hline{2} = {2-7}{.5pt},
        hline{4} = {1-7}{.5pt},
        hline{6} = {2-7}{.5pt},
        hline{8} = {3-7}{.5pt},
        hline{10}= {3-7}{.5pt},
        hline{12}= {4-7}{.5pt},
        hline{14}= {5-7}{.5pt},
        vline{3} = {1,2}{text=$.$},
        vline{4-7} = {2}{text=$'$},
        vline{2} = {4-5}{.5pt},
        vline{3} = {6-9}{.5pt},
        vline{4} = {10-11}{.5pt},
        vline{5} = {12-13}{.5pt},
    }
                              &  1 &  4 &  1 &  4 &  2 &  1 \\
                       \ghwei &  2 & 00 & 00 & 00 & 00 & 00 \\
                              &  1 &    &    &    &    &    \\
                           24 &  1 & 00 &    &    &    &    \\
                              &    & 96 &    &    &    &    \\
        \SetCell[c=2]{r}{281} &    &  4 & 00 &    &    &    \\
                              &    &  2 & 81 &    &    &    \\
        \SetCell[c=2]{r}{2824}&    &  1 & 19 & 00 &    &    \\
                              &    &  1 & 12 & 96 &    &    \\
        \SetCell[c=3]{r}{28282}&   &    &  6 & 04 & 00 &    \\
                              &    &    &  5 & 65 & 64 &    \\
        \SetCell[c=4]{r}{282841}&  &    &    & 38 & 36 & 00 \\
                              &    &    &    & 28 & 28 & 41 \\
                              &    &    &    & 10 & 07 & 59 \\
    \end{tblr} \\
    $\therefore \quad \sqrt{12.5} \approx 3.54$; &
        $\therefore \quad \sqrt{2} \approx 1.4142$。

\end{tblr}

\end{xiaoxiaotis}


例4 中的 $12.5$ 与 $2$ 都是非完全平方数,笔算开方时可以按照要求的精确度适当补零,再进行开方计算。
必须注意补零时应该使小数部分能够分成每两位一段。

