\subsection{一元二次方程的应用}\label{subsec:11-4}

\liti 两个连续奇数的积是 $323$,求这两个数。

\jie 设较小的一个奇数为 $x$,那么另一个奇数为 $x + 2$。根据题意,得
$$ x(x + 2) = 323 \juhao $$

整理后,得
$$ x^2 + 2x - 323 = 0 \juhao $$

解这个方程,得
$$ x_1 = 17 \nsep x_2 = -19 \juhao $$

奇数可以为正数,也可以为负数,所以 $x = 17$, $x = -19$ 都适合题意。

由 $x = 17$, 得 $x + 2 = 19$;

由 $x = -19$, 得 $x + 2 = -17$。

答:这两个奇数是 $17$,$19$, 或者 $-19$,$-17$。

试一试,如果设这两个奇数中较小的一个为 $x - 1$, 另一个为 $x + 1$,这个题应该怎样解。

\begin{wrapfigure}[14]{r}{6.5cm}
    \centering
    \begin{tikzpicture}[>=Stealth,
    every node/.style={fill=white, inner sep=1pt},
    scale=0.7,
]
    \pgfmathsetmacro{\a}{8}
    \pgfmathsetmacro{\b}{6}
    \pgfmathsetmacro{\c}{1.5}

    \draw [ultra thick] (0, 0) rectangle (\a, \b);

    \draw [ultra thick] (0, 0) rectangle (\c, \c);
    \draw [ultra thick] (\a - \c, 0) rectangle (\a, \c);
    \draw [ultra thick] (0, \b - \c) rectangle (\c, \b);
    \draw [ultra thick] (\a - \c, \b - \c) rectangle (\a, \b);

    \draw [dashed] (\c, \c) rectangle (\a - \c, \b - \c);

    \draw [<->] (0, \b+0.5) to [xianduan={below=0.5cm}] node {$80$} (\a, \b+0.5);
    \draw [<->] (0, \b - \c/2) to node {$x$} (\c, \b - \c/2);
    \draw [<->] (\c, \b - \c/2) to node {$80 - 2x$} (\a - \c, \b - \c/2);

    \draw [<->] (-0.5, 0) to [xianduan={below=0.5cm}]  node [rotate=90] {$60$} (-0.5, \b);
    \draw [<->] (\c/2, 0) to node [rotate=90] {$x$} (\c/2, \c);
    \draw [<->] (\c/2, \c) to node [rotate=90] {$60 - 2x$} (\c/2, \b - \c);
\end{tikzpicture}

    \caption{}\label{fig:11-1}
\end{wrapfigure}

\liti 如图,用一块长 $80$ 厘米,宽 $60$ 厘米的白铁片,在四个角上截去四个相同的小正方形,
然后把四边折起来,做成底面积为 $1500 \, \pflm$ 的没有盖的长方体盒子,截去的小正方形的边长应是多少?

\jie 设小正方形的边长为 $x$ 厘米,那么盒子底面的长及宽分别为
$(80 - 2x)$ 厘米及 $(60 - 2x)$ 厘米。根据题意,得
$$ (80 - 2x)(60 - 2x) = 1500 \juhao $$

整理后,得
$$ x^2 - 70x + 825 = 0 \juhao $$

解这个方程,得
$$ x_1 = 15 \nsep x_2 = 55 \juhao $$

当 $x = 15$ 时,$80 - 2x = 50$, $60 - 2x = 30$;

当 $x = 55$ 时,$80 - 2x = -30$, $60 - 2x = -50$;

但底面的长及宽都不能为负数,所以只能取 $x = 15$。

答:小正方形的边长应是 $15$ 厘米。


\liti 某钢铁厂去年一月份某种钢的产量为 $5000$ 吨,三月份上升到 $7200$ 吨,
这两个月平均每月增长的百分率是多少?

分析:设平均每月增长的百分率为 $x$,那么去年
二月份的产量是 $(5000 + 5000x)$ 吨,就是 $5000(1 + x)$ 吨;
三月份的产是 $[5000(1 + x) + 5000(1 + x)x]$ 吨,就是 $5000(1 + x)^2$吨。
于是可以根据题意列出方程。

\jie 设平均每月增长的百分率为 $x$。根据题意,得
$$ 5000(1 + x)^2 = 7200 \douhao $$
即
$$ (1 + x)^2 = 1.44 \juhao $$

\fengeSuoyi{1 + x = \pm 1.2 \juhao }

由此可得
$$ x_1 = 0.2 \nsep x_2 = -2.2 \juhao $$

$x_2 = -2.2$ 不合题意,所以只能取 $x = 0.2 = 20\%$。

答:平均每月增长的百分率是 $20\%$。


\lianxi
\begin{xiaotis}

\xiaoti{两个连整数的积是 $210$,求这两个数。}

\xiaoti{已知两个数的和等于 $12$,积等于 $23$,求这两个数。}

\xiaoti{解本章第 1 节开始提出的应用题。}

\begin{minipage}{10cm}
    \xiaoti{要做一个容积是 $750 \; \lflm$, 高是 $6$ 厘米,底面的长比宽多 $5$ 厘米的长方体匣子,
    底面的长及宽应该各是多少(精确到 $0.1$ 厘米)?
    }

    \xiaoti{如图,在宽为 $20$ 米、长为 $32$ 米的矩形地面上,修筑同样宽的两条互相垂直的道路,
        余下的部分作为耕地,要使耕地的面积为 $540 \, \pfm$,道路的宽应为多米?
    }

    \xiaoti{某农场的粮食产量在两年内从 $3000$ 吨增加到 $3630$ 吨,平均每年增产的百分率是多少?}
\end{minipage}
\quad
\begin{minipage}{4cm}
    \begin{figure}[H]
        \centering
        \begin{tikzpicture}[>=Stealth,
    every node/.style={fill=white, inner sep=1pt},
]
    \pgfmathsetmacro{\a}{3.2}
    \pgfmathsetmacro{\b}{2}
    \pgfmathsetmacro{\c}{0.2}

    \pgfmathsetmacro{\x}{0.8}
    \pgfmathsetmacro{\y}{1.0}

    \draw [ultra thick, pattern={mylines[angle=45, distance={5pt}]}] (0, 0) rectangle (\a, \b);

    \draw [ultra thick, fill=white] (0, 0) rectangle (\x, \y);
    \draw [ultra thick, fill=white] (\x + \c, 0) rectangle (\a, \y);
    \draw [ultra thick, fill=white] (0, \y + \c) rectangle (\x, \b);
    \draw [ultra thick, fill=white] (\x + \c, \y + \c) rectangle (\a, \b);

    \draw [<->] (0, -0.3) to [xianduan={above=0.3cm}] node {$32 \; cm$} (\a, -0.3);
    \draw [<->] (-0.3, 0) to [xianduan={below=0.3cm}]  node [rotate=90] {$20 \; cm$} (-0.3, \b);
\end{tikzpicture}


        \caption*{(第 5 题)}
    \end{figure}
\end{minipage}
\jiange

\xiaoti{1936 年,老贫农张大爷向地主借了玉米 $25$ 千克,刚过两年,地主按 “利滚利” 计算,
    硬要张爷还 $81$ 千克玉米,算一算,这笔债的年利率是多少?
}

\end{xiaotis}


