\subsection{二次根式的性质}\label{subsec:10-2}
\begin{enhancedline}

我们知道,二次根式 $\sqrt{a} \; (a \geqslant 0)$ 就是 $a$ 的算术平方根的表示式,
因此,研究二次根式的性质,只要研究算术平方根的性质就可以了。

\subsubsection{积的算术平方根}

我们看下面的例子:
\begin{gather*}
    (\sqrt{4 \times 9})^2 = 4 \times 9 = 36 \douhao \\
    (\sqrt{4 \times 9})^2 = (\sqrt{4})^2 \times (\sqrt{9})^2 = 4 \times 9 = 36 \juhao
\end{gather*}
$\sqrt{4 \times 9}$ 与 $\sqrt{4} \times \sqrt{9}$ 又都是正数。
这就说明 $\sqrt{4 \times 9}$ 与 $\sqrt{4} \times \sqrt{9}$ 都是 $36$ 的算术平方根,
而 $36$ 的算术平方根只有一个,所以
$$ \sqrt{4 \times 9} = \sqrt{4} \times \sqrt{9} \juhao $$

一般地,有
\begin{center}
    \framebox{\quad $\sqrt{ab} = \sqrt{a} \cdot \sqrt{b} \quad (a \geqslant 0,\; b \geqslant 0) \juhao$\;}
\end{center}

这就是说:\zhongdian{积的算术平方根,等于积中各因式的算术平方根的积。}

\liti 计算:
\begin{xiaoxiaotis}

    \hspace*{1.5em} \threeInLineXxt[12em]{$\sqrt{16 \times 81}$;}{$\sqrt{0.09 \times 0.25}$;}{$\sqrt{17^2 - 8^2}$。}

\resetxxt
\jie \begin{tblr}[t]{columns={colsep=0pt}}
    \xxt{$\sqrt{16 \times 81} = \sqrt{16} \times \sqrt{81} = 4 \times 9 = 36$;} \\
    \xxt{$\sqrt{0.09 \times 0.25} = \sqrt{0.09} \times \sqrt{0.25} = 0.3 \times 0.5 = 0.15$;} \\
    \xxt{$\sqrt{17^2 - 8^2} = \sqrt{(17 + 8)(17 - 8)} = \sqrt{25} \times \sqrt{9} = 5 \times 3 = 15$。}
\end{tblr}

\end{xiaoxiaotis}

\lianxi
\begin{xiaotis}

\xiaoti{计算:}
\begin{xiaoxiaotis}

    \begin{tblr}{columns={12em, colsep=0pt}}
        \xxt{$\sqrt{49 \times 121}$;}  & \xxt{$\sqrt{81 \times 169}$;}   & \xxt{$\sqrt{9 \times 25 \times 225}$;} \\
        \xxt{$\sqrt{26^2 - 10^2}$;}    & \xxt{$\sqrt{0.65^2 - 0.16^2}$;} & \xxt{$\sqrt{25a^4b^6c^2}$。}
    \end{tblr}

\end{xiaoxiaotis}

\xiaoti{下列各式的计算对不对?为什么?}
\begin{xiaoxiaotis}

    \xxt{$\sqrt{3 ^2 + 4^2} = 3 + 4 = 7$;}

    \xxt{$\sqrt{41^2 - 40^2} = 41 - 40 = 1$。}

\end{xiaoxiaotis}

\end{xiaotis}
\lianxijiange

\liti 化简:
\begin{xiaoxiaotis}

    \hspace*{1.5em} \fourInLineXxt[9em]{$\sqrt{10^2 \times 2}$;}{$\sqrt{48}$;}{$\sqrt{4a^2b^3}$;}{$\sqrt{x^4 + x^2y^2}$。}

\resetxxt
\jie \begin{tblr}[t]{columns={colsep=0pt}}
    \xxt{$\sqrt{10^2 \times 2} = \sqrt{10^2} \times \sqrt{2} = 10\sqrt{2}$;} \\
    \xxt{$\sqrt{48} = \sqrt{4^2 \times 3} = \sqrt{4^2} \times \sqrt{3} = 4\sqrt{3}$;} \\
    \xxt{$\sqrt{4a^2b^3} = \sqrt{2^2 \cdot a^2 \cdot b^2 \cdot b} = 2ab\sqrt{b}$;} \\
    \xxt{$\sqrt{x^4 + x^2y^2} = \sqrt{x^2(x^2 + y^2)} = x\sqrt{x^2 + y^2}$。}
\end{tblr}

\end{xiaoxiaotis}

从例 2 可以看出,根据积的算术平方根的性质,如果被开方数中有的因式能开得尽方,
那么这些因式可以用它们的算术平方根来代替而移到根号外面,从而将式子化简。
反过来,我们也可以把根号外面的非负因式平方后移到根号里面。


\liti 把下列各式中根号外面的因式适当改变后移到根号里面:
\begin{xiaoxiaotis}

    \hspace*{1.5em} \threeInLineXxt[12em]{$5\sqrt{3}$;}{$-3\sqrt{a}$;}{$4b\sqrt{bc}$。}

\resetxxt
\jie \begin{tblr}[t]{columns={colsep=0pt}}
    \xxt{$5\sqrt{3} = \sqrt{5^2 \times 3} = \sqrt{75}$;} \\
    \xxt{$-3\sqrt{a} = -\sqrt{3^2 \cdot a} = -\sqrt{9a}$;} \\
    \xxt{$4b\sqrt{bc} = \sqrt{(4b)^2 \cdot bc} = \sqrt{16b^3c}$。}
\end{tblr}

\end{xiaoxiaotis}

想一想,$-3\sqrt{a}$ 为什么不能写成
$$ \sqrt{(-3)^2a} = \sqrt{9a} \juhao $$


\liti 把下列各式中根号外面的因式适当改变后移到根号里面:
\begin{xiaoxiaotis}

    \hspace*{1.5em} \twoInLineXxt[12em]{$10\sqrt{0.1}$;}{$5\sqrt{\dfrac{1}{5}}$。}

\resetxxt
\jie \begin{tblr}[t]{columns={colsep=0pt}}
    \xxt{$10\sqrt{0.1} = \sqrt{10^2 \times 0.1} = \sqrt{10}$;} \\
    \xxt{$5\sqrt{\dfrac{1}{5}} = \sqrt{5^2 \times \dfrac{1}{5}} = \sqrt{5}$。}
\end{tblr}

\end{xiaoxiaotis}


\lianxi
\begin{xiaotis}

\xiaoti{化简:}
\begin{xiaoxiaotis}

    \begin{tblr}{columns={colsep=0pt}, column{1-3}={9em}}
        \xxt{$\sqrt{18}$;}   & \xxt{$-\sqrt{27 \times 15}$;} & \xxt{$\sqrt{21^2 - 4^2}$;}           & \xxt{$\sqrt{9x}$;} \\
        \xxt{$\sqrt{5a^3}$;} & \xxt{$\sqrt{8x^2y^3}$;}       & \xxt{$\dfrac{1}{6}\sqrt{9a^2bc^3}$;} & \xxt{$\sqrt{16(x + 2)^3}$。}
    \end{tblr}

\end{xiaoxiaotis}

\xiaoti{把下列各式中根号外面的因式适当改变后移到根号里面:}
\begin{xiaoxiaotis}

    \begin{tblr}{columns={9em, colsep=0pt}}
        \xxt{$5\sqrt{2}$;}   & \xxt{$-7\sqrt{3}$;}             & \xxt{$6\sqrt{5}$;} \\
        \xxt{$2\sqrt{0.5}$;} & \xxt{$-12\sqrt{\dfrac{c}{2}}$;} & \xxt{$a\sqrt{\dfrac{b}{a}}$。}
    \end{tblr}

\end{xiaoxiaotis}

\xiaoti{(口答)下列计算对不对?为什么?}
\begin{xiaoxiaotis}

    \xxt{$2a\sqrt{b} = \sqrt{2a^2b}$;}

    \xxt{$-3\sqrt{2} = \sqrt{(-3)^2 \times 2} = \sqrt{18}$;}

    \xxt{$3\sqrt{\dfrac{a}{3}} = \sqrt{a}$。}

\end{xiaoxiaotis}

\end{xiaotis}


\subsubsection{商的算术平方根}

我们看下面的例子:

\hspace*{2em} $\begin{aligned}
    & \left(\sqrt{\dfrac{2}{5}}\right)^2 = \dfrac{2}{5} \douhao \\
    & \left(\dfrac{\sqrt{2}}{\sqrt{5}}\right)^2 = \dfrac{(\sqrt{2})^2}{(\sqrt{5})^2} = \dfrac{2}{5} \juhao
\end{aligned}$ \\
$\sqrt{\dfrac{2}{5}}$ 与 $\dfrac{\sqrt{2}}{\sqrt{5}}$ 又都是正数,
这就说明 $\sqrt{\dfrac{2}{5}}$ 与 $\dfrac{\sqrt{2}}{\sqrt{5}}$
都是 $\dfrac{2}{5}$ 的算术平方根,而 $\dfrac{2}{5}$ 的算术平方根只有一个,所以
$$ \sqrt{\dfrac{2}{5}} = \dfrac{\sqrt{2}}{\sqrt{5}} \juhao $$

一般地,有
\begin{center}
    \framebox{\quad $\sqrt{\dfrac{a}{b}} = \dfrac{\sqrt{a}}{\sqrt{b}} \quad (a \geqslant 0,\; b > 0) \juhao$\;}
\end{center}

这就是说:\zhongdian{商的算术平方根等于被除式的算术平方根除以除式的算术平方根。}


\liti 计算:
\begin{xiaoxiaotis}

    \hspace*{1.5em} \fourInLineXxt[9em]{$\sqrt{\dfrac{4}{9}}$;}
        {$\sqrt{1\dfrac{15}{49}}$;}
        {$\sqrt{\dfrac{3}{100}}$;}
        {$\sqrt{\dfrac{25x^4}{81y^2}}$。}

\resetxxt
\jie \begin{tblr}[t]{columns={colsep=0pt}, rows={rowsep=.5em}}
    \xxt{$\sqrt{\dfrac{4}{9}} = \dfrac{\sqrt{4}}{\sqrt{9}} = \dfrac{2}{3}$;} &
        \xxt{$\sqrt{1\dfrac{15}{49}} = \sqrt{\dfrac{64}{49}} = \dfrac{\sqrt{64}}{\sqrt{49}} = \dfrac{8}{7} = 1\dfrac{1}{7}$;} \\
    \xxt{$\sqrt{\dfrac{3}{100}} = \dfrac{\sqrt{3}}{\sqrt{100}} = \dfrac{1}{10}\sqrt{3}$;} &
        \xxt{$\sqrt{\dfrac{25x^4}{81y^2}} = \dfrac{\sqrt{25x^4}}{\sqrt{81y^2}} = \dfrac{5x^2}{9y}$。}
\end{tblr}

\end{xiaoxiaotis}

再看一个例子:

\hspace*{2em} $\sqrt{\dfrac{a}{b}} = \sqrt{\dfrac{a \cdot b}{b \cdot b}} = \dfrac{\sqrt{ab}}{\sqrt{b^2}} = \dfrac{1}{b}\sqrt{ab} \juhao$

这就是说,如果被开方数是一个分式(或分数),就可以用一个适当的代数式同乘分子与分母,
使分母开得尽方,然后把分母用它的算术平方根来代替而移到根号外面,从而化去根号内的分母。


\liti 化去下列各式中根号内的分母:
\begin{xiaoxiaotis}

    \hspace*{1.5em} \begin{tblr}{columns={colsep=0pt}, column{1-3}={9em}}
        \xxt{$\sqrt{\dfrac{2}{3}}$;} & \xxt{$\sqrt{1\dfrac{1}{7}}$;}
            & \xxt{$\sqrt{\dfrac{4x}{3y}}$;} & \xxt{$\sqrt{\dfrac{a - 5}{a + 5}} \quad (a > 5)$。}
    \end{tblr}

\resetxxt
\jie \begin{tblr}[t]{columns={colsep=0pt}, rows={rowsep=.5em}}
    \xxt{$\sqrt{\dfrac{2}{3}} = \sqrt{\dfrac{2 \times 3}{3 \times 3}} = \dfrac{1}{3}\sqrt{6}$;} \\
    \xxt{$\sqrt{1\dfrac{1}{7}} = \sqrt{\dfrac{8}{7}} = \sqrt{\dfrac{8 \times 7}{7 \times 7}} = \sqrt{\dfrac{2^2 \times 2 \times 7}{7 \times 7}} = \dfrac{2}{7}\sqrt{14}$;} \\
    \xxt{$\sqrt{\dfrac{4x}{3y}} = \sqrt{\dfrac{4x \cdot 3y}{3y \cdot 3y}} = \dfrac{2}{3y} \sqrt{3xy}$;} \\
    \xxt{$\sqrt{\dfrac{a - 5}{a + 5}} = \sqrt{\dfrac{(a - 5)(a + 5)}{(a + 5)^2}} = \dfrac{1}{a + 5} \sqrt{a^2 - 25}$。}
\end{tblr}

\end{xiaoxiaotis}


\lianxi
\begin{xiaotis}

\xiaoti{计算:}
\begin{xiaoxiaotis}

    \begin{tblr}{columns={colsep=0pt}, column{2-4}={9em}, rows={rowsep=.5em}}
        \xxt{$\sqrt{\dfrac{25}{64}}$;}   & \xxt{$\sqrt{\dfrac{4}{225}}$;} & \xxt{$\sqrt{\dfrac{0.01}{0.16}}$;} & \xxt{$\sqrt{\dfrac{36 \times 9}{121}}$;} \\
        \xxt{$\sqrt{\dfrac{0.04 \times 144}{0.49 \times 169}}$;} & \xxt{$\sqrt{4\dfrac{1}{9}}$;} & \xxt{$\sqrt{\dfrac{6}{4a^2}}$;} & \xxt{$\sqrt{\dfrac{49m^2n}{9c^2}}$。}
    \end{tblr}

\end{xiaoxiaotis}


\xiaoti{化去下列各式中根号内的分母:}
\begin{xiaoxiaotis}

    \begin{tblr}{columns={colsep=0pt}, column{1-3}={9em}, rows={rowsep=.5em}}
        \xxt{$\sqrt{\dfrac{1}{2}}$;}   & \xxt{$\sqrt{\dfrac{7}{12}}$;} & \xxt{$\sqrt{5\dfrac{1}{3}}$;} & \xxt{$\sqrt{6\dfrac{5}{6}}$;} \\
        \xxt{$\sqrt{\dfrac{27}{2x}}$;} & \xxt{$\sqrt{\dfrac{n}{3m^2}}$;} & \xxt{$\sqrt{\dfrac{a}{50}}$;} & \xxt{$\sqrt{\dfrac{a - b}{a + b}} \quad (a > b)$。}
    \end{tblr}

\end{xiaoxiaotis}



\xiaoti{(口答)下列各式的计算对不对?为什么?}
\begin{xiaoxiaotis}

    \begin{tblr}{columns={18em, colsep=0pt}, rows={rowsep=.5em}}
        \xxt{$\sqrt{\dfrac{3}{4}} = 2\sqrt{3}$;} & \xxt{$\sqrt{\dfrac{3}{2}} = \dfrac{1}{2}\sqrt{3}$;} \\
        \xxt{$\sqrt{\dfrac{8}{2}} = \sqrt{2}$;}  & \xxt{$\sqrt{\dfrac{a}{9b}} = \dfrac{1}{3b}\sqrt{a}$。}
    \end{tblr}

\end{xiaoxiaotis}

\end{xiaotis}

\end{enhancedline}
