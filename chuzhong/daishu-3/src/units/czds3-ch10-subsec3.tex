\subsection{最简二次根式和同类二次根式}\label{subsec:10-3}
\begin{enhancedline}

\subsubsection{最简二次根式}

我们看下面的例子:

$\sqrt{a^3b} = \sqrt{a^2 \cdot ab} = \sqrt{a^2} \cdot \sqrt{ab} = a\sqrt{ab}$,

$a^2\sqrt{\dfrac{b}{a}} = a^2 \sqrt{\dfrac{ab}{a^2}} = \dfrac{a^2}{a}\sqrt{ab} = a\sqrt{ab}$。

二次根式 $\sqrt{a^3b}$ 和 $a^2\sqrt{\dfrac{b}{a}}$ \footnote{为了方便,我们把形如 $b\sqrt{a} \; (a \geqslant0)$
    的式子也叫做二次根式,如 $10\sqrt{2}$, $-\sqrt{3}$, $2ab\sqrt{c^2 +1}$ 等。
}
的形式虽然不同,但是它们都可以化成形式比较简单的二次根式 $a\sqrt{ab}$。
与二次根式和  $\sqrt{a^3b}$ 和 $a^2\sqrt{\dfrac{b}{a}}$ 比较,
二次根式 $a\sqrt{ab}$ 满足下列两个条件:

(1) 被开方数的每一个因式的指数都小于根指数 $2$ ;

(2) 被开方数不含分母。

我们把符合这两个条件的二次根式,叫做\zhongdian{最简二次根式}。
例如,$4\sqrt{5a}$, $\dfrac{\sqrt{y}}{2}$, $\sqrt{a^2 + b}$ 等都是最简二次根式;
而 $\sqrt{4a^3}$, $\sqrt{\dfrac{c}{3}}$, $\sqrt{8}$ 等就不是最简二次根式。

一个二次根式,如果不是最简二次根式,可以用上节所说的方法,
化去根号内的分母,并把被开方数中能开得尽方的因式用算术平方根代替移到根号外面,
把它化成最简二次根式。

\liti[0] 把下列根式化成最简二次根式:
\begin{xiaoxiaotis}

    \hspace*{1.5em} \fourInLineXxt[9em]{$\sqrt{12}$;}{$\sqrt{\dfrac{1}{3}}$;}{$4\sqrt{1\dfrac{1}{2}}$;}{$x^2\sqrt{\dfrac{y}{x}}$。}

\resetxxt
\jie \begin{tblr}[t]{columns={colsep=0pt}, rows={rowsep=0.5em}}
    \xxt{$\sqrt{12} = \sqrt{2^2 \times 3} = 2\sqrt{3}$;} \\
    \xxt{$\sqrt{\dfrac{1}{3}} = \sqrt{\dfrac{3}{3 \times 3}} = \dfrac{\sqrt{3}}{3}$;} \\
    \xxt{$4\sqrt{1\dfrac{1}{2}} = 4\sqrt{\dfrac{3}{2}} = 4\sqrt{\dfrac{6}{4}} = 2\sqrt{6}$;} \\
    \xxt{$x^2\sqrt{\dfrac{y}{x}} = x^2\sqrt{\dfrac{xy}{x^2}} = x\sqrt{xy}$。}
\end{tblr}

\end{xiaoxiaotis}

\zhuyi 把二次根式化成最简二次根式时,往往需要把被开方数分解质因数(或分解因式)。


\lianxi
\begin{xiaotis}

\xiaoti{下列根式中,哪些是最简二次根式? 哪些不是? 如果不是,就把它化成最简二次根式。}
\begin{xiaoxiaotis}

    \begin{tblr}{columns={12em, colsep=0pt}, rows={rowsep=0.5em}}
        \xxt{$\sqrt{45}$;}      & \xxt{$\sqrt{25a^3}$;}          & \xxt{$\sqrt{\dfrac{ab}{4}}$;} \\
        \xxt{$\sqrt{14}$;}      & \xxt{$\sqrt{\dfrac{b}{a}}$;}   & \xxt{$\dfrac{\sqrt{2}}{2}$;} \\
        \xxt{$\sqrt{6a^2b^3}$;} & \xxt{$\sqrt{\dfrac{4y}{5x}}$;} & \xxt{$\sqrt{a + b^2}$。}
    \end{tblr}

\end{xiaoxiaotis}


\xiaoti{把下列根式化成最简二次根式。}
\begin{xiaoxiaotis}

    \begin{tblr}{columns={colsep=0pt}, column{1,2} = {12em}, rows={rowsep=0.5em}}
        \xxt{$3\sqrt{216}$;}      & \xxt{$\sqrt{32}$;}          & \xxt{$\sqrt{\dfrac{8}{9}}$;} \\
        \xxt{$\sqrt{1\dfrac{1}{3}}$;}  & \xxt{$\sqrt{\dfrac{20a^2b}{a}}$;}  & \xxt{$2\sqrt{a^3b^3}$;} \\
        \xxt{$x^2\sqrt{\dfrac{1}{8x^3}}$;} & \xxt{$\sqrt{\dfrac{ab}{(a + b)^2}}$;} & \xxt{$(a - b)\sqrt{\dfrac{1}{a - b}} \; (a > b)$。}
    \end{tblr}

\end{xiaoxiaotis}

\end{xiaotis}



\subsubsection{同类二次根式}

把 $\sqrt{12}$ 与 $\sqrt{\dfrac{1}{3}}$ 化成最简二次根式,得到

\hspace*{4em} $\begin{aligned}
    & \sqrt{12} = \sqrt{2^2 \times 3} = 2\sqrt{3} \douhao \\
    & \sqrt{\dfrac{1}{3}} = \sqrt{\dfrac{3}{3 \times 3}} = \dfrac{1}{3}\sqrt{3} \juhao
\end{aligned}$

二次根式 $2\sqrt{3}$ 与 $\dfrac{1}{3}\sqrt{3}$ 的被开方数相同,都是 $3$。
几个二次根式化成最简二次根式以后,如果被开方数相同,这几个二次根式就叫做\zhongdian{同类二次根式}。
例如:$\sqrt{12}$, $\sqrt{\dfrac{1}{3}}$, $\dfrac{1}{2}\sqrt{3}$ 是同类二次根式;
$a\sqrt{ab}$, $3\sqrt{ab}$ 也是同类二次根式。
而 $\sqrt{2}$ 与 $\sqrt{3}$ 不是同类二次根式;
$\sqrt{a}$ 与 $\sqrt{3a}$ 也不是同类二次根式。

我们知道,在多项式中,遇到同类项就可以合并。
同样,在几个二次根式的和里,遇到同类二次根式也可以合并。

\liti 下列二次根式中,哪些是同类二次根式?

\hspace*{2em} $\sqrt{2}$\nsep $\sqrt{75}$\nsep $\sqrt{\dfrac{1}{50}}$\nsep $\sqrt{\dfrac{1}{27}}$\nsep $\sqrt{3}$\nsep
$\dfrac{2}{3}\sqrt{8ab^3}$\nsep $6b\sqrt{\dfrac{a}{2b}}$。

\jie $\because$ \quad \begin{tblr}[t]{columns={colsep=0pt}, rows={rowsep=0.5em}}
    $\sqrt{75} = \sqrt{5^2 \times 3} = 5\sqrt{3}$; \\
    $\sqrt{\dfrac{1}{50}} = \sqrt{\dfrac{2}{50 \times 2}} = \dfrac{1}{10}\sqrt{2}$; \\
    $\sqrt{\dfrac{1}{27}} = \sqrt{\dfrac{3}{27 \times 3}} = \dfrac{1}{9}\sqrt{3}$; \\
    $\dfrac{2}{3}\sqrt{8ab^3} = \dfrac{2}{3} \cdot 2b\sqrt{2ab} = \dfrac{4b}{3}\sqrt{2ab}$; \\
    $6b\sqrt{\dfrac{a}{2b}} = 6b\sqrt{\dfrac{a \cdot 2b}{2b \cdot 2b}} = 3\sqrt{2ab}$。
\end{tblr}

$\therefore$ \quad \begin{tblr}[t]{columns={colsep=0pt}, rows={rowsep=0.5em}}
    $\sqrt{2}$, $\sqrt{\dfrac{1}{50}}$ 是同类二次根式; \\
    $\sqrt{75}$, $\sqrt{\dfrac{1}{27}}$, $\sqrt{3}$ 是同类二次根式; \\
    $\dfrac{2}{3}\sqrt{8ab^3}$, $6b\sqrt{\dfrac{a}{2b}}$ 是同类二次根式。
\end{tblr}


\liti 合并下列各式中的同类二次根式:
\begin{xiaoxiaotis}

    \hspace*{1.5em} \begin{tblr}[t]{columns={colsep=0pt}, column{1}={22em}}
        \xxt{$2\sqrt{2} - \dfrac{1}{2}\sqrt{3} + \dfrac{1}{3}\sqrt{2} - \sqrt{2} + \sqrt{3}$;}
            & \xxt{$3\sqrt{xy} - a\sqrt{xy} + b\sqrt{xy}$。}
    \end{tblr}

\resetxxt
\jie \begin{tblr}[t]{columns={colsep=0pt}, column{1}={22em}}
    \xxt{$\begin{aligned}[t]
        & 2\sqrt{2} - \dfrac{1}{2}\sqrt{3} + \dfrac{1}{3}\sqrt{2} - \sqrt{2} + \sqrt{3} \\
        & = \left(2 + \dfrac{1}{3} - 1\right)\sqrt{2} + \left(-\dfrac{1}{2} + 1\right)\sqrt{3} \\
        &= \dfrac{4}{3}\sqrt{2} + \dfrac{1}{2}\sqrt{3} \fenhao
    \end{aligned}$} & \xxt{$\begin{aligned}[t]
        & 3\sqrt{xy} - a\sqrt{xy} + b\sqrt{xy} \\
        & = (3 - a + b)\sqrt{xy} \juhao
    \end{aligned}$}
\end{tblr}

\end{xiaoxiaotis}


\lianxi
\begin{xiaotis}

\xiaoti{下列各组里的二次根式是不是同类二次根式?}
\begin{xiaoxiaotis}

    \begin{tblr}{columns={18em, colsep=0pt}, rows={rowsep=0.5em}}
        \xxt{$\sqrt{63}$, $\sqrt{28}$;}   & \xxt{$\sqrt{12}$, $\sqrt{27}$, $4\sqrt{\dfrac{1}{3}}$;} \\
        \xxt{$\sqrt{4x^3}$, $2\sqrt{2x}$;}  & \xxt{$\sqrt{18}$, $\sqrt{50}$, $2\sqrt{\dfrac{2}{9}}$;} \\
        \xxt{$\sqrt{2x}$, $\sqrt{2a^2x^3}$, $\sqrt{50xy^2}$。}
    \end{tblr}

\end{xiaoxiaotis}


\xiaoti{合并下列各式中的同类二次根式:}
\begin{xiaoxiaotis}

    \begin{tblr}{columns={18em, colsep=0pt}, rows={rowsep=0.5em}}
        \xxt{$6\sqrt{a} + 2\sqrt{b} - 4\sqrt{a} + 3\sqrt{b}$;} & \xxt{$\sqrt{5} + \sqrt{3} + 2\sqrt{5} - \dfrac{\sqrt{3}}{3} - 3\sqrt{5}$;} \\
        \xxt{$6\sqrt{3} + \sqrt{0.12} + \sqrt{48}$;} &     \xxt{$\dfrac{5}{2}\sqrt{xy} - 2\sqrt{xy} - \dfrac{\sqrt{xy}}{2}$。}
    \end{tblr}

\end{xiaoxiaotis}

\end{xiaotis}

\end{enhancedline}
