\subsection{无理方程}\label{subsec:11-9}
\begin{enhancedline}

我们来看下面的方程:
$$ 8 + x + \sqrt{64 + x^2} = 24 \juhao $$

这个方程的未知数 $x$ 含在根号下。象这样根号下含有未知数的方程,叫做\zhongdian{无理方程}\footnote{根号下含有字母的式子叫做无理式}。
例如,$x - \sqrt{x - 1} = 3$, $\sqrt{2x - 4} + 1 = \sqrt{x + 5}$, $\sqrt{2x - 3} + \dfrac{2}{\sqrt{2x - 3}} = \sqrt{5x - 1}$
等都是无理方程。但是象 $x^2 + 2\sqrt{2}x - 1 = 0$, $\dfrac{x}{\sqrt{2} - 1} + \dfrac{1}{x - 2} = 1$ 等都不是无理方程,
而是整式方程或分式方程。

整式方程和分式方程统称\zhongdian{有理方程}。

下面我们研究无理方程的解法。例如,解方程
$$ \sqrt{2x^2 + 7x} - 2 = x \juhao $$

解这个方程,可以先移项,把被开方数中含有未知数的根式放在方程的一边,其余的移到另一边,得
$$ \sqrt{2x^2 + 7x} = x + 2 \juhao $$

两边平方,得到一个有理方程
$$ 2x^2 + 7x = x^2 + 4x + 4 \juhao $$

整理后,得
$$ x^2 + 3x - 4 = 0 \juhao $$

解这个方程,得
$$ x_1 = 1 \nsep x_2 = -4 \juhao $$

检验:把 $x = 1$ 代入原方程,

$\zuobian = \sqrt{2 \times 1^2 + 7 \times 1} - 2 = \sqrt{2 + 7} - 2 = 3 - 2 = 1$,

$\youbian = 1$,

$\therefore$ \quad  $x = 1$ 是原方程的根;

把 $x = -4$ 代入原方程,

$\zuobian = \sqrt{2 \times (-4)^2 + 7 \times (-4)} - 2 = 0$,

$\youbian = -4$,

$\therefore$ \quad  $x = -4$ 是增根。

因此原方程的根是 $x = 1$。

从上例可知,在解无理方程时,为了把无理方程变形为有理方程,需要将方程的两边都乘方相同的次数,
这样就有产生增根的可能。因此,解无理方程时,必须把变形后得到的有理方程的根,代入原方程进行检验,
如果适合,就是原方程的根,如果不适合,就是增根。

\liti 解方程 $\sqrt{2x - 4} - \sqrt{x + 5} = 1$。

\jie 移项,得
$$ \sqrt{2x - 4} = 1 + \sqrt{x + 5} \juhao $$

两边平方,得
$$ 2x - 4 = 1 + 2\sqrt{x + 5} + x + 5 \douhao $$
即
$$ x - 10 = 2\sqrt{x + 5} \juhao $$

两边再平方,得
$$ x^2 - 20x + 100 = 4(x + 5) \douhao $$
即
$$ x^2 - 24x + 80 = 0 \douhao $$

解这个方程,得
$$ x_1 = 4 \nsep x_2 = 20 \juhao $$

检验:把 $x = 4$ 代入原方程,

$\zuobian = \sqrt{2 \times 4 - 4} - \sqrt{4 + 5} = 2 - 3 = -1$,

$\youbian = 1$,

$\therefore$ \quad  $x = 4$ 是增根。

把 $x = 20$ 代入原方程,

$\zuobian = \sqrt{2 \times 20 - 4} - \sqrt{20 + 5} = 6 - 5 = 1$,

$\youbian = 1$,

$\therefore$ \quad  $x = 20$ 是原方程的根。

因此原方程的根是 $x = 20$。

注意:这个方程,先移项,使左边只有一个被开方数中含有未知数的根式,解起来比较简便。


\liti 解方程 $2x^2 + 3x - 5\sqrt{2x^2 + 3x + 9} + 3 = 0$。

分析:这个方程可变形为
$$ 2x^2 + 3x + 9 - 5\sqrt{2x^2 + 3x + 9} - 6 = 0 \juhao $$

这里,$2x^2 + 3x + 9$ 是 $\sqrt{2x^2 + 3x + 9}$ 的平方。
如果设 $\sqrt{2x^2 + 3x + 9} = y$,原方程就可变为关于 $y$ 的一元二次方程。

\jie 设 $\sqrt{2x^2 + 3x + 9} = y$,那么 $2x^2 + 3x + 9 = y^2$,
因此 $2x^2 + 3x = y^2 - 9$。于是原方程变为
$$ y^2 - 9 - 5y + 3 = 0 \douhao $$
即
$$ y^2 - 5y - 6 = 0 \juhao $$

解这个方程,得
$$ y_1 = -1 \nsep y_2 = 6 \juhao $$

当 $y = -1$ 时,$\sqrt{2x^2 + 3x + 9} = -1$,根据算术根的定义,
$\sqrt{2x^2 + 3x + 9}$ 不可能等于一个负数,所以方程 $\sqrt{2x^2 + 3x + 9} = -1$ 无解。

当 $y = 6$ 时,$\sqrt{2x^2 + 3x + 9} = 6$,两边平方,得
$$ 2x^2 + 3x + 9 = 36 \douhao $$
即
$$ 2x^2 + 3x - 27 = 0 \juhao $$

解这个方程,得
$$ x_1 = 3 \nsep x_2 = -\dfrac{9}{2} \juhao $$

检验:把 $x = 3$, $x = -\dfrac{9}{2}$ 分别代入原方程都适合,所以它们都是原方程的根。

从而原方程的根是
$$ x_1 = 3 \nsep x_2 = -\dfrac{9}{2} \juhao $$


\lianxi
\begin{xiaotis}

\xiaoti{(口答)下列方程是否有解?为什么?}
\begin{xiaoxiaotis}

    \begin{tblr}{columns={18em, colsep=0pt}}
        \xxt{$\sqrt{x^2 + 3x + 2} = -4$;} & \xxt{$\sqrt{x + 1} + 3 = 2$。}
    \end{tblr}
\end{xiaoxiaotis}


\xiaoti{解下列方程:}
\begin{xiaoxiaotis}

    \begin{tblr}{columns={18em, colsep=0pt}}
        \xxt{$\sqrt{2x + 3} = x$;}                 & \xxt{$\sqrt{2x + 3} = -x$;} \\
        \xxt{$x + \sqrt{x - 2} = 2$;}              & \xxt{$x - \sqrt{x - 2} = 2$;} \\
        \xxt{$\sqrt{1 - x} + \sqrt{12 + x} = 5$;}  & \xxt{$\sqrt{3x - 2} + \sqrt{x + 3} = 3$。}
    \end{tblr}
\end{xiaoxiaotis}


\xiaoti{用换元法解下列方程:}
\begin{xiaoxiaotis}

    \begin{tblr}{columns={18em, colsep=0pt}}
        \xxt{$x^2 + 8x + \sqrt{x^2 + 8x} = 12$;} & \xxt{$x^2 - 3x - \sqrt{x^2 - 3x + 5} = 1$。}
    \end{tblr}
\end{xiaoxiaotis}

\end{xiaotis}

\end{enhancedline}

