\fuxiti
\begin{enhancedline}
\begin{xiaotis}

\xiaoti{\footnotemark 计算}
\footnotetext{第1 ~\, 5 题是复习二次根式的内容的,可以在学习本章的过程中,根据情况选用。}
\begin{xiaoxiaotis}

    \begin{tblr}{columns={colsep=0pt}, column{1,2}={12em}}
        \xxt{$\sqrt{7^2}$;} & \xxt{$\sqrt{(-7)^2}$;} & \xxt{$\sqrt{(x - y)^2} \quad (x > y)$;} \\
        \xxt{$\left(\sqrt{\dfrac{2}{15}}\right)^2$;} & \xxt{$\sqrt{0.49^2}$;} & \xxt{$\sqrt{a^2 - 14a + 49} \quad (a < 7)$。}
    \end{tblr}
\end{xiaoxiaotis}


\xiaoti{计算:}
\begin{xiaoxiaotis}

    \begin{tblr}{columns={12em, colsep=0pt}, rows={rowsep=0.5em}}
        \xxt{$\sqrt{529 \times 289}$;} & \xxt{$\sqrt{68.89 \times 0.0009}$;} & \xxt{$\sqrt{65^2 - 16^2}$;} \\
        \xxt{$\sqrt{0.17^2 - 0.08^2}$;} & \xxt{$\sqrt{\dfrac{625}{1089}}$;} & \xxt{$\sqrt{\dfrac{0.49 \times 121}{361 \times 0.81}}$;} \\
        \xxt{$\sqrt{\dfrac{1.21}{4.41} \times 49}$;} & \SetCell[c=2]{l}\xxt{$\sqrt{\dfrac{2.25x^6}{0.25y^2}} \quad (x > 0,\; y < 0)$。}
    \end{tblr}
\end{xiaoxiaotis}


\xiaoti{计算:}
\begin{xiaoxiaotis}

    \begin{tblr}{columns={colsep=0pt}, column{1}={18em}, rows={rowsep=0.5em}}
        \xxt{$3\sqrt{a} + 5\sqrt{b^3} + 6\sqrt{a^5} - 2\sqrt{b}$;}
            & \xxt{$\sqrt{27x} + \sqrt{\dfrac{x}{3}} - \sqrt{0.03x}$;} \\
        \xxt{$a\sqrt{\dfrac{b}{a}} + b\sqrt{ab} - b\sqrt{\dfrac{1}{ab}}$;}
            & \xxt{$\sqrt{\dfrac{x}{y}} + \sqrt{\dfrac{y}{x}} - \sqrt{xy} + y\sqrt{\dfrac{1}{xy}} - x\sqrt{\dfrac{1}{xy}}$。}
    \end{tblr}
\end{xiaoxiaotis}


\xiaoti{计算:}
\begin{xiaoxiaotis}

    \xxt{$\left(\sqrt{3ab} + \sqrt{\dfrac{b}{3a}} - \sqrt{\dfrac{27a}{b}} + \sqrt{\dfrac{a}{3b}}\right) \cdot \sqrt{3ab}$;}

    \xxt{$(3\sqrt{x} - \sqrt{5}) (\sqrt{5} + 3\sqrt{x})$;}

    \xxt{$(m - \sqrt{n}) (\sqrt{n} + m)$;}

    \xxt{$(\sqrt{x} + \sqrt{y}) (x + y - \sqrt{xy})$;}

    \xxt{$(\sqrt{a} + \sqrt{b} - \sqrt{c}) (\sqrt{a} - \sqrt{b} + \sqrt{c})$。}

\end{xiaoxiaotis}


\xiaoti{把下列各式的分母有理化:}
\begin{xiaoxiaotis}

    \begin{tblr}{columns={12em, colsep=0pt}, rows={rowsep=0.5em}}
        \xxt{$\dfrac{\sqrt{3}}{\sqrt{8}}$;} & \xxt{$\dfrac{a}{\sqrt{27a}}$;} & \xxt{$\dfrac{a - b}{\sqrt{a} + \sqrt{b}}$;} \\
        \xxt{$\dfrac{ab}{a\sqrt{b} + b\sqrt{a}}$;} & \xxt{$\dfrac{\sqrt{b} - a}{a + \sqrt{b}}$;} & \xxt{$\dfrac{1 - xy}{\sqrt{x} + x\sqrt{y}}$;} \\
        \xxt{$\dfrac{x - 1}{\sqrt{xy} + \sqrt{y}}$;} & \xxt{$\dfrac{x - y}{\sqrt{\dfrac{1}{x}} + \sqrt{\dfrac{1}{y}}}$。}
    \end{tblr}
\end{xiaoxiaotis}

\begin{center}
    * \hspace*{3em} * \hspace*{3em} * \hspace*{3em} * \hspace*{3em} *
\end{center}

\xiaoti{指出下列各式中的 $x$:}
\begin{xiaoxiaotis}

    \begin{tblr}{columns={18em, colsep=0pt}}
        \xxt{$35.23 = 3.523 \times 10^x$;}     & \xxt{$3.523 \times 10^x = 0.003523$;} \\
        \xxt{$3.523 \times 10^x = 35230000$;} & \xxt{$3.523 \times 10^x = 3.523$。}
    \end{tblr}
\end{xiaoxiaotis}


\xiaoti{光的速度每秒约 $3 \times 10^5$ 公里,太阳与地球的距离约是 $1.5 \times 10^8$ 公里,
    求光从太阳射到地球约需多少时间(保留一个有效数字)。
}


\xiaoti{$18$ 克水中有 $6.02 \times 10^{23}$ 个水分子,计算一个水分子的重量,用科学记数法表示(保留两个有效数字)。}

\xiaoti{下列各式在什么条件下有意义?}
\begin{xiaoxiaotis}

    \begin{tblr}{columns={9em, colsep=0pt}}
        \xxt{$\sqrt{x}$;} & \xxt{$\sqrt[4]{x - 1}$;} & \xxt{$\dfrac{1}{\sqrt{x}}$;} & \xxt{$\sqrt[3]{-x}$。}
    \end{tblr}
\end{xiaoxiaotis}


\xiaoti{计算:}
\begin{xiaoxiaotis}

    \begin{tblr}{columns={colsep=0pt}, column{1}={12em}}
        \xxt{$(\sqrt[3]{-3.8})^3$;} & \xxt{$\sqrt[3]{-27}$;} & \xxt{$\sqrt[\uproot{6}5]{\dfrac{32}{243}}$;} \\
        \xxt{$\sqrt[6]{(-5)^6}$;} & \xxt{$\sqrt[4]{(1 - a)^4} \; (a > 1)$;} & \xxt{$\sqrt[8]{(m - n)^8} \; (m < n)$。}
    \end{tblr}
\end{xiaoxiaotis}


\xiaoti{根据下列条件计算 $a + \sqrt[6]{(a - 1)^6}$:}
\begin{xiaoxiaotis}

    \begin{tblr}{columns={18em, colsep=0pt}}
        \xxt{$a \geqslant 1$;} & \xxt{$a < 1$。}
    \end{tblr}
\end{xiaoxiaotis}


\xiaoti{化下列双重根式为单根式:}
\begin{xiaoxiaotis}

    \begin{tblr}{columns={18em, colsep=0pt}}
        \xxt{$\sqrt{x^{-3}y^2 \sqrt[3]{xy^2}} \quad (y \geqslant 0)$;} & \xxt{$\sqrt[3]{-2\sqrt{2}}$。}
    \end{tblr}
\end{xiaoxiaotis}


\xiaoti{计算下列各式:}
\begin{xiaoxiaotis}

    \begin{tblr}{columns={18em, colsep=0pt}, rows={rowsep=0.5em}}
        \xxt{$\dfrac{a^{\frac{1}{2}} - b^{\frac{1}{2}}}{a^{\frac{1}{2}} + b^{\frac{1}{2}}} + \dfrac{a^{\frac{1}{2}} + b^{\frac{1}{2}}}{a^{\frac{1}{2}} - b^{\frac{1}{2}}}$;}
            & \xxt{$\dfrac{(a + b)^{-1} - (a - b)^{-1}}{(a + b)^{-1} + (a - b)^{-1}}$;} \\
        \xxt{$\dfrac{a^{-2} - b^{-2}}{a^{-1} + b^{-1}} + \dfrac{1}{b} - \dfrac{1}{a}$;}
            & \xxt{$\left(\dfrac{e^2 + e^{-2}}{2}\right)^2 - \left(\dfrac{e^2 - e^{-2}}{2}\right)^2$;} \\ % 原书中,本题中 e 的指数看不清楚。盲测为 2。
        \xxt{$(a^2 - 2 + a^{-2}) \div (a^2 - a^{-2})$。}
    \end{tblr}
\end{xiaoxiaotis}


\xiaoti{计算:}
\begin{xiaoxiaotis}

    \xxt{$125^{\frac{2}{3}} + \left(\dfrac{1}{2}\right)^{-2} + 343^{\frac{1}{3}} - \left(\dfrac{1}{27}\right)^{-\frac{1}{3}}$;}

    \xxt{$\left(\dfrac{4}{9}\right)^{\frac{1}{2}} + (-5.6)^0 - \left(2\dfrac{10}{27}\right)^{-\frac{2}{3}} + 0.125^{-\frac{1}{3}}$;}

    \xxt{$\dfrac{(a^{-2}b^{-2}) (-4a^{-1}b)}{12a^{-4}b^{-2}c}$;}

    \xxt{$(a^2b)^{\frac{1}{2}} \times (ab^2)^{-2} \div \left(\dfrac{b}{a^2}\right)^{-3}$。}

\end{xiaoxiaotis}


\xiaoti{指出下列各式中的 $x$:}
\begin{xiaoxiaotis}

    \begin{tblr}{columns={12em, colsep=0pt}}
        \xxt{$5^x = 125$;} & \xxt{$4^x = 1$;} & \xxt{$7^x = \sqrt[3]{7}$;} \\
        \xxt{$\dfrac{1}{8} = 2^x$;} & \xxt{$\sqrt{3} = 3^x$;} & \xxt{$\dfrac{1}{\sqrt[3]{10^2}} = 10^x$。}
    \end{tblr}
\end{xiaoxiaotis}

\end{xiaotis}
\end{enhancedline}

