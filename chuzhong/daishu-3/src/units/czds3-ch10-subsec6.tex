\subsection{二次根式的除法}\label{subsec:10-6}
\begin{enhancedline}

把公式 $\sqrt{\dfrac{a}{b}} = \dfrac{\sqrt{a}}{\sqrt{b}}$ 反过来,就得
$$ \dfrac{\sqrt{a}}{\sqrt{b}} = \sqrt{\dfrac{a}{b}} \juhao $$

运用这个公式,可以进行一些简单的二次根式的除法运算。
就是说,二次根式相除,仍得二次根式,把被开方数相除所得的商作为商的被开方数。

\liti 计算:
\begin{xiaoxiaotis}

    \hspace*{1.5em} \begin{tblr}[t]{columns={18em, colsep=0pt}}
        \xxt{$\sqrt{72} \div \sqrt{6}$;} & \xxt{$\sqrt{1\dfrac{1}{2}} \div \sqrt{\dfrac{1}{6}}$。}
    \end{tblr}

\resetxxt
\jie \begin{tblr}[t]{columns={18em, colsep=0pt}}
    \xxt{$\begin{aligned}[t]
        &\sqrt{72} \div \sqrt{6} \\
        &= \dfrac{\sqrt{72}}{\sqrt{6}} = \sqrt{\dfrac{72}{6}} \\
        &= \sqrt{12} = 2\sqrt{3} \fenhao
    \end{aligned}$} & \xxt{$\begin{aligned}[t]
        & \sqrt{1\dfrac{1}{2}} \div \sqrt{\dfrac{1}{6}} \\
        &= \dfrac{\sqrt{\dfrac{3}{2}}}{\sqrt{\dfrac{1}{6}}} = \sqrt{\dfrac{3}{2} \times 6} \\
        &= \sqrt{9} = 3 \juhao
    \end{aligned}$}
\end{tblr}

\end{xiaoxiaotis}

二次根式的除法运算,还可以利用化去分母中的根号的方法来进行。
例如,计算 $\sqrt{3} \div \sqrt{2}$。
先将 $\sqrt{3} \div \sqrt{2}$ 写成 $\dfrac{\sqrt{3}}{\sqrt{2}}$,
然后把分子和分母都乘以 $\sqrt{2}$,化去分母中的根号,就得
\begin{align*}
    \dfrac{\sqrt{3}}{\sqrt{2}} &= \dfrac{\sqrt{3} \times \sqrt{2}}{\sqrt{2} \times \sqrt{2}} \\
                               &= \dfrac{\sqrt{6}}{(\sqrt{2})^2} \\
                               &= \dfrac{1}{2}\sqrt{6} \juhao
\end{align*}


同样,在计算 $\dfrac{1}{\sqrt{3} - \sqrt{2}}$ 的时候,也是先把分母中的根号化去。
\begin{align*}
    \dfrac{1}{\sqrt{3} - \sqrt{2}} &= \dfrac{\sqrt{3} + \sqrt{2}}{(\sqrt{3} - \sqrt{2}) (\sqrt{3} + \sqrt{2})} \\
                                   &= \dfrac{\sqrt{3} + \sqrt{2}}{3 - 2} \\
                                   &= \sqrt{3} + \sqrt{2} \juhao
\end{align*}


把分母中的根号化去,叫做\zhongdian{分母有理化}。
把分母有理化时,一般是把分子和分母都乘以同一个适当的代数式,使分母不含根号。

两个含有二次根式的代数式相乘,如果它们的积不含有二次根式,
我们说这两个代数式互为\zhongdian{有理化因式}。在上面例子中,
$\sqrt{2}$ 与 $\sqrt{2}$, $\sqrt{3} + \sqrt{2}$ 与 $\sqrt{3} - \sqrt{2}$
互为有理化因式。


\liti 把下列各式的分母有理化:
\begin{xiaoxiaotis}

    \hspace*{1.5em} \begin{tblr}[t]{columns={18em, colsep=0pt}, rows={rowsep=0.5em}}
        \xxt{$\dfrac{1}{\sqrt{5}}$;} & \xxt{$\dfrac{4}{3\sqrt{7}}$;} \\
        \xxt{$\dfrac{a}{\sqrt{a + b}}$;} & \xxt{$\dfrac{\sqrt{5}a}{\sqrt{20a}}$。}
    \end{tblr}

\resetxxt
\jie \begin{tblr}[t]{columns={colsep=0pt}, rows={rowsep=0.5em}}
    \xxt{$\dfrac{1}{\sqrt{5}} = \dfrac{\sqrt{5}}{\sqrt{5} \cdot \sqrt{5}} = \dfrac{\sqrt{5}}{5}$;} \\
    \xxt{$\dfrac{4}{3\sqrt{7}} = \dfrac{4 \cdot \sqrt{7}}{3\sqrt{7} \cdot \sqrt{7}} = \dfrac{4}{21}\sqrt{7}$;} \\
    \xxt{$\dfrac{a}{\sqrt{a + b}} = \dfrac{a \cdot \sqrt{a + b}}{\sqrt{a + b} \cdot \sqrt{a + b}} = \dfrac{a}{a + b}\sqrt{a + b}$;} \\
    \xxt{$\dfrac{\sqrt{5}a}{\sqrt{20a}} = \dfrac{\sqrt{5} (\sqrt{a})^2}{2\sqrt{5} \cdot \sqrt{a}} = \dfrac{1}{2}\sqrt{a}$。}
\end{tblr}

\end{xiaoxiaotis}

从上例第 (1)~(3) 小题可以看出,$\sqrt{a}$ 的有理化因式是 $\sqrt{a}$。
在分母有理化时,有时也可直接利用约分,例如第 (4) 小题。


\liti 把下列各式的分母有理化:
\begin{xiaoxiaotis}

    \hspace*{1.5em} \begin{tblr}[t]{columns={18em, colsep=0pt}, rows={rowsep=0.5em}}
        \xxt{$\dfrac{1}{\sqrt{2} + 1}$;} & \xxt{$\dfrac{\sqrt{2}}{3 - \sqrt{3}}$;} \\
        \xxt{$\dfrac{\sqrt{x} - \sqrt{y}}{\sqrt{x} + \sqrt{y}} \quad (x \neq y)$;} & \xxt{$\dfrac{x - y}{\sqrt{x} + \sqrt{y}}$。}
    \end{tblr}

\resetxxt
\jie \begin{tblr}[t]{columns={18em, colsep=0pt}, rows={rowsep=0.5em}}
    \xxt{$\begin{aligned}[t]
            \dfrac{1}{\sqrt{2} + 1} &= \dfrac{\sqrt{2} - 1}{(\sqrt{2} + 1) (\sqrt{2} - 1)} \\
                                &= \dfrac{\sqrt{2} - 1}{2 - 1} \\
                                &= \sqrt{2} - 1 \fenhao
        \end{aligned}$} & \xxt{$\begin{aligned}[t]
            \dfrac{\sqrt{2}}{3 - \sqrt{3}} &= \dfrac{\sqrt{2} (3 + \sqrt{3})}{(3 - \sqrt{3}) (3 + \sqrt{3})} \\
                                           &= \dfrac{3\sqrt{2} + \sqrt{6}}{9 - 3} \\
                                           &= \dfrac{3\sqrt{2} + \sqrt{6}}{6} \fenhao
        \end{aligned}$} \\
    \xxt{$\begin{aligned}[t]
        &\dfrac{\sqrt{x} - \sqrt{y}}{\sqrt{x} + \sqrt{y}} \\
        &= \dfrac{(\sqrt{x} - \sqrt{y})^2}{(\sqrt{x} + \sqrt{y}) (\sqrt{x} - \sqrt{y})} \\
        &= \dfrac{x + y - 2\sqrt{xy}}{x - y} \fenhao
    \end{aligned}$} & \xxt{$\begin{aligned}[t]
        &\dfrac{x - y}{\sqrt{x} + \sqrt{y}} \\
        &= \dfrac{(\sqrt{x})^2 - (\sqrt{y})^2}{\sqrt{x} + \sqrt{y}} \\
        &= \dfrac{(\sqrt{x} + \sqrt{y}) (\sqrt{x} - \sqrt{y}))}{\sqrt{x} + \sqrt{y}} \\
        &= \sqrt{x} - \sqrt{y} \juhao
    \end{aligned}$}
\end{tblr}

\end{xiaoxiaotis}

从上例第 (1)~(3) 小题可以看出,
$a\sqrt{x} + b\sqrt{y}$ 与 $a\sqrt{x} - b\sqrt{y}$ 互为有理化因式。
在分母有理化时,有时也可先分解因式,再约分,例如第 (4) 小题。

\liti 计算:
\begin{xiaoxiaotis}

    \hspace*{1.5em} \begin{tblr}[t]{columns={18em, colsep=0pt}}
        \xxt{$(6\sqrt{7} - 4\sqrt{2}) \div \sqrt{3}$;} & \xxt{$(\sqrt{12} - 5\sqrt{8}) \div (\sqrt{6} + \sqrt{2})$。}
    \end{tblr}

\resetxxt
\jie \begin{tblr}[t]{columns={18em, colsep=0pt}, rows={rowsep=0.5em}}
    \xxt{$\begin{aligned}[t]
        & (6\sqrt{7} - 4\sqrt{2}) \div \sqrt{3} \\
        &= \dfrac{(6\sqrt{7} - 4\sqrt{2}) \cdot \sqrt{3}}{\sqrt{3} \cdot \sqrt{3}} \\
        &= 2\sqrt{21} - \dfrac{4}{3}\sqrt{6} \fenhao
    \end{aligned}$} & \xxt{$\begin{aligned}[t]
        & (\sqrt{12} - 5\sqrt{8}) \div (\sqrt{6} + \sqrt{2}) \\
        &= \dfrac{\sqrt{12} - 5\sqrt{8}}{\sqrt{6} + \sqrt{2}} \\
        &= \dfrac{(2\sqrt{3} - 10\sqrt{2}) (\sqrt{6} - \sqrt{2})}{(\sqrt{6} + \sqrt{2}) (\sqrt{6} - \sqrt{2})} \\
        &= \dfrac{6\sqrt{2} - 2\sqrt{6} - 20\sqrt{3} + 20}{4} \\
        &= \dfrac{3}{2}\sqrt{2} - \dfrac{1}{2}\sqrt{6} - 5\sqrt{3} + 5 \juhao
    \end{aligned}$}
\end{tblr}

\end{xiaoxiaotis}


一般地,对于二次根式的除法,可以先写成分式的形式,然后通过分母有理化进行运算。

\lianxi
\begin{xiaotis}

\xiaoti{计算:}
\begin{xiaoxiaotis}

    \begin{tblr}{columns={colsep=0pt}, rows={rowsep=0.5em}}
        \xxt{$-\sqrt{54} \div \sqrt{3}$;} & \xxt{$\sqrt{1\dfrac{3}{5}} \div \sqrt{3\dfrac{1}{5}}$;} & \xxt{$6\sqrt{3} \div 3\sqrt{6}$;} \\
        \xxt{$\dfrac{1}{2}\sqrt{6} \cdot 4\sqrt{\dfrac{1}{12}} \div \dfrac{2}{3}\sqrt{1\dfrac{1}{2}}$;} & \xxt{$4\sqrt{6a^3} \div 2\sqrt{\dfrac{a}{3}}$;} & \xxt{$a^2x \div \sqrt{ax^3}$。}
    \end{tblr}
\end{xiaoxiaotis}


\xiaoti{把下列各式的分母有理化:}
\begin{xiaoxiaotis}

    \begin{tblr}{columns={12em, colsep=0pt}, rows={rowsep=0.5em}}
        \xxt{$\dfrac{1}{\sqrt{3}}$;}   & \xxt{$\dfrac{\sqrt{3}}{\sqrt{40}}$;} & \xxt{$\dfrac{x^2}{\sqrt{4xy}}$;} \\
        \xxt{$\dfrac{2n}{3\sqrt{n}}$;} & \xxt{$\dfrac{a^2 - b^2}{\sqrt{a + b}}$。}
    \end{tblr}
\end{xiaoxiaotis}


\xiaoti{把下列各式的分母有理化:}
\begin{xiaoxiaotis}

    \begin{tblr}{columns={12em, colsep=0pt}, rows={rowsep=0.5em}}
        \xxt{$\dfrac{1}{1 - \sqrt{2}}$;} & \xxt{$\dfrac{\sqrt{3}}{5 + \sqrt{7}}$;} & \xxt{$\dfrac{\sqrt{5} - \sqrt{7}}{\sqrt{5} + \sqrt{7}}$;} \\
        \xxt{$\dfrac{a - b}{\sqrt{a} - \sqrt{b}} \; (a \neq b)$;} & \xxt{$\dfrac{1}{x + \sqrt{1 + x^2}}$;} & \xxt{$\dfrac{3\sqrt{3} + 4\sqrt{2}}{3\sqrt{3} - 4\sqrt{2}}$。}
    \end{tblr}
\end{xiaoxiaotis}

\xiaoti{计算:}
\begin{xiaoxiaotis}

    \begin{tblr}{columns={18em, colsep=0pt}} %, rows={rowsep=0.5em}}
        \xxt{$\left(5\sqrt{15} + \sqrt{\dfrac{3}{5}}\right) \div \sqrt{5}$;} & \xxt{$\sqrt{3} \div (4 - 3\sqrt{5})$;} \\
        \xxt{$(14 + 6\sqrt{5}) \div (3 + \sqrt{5})$;} & \xxt{$\left(\sqrt{mn} + \sqrt{\dfrac{m}{n}}\right) \div \sqrt{\dfrac{m}{n}}$;} \\
        \SetCell[c=2]{l}{\xxt{$(\sqrt{a + b} - \sqrt{a - b}) \div (\sqrt{a + b} + \sqrt{a - b}) \; (a > b)$;}} \\
        \xxt{$(x + 2\sqrt{xy} + y) \div (\sqrt{x} + \sqrt{y})$。}
    \end{tblr}
\end{xiaoxiaotis}

\end{xiaotis}
\lianxijiange


\liti 求下列各式的值(精确到 $0.01$):
\begin{xiaoxiaotis}

    \hspace*{1.5em} \begin{tblr}[t]{columns={18em, colsep=0pt}}
        \xxt{$\dfrac{4}{\sqrt{5} - \sqrt{3}}$;} & \xxt{$\dfrac{1}{\sqrt{2}} + \dfrac{1}{\sqrt{3}}$。}
    \end{tblr}

\resetxxt
\jie \begin{tblr}[t]{columns={18em, colsep=0pt}, rows={rowsep=0.5em}}
    \xxt{$\begin{aligned}[t]
        & \dfrac{4}{\sqrt{5} - \sqrt{3}} \\
        &= \dfrac{4(\sqrt{5} + \sqrt{3})}{(\sqrt{5} - \sqrt{3}) (\sqrt{5} + \sqrt{3})} \\
        &= \dfrac{4(\sqrt{5} + \sqrt{3})}{2} \\
        &= 2(\sqrt{5} + \sqrt{3}) \\
        &\approx 2(2.236 + 1.732) \\
        &\approx 7.94 \fenhao
    \end{aligned}$} & \xxt{$\begin{aligned}[t]
        & \dfrac{1}{\sqrt{2}} + \dfrac{1}{\sqrt{3}} \\
        &= \dfrac{\sqrt{2}}{2} + \dfrac{\sqrt{3}}{3} \\
        &\approx 0.707 + 0.577 \\
        &\approx 1.28 \juhao
    \end{aligned}$}
\end{tblr}

\end{xiaoxiaotis}


如上例,在求含有二次根式的式子的值时,往往先把式子化简,然后再求值。

\liti 计算 $\left(\dfrac{3 - \sqrt{7}}{2}\right)^2 + \dfrac{\sqrt{7} + \sqrt{3}}{\sqrt{7} - \sqrt{3}}$。

\jie $\begin{aligned}[t]
    & \left(\dfrac{3 - \sqrt{7}}{2}\right)^2 + \dfrac{\sqrt{7} + \sqrt{3}}{\sqrt{7} - \sqrt{3}} \\
    &= \dfrac{16 - 6\sqrt{7}}{4} + \dfrac{(\sqrt{7} + \sqrt{3})^2}{(\sqrt{7})^2 - (\sqrt{3})^2} \\
    &= \dfrac{16 - 6\sqrt{7}}{4} + \dfrac{10 + 2\sqrt{21}}{4} \\
    &= 6\dfrac{1}{2} - \dfrac{3}{2}\sqrt{7} + \dfrac{1}{2}\sqrt{21} \juhao
\end{aligned}$


\lianxi
\begin{xiaotis}

\xiaoti{求下列各式的值(精确到 $0.01$):}
\begin{xiaoxiaotis}

    \begin{tblr}{columns={18em, colsep=0pt}, rows={rowsep=0.5em}}
        \xxt{$\dfrac{3}{\sqrt{7} - 2}$;} & \xxt{$\dfrac{3 - \sqrt{2}}{3 + \sqrt{2}}$;} \\
        \xxt{$\dfrac{10}{\sqrt{5}} - \dfrac{4}{\sqrt{5} - 1}$;} & \xxt{$\dfrac{\sqrt{5} + 6}{\sqrt{5} - 6} + \dfrac{\sqrt{7} + 2}{\sqrt{7} - 2}$。}
    \end{tblr}
\end{xiaoxiaotis}


\xiaoti{计算:}
\begin{xiaoxiaotis}

    \xxt{$\sqrt{3} \div (5 - \sqrt{7}) - (2\sqrt{3} - 5\sqrt{7})^2$;}

    \xxt{$(2\sqrt{3} - 2) (3\sqrt{6} + \sqrt{2}) + (3\sqrt{3} + 5\sqrt{2}) \div (3\sqrt{3} - 5\sqrt{2})$。}

\end{xiaoxiaotis}
\end{xiaotis}

\end{enhancedline}

