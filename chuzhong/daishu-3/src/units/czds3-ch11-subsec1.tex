\subsection{一元二次方程}\label{subsec:11-1}

我们已经学过一元一次方程的解法及其应用,现在来看下面的问题:

要剪一块面积是 $150$ 平方厘米的长方形铁片,使它的长比宽多 $5$ 厘米,应该怎样剪法?

要解决这个问题,就要求出铁片的长与宽,可以设宽是 $x$ 厘米,那么长是 $(x + 5)$ 厘米。
根据题意,得
$$ x(x + 5) = 150 \douhao $$
去括号,得
$$ x^2 + 5x = 150 \juhao $$

上面这个方程的两边都是关于未知数的整式,这样的方程是一个\zhongdian{整式方程}。
在这个整式方程中,只含有一个未知数,并且未知数的最高次数是 $2$,
这样的整式方程叫做\zhongdian{一元二次方程}。

上面的方程,经过移项可以化成下面的形式:
$$ x^2 + 5x - 150 = 0 \juhao $$

任何一个关于 $x$ 的一元二次方程,经过整理,都可以化成
$$ ax^2 + bx + c = 0 \quad (a \neq 0) $$
的形式。这种形式叫做一元二次方程的一般形式。其中
$ax^2$ 叫做二次项,$a$ 叫做二次项系数;
$bx$   叫做一次项,$b$ 叫做一次项系数;
$c$    叫做常数项。
一次项系数 $b$ 和常数项 $c$ 可以是任何实数,二次项系数 $a$ 是不等于零的实数。
因为如果 $a$ 等于零,那么这样的方程就不是二次方程了。

\liti[0] 把方程 $4x(x + 3) = 5(x - 1) + 8$ 化成一般形式,
并写出它的二次项系数、一次项系数及常数项。

\jie 去括号,得
$$ 4x^2 + 12x = 5x - 5 + 8 \juhao $$
移项,合并同类项,得
$$ 4x^2 + 7x - 3 = 0 \juhao $$

方程的二次项系数是 $4$,一次项系数是 $7$, 常数项是 $-3$。


\lianxi
\begin{xiaotis}

\xiaoti{(口答)说出一元二次方程
    $$ 2x^2 + x + 4 = 0 $$
    的二次项系数、一次项系数及常数项。
}

\xiaoti{写出下列一元二次方程的的二次项系数、一次项系数及常数项:}
\begin{xiaoxiaotis}

    \begin{tblr}{columns={12em, colsep=0pt}}
        \xxt{$4x^2 + 3x - 2 = 0$;} & \xxt{$3x^2 - 5 = 0$;} &  \xxt{$6x^2 - x = 0$。}
    \end{tblr}
\end{xiaoxiaotis}


\xiaoti{把下列方程先化成一元二次方程的一般形式,再写出它的二次项系数、一次项系数及常数项:}
\begin{xiaoxiaotis}

    \begin{tblr}{columns={colsep=0pt}, column{1}={18em}}
        \xxt{$3x^2 = 5x + 2$;} & \xxt{$(x + 3)(x - 4) = -6$;} \\
        \xxt{$3x(x - 1) = 2(x + 2) - 4$;} & \xxt{$(2x - 1)(3x + 2) = x^2 + 2$;} \\
        \xxt{$(t + 1)^2 - 2(t - 1)^2 = 6t - 5$;} & \xxt{$(y + \sqrt{6})(y - \sqrt{6}) + (2y + 1)^2$ = 4y - 5。}
    \end{tblr}
\end{xiaoxiaotis}

\end{xiaotis}

