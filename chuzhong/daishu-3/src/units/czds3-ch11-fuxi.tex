\fuxiti
\begin{enhancedline}
\begin{xiaotis}

\xiaoti{解下列方程:}
\begin{xiaoxiaotis}

    \xxt{$(2x + 1)^2 + (x - 2)^2 - (2x + 1)(x - 2) = 43$;}

    \xxt{$x^2 - (1 + 2\sqrt{3})x - 3 + \sqrt{3} = 0$;}

    \xxt{$(x^2 - 10)^2 + 3x^2 = 28$;}

    \xxt{$(2x - 3)^4 - 6(2x - 3)^2 + 9 = 0$;}

    \xxt{$(x^2 - x)^2 - 4(2x^2 - 2x - 3) = 0$;}

    \xxt{$(x^2 + 3x + 4) (x^2 + 3x + 5) = 6$;}

    \xxt{$(x + 1) (x + 2) (x + 3) (x + 4) + 1 = 0$;}

    \xxt{$\dfrac{3}{x^2 - 3x + 2} - \dfrac{1}{x^2 - x} = \dfrac{1}{x - 2} + \dfrac{4}{x^2 - 2x}$;}

    \xxt{$\dfrac{x + 1}{(x + 3)(x - 1)} - \dfrac{2x - 2}{(x + 3)(2 - x)} = \dfrac{6x}{(1 - x)(x - 2)}$;}

    \xxt{$\dfrac{3}{x - 2} - \dfrac{4}{x - 1} = \dfrac{1}{x - 4} - \dfrac{2}{x - 3}$;}

    \xxt{$x^2 + 3x - \dfrac{20}{x^2 + 3x} = 8$;}

    \xxt{$\left(\dfrac{x^2 - 1}{x}\right)^2 - \dfrac{7}{2}\left(\dfrac{x^2 - 1}{x}\right) + 3 = 0$;}

    \xxt{$2\left(x^2 + \dfrac{1}{x^2}\right) - 3\left(x + \dfrac{1}{x}\right) = 1$;}

    \xxt{$\sqrt{x + 5} + \dfrac{3}{\sqrt{x + 5}} = \sqrt{3x + 4}$;}

    \xxt{$\sqrt{2x - 5} + \sqrt{x - 3} = \sqrt{3x + 4}$;}

    \xxt{$\sqrt{3x + 1} + \sqrt{4x - 3} - \sqrt{5x + 4} = 0$;}

    \xxt{$2x^2 - 14x - 3\sqrt{x^2 - 7x + 10} + 18 = 0$;}

    \xxt{$\dfrac{\sqrt{x + 1} - \sqrt{x - 1}}{\sqrt{x + 1} + \sqrt{x - 1}} = 2 - x$。}

\end{xiaoxiaotis}


\xiaoti{下列各小题中,已知字母都表示正数。}
\begin{xiaoxiaotis}

    \xxt{在公式 $S = \dfrac{\pi D^2}{2} + \pi Dh$ 中,已知 $S$,$\pi$,$h$,求 $D$;}

    \xxt{在公式 $Q = mg + \dfrac{mv^2}{R} \quad (Q > mg)$ 中,已知 $Q$,$m$,$g$,$R$,求 $v$;}

    \xxt{在公式 $mf = \left(\dfrac{m}{K} - 1\right) \dfrac{1}{K}$ 中,已知 $m$,$f$,求 $K$;}

    \xxt{在公式 $\dfrac{1}{R} = \dfrac{1}{r} - \dfrac{1}{r - r_1} \quad (r_1 > 4R)$ 中,已知 $R$,$r_1$,求 $r$;}

    \xxt{在公式 $c = \sqrt{a^2 + b^2}$ 中,已知 $a$,$c$,求 $b$。}

\end{xiaoxiaotis}


\xiaoti{如果一元二次方程 $ax^2 + bx + c = 0$ 的二根之比是 $2:3$,求证 $6b^2 = 25ac$。}

\xiaoti{利用根与系数的关系,求一个一元二次方程,使它的根分别是方程 $x^2 + px + q = 0$ 的各根的}
\begin{xiaoxiaotis}

    \xxt{相反数;} \xxt{倒数;} \xxt{$k$ 倍;} \xxt{平方。}

\end{xiaoxiaotis}


\xiaoti{把下列各式分解因式:}
\begin{xiaoxiaotis}

    \begin{tblr}{columns={18em, colsep=0pt}}
        \xxt{$a^2 + 2a - 120$;}    & \xxt{$-6p^2 + 11p + 10$;} \\
        \xxt{$3x^2 - xy - 10y^2$;} & \xxt{$15a^2 - 8ab - 12b^2$。}
    \end{tblr}
\end{xiaoxiaotis}


\xiaoti{在实数范围内把下列各式分解因式:}
\begin{xiaoxiaotis}

    \begin{tblr}{columns={18em, colsep=0pt}} %, rows={rowsep=0.5em}}
        \xxt{$\sqrt{3}a^2 - \sqrt{6}a - \sqrt{2}a + 2$;} & \xxt{$9m^4 - \dfrac{1}{4}n^4$;} \\
        \xxt{$x^4 - x^2 - 6$;} & \xxt{$6x^4 - 7x^2 - 3$;} \\
        \xxt{$9x^2 - 12xy + y^2$;} & \xxt{$5x^2y^2 + xy - 7$。}
    \end{tblr}
\end{xiaoxiaotis}


\xiaoti{用长为 $100$ 厘米的金属丝制成一个矩形框子,框子各边的长取多少厘米时,框子的面积是}
\begin{xiaoxiaotis}

    \xxt{$500 \; \pflm$?}
    \xxt{$625 \; \pflm$?}
    \xxt{$800 \; \pflm$?}
\end{xiaoxiaotis}


\xiaoti{已知一个多边形,它的对角线共有 $35$ 条。这个多边形是几边形?}


\xiaoti{}%
\begin{xiaoxiaotis}%
    \xxt[\xxtsep]{有四个连续整数。已知它们的和等于其中最大的与最小的两个整数的积,求这四个数。}

    \xxt{有三个连续奇数。已知它们的平方和等于 $251$,求这三个数。}

\end{xiaoxiaotis}

\begin{minipage}{11cm}
\xiaoti{如图, 在 $\triangle ABC$ 中,$\angle B = 90^\circ$,
    点 $P$ 从点 $A$ 开始沿 $AB$ 边向点 $B$ 以 $1 \; \lmmm$ 的速度移动,
    点 $Q$ 从点 $B$ 开始沿 $BC$ 边向点 $C$ 以 $2 \; \lmmm$ 的速度移动,
    如果 $P$,$Q$ 分别从 $A$,$B$ 同时出发,几秒以后 $\triangle PBQ$ 的面积等于 $8 \; \pflm$?
}

\xiaoti{一个容器盛满纯药液 $63$ 升,第一次倒出一部分纯药液后,用水加满,
    第二次又倒出同样多的药液,再用水加满,这时,容器内剩下的纯药液是 $28$ 升。
    每次倒出液体多少升?
}

\begin{enhancedline}
\xiaoti{一个容器盛满烧碱溶液,第一次倒出 $10$ 升后,用水加满,
    第二次又倒出 $10$ 升,再用水加满,这时容器内的溶液浓度是原来浓度的 $\dfrac{1}{4}$。
    求容器的容积。
}
\end{enhancedline}

\end{minipage}
\quad
\begin{minipage}{4cm}
    \begin{figure}[H]
        \centering
        \begin{tikzpicture}[>=Stealth,
    every node/.style={fill=white, inner sep=1pt},
]
    \pgfmathsetmacro{\a}{6*0.3}
    \pgfmathsetmacro{\b}{12*0.3}
    \coordinate (A) at (0, 0);
    \coordinate (B) at (\a, 0);
    \coordinate (C) at (\a, \b);
    \coordinate (P) at (2*0.3, 0);
    \coordinate (Q) at (\a, 4*0.3);

    \draw [ultra thick] (A) -- (B) -- (C) -- cycle;
    \draw (A) node [anchor=north east] {$A$};
    \draw (B) node [anchor=north west] {$B$};
    \draw (C) node [above] {$C$};

    \draw (P) + (0.5, 0.2) node {$P$};
    \draw (Q) node [left=0.25, above] {$Q$};
    \draw [ultra thick] (P) -- (Q);

    \draw [<->] (0, -0.2) to [xianduan] node {$6$\; cm} (\a, -0.2);
    \draw [<->] (\a + 0.2, 0) to [xianduan] node [rotate=90] {$12$\; cm} (\a + 0.2, \b);

    \draw [->] (0.1, 0.2) -- (2*0.3, 0.2);
    \draw [->] (\a - 0.2, 0) -- (\a - 0.2, 2*0.3);
\end{tikzpicture}


        \caption*{(第 10 题)}
    \end{figure}
\end{minipage}


\xiaoti{解下列方程组:}
\begin{xiaoxiaotis}

    \begin{tblr}{columns={colsep=0pt}, rows={rowsep=0.5em}}
        \xxt{$\begin{cases}
                x^2 - y^2 - 3x + 2y = 10 \douhao \\
                x + y = 7 \fenhao
            \end{cases}$} & \xxt{$\begin{cases}
                x + y = 17 \douhao \\
                x^2 + y^2 = 169 \fenhao
            \end{cases}$} \\
        \xxt{$\begin{cases}
                (x - 2)^2 + (y + 3)^2 = 9 \douhao \\
                3x - 2y = 6 \fenhao
            \end{cases}$} & \xxt{$\begin{cases}
                4x^2 - 9y^2 = 15 \douhao \\
                2x - 3y = 5 \fenhao
            \end{cases}$}
    \end{tblr}

    \begin{tblr}{columns={colsep=0pt}, rows={rowsep=0.5em}}
        \xxt{$\begin{cases}
            x^2 + y^2 = 5 \douhao \\
            y^2 = 4x \fenhao
        \end{cases}$} & \xxt{$\begin{cases}
            x^2 + y^2 = 101 \douhao \\
            xy = -10 \fenhao
        \end{cases}$} \\
    \xxt{$\begin{cases}
            (x - 2)^2 + (y - 1)^2 = 25 \douhao \\
            2(x - 2)^2 - 3(y - 1)^2 = 5 \fenhao
        \end{cases}$} & \xxt{$\begin{cases}
            (x + 3)^2 + y^2 = 9 \douhao \\
            9(x - 2)^2 + 4y^2 = 36 \fenhao
        \end{cases}$} \\
    \xxt{$\begin{cases}
            x^2 + y^2 - 3x - 3y = 8 \douhao \\
            xy = 10 \fenhao
        \end{cases}$} & \xxt{$\begin{cases}
            2x^2 - xy - y^2 + 3x + 2y = 3 \douhao \\
            x^2 - 3x + 2 = 0 \fenhao
        \end{cases}$} \\
    \xxt{$\begin{cases}
            \dfrac{y}{x} + \dfrac{2x}{y} = 3 \douhao \\
            2x + 3y = 4 \fenhao
        \end{cases}$} & \xxt{$\begin{cases}
            \dfrac{2}{x} + \dfrac{5}{y} = 1 \douhao \\[1em]
            \dfrac{4}{x^2} + \dfrac{25}{y^2} = 25 \fenhao
        \end{cases}$} \\
        \xxt{$\begin{cases}
                \sqrt{x + 1} + \sqrt{y - 2} = 5 \douhao \\
                x - y = 12 \fenhao
            \end{cases}$} & \xxt{$\begin{cases}
                \sqrt{x + 1} + \sqrt{y - 1} = 5 \douhao \\
                x + y = 13 \fenhao
            \end{cases}$} \\
        \xxt{$\begin{cases}
                \sqrt{\dfrac{x}{y}} + \sqrt{\dfrac{y}{x}} = \dfrac{5}{2} \douhao \\
                x + y = 10 \fenhao
            \end{cases}$} & \xxt{$\begin{cases}
                xy = 3 \douhao \\
                yz = 6 \douhao \\
                xz = 2 \juhao
            \end{cases}$}
    \end{tblr}
\end{xiaoxiaotis}


\xiaoti{}%
\begin{xiaoxiaotis}%
    \xxt[\xxtsep]{已知
        $$ \begin{cases}
            x = 1 \douhao \\
            y = 3 \douhao
        \end{cases} \qquad \begin{cases}
            x = 7 \douhao \\
            y = -3
        \end{cases}$$
        是方程 $(y + k)^2 = 2(x + h)$ 的两个解,求 $h$, $k$ 的值;
    }

    \xxt{已知
        $$ \begin{cases}
            x = 0 \douhao \\
            y = 0 \douhao
        \end{cases} \qquad \begin{cases}
            x = 2 \douhao \\
            y = -4
        \end{cases}$$
        是方程 $(x + h)^2 + (y + k)^2 = 10$ 的两个解,求 $h$, $k$ 的值。
    }

\end{xiaoxiaotis}


\xiaoti{}%
\begin{xiaoxiaotis}%
    \xxt[\xxtsep]{已知
        $$ \begin{cases}
            x = 1 \douhao \\
            y = \dfrac{3}{2}\sqrt{3} \douhao
        \end{cases} \qquad \begin{cases}
            x = \dfrac{2}{3}\sqrt{5} \douhao \\
            y = -2
        \end{cases}$$
        是方程 $\dfrac{x^2}{a^2} + \dfrac{y^2}{b^2} = 1$ 的两个解,求正数 $a$, $b$ 的值;
    }

    \xxt{已知
        $$ \begin{cases}
            x = -5\sqrt{2} \douhao \\
            y = 2 \douhao
        \end{cases} \qquad \begin{cases}
            x = 6 \douhao \\
            y = -\dfrac{2}{5}\sqrt{11}
        \end{cases}$$
        是方程 $\dfrac{x^2}{a^2} - \dfrac{y^2}{b^2} = 1$ 的两个解,求正数 $a$, $b$ 的值。
    }

\end{xiaoxiaotis}


\xiaoti{}%
\begin{xiaoxiaotis}%
    \xxt[\xxtsep]{从方程组
        $$\begin{cases}
            x^2 + y^2 = 8 \douhao \\
            x + y = b
        \end{cases}$$
        中消去 $y$,得出关于 $x$ 的二次方程;
    }

    \xxt{当 $b = 3$ 时,这个关于 $x$ 的二次方程有几个实数解?\\
        当 $b = 4$ 时呢?当 $b = 5$ 时呢?
    }

\end{xiaoxiaotis}


\xiaoti{}%
\begin{xiaoxiaotis}%
    \xxt[\xxtsep]{$m$ 取什么值时,方程组
        $$\begin{cases}
            y^2 = 4x \douhao \\
            y = 2x + m
        \end{cases}$$
        有两个相等的实数解?并求出这时方程组的解;
    }

    \xxt{$m$ 取什么值时,方程组
        $$\begin{cases}
            x^2 + 2y^2 = 6 \douhao \\
            mx + y = 3
        \end{cases}$$
        有两个相等的实数解?并求出这时方程组的解。
    }

\end{xiaoxiaotis}


\xiaoti{在什么情况下,关于 $x$,$y$ 的方程组
    $$\begin{cases}
        x + y = a \douhao \\
        xy = b
    \end{cases}$$
}
\begin{xiaoxiaotis}

    \xxt{有两个不相等的实数解?}

    \xxt{有两个相等的实数解?}

    \xxt{没有实数解?}

\end{xiaoxiaotis}


\xiaoti{甲乙二人共同完成一项工程需要 $4$ 天。甲单独工作 $3$ 天后,剩下的部分由乙去做,
    乙还需工作的天数等于甲单独完成此项工程的天数。两人单独完成此项工程各需多少天?
}

\xiaoti{$A$,$B$ 两地间的路程为 $18$ 公里。甲从 $A$ 地, 乙从 $B$ 地同时出发相向而行。
    二人相遇后,甲再走 $2$ 小时 $30$ 分到达 $B$ 地,乙再走 $1$ 小时 $36$ 分到达 $A$ 地。
    求二人的速度。
}

\end{xiaotis}
\end{enhancedline}

