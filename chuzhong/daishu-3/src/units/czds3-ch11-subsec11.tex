\subsection{由一个二元二次方程和一个二元一次方程组成的方程组}\label{subsec:11-11}

\begin{enhancedline}
这种形式的方程组一般都可以用代入法来解。

\liti 解方程组
\begin{numcases}{}
    x^2 - 4y^2 + x + 3y - 1 = 0 \douhao \tag{1} \\
    2x - y - 1 = 0 \juhao \tag{2}
\end{numcases}

\jie 由 (2),得
\begin{equation}
    y = 2x - 1 \juhao \tag{3}
\end{equation}

把 (3) 代入 (1),得
$$ x^2 - 4(2x - 1)^2 + x + 3(2x - 1) - 1 = 0 \juhao $$

整理,得
$$ 15x^2 - 23x + 8 = 0 \juhao $$

解这个方程,得
$$ x_1 = 1 \nsep x_2 = \dfrac{8}{15} \juhao $$

把 $x_1 = 1$ 代入 (3),得
$$ y_1 = 1 \fenhao $$

把 $x_2 = \dfrac{8}{15}$ 代入 (3),得
$$ y_2 = \dfrac{1}{15} \juhao $$

所以原方程组的解是
$$ \begin{cases}
    x_1 = 1 \douhao \\
    y_1 = 1 \fenhao
\end{cases} \qquad \begin{cases}
    x_2 = \dfrac{8}{15} \douhao \\[1em]
    y_2 = \dfrac{1}{15} \juhao
\end{cases}$$



\liti 解方程组
$$\begin{cases}
    x + y = 7 \douhao \\
    xy = 12 \juhao
\end{cases}$$

分析:这个方程组可以用代入法解,也可以根据一元二次方程的根与系数的关系,
把 $x$,$y$ 看作一个一元二次方程的两个根,通过解这个一元二次方程来求 $x$,$y$。

\jie 这个方程组的 $x$,$y$ 是一元二次方程
$$ z^2 - 7z + 12 = 0 $$
的两个根。解这个方程,得
$$ z = 3 \text{,或} \quad z = 4 \juhao $$

所以原方程组的解是
$$\begin{cases}
    x_1 = 3 \douhao \\
    y_1 = 4 \fenhao \\
\end{cases} \qquad \begin{cases}
    x_2 = 4 \douhao \\
    y_2 = 3 \juhao
\end{cases}$$


\lianxi
\begin{xiaotis}

\xiaoti{下列各组中  $x$,$y$ 的值是不是方程组
    $$\begin{cases}
        x^2 + y^2 = 13 \douhao \\
        x + y = 5 \juhao
    \end{cases}$$
    的解?
}
\begin{xiaoxiaotis}

    \begin{tblr}{columns={18em, colsep=0pt}}
        \xxt{$\begin{cases}
                x = 2 \douhao \\
                y = 3 \fenhao
            \end{cases}$} & \xxt{$\begin{cases}
                x = 3 \douhao \\
                y = 2 \fenhao
            \end{cases}$} \\
        \xxt{$\begin{cases}
                x = 1 \douhao \\
                y = 4 \fenhao
            \end{cases}$} & \xxt{$\begin{cases}
                x = -2 \douhao \\
                y = -3 \juhao
            \end{cases}$}
    \end{tblr}
\end{xiaoxiaotis}


\xiaoti{解下列方程组:}
\begin{xiaoxiaotis}

    \begin{tblr}{columns={18em, colsep=0pt}}
        \xxt{$\begin{cases}
                y = x + 5 \douhao \\
                x^2 + y^2 = 625 \fenhao
            \end{cases}$} & \xxt{$\begin{cases}
                x^2 - 6x - 2y + 11 = 0 \douhao \\
                2x - y + 1 = 0 \fenhao
            \end{cases}$} \\
        \SetCell[c=2]{l}\xxt{$\begin{cases}
                x^2 + xy + y^2 + x + 5y = 0 \fenhao \\
                x + 2y = 0 \juhao
            \end{cases}$}
    \end{tblr}
\end{xiaoxiaotis}


\xiaoti{解下列方程组:}
\begin{xiaoxiaotis}

    \begin{tblr}{columns={18em, colsep=0pt}}
        \xxt{$\begin{cases}
                x + y = 3 \douhao \\
                xy = -10 \fenhao
            \end{cases}$} & \xxt{$\begin{cases}
                \dfrac{1}{x} + \dfrac{1}{y} = 5 \douhao \\[1em]
                \dfrac{1}{xy} = 6 \juhao
            \end{cases}$}
    \end{tblr}
\end{xiaoxiaotis}

\end{xiaotis}
\end{enhancedline}

