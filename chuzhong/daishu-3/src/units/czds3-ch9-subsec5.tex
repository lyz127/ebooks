\subsection{立方根表}\label{subsec:9-5}

《中学数学用表》中表五是立方根表,它的查法同平方根表的查法类似,但要注意下列两点:

1. 《中学数学用表》中的立方根表没有修正值,只能直接查出 $0.100$ 到 $99.9$ 之间的有三个数位的数的立方根,
如果被开方数有三个以上数位,可以先四舍五入成有三个数位的数,再查表;

2. 要查小于 $0.1$ 或大于 $100$ 的数的立方根,需要先移动被开方数的小数点,
移动的法则是把被开方数的小数点向右或者向左三位三位地移,使它成为表内可以查到的数,
被开方数的小数点每移动三位,查得的立方根的小数点向相反方向移动一位。

\liti 查表求下列各式的值:
\begin{xiaoxiaotis}

    \hspace*{1.5em} \fourInLineXxt[9em]{$\sqrt[3]{3.78}$;}{$\sqrt[3]{0.854}$;}{$\sqrt[3]{63.54}$;}{$\sqrt[3]{15.857}$。}

\resetxxt
\jie \begin{tblr}[t]{columns={colsep=0pt}}
    \xxt{$\sqrt[3]{3.78} = 1.558$;} \\
    \xxt{$\sqrt[3]{0.854} = 0.9488$;} \\
    \xxt{$\sqrt[3]{63.54} \approx \sqrt[3]{63.5} = 3.990$;} \\
    \xxt{$\sqrt[3]{15.857} \approx \sqrt[3]{15.9} = 2.515$。}
\end{tblr}

\end{xiaoxiaotis}

\liti 查表求下列各式的值:
\begin{xiaoxiaotis}

    \hspace*{1.5em} \twoInLineXxt[9em]{$\sqrt[3]{0.00525}$;}{$\sqrt[3]{525000}$。}

\resetxxt
\vspace*{.5em}
\jie \xxt{\hspace*{.5em} $\begin{aligned}[t]
    & \tikz [overlay, >=Stealth] {
        \draw [dashed] (-.5em, -.8em) rectangle (4.5em, 1.5em);
        \draw [->] (1.8em, -.8em) -- (1.8em, -5.5em) node[midway, align=center, fill=white, inner sep=3pt] {小数点向右 \\[-.6em] 移动三位};
      } \sqrt[3]{0.00525}
      \hspace*{.5em} \xlongequal{\text{\phantom{查表}}} \hspace*{.5em}
      \tikz [overlay, >=Stealth] {
        \draw [dashed] (-.5em, -.8em) rectangle (3.5em, 1.5em);
        \draw [<-] (1.8em, -.8em) -- (1.8em, -5.5em) node[midway, align=center, fill=white, inner sep=3pt] {小数点向左 \\[-.6em] 移动一位};
      }
      0.1738 \\[5em]
    & \tikz [overlay, >=Stealth] {
        \draw [dashed] (-.5em, -.8em) rectangle (4.5em, 1.5em);
      } \phantom{0}\sqrt[3]{5.25}\phantom{00}
      \hspace*{.5em} \xlongequal{\text{查表}} \hspace*{.5em}
      \tikz [overlay, >=Stealth] {
        \draw [dashed] (-.5em, -.8em) rectangle (3.5em, 1.5em);
      } 1.738
\end{aligned}$}

\vspace*{1em}\hspace*{3em}
$\therefore \hspace*{1em} \sqrt[3]{0.00525} = 0.1738$;

\vspace*{1em}
\hspace*{1.5em} \xxt{\hspace*{.5em}$\begin{aligned}[t]
    & \tikz [overlay, >=Stealth] {
        \draw [dashed] (-.5em, -.8em) rectangle (5em, 1.5em);
        \draw [->] (1.8em, -.8em) -- (1.8em, -5.5em) node[midway, align=center, fill=white, inner sep=3pt] {小数点向左 \\[-.6em] 移动六位};
      } \sqrt[3]{525000}\phantom{0.}
      \hspace*{.5em} \xlongequal{\text{\phantom{查表}}} \hspace*{.5em}
      \tikz [overlay, >=Stealth] {
        \draw [dashed] (-.5em, -.8em) rectangle (3.5em, 1.5em);
        \draw [<-] (1.8em, -.8em) -- (1.8em, -5.5em) node[midway, align=center, fill=white, inner sep=3pt] {小数点向右 \\[-.6em] 移动二位};
      }
      80.67 \\[5em]
    & \tikz [overlay, >=Stealth] {
        \draw [dashed] (-.5em, -.8em) rectangle (5em, 1.5em);
      } \sqrt[3]{0.525000}
      \hspace*{.5em} \xlongequal{\text{查表}} \hspace*{.5em}
      \tikz [overlay, >=Stealth] {
        \draw [dashed] (-.5em, -.8em) rectangle (3.5em, 1.5em);
      } 0.8067
\end{aligned}$}

\vspace*{1em}\hspace*{3em}
$\therefore \hspace*{1em} \sqrt[3]{525000} = 80.67$。

\end{xiaoxiaotis}


\lianxi
\begin{xiaotis}

\xiaoti{查表求下列各数的立方根:}
\begin{xiaoxiaotis}

    \begin{tblr}{columns={10em, colsep=0pt}}
        \xxt{$3$;}              & \xxt{$8.7$;}    & \xxt{$98.7$;}    & \xxt{$0.35$;} \\
        \xxt{$65\dfrac{4}{5}$;} & \xxt{$1.847$;}  & \xxt{$25.738$;}  & \xxt{$8\dfrac{1}{8}$;} \\
        \xxt{$-80$;}            & \xxt{$-16.5$。}
    \end{tblr}

\end{xiaoxiaotis}


\xiaoti{查表求下列各式的值:}
\begin{xiaoxiaotis}

    \begin{tblr}{columns={10em, colsep=0pt}}
        \xxt{$\sqrt[3]{5}$;}       & \xxt{$\sqrt[3]{68.8}$;}        & \xxt{$\sqrt[3]{0.8725}$;}  & \xxt{$\sqrt[3]{-43.6}$;} \\
        \xxt{$-\sqrt[3]{0.459}$;}  & \xxt{$\sqrt[3]{0.00094}$;}     & \xxt{$\sqrt[3]{0.037}$;}   & \xxt{$\sqrt[3]{6570}$;} \\
        \xxt{$\sqrt[3]{420000}$;}  & \xxt{$\sqrt[3]{0.00005087}$;}  & \xxt{$\sqrt[3]{-258}$;}    & \xxt{$-\sqrt[3]{0.0062548}$。}
    \end{tblr}

\end{xiaoxiaotis}

\xiaoti{求下列各式中的 $x$:}
\begin{xiaoxiaotis}

    \begin{tblr}{columns={18em, colsep=0pt}}
        \xxt{$x^3 = 27$;}       & \xxt{$x^3 = \dfrac{1}{8}$;} \\
        \xxt{$27x^3 + 64 = 0$;} & \xxt{$4x^3 - 5 = 0$(精确到 $0.01$)。}
    \end{tblr}

\end{xiaoxiaotis}

\end{xiaotis}
