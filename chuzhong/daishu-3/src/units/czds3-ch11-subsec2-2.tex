\subsubsection{配方法}

我们已经解过方程
$$ (x + 3)^2 = 2 \juhao $$
因为方程中 $x + 3$ 是 $2$ 的平方根,所以运用了直接开平方法来解。如果我们把方程
$$ (x + 3)^2 = 2 $$
的左边展开并整理,就得
$$ x^2 + 6x + 7 = 0 \juhao $$
因此,要解方程
$$ x^2 + 6x + 7 = 0 \douhao $$
我们可以先把它化成
$$ (x + 3)^2 = 2 $$
来解,化法如下:把方程
$$ x^2 + 6x + 7 = 0 $$
的常数项移到右边,得
$$ x^2 + 6x = -7 \juhao $$
为了使左边成为一个完全平方式,在方程的两边各边上一次项系数一半的平方。
\begin{align*}
    x^2 + 6x + 3^2 &= -7 + 3^2 \douhao \\
    (x + 3)^2      &= 2 \juhao
\end{align*}

解这个方程,得
$$ x + 3 = \pm \sqrt{2} \douhao $$
所以
$$ x = -3 \pm \sqrt{2} \juhao $$
即
$$ x_1 = -3 + \sqrt{2} \nsep x_2 = -3 - \sqrt{2} \juhao $$

这种解一元二次方程的方法叫做\zhongdian{配方法}。
这个方法就是先把常数项移到方程的右边,再把左边配成一个完全平方式,如果右边是非负常数,
就可以进一步通过直接开平方法来求出它的根。

\liti 解方程 $x^2 - 4x - 3 = 0$。

\jie 移项,得
$$ x^2 - 4x = 3 \juhao $$

配方,得
\begin{align*}
    x^2 - 4x + (-2)^2 &= 3 + (-2)^2 \douhao \\
    (x - 2)^2         &= 7 \juhao
\end{align*}

解这个方程,得
$$ x - 2 = \pm \sqrt{7} \juhao $$

\fengeSuoyi{x = 2 \pm \sqrt{7} \juhao} \\
即
$$ x_1 = 2 + \sqrt{7} \nsep x_2 = 2 - \sqrt{7} \juhao $$


\liti 解方程 $2x^2 + 5x - 1 = 0$。

分析:这个方程的二次项系数是 $2$,为了便于配方,可以先把二次项系数化为 $1$,为此方程的各项都除以 $2$。

\begin{enhancedline}
\jie 把方程的各项除以 $2$,得
$$ x^2 + \dfrac{5}{2}x - \dfrac{1}{2} = 0 \douhao $$

移项,得
$$ x^2 + \dfrac{5}{2}x = \dfrac{1}{2} \juhao $$
配方,得
\begin{align*}
    x^2 + \dfrac{5}{2}x + \left(\dfrac{5}{4}\right)^2 &= \dfrac{1}{2} + \left(\dfrac{5}{4}\right)^2 \douhao \\
    \left(x + \dfrac{5}{4}\right)^2 &= \dfrac{33}{16} \juhao
\end{align*}

解这个方程,得
$$ x + \dfrac{5}{4} = \pm \dfrac{\sqrt{33}}{4} \douhao $$

\fengeSuoyi{x = -\dfrac{5}{4} \pm \dfrac{\sqrt{33}}{4} = \dfrac{-5 \pm \sqrt{33}}{4} \douhao} \\
即
\begin{align*}
    x_1 &= \dfrac{-5 + \sqrt{33}}{4} \douhao \\
    x_2 &= \dfrac{-5 - \sqrt{33}}{4} \juhao
\end{align*}
\end{enhancedline}

\lianxi
\begin{xiaotis}

\xiaoti{用适当的数填空:}
\begin{xiaoxiaotis}

    \begin{tblr}{columns={18em, colsep=0pt}, rows={rowsep=0.5em}}
        \xxt{$x^2 + 6x + \hspace*{2em} = (x + \hspace*{2em})^2$;} & \xxt{$x^2 - 4x + \hspace*{2em} = (x - \hspace*{2em})^2$;} \\
        \xxt{$x^2 + 3x + \hspace*{2em} = (x + \hspace*{2em})^2$;} & \xxt{$x^2 - \dfrac{5}{2}x + \hspace*{2em} = (x - \hspace*{2em})^2$;} \\
        \xxt{$x^2 + px + \hspace*{2em} = (x + \hspace*{2em})^2$;} & \xxt{$x^2 + \dfrac{b}{a}x + \hspace*{2em} = (x + \hspace*{2em})^2$。}
    \end{tblr}
\end{xiaoxiaotis}

\xiaoti{用配方法解下列方程:}
\begin{xiaoxiaotis}

    \begin{tblr}{columns={12em, colsep=0pt}}
        \xxt{$x^2 - 6x + 4 = 0$;} & \xxt{$2t^2 - 7t - 4 = 0$;} &  \xxt{$3x^2 - 1 = 6x$。}
    \end{tblr}
\end{xiaoxiaotis}

\end{xiaotis}
