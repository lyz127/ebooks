% 原书的目录结构就是如此(缺少 section)
% 忽略这里的报错: Difference (2) between bookmark levels is greater (hyperref)	than one, level fixed.
\subsection{平方根}\label{subsec:9-1}

\begin{enhancedline}
如果已经知道一个正方形的边长,我们可以通过平方运算求出正方形的面积。
反过来,如果已知正方形的面积,这个正方形的边长应该怎样计算?

例如,我们需要一个面积是 $9$ 平方米的正方形钢板,就要先求出这个正方形钢板的边长是多少米,
也就是要求出一个平方后等于 $9$ 的数。因为 $3^2 = 9$,$(-3)^2 = 9$,
而实际上正方形钢板的边长不能是负数,所以,这个正方形钢板的边长是 $3$ 米。

一般地,如果一个数的平方等于 $a$,这个数就叫做 $a$ 的\zhongdian{平方根}(也叫做\zhongdian{二次方根})。
换句话说,如果 $x^2 = a$,那么,$x$ 就叫做 $a$ 的平方根。
例如,因为 $3^2 = 9$,$(-3)^2 = 9$,所以 $3$ 与 $-3$ 都是 9 的平方根。
又如,因为 $7^2 = 49$,$(-7)^2 = 49$, 所以 $7$ 与 $-7$ 都是 49 的平方根。
类似地,因为 $\left(\dfrac{2}{5}\right)^2 = \dfrac{4}{25}$,$\left(-\dfrac{2}{5}\right)^2 = \dfrac{4}{25}$,
所以 $\dfrac{2}{5}$ 与 $-\dfrac{2}{5}$ 都是 $\dfrac{4}{25}$ 的平方根。

根据平方运算可以知道,不为零的两个数互为相反数时,它们的平方是同一个正数。
一般地,\zhongdian{一个正数有两个平方根,这两个平方根互为相反数}。

因为 $0^2 = 0$, 所以\zhongdian{零的平方根是零}。

任何正数、负数的平方都是正数,零的平方是零,也就是正数、负数和零的平方都不是负数,
因此\zhongdian{负数没有平方根}。例如,$-4$ 没有平方根。

求一个数 $a$ 的平方根的运算,叫做\zhongdian{开平方}。

开平方与平方互为逆运算,因此,我们可以运用平方运算求一个数的平方根,
也可以用平方运算检验一个数是不是另一个数的平方根,例如,

$\because \quad 3^2 = 9 \nsep (-3)^2 = 9$,

$\therefore$ \quad 9 的两个平方根是 $3$ 和 $-3$。

一个正数 $a$ 的正的平方根,用符号 “$\sqrt[2]{a}$” 表示;负的平方根, 用符号 “$-\sqrt[2]{a}$” 表示。
这两个平方根合起来可以记做 “$\pm\sqrt[2]{a}$”, 这里符号 “$\sqrt[2]{\phantom{a}}$” 读作 “二次根号”,
$a$ 叫做\zhongdian{被开方数}, $2$ 叫做\zhongdian{根指数}。
根指数是 $2$ 时,通常省略不写,如 $\pm\sqrt[2]{a}$ 写做 $\pm \sqrt{a}$,读作 “正、负根号 $a$ ”。
$\pm\sqrt{0}$就是 0 。

\zhuyi 因为负数没有平方根,所以 $\sqrt{a}$ 中的被开方数要大于或等于零,即 $a \geqslant 0$。

\liti 求下列各数的平方根:
\begin{xiaoxiaotis}

    \hspace*{1.5em} \fourInLineXxt[6em]{$36$;}{$\dfrac{16}{25}$;}{$2\dfrac{1}{4}$;}{$0.49$。}

\resetxxt
\jie \xxt{\begin{tblr}[t]{}
        $\because$   & $(\pm 6)^2 = 36$, \\
        $\therefore$ & 36 的平方根是 $\pm 6$,即 \\
                     & $\pm \sqrt{36} = \pm 6$;
\end{tblr}}

\hspace*{1.5em} \xxt{\begin{tblr}[t]{}
    $\because$   & $\left(\pm \dfrac{4}{5}\right)^2 = \dfrac{16}{25}$, \\
\end{tblr}}

\hspace*{4.5em} \begin{tblr}[t]{} % 为了让本小节“练习”第4题的图片不会被显示到下一页去,这里手工将表格拆分,让第一行显示到上一页。
    $\therefore$ & $\dfrac{16}{25}$ 的平方根是 $\pm \dfrac{4}{5}$,即 \\
                 & $\pm \sqrt{\dfrac{16}{25}} = \pm \dfrac{4}{5}$;
\end{tblr}

\hspace*{1.5em} \xxt{\begin{tblr}[t]{}
    $\because$   & $2\dfrac{1}{4} = \dfrac{9}{4} \nsep \left(\pm \dfrac{3}{2}\right)^2 = \dfrac{9}{4}$, \\
    $\therefore$ & $2\dfrac{1}{4}$ 的平方根是 $\pm \dfrac{3}{2}$,即 \\
                 & $\pm \sqrt{2\dfrac{1}{4}} = \sqrt{\dfrac{9}{4}} = \pm \dfrac{3}{2}$;
\end{tblr}}

\hspace*{1.5em} \xxt{\begin{tblr}[t]{}
    $\because$   & $(\pm 0.7)^2 = 0.49$, \\
    $\therefore$ & $0.49$ 的平方根是 $\pm 0.7$,即 \\
                 & $\pm \sqrt{0.49} = \pm 0.7$。
\end{tblr}}

\end{xiaoxiaotis}

\liti 判断下列各数有没有平方根:
\begin{xiaoxiaotis}

    \hspace*{1.5em} \fourInLineXxt[6em]{64;}{$-64$;}{0;}{($-4)^2$。}

\resetxxt
\jie \xxt{因为 $64 > 0$,所以 64 有平方根;}

\hspace*{1.5em} \xxt{因为 $-64 < 0$,所以 $-64$ 没有平方根;}

\hspace*{1.5em} \xxt{0 有平方根;}

\hspace*{1.5em} \xxt{因为 $(-4)^2 = 16 > 0$,所以 $(-4)^2$ 有平方根。}

\end{xiaoxiaotis}

\lianxi
\begin{xiaotis}

\xiaoti{(口答)}
\begin{xiaoxiaotis}

    \xxt{什么数的平方等于 $81$?}

    \xxt{什么数的平方等于 $\dfrac{4}{9}$?}

    \xxt{什么数的平方等于 $0.25$?}

\end{xiaoxiaotis}

\xiaoti{求下列各数的平方根:\\
    \hspace*{2em}$1 \nsep 64  \nsep 1600 \nsep 0 \nsep 0.0081 \nsep \dfrac{49}{100} \nsep 2.25 \nsep \dfrac{25}{144}$。
}

\xiaoti{求下列各数的平方根:\\
    \hspace*{2em}$729 \nsep 324 \nsep 0.81 \nsep 0.0225 \nsep \dfrac{16}{49} \nsep \dfrac{25}{64}$。
}

\xiaoti{如图,求左圈和右圈中的 “?”:}


\begin{figure}[htbp]
    \centering
    \begin{tikzpicture}[>=Stealth]
    \foreach \a/\b/\c [count=\xi] in {
            \hphantom{+}8/-8/?,
            \hphantom{+}\frac{3}{4}/-\frac{3}{4}/?,
            \hphantom{+}?/\hphantom{+}?/121,
            \hphantom{+}?/\hphantom{+}?/0.36
    } {
        \coordinate (A) at (0, 4-\xi);
        \coordinate (B) at (0, 4-\xi - 0.5);
        \coordinate (C) at (4, 4-\xi-0.25);
        \path
            let
                \p{ac} = ($ (A) !.5! (C) $),
                \p{bc} = ($ (B) !.5! (C) $)
            in
                coordinate (AC) at (\p{ac})
                coordinate (BC) at (\p{bc});
        \draw [->] (A) node [anchor=east] {$\a$} -- (AC);
        \draw (AC) -- (C);
        \draw [->] (B) node [anchor=east] {$\b$} -- (BC);
        \draw (BC) -- (C) node [anchor=west] {$\c$};
    }

    \draw (0,   1.3) ellipse [x radius=1.0, y radius=2.2];
    \draw (4.2, 1.3) ellipse [x radius=1.0, y radius=2.2];
    \draw (0,   3.8) node {$x$};
    \draw (4.2, 3.8) node {$x^2$};
\end{tikzpicture}

    \caption*{(第 4 题)}
\end{figure}


\end{xiaotis}
\end{enhancedline}