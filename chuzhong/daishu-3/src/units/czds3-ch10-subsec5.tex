\subsection{二次根式的乘法}\label{subsec:10-5}
\begin{enhancedline}

把公式 $\sqrt{ab} = \sqrt{a} \cdot \sqrt{b}$ 反过来,就得
$$ \sqrt{a} \cdot \sqrt{b} = \sqrt{ab} \juhao $$

运用这个公式,可以进行二次根式的乘法运算。
就是说,二次根式相乘,仍得二次根式,把被开方数的积作为积的被开方数。

\liti 计算:
\begin{xiaoxiaotis}

    \hspace*{1.5em} \begin{tblr}[t]{columns={colsep=0pt}, column{1}={20em}}
        \xxt{$\sqrt{14} \cdot \sqrt{7}$;} & \xxt{$3\sqrt{5a} \cdot 2\sqrt{10b}$。}
    \end{tblr}

\resetxxt
\jie \begin{tblr}[t]{columns={colsep=0pt}, column{1}={20em}}
    \xxt{$\begin{aligned}[t]
        \sqrt{14} \cdot \sqrt{7} &= \sqrt{14 \times 7} = \sqrt{7^2 \times 2} \\
                                 &= 7\sqrt{2} \fenhao
    \end{aligned}$} & \xxt{$\begin{aligned}[t]
        3\sqrt{5a} \cdot 2\sqrt{10b} &= 3 \times 2\sqrt{5a \cdot 10b} \\
                                     &= 30\sqrt{2ab} \juhao
    \end{aligned}$}
\end{tblr}

\end{xiaoxiaotis}


\zhuyi 二次根式运算的结果,如果含有二次根式,一般要化成最简二次根式。


\liti 计算:
\begin{xiaoxiaotis}

    \hspace*{1.5em} \begin{tblr}[t]{columns={colsep=0pt}, column{1}={20em}}
        \xxt{$\left(\sqrt{\dfrac{8}{27}} - 5\sqrt{3}\right) \cdot \sqrt{6}$;} & \xxt{$(5 + \sqrt{6}) (5\sqrt{2} - 2\sqrt{3})$。}
    \end{tblr}

\resetxxt
\jie \begin{tblr}[t]{columns={colsep=0pt}, column{1}={20em}}
    \xxt{$\begin{aligned}[t]
        & \left(\sqrt{\dfrac{8}{27}} - 5\sqrt{3}\right) \cdot \sqrt{6} \\
        &= \sqrt{\dfrac{8}{27}} \cdot \sqrt{6} - 5\sqrt{3} \cdot \sqrt{6} \\
        &= \sqrt{\dfrac{8}{27} \times 6} - 5\sqrt{3 \times 6} \\
        &= \dfrac{4}{3} - 15\sqrt{2} \fenhao
    \end{aligned}$} & \xxt{$\begin{aligned}[t]
        & (5 + \sqrt{6}) (5\sqrt{2} - 2\sqrt{3}) \\
        &= 25\sqrt{2} - 10\sqrt{3} + 5\sqrt{12} - 2\sqrt{18} \\
        &= 25\sqrt{2} - 10\sqrt{3} + 10\sqrt{3} - 6\sqrt{2} \\
        &= 19\sqrt{2} \juhao
    \end{aligned}$}
\end{tblr}

\end{xiaoxiaotis}


二次根式的和相乘,与多项式的乘法相类似。
遇到适用多项式乘法公式的时候,也可以运用乘法公式。

\liti 计算:
\begin{xiaoxiaotis}

    \hspace*{1.5em} \begin{tblr}[t]{columns={colsep=0pt}, column{1}={20em}}
        \xxt{$(2\sqrt{3} + 3\sqrt{2}) (2\sqrt{3} - 3\sqrt{2})$;} & \xxt{$(4 + 3\sqrt{5})^2$;} \\
        \xxt{$(\sqrt{6} - 3\sqrt{3})^2$。}
    \end{tblr}

\resetxxt
\jie \begin{tblr}[t]{columns={colsep=0pt}, column{1}={20em}}
    \xxt{$\begin{aligned}[t]
        & (2\sqrt{3} + 3\sqrt{2}) (2\sqrt{3} - 3\sqrt{2}) \\
        &= (2\sqrt{3})^2 - (3\sqrt{2})^2 \\
        &= 12 - 18 \\
        &= -6 \fenhao
    \end{aligned}$} & \xxt{$\begin{aligned}[t]
        & (4 + 3\sqrt{5})^2 \\
        &= 4^2 + 2 \cdot 4 \cdot 3\sqrt{5} + (3\sqrt{5})^2 \\
        &= 16 + 24\sqrt{5} + 45 \\
        &= 61 + 24\sqrt{5} \juhao
    \end{aligned}$} \\
    \xxt{$\begin{aligned}[t]
        & (\sqrt{6} - 3\sqrt{3})^2 \\
        &= (\sqrt{6})^2 - 2\cdot \sqrt{6} \cdot 3\sqrt{3} + (3\sqrt{3})^2 \\
        &= 6 - 18\sqrt{2} + 27 \\
        &= 33 - 18\sqrt{2} \juhao
    \end{aligned}$}
\end{tblr}

\end{xiaoxiaotis}




\liti 计算:
\begin{xiaoxiaotis}

    \hspace*{1.5em} \begin{tblr}[t]{columns={colsep=0pt}, column{1}={20em}}
        \xxt{$(\sqrt{3} +  \sqrt{6}) (\sqrt{3} - \sqrt{6})$;} & \xxt{$(2\sqrt{ax} - 5\sqrt{by}) (2\sqrt{ax} + 5\sqrt{by})$。}
    \end{tblr}

\resetxxt
\jie \begin{tblr}[t]{columns={colsep=0pt}, column{1}={20em}}
    \xxt{$\begin{aligned}[t]
        & (\sqrt{3} +  \sqrt{6}) (\sqrt{3} - \sqrt{6}) \\
        &= (\sqrt{3})^2 - (\sqrt{6})^2 \\
        &= 3 - 6 \\
        &= -3 \fenhao
    \end{aligned}$} & \xxt{$\begin{aligned}[t]
        & (2\sqrt{ax} - 5\sqrt{by}) (2\sqrt{ax} + 5\sqrt{by}) \\
        &= (2\sqrt{ax})^2 - (5\sqrt{by})^2 \\
        &= 4ax - 25by \juhao
    \end{aligned}$}
\end{tblr}

\end{xiaoxiaotis}


\lianxi
\begin{xiaotis}

\xiaoti{计算:}
\begin{xiaoxiaotis}

    \begin{tblr}{columns={12em, colsep=0pt}, rows={rowsep=0.5em}}
        \xxt{$\sqrt{5} \cdot \sqrt{3}$;}   & \xxt{$6\sqrt{27} \cdot (-2\sqrt{3})$;}          & \xxt{$9\sqrt{45} \times \dfrac{3}{2}\sqrt{2\dfrac{2}{3}}$;} \\
        \xxt{$\sqrt{6x} \cdot \sqrt{2x}$;} & \xxt{$\dfrac{a}{b}\sqrt{\dfrac{b}{a}} \cdot \dfrac{b}{a}\sqrt{\dfrac{a}{b}}$;} & \xxt{$10x\sqrt{y} \cdot \sqrt{\dfrac{1}{x}}$。}
    \end{tblr}

\end{xiaoxiaotis}



\xiaoti{计算:}
\begin{xiaoxiaotis}

    \begin{tblr}{columns={18em, colsep=0pt}}
        \xxt{$(\sqrt{12} - 3\sqrt{75}) \cdot \sqrt{3}$;}   & \xxt{$2\sqrt{5} (\sqrt{10} + 4\sqrt{12})$;} \\
        \xxt{$(\sqrt{2} + 2\sqrt{12} - \sqrt{6}) \cdot 2\sqrt{3}$;} & \xxt{$3\sqrt{6} (3\sqrt{2} - \sqrt{15})$。}
    \end{tblr}

\end{xiaoxiaotis}


\xiaoti{计算:}
\begin{xiaoxiaotis}

    \begin{tblr}{columns={18em, colsep=0pt}}
        \xxt{$(2\sqrt{3} - 2) (3\sqrt{2} - 3)$;}   & \xxt{$\left(\dfrac{\sqrt{5}}{3} - 2\sqrt{3}\right) \left(3\sqrt{5} - \dfrac{1}{2}\sqrt{3}\right)$;} \\
        \xxt{$(\sqrt{a} + \sqrt{b}) (\sqrt{a} - \sqrt{c})$;} & \xxt{$(2\sqrt{x} + \sqrt{y}) (\sqrt{x} - \sqrt{y})$。}
    \end{tblr}

\end{xiaoxiaotis}


\xiaoti{计算:}
\begin{xiaoxiaotis}

    \begin{tblr}{columns={colsep=0pt}}
        \xxt{$(4 - 3\sqrt{5}) (4 + 3\sqrt{5})$;}   & \xxt{$(7\sqrt{2} + 2\sqrt{6}) (2\sqrt{6} - 7\sqrt{2})$;} \\
        \xxt{$(\sqrt{4x + 3} - \sqrt{2x}) (\sqrt{4x + 3} + \sqrt{2x})$;}   & \xxt{$(\sqrt{3} + 2\sqrt{2})^2$;} \\
        \xxt{$\left(\dfrac{-1 - \sqrt{3}}{2}\right)^2$;}   & \xxt{$(4\sqrt{7} - 7\sqrt{3})^2$;} \\
        \xxt{$\left(\sqrt{\dfrac{a}{b}} + \sqrt{\dfrac{b}{a}}\right)^2$;}   & \xxt{$(\sqrt{x} + \sqrt{y})^2 + (\sqrt{x} - \sqrt{y})^2$;} \\
        \xxt{$(\sqrt{2} + \sqrt{3} - \sqrt{6})^2 - (\sqrt{2} - \sqrt{3} + \sqrt{6})^2$;}   & \xxt{$(1 + \sqrt{2} - \sqrt{3}) (1 - \sqrt{2} + \sqrt{3})$。}
    \end{tblr}

\end{xiaoxiaotis}

\end{xiaotis}
\end{enhancedline}

