\subsection{一元二次方程的根与系数的关系}\label{subsec:11-5}
\begin{enhancedline}

在解一元二次方程时,我们发现它的根与系数之间有一定的关系。

例如,解方程 $x^2 -5x + 6 = 0$,得
$$ x_1 = 2 \nsep x_2 = 3 \juhao $$

可以看出,$x_1 + x_2 = 5$,是一次项系数 $-5$ 的相反数; $x_1 \cdot x_2 = 6$,是常数项。

又如,解方程 $2x^2 + 5x - 3 = 0$,得
$$ x_1 = \dfrac{1}{2} \nsep x_2 = -3 \juhao $$

可以看出,$x_1 + x_2 = -\dfrac{5}{2}$,是一次项系数 $5$ 除以二次项系数 $2$ 所得的商的相反数;
$x_1 \cdot x_2 = -\dfrac{3}{2}$,是常数项 $-3$ 除以二次项系数 $2$ 所得的商。

一般地,对于一元二次方程 $ax^2 + bx + c = 0 \quad (a \neq 0)$,

\begin{tblr}[]{columns={$}, column{1}={2em}, rows={rowsep=0.5em}}
    \because   &  x_1 = \dfrac{-b + \sqrt{b^2 - 4ac}}{2a}, \quad  x_2 = \dfrac{-b - \sqrt{b^2 - 4ac}}{2a},  \\
    \therefore & \begin{aligned}[t]
                    x_1 + x_2 &= \dfrac{-b + \sqrt{b^2 - 4ac}}{2a} + \dfrac{-b - \sqrt{b^2 - 4ac}}{2a} \\[.5em]
                              &= \dfrac{-2b}{2a} = -\dfrac{b}{a} \fenhao
                 \end{aligned}
\end{tblr}

\begin{tblr}[]{columns={$}, column{1}={2em}, rows={rowsep=0.5em}} % 由于分页,此处手工分拆表格
               & \begin{aligned}[t]
                    x_1 \cdot x_2 &= \dfrac{-b + \sqrt{b^2 - 4ac}}{2a} \cdot \dfrac{-b - \sqrt{b^2 - 4ac}}{2a} \\[.5em]
                                  &= \dfrac{(-b)^2 - (\sqrt{b^2 - 4ac})^2}{4a^2} \\[.5em]
                                  &= \dfrac{4ac}{4a^2} = \dfrac{c}{a} \juhao
                 \end{aligned}
\end{tblr}


由此得出,一元二次方程的根与系数有下列关系:

\jiange
\framebox{\begin{minipage}{0.93\textwidth}
    \zhongdian{如果 $\bm{ax^2 + bx + c = 0 \quad (a \neq 0)}$ 的两个根是 $\bm{x_1}$, $\bm{x_2}$,}

    \begin{enhancedline}
        \zhongdian{那么 $\bm{x_1 + x_2 = -\dfrac{b}{a}}$, $\bm{x_1 \cdot x_2 = \dfrac{c}{a}}$。}
    \end{enhancedline}
\end{minipage}}
\jiange

如果把方程 $ax^2 + bx + c = 0 \quad (a \neq 0)$ 变形为
$$ x^2 + \dfrac{b}{a}x + \dfrac{c}{a} = 0 \douhao $$
我们就可以把它写成
$$ x^2 + px + q = 0 $$
的形式,其中 $p = \dfrac{b}{a}$, $q = \dfrac{c}{a}$。从而得出:

\zhongdian{如果方程 $\bm{x^2 + px + q = 0}$ 的两个根是 $\bm{x_1}$, $\bm{x_2}$, 那么}
$$ \bm{x_1 + x_2 = -p \nsep x_1 \cdot x_2 = q} \juhao $$

\liti 已知方程 $5x^2 + kx - 6 = 0$ 的一个根是 $2$,求它的另一个根及 $k$ 的值。

\jie 设方程的另一个根是 $x_1$,那么
$$ x_1 \cdot 2 = -\dfrac{6}{5} \douhao $$

\fengeSuoyi{x_1 = -\dfrac{3}{5} \juhao}

又
$$ \left(-\dfrac{3}{5}\right) + 2 = -\dfrac{k}{5} \douhao $$

\fengeSuoyi{k = -5\left[\left(-\dfrac{3}{5}\right) + 2\right] = -7 \juhao}

答:方程的另一个根是 $-\dfrac{3}{5}$, $k = -7$。

试一试,能不能把 $x = 2$ 代入原方程,先求出 $k$ 的值,再求出另一个根。


\liti 利用根与系数的关系,求一元二次方程 $2x^2 + 3x - 1 = 0$ 两个根的
\begin{xiaoxiaotis}

    \hspace*{2em} \xxt{平方和}; \xxt{倒数和}。

\resetxxt
\jie 设方程的两个根是 $x_1$,$x_2$,那么
$$ x_1 + x_2 = -\dfrac{3}{2} \nsep x_1 \cdot x_2 = -\dfrac{1}{2} \juhao $$

\xxt{\begin{tblr}[t]{columns={$}}
    \because   & (x_1 + x_2)^2 = x_1^2 + 2x_1x_2 + x_2^2 \douhao \\
    \therefore & \begin{aligned}[t]
                    x_1^2 + x_2^2 &= (x_1 + x_2)^2 - 2x_1x_2 \\
                                  &= \left(-\dfrac{3}{2}\right)^2 - 2 \times \left(-\dfrac{1}{2}\right) = \dfrac{13}{4} \fenhao
                 \end{aligned}
\end{tblr}}

\xxt{$\begin{aligned}[t]
    \dfrac{1}{x_1} + \dfrac{1}{x_2} &= \dfrac{x_1 + x_2}{x_1 x_2} \\
                                    &= \dfrac{-\dfrac{3}{2}}{-\dfrac{1}{2}} = 3 \juhao
\end{aligned}$}

答:方程两个根的平方和是 $\dfrac{13}{4}$,倒数和是 $3$。

\end{xiaoxiaotis}


\lianxi
\begin{xiaotis}

\xiaoti{(口答)下列各方程中,两根的和与两根的积各是多少?}
\begin{xiaoxiaotis}

    \begin{tblr}{columns={18em, colsep=0pt}}
        \xxt{$x^2 - 3x + 1 = 0$;}  & \xxt{$3x^2 - 2x - 2 = 0$;} \\
        \xxt{$2x^2 - 9x + 5 = 0$;} & \xxt{$4x^2 - 7x + 1 = 0$;} \\
        \xxt{$2x^2 + 3x = 0$;}     & \xxt{$3x^2 - 1 = 0$。}
    \end{tblr}
\end{xiaoxiaotis}


\xiaoti{已知方程 $3x^2 - 19x + m = 0$ 的一个根是 $1$,求它的另一个根及 $m$ 的值。}

\xiaoti{设 $x_1$,$x_2$ 是方程 $2x^2 + 4x - 3 = 0$ 的两个根,利用根与系数的关系,求下列各式的值:}
\begin{xiaoxiaotis}

    \begin{tblr}{columns={18em, colsep=0pt}}
        \xxt{$(x_1 + 1)(x_2 + 1)$;} & \xxt{$\dfrac{x_2}{x_1} + \dfrac{x_1}{x_2}$。}
    \end{tblr}
\end{xiaoxiaotis}

\end{xiaotis}
\lianxijiange

设有两个数 $x_1$,$x_2$。假定以这两个数为根的一元二次方程(二次项系数为 $1$)是
$$ x^2 + px + q = 0 \douhao $$
那么,由根与系数的关系,得
$$ x_1 + x_2 = -p \nsep x_1 \cdot x_2 = q \juhao $$
因此,
$$ p = -(x_1 + x_2) \nsep q = x_1 \cdot x_2 \douhao $$
方程 $x^2 + px + q = 0$ 就是
$$ x^2 - (x_1 + x_2)x + x_1 \cdot x_2 = 0 \juhao $$

这就是说,\zhongdian{以两个数 $\bm{x_1}$, $\bm{x_2}$ 为根的一元二次方程(二次项系数为 $\bm{1}$)是}
$$ \bm{x^2 - (x_1 + x_2)x + x_1 \cdot x_2 = 0} \juhao $$

\liti 求一个一元二次方程,使它的两个根是 $-3\dfrac{1}{3}$, $2\dfrac{1}{2}$。

\jie 所求的方程是
$$ x^2 - \left(-3\dfrac{1}{3} + 2\dfrac{1}{2}\right)x + \left(-3\dfrac{1}{3}\right) \times 2\dfrac{1}{2} = 0 \juhao $$
即
$$ x^2 + \dfrac{5}{6}x - \dfrac{25}{3} = 0 \douhao $$
或
$$ 6x^2 + 5x - 50 = 0 \juhao $$


\liti 已知两个数的和等于 $8$,积等于 $9$,求这两个数。

\jie 因为这两个数的和等于 $8$,积等于 $9$,所以这两个数是方程
$$ x^2 - 8x + 9 = 0 $$
的两个根。

解这个方程,得
$$ x_1 = 4 + \sqrt{7} \nsep x_2 = 4 - \sqrt{7} \juhao $$

因此,这两个数是 $4 + \sqrt{7}$, $4 - \sqrt{7}$。

\end{enhancedline}


\liti 已知方程 $x^2 - 2x - 1 = 0$,利用根与系数的关系求一个一元二次方程,使它的根是原方程的各根的立方。

\jie 设方程 $x^2 - 2x - 1 = 0$ 的两个根是 $x_1$, $x_2$, 那么所求的方程的根是 $x_1^3$, $x_2^3$。

$\because \quad x_1 + x_2 = 2 \nsep x_1 \cdot x_2 = 1 \juhao$

$\therefore \quad \begin{aligned}[t]
    x_1^3 + x_2^3 &= (x_1 + x_2) (x_1^2 - x_1x_2 + x_2^2) \\
                  &= (x_1 + x_2)[(x_1 + x_2)^2 - 3x_1x_2] \\
                  &= 2[2^2 - 3 \times (-1)] = 14 \douhao
\end{aligned}$

\hspace*{2.5em} $x_1^3 \cdot x_2^3 = (x_1 \cdot x_2)^3 = (-1)^3 = -1 \juhao $

因此,所求的方程是
$$ y^2 - 14y - 1 = 0 \juhao $$


\lianxi
\begin{xiaotis}

\xiaoti{(口答)判定下列各方程后面括号内的两个数是不是它的根:}
\begin{xiaoxiaotis}
\begin{enhancedline}

    \xxt{$x^2 + 5x + 4 = 0$,\quad $(1 \nsep 4)$;}

    \xxt{$x^2 - 6x - 7 = 0$,\quad $(-1 \nsep 7)$;}

    \xxt{$2x^2 - 3x + 1 = 0$,\quad $\left(\dfrac{1}{2} \nsep 1\right)$;}

    \xxt{$3x^2 + 5x - 2 = 0$,\quad $\left(-\dfrac{1}{3} \nsep 2\right)$;}

    \xxt{$x^2 - 8x + 11 = 0$,\quad $(4 - \sqrt{5} \nsep 4 + \sqrt{5})$;}

    \xxt{$x^2 - 4x + 1 = 0$,\quad $(-2 - \sqrt{3} \nsep -2 + \sqrt{3})$。}

\end{enhancedline}
\end{xiaoxiaotis}


\xiaoti{求一个一元二次方程,使它的两个根分别为}
\begin{xiaoxiaotis}

    \begin{tblr}{columns={18em, colsep=0pt}}
        \xxt{$4 \nsep -7$;} & \xxt{$1 + \sqrt{3} \nsep 1 - \sqrt{3}$。}
    \end{tblr}
\end{xiaoxiaotis}


\xiaoti{已知两个数的和等于 $-6$, 积等于 $2$, 求这两个数。 }

\xiaoti{利用根与系数的关系,求一个一元二次方程,使它的根分别是方程 $x^2 + 3x - 2 = 0$ 的各根的 $2$ 倍。}

\end{xiaotis}


