\subsubsection{直接开平方法}

我们来解方程
$$ x^2 = 4 \juhao $$
因为 $x$ 就是 $4$ 的平方根,所以
$$ x = \pm \sqrt{4} \douhao $$
即
$$ x_1 = 2 \nsep x_2 = -2 \;\footnote{通常用 $x_1, \; x_2$ 表示未知数为 $x$ 的一元二次方程的两个根。} \juhao $$

这种解一元二次方程的方法叫做\zhongdian{直接开平方法}。

\liti 解方程 $x^2 - 25 = 0$。

\jie 移项,得
$$ x^2 = 25 \juhao $$
所以
$$ x = \pm \sqrt{25} \douhao $$
即
$$ x_1 = 5 \nsep x_2 = -5 \juhao $$


\liti 解方程 $(x + 3)^2 = 2$。

分析:原方程中 $x + 3$ 是 $2$ 的平方根,因此,也可运用直接开平方法来解。

\jie $x + 3 = \pm \sqrt{2}$,

即

\hspace*{1.5em} $x + 3 = \sqrt{2} \text{,或 \quad} x + 3 = -\sqrt{2}$。

$\therefore$ \quad  $x_1 = -3 + \sqrt{2} \nsep x_2 = -3 - \sqrt{2}$。

这就是说,如果一元二次方程的一边是一个含有未知数的式子的平方,另一边是一个非负常数,同样可以用直接开平方法来解。


\lianxi

用直接开平方法解下列方程:
\begin{xiaotis}

\xiaoti{$x^2 = 256$。}

\xiaoti{$4y^2 = 9$。}

\xiaoti{$16x^2 - 49 = 0$。}

\xiaoti{$t^2 - 45 = 0$。}

\xiaoti{$(2x - 3)^2 = 5$。}

\xiaoti{$(x + 1)^2 - 12 = 0$。}

\end{xiaotis}


