% 原书的目录结构就是如此(缺少 section)
% 忽略这里的报错: Difference (2) between bookmark levels is greater (hyperref)	than one, level fixed.
\subsection{二次根式}\label{subsec:10-1}
\begin{enhancedline}

式子 $\sqrt{a} \; (a \geqslant 0)$ 叫做\zhongdian{二次根式}。
例如,$\sqrt{3}$, $\sqrt{\dfrac{3}{5}}$, $\sqrt{b^2 + 1}$,
$\sqrt{(a - b)^2}$ 等都是二次根式。
在实数范围内,负数没有平方根,所以 $\sqrt{-5}$, $\sqrt{a} \; (a < 0)$ 没有意义。
在本章里,如果没有特别说明,所有字母都表示正数。

我们在前一章学过,正数 $a$ 的正的平方根,也叫做 $a$ 的算术平方根,记作 $\sqrt{a}$。
这表明,$\sqrt{a}$ 是一个正数,即 $\sqrt{a} > 0$。
零的平方根也叫做零的算术平方根,记作 $\sqrt{0}$。我们有 $\sqrt{0} = 0$。
从上面的分析可以看出,$\sqrt{a} \geqslant 0 \; (a \geqslant 0)$,
即 $\sqrt{a} \; (a \geqslant 0)$ 总是一个非负数(正数和零统称非负数)。

根据平方根的意义,如果一个数的平方等于 $2$,这个数就叫做 $2$ 的平方根。因此,可以知道,
$$ (\sqrt{2})^2 = 2 \juhao $$

一般地,如果一个数的平方等于 $a$,这个数就叫做 $a$ 的平方根。因此,我们有
\begin{center}
    \framebox{\quad $(\sqrt{a})^2 = a \quad (a \geqslant 0)$。\;}
\end{center}

\liti 计算:
\begin{xiaoxiaotis}

    \hspace*{1.5em} \fourInLineXxt[8em]{$(\sqrt{4})^2$;}
        {$(\sqrt{5})^2$;}
        {$\left(\sqrt{\dfrac{3}{5}}\right)^2$;}
        {$(2\sqrt{3})^2$。\footnotemark}
\footnotetext{$(2\sqrt{3})^2$ 表示 $2 \times \sqrt{3}$;一般地,$b\sqrt{a}$ 表示 $b \times \sqrt{a}$。}

\resetxxt
\jie \begin{tblr}[t]{columns={colsep=0pt}}
    \xxt{$(\sqrt{4})^2 = 2^2 = 4$;\\
        或 $(\sqrt{4})^2 = 4$;
    } \\
    \xxt{$(\sqrt{5})^2 = 5$;} \\
    \xxt{$\left(\sqrt{\dfrac{3}{5}}\right)^2 = \dfrac{3}{5}$;} \\
    \xxt{$(2\sqrt{3})^2 = 2^2 \times (\sqrt{3})^2 = 4 \times 3 = 12$。}
\end{tblr}

\end{xiaoxiaotis}

把上面的公式 $(\sqrt{a})^2 = a \; (a \geqslant 0)$ 反过来,就得到
$$ a = (\sqrt{a})^2 \quad (a \geqslant 0) \juhao$$
利用这个公式,可以把任何一个非负数写成一个数的平方的形式。


\liti 把下面的非负数写成平方的形式:
\begin{xiaoxiaotis}

    \hspace*{1.5em} \fourInLineXxt[8em]{$2$;}{$0.5$;}{$\dfrac{1}{7}$;}{$ab$。}

\resetxxt
\jie \begin{tblr}[t]{columns={18em, colsep=0pt}, rows={rowsep=0.5em}}
    \xxt{$(2 = (\sqrt{2})^2$;}  &  \xxt{$0.5 = (\sqrt{0.5})^2$;} \\
    \xxt{$\dfrac{1}{7} = \left(\sqrt{\dfrac{1}{7}}\right)^2$;} & \xxt{$ab = (\sqrt{ab})^2$。}
\end{tblr}

\end{xiaoxiaotis}


\lianxi
\begin{xiaotis}

\xiaoti{计算:}
\begin{xiaoxiaotis}

    \fourInLineXxt[9em]{$(\sqrt{0.5})^2$;}
        {$\left(\sqrt{\dfrac{2}{7}}\right)^2$;}
        {$(5\sqrt{7})^2$;}
        {$\left(-3\sqrt{\dfrac{1}{3}}\right)^2$。}

\end{xiaoxiaotis}

\xiaoti{把下面的非负数写成平方的形式:}
\begin{xiaoxiaotis}

    \begin{tblr}{columns={12em, colsep=0pt}}
        \xxt{$9$;}     & \xxt{$6$;} & \xxt{$2.5$;} \\
        \xxt{$0.25$;}  & \xxt{$b$;} & \xxt{$4a$。}
    \end{tblr}

\end{xiaoxiaotis}
\end{xiaotis}
\lianxijiange

根据算术平方根的意义,我们分 $a > 0$, $a = 0$, $a < 0$ 三种情况来研究根式 $\sqrt{a^2}$。

(1) $\sqrt{2^2} = 2$, $\sqrt{3^2} = 3$。

一般地,当 $a > 0$ 时, $\sqrt{a^2} = a$。

(2) $\sqrt{0^2} = 0$。

也就是说,当 $a = 0$ 时, $\sqrt{a^2} = 0$。

(3) $\sqrt{(-2)^2} = \sqrt{4} = 2$, $2$ 与 $-2$ 互为相反数,即 $2 = -(-2)$;

\hspace*{1.5em} $\sqrt{(-3)^2} = \sqrt{9} = 3$, $3$ 与 $-3$ 互为相反数,即 $3 = -(-3)$。

一般地,当 $a < 0$ 时, $\sqrt{a^2} = -a$。

综合上面的结果,有
$$\sqrt{a^2} = \begin{cases}
    \phantom{-}a  \quad (a > 0) \douhao \\
    \phantom{-}0  \quad (a = 0) \douhao \\
             -a   \quad (a < 0) \juhao
\end{cases}$$

我们已经知道,
$$|a| = \begin{cases}
    \phantom{-}a  \quad (a > 0) \douhao \\
    \phantom{-}0  \quad (a = 0) \douhao \\
             -a   \quad (a < 0) \juhao
\end{cases}$$

比较 $\sqrt{a^2}$ 与 $|a|$,就得到
\begin{center}
    \framebox{\quad $\sqrt{a^2} = |a| = \begin{cases}
        \phantom{-}a  \quad (a > 0) \douhao \\
        \phantom{-}0  \quad (a = 0) \douhao \\
                 -a   \quad (a < 0) \juhao
    \end{cases}$\;}
\end{center}


\liti 计算:
\begin{xiaoxiaotis}

    \hspace*{1.5em} \begin{tblr}{columns={colsep=0pt}}
        \xxt{$\sqrt{(-1.5)^2}$;} & \xxt{$\sqrt{(a - 3)^2} \quad (a < 3)$。}
    \end{tblr}
    %\twoInLineXxt[12em]{$\sqrt{(-1.5)^2}$;}{$\sqrt{(a - 3)^2} \; (a < 3)$。}

\resetxxt
\jie \begin{tblr}[t]{columns={colsep=0pt}}
    \xxt{}  & $\sqrt{(-1.5)^2} = |-1.5| = 1.5$;\\
    \xxt{}  & $\sqrt{(a - 3)^2} = |a - 3|$, \\
            & $\because \quad a < 3$, \\
            & $\therefore \quad a - 3 < 0$,\\
            & $\therefore \quad \sqrt{(a - 3)^2} = |a - 3| = -(a - 3) = 3 - a$。
\end{tblr}

\end{xiaoxiaotis}


\lianxi
\begin{xiaotis}

\xiaoti{(口答)下列等式能不能成立?为什么?}
\begin{xiaoxiaotis}

    \begin{tblr}{columns={18em, colsep=0pt}}
        \xxt{$(\sqrt{7})^2 = 7$;}  & \xxt{$(-\sqrt{7})^2 = -7$;} \\
        \xxt{$\sqrt{6^2} = 6$;}    & \xxt{$\sqrt{(-6)^2} = -6$。}
    \end{tblr}

\end{xiaoxiaotis}


\xiaoti{(口答)说出下列各式的值:}
\begin{xiaoxiaotis}

    \begin{tblr}{columns={18em, colsep=0pt}}
        \xxt{$(\sqrt{0.8})^2$;}    & \xxt{$\sqrt{0.8^2}$;} \\
        \xxt{$\sqrt{(-0.8)^2}$;}   & \xxt{$-\sqrt{(-0.8)^2}$。}
    \end{tblr}

\end{xiaoxiaotis}



\xiaoti{化简下列各式:}
\begin{xiaoxiaotis}

    \begin{tblr}{columns={18em, colsep=0pt}}
        \xxt{$\sqrt{(5 - 9)^2}$;}              & \xxt{$\sqrt{\left(3\dfrac{1}{2} - 2\right)^2}$;} \\
        \xxt{$\sqrt{(b - 4)^2} \quad (b > 4)$;}   & \xxt{$\sqrt{(m - n)^2} \quad (m < n)$。}
    \end{tblr}

\end{xiaoxiaotis}


\xiaoti{甲乙两人计算 $a + \sqrt{1 - 2a + a^2}$ 的值,当 $a = 5$ 的时候,得到不同的答案,
    甲的解答是 \\
    \hspace*{2em} $a + \sqrt{1 - 2a + a^2} = a + \sqrt{(1 - a)^2} = a + 1 - a = 1 \fenhao$ \\
    乙的解答是 \\
    \hspace*{2em} $\begin{aligned}
        a + \sqrt{1 - 2a + a^2} &= a + \sqrt{(a - 1)^2} = a + a - 1  \\
                                &= 2a - 1 = 2 \times 5 - 1 = 9 \juhao
    \end{aligned}$ \\
    哪一个答案是正确的?错误的解答,错在哪里?为什么?
}

\end{xiaotis}

\end{enhancedline}

