\xiaojie

一、本章主要内容是零指数、负整数指数、分数指数幂的概念和性质。


二、零指数, 负整数指数, 分数指数幂的意义分别
规定如下:

\hspace*{2em}\begin{tblr}{}
    $a^0 = 1 \quad (a \neq 0)$ \\
    $a^{-m} = \dfrac{1}{a^m} \quad (a \neq 0 \douhao m \text{是正整数})$; \\
    $a^{\frac{m}{n}} = \sqrt[n]{a^m} \quad (a \geqslant 0 \douhao m \douhao n \text{都是正整数,} n > 1)$; \\
    $a^{-\frac{m}{n}} = \dfrac{1}{a^{\frac{m}{n}}} = \dfrac{1}{\sqrt[n]{a^m}} \quad (a > 0 \douhao m \douhao n \text{都是正整数,} n > 1)$。
\end{tblr}

\begin{enhancedline}
这样,我们就把指数概念由正整数范围推广到了有理数范围。
关于正整数指数幂的运算性质,对于有理数指数幂也同样适用。又因为
\begin{gather*}
    \dfrac{a^m}{a^n} = a^m \cdot a^{-n} = a^{m + (-n)} \quad (a > 0) \douhao \\
    \left(\dfrac{a}{b}\right)^n = (a \cdot b^{-1})^n = a^n \cdot b^{-n} \quad (a > 0 \douhao  b > 0) \douhao
\end{gather*}
所以,对于有理数指数幂,运算性质 $\dfrac{a^m}{a^n} = a^{m - n}$ 与 $\left(\dfrac{a}{b}\right)^n = \dfrac{a^n}{b^n}$
可以分别被包含在 $a^m \cdot a^n = a^{m + n}$ 与 $(a \cdot b)^n = a^n \cdot b^n$ 之中。
正整数指数幂的五个运算性质,对于有理数指数幂来说,可以合并为下面三个。
\end{enhancedline}

\hspace*{2em}\begin{tblr}{}
    $a^m \cdot a^n = a^{m + n} \quad (a > 0 \douhao m \douhao n \text{为有理数})$; \\
    $(a^m)^n = a^{mn} \quad (a > 0 \douhao m \douhao n \text{为有理数})$; \\
    $(ab)^n = a^n b^n \quad (a > 0 \douhao b > 0 \douhao n \text{为有理数})$。
\end{tblr}

我们还可以规定无理数指数幂的意义(本书略去不讲),把指数概念推广到实数范围。
关于有理数指数幂的运算性质,对于实数指数幂也同样适用。



三、所谓科学记数法,就是把一个数写成绝对值在 $1$ 与 $10$ 之间(可以是 $1$)的数与 $10$ 的整数次幂的积
(即写成 $\pm a \times 10^n$,其中 $n$ 为整数, $a$ 大于或等于 $1$ 而小于 $10$)这样一种方法。


四、分数指数与根式紧密相关。根式有下列性质:

\hspace*{2em}\begin{tblr}{}
    $(\sqrt[n]{a})^n = a$; \\
    当 $n$ 为奇数时,$\sqrt[n]{a^n} = a$; \\
    当 $n$ 为偶数时,$\sqrt[n]{a^n} = |a| = \begin{cases}
        \phantom{-}a \quad (a \geqslant 0) \douhao \\
        -a           \quad (a < 0) \fenhao
    \end{cases}$ \\
    $\sqrt[np]{a^{mp}} = \sqrt[n]{a^m} \quad (a \geqslant 0)$ (根式的基本性质); \\
    $\sqrt[n]{ab} = \sqrt[n]{a} \cdot \sqrt[n]{b} \quad (a \geqslant 0 \douhao b \geqslant 0)$; \\
    $\sqrt[\uproot{6}n]{\dfrac{a}{b}} = \dfrac{\sqrt[n]{a}}{\sqrt[n]{b}} \quad (a \geqslant 0 \douhao b > 0)$; \\
    $(\sqrt[n]{a})^m = \sqrt[n]{a^m} \quad (a \geqslant 0)$; \\
    $\sqrt[m]{\sqrt[n]{a}} = \sqrt[mn]{a} \quad (a \geqslant 0)$。
\end{tblr}

根式的加减,就是合并同类根式的过程。根式的乘、除、乘方、开方,一般利用分数指数来进行比较简便。

