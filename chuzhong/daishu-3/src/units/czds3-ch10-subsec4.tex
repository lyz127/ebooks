\subsection{二次根式的加减}\label{subsec:10-4}
\begin{enhancedline}

二次根式的加减同整式的加减类似,只要合并同类二次根式就可以了。
为了合并同类二次根式,应当先把各个二次根式化成最简二次根式。
也就是说,二次根式相加减,先把各个二次根式化成最简二次根式,再把同类二次根式分别合并。

\liti 计算 $2\sqrt{12} - 4\sqrt{\dfrac{1}{27}} + 3\sqrt{48}$。

\jie $\begin{aligned}[t]
    &2\sqrt{12} - 4\sqrt{\dfrac{1}{27}} + 3\sqrt{48} \\
    &= 4\sqrt{3} - \dfrac{4}{9}\sqrt{3} + 12\sqrt{3} \\
    &= \left(4 - \dfrac{4}{9} + 12\right)\sqrt{3} \\
    &= \dfrac{140}{9} \sqrt{3} \juhao
\end{aligned}$


\liti 计算 $\dfrac{2}{3}\sqrt{9x} + 6\sqrt{\dfrac{x}{4}} - 2x\sqrt{\dfrac{1}{x}}$。

\jie $\begin{aligned}[t]
    &\dfrac{2}{3}\sqrt{9x} + 6\sqrt{\dfrac{x}{4}} - 2x\sqrt{\dfrac{1}{x}} \\
    &= 2\sqrt{x} + 3\sqrt{x} - 2\sqrt{x} \\
    &= 3\sqrt{x} \juhao
\end{aligned}$

\liti 计算 $\left(\sqrt{32} + \sqrt{0.5} - 2\sqrt{\dfrac{1}{3}}\right) - \left(\sqrt{\dfrac{1}{8}} - \sqrt{75}\right)$。

\jie $\begin{aligned}[t]
    &\left(\sqrt{32} + \sqrt{0.5} - 2\sqrt{\dfrac{1}{3}}\right) - \left(\sqrt{\dfrac{1}{8}} - \sqrt{75}\right) \\
    &= \sqrt{32} + \sqrt{0.5} - 2\sqrt{\dfrac{1}{3}} - \sqrt{\dfrac{1}{8}} + \sqrt{75} \\
    &= 4\sqrt{2} + \dfrac{1}{2}\sqrt{2} - \dfrac{2}{3}\sqrt{3} - \dfrac{1}{4}\sqrt{2} + 5\sqrt{3} \\
    &= \left(4 + \dfrac{1}{2} - \dfrac{1}{4}\right)\sqrt{2} + \left(5 - \dfrac{2}{3}\right)\sqrt{3} \\
    &= \dfrac{17}{4}\sqrt{2} + \dfrac{13}{3}\sqrt{3} \juhao
\end{aligned}$


\lianxi
\begin{xiaotis}

\xiaoti{计算:}
\begin{xiaoxiaotis}

    \begin{tblr}{columns={colsep=0pt}, rows={rowsep=0.5em}}
        \xxt{$5\sqrt{2} + \sqrt{8} - 7\sqrt{18}$;} & \xxt{$\sqrt{28} + 9\sqrt{112}$;} \\
        \xxt{$3\sqrt{40} - \sqrt{\dfrac{2}{5}} - 2\sqrt{\dfrac{1}{10}}$;} & \xxt{$\sqrt{12} + \sqrt{\dfrac{1}{27}} - \sqrt{\dfrac{1}{3}}$;} \\
        \xxt{$\dfrac{1}{3}\sqrt{32} + \dfrac{\sqrt{8}}{2} - \dfrac{1}{5}\sqrt{50}$;} & \xxt{$\sqrt{2x} - \sqrt{8x^3} + 2\sqrt{2xy^2}$;} \\
        \xxt{$x\sqrt{\dfrac{1}{x}} + \sqrt{4y} - \dfrac{\sqrt{x}}{2} + y\sqrt{\dfrac{1}{y}}$;} & \xxt{$\dfrac{-b + \sqrt{b^2 - 4ac}}{2a} + \dfrac{-b - \sqrt{b^2 - 4ac}}{2a} \; (b^2 > 4ac)$。} \\
    \end{tblr}

\end{xiaoxiaotis}


\xiaoti{计算:}
\begin{xiaoxiaotis}

    \xxt{$\sqrt{18} - (\sqrt{98} - 2\sqrt{75} + \sqrt{27})$;}

    \xxt{$(\sqrt{45} + \sqrt{108}) + \left(\sqrt{1\dfrac{1}{3}} - \sqrt{125}\right)$;}

    \xxt{$\left(\sqrt{24} - \sqrt{0.5} - 2\sqrt{\dfrac{2}{3}}\right) - \left(\sqrt{\dfrac{1}{8}} - \sqrt{6}\right)$;}

    \xxt{$\dfrac{1}{2}(\sqrt{2} + \sqrt{3}) - \dfrac{3}{4}(\sqrt{2} - \sqrt{27})$;}

    \xxt{$a^2\sqrt{8a} + 3a\sqrt{50a^3} - \dfrac{a}{2}\sqrt{18a^3}$;}

    \xxt{$\left(4b\sqrt{\dfrac{a}{b}} + \dfrac{2}{a}\sqrt{a^3b}\right) - \left(3a\sqrt{\dfrac{b}{a}} + \sqrt{9ab}\right)$。}

\end{xiaoxiaotis}


\xiaoti{求下列各式的近似值(精确到 $0.01$)}:
\begin{xiaoxiaotis}

    \begin{tblr}{columns={colsep=0pt}, column{1}={18em}, rows={rowsep=0.5em}}
        \xxt{$\sqrt{2\dfrac{2}{3}} + \sqrt{\dfrac{2}{3}} - \dfrac{1}{5}\sqrt{54}$;} &
            \xxt{$\left(5\sqrt{\dfrac{1}{5}} + \dfrac{1}{2}\sqrt{20}\right) - \left(\dfrac{5}{4}\sqrt{\dfrac{4}{5}} - \sqrt{45}\right)$。}
    \end{tblr}

\end{xiaoxiaotis}


\xiaoti{下列计算是否正确?为什么?}
\begin{xiaoxiaotis}

    \begin{tblr}{columns={colsep=0pt}, column{1}={18em}, rows={rowsep=0.5em}}
        \xxt{$\sqrt{2} + \sqrt{3} = \sqrt{5}$;} & \xxt{$2 + \sqrt{2} = 2\sqrt{2}$;} \\
        \xxt{$a\sqrt{x} - b\sqrt{x} = (a - b)\sqrt{x}$;} & \xxt{$\dfrac{\sqrt{8} + \sqrt{18}}{2} = \sqrt{4} + \sqrt{9} = 2 + 3 = 5$。}
    \end{tblr}

\end{xiaoxiaotis}
\end{xiaotis}
\end{enhancedline}

