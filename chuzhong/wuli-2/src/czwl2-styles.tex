% 定义本书中使用到的格式

\ctexset {
  chapter = {
    pagestyle = headings,
  },
  section = {
     aftername={、},
  }
}

\usepackage{array}
\usepackage{amsmath}
\allowdisplaybreaks[1]
\usepackage{amssymb}
\usepackage{calc}
\usepackage{caption}
\captionsetup[figure]{labelsep=none}

\usepackage{CJKfntef}
\usepackage{enumitem}
\usepackage{etoolbox}
\usepackage{float}
\usepackage[stable]{footmisc}
\usepackage{geometry}
\geometry{a4paper,left=2cm,right=2cm,top=2cm,bottom=1cm}

\usepackage{hyperref}
\hypersetup{colorlinks=true, linkcolor=red}

\usepackage{tabularray}
\UseTblrLibrary{siunitx}

\usepackage{tikz}
\usetikzlibrary{
  arrows.meta,
  calc,
  math,
}

\usepackage{wrapfig}
\usepackage{xparse}
\usepackage{xpinyin}

% 确保 有行内公式 的行,与其上、下行之间,有足够的行距
% 参数1: 最后一行需要额外增加的高度(用于调整和后继行的间隔)
\NewDocumentEnvironment{enhancedline}{o}{
  \setlength{\lineskip}{\baselineskip-\ccwd}
  \setlength{\lineskiplimit}{2.5pt}
}{
  \IfValueT{#1}{
    \vspace{#1}
  }
  \par
}

% 绘制带圈的数字
\newcommand{\circled}[2][]{\tikz[baseline=(char.base)]
    {\node[shape = circle, draw, inner sep = 1pt]
    (char) {\phantom{\ifblank{#1}{#2}{#1}}};%
    \node at (char.center) {\makebox[0pt][c]{#2}};}}
\robustify{\circled}
\newcommand{\tc}[1]{\text{\circled{#1}}}

\counterwithout*{subsection}{section}
\counterwithin*{subsection}{chapter}
\setcounter{secnumdepth}{3}
\renewcommand{\thesection}{\chinese{section}}
\renewcommand{\thesubsection}{\arabic{chapter}.\arabic{subsection}}
\renewcommand{\thesubsubsection}{\arabic{subsubsection}.}

\renewcommand{\thefigure}{\thechapter-\arabic{figure}\;}
\renewcommand{\thefootnote}{\circled{\arabic{footnote}}}

% 在不改变 counter 值的情况下,重置 counter 的子 counter
\newcommand{\touchcounter}[1]{\stepcounter{#1} \addtocounter{#1}{-1}}

\newenvironment{starred}{
  \ctexset {
    chapter/name={*第,章},
    section/name={*}
  }
}{}


% 没有编号但又需要添加到目录中的 section (No num(ber) section)。
% 参数1: 目录中的文字(可选,如果没有指定,则使用参数2的值)
% 参数2: 正文中的文字
\NewDocumentCommand{\nonumsection}{o m}{%
  \touchcounter{section}
  \section*{#2}
  \IfNoValueTF{#1}
    {\addcontentsline{toc}{section}{#2}}
    {\addcontentsline{toc}{section}{#1}}
}

% 没有编号但又需要添加到目录中的 subsection (No num(ber) subsection)。
% 参数1: 目录中的文字(可选,如果没有指定,则使用参数2的值)
% 参数2: 正文中的文字
\NewDocumentCommand{\nonumsubsection}{o m}{%
  \touchcounter{subsection}
  \subsection*{#2}
  \IfNoValueTF{#1}
    {\addcontentsline{toc}{subsection}{#2}}
    {\addcontentsline{toc}{subsection}{#1}}
}


% (编号前)带星号的 subsection
% 参数1: subsection 的标题
\NewDocumentCommand{\starredsubsection}{o m}{
  \refstepcounter{subsection}
  \subsection*{*\thesubsection\quad #2}
  \IfNoValueTF{#1}
    {\addcontentsline{toc}{subsection}{\makebox{*}\thesubsection\qquad #2}}
    {\addcontentsline{toc}{subsection}{\makebox{*}\thesubsection\qquad #1}}
}

\newcounter{cntliti}[section]           % 例题的计数器
\newcommand{\liti}{ % 例题的标题
  \stepcounter{cntliti}
  \textbf{[例题 \thecntliti]}
}

\newcommand{\jie}{\textbf{解: }}
\newcommand{\zhengming}{\textbf{证明: }}

\newcounter{cntlianxi}[chapter]               % “练习”的计数器
\newcommand{\lianxi}{%
  \stepcounter{cntlianxi}
  \vspace{1em}
  {\centering \section*{\labelxiti}}
}
\newcommand{\labelxiti}{练 \; 习 \; \chinese{cntlianxi}}

\newcommand{\jiange}{\vspace{0.5em}} % 手工调整垂直间隔(测试发现 0.5em 比较合适。不支持参数,特殊情况直接使用 \vspace{} 命令。)

% 修改数学公式与上下文的距离
\makeatletter
\renewcommand\normalsize{%
    \abovedisplayskip 1\p@ \@plus1\p@ \@minus6\p@
    \belowdisplayskip \abovedisplayskip
}
\makeatother


%----------------------------------
\newcounter{mylabel}
% 针对计数器 mylabel 创建(指定名字)标签
% 参数1: 标签的id
% 参数2: (可选) 标签的名字 (\nameref{标签id} 所得到的结果)。默认为标签归属章节的名字
\NewDocumentCommand{\mylabel}{m o}{%
  \refstepcounter{mylabel}%
  \IfValueTF{#2}{%
    \namedlabel{#1}{#2}%
  }{%
    \label{#1}%
  }%
}


%----------------------------------
% 实现类似如下效果:
%  name1 .............. value1
%  another-name2 .......... v2
\makeatletter
\newcommand \cdotfill {\leavevmode \cleaders \hb@xt@ .33em{\hss $\cdot$ \hss }\hfill \kern \z@}
\makeatother

\newenvironment{dottedlist}[2]{%
    \newcommand{\pair}[2]{\item ##1 \cdotfill ##2}
    \begin{itemize}[nosep, leftmargin=#1, rightmargin=#2]
        \renewcommand{\labelitemi}{}
}{%
    \end{itemize}
}


%---------------------------
\newcommand{\shiyan}[1]{\textbf{【#1】}}
\newcommand{\xiaobiaoti}[1]{\textbf{#1} \hspace{1em}}
