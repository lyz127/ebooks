\section{决定电阻大小的因素}\label{sec:8-9}

从前面的学习中我们知道,不同导体的电阻一般是不同的。可见电阻是导体本身的一种性质。
现在我们来进一步研究,导体电阻的大小跟它本身的哪些因素有关系呢?

在图 \ref{fig:8-16} 所示的实验中,我们将 $AB$、$CD$ 换成横截面积相同而长度不同的两条镍铬合金线做实验。
当加上相同的电压时,较长的一条电流强度小。
可见,导体的电阻跟它的长度有关系,导体越长,电阻越大。

再换用长度相同而横截面积不同的两条镍铬合金线做实验。
当加上相同的电压时,横截面积较小的一条电流强度小。
可见,导体的电阻跟它的横截面积有关系,导体的横截面积越小,电阻越大。

再换用长度和横截面积都相同而材料不同的导线,例如一条镍铬合金线和一条锰铜线,来做实验。
当加上相同的电压时,通过的电流强度不同。
可见,导体的电阻还跟导体的材料有关系。
下面列出了一些材料制成的长 1 米、横截面积 1 $\pfhm$ 的导线在 20 ℃ 的电阻:
\footnote{某种材料制成的长度为 1 米、横截面积 是1 $\pfhm$ 的导线的电阻,叫做这种材料的电阻率。}

\begin{dottedlist}{5em}{5em}
    \pair{银}{0.016 欧姆}
    \pair{铜}{0.017 欧姆}
    \pair{铝}{0.029 欧姆}
    \pair{钨}{0.053 欧姆}
    \pair{铁}{0.10 欧姆}
    \pair{锰铜(铜、锰、镍的合金)}{0.44 欧姆}
    \pair{镍铬合金(镍、铬、铁、锰的合金)}{1.0 欧姆}
\end{dottedlist}

综合上面的实验研究知道:
\textbf{导体的电阻是导体本身的一种性质,它的大小决定于导体的长度、横截面积和材料}。



\section*{阅读材料:导体的电阻总是不变吗?}

你注意过吗?电灯泡的灯丝,很少有正在发光时突然烧断,通常是在开灯的瞬间,
灯丝被烧断,电灯不亮了。这是什么原因呢?

原来,导体的电阻还随温度改变而改变。金属的电阻都随温度升高而增大。
一般金属导体温度变化几度或十几度,电阻值变化不过百分之几,可以忽略不计。
但是电灯的灯丝(钨丝),不发光时温度不过几十度,发光时却能达到两千多度,电阻值就要增大许多倍。
因而,在刚刚闭合电键,灯丝温度还没有升高的瞬间,灯丝中的电流强度将比发光时大得多,
就是由于这个缘故,通常灯丝才在这一瞬间烧断。

利用金属的电阻随温度而改变的现象,可以制成电阻温度计。电阻温度计常用铂丝制成。
预先测出铂丝在不同温度时的电阻值,测温时再把铂丝插在被测物体里,
测出铂丝的电值也就知道了待测的温度。铂电阻温度计的测量范围大约是 $-263$ ℃ ~ 1000 ℃。

1911 年,荷兰物理学家昂尼斯(1853 ~ 1926)测定水银在低温下的电阻值时发现,
当温度降到 $-269$ ℃ 左右时水银的电阻突然消失。
以后又发现还有一些金属、合金当温度降到某一温度(所谓转变温度)时,电阻也突然消失。
人们把这种现象叫做超导性,处于超导状态的物体叫超导体。

目前的输电线路,由于输电导线有电阻而大约损失输送能量的四分之一,假如能用超导体输电,这种损失就可以避免。
可惜,到现在为止,人们发现具有超导性的金属、合金的转变温度都很低,
例如铝是 $-271.96$ ℃, 铅是 $-265.97$ ℃,铌是 $-264.01$ ℃。
要使输电导线保持这徉低的温度是很困难的。因此,人们正在努力寻找转变温度高的超导材料,
同时也在努力把已经取得的关于超导体的研究成果应用于科学技术,这些方面都是大有可为的。

