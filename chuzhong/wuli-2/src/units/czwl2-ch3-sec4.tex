\section{热量的计算}\label{sec:3-4}

知道了物质的比热,可以利用它计算这种物质做成的物体在温度改变时吸收或者放出的热量。

\liti 把质量为 500 克、温度为 30 ℃ 的铝加热到 100 ℃,铝块吸收了多少热量?

解:从比热表中查到铝的比热是 0.21 卡/(克·℃)。
就是说,1 克铝温度升高 1 ℃ 吸收的热量是 0.21 卡。

500 克铝温度升高 1 ℃ 时,吸收的热量是
$$ 0.21 \ka \times 500 = 105 \ka \;\juhao $$

500 克铝温度升高 $100\celsius - 30\celsius = 70\celsius$ 时,吸收的热量是
$$ 105 \ka \times 70 = 7350 \ka = 7.35 \qianka \;\juhao $$

答:铝块吸收的热量是 7.35 千卡。

上面的计算可以写成一个算式:
\begin{align*}
    \text{吸收的热量} &= 0.21 \kmkssd \times 500 \ke \times (100 \celsius - 30 \celsius) \\
        &= 7350 \ka = 7.35 \qianka
\end{align*}


如果用 $c$ 表示比热,$m$ 表示物体的质量,$t_0$ 表示物体原来的温度,
$t$ 表示物体后来的温度,那么,物体温度升高时吸收的热量
$$ Q_\text{吸} = cm(t - t_0) \;\juhao $$

通过同样的分析,物体温度降低时放出的热量
$$ Q_\text{放} = cm(t_0 - t) \;\juhao $$

\liti 把 100 克、80 ℃ 的热水跟 300 克、20 ℃ 室温的水相混合。
经测量,混合后的共同温度为 35 ℃。
试计算热水温度降低时放出的热量和室温的水温度升高时吸收的热量。
把这两个热量加以比较。

用 $m_1$、$t_{0\,1}$ 分别表示热水的质量和原来的温度,
用 $m_2$、$t_{0\,2}$ 分别表示室温水的质量和原来的温度,
用 $c$ 表示水的比热,用 $t$ 表示混合后的共同温度。

已知:
$m_1 = 100\ke$,  $t_{0\,1} = 80\celsius$,
$m_2 = 300\ke$,  $t_{0\,2} = 20\celsius$,
$c = 1 \kmkssd$, $t = 35\celsius$。

求:热水放出的热量 $Q_\text{放}$ 和室温的水吸收的热量 $Q_\text{吸}$。

解:$\begin{aligned}[t]
    Q_\text{放} &= cm_1(t_{0\,1} - t) \\
        &= 1 \kmkssd \times 100 \ke \times (80 \celsius - 35 \celsius) \\
        &= 4500 \ka \;\juhao \\
     Q_\text{吸} &= cm_2(t - t_{0\,2}) \\
        &= 1 \kmkssd \times 300 \ke \times (35 \celsius - 20 \celsius) \\
        &= 4500 \ka \;\juhao
\end{aligned}$

答:热水温度降低时放出的热量跟室温的水温度升高时吸收的热量相等,都是 4500 卡。

我们知道,温度不同的两个物体互相接触时,热就要从高温物体传递到低温物体,
并且一直继续到两个物体的温度相同时为止。例题 2 的计算结果表明,在热传递过程中,
低温物体吸收的热量 $Q_\text{吸}$ 等于高温物体放出的热量 $Q_\text{放}$, 即
$$ Q_\text{吸} = Q_\text{放} \;\juhao $$

实际上,如果利用比较精确的测量数据计算冷热水混合时吸收和放出的热量时,
你会发现,在热传递过程中,吸收和放出的热量并不相等。
这是因为容器要吸收一些热量,另外还传递给周围空气一些热量。
如果我们把容器吸收的热量也计算在内,并且尽可能地减少热量的损失,
那么高温物体放出的热量和低温物体吸收的热量就会相差很少了。


\lianxi

(1) 砂石的比热是 0.22 卡/(克·℃),它表示的是什么意思?

(2) 质量相等的铁块和铝块,吸收了相等的热量,升高的温度是否相同?为什么?
如果要使它们升高相同的温度,哪个吸收的热量多?

(3) 有人说:“热水和冷水混合的时候,热水降低的温度等于冷水升高的温度。” 这句话对吗? 为什么?

(4) 有一根烧红的铁钉,温度是 800 ℃ ,质量是 1 克。
另有一壶水,温度是 80 ℃ ,质量是 1 千克,
如果铁钉和水的温度都降到 20 ℃,哪一个放出的热量多?

(5) 1 克铝和 1 克铅,它们都吸收了 6.2 卡的热量,铝和铅升高的温度各是多少?

(6) 在质量是 500 克的铝壶里装了 2 千克水,把这壶水从 15 ℃ 加热到 100 ℃,铝壶和水总共吸收了多少热量?

(7) 在第 6 题中,如果忽略铝壶吸收的热量,计算得出来的答数跟第 6 题的答数相比较差百分之几?
为什么近似计算时可以不考虑铝壶吸收的热量?

