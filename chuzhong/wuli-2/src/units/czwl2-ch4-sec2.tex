\section{实验:研究萘的熔解过程}\label{sec:4-2}

这个实验是要研究萘受热熔解时,它的温度变化情况。

把萘粉装进试管,试管里插入温度计和搅动器,温度计的玻璃泡要插进萘粉中。
把试管放到盛有热水的烧杯里,用酒精灯给烧杯缓缓加热,同时用搅动器搅动萘粉,
使萘的各部分温度均匀。实验装置跟图 \ref{fig:4-1} 一样。

观察温度计的示数,等萘的温度升到 50 ℃ 左右,开始每隔一分钟或两分钟记录一次萘的温度。
直到萘的温度大约升到 85 ℃ 为止。把实验数据记录在表格里。

\jiange
\begin{tblr}{
    hlines,
    colspec={|c|*{7}{c}c|},
    column{2-8} = {wd=2em},
}
    时间(分) & 0, & 1, & 2, & 3, & 4, & 5, & 6, & …… \\
    温度(℃) & & & & & & & & \\
\end{tblr}
\jiange

以横坐标表示时间,以纵坐标表示温度,在方格纸上标出一些点,用来表示实验得到的萘在各个时刻的温度。
用平滑的曲线把这些点连接起来,就得到了萘的熔解图象。从你画的图象中,得出萘的熔点是多少度?

如果有时间的话,还可以做萘的凝固图象。看看你得出的萘的凝固点跟它的熔点是否相同?

