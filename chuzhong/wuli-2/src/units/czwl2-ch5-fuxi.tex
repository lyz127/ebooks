\section*{复习题}

(1) 什么现象表明物体的分子在不停地做无规则的运动?
为什么把物体里大量分子的这种无规则运动叫做热运动?

(2) 物体里分子间的作用,在什么情况下表现出斥力?
在什么情况下表现出引力?在什么情况下可以认为没有作用力?

(3) 气体、液体和固体三者之间的主要差别是什么?怎样用物质的分子结构来说明这种差别?
在气体、液体和固体这三种状态中,物质究竟以哪种状态存在是由什么来决定的?

(4) 什么叫做热能?改变物体的热能有哪两种方法?

(5) 热功当量表示的是什么关系?写出焦耳和卡这两种单位的换算关系。

(6) 什么是能的转化和守恒定律?

