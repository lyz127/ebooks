\section*{复习题}

(1) 什么现象可以说明光在同一种物质里传播的路线是直的?光在真空里的传播速度是多少

(2) 叙述光的反射定律。

(3) 平面镜里成的像跟物体是怎样的关系?

(4) 凹镜有什么重要的光学性质和应用?

(5) 当光从一种物质进入另一种物质的时候,折射光线跟入射光线有什么关系?

(6) 什么是凸透镜的主轴、焦点和焦距?

(7) 什么情况下凸透镜成实像?实像的位置和大小跟物体的所在位置有什么关系?

(8) 什么情况下凸透镜成虚像?虚像跟实像的区别是什么?

(9) 照相机、幻灯机、放大镜分别利用了凸透镜的哪种成像情况?

(10) 什么叫光的色散?

(11) 透明体的颜色是由什么决定的?不透明体呢?

