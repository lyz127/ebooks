\section{热能}\label{sec:5-3}

一切物体的分子都在不停地做无规则的运动。分子做无规则运动的快慢跟温度有关系。
把红墨水分别滴到热水和冷水中,可以看到,热水变色比冷水变色快。
在两种气体扩散的实验中,温度越高,两种气体扩散得就越快。
这些现象表明:温度越高,分子做无规则运动的速度就越大。
因为大量分子做无规则运动的速度跟温度有关,所以我们把物体里大量分子的无规则运动叫做\textbf{热运动}。

我们知道,在机械运动中,任何运动的物体都具有能,这种能叫做机械能。
在热运动中,任何运动的分子也具有能,物体中大量的做无规则运动的分子具有的能叫做\textbf{热能}。
机械能是能的一种形式,热能也是能的一种形式。

一个物体的温度升高,它的分子的运动加快,它的热能就增加;
    物体的温度降低,它的分子的运动变慢,它的热能就减少。


\section*{小实验}

倒一杯凉开水,把一块糖放到杯底。
然后用小勺取表面附近的水品尝,看看经过多长时间才能尝到有甜味。
取水时要小心,不要把水搅动。

再换用温开水来做这个实验,看看尝到表面附近的水有甜味经过的时间,
跟凉开水相比,是长了还是短了。

试解释你的实验结果。

