\section{实验:测定物质的比热}\label{sec:3-6}

这个实验的目的是学习用量热器测定物质的比热。

用天平称出铜块(或其他金属块)的质量,把测量的结果记在下面的表里
(以后各个步骤的测量结果,也要随时记在表里)。
用线拴好铜块,把它放进装有开水的烧杯里,用酒精灯加热。
在给铜块加热的同时,可以继续进行下面的实验。

用量筒量一些水倒进量热器的小筒里(水的多少,要能浸没铜块),算出水的质量。
把装着水的量热器的小筒放在大筒里的木架上,注意不要使两个筒接触。

用温度计测出量热器小筒里的水的温度。
估计铜块加热已超过 10 分钟,即可用温度计测烧杯里的水的温度,这个温度就是铜块的温度。

从烧杯里取出铜块,立刻把它投进量热器小筒的水里,随后就把木盖盖好。
用搅动器上下搅动小筒里的水,注意插在水里的温度计的示数,记下最高温度,这就是混合后的共同温度。

\jiange
\begin{tblr}{
    hlines, vlines,
    colspec={ccccc},
    cells={valign=m},
}
    {铜块的质量\\(克)} & {水的质量\\(克)} & {混合前水的\\温度(℃)} & {混合前铜块\\的温度(℃)} & {混合后的共\\同温度(℃)} \\
    & & & &
\end{tblr}
\jiange

利用测得的数值,算出铜的比热。你得到的铜的比热的数值,跟课本里给的相比,是大还是小?
试分析产生误差的原因。想想看,怎样做实验,结果可以更准确些?



\lianxi

(1) 比热和密度都是物质的特性,因此比热和密度都可以用来鉴别物质。
但是实际鉴别物质的时候常用密度,而很少用比热。这是为什么?

(2) 把 200 克铅加热到 98 ℃,然后投进温度为 12 ℃ 的 80 克水里,混合后的温度是 18 ℃。求铅的比热。

(3) 取 100 克煤油,测得它的温度是 20 ℃。把 80 克的铁块加热到 100 ℃ 后投进煤油里。
测出混合的温度是 31.8 ℃。铁的比热已知是 0.11 卡/(克·℃),求煤油的比热。

