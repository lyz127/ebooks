\section*{复习题}

(1) 举出气体、液体、固体热胀冷缩的实例。气体、液体、固体的热膨胀有什么不同?

(2) 举例说明怎样预防热膨胀所带来的危害?怎样利用热膨胀来做有益的事情?

(3) 采用摄氏度为单位的温度计是怎样确定刻度的?

(4) 使用液体温度计要注意些什么?为什么?

(5) 什么叫做热传递?热传递有哪几种方式?这些热传递的方式各是怎样进行的?

(6) 举例说明怎样尽可能利用或防止热传递?

