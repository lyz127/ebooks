% 定义本书中使用到的格式

\ctexset {
  chapter = {
    pagestyle = headings,
  },
  section = {
    aftername={、},
  }
}

\usepackage{array}
\usepackage{amsmath}
\allowdisplaybreaks[1]
\usepackage{amssymb}
\usepackage{calc}
\usepackage{caption}
\captionsetup[figure]{labelsep=none}

\usepackage{CJKfntef}
\usepackage{etoolbox}
\usepackage{float}
\usepackage[stable]{footmisc}
\usepackage{geometry}
\geometry{a4paper,left=2cm,right=2cm,top=2cm,bottom=1cm}

\usepackage{hyperref}
\hypersetup{colorlinks=true, linkcolor=red}

\usepackage{makecell}
\usepackage{multirow}
\usepackage{tabularray}

\usepackage{tikz}

\usepackage{wrapfig}
\usepackage{xparse}
\usepackage{xpinyin}

% 确保 有行内公式 的行,与其上、下行之间,有足够的行距
\setlength{\lineskip}{\baselineskip-\ccwd}
\setlength{\lineskiplimit}{2.5pt}

% 绘制带圈的数字
\newcommand{\circled}[2][]{\tikz[baseline=(char.base)]
    {\node[shape = circle, draw, inner sep = 1pt]
    (char) {\phantom{\ifblank{#1}{#2}{#1}}};%
    \node at (char.center) {\makebox[0pt][c]{#2}};}}
\robustify{\circled}
\newcommand{\tc}[1]{\text{\circled{#1}}}

\counterwithout*{subsection}{section}
\counterwithin*{subsection}{chapter}
\setcounter{secnumdepth}{3}
\renewcommand{\thesection}{\chinese{section}}
\renewcommand{\thesubsection}{\arabic{chapter}.\arabic{subsection}}
\renewcommand{\thesubsubsection}{\arabic{subsubsection}.}

\renewcommand{\thefigure}{\thechapter-\arabic{figure}\;}
\renewcommand{\thefootnote}{\circled{\arabic{footnote}}}

% 在不改变 counter 值的情况下,重置 counter 的子 counter
\newcommand{\touchcounter}[1]{\stepcounter{#1} \addtocounter{#1}{-1}}

\newenvironment{starred}{
  \ctexset {
    chapter/name={*第,章},
    section/name={*}
  }
}{}


% 没有编号但又需要添加到目录中的 section (No num(ber) section)。
% 参数1: 目录中的文字(可选,如果没有指定,则使用参数2的值)
% 参数2: 正文中的文字
\NewDocumentCommand{\nonumsection}{o m}{%
  \touchcounter{section}
  \section*{#2}
  \IfNoValueTF{#1}
    {\addcontentsline{toc}{section}{#2}}
    {\addcontentsline{toc}{section}{#1}}
}

% 没有编号但又需要添加到目录中的 subsection (No num(ber) subsection)。
% 参数1: 目录中的文字(可选,如果没有指定,则使用参数2的值)
% 参数2: 正文中的文字
\NewDocumentCommand{\nonumsubsection}{o m}{%
  \touchcounter{subsection}
  \subsection*{#2}
  \IfNoValueTF{#1}
    {\addcontentsline{toc}{subsection}{#2}}
    {\addcontentsline{toc}{subsection}{#1}}
}


% (编号前)带星号的 subsection
% 参数1: subsection 的标题
\NewDocumentCommand{\starredsubsection}{o m}{
  \refstepcounter{subsection}
  \subsection*{*\thesubsection\quad #2}
  \IfNoValueTF{#1}
    {\addcontentsline{toc}{subsection}{\makebox{*}\thesubsection\qquad #2}}
    {\addcontentsline{toc}{subsection}{\makebox{*}\thesubsection\qquad #1}}
}

\newcounter{cntliti}[subsubsection]           % 例题的计数器
\newcommand{\liti}{ % 例题的标题
  \stepcounter{cntliti}
  \textbf{[例题 \thecntliti]}
}

\newcommand{\jie}{\textbf{解: }}
\newcommand{\zhengming}{\textbf{证明: }}

\newcounter{cntlianxi}[chapter]               % “练习”的计数器
\newcommand{\lianxi}{%
  \stepcounter{cntlianxi}
  \vspace{1em}
  {\centering \nonumsubsection{\labelxiti}}
}
\newcommand{\labelxiti}{练 \; 习 \; \chinese{cntlianxi}}

\newcommand{\jiange}{\vspace{0.5em}} % 手工调整垂直间隔(测试发现 0.5em 比较合适。不支持参数,特殊情况直接使用 \vspace{} 命令。)

% 修改数学公式与上下文的距离
\makeatletter
\renewcommand\normalsize{%
    \abovedisplayskip 1\p@ \@plus1\p@ \@minus6\p@
    \belowdisplayskip \abovedisplayskip
}
\makeatother


%----------------------------------
% (重)定义一些数学符号
%  一是为了使用/记忆上的方便,二是如果以后有变动,只需要修改一处。
\newcommand{\celsius}{\ensuremath{^\circ\hspace{-0.09em}\mathrm{C}}} % 摄氏度


%----------------------------------
\newcounter{mylabel}
% 针对计数器 mylabel 创建(指定名字)标签
% 参数1: 标签的id
% 参数2: (可选) 标签的名字 (\nameref{标签id} 所得到的结果)。默认为标签归属章节的名字
\NewDocumentCommand{\mylabel}{m o}{%
  \refstepcounter{mylabel}%
  \IfValueTF{#2}{%
    \namedlabel{#1}{#2}%
  }{%
    \label{#1}%
  }%
}

%---------------------------
\newcommand{\shiyan}[1]{\textbf{【#1】}}

% 实现单元格内文字换行的命令
% 参数1:对齐方式
% 参数2: 单元格的内容。使用 \\ 进行换行。
\newcommand{\tabincell}[2]{\begin{tabular}{@{}#1@{}}#2\end{tabular}}

%特殊符号
\newcommand{\juhao}{\mathord{\text{。}}}%(中文的)句号

%------------------ 物理单位
% 长度单位
\newcommand{\qianmi}{\mathord{\text{千米}}}%千米
\newcommand{\mi}{\mathord{\text{米}}}%米
\newcommand{\fenmi}{\mathord{\text{分米}}}%分米
\newcommand{\limi}{\mathord{\text{厘米}}}%厘米
\newcommand{\haomi}{\mathord{\text{毫米}}}%毫米

% 面积单位
\newcommand{\pfm}{\mathord{\text{米}^2}}%平方米
\newcommand{\pflm}{\mathord{\text{厘米}^2}}%平方厘米
\newcommand{\pfhm}{\mathord{\text{毫米}^2}}%平方毫米

% 体积单位
\newcommand{\lfm}{\mathord{\text{米}^3}}%立方米
\newcommand{\lffm}{\mathord{\text{分米}^3}}%立方分米
\newcommand{\lflm}{\mathord{\text{厘米}^3}}%立方厘米
\newcommand{\lfhm}{\mathord{\text{毫米}^3}}%立方毫米

% 质量单位
\newcommand{\haoke}{\mathord{\text{毫克}}}%毫克
\newcommand{\ke}{\mathord{\text{克}}}%克
\newcommand{\qianke}{\mathord{\text{千克}}}%千克

\newcommand{\qkmlfm}{\mathord{\text{千克}/\text{米}^3}}%千克每立方米
\newcommand{\kmlflm}{\mathord{\text{克}/\text{厘米}^3}}%克每立方厘米

% 力的单位
\newcommand{\niudun}{\mathord{\text{牛顿}}}%牛顿
\newcommand{\ndmqk}{\mathord{\text{牛顿/千克}}}%牛顿每千克
\newcommand{\ndmpfm}{\mathord{\text{牛顿}/\text{米}^2}}%牛顿每平方米
\newcommand{\pasika}{\mathord{\text{帕斯卡}}}%帕斯卡

% 功的单位
\newcommand{\niudunmi}{\mathord{\text{牛顿·米}}}%牛顿·米
\newcommand{\jiaoer}{\mathord{\text{焦耳}}}%焦耳
\newcommand{\jemm}{\mathord{\text{焦耳/秒}}}%焦耳每秒
\newcommand{\wate}{\mathord{\text{瓦特}}}%瓦特
\newcommand{\qianwa}{\mathord{\text{千瓦}}}%千瓦
\newcommand{\mali}{\mathord{\text{马力}}}%马力

% 时间单位
\newcommand{\miao}{\mathord{\text{秒}}}%秒
