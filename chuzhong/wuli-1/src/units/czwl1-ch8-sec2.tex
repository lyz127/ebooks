\section{功率}\label{sec:8-2}

在建筑工地上,有一大堆砖需要搬到正在修建的楼房上去。
我们可以用人力把它们分批搬上去,也可以用滑轮把它们分批提上去,还可以用起重机把它们一次吊上去。
这三种办法做的功一样多,但区别是很明显的,第一种办法最慢,第二种办法快一些,第三种办法最快。
所以,做功不但有个多少问题,还有一个快慢问题。在生产建设中,做功的快慢常常是很重要的。

我们知道,运动的快慢是用单位时间里通过的路程来表示的,
与此类似,做功的快慢是用单位时间里完成的功来表示的。

\textbf{单位时间里完成的功,叫做功率}。

一架机器在 5 秒钟里完成 $2 \times 10^4$ 焦耳的功,它的功率就是
$\dfrac{2 \times 10^4 \jiaoer}{5\miao} = 4000 \jemm$。所以
$$ \text{功率} = \dfrac{\text{功}}{\text{时间}} \;\juhao $$

如果用 $W$ 表示功,$t$ 表示时间, $P$ 表示功率,上式就可以写成
$$ P = \dfrac{W}{t} \;\juhao $$

功率的单位是由功的单位和时间的单位决定的。
在国际单位制中,功的单位是焦耳,时间的单位是秒,功率的单位就是焦耳/秒。
焦耳/秒有一个专门的名称,叫做\textbf{瓦特}\footnotemark。
\footnotetext{工程还常用马力作功率单位,$1 \mali \approx 735 \wate$。马力不是国际位制单位。}
$$ 1\wate = 1\jemm \;\juhao $$

工程上还用\textbf{千瓦}作功率单位, $1 \qianwa = 1000 \wate$。


\begin{table}[H]
    \centering
    \caption*{一些机器的功率}
    \begin{tabular}{w{l}{20em}w{r}{8em}}
        AO 4514 微型电动机                  & 15 瓦 \\
        YLF-JO 1/2 高精度精密机床用电动机   & 80 瓦 \\
        25 厘米台扇电动机                   & 50 瓦 \\
        YSF 12/2 型工业缝纫机电动机         & 270 瓦 \\
        YK 1000-2/990 三相异步电动机       & 1000 千瓦 \\
        ZJD 250/145-12 型直流电动机         & 4560 千瓦 \\
        双水内冷汽轮发电机(QFS-300-2 型)  & $3 \times 10^5$ 千瓦 \\
    \end{tabular}
\end{table}


一般机器上都有一个小牌子,叫铭牌,上面写着表明这台机器的工作性能和结构特征的一些数据,功率是其中的一项。
了解机器的功率数值在选用机器时很重要。例如我们有一台水泵,要买一台电动机来带动它,
或者说要买一台电动机来跟它配套,我们就需要按照水泵铭牌上写的“配套功率”的数值去买。
假如水泵铭牌上写的配套功率是 5.5 千瓦,而买了一台功率是 4 千瓦的电动机,电动机就带不动水泵。
如果买了一台 7.5 千瓦的电动机,电动机就不能充分发挥作用,对设备是一种浪费。


\liti 一架起重机在 2 分钟里把 50 吨的物体举高了 20 米,起重机的功率是多少千瓦?

首先要求出起重机举高物体所做的功 $W = Fs$,举起物体所用的力 $F$ 应该等于物体重 $G$。

解:$F = G = mg = 50 \times 10^3 \qianke \times 9.8 \ndmqk = 4.9 \times 10^5 \niudun$。

把物体举高 20 米所做的功

$W = Fs = 4.9 \times 10^5 \niudun \times 20 \mi = 9.8 \times 10^6 \jiaoer$。

所以起重机的功率

$P = \dfrac{W}{t} = \dfrac{9.8 \times 10^6 \jiaoer}{2 \times 60 \miao} \approx 8.2 \times 10^4 \wate = 82 \qianwa$。

答:起重机的功率是 82 千瓦。


\lianxi

(1) 两个体重相同的人都从一楼上到三楼,一个是慢慢走上去的,一个是很快跑上去的,
哪个人做的功多?哪个人的功率大?

(2) 一台拖拉机耕地时的牵引力是 $2.85 \times 10^4$ 牛顿,每小时行驶 3600 米,
每小时做的功是多少?拖拉机的功率是多大?

(3) 有一台拖拉机,它的功率是 25 千瓦。一头牛,它的平均功率是 300 瓦,
这台拖拉机 4 小时做的功由这头牛来完成,需要多少时间?



\nonumsection{小实验:测量自己的最大功率}

两三个同学合作测定每个同学上楼(三层)的最大功率,看看谁的功率最大。
做实验前先想好需要测出哪几个量,用什么来测和怎样测。

