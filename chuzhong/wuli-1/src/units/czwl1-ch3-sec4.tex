\begin{starred}
\section{变速直线运动的平均速度}\label{sec:3-4}
\end{starred}

汽车、火车、轮船、飞机等都是做变速运动的,它们在整个运动过程中,速度有时大些,有时小些,有时由小变大,有时又由大变小,不是固定不变的。
但是,我们还是经常说汽车、火车、轮船、飞机等做变速运动的物体的速度是多少,这又是什么意思呢?
原来,对于做变速直线运动的物体,我们所说的速度,指的是它在一段时间内或一段路程中的\textbf{平均速度}。

例如,北京和天津之间的铁路长 140 千米,从北京开出的火车经过两小时到达天津。
火车在这段路程中的运动是变速的,有时加快,有时减慢,有时还停在中间站上。
火车平均在单位时间内通过的路程,就是它的平均速度。火车从北京到天津的平均速度是
$$ \dfrac{140\text{千米}}{2\text{小时}} = 70\text{千米/小时。} $$

知道了这列火车的平均速度,我们就大体上知道了它的运动的快慢程度,但是并不能知道这列火车在什么时候加快,
在什么时候减慢,在什么时候停在中间站上。

平均速度通常用 $\overline{v}$ 来表示。一个做变速直线运动的物体,如果它在 $t$ 时间内通过的路程是 $s$,
它在这段时间内的平均速度,或者说它在这段路程中的平均速度
$$ \overline{v} = \dfrac{s}{t} \text{。} $$

可见,对于做变速直线运动的物体,我们如果把它在一段时间内或一段路程中的运动当成匀速直线运动来处理,
就可以计算出它在这段时间内或这段路程中的平均速度。正是由于这个缘故,平均速度只能大体上反映物体的运动情况,
而不能精确地反映物体的运动情况。

下面的表是一些物体的平均速度。

\begin{table}[H]
    \centering
    \newcommand{\tupian}[1]{\includegraphics[width=2cm]{#1}}
    \renewcommand\arraystretch{1.5}
    \begin{tabular}{|c|m{4cm}|p{2cm}|m{4cm}|}
        \hline
        \tupian{../pic/pjsd/01-woniu} & 蜗牛:1.5毫米/秒 &  \tupian{../pic/pjsd/02-feiji} & 一般军用喷气式飞机:800千米/小时 \\ \hline
        \tupian{../pic/pjsd/03-xingren} & 步行人:4 ~ 5千米/小时 & \tupian{../pic/pjsd/04-paodan} & 普通炮弹:1000米/秒 \\ \hline
        \tupian{../pic/pjsd/05-tuolaji} & 手扶拖拉机(耕地):1.3 ~ 3.9千米/小时 & \tupian{../pic/pjsd/06-huojian} & 单级火箭:4.5千米/秒 \\ \hline
        \tupian{../pic/pjsd/07-zixingche} & 自行车(一般):15千米/小时 & \tupian{../pic/pjsd/08-weixing} & 人造地球卫星(第一宇宙速度):7.9千米/秒 \\ \hline
        \tupian{../pic/pjsd/09-kache} & 卡车(一般):40千米/小时 & \tupian{../pic/pjsd/10-diqiu} & 地球公转:30千米/秒 \\ \hline
        \tupian{../pic/pjsd/11-huoche} & 火车(快车):60 ~ 120千米/小时 & \tupian{../pic/pjsd/12-dianbo} & 光和无线电波:300\,000千米/秒 \\ \hline
    \end{tabular}
\end{table}
