\section{密度的应用}\label{sec:4-3}

密度是物质的特性之一,每种物质都有一定的密度,不同物质的密度一般是不同的。因此我们可以利用密度来鉴别物质。
例如,镀了金的工艺品外表跟纯金的一样,真假难辨,但是只要测出它的密度,跟金的密度对比,立刻就能辨出真伪。
地质勘探人员在找到矿石以后,常常根据矿石的颜色、硬度、密度和其它特性来初步判断是什么矿石。

在体积相等的条件下,密度越小的物质,质量越小,也越轻。
因此,在生产技术中需要减轻产品时,就尽可能选用密度小的材料。
例如,飞机越轻越容易起飞,制造飞机就要尽可能选用密度小的铝合金、塑料等材料,尽可能少用密度大的钢、铁等金属。
摄制电影片在拍摄房屋倒塌伤人的特技镜头时,作为道具的房屋构件必须很轻,现在多采用密度很小的泡沫塑料制作,
演员即使被埋在倒塌的房屋下面,实际却安然无恙。

从密度的公式 $\rho = \dfrac{m}{V}$,利用公式变形可以得出
\begin{gather}
    m = \rho \, V \text{。} \label{eq:midu-1}
\end{gather}
和
\begin{gather}
    V = \dfrac{m}{\rho} \text{。} \label{eq:midu-2}
\end{gather}

这两个公式为应用密度知识解决实际问题提供了新的根据,但是要想用得正确,必须先弄清它们表示什么意思。

公式 \eqref{eq:midu-1} 表示一个物体的质量等于它的密度乘它的体积。如果物体(例如一座花岗岩石碑)的质量不便直接称量,
但体积容易测出,那么测出它的体积,再从书中查出它的密度值,利用公式 \eqref{eq:midu-1} 就可以算出它的质量。

公式 \eqref{eq:midu-2} 表示一个物体的体积等于它的质量除以它的密度。如果物体的体积不便测量,但质量容易称出,
那么称出质量,再查出密度,就可以利用公式 \eqref{eq:midu-2} 算出它的体积。


\liti 煤油可以用油罐车来运输。如果每节油罐车的容量是 50 $\lfm$, 运输 1000 吨煤油需要多少节油罐车?

要求出油罐车的节数,需要先利用公式 \eqref{eq:midu-2} 求出 1000 吨煤油的体积。

解:煤油的质量
$$ m = 1000 \text{吨} = 1000 \times 1000 \text{千克} = 10^6 \text{千克。} $$

从密度表查出煤油的密度 $\rho = 0.8 \times 10^3 \qkmlfm$。

煤油的体积
$$ V = \dfrac{m}{\rho} = \dfrac{10^6 \text{千克}}{0.8 \times 10^3 \qkmlfm} = 1250 \lfm \, \text{。} $$

需要的油罐车的节数
$$ n = \dfrac{1250 \lfm}{50 \lfm} = 25 \; \text{(节)。} $$

答:需要 25 节油罐车。


\liti 某工厂要用横截面积是 $25 \pfhm$ 的铜线 4000 米,应该买这种铜线多少千克?

只要先算出铜线的体积,就可以利用公式 \eqref{eq:midu-1} 求出铜线的质量。
为了计算方便,体积的单位用 $\pflm$,密度的单位用 $\kmlflm$。

解:铜线的横截面积 $S = 25 \;\pfhm = 0.25 \;\pflm$,铜线的长度 $l = 4000 \;\text{米} = 4 \times 10^5 \;\text{厘米}$,
因此铜线的体积
$$ V = S\,l = 0.25 \pflm \times 4 \times 10^5 \text{厘米} = 10^5 \lflm \text{。} $$

从密度表中查出铜的密度 $\rho = 8.9 \times 10^3 \qkmlfm = 8.9 \kmlflm$。

铜线的质量
$$ m = p\,V  = 8.9 \kmlflm \times 10^5 \lflm = 8.9 \times 10^5 \text{克} = 890 \text{千克。} $$

答:应该买 890 千克的铜线。

从上面的计算中可以看出,计算有关密度的问题时,要注意统一单位。
如果密度的单位用 $\qkmlfm$,计算中质量的单位要用千克,体积的单位要用$\lfm$。
如果密度的单位用 $\kmlflm$,计算中质量的单位要用克,体积的单位要用 $\lflm$。



\lianxi

(1) 体积相等的铁块和铜块,哪一个的质量大?哪一个重?

(2) 有两把形状和大小完全相同的小勺,一把是不锈钢的,一把是铝合金。
不用任何仪器,利用学过的密度知识,你能不能分辨出它们?说出你的办法来。

(3) 在三个同样的瓶子中,分别装着质量相等的水、煤油和汽油。
根据它们的体积,你能不能确定每个子里装的是哪种液体?

(4) 不用天平,只用量筒,你能不能量出 100 克酒精?

(5) 体育课上用的铅球,质量是 4 千克,体积约 $0.57 \lffm$,这种铅球是用铅做的吗?

(6) 量出你们教室的长、宽、高,算一算教室里的空气有多少千克?

(7) 一个瓶子能装 1 千克的水,至多能装多少千克的煤油?

(8) $1 \lfm$ 的水结成冰后体积是多大?体积是增大了还是缩小了?

