\section{匀速直线运动}\label{sec:3-3}

拖拉机耕田,汽车行驶,飞机飞行,假定它们都做匀速直线运动,它们的运动情形还是有区别的。
在相等的时间内,拖拉机通过的路程比较短,汽车通过的路程比较长,飞机飞过的路程就更长。
也就是说,拖拉机运动得慢,汽车运动得快,飞机运动得更快。
在物理学里,用\textbf{速度}表示物体运动的快慢。

\textbf{在匀速直线运动中,速度在数值上等于运动物体在单位时间内通过的路程}。

例如,做匀速直线飞行的飞机,它每小时飞过的路程是 600 千米,我们就说它的速度是每小时 600 千米。
在物理学中,更常用的说法是飞机的速度是 600千米每小时。
因为飞机飞行的速度既要用路程,又要用时间来说明,所以我们必须同时用上路程和时间的单位,才能把它的速度说清楚。
这个“千米每小时”,就是速度的单位,通常写作“千米/小时”。

可见,速度的单位要由长度的单位和时间的单位来组成。我们在后面的学习中,还会遇到一些这样的单位。

千米每小时是交通运输中常用的速度单位。
在国际单位制中,长度单位是米,时间单位是秒,速度的单位就是米每秒,写作“米/秒”。

怎样计算匀速直线运动的速度呢?一个做匀速直线运动的物体,如果它在 5 秒内通过的路程是 30 米,
它在 1 秒内通过的路程就是 6 米,它的速度就是 6 米每秒,算式的写法是
$$ \dfrac{30\text{米}}{5\text{秒}} = 6 \text{米/秒。}$$

可见,\CJKunderwave{只要我们用时间去除物体在这段时间内通过的路程,就可以求出匀速直线运动的速度},
也就是说
$$ \text{速度} = \dfrac{\text{路程}}{\text{时间}}\text{。} $$

如果用 $v$ 表示速度,$s$ 表示路程, $t$ 表示时间,速度的公式就是
$$ v = \dfrac{s}{t} \text{。}$$

\liti 一架飞机匀速飞行,它在 5 分钟内飞过的路程是 60 千米,它的速度是多少千米/小时?

为了求出速度是多少千米/小时,我们要先把时间改用小时作单位。

\jie 飞机飞行的时间

$ \qquad t = 5\text{分} = 5 \times \dfrac{1}{60}\text{小时} = \dfrac{1}{12}\text{小时。} $

飞机的速度

$ \qquad v = \dfrac{s}{t} = \dfrac{60\text{千米}}{1 / 12 \text{小时}} = 720 \text{千米/小时。} $

答:这架飞机的速度是 720 千米/小时。


如果我们要用米/秒作单位来表示上述飞机的速度,就需要把千米/小时换算成米/秒。
在换算时可以把千米/小时中的千米和小时分别换算成米和秒。即

$\qquad v = 720\text{千米/小时} = 720 \times \dfrac{1000\text{米}}{3600\text{秒}} = 200 \text{米/秒。}$

可以看出,同一个速度用不同的单位来表示时,数值是不同的。
所以,我们在写速度的时候,不能只写数值,一定要写上单位。

\lianxi

(1) 小船在河里顺流而下,船上坐着一个人,河岸上有树,如果用树作参照物,人是运动的,小船是运动的,河岸是静止的。
如果用人作参照物, 小船是……,河岸是……, 河岸上的树是……。

(2) 一个人坐在行驶的汽车里,看到公路旁的树木或电线杆向后退,这是用……作参照物。
如果用树木或电线杆作参照物,那么汽车是在向……行驶。

(3) 两个同学并肩前进,如果用路旁的树木作参照物,两个人都是……。
如果用一个同学作参照物,另一个同学就是……。

(4) 物理学中常用米/秒作速度的单位,但在计量较小的速度时,也常用厘米/秒作单位。1 米/秒合多少厘米/秒?

(5) 1 千米/小时合多少米/秒? 1 米/秒合多少千米/小时?千米/小时和米/秒哪个是较大的单位?

(6) 一种喷气式飞机的速度是 1200 千米/小时,普通炮弹飞离炮口的速度是 1000 米/耖。这两个速度哪个大?

(7) 飞机在 15 分钟内飞行了 270 千米。它的速度是多少千米/小时? 合多少米/秒?

(8) 你 60 米赛跑的记录是多少?算算你尽力快跑时能达到的速度。

