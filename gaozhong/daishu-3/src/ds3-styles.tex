% 定义本书中使用到的格式

\ctexset {chapter = {pagestyle = headings}}

\usepackage{array}
\usepackage{amsmath}
\allowdisplaybreaks[1]
\usepackage{amssymb}
\usepackage{bm}
\usepackage{calc}
\usepackage{caption}
\captionsetup[figure]{labelsep=none}

\usepackage{cases}
\usepackage{diagbox}
\usepackage[shortlabels]{enumitem}
\usepackage{etoolbox}
\usepackage{extarrows}
\usepackage{float}
\usepackage[stable]{footmisc}
\usepackage{geometry}
\geometry{a4paper,left=2cm,right=2cm,top=2cm,bottom=1cm}

\usepackage{hyperref}
\hypersetup{colorlinks=true, linkcolor=red}

\usepackage{longdivision}
\usepackage{longtable}
\setlength\LTleft\parindent
\setlength\LTright\fill

\usepackage{makecell}
\usepackage{mathtools}
\usepackage{multirow}
\usepackage{nicematrix}
\usepackage{polynom}
\usepackage[figuresright]{rotating}
\usepackage{scalerel}
\usepackage{tablefootnote}
\usepackage{tikz}
\usetikzlibrary{
  arrows.meta,
  calc,
  decorations.pathmorphing,
  decorations.pathreplacing,
  intersections,
  patterns,
  patterns.meta,
  through,
}

\usepackage{wrapfig}
\usepackage{xlop}
\usepackage{xparse}

% 确保 有行内公式 的行,与其上、下行之间,有足够的行距
\setlength{\lineskip}{\baselineskip-\ccwd}
\setlength{\lineskiplimit}{2.5pt}

% 绘制带圈的数字
\newcommand{\circled}[2][]{\tikz[baseline=(char.base)]
    {\node[shape = circle, draw, inner sep = 1pt]
    (char) {\phantom{\ifblank{#1}{#2}{#1}}};%
    \node at (char.center) {\makebox[0pt][c]{#2}};}}
\robustify{\circled}
\newcommand{\tc}[1]{\text{\circled{#1}}}

% 罗马数字
\makeatletter
\newcommand{\rmnum}[1]{\romannumeral #1}
\newcommand{\Rmnum}[1]{\expandafter\@slowromancap\romannumeral #1@}
\makeatother

\newcommand{\gongshishangyi}{\setlength\abovedisplayskip{-1.5em}} % 当 “解:” 后直接写公式时,为了“解”与公式水平对齐所需要的偏移。
\newcommand{\lianxi}{\hspace{1em}\par \textbf{练\quad 习}}
\newcommand{\xhx}[1][2em]{\underline{\hspace{#1}}}

\counterwithout*{subsection}{section}
\counterwithin*{subsection}{chapter}
\setcounter{secnumdepth}{3}
\renewcommand{\thesection}{\chinese{section}}
\renewcommand{\thesubsection}{\arabic{chapter}.\arabic{subsection}}
\renewcommand{\thesubsubsection}{\arabic{subsubsection}.}

\renewcommand{\thefigure}{\thechapter-\arabic{figure}}
\renewcommand{\thefootnote}{\circled{\arabic{footnote}}}

% 在不改变 counter 值的情况下,重置 counter 的子 counter
\newcommand{\touchcounter}[1]{\stepcounter{#1} \addtocounter{#1}{-1}}

\newenvironment{starred}{
  \ctexset {
    chapter/name={*第,章},
    section/name={*}
  }
}{}


% 没有编号但又需要添加到目录中的 section (No num(ber) section)。
% 参数1: 目录中的文字(可选,如果没有指定,则使用参数2的值)
% 参数2: 正文中的文字
\NewDocumentCommand{\nonumsection}{o m}{%
  \touchcounter{section}
  \section*{#2}
  \IfNoValueTF{#1}
    {\addcontentsline{toc}{section}{#2}}
    {\addcontentsline{toc}{section}{#1}}
}

% 没有编号但又需要添加到目录中的 subsection (No num(ber) subsection)。
% 参数1: 目录中的文字(可选,如果没有指定,则使用参数2的值)
% 参数2: 正文中的文字
\NewDocumentCommand{\nonumsubsection}{o m}{%
  \touchcounter{subsection}
  \subsection*{#2}
  \IfNoValueTF{#1}
    {\addcontentsline{toc}{subsection}{#2}}
    {\addcontentsline{toc}{subsection}{#1}}
}


% (编号前)带星号的 subsection
% 参数1: subsection 的标题
\NewDocumentCommand{\starredsubsection}{o m}{
  \refstepcounter{subsection}
  \subsection*{*\thesubsection\quad #2}
  \IfNoValueTF{#1}
    {\addcontentsline{toc}{subsection}{\makebox{*}\thesubsection\qquad #2}}
    {\addcontentsline{toc}{subsection}{\makebox{*}\thesubsection\qquad #1}}
}

\newcounter{cntliti}[subsubsection]           % 例题的计数器
\newcommand{\liti}{ % 例题的标题
  \stepcounter{cntliti}
  \textbf{例\thecntliti}
}

\newcommand{\jie}{\textbf{解: }}
\newcommand{\zhengming}{\textbf{证明: }}

\newcounter{cntxiti}                       % “习题”的计数器
\newcommand{\xiti}{%
  \stepcounter{cntxiti}
  \vspace{1em}
  {\centering \nonumsubsection{\labelxiti}}
}
\newcommand{\labelxiti}{习题 \chinese{cntxiti}}

\newcommand{\xiaojie}{%
  \nonumsection[\labelxiaojie]{\huge \labelxiaojie}
}
\newcommand{\labelxiaojie}{小 \hspace{2em} 结}

\newcounter{cntxiaoti}[subsubsection]      % 小题的计数器
\newcounter{cntxiaoxiaoti}[cntxiaoti]      % 小小题的计数器
\counterwithin*{cntxiaoxiaoti}{cntliti}

\newlength{\lenLabel}                     % 内部变量:用于计算题目前编号所占的长度
\newlength{\lenParent}                    % 内部变量:用于记录父题目编号所占的长度
\setlength{\lenParent}{0em}

\newenvironment{xiaotis}{% “小题” 环境
  \NewDocumentCommand \xiaoti {s m} {% 小题的标题
    \IfBooleanTF {##1}%
      {%
        \setlength{\lenLabel}{\widthof{\labelxiaoti}}%
        \hangafter 1\setlength{\hangindent}{\parindent + \lenLabel}{\hspace{\lenLabel}##2}%
      }%
      {%
        \stepcounter{cntxiaoti}%
        \setlength{\lenLabel}{\widthof{\labelxiaoti}}%
        \hangafter 1\setlength{\hangindent}{\parindent + \lenLabel}{\labelxiaoti ##2}%
      }%
  }%
  \newcommand{\labelxiaoti}{\arabic{cntxiaoti}. }% 1. 2. 3. ……
}{%
}

% 为命令 `\xiaoxiaoti' 增加一个可选参数,是为了实现:当“小题”没有文字时,第一个“小小题” 与“小题”同行。
% 对应 《代数二》“习题 十四” 中的第 10 小题。
% 在此之前,是通过将 “小小题” 上移一行来实现同行显示。如:《代数二》“习题 七” 中的第 10 小题。
% 但当“小题”本身处于页的最后一行时,上移不能生效,会导致“小题”显示在上一页的页末,而“小小题”显示在下一页的页首。
\newenvironment{xiaoxiaotis}{% “小小题” 环境
  \setlength{\lenParent}{\lenParent + \lenLabel}%
  \NewDocumentCommand{\xiaoxiaoti}{o m} {% 小小题的标题
    \stepcounter{cntxiaoxiaoti}%
    \setlength{\lenLabel}{\widthof{\labelxiaoxiaoti}}%
    \IfNoValueTF{##1}%
      {\hangafter 1\setlength{\hangindent}{\parindent + \lenParent + \lenLabel + 0.5em}{\hspace{\lenParent}~\labelxiaoxiaoti ##2}}%
      {\hangafter 1\setlength{\hangindent}{\parindent + \lenParent + \lenLabel + 0.5em}{~\labelxiaoxiaoti ##2}}%
  }%
  \newcommand{\labelxiaoxiaoti}{(\arabic{cntxiaoxiaoti})}% (1) (2) (3) ……
  % 小小题 与 小题 同行时的默认缩进。调用形式为: \xiaoxiaoti[\xxtsep]{小小题的内容}
  \newcommand{\xxtsep}{0em}%
}{%
}

\newcommand{\twoInLine}  [3][10em] {\begin{tabular}[t]{*{2}{@{}p{#1}}} #2 & #3\end{tabular}}
\newcommand{\threeInLine}[4][10em] {\begin{tabular}[t]{*{3}{@{}p{#1}}} #2 & #3 & #4\end{tabular}}
\newcommand{\fourInLine} [5][10em] {\begin{tabular}[t]{*{4}{@{}p{#1}}} #2 & #3 & #4 & #5\end{tabular}}

\newcommand{\twoInLineXxt}  [3][10em] {\begin{tabular}[t]{*{2}{@{}p{#1}}} \xiaoxiaoti{#2} & \xiaoxiaoti{#3}\end{tabular}}
\newcommand{\threeInLineXxt}[4][10em] {\begin{tabular}[t]{*{3}{@{}p{#1}}} \xiaoxiaoti{#2} & \xiaoxiaoti{#3} & \xiaoxiaoti{#4}\end{tabular}}
\newcommand{\fourInLineXxt} [5][10em] {\begin{tabular}[t]{*{4}{@{}p{#1}}} \xiaoxiaoti{#2} & \xiaoxiaoti{#3} & \xiaoxiaoti{#4} & \xiaoxiaoti{#5}\end{tabular}}

\newcommand{\jiange}{\vspace{0.5em}} % 手工调整垂直间隔(测试发现 0.5em 比较合适。不支持参数,特殊情况直接使用 \vspace{} 命令。)
\newcommand{\shangyihang}{\vspace{-1.5em}} % 上移一行

% 修改数学公式与上下文的距离
\makeatletter
\renewcommand\normalsize{%
    \abovedisplayskip 1\p@ \@plus1\p@ \@minus6\p@
    \belowdisplayskip \abovedisplayskip
}
\makeatother


%----------------------------------
% (重)定义一些数学符号
%  一是为了使用/记忆上的方便,二是如果以后有变动,只需要修改一处。
\newcommand{\kongji}{\varnothing}      % 空集
\newcommand{\buji}{\overline}          % 补集
\newcommand*\pxdy{%                    % 平行且等于
    \mathrel{\hspace{.03555em}\text{\tikz[baseline]
            \draw (.1em,0ex) -- (.9em,0ex)
            (.1em,.3ex) -- (.9em,.3ex)
            (.375em,.4ex) -- (.675em,1.8ex)
            (.55em,.4ex) -- (.85em,1.8ex);}\hspace{.03555em}}%
}
\newcommand*\xiangsi{%                 % 相似 (如果不在意有些不同,可以使用 \backsim )
    \mathrel{\text{%
            \tikz \draw[baseline] (-.25em,1.15ex) .. controls (-.55em,1.15ex) and (-.51em,.23ex) .. (-.275em,.23ex) .. controls (0,.23ex) and (0,1.15ex) .. (.275em,1.15ex) .. controls (.51em,1.15ex) and (.55em,.23ex) .. (.25em,.23ex);%
}}}
\newcommand*\quandeng{%                % 全等
    \mathrel{\text{%\small%
            \tikz \draw[baseline] (-.2em,1.35ex) .. controls (-.46em,1.6ex) and (-.54em,.65ex) .. (-.25em,.65ex) .. controls (-.06em,.65ex) and (.06em,1.35ex) .. (.25em,1.35ex) .. controls (.54em,1.35ex) and (.46em,.4ex) .. (.2em,.65ex) (-.46em,.4ex) -- (.46em,.4ex) (-.46em,0ex) -- (.46em,0ex);%
}}}
\newcommand{\arccot}{\mathrm{arccot}\,} % 反余切函数
\newcommand{\celsius}{\ensuremath{^\circ\hspace{-0.09em}\mathrm{C}}} % 摄氏度

% 绘制 弧度 符号。代码来自
%    https://tex.stackexchange.com/questions/96680/
\makeatletter
\DeclareFontFamily{U}{tipa}{}
\DeclareFontShape{U}{tipa}{m}{n}{<->tipa10}{}
\newcommand{\hudu@char}{{\usefont{U}{tipa}{m}{n}\symbol{62}}}%

\newcommand{\hudu}[1]{\mathpalette\hudu@hudu{#1}}

\newcommand{\hudu@hudu}[2]{%
  \sbox0{$\m@th#1#2$}%
  \vbox{
    \hbox{\resizebox{\wd0}{\height}{\hudu@char}}
    \nointerlineskip
    \box0
  }%
}
\makeatother

%----------------------------------
\newlength{\defaultParIndent}  % 页面缺省的 \parindent 长度。
\setlength{\defaultParIndent}{\parindent}

%----------------------------------
\counterwithin*{equation}{subsection}
\renewcommand{\theequation}{\arabic{equation}}
%\newtheorem{theorem}{定理}[subsection]
%\renewcommand\thetheorem{\arabic{theorem}}
%\newtheorem{corollary}{推论}[theorem]
%\renewcommand\thecorollary{\arabic{corollary}}

\makeatletter
% 创建指定名字的标签
% 参数1: 标签的id
% 参数2: 标签的名字 (\nameref{标签id} 所得到的结果)
\newcommand{\namedlabel}[2]{%
  \@bsphack%
  \protected@write\@auxout{}{%
    \string\newlabel{#1}{%
        {\@currentlabel}%
        {\thepage}%
        {{#2}}%\@currentlabelname
        {\@currentHref}{}%
      }%
  }%
  \@bsphack%
}%
\makeatother

% 创建定理环境
% 参数1: 环境名称
% 参数2: 环境的标题
% 参数3: 环境的counter对应的父counter
% 参数4: 创建label时,所写内容的模板(一般要包含环境名称对应计数器)
% 注:前3个参数,就是 \newtheorem{foo3}{bar}[section] 中的三个参数。
\newcommand{\createtheorem}[4]{
  \newcounter{#1}[#3]
  \newtheorem{inner#1}[#1]{#2}
  \expandafter\renewcommand\csname the#1\endcsname{\arabic{#1}}
  \newenvironment{#1}{
    \expandafter\renewcommand{\label}[1]{\namedlabel{####1}{#4}}
    \begin{inner#1}
  }{
    \end{inner#1}
  }
}

\createtheorem{theorem}{定理}{subsection}{定理 \thetheorem}
\createtheorem{corollary}{推论}{theorem}{定理 \thetheorem 推论 \thecorollary}

\newcommand{\eqnameref}[1]{(\nameref{#1})}

%----------------------------------
\newcounter{mylabel}
% 针对计数器 mylabel 创建(指定名字)标签
% 参数1: 标签的id
% 参数2: (可选) 标签的名字 (\nameref{标签id} 所得到的结果)。默认为标签归属章节的名字
\NewDocumentCommand{\mylabel}{m o}{%
  \refstepcounter{mylabel}%
  \IfValueTF{#2}{%
    \namedlabel{#1}{#2}%
  }{%
    \label{#1}%
  }%
}


%
% 创建计数器及其同名的生成label的命令模板。
% 参数1: 计数器的名称(同时也是为计数器创建 label 的命令的名字)
% 参数2: (可选)\the<counter> 显示的内容。默认为 \arabic{<counter>}
% 参数3: (可选)上级计数器的名字。默认为没有上级计数器。
\NewDocumentCommand{\createcounter}{m o o}{
  \IfValueTF{#3}
    {\newcounter{#1}[#3]}
    {\newcounter{#1}}
  \IfValueT{#2}
    {\expandafter\renewcommand\csname the#1\endcsname{#2}}
  \expandafter\newcommand\expandafter*\csname #1\endcsname[1]{
    \refstepcounter{#1} \label{##1}}
}

\createcounter{fangchengzu}[\Roman{fangchengzu}]

%---------------------------
% 在数学的长符号(如:\xlongrightarrow)添加脚注时,遇到脚注重复的问题。
% 下面的代码用于避免脚注重复。
% 注意:不建议在数学公式中使用脚注,这里是为了尽可能展现原书的内容。
\makeatletter
\newif\ifext@arrow@measuring@
\let\saved@ext@arrow\ext@arrow % 保存原始定义
\def\fake@footnotemark{\textsuperscript{\the\numexpr\value{footnote}+1\relax}}
\patchcmd\ext@arrow{\mathop}{\ext@arrow@measuring@true\mathop}{}{}
\patchcmd\ext@arrow{\limits}{\ext@arrow@measuring@false\limits}{}{}
\def\arrowmeasuring#1#2{\ifext@arrow@measuring@#1\else#2\fi}
\makeatother
\def\mytablefootnote#1{\arrowmeasuring{\csname fake@footnotemark\endcsname}{\tablefootnote{#1}}}

