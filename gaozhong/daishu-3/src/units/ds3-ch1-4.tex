\subsection{因式定理}\label{subsec:1-4}

从余数定理可以推出一个重要的定理——因式定理。

\textbf{因式定理\mylabel{theorem:yinshi}[因式定理] \quad 多项式 $f(x)$ 有一个因式 $x -b$ 的充要条件是 $f(b) = 0$。}

\zhengming (1) 充分性。设 $f(b) = 0$,则根据余数定理,$f(x)$ 除以 $x - b$ 所得的余数
也等于 $0$。因此 $f(x)$ 有一个因式 $x - b$。

(2) 必要性。设 $f(x)$ 有一个因式 $x - b$,则 $f(x)$ 除以 $x - b$ 所得的余数等于 $0$。
根据余数定理,有 $f(b) = 0$。


\liti 求证 $n$ 为任何正整数时,$x^n - a^n$ 都有因式 $x - a$。

\zhengming 设 $f(x) = x^n - a^n$,那么 $f(a) = a^n - a^n = 0$。
根据因式定理,$x^n - a^n$ 有因式 $x - a$。


\liti $m$ 为何值时,多项式 $f(x) = x^5 -3x^4 + 8x^3 + 11x + m$ 能被 $x - 1$ 整除?

\jie $f(x)$ 能被 $x - 1$ 整除,就是 $f(x)$ 有因式 $x - 1$。根据因式定理,充要条件是
$f(1) = 0$,即 $1 - 3 + 8 + 11 + m = 0$。由此可得 $m = -17$。


\lianxi
\begin{xiaotis}

\xiaoti{不用除法,求证多项式 $x^4 - 5x^3 - 7x^2 + 15x - 4$ 有因式 $x - 1$。}

\xiaoti{求证 $n$ 为正偶数时,$x^n - a^n$ 有因式 $x + a$;$n$ 为正奇数时,$x^n + a^n$ 有因式 $x + a$。}

\xiaoti{求证 $x^{4n} - 1 \; (n \in N)$ 有因式 $x - i$,又有因式 $x + i$。}

\xiaoti{已知 $f(x) = x^3 - 8x + l$ 有因式 $x + 2$,确定 $l$ 的值。}

\end{xiaotis}

