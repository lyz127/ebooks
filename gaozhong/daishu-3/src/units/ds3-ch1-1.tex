\subsection{一元 $n$ 次多项式}\label{subsec:1-1}

我们在初中学习过整式。单项式与多项式都是整式(单项式可以看作是特殊的多项式)。

以 $x$ 为元的一元多项式的一般形式有
\begin{itemize}[nosep, left=2cm, label=]
    \item 一元一次式 \quad $ax + b$,
    \item 一元二次式 \quad $ax^2 + bx + c$,
    \item 一元三次式 \quad $ax^3 + bx^2 + cx + d$,
\end{itemize}
等等, 其中 $a \neq 0$。

一般地, 以 $x$ 为元的一元 $n$ 次多项式的一般形式可以写成
$$ a_nx^n + a_{n-1}x^{n-1} + a_{n-2}x^{n-2} + \cdots + a_1x + a_0 \text{,} $$
这里 $n$ 是确定的自然数,$a_n \neq 0$。

我们把系数 $a_i \; (i = 0,\, 1,\, \cdots,\, n)$ 都是复数的一元 $n$ 次多项式叫做\textbf{复系数一元 $n$ 次多项式}。
类似地,把系数都是实数(或有理数、整数等)的一元 $n$ 次多项式叫做\textbf{实系数}( 或\textbf{有理系数}、
\textbf{整系数}等)\textbf{一元 $n$ 次多项式},它们都是复系数一元 $n$ 次多项
式的特殊情形。在本章中提到的多项式,如果不特别说明,都是指复系数多项式。


单独的一个非零数 $a_0$,可以看作零次多项式(事实上,当 $x \neq 0$ 时,$a_0 = a_0 x^0$)。
系数都是零的多项式叫做\textbf{零多项式},零多项式没有确定的项数与次数。


当 $x$ 在复数集 $C$ 上取值时,由于复数集中加法、减法、乘法总可以实施,
一元 $n$ 次多项式总有确定的值,所以,当 $x$ 表
示复数,我们可以把一元 $n$ 次多项式看作定义在复数集 $C$
上的函数,并记作 $f(x)$,$g(x)$ 等。当 $x = a + b\,i$ 时,$f(x)$ 的值
记作 $f(a + b\,i)$。很明显,不论 $x$ 在 $C$ 上取什么值,零多项式
的值都等于 $0$ , 所以零多项式可以记作数 $0$。

本章中,我们规定 $x$ 表示复数。


\lianxi
\begin{xiaotis}

\xiaoti{设 $f(x) = x^2 - 5x + 7$,求:}
\begin{xiaoxiaotis}

    \renewcommand\arraystretch{1.2}
    \begin{tabular}[t]{*{2}{@{}p{16em}}}
        \xiaoxiaoti{$f(0)$;} & \xiaoxiaoti{$f\left( -\dfrac{i}{5} \right)$;} \\[1em]
        \xiaoxiaoti{$f(3 + 2\,i)$;} & \xiaoxiaoti{$f\left( \dfrac{5}{2} + \dfrac{\sqrt[]{3}}{2}\,i \right)$。}
    \end{tabular}

\end{xiaoxiaotis}


\xiaoti{零次多项式与零多项式有什么区别?}

\end{xiaotis}

