\subsection{排列数公式}\label{subsec:2-3}

从 $n$ 个不同元素中取出 $m \; (m \leqslant n)$ 个元素的所有排列的个数,
叫做从 $n$ 个不同元素中取出 $m$ 个元素的\textbf{排列数},用符号 $P_n^m$ 表示。
\footnote{$P$ 是英文 Permutation (排列)的第一个字母。}

例如,从 $8$ 个不同元素中取出 $5$ 个元素的排列数表示为 $P_8^5$,
从 $7$ 个不同元素中取出 $6$ 个元素的排列数表示为 $P_7^6$。

现在我们研究计算排列数的公式。

求排列数 $P_n^2$ 可以这样考虑:假定有排好顺序的 $2$ 个空位(图 \ref{fig:2-3}),
从 $n$ 个不同元素 $a_1,\, a_2,\, \cdots,\, a_n$ 中任意取 $2$ 个去填空,
一个空位填一个元素,每一种填法就得到一个排列;反过来,任一个排列总可以由这样的
一种填法得到。因此,所有不同填法的种数就是排列数 $P_n^2$。

\begin{figure}[htbp]
    \centering
    \begin{tikzpicture}[>=Stealth, scale=0.8]
    \draw (0, 0) rectangle (6, 1);
    \draw (3, 0) -- (3, 1);
    \node at (1.5, 1.3) {第 $1$ 位};
    \node at (4.5, 1.3) {第 $2$ 位};
    \draw [<-] (1.5, -0.1) -- (1.5, -0.5) node[below] {$n$};
    \draw [<-] (4.5, -0.1) -- (4.5,  -0.5) node[below] {$n - 1$};
\end{tikzpicture}

    \caption{}\label{fig:2-3}
\end{figure}

现在我们计算有多少种不同的填法,完成这件事可分为两个步骤:

第一步,先排第一个位置的元素,可以从这 $n$ 个元素中任选一个填空,有 $n$ 种方法;

第二步,确定排在第二个位置的元素,可以从剩下的 $n - 1$ 个元素中任选一个填空,有 $n - 1$ 种方法。

于是,根据乘法原理,得到排列数为
$$ P_n^2 = n (n - 1) \text{。} $$

求排列数 $P_n^3$ 可以按依次填 $3$ 个空位来考虑,得到
$$ P_n^3 = n (n - 1) (n - 2) \text{。} $$

同样,求排列数 $P_n^m$ 可以这样考虑:假定有排好顺序的 $m$ 个空位(图\ref{fig:2-4}),
从 $n$ 个不同元素 $a_1,\, a_2,\, \cdots,\, a_n$ 中任意取 $m$ 个去填空,一个空位填一个元素,
每一种填法就得到一个排列;反过来,任一个排列总可以由一种填法得到。
因此,所有不同填法的种数就是排列数 $P_n^m$。

现在我们计算共有多少种不同的填法(图\ref{fig:2-4}):

\begin{figure}[htbp]
    \centering
    \begin{tikzpicture}[>=Stealth, scale=0.8, transform shape]
    \tikzset{
        pics/element/.style args={#1/#2}{
            code = {
            \node at (0, 1.3) {第 $#1$ 位};
            \draw [<-] (0, -0.1) -- (0, -0.5) node[below] {$#2$};
            }
        }
    }
    \draw (0, 0) rectangle (12, 1);
    \draw (2, 0) -- (2, 1);
    \draw (4, 0) -- (4, 1);
    \draw (6, 0) -- (6, 1);
    \draw (10, 0) -- (10, 1);

    \draw (1, 0) pic {element=1/n};
    \draw (3, 0) pic {element=2/n-1};
    \draw (5, 0) pic {element=3/n-2};
    \draw (11, 0) pic {element=m/n-m+1};
    \node at (8, 0.5) {$\cdots \cdots$};
\end{tikzpicture}

    \caption{}\label{fig:2-4}
\end{figure}

第一步,第 $1$ 位可以从 $n$ 个元素中,任选一个填上,共有 $n$ 种填法;

第二步,第 $2$ 位只能从余下的 $n-1$ 个元素中,任选一个填上, 共有 $n-1$ 种填法;

第三步,第 $3$ 位只能从余下的 $n-2$ 个元素中,任选一个填上, 共有 $n-2$ 种填法;

依此类推,当前面的 $m-1$ 个空位都填上后,第 $m$ 位只能从余下的 $n - (m-1)$ 个元素中,
任选一个填上,共有 $n - (m+1)$ 种填法。

根据乘法原理,全部填满 $m$ 个空位共有
$$ n (n-1) (n-2) \cdots (n-m+1) $$
种填法。

所以得到公式
\begin{center}
    \framebox{\begin{minipage}{18em}
        \begin{gather*}
            P_n^m = n (n-1) (n-2) \cdots (n-m+1) \text{。}
        \end{gather*}
    \end{minipage}}
\end{center}
这里 $n,\, m \in N$,并且 $m \leqslant n$。这个公式叫做\textbf{排列数公式}。
其中,公式右边中第一个因数是 $n$,后面的每个因数都比它前面一个因数少 $1$,
最后一个因数为 $n-m+1$,共有 $m$ 个因数相乘。

例如,$P_8^3 = 8 \times 7 \times 6 = 336$。

排列数公式中,当 $m=n$ 时,有
$$ P_n^n = n \cdot (n-1) \cdot (n-2) \cdot \; \cdots \; \cdot 3 \cdot 2 \cdot 1 \text{。} $$
这个公式指出,$n$ 个不同元素全部取出的排列数,等于自然数 $1$ 到 $n$ 的连乘积。
$n$ 个不同元素全部取出的一个排列,叫做 $n$ 个不同元素的一个\textbf{全排列}。
自然数 $1$ 到 $n$ 的连乘积,叫做\textbf{$n$ 的阶乘},用 $n!$ 表示。
所以 $n$ 个不同元素的全排列数公式可以写成
\begin{center}
    \framebox{\begin{minipage}{8em}
        \begin{gather*}
            P_n^n = n! \text{。}
        \end{gather*}
    \end{minipage}}
\end{center}

排列数公式可作如下变形:
$$\begin{aligned}
    P_n^m &= n (n-1) (n-2) \cdots (n-m+1) \\
        &= \dfrac{n \cdot (n-1) \cdot (n-2) \cdot \; \cdots \; \cdot (n-m+1) \cdot (n-m) \cdot \; \cdots \; \cdot 2 \cdot 1}{(n-m) \cdot \; \cdots \; \cdot 2 \cdot 1} \\
        &= \dfrac{n!}{(n-m)!} \text{。}
\end{aligned}$$
因此,排列数公式还可以写成
\begin{center}
    \framebox{\begin{minipage}{8em}
        \begin{gather*}
            P_n^m = \dfrac{n!}{(n-m)!} \text{。}
        \end{gather*}
    \end{minipage}}
\end{center}

\textbf{注意} \quad 为了使这个公式在 $m = n$ 时也能成立,我们规定
$$ 0! = 1 \text{。} $$

\liti 计算 $P_{16}^3$ 及 $P_6^6$ 。

\jie \quad $\begin{aligned}[t]
    P_{16}^3 &= 16 \times 15 \times 14 = 3360; \\
    P_6^6 &= 6! = 720 \text{。}
\end{aligned}$


\liti 求证 $P_n^m + m P_n^{m-1} = P_{n+1}^m$ 。

\zhengming $\begin{aligned}[t]
    P_n^m + m P_n^{m-1} &= \dfrac{n!}{(n-m)!} + m \cdot \dfrac{n!}{[n - (m-1)]!} \\
        &= \dfrac{n! (n-m+1)}{(n-m+1)!} + \dfrac{m \cdot n!}{(n-m+1)!} \\
        &= \dfrac{n! ((n-m+1+m))}{(n-m+1)!} \\
        &= \dfrac{(n+1)!}{[(n+1)-m]!} = P_{n+1}^m \text{。}
\end{aligned}$

$\therefore \quad P_n^m + m P_n^{m-1} = P_{n+1}^m$。


\liti 某段铁路上有 $12$ 个车站,共需要准备多少种普通客票?

\jie 因为每一张车票对应着 $2$ 个车站的一个排列,因此需要准备的车票种数,
就是从 $12$ 个车站中任取 $2$ 个的排列数:
$$ P_{12}^2 = 12 \times 11 = 132 \; \text{(种)。} $$

答: 一共需要准备 $132$ 种普通客票。


\liti 某信号兵用红、黄、蓝三面旗从上到下挂在竖直的旗杆上表示信号,每次可以任挂一面、二面或三面,
并且不同的顺序表示不同的信号,一共可以表示多少种不同的信号?

\jie 如果把 $3$ 面旗看成 $3$ 个元素, 则从 $3$ 个元素里每次取出 $1$ 个元素的一个排列,
对应一种信号。于是,只 $1$ 面旗表示的信号是 $P_3^1$ 种。

同样,只用二面旗表示的信号共有 $P_3^2$ 种,只用 $3$ 面旗表示的信号共有 $P_3^3$ 种。

根据加法原理,所求的信号种数是
$$ P_3^1 + P_3^2 + P_3^3 = 3 + 3 \times 2 + 3 \times 2 \times 1 = 15 \; \text{(种)。} $$

答:一共可以表示 $15$ 种不同的信号。


\liti 用 $0$ 到 $9$ 这十个数字,可以组成多少个没有重复数字的三位数?

分析一:因为要用 $0$ 到 $9$ 这十个数字组成三位数, 每一个三位数可以看成是从这十个数字中
任取 $3$ 个的一个排列($0$ 排在首位的除外),由于百位上的数字不能是 $0$,我们可以分成
两个步骤考虑:先排百位上的数字,再排十位和个位上的数字。

\textbf{解法一:} 百位上的数字只能从除 $0$ 以外的 $1$ 到 $9$ 这九个数字中任选一个,有 $P_9^1$ 种;
十位和个位上的数字,可以从余下的九个数字中任选两个,有 $P_9^2$ 种 (如图\ref{fig:2-5})。
根据乘法原理,所求的三位数的个数是
$$ P_9^1 \cdot P_9^2 = 9 \times 9 \times 8 = 648 \text{。} $$

\begin{figure}[htbp]
    \centering
    \begin{tikzpicture}[>=Stealth, scale=0.8]
    \draw (0, 0) rectangle (6, 1);
    \draw (2, 0) -- (2, 1);
    \draw (4, 0) -- (4, 1);
    \node at (1, 1.3) {百位};
    \node at (3, 1.3) {十位};
    \node at (5, 1.3) {个位};
    \node at (1, -0.8) {$P_9^1$};
    \draw[decorate,decoration={brace,mirror,amplitude=0.2cm}] (2, -0.1) -- (6, -0.1);
    \node at (4, -0.8) {$P_9^2$};
\end{tikzpicture}

    \caption{}\label{fig:2-5}
\end{figure}

分析二:从 $0$ 到 $9$ 这十个数字中任取三个数字的排列数,减去其中以 $0$ 为排头的排列数,
就是用这十个数字组成的没有重复数字的三位数的个数。

\textbf{解法二:} 从 $0$ 到 $9$ 这十个数字中任取三个数字的排列数为 $P_{10}^3$,
其中以 $0$ 为排头的排列数为 $P_9^2$,因此所求的三位数的个数是
$$ P_{10}^3 - P_9^2 = 10 \times 9 \times 8 - 9 \times 8 = 648 \text{。} $$

\textbf{解法三:} 如图 \ref{fig:2-6},符合条件的三位数可以分为三类:

每一位数字都不是 0 的三位数有 $P_9^3$ 个;

个位数字是 $0$ 的三位数有 $P_9^2$ 个;

十位数字是 $0$ 的三位数有 $P_9^2$ 个。

\begin{figure}[htbp]
    \centering
    \begin{tikzpicture}[>=Stealth, transform shape]
    \tikzset{
        numbers/.pic ={
            \draw (0, 0) rectangle (3, 1);
            \draw (1, 0) -- (1, 1);
            \draw (2, 0) -- (2, 1);
            \node at (0.5, 1.3) {百位};
            \node at (1.5, 1.3) {十位};
            \node at (2.5, 1.3) {个位};
        }
    }

    \draw (0, 0) pic {numbers};
    \draw (0.5, 0) -- (1.5, -0.5);
    \draw (1.5, 0) -- (1.5, -0.5);
    \draw (2.5, 0) -- (1.5, -0.5) node[below] {$P_9^3$};
    \node at (1.5, -1.4) {(1)};

    \draw (4, 0) pic {numbers};
    \node at (6.5, 0.5) {$0$};
    \draw (4.5, 0) -- (5.0, -0.5);
    \draw (5.5, 0) -- (5.0, -0.5) node[below] {$P_9^2$};
    \node at (5.5, -1.4) {(2)};

    \draw (8, 0) pic {numbers};
    \node at (9.5, 0.5) {$0$};
    \draw (8.5, 0) -- (9.5, -0.5);
    \draw (10.5, 0) -- (9.5, -0.5) node[below] {$P_9^2$};
    \node at (9.5, -1.4) {(3)};
\end{tikzpicture}

    \caption{}\label{fig:2-6}
\end{figure}

根据加法原理,符合条件的三位数的个数是
$$ P_9^3 + P_9^2 + P_9^2 = 648 \text{。} $$

答:可以组成 $648$ 个没有重复数字的三位数。



\lianxi
\begin{xiaotis}

\xiaoti{写出:}
\begin{xiaoxiaotis}

    \xiaoxiaoti{从四个元素 $a,\, b,\, c,\, d$ 中任取两个元素的所有排列;}

    \xiaoxiaoti{从五个元素 $a,\, b,\, c,\, d,\, e$ 中任取两个元素的所有排列。}

\end{xiaoxiaotis}


\xiaoti{计算:}
\begin{xiaoxiaotis}

    %\renewcommand\arraystretch{1.2}
    \begin{tabular}[t]{p{5em}@{}p{8em}@{}p{7em}@{}p{15em}}
        \xiaoxiaoti{$P_5^2$;} & \xiaoxiaoti{$P_{15}^4$;} & \xiaoxiaoti{$P_{100}^3$;} & \xiaoxiaoti{$P_7^7$;} \\[1em]
        \xiaoxiaoti{$P_6^3$;} & \xiaoxiaoti{$P_8^4 - 2 P_8^2$;} & \xiaoxiaoti{$\dfrac{P_{12}^8}{P_{12}^7}$;} & \xiaoxiaoti{$P_4^1 + P_4^2 + P_4^3 + P_4^4$。}
    \end{tabular}

\end{xiaoxiaotis}


\xiaoti{填写下面的阶乘表,并计算出各阶乘数:\\
    \begin{tabular}{|w{c}{4em}|*{7}{w{c}{2em}|}}
        \hline
        $n$  & $2$ & $3$ & $4$ & $5$ & $6$ & $7$ & $8$ \\ \hline
        $n!$ &     &     &     &     &     &     & \\ \hline
    \end{tabular}
}


\xiaoti{求证:}
\begin{xiaoxiaotis}

    \twoInLineXxt[16em]{$n! = \dfrac{(n + 1)!}{n + 1}$;}
        {$P_8^8 - 8P_7^7 + 7P_6^6 = P_7^7$。}

\end{xiaoxiaotis}


\xiaoti{求 $n$: $\dfrac{P_n^7 - P_n^5}{P_n^5} = 89$。}


\xiaoti{$6$ 名同学排成一排照相,有多少种排法?}

\xiaoti{从 $4$ 种蔬菜品种中选出 $3$ 种,分别种植在不同土质的 $3$ 块
    土地上进行试验,有多少种种植方法?}

\xiaoti{用 $1,\, 2,\, 3,\, 4,\, 5$ 这五个数字,可以组成多少个没
    有重复数字的四位数?其中有多少个四位数是 $5$ 的倍数?}

\end{xiaotis}




