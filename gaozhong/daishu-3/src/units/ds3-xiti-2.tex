\xiti

\begin{xiaotis}

\xiaoti{求证任何复系数一元 $n$ 次方程都可化成
    $$ x^n + b_{n-1} x^{n-1} + \cdots + b_1 x + b_0 = 0$$
    的形式,其中 $b_0,\, b_1,\, \cdots,\, b_{n-1} \in C$。
}


\xiaoti{求下列方程在复数集 $C$ 中的解集:}
\begin{xiaoxiaotis}

    \xiaoxiaoti{$x^3 - 8x^2 + 20x - 16 = 0$;}

    \xiaoxiaoti{$x^4 + x^3 - 5x^2 + x - 6 = 0$;}

    \xiaoxiaoti{$2x^4 + 9x^3 -27x^2 + 53x - 21 = 0$;}

    \xiaoxiaoti{$5x^4 + 6x^3 - 5x - 6 = 0$。}

\end{xiaoxiaotis}


\xiaoti{求最简整系数方程 $f(x) = 0$,已知它在复数集 $C$ 中的解集是:}
\begin{xiaoxiaotis}

    \xiaoxiaoti{$\{ 0,\; 2 - \sqrt{3},\; 2 + \sqrt{3},\; 2\,i,\, -2\,i \}$;}

    \xiaoxiaoti{$\left\{ \dfrac{1}{2}_{(2)},\; -\dfrac{2}{3}_{(3)} \right\}$。}

\end{xiaoxiaotis}


\xiaoti{求证:}
\begin{xiaoxiaotis}

    \xiaoxiaoti{如果一元 $n$ 次方程 $f(x) = 0$ 各项的系数都是正数,那么它没有正数根;}

    \xiaoxiaoti{如果一元 $n$ 次方程 $f(x) = 0$ 各奇次项的系数都是正数,各偶次项
        (包括常数项 $a_0$)的系数都是负数,那么它没有负数根;}

    \xiaoxiaoti{方程 $2x^6 + 3x^4 + 5x^2 + 7 = 0$ 没有实数根。}

\end{xiaoxiaotis}


\xiaoti{利用第 $4$ 题的结论,求下列方程在复数集 $C$ 中的解集:}
\begin{xiaoxiaotis}

    \xiaoxiaoti{$x^3 + \dfrac{7}{2}x^2 + \dfrac{5}{2}x + \dfrac{1}{2} = 0$;}

    \xiaoxiaoti{$x^3 - \dfrac{2}{3}x^2 + 3x - 2 = 0$。}

\end{xiaoxiaotis}


利用一元 $n$ 次方程根与系数的关系解下列各题(第 $6 \sim 9$ 题):

\xiaoti{}%
\begin{xiaoxiaotis}%
    \xiaoxiaoti[\xxtsep]{已知方程 $18x^3 + 9x^2 - 74x + 40 = 0$ 的根中有一个是另一个的 $2$ 倍,
        求这个方程在复数集 $C$ 中的解集;}

    \xiaoxiaoti{已知方程 $x^4 + 4x^3 + 10x^2 + 12x + 9 = 0$ 在复数集 $C$ 中
        的四个根是 $2$ 重根 $a$,$2$ 重根 $b$,求 $a,\, b$ 的值。}

\end{xiaoxiaotis}


\xiaoti{}%
\begin{xiaoxiaotis}%
    \xiaoxiaoti[\xxtsep]{已知方程 $x^4 - 4x^3 - 34x^2 + ax + b = 0$ 的四个根成等差数列,
        求 $a,\, b$ 的值,并且求这个方程的解集;}

    \xiaoxiaoti{已知方程 $8x^3 - 14x^2 + kx + 27 = 0$ 的三个根成等比数列,
        求 $k$ 的值,并且求这个方程的解集。}

\end{xiaoxiaotis}


\xiaoti{已知方程 $x^3 + px^2 + qx + r = 0 \; (p,\, q,\, r \in C)$
    在复数集 $C$ 中的根是 $x_1,\, x_2,\, x_3$,求下列各式的值:}
\begin{xiaoxiaotis}

    \xiaoxiaoti{$\dfrac{1}{x_1 x_2} + \dfrac{1}{x_1 x_3} + \dfrac{1}{x_2 x_3}$;}

    \xiaoxiaoti{$\dfrac{1}{x_1} + \dfrac{1}{x_2} + \dfrac{1}{x_3}$;}

    \xiaoxiaoti{$x_1^2 + x_2^2 + x_3^2$;}

    \xiaoxiaoti{$x_1^2 x_2^2 + x_1^2 x_3^2 + x_2^2 x_3^2$。}

\end{xiaoxiaotis}


\xiaoti{设方程 $x^3 + 2x^2 - x + 3 = 0$ 在复数集 $C$ 中的根是 $x_1,\, x_2,\, x_3$,
    求一元三次方程,使它在 $C$ 中的根是:}
\begin{xiaoxiaotis}

    \renewcommand\arraystretch{1.2}
    \begin{tabular}[t]{*{2}{@{}p{16em}}}
        \xiaoxiaoti{$2x_1,\, 2x_2,\, 2x_3$;} & \xiaoxiaoti{$-x_1,\, -x_2,\, -x_3$;} \\
        \xiaoxiaoti{$\dfrac{1}{x_1},\, \dfrac{1}{x_2},\, \dfrac{1}{x_3}$。}
    \end{tabular}

\end{xiaoxiaotis}


\xiaoti{根据已知条件,求下列方程在复数集 $C$ 中的解集:}
\begin{xiaoxiaotis}

    \xiaoxiaoti{$x^4 - 3x^3 - 10x^2 + 42x - 20 = 0$,已知它的根中有一个是 $3 + i$;}

    \xiaoxiaoti{$x^4 + 3x^3 + 5x^2 + 4x + 2 = 0$,已知它的根中有一个是 $i - 1$;}

    \xiaoxiaoti{$x^4 - 4x^3 + 11x^2 - 14x + 10 = 0$,已知它的根中有两个是 $a + b\,i$,$a + 2b\,i$,其中 $a,\, b \in R$,且 $b \neq 0$。}

\end{xiaoxiaotis}


\xiaoti{求次数最低的实系数方程 $f(x) = 0$,已知它在复数集 $C$ 中的解集 含有下列数:}
\begin{xiaoxiaotis}

    \renewcommand\arraystretch{1.2}
    \begin{tabular}[t]{*{2}{@{}p{16em}}}
        \xiaoxiaoti{$1,\; \dfrac{-1 + \sqrt{3}\,i}{2}$;} & \xiaoxiaoti{$2 + i,\; -1 + i$;} \\
        \xiaoxiaoti{$\pm 1,\; i$;} & \xiaoxiaoti{$\sqrt{2},\; \sqrt{2}\,i$。}
    \end{tabular}

\end{xiaoxiaotis}


\xiaoti{求证实系数一元 $n$ 次方程在 $n$ 为奇数时,有奇数个实根;在 $n$ 为偶数时,有偶数个实根,或者没有实根。}

\xiaoti{已知虚数 $a + b\,i \; (a,\, b \in R)$ 是实系数方程 $x^3 + px + q = 0$ 的根,求证 $2a$ 是方程 $x^3 + px -q = 0$ 的根。}

\xiaoti{一个长方体的长、宽、高分别是 $12cm$,$5cm$,$6cm$。要使各度(即长、宽、高)
    都增加一个相同的长度,体积增加 $186cm^3$,这个增加的长度应是多少?
}

\xiaoti{把边长为 $6dm$ 的正方形铁板的四角各截去一个相同的小正方形,然后把各边折起来
    做成一个无盖的长方体盒。已知这个长方体盒的容积(铁板厚度不计)是 $16dm^3$,
    求截去的小正方形每边的长。
}

\end{xiaotis}

