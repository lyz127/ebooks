\subsection{独立重复试验}\label{subsec:3-5}

某射手射击一次,击中目标的概率是 $0.9$,他射击 $4$ 次恰好击中 $3$ 次的概率是多少?

分别记在第 $1,\, 2,\, 3,\, 4$ 次射击中,这个射手击中目标为事件 $A_1,\, A_2,\, A_3,\, A_4$,
未击中目标为事件 $\buji{A_1},\, \buji{A_2},\, \buji{A_3},\, \buji{A_4}$。
那么,射击次 $4$ 次、击中 $3$ 次共有下面 $4$ 种情况:
$$ A_1A_2A_3\buji{A_4},\quad  A_1A_2\buji{A_3}A_4,\quad  A_1\buji{A_2}A_3A_4,\quad \buji{A_1}A_2A_3A_4 \text{。} $$

上述每一种情况,都可看成是在 $4$ 个位置上取出 $3$ 个写上 $A$,另一个写上 $\buji{A}$,
所以这些情况的种数等于从 $4$ 个元素中取出 $3$ 个的组合数 $C_4^3$,即 $4$ 种。

由于各次射击是否击中相互之间没有影响,根据公式 \eqref{eq:du-li-ex},前 $3$ 次击中、第 $4$ 次未击中的概率
\begin{align*}
    P(A_1 \cdot A_2 \cdot A_3 \cdot \buji{A_4}) &= P(A_1) \cdot P(A_2) \cdot P(A_3) \cdot P(\buji{A_4}) \\
        &= 0.9 \times 0.9 \times 0.9 \times (1 - 0.9) \\
        &= 0.9^3 \times (1 - 0.9)^{4-3}
\end{align*}

同理,
\begin{align*}
    P(A_1 \cdot A_2 \cdot \buji{A_3} \cdot A_4) &= P(A_1 \cdot \buji{A_2} \cdot A_3 \cdot A_4) \\
        &= P(\buji{A_1} \cdot A_2 \cdot A_3 \cdot A_4) \\
        &= 0.9^3 \times (1 - 0.9)^{4-3}
\end{align*}

这就是说,在上面射击 $4$ 次、击中 $3$ 次的 $4$ 种情况中,
每一种发生的概率都是 $0.9^3 \times (1 - 0.9)^{4-3}$。
因为这 $4$ 种情况彼此互斥,根据公式 \eqref{eq:hu-chi-ex},射击 $4$ 次、击中 $3$ 次的概率
\begin{align*}
    P &= P(A_1 \cdot A_2 \cdot A_3 \cdot \buji{A_4})
        + P(A_1 \cdot A_2 \cdot \buji{A_3} \cdot A_4)
        + P(A_1 \cdot \buji{A_2} \cdot A_3 \cdot A_4)
        + P(\buji{A_1} \cdot A_2 \cdot A_3 \cdot A_4) \\
      &= C_4^3 \times 0.9^3 \times (1 - 0.9)^{4-3} \\
      &= 4 \times 0.9^3 \times 0.1 \approx 0.29 \text{。}
\end{align*}

在上面的例子中,$4$ 次射击可以看成是进行 $4$ 次独立重复试验。

一般地,\textbf{如果在一次试验中某事件发生的概率是 $P$,那么 $n$ 次独立重复试验中
这个事件恰好发生 $k$ 次的概率}
\begin{align}
    \boxed{P_n(k) = C_n^k P^k (1 - P)^{n-k} \text{。}} \tag{4} \label{eq:du-li-chong-fu}
\end{align}


\textbf{例} 某气象站天气预报的准确率为 $80\%$ ,计算(结果保留两个有效数字):

(1) $5$ 次预报中恰有 $4$ 次准确的概率;

(2) $5$ 次预报中至少有 $4$ 次准确的概率。

\jie (1) 记 “预报一次,结果准确” 为事件 $A$。预报 $5$ 次相当于作 $5$ 次独立重复试验,
根据公式 \eqref{eq:du-li-chong-fu}, $5$ 次预报中恰有 $4$ 次准确的概率
\begin{align*}
    P_5(4) &= C_5^4 \times 0.8^4 \times (1 - 0.8)^{5-4} \\
            &= 5 \times 0.8^4 \times 0.2 \approx 0.41 \text{。}
\end{align*}

答:$5$ 次预报中恰有 $4$ 次准确的概率约为 $0.41$。

(2) $5$ 次预报中至少有 $4$ 次准确的概率,就是 $5$ 次预报中恰有 $4$ 次准确的概率
与 $5$ 次预报都准确的概率的和,即
\begin{align*}
    P &= P_5(4) + P_5(5) \\
        &= C_5^4 \times 0.8^4 \times (1 - 0.8)^{5-4} + C_5^5 \times 0.8^5 \times (1 - 0.8)^{5-5} \\
        &= 5 \times 0.8^4 \times 0.2 + 0.8^5 \\
        &\approx 0.410 + 0.328 \\
        &\approx 0.74 \text{。}
\end{align*}

答:$5$ 次预报中至少有 $4$ 次准确的概率约为 $0.74$。


\lianxi
\begin{xiaotis}

\xiaoti{生产一种零件,出现次品的概率是 $0.04$。生产这种零件 $4$ 件,其中恰有 $1$ 件次品,
    恰有 $2$ 件次品,至多有 $1$ 件次品的概率各是多少?
}

\xiaoti{在本节开始关于射手 $4$ 次射击的问题中,分别写出射手恰好击中 $4$ 次,$3$ 次,$2$ 次,
    $1$ 次,$0$ 次的概率的计算式子,并将它们与 $(0.9 + 0.1)^4$ 的展开式的各项进行比较。
}

\end{xiaotis}


