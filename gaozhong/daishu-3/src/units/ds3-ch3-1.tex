\subsection{随机事件的概率}\label{subsec:3-1}

在实际生活中,我们会碰到许多事件。有些事件,例如
“在标准大气压下,水的温度达到 $100 \, \celsius$ 时沸腾” ,
“抛一石块,下落” 等,在一定的条件下是必然要发生的,
这种在一定的条件下必然要发生的事件,叫做\textbf{必然車件}。
有些事件, 例如
“在标准大气压下且温度低于 $0 \, \celsius$ 时,冰融化”,
“在常温下,焊锡熔化”等,在一定的条件下是不可能发生的,
这种在一定的条件下不可能发生的事件,叫做\textbf{不可能事件}。


此外,还有一些事件,它们在一定的条件下可能发生也可能不发生。例如:
某人射击一次,可能中靶,也可能不中靶;
掷一枚硬币,可能出现正面,也可能出现反面;
检验某件产品,可能合格,也可能不合格;
某地五月一日,可能下雨,也可能不下雨等等。
这就是说,“某人射击一次,中靶”,“掷一枚硬币,出现正面”,
“检验某件产品,合格” ,“某地五月一日,下雨” 等事件
在一定的条件下是否发生,不能事先确定。
这种在一定的条件下可能发生也可能不发生的事件,叫做\textbf{随机事件}。


随机事件在一次试验\footnotemark 中是否发生虽然不能事先确定,
但是在大量重复试验的情况下,它的发生呈现出一定的规律性。

\footnotetext{一次试验就是将事件的条件实现一次。
例如对“掷一枚硬币,出现正面” 这个事件来说,作一次试验就是将硬币掷一次。}

例如,对生产的一批乒乓球进行抽查,结果如下表所示:

\begin{tabular}{|w{c}{7em}|*{7}{w{c}{2.5em}|}}
    \hline
    抽取球数 $n$              & $50$  & $100$  & $200$  & $500$  & $1000$  & $2000$  \\ \hline
    优等品数 $m$              & $45$  & $92$   & $194$  & $470$  & $954$   & $1902$  \\ \hline
    优等品频率 $\dfrac{m}{n}$ & $0.9$ & $0.92$ & $0.97$ & $0.94$ & $0.954$ & $0.951$ \rule{0pt}{1.5em} \\[0.5em] \hline
\end{tabular}


我们看到,当抽查的球数很多时,抽到优等品的频率 $\dfrac{m}{n}$
(优等品的个数 $m$ 与抽取的球数 $n$ 的比)接近于常数 $0.95$,在它附近摆动。

又如,在相同条件下对某种油菜籽进行发芽试验,结果如下表所示:

\begin{tabular}{|w{c}{7em}|*{10}{w{c}{2em}|}}
    \hline
    每批试验粒数 $n$          & $2$ & $5$   & $10$  & $70$    & $130$   & $310$   & $700$   & $1500$  & $2000$  & $3000$ \\ \hline
    发芽的粒数 $m$            & $2$ & $4$   & $9$   & $60$    & $116$   & $282$   & $639$   & $1339$  & $1806$  & $2715$ \\ \hline
    发芽的频率 $\dfrac{m}{n}$ & $1$ & $0.8$ & $0.9$ & $0.857$ & $0.892$ & $0.910$ & $0.913$ & $0.893$ & $0.903$ & $0.905$ \rule{0pt}{1.5em} \\[0.5em] \hline
\end{tabular}


我们看到,当试验的油菜籽的粒数很多时,油菜籽发芽的频率接近于常数 $0.9$,在它附近摆动。

一般地,\textbf{在大量复进行同一试验时,事件 $A$ 发生的频率 $\dfrac{m}{n}$
总是接近于某个常数,在它附近摆动,}这时就把这个常数叫做\textbf{事件 $A$ 的概率},
记作 $P(A)$\footnotemark 。根据这个定义,求一个事件的概率的基本方法,是通过大量的重复试验,
用这个事件发生的频率近似地作为它的概率。 概率从数量上反映了一个事件发生的可能性的大小。在上面的例子中,
抽查乒乓球得到优等品的率是 $0.95$,就是说,从一批乒乓球中抽取一个,取到优等品的可能性 $95\%$;
油菜籽发芽的概率是 $0.9$,就是说,从进行发芽试验的一批油菜籽中任选一粒,它发芽的可能性是 $90\%$。
\footnotetext{$P$ 是英文 Probability (概率) 的第一个宇母。}


由于任何事件 $A$ 发生的次数 $m$ 不会是负数,也不可能大于试验次数 $n$,事件 $A$ 的概率满足
\begin{center}
    \framebox{\begin{minipage}{12em}
        \begin{gather*}
            0 \leqslant P(A) \leqslant 1 \text{。}
        \end{gather*}
    \end{minipage}}
\end{center}


很明显,必然事件的概率是 $1$ , 不可能事件的概率是 $0$ 。


\lianxi
\begin{xiaotis}

\xiaoti{指出下列事件是必然事件,不可能事件,还是随机事件。}
\begin{xiaoxiaotis}

    \xiaoxiaoti{如果 $a,\, b$ 都是实数, 那么 $a + b = b + a$;}

    \xiaoxiaoti{从分别标有号数 $1,\, 2,\, 3,\, 4,\, 5,\, 6,\, 7,\, 8,\, 9,\, 10$
        的十张号签中任取一张,得到 $4$ 号签;}

    \xiaoxiaoti{没有水分, 种籽发芽;}

    \xiaoxiaoti{某电话总机在一分钟内接到至少 $15$ 次呼唤。}

\end{xiaoxiaotis}


\xiaoti{某射手在同一条件下进行射击, 结果如下表所示:\\
    \begin{tabular}{|w{c}{7em}|*{7}{w{c}{2.5em}|}}
        \hline
        射击次数 $n$                & $10$ & $20$ & $50$ & $100$ & $200$ & $500$  \\ \hline
        击中靶心次数 $m$            & $8$  & $19$ & $44$ & $92$  & $178$ & $455$  \\ \hline
        击中靶心频率 $\dfrac{m}{n}$ &      &      &      &       &       & \rule{0pt}{1.5em} \\[0.5em] \hline
    \end{tabular}
}
\begin{xiaoxiaotis}

    \xiaoxiaoti{计算表中击中靶心的各个频率;}

    \xiaoxiaoti{这个射手射击一次,击中靶心的概率约是多少?}

\end{xiaoxiaotis}

\end{xiaotis}

