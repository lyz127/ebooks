\subsection{二项式定理}\label{subsec:2-7}

我们已经知道,
\begin{align*}
    (a + b)^2 &= a^2 + 2ab + b^2 , \\
    (a + b)^3 &= a^3 + 3a^2b + 3ab^2 + b^3 \text{。}
\end{align*}

现在研究 $(a + b)^n$ 的展开式,这里 $n \in N$ 。

首先,研究 $(a + b)^4$ 的展开式的各项,即研究
$$ (a + b)^4  = (a + b) (a + b) (a + b) (a + b) $$
的展开式的各项。

等号右边的积的展开式的每一项,是从四个括号中每个里任取一个字母的乘积,
因而各项都是 $4$ 次式,即展开式应有下面形式的各项:
$$ a^4,\; a^3b,\; a^2b^2,\; ab^3,\; b^4 \text{。} $$

运用组合的知识,就可以得出展开式各项的系数规律:

在上面四个括号中,都不取 $b$,共有 $1$ 种,即 $C_4^0$种,所以 $a^4$ 的系数是 $C_4^0$;

在四个括号中,恰有 $1$ 个取 $b$,共有 $C_4^1$ 种,所以 $a^3b$ 的系数是 $C_4^1$;

在四个括号中,恰有 $2$ 个取 $b$,共有 $C_4^2$ 种,所以 $a^2b^2$ 的系数是 $C_4^2$;

在四个括号中,恰有 $3$ 个取 $b$,共有 $C_4^3$ 种,所以 $ab^3$ 的系数是 $C_4^3$;

在四个括号中,$4$ 个都取 $b$,共有 $C_4^4$ 种,所以 $b^4$ 的系数是 $C_4^4$。

因此,
$$ (a + b)^4 = C_4^0 a^4 + C_4^1 a^3b + C_4^2 a^2b^2 + C_4^3 ab^3 + C_4^4 b^4 \text{。} $$

一般地,有以下公式:
\begin{center}
    \framebox{\begin{minipage}{32em}
        \begin{gather*}
            (a + b)^n = C_n^0 a^n + C_n^1 a^{n-1}b^1 + \cdots + C_n^r a^{n-r}b^r + \cdots + C_n^n b^n,\; (n \in N) \text{。}
        \end{gather*}
    \end{minipage}}
\end{center}

下面,我们用数学归纳法来证明这一公式。

\zhengming (1) 当 $n = 1$ 时,等式的左边是
$$ (a + b)^1 = a + b \text{;} $$
等式的右边是
$$ C_1^0 a + C_1^1 b = a + b \text{。} $$
于是,当 $n = 1$ 时等式成立。

(2) 假设 $n = k$ 时等式成立,即
$$ (a + b)^k = C_k^0 a^k + C_k^1 a^{k-1}b^1 + \cdots + C_k^r a^{k-r}b^r + \cdots + C_k^k b^k \text{。} $$
现在证明当 $n = k + 1$ 时等式也成立。

由于
\begin{align*}
        & (a + b)^{k+1} \\
    ={} & (a + b)^k (a + b) \\
    ={} & (C_k^0 a^k + C_k^1 a^{k-1}b^1 + \cdots + C_k^r a^{k-r}b^r + \cdots + C_k^k b^k) (a + b) \\
    ={} & C_k^0 a^{k+1} + C_k^1 a^kb^1 + \cdots + C_k^{r+1} a^{k-r}b^{r+1} + \cdots + C_k^k ab^k \\
        & \hspace{3.3em} + C_k^0 a^kb^1 + \cdots + C_k^r a^{k-r}b^{r+1} + \cdots + C_k^{k-1} ab^k + C_k^k b^{k+1} \\
    ={} & C_k^0 a^{k+1} + (C_k^1 + C_k^0) a^kb^1 + \cdots + (C_k^{r+1} + C_k^r) a^{k-r}b^{r+1} + \cdots + (C_k^k + C_k^{k-1}) ab^k + C_k^k b^{k+1} \text{,}
\end{align*}
利用
\begin{align*}
    C_k^0 = C_{k+1}^0,\quad C_k^1 + C_k^0 = C_{k+1}^1,\quad \cdots,\quad C_k^{r+1} + C_k^r = C_{k+1}^{r+1},\quad  \cdots,\\
    C_k^k + C_k^{k-1} = C_{k+1}^k,\quad  C_k^k = C_{k+1}^{k+1},
\end{align*}
则得到
$$ (a+b)^{k+1} = C_{k+1}^0 a^{k+1} + C_{k+1}^1 a^kb^1 + \cdots + C_{k+1}^{r+1} a^{k-r}b^{r+1} + \cdots + C_{k+1}^k a^1b^k + C_{k+1}^{k+1}b^{k+1} \text{。} $$
这就是说,如果 $n=k$ 时等式成立,那么 $n=k+1$ 时等式也成立。

根据 (1) 和 (2) ,可知对于任意自然数 $n$,公式都成立。

这个公式所表示的定理叫做\textbf{二项式定理},
右边的多项式叫做 $(a+b)^n$ 的 \textbf{二项展开式},
其中的系数 $C_n^r \; (r = 0,\, 1,\, \cdots,\, n)$ 叫做 \textbf{二项式系数}。
式中的 $C_n^r a^{n-r}b^r$ 叫做二项展开式的\textbf{通项},用 $T_{r+1}$ 表示,即通项为展开式的第 $r+1$ 项:
\begin{center}
    \framebox{\begin{minipage}{12em}
        \begin{gather*}
            T_{r+1} = C_n^r a^{n-r}b^r \text{。}
        \end{gather*}
    \end{minipage}}
\end{center}

在二项式定理中,如果设 $a = 1$,$b = x$,则得到公式:
$$ (1 + x)^n = 1 + C_n^1 x + C_n^2 x^2 + \cdots + C_n^r x^r + \cdots + x^n \text{。} $$

遇到 $n$ 是较小的正整数时,二项式系数也可以直接用下表计算:

\begin{figure}[htbp]
    \centering
    \begin{tikzpicture}[>=Stealth, scale=0.8]
    \ExplSyntaxOn
    \tikzset{
       pics/mydots/.style~n~args={1}{
            code = {
                \draw (0, 0) node {\int_step_inline:nn {#1} {$\cdot$~} } ;
            }}}
    \ExplSyntaxOff
    \draw (0, 6) node { $(a + b)^1$} (3.5, 6) pic {mydots={16}} (7, 6) node {$1 \quad 1$ };
    \draw (0, 5) node { $(a + b)^2$} (3.3, 5) pic {mydots={15}} (7, 5) node {$1 \quad 2 \quad 1$ };
    \draw (0, 4) node { $(a + b)^3$} (3.1, 4) pic {mydots={14}} (7, 4) node {$1 \quad 3 \quad 3 \quad 1$ };
    \draw (0, 3) node { $(a + b)^4$} (2.9, 3) pic {mydots={13}} (7, 3) node {$1 \quad 4 \quad 6 \quad 4 \quad 1$ };
    \draw (0, 2) node { $(a + b)^5$} (2.7, 2) pic {mydots={12}} (7, 2) node {$1 \quad 5 \quad 10 \quad 10 \quad 5 \quad 1$ };
    \draw (0, 1) node { $(a + b)^6$} (2.5, 1) pic {mydots={11}} (7, 1) node {$1 \quad 6 \quad 15 \quad 20 \quad 15 \quad 6 \quad 1$ };
   \draw (7, 0) pic {mydots={20}};
\end{tikzpicture}

\end{figure}

表中每行两端都是 $1$,而且除 $1$ 以外的每一个数都等于它肩上两个数的和。

类似这样的表,早在我国宋朝数学家杨辉 $1261$ 年所著的《详解九章算法》一书里就已出现,
这本书里记载着下面的表(图 \ref{fig:2-8}),我们称它为\textbf{杨辉三角}\footnotemark。
\footnotetext{在欧洲,人们认为这个表是法国数学家帕斯卡 (Blaise Pasca1,1623 一 1662 年)首先发现的,他们把这个表叫做帕斯卡三角。}

\begin{figure}[htbp]
    \centering
    \begin{tikzpicture}[>=Stealth, scale=0.8]
    \NewDocumentCommand{\num}{m o} {
        \IfNoValueTF{#2}
            {\hbox {#1}}
            {\hbox{\lower-1.0ex\hbox{\scalebox{1}[0.4]{#1}}\lower.1ex\hbox{\kern-1em \scalebox{1}[0.5]{#2}}}}
    }

    \ExplSyntaxOn
    \tikzset{
       pics/line/.style~n~args={1}{
            code = {
                \int_step_inline:nnn {1}{\clist_count:n{#1}} {
                     \draw [fill=white] (##1 -1, 0) node {$\clist_item:nn{#1}{##1}$} circle (0.28);
                }
            }}}
    \ExplSyntaxOff

    \ExplSyntaxOn
    \int_step_inline:nnn {0}{5} {
        \draw (0 + 0.6*#1, 0 - #1) -- (-3.9 + 1.25 * #1, -6);
        \draw (0 - 0.65*#1, 0 - #1) -- (3.7 - 1.25 * #1, -6);
    }
    \ExplSyntaxOff

    \draw (0, 0) pic {line={\num{一}}};
    \draw (-0.65, -1) pic {line={\num{一},\num{一}}};
    \draw (-1.30, -2) pic {line={\num{一},\num{二},\num{一}}};
    \draw (-1.95, -3) pic {line={\num{一},\num{三},\num{三},\num{一}}};
    \draw (-2.60, -4) pic {line={\num{一},\num{四},\num{六},\num{四},\num{一}}};
    \draw (-3.25, -5) pic {line={\num{一},\num{五},\num{十},\num{十},\num{五},\num{一}}};
    \draw (-3.90, -6) pic {line={\num{一},\num{六},\num{十}[五],\num{二}[十],\num{十}[五],\num{六},\num{一}}};
\end{tikzpicture}

    \caption{}\label{fig:2-8}
\end{figure}


\liti 展开 $\left( 1 + \dfrac{1}{x} \right)^4$。

\jie \; $\begin{aligned}[t]
    \left( 1 + \dfrac{1}{x} \right) &= 1 + 4 \left( \dfrac{1}{x} \right) + 6 \left( \dfrac{1}{x} \right)^2 + 4 \left( \dfrac{1}{x} \right)^3 + \left( \dfrac{1}{x} \right)^4 \\
        &= 1 + \dfrac{4}{x} + \dfrac{6}{x^2} + \dfrac{4}{x^3} + \dfrac{1}{x^4} \text{。}
\end{aligned}$



\liti 展开 $\left( 2\sqrt{x} - \dfrac{1}{\sqrt{x}} \right)^6$ 。

\jie \; $\begin{aligned}[t]
    \left( 2\sqrt{x} - \dfrac{1}{\sqrt{x}} \right)^6 &= \left( \dfrac{2x - 1}{\sqrt{x}} \right)^6 = \dfrac{1}{x^3}(2x - 1)^6 \\
        &= \dfrac{1}{x^3} [(2x)^6 - C_6^1 (2x)^5 + C_6^2 (2x)^4 - C_6^3 (2x)^3 + C_6^4 (2x)^2 - C_6^5 (2x) + C_6^6] \\
        &= \dfrac{1}{x^3} (64x^6 - 6 \cdot 32 x^5 + 15 \cdot 16 x^4 - 20 \cdot 8 x^3 + 15 \cdot 4x^2 - 6 \cdot 2x + 1) \\
        &= 64x^3 - 192 x^2 + 240x - 160 + \dfrac{60}{x} - \dfrac{12}{x^2} + \dfrac{1}{x^3} \text{。}
\end{aligned}$



\liti 求 $\left( x - \dfrac{1}{x} \right)^9$ 的展开式中 $x^3$ 的系数。

\jie 展开式的通项是
$$ C_9^r x^{9-r} \left( -\dfrac{1}{x} \right)^r = (-1)^r C_9^r x^{9-2r} \text{。} $$
根据题意,得
\begin{align*}
    9 - 2r &= 3, \\
         r &= 3 \text{。}
\end{align*}
因此,$x^3$ 的系数是
$$ (-1)^3 C_9^3 = -84 \text{。} $$


\textbf{注意} \quad 展开式中第 $r+1$ 项的二项式系数 $C_n^r$ 与第 $r+1$ 项的系数不同,
例如在 $(1 + 2x)^7$ 的展开式中,第四项为 $T_4 = C_7^3 \cdot 1^{7-3} \cdot (2x)^3$,
其二项式系数是 $C_7^3 = 35$,而第四项(即含 $x^3$ 的项)的系数是 $C_7^3 \cdot 2^3 = 280$。



\liti 计算 $(0.997)^3$ 的近似值(精确到 $0.001$)。

\jie \; $\begin{aligned}[t]
    (0.997)^3 &= (1 - 0.003)^3 \\
        &= 1 - 3 \times 0.003 + 3 \times (0.003)^2 - \cdots \text{。}
\end{aligned}$ \\
根据题中精确度的要求,从第三项起以后的各项都可以删去,所以
$$ (0.997)^3 \approx 1 - 3 \times 0.003 = 0.991 \text{。} $$



\lianxi
\begin{xiaotis}

\xiaoti{写出 $(p + q)^7$ 的展开式。}

\xiaoti{求 $(2a + 3b)^6$ 的展开式的第 $3$ 项。}

\xiaoti{求 $(3b + 2a)^6$ 的展开式的第 $3$ 项。}

\xiaoti{写出 $\left( \sqrt[3]{x} - \dfrac{1}{2 \sqrt[3]{x}} \right)^n$ 的展开式的第 $r+1$ 项。}

\xiaoti{求 $(x^3 + 2x)^7$ 的展开式的第 $4$ 项的二项式系数,并求第 $4$ 项的系数。}

\xiaoti{计算 $(1.002)^6$ 的近似值(精确到 $0.001$)。}

\end{xiaotis}

