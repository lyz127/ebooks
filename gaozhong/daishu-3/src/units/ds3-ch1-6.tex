\subsection{一元 $n$ 次方程的根的个数}\label{subsec:1-6}

如果
$$ f(x) = a_n x^n + a_{n-1}x^{n-1} + \cdots + a_1x + a_0 \quad (a_n \neq 0) $$
是复系数一元 $n$ 次多项式,那么方程 $f(x) = 0$,即
$$ a_n x^n + a_{n-1}x^{n-1} + \cdots + a_1x + a_0 = 0 \text{,} $$
叫做\textbf{复系数一元 $n$ 次方程}。 当 $n > 2$ 时,通常也叫做\textbf{复系数高次方程}。
我们过去学过的二项方程是复系数高次方程的特殊情形。

类似地,如果 $f(x)$ 是实系数(或有理系数、整系数等)一元 $n$ 次多项式,那么方程 $f(x) = 0$
叫做\textbf{实系数(或有理系数、整系数等)一元 $n$ 次方程}。 当 $n > 2$ 时,通常也叫做
\textbf{实系数(或有理系数、整系数等)高次方程}。
很明显,实系数、有理系数、整系数一元 $n$ 次方程都是复系数一元 $n$ 次方程的特殊情形。
在本章中所提到的一元 $n$ 次方程,如果不特别说明,都是指复系数一元 $n$ 次方程。

复系数一元 $n$ 次方程 $f(x) = 0$ 的根与多项式 $f(x)$ 的一次因式之间有着极为密切的关系。
首先,根据\nameref{theorem:yinshi},我们有

\begin{theorem} \label{theorem:yyncfc-1}
一元 $n$ 次方程 $f(x) = 0$ 有一个根 $x = b$ 的充要条件是多项式 $f(x)$ 有一个一次因式 $x — b$。
\end{theorem}

在第 \ref{subsec:1-5} 节中,我们还知道,任何一个复系数一元 $n$ 次多项式 $f(x)$ 具有唯一确定的因式分解形式:
$$ f(x) = a_n(x - x_1)^{k_1} (x - x_2)^{k_2} \cdots (x - x_m)^{k_m} \text{,} $$
其中, $k_1,\, k_2,\, \cdots,\, k_m \in N$,且 $k_1 + k_2 + \cdots + k_m = n$,
复数 $x_1,\, x_2,\, \cdots,\, x_m$ 两两不等。
由 \nameref{theorem:yyncfc-1},可知 $x_1,\, x_2,\, \cdots,\, x_m$ 都是方程 $f(x) = 0$ 的根,
且 $f(x) = 0$  没有其他的根。
由于 $x - x_i \; (i = 1,\, 2,\, \cdots,\, m)$ 是多项式 $f(x)$
的 $k_i$ 重一次因式,我们相应地把 $x_i$ 叫做\textbf{方程 $f(x) = 0$ 的 $k_i$ 重根}。


例如: 方程 $x^2 - 6x + 9 = 0$,即 $(x - 3)^2 = 0$ 有 $2$ 重根 $3$;
方程 $(x -4) (x + 2)^2 (x - 5)^3 = 0$ 有 $1$ 重根 $4$,$2$ 重根 $-2$, $3$ 重根 $5$。
这两个方程的解集可以分别表示为 $\{ 3_{(2)} \}$,$\{ 4,\, -2_{(2)},\, 5_{(3)} \}$,
其中元素右边下标括号中的数 $k \; (k \geqslant 2)$ 表示这个元素是相应方程的 $k$ 重根。
例如,元素 $5$ 右边的下标 $(3)$ , 表示 $5$ 是方程 $(x -4) (x + 2)^2 (x - 5)^3 = 0$ 的 $3$ 重根,
即此方程有 $3$ 个相等的根 $5$,但在解集中 $5$ 只能算一个元素。


复系数一元 $n$ 次方程有多少个根呢? 由第 \ref{subsec:1-5} 节的 \nameref{theorem:dxs-1} ,容易得到


\begin{theorem} \label{theorem:yyncfc-2}
    复系数一元 $n$ 次方程在复数集 $C$ 中有且仅有 $n$ 个根($k$ 重根算作 $k$ 个根)。
\end{theorem}


\liti 求方程
$$ f(x) = x^4 + 3x^3 - 2x^2 - 9x + 7 = 0 $$
在复数集 $C$ 中的解集。

\jie 方程 $f(x) = 0$ 的系数 $1$,$3$,$-2$,$-9$,$7$ 的和为 $0$,即 $f(1) = 0$,可知 $1$ 是原方程的根,
从而 $x - 1$ 是多项式 $f(x)$ 的一次因式。利用综合除法,得
$$
\begin{array}{*{4}{c@{\hspace{0.5cm}}}c|l}
    1 & +3 & -2 & -9 & +7 & 1 \\
      &  1 & +4 & +2 & -7 &  \\
    \cline{1-5}
    1 & +4 & +2 & -7 & \multicolumn{1}{|l}{ 1 } & \\
      &  1 & +5 & +7 & \multicolumn{1}{|l}{ }   & \\
    \cline{1-4}
    1 & +5 & +7
\end{array}
$$
(说明:这里第一次除以 $x - 1$,所得商式的系数 $1,\, 4,\, 2,\, -7$ 的和又为 $0$,
可知 $1$ 又是方程 $x^3 + 4x^2 + 2x - 7 = 0$ 的根,所以利用综合除法,再次商式除以 $x - 1$,
得到 $x^2 + 5x + 7$.)\footnote{实际解题时,括号中的说明都可以省去。}即
$$ f(x)  = (x - 1)^2 (x^2 + 5x + 7) = 0 \text{。} $$

这时商式 $x^2 + 5x + 7$ 已降为二次式了,解方程 $x^2 + 5x + 7 = 0$,得原方程的另外两个根
$$ x = \dfrac{-5 \pm \sqrt{3}\,i}{2} \text{。} $$

由 \nameref{theorem:yyncfc-2},原方程有且仅有四个根。从而原方程在复数集 $C$ 中的解集是
$$ \left\{ 1_{(2)},\, \dfrac{-5 + \sqrt{3}\,i}{2},\, \dfrac{-5 - \sqrt{3}\,i}{2} \right\} \text{。} $$

由第 \ref{subsec:1-5} 节的 \nameref{theorem:dxs-2} 及其 \hyperref[corollary:dxs-2-1]{推论},
我们还可以得到:

\begin{theorem} \label{theorem:yyncfc-3}
    如果既约分数 $\dfrac{q}{p}$ 是整系数一元 $n$ 次方程
    $$ a_n x^n + a_{n-1}x^{n-1} + \cdots + a_1x + a_0 = 0 $$
    的根,那么 $p$ 一定是 $a_n$ 的约数,$q$ 一定是 $a_0$ 的约数。
\end{theorem}

\begin{corollary} \label{corollary:yyncfc-3-1}
    如果整系数一元 $n$ 次方程的首项系数是 $1$,那么这个方程的有理数根只可能是整数。
\end{corollary}

\begin{corollary} \label{corollary:yyncfc-3-2}
    如果整系数一元 $n$ 次方程有整数根,那么它一定是常数项的约数。
\end{corollary}

\liti 求方程 $f(x) = 2x^6 + x^5 - 16x^4 - 6x^3 + 25x^2 + 20x + 4 = 0$ 在复数集 $C$ 中的解集。

\jie 原方程是一个整系数一元六次方程。由 \nameref{theorem:yyncfc-3},如果它有有理数根,只可能是
$\pm 1$,$\pm 2$,$\pm 4$,$\pm \dfrac{1}{2}$。因为它的系数之和不为 $0$,可知 $1$ 不是它的根,
利用综合除法,得
$$
\begin{array}{*{6}{c@{\hspace{0.5cm}}}c|l}
    2 & +1 & -16 &  -6 & +25 & +20 & +4 & -1 \\
      & -2 &  +1 & +15 &  -9 & -16 & -4 &    \\
    \cline{1-7}
    2 & -1 & -15 &  +9 & +16 &  +4 & \multicolumn{1}{|l}{ 2 } & \\
      & +4 &  +6 & -18 & -18 &  -4 & \multicolumn{1}{|l}{ }   & \\
    \cline{1-6}
    2 & +3 &  -9 &  -9 &  -2 &  \multicolumn{1}{|l}{ 2 } \\
      & +4 & +14 & +10 &  +2 &  \multicolumn{1}{|l}{   } \\
    \cline{1-5}
    2 & +7 &  +5 &  +1 & \multicolumn{1}{|l}{ -\dfrac{1}{2} } \rule{0pt}{2em} \\
      & -1 &  -3 &  -1 & \multicolumn{1}{|l}{ } \\
    \cline{1-4}
    2 & +6 & +2
\end{array}
$$
(说明:这里先除以 $x + 1$,得商式 $2x^5 - x^4 -15x^3 + 9x^2 + 16x + 4$,余数为 $0$。
方程 $2x^5 - x^4 -15x^3 + 9x^2 + 16x + 4 = 0$ 的有理数根只可能是 $-1$,$\pm 2$,
$\pm 4$,$\pm \dfrac{1}{2}$。用心算可知,$-1$ 不是它的根。用综合除法除以 $x - 2$ 后,
得商式 $2x^4 + 3x^3 -9x^2 -9x - 2$,余数为 $0$。
方程 $2x^4 + 3x^3 -9x^2 -9x - 2 = 0$ 的有理数根只可能是 $\pm 2$,$\pm \dfrac{1}{2}$。
用综合除法除以 $x - 2$ 后,得商式 $2x^3 + 7x^2 + 5x + 1$,余数为 $0$。
方程 $2x^3 + 7x^2 + 5x + 1 = 0$ 的系数都是正数,所以它没有正数根,它的有理数根只可能是 $-\dfrac{1}{2}$。
用综合除法除以 $x + \dfrac{1}{2}$ 后,得商式 $2x^2 + 6x + 2$,余数为 $0$。
$2x^2 + 6x + 2$ 已经是二次式了。)即
$$ f(x) = (x + 1) (x - 2)^2 \left( x + \dfrac{1}{2} \right) (2x^2 + 6x + 2) = 0 \text{。} $$
解方程 $2x^2 + 6x + 2 = 0$,得原方程的另外两个根
$$ x = \dfrac{-3 \pm \sqrt{5}}{2} $$
从而原方程在复数集 $C$ 中的解集是
$$ \left\{ -1,\, 2_{(2)},\, -\dfrac{1}{2},\, \dfrac{-3 + \sqrt{5}}{2},\, \dfrac{-3 - \sqrt{5}}{2} \right\} \text{。} $$



\liti 求\textbf{最简整系数方程}(就是求一个整系数方程,并使最高次项系数取尽可能小的自然数) $f(x) = 0$,
已知它在复数集 $C$ 中的解集为 $\left\{ \dfrac{1}{2}_{(2)},\, i,\, -i \right\}$。

\jie 设所求的方程是
$$ a \left( x - \dfrac{1}{2} \right)^2 (x - i) (x + i) \quad (a \in N,\, \text{且} a \neq 0) \text{,} $$
即
$$ a \left( x^2 -x + \dfrac{1}{4} \right) (x^2 + 1) = 0 \text{,} $$

因为要求上式具有最简单的整系数,所以取 $a = 4$,代入上式,得
$$ 4 \left( x^2 -x + \dfrac{1}{4} \right) (x^2 + 1) = 0 \text{,} $$
即
$$ 4x^4 - 4x^3 + 5x^2 -4x + 1 = 0 \text{。} $$



\lianxi
\begin{xiaotis}

\xiaoti{(口答)在复数集 $C$ 中,下列方程有且仅有多少个根?}
\begin{xiaoxiaotis}

    \renewcommand\arraystretch{1.2}
    \begin{tabular}[t]{@{}p{16em}@{}p{17em}}
        \xiaoxiaoti{$x^4 + 3x^2 + 4x + 5 = 0$;} & \xiaoxiaoti{$x^7 = 1$;} \\
        \xiaoxiaoti{$(x + 1)^4 - (x - 1)^4 = 0$;} & \xiaoxiaoti{$(x - 1)^2 (x - 2)^3 (x + 3)^4 = 0$。}
    \end{tabular}

\end{xiaoxiaotis}


\xiaoti{}%
\begin{xiaoxiaotis}%
    \xiaoxiaoti[\xxtsep]{用综合除法验证 $3$ 是方程 $2x^3 - 5x^2 - 9x + 18 = 0$ 的一个根;}

    \xiaoxiaoti{把方程 $2x^3 - 5x^2 - 9x + 18 = 0$ 先化成
        $$ (x - x_1) (ax^2 + bx + c) = 0$$
        的形式,再化成
        $$ a (x - x_1) (x - x_2) (x - x_3) = 0$$
        的形式。
    }

\end{xiaoxiaotis}


\xiaoti{求下列方程在复数集 $C$ 中的解集:}
\begin{xiaoxiaotis}

    \xiaoxiaoti{$3x^3 - 11x^2 + 5x + 3 = 0$;}

    \xiaoxiaoti{$6x^4 + 31x^3 + 25x^2 - 39x + 9 = 0$;}

    \xiaoxiaoti{$3x^5 + 4x^4 - 10x^3 - 14x^2 + 3x + 6 = 0$。}

\end{xiaoxiaotis}


\xiaoti{求最简整系数方程 $f(x) = 0$,已知它在复数集 $C$ 中的解集是:}
\begin{xiaoxiaotis}

    \xiaoxiaoti{$\{ -1,\, -2,\, 3 \}$;}

    \xiaoxiaoti{$\left\{ -\dfrac{1}{2},\, \dfrac{2}{3},\, 1 \right\}$;}

    \xiaoxiaoti{$\{ -2,\, 2_{(2)} \}$;}

    \xiaoxiaoti{$\left\{ 1 + i,\, 1 - i,\, -\dfrac{\sqrt{3}}{2},\, \dfrac{\sqrt{3}}{2} \right\}$。}

\end{xiaoxiaotis}

\end{xiaotis}

