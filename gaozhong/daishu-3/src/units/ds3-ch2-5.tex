\subsection{组合数公式}\label{subsec:2-5}

从 $n$ 个不同元素中取出 $m \; (m \leqslant n)$ 个元素的所有组合的个数,
叫做从 $n$ 个不同元素中取出 $m$ 个元素的\textbf{组合数},用符号 $C_n^m$ 表示。
\footnote{$C$ 是英文 Combination (组合)的第一个字母。}

例如,从 $8$ 个不同元素中取出 $5$ 个元素的组合数表示为 $C_8^5$;
从 $7$ 个不同元素中取出 $6$ 个元素的组合数表示为 $C_7^6$。

现在我们从研究组合数 $C_n^m$ 与排列数 $P_n^m$ 的关系入手,找出组合数 $C_n^m$ 的计算公式。

例如,从 $4$ 个不同元素 $a,\, b,\, c,\, d$ 中取出 $3$ 个元素的排列与组合的关系如下表所示:

\begin{figure}[htbp]
    \centering
    \begin{tikzpicture}[>=Stealth, transform shape]
    \tikzset{
        pics/zuhe pailie/.style args={#1/#2/#3}{
            code = {
                \draw (-0.9, -0.4) rectangle (0.9, 0.4);
                \node at (0, 0) {$#1 \quad #2 \quad #3$};

                \draw[->] (1, 0) -- (2, 0);

                \draw (2.1, -0.6) rectangle (8.9, 0.6);
                \node at (3,  0.3) {$#1 \quad #2 \quad #3$};
                \node at (3, -0.3) {$#1 \quad #3 \quad #2$};
                \node at (5.5,  0.3) {$#2 \quad #1 \quad #3$};
                \node at (5.5, -0.3) {$#2 \quad #3 \quad #1$};
                \node at (8,  0.3) {$#3 \quad #1 \quad #2$};
                \node at (8, -0.3) {$#3 \quad #2 \quad #1$};
            }
        }
    }

    \node at (0, 1) {组 \quad 合};
    \node at (5.5, 1) {排 \quad 列};

    \draw (0, 0) pic {zuhe pailie=a/b/c};
    \draw (0, -1.6) pic {zuhe pailie=a/b/d};
    \draw (0, -3.2) pic {zuhe pailie=a/c/d};
    \draw (0, -4.8) pic {zuhe pailie=b/c/d};
\end{tikzpicture}

\end{figure}

由表中可以看出,对于每一个组合都有 $6$ 个不同的排列,因此,求从 $4$ 个不同元素中取 $3$ 个
元素的排列数 $P_4^3$,可以分以下两步完成:

第一步,从 $4$ 个不同元素中取出 $3$ 个元素作组合,共有 $C_4^3 (=4)$ 个;

第二步,对每一个组合中的 $3$ 个不同元素作全排列,各有 $P_3^3 (=6)$ 个。

根据乘法原理,得
$$ P_4^3 = C_4^3 \cdot P_3^3 \text{,}$$

因此,
$$ C_4^3 = \dfrac{P_4^3}{P_3^3} \text{,} $$

一般地,求从 $n$ 个不同元素中取出 $m$ 个元素的排列数 $P_n^m$,可分以下两步完成:

第一步,先求出从这 $n$ 个不同的元素中取出 $m$ 个元素的组合数 $C_n^m$;

第二步,求每一个组合中 $m$ 个元素的全排列数 $P_m^m$。

根据乘法原理,得到
$$ P_n^m = C_n^m \cdot P_m^m \text{,} $$
因此
\begin{center}
    \framebox{\begin{minipage}{22em}
        \begin{gather*}
            C_n^m = \dfrac{P_n^m}{P_m^m} = \dfrac{n (n-1) (n-2) \cdots (n-m+1)}{m!} \text{。}
        \end{gather*}
    \end{minipage}}
\end{center}
这里 $n,\, m \in N$,并且 $m \leqslant n$。这个公式叫做\textbf{组合数公式}。

因为
$$ P_n^m = \dfrac{n!}{(n-m)!} \text{,} $$
所以,上面的组合数公式还可以写成
\begin{center}
    \framebox{\begin{minipage}{12em}
        \begin{gather*}
            C_n^m = \dfrac{n!}{m! (n-m)!} \text{。}
        \end{gather*}
    \end{minipage}}
\end{center}
这也是组合数的一个常用公式。

\liti 计算 $C_{10}^4$ 及 $C_7^3$ 。

\jie \; $\begin{aligned}[t]
    C_{10}^4 &= \dfrac{10 \times 9 \times 8 \times 7}{4 \times 3 \times 2 \times 1} = 210;\\
    C_7^3 &= \dfrac{7 \times 6 \times 5}{3 \times 2 \times 1} = 35 \text{。}
\end{aligned}$


\liti 求证 $C_n^m = \dfrac{m+1}{n-m} \cdot C_n^{m+1}$。

\zhengming \; $\because \; C_n^m = \dfrac{n!}{m! (n-m)!}$,

$\begin{aligned}[t]
    \dfrac{m+1}{n-m} \cdot C_n^{m+1} &= \dfrac{m+1}{n-m} \cdot \dfrac{n!}{(m+1)! (n-m-1)!} \\
        &=\dfrac{m+1}{(m+1)!} \cdot \dfrac{n!}{(n-m) (n-m-1)} \\
        &= \dfrac{n!}{m! (n-m)!} \text{,}
\end{aligned}$

$\therefore \quad C_n^m = \dfrac{m+1}{n-m} \cdot C_n^{m+1}$。


