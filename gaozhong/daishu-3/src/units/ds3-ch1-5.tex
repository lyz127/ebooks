\subsection{利用综合除法、因式定理来分解因式}\label{subsec:1-5}

设有多项式 $x^6 + x^4 - x^2 - 1$,我们把它在复数集 $C$ 中分解因式,得
\begin{align*}
        & x^6 + x^4 - x^2 - 1 \\
    ={} & x^4(x^2 + 1) - (x^2 + 1) \\
    ={} & (x^4 - 1) (x^2 + 1) \\
    ={} & (x^2 - 1) (x^2 + 1) (x^2 + 1) \\
    ={} & (x + 1) (x - 1) (x + i)^2 (x - i)^2 \text{。}
\end{align*}

这个一元六次式有六个一次因式,其中有两个相同因式 $x + i$,两个相同因式 $x - i$。

关于复系数一元 $n$ 次多项式的因式分解, 有下面的定理:

\begin{theorem} \label{theorem:dxs-1}
    任何一个复系数一元 $n$ 次多项式 $f(x)$ 有且仅有 $n$ 个一次因式 $x - x_i \; (i = 1,\, 2,\, \cdots,\, n)$,
    把其中相同的因式的积用幂表示后,$f(x)$ 就具有唯一确定\footnotemark 的因式分解的形式:
    \begin{align*}
        f(x) = a_n(x - x_1)^{k_1} (x - x_2)^{k_2} \cdots (x - x_m)^{k_m} \tag{*}\mylabel{eq:dxs-1}[*]
    \end{align*}
    其中, $k_1,\, k_2,\, \cdots,\, k_m \in N$,且 $k_1 + k_2 + \cdots + k_m = n$,
    复数 $x_1,\, x_2,\, \cdots,\, x_m$ 两两不等。
\end{theorem}
\footnotetext{这里所说的“唯一确定”,不考虑各一次因式的书写顺序,也不考虑常数因子。例如,我们把
$4x^2 -16 = (2x + 4) (2x -4)$ 与 $4x^2 -16 = 4 (x - 2) (x + 2)$ 等等看成同一种分解形式。}

这个定理的证明超出中学数学范围,本书从略。

我们把分解结果 (\nameref{eq:dxs-1}) 中的 $x - x_i \; (i = 1,\, 2,\, \cdots,\, m)$ 叫做\textbf{多项式 $f(x)$ 的 $k_i$ 重一次因式。}
例如:多项式 $x^2 - 6x + 9$ 有 $2$ 重一次因式 $x - 3$;
多项式 $(x - 4) (x-2)^2 (x - 5)^3$ 有 $1$ 重一次因式 $x - 4$,$2$ 重一次因式 $x + 2$,$3$ 重一次因式 $x - 5$。

由 \nameref{theorem:dxs-1} 可以得到:

\begin{corollary} \label{corollary:dxs-1-1}
    如果 $x - a,\, x - b \; (a \neq b)$ 都是复系数一元 $n$ 次多项式 $f(x)$ 的因式,那么它们的积 $(x - a) (x - b)$ 也是 $f(x)$ 的因式。
\end{corollary}

\zhengming 因为 $f(x)$ 的分解结果 \eqnameref{eq:dxs-1} 是唯一确定的,所以 $a$ 一定等于某个 $x_i$,
$b$ 一定等于某个 $x_j$ ($i = 1,\, 2\, \cdots,\, m,$ $j = 1,\, 2\, \cdots,\, m$,且 $i \neq j$),
即 $(x - a) (x - b) = (x - x_i) (x - x_j)$,由此可见,$(x - x_i) (x - x_j)$ 是 $f(x)$ 的因式。


对于一个任意的复系数一元 $n$ 次多项式,要求出它的一次因式,没有一般的方法。
但是,如果 $f(x)$ 是整系数多项式,那么进一步运用下列定理,就能使我们较快地求得它的形如
$x - \dfrac{q}{p}$ (其中$p,\, q$ 是互质的整数)的因式,或者确定它没有这种形式的因式。

\begin{theorem} \label{theorem:dxs-2}
    如果整系数多项式 $f(x) = a_nx^n + a_{n-1}x^{n-1} + \cdots + a_1x + a_0$ 有因式
    $x - \dfrac{q}{p}$ (其中$p,\, q$ 是互质的整数),那么
    $p$ 一定是首项系数 $a_n$ 的约数,$q$ 一定是末项系数 $a_0$ 的约数。
\end{theorem}

例如,$15x^2 - 17x + 4$ 有因式 $3x - 1$,$5x - 4$,即 $3 \left( x - \dfrac{1}{3} \right)$,
$5 \left( x - \dfrac{4}{5} \right)$,$3$ 与 $5$ 都是首项系数 $15$ 的约数,
$1$ 与 $4$ 都是末项系数 $4$ 的约数。
又如, 如果 $2x^4 - x^3 - 13x^2 - x - 15$ 有 $x - \dfrac{q}{p}$ 形式的因式
(其中 $p$,$q$ 是互质的整数,下同),那么 $p$ 只可能是 $1$,$2$,
$q$ 只可能是 $\pm 1$,$\pm 3$,$\pm 5$,$\pm 15$。


要注意定理中“$p$ 是 $a_n$ 的约数,$q$ 是 $a_0$ 的约数” 只是 “整系数多项式
$a_nx^n + a_{n-1}x^{n-1} + \cdots + a_1x + a_0$ 有因式 $x - \dfrac{q}{p}$
的必要条件,而不是充分条件(为什么)。


下面证明 \nameref{theorem:dxs-2}。

\zhengming 因为 $f(x)$ 有因式 $x - \dfrac{q}{p}$,所以 $f \left( \dfrac{q}{p} \right) = 0$,即
$$ a_n \left( \dfrac{q}{p} \right)^n + a_{n-1} \left( \dfrac{q}{p} \right)^{n-1} + \cdots + a_1 \left( \dfrac{q}{p} \right) + a_0 = 0 \text{。} $$

把第二项起的各项移到右边,并将两边都乘以 $p^{n-1}$,得
$$ \dfrac{a_n q^n}{p} = - (a_{n-1} q^{n-1} + \cdots + a_1 q p^{n-2} + a_0 p^{n-1}) \text{。} $$

等式的右边是一个整数,所以 $\dfrac{a_n q^n}{p}$ 也是一个整数,即 $p$ 能整除 $a_n q^n$。
但因 $p$,$q$ 互质,所以 $p$ 的任何一个因数都不是 $q$ 的约数,从而也不是 $q^n$ 的约数\footnotemark。
由此可知,$p$ 一定是 $a_n$ 的约数。
\footnotetext{例如:$p = 2 \times 5 = 10$,$q = 3 \times 7 = 21$,$p$ 的任何一个质因数($2$ 或 $5$)
都不是$q$ 的约数,从而也不是 $q^n = 3^n \times 7^n$ 的约数。}

同理,把上面的等式写成
$$ \dfrac{a_0 p^n}{q} = - (a_{n} q^{n-1} + a_{n-1} q^{n-2} p + \cdots + a_1 p^{n-1}) \text{,} $$
可以证明 $q$ 一定是 $a_0$ 的约数。

\begin{corollary} \label{corollary:dxs-2-1}
    如果首项系数为 $1$ 的整系数多项式  $f(x) = x^n + a_{n-1}x^{n-1} + \cdots + a_1x + a_0$
    有因式 $x - q$,其中 $q \in Z$,那么 $q$ 一定是常数项 $a_0$ 的约数。
\end{corollary}

利用 \nameref{theorem:dxs-2} 及其 \hyperref[corollary:dxs-2-1]{推论},我们可以较快地确定一个整系数一元一次式
是不是某整系数一元 $n$ 次多项式的因式。


\liti 把 $f(x) = x^3 + x^2 - 10x - 6$ 分解因式\footnotemark。
\footnotetext{如果没有特别说明,本章中所说的因式分解,都是指在复数集 $C$ 中的因式分解。}

分析:先考虑 $x - q \; (q \in Z)$ 形式的因式,因为 $f(x)$ 的首项系数为 $1$ 的整系数多项式,
根据 \nameref{theorem:dxs-2} 的 \hyperref[corollary:dxs-2-1]{推论},可能出现的 $x - q$ 这样的因式
有 $x \pm 1$,$x \pm 2$,$x \pm 3$,$x \pm 6$。

判断 $x - 1$,$x + 1$ 是不是 $f(x)$ 的因式时,只要根据 \nameref{theorem:yinshi},
计算 $f(1)$,$f(-1)$ 是不是等于零就可以了。因为 $f(1) = -14 \neq 0$,$f(-1) = 4 \neq 0$,
所以 $x - 1$,$x + 1$ 都不是 $f(x)$ 的因式。

判断 $x - 2$,$x + 2$,$\cdots$ 是不是 $f(x)$ 的因式时,可以计算 $f(2)$,$f(-2)$,$\cdots$
是不是等于零。用综合除法,由于

\begin{minipage}{6cm}
    $$
    \begin{array}{*{3}{c@{\hspace{0.5cm}}}c|l}
        1 & +1 & -10 & -6 & 2 \\
          & +2 &  +6 &    &   \\
        \cline{1-4}
        1 & +3 &  -4 & \multicolumn{1}{r}{ } &
    \end{array}
    $$
\end{minipage}
\begin{minipage}{6cm}
    $$
    \begin{array}{*{3}{c@{\hspace{0.5cm}}}c|l}
        1 & +1 & -10 & -6 & -2 \\
          & -2 &  +2 &    &    \\
        \cline{1-4}
        1 & -1 &  -8 & \multicolumn{1}{r}{ } &
    \end{array}
    $$
\end{minipage}\\
(上面左式中 $-4 \times 2$ 不是 $-6$ 的相反数,右式中 $-8 \times (-2)$ 不是 $-6$ 的相反数,
已经说明相应的余数都不是零,所以不必继续演算了。)可见 $x - 2$,$x + 2$ 都不是 $f(x)$ 的因式。但
$$
\begin{array}{*{3}{c@{\hspace{0.5cm}}}c|l}
    1 & +1 & -10 & -6 & 3 \\
      & +3 & +12 & +6 &   \\
    \cline{1-4}
    1 & +4 &  +2 & \multicolumn{1}{|r}{ 0 } & \\
    \cline{4-4}
\end{array}
$$
可知 $x - 3$ 是 $f(x)$ 的因式。所以
$$ x^3 + x^2 - 10x - 6 = (x - 3) (x^2 + 4x + 2) \text{。} $$

因为方程 $x^2 + 4x + 2 = 0$ 的两个根是 $-2 \pm \sqrt{2}$,于是
$$ x^3 + x^2 - 10x - 6 = (x - 3) (x + 2 + \sqrt{2}) (x + 2 - \sqrt{2}) \text{。} $$

解答时,只需写出结果是因式的试验过程,其他过程不必写出。

\jie \qquad $\begin{array}[t]{*{3}{c@{\hspace{0.5cm}}}c|l}
    1 & +1 & -10 & -6 & 3 \\
      & +3 & +12 & +6 &   \\
    \cline{1-4}
    1 & +4 &  +2 & \multicolumn{1}{|r}{ 0 } & \\
    \cline{4-4}
\end{array}$

$\therefore \quad \begin{aligned}[t]
        & x^3 + x^2 - 10x - 6 \\
    ={} & (x - 3) (x^2 + 4x + 2) \\
    ={} & (x - 3) (x + 2 + \sqrt{2}) (x + 2 - \sqrt{2}) \text{。}
\end{aligned}$



\liti 把 $f(x) = 2x^4 - x^3 - 13x^2 -x -15$ 分解因式。

分析:$f(x)$ 首项系数不是 $1$,根据 \nameref{theorem:dxs-2},可试验
$x \pm 1$,$x \pm 3$,$x \pm 5$,$x \pm 15$,
$x \pm \dfrac{1}{2}$,$x \pm \dfrac{3}{2}$,
$x \pm \dfrac{5}{2}$,$x \pm \dfrac{15}{2}$。

因为 $f(1) \neq 0$,$f(-1) \neq 0$,所以 $x + 1$,$x - 1$ 不是 $f(x)$ 的因式。但
$$
\begin{array}{*{4}{c@{\hspace{0.5cm}}}c|l}
    2 & -1 & -13 & -1 & -15 & 3 \\
      & +6 & +15 & +6 & +15 &   \\
    \cline{1-5}
    2 & +5 &  +2 & +5 & \multicolumn{1}{|r}{ 0 } & \\
    \cline{5-5}
\end{array}
$$
所以,
$$ f(x) = (x - 3) (2x^3 + 5x^2 + 2x + 5) \text{。} $$

继续分解 $2x^3 + 5x^2 + 2x + 5$。这个多项式的首项系数是 $2$,末项系数是 $5$,
所以只要试验 $x \pm 1$,$x \pm 5$,$x \pm \dfrac{1}{2}$,$x \pm \dfrac{5}{2}$
就可以了。但因 $x \pm 1$ 不是原来多项式 $f(x)$ 的因式,所以也不是这个多项式的因式。由
$$
\begin{array}{*{3}{c@{\hspace{0.5cm}}}c|l}
    2 & +5 & +2 & +5 & -\dfrac{5}{2} \\
      & -5 & +0 & -5 &  \\
    \cline{1-4}
    2 & +0 & +2 & \multicolumn{1}{|r}{ 0 } & \\
    \cline{4-4}
\end{array}
$$
得
\begin{align*}
    2x^3 + 5x^2 + 2x + 5 &= \left( x + \dfrac{5}{2} \right) (2x^2 + 2) \\
        &= (2x + 5)(x^2 + 1) \text{。}
\end{align*}

(实际上,利用分组分解法也容易得到这个结果。)

$x^2 + 1$ 在复数集 $C$ 中还能继续分解因式,所以

$\begin{aligned}[t]
        & 2x^4 - x^3 - 13x^2 -x -15 \\
    ={} & (x - 3) (2x + 5) (x^2 + 1) \\
    ={} & (x - 3) (2x + 5) (x + i) (x - i) \text{。}
\end{aligned}$

\jie \qquad $\begin{array}[t]{*{4}{c@{\hspace{0.5cm}}}c|l}
    2 & -1 & -13 & -1 & -15 & 3 \\
      & +6 & +15 & +6 & +15 &   \\
    \cline{1-5}
    2 & +5 &  +2 & +5 & \multicolumn{1}{|r}{ 0 } & \\
    \cline{5-5}
\end{array}
$

\phantom{\jie} \qquad $\begin{array}[t]{*{3}{c@{\hspace{0.5cm}}}c|l}
    2 & +5 & +2 & +5 & -\dfrac{5}{2} \\
      & -5 & +0 & -5 &  \\
    \cline{1-4}
    2 & +0 & +2 & \multicolumn{1}{|r}{ 0 } & \\
    \cline{4-4}
\end{array}
$

$\therefore \quad \begin{aligned}[t]
    & 2x^4 - x^3 - 13x^2 -x -15 \\
={} & (x - 3) (2x^3 + 5x^2 + 2x + 5) \\
={} & (x - 3) (2x + 5) (x^2 + 1) \\
={} & (x - 3) (2x + 5) (x + i) (x - i) \text{。}
\end{aligned}$

\lianxi
\begin{xiaotis}

\xiaoti{判断下列各命题的真假,并说明理由:}
\begin{xiaoxiaotis}

    \xiaoxiaoti{如果整数 $p$,$q$ 互质,且 $p$ 为整系数多项式
        $$ f(x) = a_n x^n + a_{n-1} x^{n-1} + \cdots + a_1 x + a_0 $$
        的首项系数 $a_n$ 的约数,$q$ 为末项系数 $a_0$ 的约数,那么
        $x - \dfrac{q}{p}$ 一定是 $f(x)$ 的因式;
    }

    \xiaoxiaoti{如果多项式 $f(x) = g(x) q(x)$,其中 $g(x)$,$q(x)$ 也是多项式,
        且 $x - a$ 不是 $f(x)$ 的因式,那么 $x -a$ 也不是 $q(x)$ 的因式。
    }

\end{xiaoxiaotis}


\xiaoti{把下列多项式分解因式:}
\begin{xiaoxiaotis}

    \xiaoxiaoti{$x^3 + x^2 - 10x + 8$;}

    \xiaoxiaoti{$2x^3 - 9x^2 + x + 12$;}

    \xiaoxiaoti{$x^4 + 3x^3 - 3x^2 -12x - 4$;}

    \xiaoxiaoti{$4x^5 - 13x^3 + 10x^2 - 42x + 20$。}

\end{xiaoxiaotis}

\end{xiaotis}

