\subsection{一元 $n$ 次方程的根与系数的关系}\label{subsec:1-7}

我们知道,如果一元二次方程
$$ ax^2 + bx + c = 0 $$
的两个根是 $x_1,\, x_2$,那么根与系数之间有下列关系:
$$
\begin{cases}
    x_1 + x_2 = -\dfrac{b}{a}, \\[1em]
    x_1 x_2 = \dfrac{c}{a} \text{。}
\end{cases}
$$

一般地说,我们有如下的定理:

\begin{theorem}\footnotemark
    如果一元 $n$ 次方程
    $$ a_n x^n + a_{n-1}x^{n-1} + \cdots + a_1x + a_0 = 0 $$
    在复数集 $C$ 中的根是 $x_1,\, x_2,\, \cdots,\, x_n$,那么
    \begin{equation}
    \begin{cases}
        x_1 + x_2 + \cdots + x_n = -\dfrac{a_{n-1}}{a_n}, \\
        x_1 x_2 + x_1 x_3 + \cdots + x_{n-1} x_n = \dfrac{a_{n-2}}{a_n}, \\
        x_1 x_2 x_3 + x_1 x_2 x_4 + \cdots + x_{n-2} x_{n-1} x_n = -\dfrac{a_{n-8}}{a_n}, \\
        \rule{2em}{0pt} \cdots \cdots \cdots  \hspace{6em} \cdots \cdots \cdots \\
        x_1 x_2 \cdots x_n = (-1)^n \dfrac{a_0}{a_n}
    \end{cases} \tag{*} \label{eq:wddl}
    \end{equation}
\end{theorem}
\footnotetext{此定理又叫做\textbf{韦达定理}。韦达(Francois Viete, 1540 - 1603 年),法国数学家。}

例如,当 $n = 3$ 时,有
$$\begin{cases}
    x_1 + x_2 + x_3 = -\dfrac{a_2}{a_3}, \\
    x_1 x_2 + x_1 x_3 + x_2 x_3 = \dfrac{a_1}{a_3}, \\
    x_1 x_2 x_3 = -\dfrac{a_0}{a_3} \text{。}
\end{cases}$$

下面我们证明上述定理。

\zhengming 因为方程 $a_n x^n + a_{n-1}x^{n-1} + \cdots + a_1x + a_0 = 0$ 的根是 $x_1,\, x_2,\, \cdots,\, x_n$,
由上节\nameref{theorem:yyncfc-1},可以把 $f(x) = a_n x^n + a_{n-1}x^{n-1} + \cdots + a_1x + a_0$
分解成 $n$ 个一次因式与 $a_n$ 的积:
$$ a_n x^n + a_{n-1}x^{n-1} + \cdots + a_1x + a_0 = a_n (x - x_1) (x - x_2) \cdots (x - x_n) \text{。} $$

因为
\begin{align*}
        & (x - x_1) (x - x_2) \cdots (x - x_n) \\
    ={} & x^n - (x_1 + x_2 + \cdots x_n) x^{n-1}  \\
        & + (x_1 x_2 + x_1 x_3 + \cdots + x_{n-1} x_n) x^{n-2}  \\
        & + \cdots + (-1)^n x_1 x_2 \cdots x_n \text{,}
\end{align*}
代入上式后,把每一项与 $a_n$ 相乘。现将等号左边的多项式减去等号右边的多项式,所得的差 $F(x)$
是一个零多项式。对 $F(x)$ 进行整理,可知
\begin{align*}
    F(x) ={} & [a_{n-1} + a_n (x_1 + x_2 + \cdots + x_n)] x^{n-1} \\
             & + [ a_{n-2} - a_n (x_1 x_2 + x_1 x_3 + \cdots + x_{n-1} x_n)] x^{n-2} \\
             & + \cdots + [ a_0 - (-1)^n a_n x_1 x_2 \cdots x_n ] \text{。}
\end{align*}

根据零多项式的定义,$F(x)$ 的系数都是 $0$,所以
$$\begin{cases}
    a_{n-1} + a_n (x_1 + x_2 + \cdots + x_n) = 0, \\
    a_{n-2} - a_n (x_1 x_2 + x_1 x_3 + \cdots + x_{n-1} x_n) = 0, \\
    \rule{2em}{0pt} \cdots\cdots\cdots  \qquad \cdots\cdots\cdots \\
    a_0 - (-1)^n a_n x_1 x_2 \cdots x_n = 0 \text{。}
\end{cases}$$

由此即得
$$\begin{cases}
    x_1 + x_2 + \cdots + x_n = -\dfrac{a_{n-1}}{a_n}, \\
    x_1 x_2 + x_1 x_3 + \cdots + x_{n-1} x_n = \dfrac{a_{n-2}}{a_n}, \\
    \rule{2em}{0pt} \cdots \cdots \cdots  \hspace{6em} \cdots \cdots \cdots \\
    x_1 x_2 \cdots x_n = (-1)^n \dfrac{a_0}{a_n} \text{。}
\end{cases}$$

这个定理的逆命题也成立,即对于任何一元 $n$ 次方程
$$ f(x) = a_n x^n + a_{n-1}x^{n-1} + \cdots + a_1x + a_0 = 0 \text{,} $$
如果有 $n$ 个数 $x_1,\, x_2,\, \cdots,\, x_n$ 满足\eqref{eq:wddl}式,
那么 $x_1,\, x_2,\, \cdots,\, x_n$ 一定是方程 $f(x) = 0$ 的根。


\liti 已知方程 $2x^3 -5x^2 -4x + 12 = 0$ 有 $2$ 重根,利用
一元 $n$ 次方程的根与系数的关系,求这个方程在复数集 $C$ 中的解集。

\jie 设原方程在 $C$ 中的解集为 $\{ \alpha_{(2)},\, \beta \}$,那么
\begin{numcases}{}
    2\alpha + \beta = \dfrac{5}{2} \text{,} \tag{1} \\
    \alpha^2 + 2\alpha\beta = -2 \text{,} \tag{2} \\
    \alpha^2 \beta = -6 \text{。} \tag{3}
\end{numcases}
(说明:这里有两个未知数、三个方程。我们可以选其中两个方程,求出满足这两个方程的
$\alpha,\, \beta$,再代入另一个方程,如能满足,求出满足这两个方程的 $\alpha,\, \beta$,
再代入另一个方程,如能满足,就是方程组的解,否则不是。)解(1),(2)两式组成的方程组,得

\rule{4cm}{0pt}
\begin{minipage}{2.5cm}
    $\begin{cases}
        \alpha = 2, \\[1em]
        \beta = -\dfrac{3}{2},
    \end{cases}$
\end{minipage}
或 \quad
\begin{minipage}{3cm}
    $\begin{cases}
        \alpha = -\dfrac{1}{3}, \\[1em]
        \beta = \dfrac{19}{6} \text{。}
    \end{cases}$
\end{minipage}

第一个解满足 (3) 式;第二个解不满足 (3) 式,应舍去。
所以原方程在 $C$ 中的解集为 $\left\{ 2_{(2)},\, -\dfrac{3}{2} \right\}$。


\liti 当且仅当 $k$ 是什么数的时候,方程 $x^3 - 6x^2 + 3x + k = 0$ 的三个根成
等差数列\footnote{本书中涉及等差、等比数列问题,都限于在实数集内讨论。}?

\jie 设原方程在 $C$ 中的三个根成等差数列,并分别记作
$a - d,\, a,\, a + d \; (d \geqslant 0)$,那么
$$\begin{cases}
    (a - d) + a + (a + d) = 6, \\
    a (a - d) + a (a + d) + (a + d) (a - d) = 3, \\
    a (a - d) (a + d) = -k \text{。}
\end{cases}$$
整理后,得
\begin{numcases}{}
    3a = 6 \text{,} \tag{1} \\
    3a^2 - d^2 = 3 \text{,} \tag{2} \\
    a^3 - ad^2 = -k \text{。} \tag{3}
\end{numcases}

由 (1) 式,得 $a = 2$;代入 (2) 式,得 $d = 3$;再代入 (3) 式,便得
$$ k = 10 \text{。} $$

这就是说,要使原方程的三个根成等差数列,$k$ 必须等于 $10$。
反过来,容易验证,当方程中的 $k = 10$ 时,$-1 \; (= a - d)$,
$2 \; (= a)$, $5 \; (= a + d)$ 三个数确实是原方程的根,且成等差数列。
所以当且仅当 $k = 10$ 时,原方程的三个根成等差数列。

\lianxi

利用一元 $n$ 次方程的根与系数的关系解下列各题:

\begin{xiaotis}

\xiaoti{已知方程 $6x^4 + 7x^3 - 36x^2 - 7x + 6 = 0$ 的根中有三个是
    $-\dfrac{1}{2}$,$\dfrac{1}{3}$,$2$,求这个方程在复数集 $C$ 中的解集。}


\xiaoti{已知方程 $2x^3 + x^2 -8x - 4 = 0$ 的根都是实数,且有两个互为相反数,求这个方程的解集。}

\xiaoti{已知方程 $x^3 - 9\sqrt{2}x^2 + 46x - 30\sqrt{2} = 0$ 的三个根成等差数列,求这个方程的解集。}

\end{xiaotis}

