\xiaojie

一、本章主要内容是复系数一元 $n$ 次多项式和一元 $n$ 次方程的一些基本概念和重要性质。


二、复系数一元 $n$ 次多项式可以写成
$$ a_n x^n + a_{n-1}x^{n-1} + \cdots + a_1x + a_0 \text{,} $$
其中 $n \in N$,系数 $a_0,\, a_1,\, \cdots,\, a_n \in C$,且 $a_n \neq 0$。
在需要的时候,我们可以把一元 $n$ 次多项式看作定义在 $C$ 上的函数,并记作 $f(x)$,$g(x)$ 等。

单独的一个非零复数,可以看作零次多项式;系数都是零的多项式叫做零多项式。
不论 $x$ 在 $C$ 上取什么值,零多项式的值都等于 $0$。


三、余数定理:多项式 $f(x)$ 除以 $x - b\; (b \in C)$ 所得的余数等于 $f(b)$。余数定理有一个
重要的推论 —— 因式定理:多项式 $f(x)$ 有一个因式 $x - b$ 的充要条件是 $f(b) = 0$。


四、任何一个复系数一元 $n$ 次多项式 $f(x)$ 有唯一确定
(不考虑各一次因式的书写顺序,也不考虑常数因子)的因式分解的形式:
$$ f(x) = a_n(x - x_1)^{k_1} (x - x_2)^{k_2} \cdots (x - x_m)^{k_m} \text{,} $$
其中 $k_1,\, k_2,\, \cdots,\, k_m \in N$,且 $k_1 + k_2 + \cdots + k_m = n$,
复数 $x_1,\, x_2,\, \cdots,\, x_m$ 两两不等。
$x - x_i \; (i = 1,\, 2,\, \cdots,\, m)$ 叫做多项式 $f(x)$ 的 $k_i$ 重因式。


五、如果整系数多项式
$f(x) = a_n x^n + a_{n-1}x^{n-1} + \cdots + a_1x + a_0$
有因式$x - \dfrac{q}{p}$(其中 $p$,$q$ 是互质的整数),那么 $p$ 一定是
首项系数 $a_n$ 的约数,$q$ 一定是末项系数 $a_0$ 的约数。
根据这个定理,依靠综合除法的帮助,或者可以求出整系数项式的
整系数一次因式,或者可以证明它没有这种因式。


六、多项式因式分解与解方程密切相关。如果 $x - x_i \; (i = 1,\, 2,\, \cdots,\, m)$
是多项式 $f(x)$ 的 $k_i$ 重一次因式,那么 $x_i$ 叫做方程 $f(x) = 0$ 的 $k_i$ 重根。
由此可以推出重要定理: 复系数一元 $n$ 次方程在复数集 $C$ 中有且仅有 $n$ 个根
($k$ 重根算作 $k$ 个根)。这个定理确定了复系数一元 $n$ 次方程在 $C$ 中的根的个数,
显然,实系数一元 $n$ 次方程在实数集 $R$ 中的根的个数不具有这一性质。


七、如果一元 $n$ 次方程 $a_n x^n + a_{n-1}x^{n-1} + \cdots + a_1x + a_0 = 0$
在复数集 $C$ 中的根是 $x_1,\, x_2,\, \cdots,\, x_n$,那么
\begin{equation}
    \begin{cases}
        x_1 + x_2 + \cdots + x_n = -\dfrac{a_{n-1}}{a_n}, \\
        x_1 x_2 + x_1 x_3 + \cdots + x_{n-1} x_n = \dfrac{a_{n-2}}{a_n}, \\
        x_1 x_2 x_3 + x_1 x_2 x_4 + \cdots + x_{n-2} x_{n-1} x_n = -\dfrac{a_{n-8}}{a_n}, \\
        \rule{2em}{0pt} \cdots \cdots \cdots  \hspace{6em} \cdots \cdots \cdots \\
        x_1 x_2 \cdots x_n = (-1)^n \dfrac{a_0}{a_n}
    \end{cases} \tag{*}
\end{equation}

这个定理确定了一元 $n$ 次方程的根与系数的关系,它的逆命题也成立,即对于任何一元 $n$ 次方程
$$ f(x) = a_n x^n + a_{n-1}x^{n-1} + \cdots + a_1x + a_0 = 0 \text{,} $$
如果有 $n$ 个数 $x_1,\, x_2,\, \cdots,\, x_n$ 满足 (*) 式,
那么 $x_1,\, x_2,\, \cdots,\, x_n$ 一定是方程 $f(x) = 0$ 的根。


八、如果虚数 $a + b\,i$ 是实系数一元 $n$ 次方程 $f(x) = 0$ 的根,那么它的共轭虚数 $a - b\,i$
也是这个方程的根, 并且它们的重数相等。这个定理确定了实系数一元 $n$ 次方程虚根成对的性质。

九、根据本章的知识,我们可以把某些一元 $n$ 次多项式分解因式,也可以求出相应的一元 $n$ 次方程的解集。
在解决这类问题时,要认真分析已知条件,选择较为简便的解法。例如:对于整系数一元 $n$ 次方程,
可以利用综合除法求有理根;已知实系数一元 $n$ 次方程的一个虚数根 $a + b\,i$,就可知道它有另
一个虚数根 $a — b\,i$;有时还可根据已知条件,利用一元 $n$ 次方程的根与系数的关系;也可先观察
方程的系数有什么特点,确定根的范围;等等。“降次”是在分解因式和解方程时经常采用的一种基本思想方法。

