{\centering \nonumsubsection{A \hspace{1em} 组}}

\begin{xiaotis}

\xiaoti{求证:}
\begin{xiaoxiaotis}

    \xiaoxiaoti{$\dfrac{(2n)!}{2^n \cdot n!} = 1 \cdot 3 \cdot 5 \cdot \, \cdots \, \cdot (2n-1)$;}

    \xiaoxiaoti{$1! + 2 \cdot 2! + 3 \cdot 3!  + \cdots + n \cdot n! = (n + 1)! - 1$,(提示:考虑等式 $n \cdot n! = (n + 1)! - n!$)。}

\end{xiaoxiaotis}


\xiaoti{}%
\begin{xiaoxiaotis}%
    \xiaoxiaoti[\xxtsep]{已知 $\dfrac{1}{C_5^m} - \dfrac{1}{C_6^m} - \dfrac{7}{10 \cdot C_7^m}$,求 $C_8^m$;}

    \xiaoxiaoti{已知 $\dfrac{C_n^{m-1}}{2} = \dfrac{C_n^m}{3} = \dfrac{C_n^{m+1}}{4}$,求 $n$ 与 $m$。}

\end{xiaoxiaotis}


\xiaoti{乘积 $\displaystyle \sum_{i=1}^m a_i \, \sum_{j=1}^n b_j$ 一共有多少项?}


\xiaoti{$6$ 名同学站成一排,其中某一名不站在排头,也不站在排尾,共有多少种站法?}

\xiaoti{由数字 $1,\, 2,\, 3,\, 4,\, 5,\, 6$ 可以组成多少个没有重复数字的自然数?}

\xiaoti{由数字 $1,\, 2,\, 3,\, 4,\, 5,\, 6$ 可以组成多少个没有重复数字,并且比 $500000$ 大的自然数?}

\xiaoti{一个集合由 $8$ 个不同的元素组成,这个集合中含 $3$ 个元素的子集有几个?}

\xiaoti{一个集合由 $5$ 个不同的元素组成,其中含 $1$ 个、$2$ 个、$3$ 个、$4$ 个元素的子集共有几个?}

\xiaoti{}%
\begin{xiaoxiaotis}%
    \xiaoxiaoti[\xxtsep]{平面内有 $n$ 条直线,其中没有两条互相平行,也没有三条相交于一点,一共有多少个交点?}

    \xiaoxiaoti{空间有 $n$ 个平面,其中没有两个互相平行,也没有三个相交于一直线,一共有多少条交线?}

\end{xiaoxiaotis}


\xiaoti{$100$ 件产品中有 $97$ 件合品,$3$ 件次品,从中任意抽取 $5$ 件进行检查。}
\begin{xiaoxiaotis}

    \xiaoxiaoti{抽出的 $5$ 件都是合格品的抽法有多少种?}

    \xiaoxiaoti{抽出的 $5$ 件恰好有 $2$ 件是次品的抽法有多少种?    }

    \xiaoxiaoti{抽出的 $5$ 件至少有 $2$ 件是次品的抽法有多少种?}

\end{xiaoxiaotis}


\xiaoti{书架上有 $4$ 本不同的数学书,$5$ 本不同的物理书,$3$ 本不同的化学书,
    全部竖起排成一排,如果不使同类的书分开,一共有多少种排法?}


\xiaoti{当 $a$ 的绝对值与 $1$ 相比很小时,$(1 + a)^n$ 的近似值可以用公式
    $(1 + a)^n \approx 1 + na$ 来计算,用这个近似公式计算:}
\begin{xiaoxiaotis}

    \renewcommand\arraystretch{1.2}
    \begin{tabular}[t]{*{2}{@{}p{16em}}}
        \xiaoxiaoti{$(1.002)^5$;} & \xiaoxiaoti{$(0.997)^6$;} \\
        \xiaoxiaoti{$(1.005)^{10}$;} & \xiaoxiaoti{$(0.9995)^9$。}
    \end{tabular}

\end{xiaoxiaotis}


\xiaoti{分别求当 $n = 1,\, 2,\, 3,\, 4$ 时,$\left( 1 + \dfrac{1}{n} \right)^n$ 的值。}



\xiaoti{}%
\begin{xiaoxiaotis}%
    \xiaoxiaoti[\xxtsep]{求 $(a + \sqrt{b})^{12}$ 展开式中第 $9$ 项;}

    \xiaoxiaoti{求 $(1 - 2x)^5 (1 + 3x)^4$ 展开式中按 $x$ 升幂排列的前三项;}

    \xiaoxiaoti{求 $\left( 9x - \dfrac{1}{3\sqrt{x}} \right)^{18}$ 展开式的常数项;}

    \xiaoxiaoti{求 $n$:已知 $(1 + \sqrt{x})^n$ 的展开式中第 $9$ 项、第 $10$ 项、
        第 $11$ 项的二项式系数成等差数列;}

    \xiaoxiaoti{求 $(1 + x + x^2) (1 - x)^{10}$ 展开式中 $x^4$ 的系数。}

\end{xiaoxiaotis}


\xiaoti{}%
\begin{xiaoxiaotis}%
    \xiaoxiaoti[\xxtsep]{用二项式定理证明 $55^{55} + 9$ 能被 $8$ 整除;}

    \xiaoxiaoti{用二项式定理求 $89^{10}$ 除以 $88$ 的余数。}

\end{xiaoxiaotis}


\xiaoti{证明 $(1 + x)^{2n}$ 展开式中 $x^n$ 的系数等于
    $(1 + x)^{2n-1}$ 展开式中 $x^n$ 的系数的 $2$ 倍。}


\xiaoti{已知 $\left( \sqrt{x} + \dfrac{1}{\sqrt[3]{x}} \right)^n$ 展开式的二项式系数之和
    比 $(a + b)^{2n}$ 展开式的二项式系数之和小 $240$,求:
}
\begin{xiaoxiaotis}

    \xiaoxiaoti{$\left( \sqrt{x} + \dfrac{1}{\sqrt[3]{x}} \right)^n$ 展开式的第 $3$ 项;}

    \xiaoxiaoti{$(a + b)^{2n}$ 展开式的中间项。}

\end{xiaoxiaotis}
\end{xiaotis}

