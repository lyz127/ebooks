\xiaojie

一、在这一章中,我们初步介绍了事件的概率的概念及其计算。


二、随机事件在现实世界中是广泛存在的。一次试验中,事件是否发生虽然带有偶然性,
但在大量重复试验下,它的发生呈现出一定的规律性,即事件发生的频率总是接近于某
个常数,在它附近摆动,这个常数就叫做这一事件的概率。


三、对于某些事件,也可以直接通过分析来计算其概率。如果一次试验中共有 $n$ 种等可能出现的结果,
其中事件 $A$ 包含的结果有 $m$ 种,那么事件 $A$ 的概率 $P(A)$ 是 $\dfrac{m}{n}$。


四、不可能同时发生的两个事件叫做互斥事件。当 $A,\, B$ 是互斥事件时,
$$ P(A + B) = P(A) + P(B) \text{。} $$

如果一个事件是否发生对另一个事件发生的概率没有影响,那么这两个事件叫做相互独立事件。
当 $A,\, B$ 是相互独立事件时,
$$ P(A \cdot B) = P(A) \cdot P(B) \text{。} $$

应注意上面两个概念的区别,注意运用上面两个公式的前提条件。

其中必有一个发生的两个互斥事件叫做对立事件。
当 $A,\, B$ 是对立事件时,$P(B) = 1 - P(A)$。
利用这个公式,常可使概率的计算简化。


五、如果事件 $A$ 在一次试验中发生的概率是 $P$,那么它在 $n$ 次独立重复试验中恰好发生 $k$ 次的概率是
$$ P_n(k) = C_n^k P^k (1 - P)^{n-k} \text{。} $$

