\subsection{基本原理}\label{subsec:2-1}

我们先看下面的问题:

从甲地到乙地,可以乘火车,也可以乘汽车,还可以乘轮船。一天中,火车有 $4$ 班,汽车有 $2$ 班,
轮船有 $3$ 班。那么一天中乘坐这些交通工具从甲地到乙地共有多少种不同的走法?

因为一天中乘火车有 $4$ 种走法,乘汽车有 $2$ 种走法,乘轮船有 $3$ 种走法,每一种走法都可以
从甲地到达乙地,因此,一天中乘坐这些交通工具从甲地到乙地共有
$$ 4 + 2 + 3 = 9 $$
种不同的走法。

一般地,有如下原理:

\textbf{加法原理\mylabel{theorem:jiafa}[加法原理] \quad 做一件事,完成它可以有 $n$ 类办法,
    在第一类办法中有 $m_1$ 种不同的方法,
    在第二类办法中有 $m_2$ 种不同的方法,……,
    在第 $n$ 类办法中有 $m_n$ 种不同的方法,那么完成这件事共有
    $$ N = m_1 + m_2 + \cdots + m_n $$
    种不同的方法。
}

我们再看下面的问题:

由 $A$ 村去 $B$ 村的道路有 $3$ 条,由 $B$ 村去 $C$ 村的道路有 $2$ 条(图 \ref{fig:2-1})。
从 $A$ 村经 $B$ 村去 $C$ 村,共有多少种不同的走法?

\begin{figure}[htbp]
    \centering
    \begin{tikzpicture}[>=Stealth]
    \coordinate (A) at (0, 0);
    \coordinate (B) at (3, 0);
    \coordinate (C) at (6, 0);

    \draw (A) .. controls (1.5, 1) and (1, 0.5) .. (B);
    \draw (A) .. controls (1, 0.3) and (1.5, -0.3) .. (B);
    \draw (A) .. controls (1.3, -0.6) and (1.5, -0.6) .. (B);
    \node at (2, 0.6) {北};
    \node at (1.2, 0.3) {中};
    \node at (1.5, -0.7) {南};

    \draw (B) .. controls (4.5, 0.5) and (4.5, 0.5) .. (C);
    \draw (B) .. controls (4.5, -0.5) and (4.5, -0.5) .. (C);
    \node at (4.5, 0.6) {北};
    \node at (4.5, -0.6) {南};

    \draw [very thick, fill=white] (A) circle [radius = 0.1] node [below] {$A$村};
    \draw [very thick, fill=white] (B) circle [radius = 0.1] node [below] {$B$村};
    \draw [very thick, fill=white] (C) circle [radius = 0.1] node [below] {$C$村};

\end{tikzpicture}

    \caption{}\label{fig:2-1}
\end{figure}

这里,从 $A$ 村到 $B$ 村有 $3$ 种不同的走法,按这 $3$ 种走法中的每一种走法到达 $B$ 村后,
再从 $B$ 村到 $C$ 村又有 $2$ 种不同的走法。因此,从 $A$ 村经 $B$ 村去 $C$ 村共有
$$ 3 \times 2 = 6 $$
种不同的走法。

一般地,有如下原理:

\textbf{乘法原理\mylabel{theorem:chefa}[乘法原理] \quad
    做一件事, 完成它需要分成 $n$ 个步骤,
    做第一步有 $m_1$ 种不同的方法,
    做第二步有 $m_2$ 种不同的方法,……,
    做第 $n$ 步有 $m_n$ 种不同的方法。那么完成这件事共有
    $$ N = m_1 \times m_2 \times \cdots \times m_n $$
    种不同的方法。
}

\liti 书架上层放有 $6$ 本不同的数学书,下层放有 $5$ 本不同的语文书。

(1) 从中任取一本,有多少种不同的取法?

(2) 从中任取数学书与语文书各一本,有多少种不同的取法?

\jie (1) 从书架上任取一本书,有两类办法:
第一类办法是从上层取数学书,可以从 $6$ 本书中任取一本,有 $6$ 种方法;
第二类办法是从下层取语文书,可以从 $5$ 本书中任取一本,有 $5$ 种方法。
根据加法原理,得到不同的取法的种数是
$$ N = m_1 + m_2 = 6 + 5 = 11 \text{。} $$

\textbf{答:} 从书架上任取一本书,有 $11$ 种不同的取法。

(2) 从书架上任取数学书与语文书各一本,可以分成两个步骤完成:
第一步取一本数学书,有 $6$ 种方法;
第二步取一本语文书,有 $5$ 种方法。
根据乘法原理,得到不同的取法的种数是
$$ N = m_1 \times m_2 = 6 \times 5 = 30 \text{。} $$

\textbf{答:} 从书架上取数学书与语文书各一本,有 $30$ 种不同的方法。

\liti 由数字 $1,\, 2,\, 3,\, 4,\, 5$ 可以组成多少个三位数(各位上的数字允许重复)?

\jie 要组成一个三位数可以分成三个步骤完成:
第一步确定百位上的数字,从 $5$ 个数字中任选一个数字,共有 $5$ 种选法;
第二步确定十位上的数字,由于数字允许重复,这仍有 $5$ 种选法;
第三步确定个位上的数字,同理,它也有 $5$ 种选法。
根据乘法原理得到可以组成的三位数的个数是
$$ N = 5 \times 5 \times 5 = 5^3 = 125 \text{。} $$

答:可以组成 $125$ 个三位数。


\lianxi
\begin{xiaotis}

\xiaoti{(口答)一件工作可以用两种方法完成。有 $5$ 人会用第一种方法完成,另有 $4$ 人会用第二种方法完成。
    选出一个人来成这件作,共有多少种选法?
}

\xiaoti{在读书活动中,一个学生要从 $2$ 本科技书、$2$ 本政治书、$3$ 本文艺书里任选一本,
    共有多少种不同的选法?
}

\xiaoti{一名儿童做加法游戏。
    在一个红口袋中装着 $20$ 张分别标有数 $1,\, 2,\, \cdots,\, 19,\, 20$ 的红卡片,从中任抽一张,把上面的数作为被加数;
    在另一个黄口袋中装着 $10$ 张分别标有数 $1,\, 2,\, \cdots,\, 9,\, 10$ 的黄卡片,从中任抽一张,把上面的数作为加数。
    这名儿童一共可以列出多少个加法式子?
}

\xiaoti{乘积 $(a_1 + a_2 + a_3) (b_1 + b_2 + b_3 + b_4) (c_1 + c_2 + c_3 + c_4 + c_5)$ 展开后共有多少项?}


\xiaoti{如图,从甲地到乙地有 $2$ 条路可通,从乙地到丙地有 $3$ 条路可通;
    从甲地到丁地有 $4$ 条路可通,从丁地到丙地有 $2$条路可通。
    从甲地到丙地共有多少种不同的走法?
    \begin{figure}[htbp]
        \centering
        \begin{tikzpicture}[>=Stealth, scale=0.8]
    \draw [very thick] (0, 4) circle (0.6) node {甲地};
    \draw [very thick] (6, 4) circle (0.6) node {乙地};
    \draw [very thick] (6, 0) circle (0.6) node {丙地};
    \draw [very thick] (0, 0) circle (0.6) node {丁地};

    \draw [->] (0.5, 4.3) -- (5.5, 4.3);
    \draw [->] (0.5, 3.7) -- (5.5, 3.7);

    \draw [->] (5.7, 3.5) -- (5.7, 0.5);
    \draw [->] (6, 3.4) -- (6, 0.6);
    \draw [->] (6.3, 3.5) -- (6.3, 0.5);

    \draw [->] (-0.3, 3.5) -- (-0.3, 0.5);
    \draw [->] (-0.1, 3.4) -- (-0.1, 0.6);
    \draw [->] (0.1, 3.4) -- (0.1, 0.6);
    \draw [->] (0.3, 3.5) -- (0.3, 0.5);

    \draw [->] (0.5, 0.3) -- (5.5, 0.3);
    \draw [->] (0.5, -0.3) -- (5.5, -0.3);
\end{tikzpicture}

        \caption*{(第 5 题)}
    \end{figure}
}

\xiaoti{一个口袋内装有 $5$ 个小球,另一个口袋内装有 $4$ 个小球,所有这些小球的颜色互不相同。}
\begin{xiaoxiaotis}

    \xiaoxiaoti{从两个口袋内任取一个小球,有多少种不同的取法?}

    \xiaoxiaoti{从两个口袋内各取一个小球,有多少种不同的取法?}

\end{xiaoxiaotis}


\xiaoti{如图,从甲地到乙地有 $2$ 条陆路可走,从乙地到丙地有 $3$ 条陆路可走,
    又从甲地不经过乙地到丙地有 $2$ 条水路可走。
    \begin{figure}[htbp]
        \centering
        \begin{tikzpicture}[>=Stealth, scale=0.8]
    \coordinate (A) at (0, 1);
    \coordinate (B) at (4, 0);
    \coordinate (C) at (6, 1.5);

    \draw (A) .. controls (2, 0.5) and (1, 0.5) .. (B);
    \draw (A) .. controls (1.5, 0) and (3, 0) .. (B);

    \draw (A) .. controls (2, 2) and (5, 2) .. (C);
    \draw (A) .. controls (3, 0.5) and (3.5, 1.5) .. (C);

    \draw (B) .. controls (5, 1) and (5, 1) .. (C);
    \draw (B) .. controls (5, 0.5) and (6, 1) .. (C);
    \draw (B) .. controls (5, 0) and (6, 0.5) .. (C);

    \draw [very thick, fill=white] (A) circle (0.1) node[below left] {甲地};
    \draw [very thick, fill=white] (B) circle (0.1) node[below] {乙地};
    \draw [very thick, fill=white] (C) circle (0.1) node[below right] {丙地};
\end{tikzpicture}

        \caption*{(第 7 题)}
    \end{figure}
}
\begin{xiaoxiaotis}

    \xiaoxiaoti{从甲地经乙地到丙地有多少种不同的走法?}

    \xiaoxiaoti{从甲地到丙地共有多少种不同的走法?}

\end{xiaoxiaotis}

\end{xiaotis}

