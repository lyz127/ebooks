\subsection{指数函数}\label{subsec:1-12}

我们来研究下面的问题:

某种细胞分裂时,由1个分裂成2个,2个分裂成4个,……。一个这样的细胞分裂 $x$ 次后,得到的细胞的个数 $y$ 与 $x$ 的函数关系式是
$$y = 2^x \text{。}$$

在这个函数里,自变量 $x$ 出现在指数的位置上,而底数2是一个大于零且不等于1的常量。

一般地,函数 $y = a^x$ 叫做\textbf{指数函数},其中 $a$ 是一个大于零且不等于 $1$ 的常量。函数的定义域是实数集 $R$。
\footnote{$a > 0$,$x$ 是一个无理数时,$a^x$ 是一个确定的实数。对于无理数指数幂,过去学过的有理
    数指数幂的性质和运算法则都适用。有关概念和定理证明在本书中从略。}

现在研究指数函数 $y = a^x$ 的图象和性质。先画出一些指数函数的图象,例如,画出 $y = 2^x$,
$y = 10^x$,$y = \left( \dfrac 1 2 \right)^x$ 的图象。

列出 $x$,$y$ 的对应值表,用描点法画出图象(图\ref{fig:1-28}):

\begin{table}[H]
\renewcommand\arraystretch{2}
\begin{tabular}{|w{c}{5em}|*{9}{w{c}{2em}|}}
    \hline
    $x$ & $\dots$ & $-3$ & $-2$ & $-1$ & $0$ & $1$ & $2$ & $3$ & $\dots$ \\
    \hline
    $y=2^x$ & $\dots$ & $\dfrac 1 8$ & $\dfrac 1 4$ & $\dfrac 1 2$ & $1$ & $2$ & $4$ & $8$ & $\dots$ \\
    \hline
\end{tabular}
\end{table}

\begin{table}[H]
\renewcommand\arraystretch{2}
\begin{tabular}{|w{c}{5em}|*{7}{w{c}{2.9em}|}}
    \hline
    $x$ & $\dots$ & $-1$ & $-\dfrac 1 2$ & $0$ & $\dfrac 1 2$ & $1$ & $\dots$ \\
    \hline
    $y=10^x$ & $\dots$ & $0.1$ & $0.32$ & $1$ & $3.16$ & $10$ & $\dots$ \\
    \hline
\end{tabular}
\end{table}

\begin{table}[H]
\renewcommand\arraystretch{2}
\begin{tabular}{|w{c}{5em}|*{9}{w{c}{2em}|}}
    \hline
    $x$ & $\dots$ & $-3$ & $-2$ & $-1$ & $0$ & $1$ & $2$ & $3$ & $\dots$ \\
    \hline
    $y=\left( \dfrac 1 2 \right)^x$ & $\dots$ & $8$ & $4$ & $2$ & $1$ & $\dfrac 1 2$ & $\dfrac 1 4$ & $\dfrac 1 8$ & $\dots$ \\
    \hline
\end{tabular}
\end{table}

\newpage

\begin{figure}[H]
    \centering
    \begin{tikzpicture}[>=Stealth]
    \draw [->] (-3.5,0) -- (3.5,0) node[anchor=north] {$x$};
    \draw [->] (0,-1.0) -- (0,10.5) node[anchor=east] {$y$};
    \node at (0.3,-0.3) {$O$};
    \node at (0.2,0.6) {$1$};
    \foreach \x in {-3,-2,-1,1,2,3} {
        \draw (\x,0.2) -- (\x,0) node[anchor=north] {$\x$};
    }
    \foreach \y in {2,3,...,10} {
        \draw (0.2,\y) -- (0,\y) node[anchor=east] {$\y$};
    }
    
    \draw[domain=-3.1:3.1,samples=50] plot (\x, {2^\x}) +(0.5, -1.3) node {$y = 2^x$};
    \draw[domain=-1.5:0.95,samples=50] plot (\x, {10^\x}) +(0.7, -0.3) node {$y = 10^x$};
    \draw[domain=3.1:-3.1,samples=50] plot (\x, {(1/2)^\x}) +(1.1, -0.3) node {$y = \displaystyle \left(\frac 1 2 \right)^x$};
\end{tikzpicture}

    \caption{}\label{fig:1-28}
\end{figure}

一般地,指数函数 $y = a^x$ 在其底数 $a > 1$ 及 $0 < a < 1$ 这两种情况下的图象和性质如下表所示:

\begin{table}[H]
\begin{tabular}{|c|l|l|}
    \hline
    \multirow{2}{*}{图象} & \makecell[c]{$a > 1$} & \makecell[c]{$0 < a < 1$} \\
    \cline{2-3}
    & \includegraphics{../pic-pdf/zhi-shu-han-shu-1.pdf} & \includegraphics{../pic-pdf/zhi-shu-han-shu-2.pdf} \\
    \hline
    \multirow{4}{*}{性质} & \multicolumn{2}{l|}{(1)$y > 0$;} \\
    \cline{2-3}
    & \multicolumn{2}{l|}{(2)当 $x = 0$ 时,$y = 1$;} \\
    \cline{2-3}
    &  \makecell[l]{(3)当 $x>0$ 时,$y>1$,\\ \hspace{2em} $x<0$ 时,$0<y<1$ ;}  & \makecell[l]{(3)当 $x>0$ 时,$0<y<1$,\\ \hspace{2em} $x<0$ 时,$y>1$ ;} \\
    \cline{2-3}
    & (4)在 $(-\infty, +\infty)$ 上是增函数。 & (4)在 $(-\infty, +\infty)$ 上是减函数。 \\
    \hline
\end{tabular}
\end{table}

\liti 一种放射性物质不断变化为其他物质,每经过一年剩留的质量约是原来的 $84\%$。画出这种物质的剩留量
随时间变化的图象,并从图象上求出约经过多少年,剩留量是原来的一半(结果保留一个有效数字)。

\jie 设最初的质量是 $1$,经过 $x$ 年,剩留量是 $y$。则
经过 $1$年,$y = 1 \times 84\% = 0.84^1$;
经过 $2$年,$y = 0.84 \times 0.84 = 0.84^2$。
一般地,经过 $x$ 年则 $y = 0.84^x$。这就是所求的函数关系式。据此可以列出下表:

\begin{table}[H]
\begin{tabular}{|w{c}{5em}|*{7}{w{c}{2em}|}}
    \hline
    $x$ & $0$ & $1$ & $2$ & $3$ & $4$ & $5$ & $6$ \\
    \hline
    $y$ & $1$ & $0.84$ & $0.71$ & $0.59$ & $0.50$ & $0.42$ & $0.35$ \\
    \hline
\end{tabular}
\end{table}

画出指数函数 $y = 0.84^x$ 的图象(图 \ref{fig:1-29})。从图上看出 $y = 0.5$ 必须并且只需 $x \approx 4$。

\begin{figure}[htbp]
    \centering
    \begin{tikzpicture}[>=Stealth]
    \draw [->] (-0.5,0) -- (6.5,0) node[anchor=north] {$x$};
    \draw [->] (0,-0.5) -- (0,5.5) node[anchor=east] {$y$};
    \node at (0.3,-0.3) {$O$};
    \foreach \x in {1,2,...,6} {
        \draw (\x,0.2) -- (\x,0) node[anchor=north] {$\x$};
    }
    \foreach \y in {0.2,0.4,0.6,0.8,1} {
        \draw (0.2,\y*5) -- (0,\y*5) node[anchor=east] {\y};
    }
    
    \draw[domain=0:6.3,samples=50] plot (\x, {0.84^\x * 5}) +(-3, +2.3) node {$y = 0.84^x$};
    \draw[dash pattern=on 5mm off 2mm] (0, 2.5) -- (6, 2.5);
\end{tikzpicture}

    \caption{}\label{fig:1-29}
\end{figure}

答:约经过 $4$ 年,剩留量是原来的一半。

\liti 比较下列各题中两个值的大小:

\begin{xiaoxiaotis}

    \hspace{1em}\twoInLine[10em]{\xiaoxiaoti{$1.7^{2.5}$,$1.7^3$;}}{\xiaoxiaoti{$0.8^{-0.1}$,$0.8^{-0.2}$。}}

\end{xiaoxiaotis}

\jie 分别考察指数函数 $y = 1.7^x$ 与 $y = 0.8^x$,根据指数函数的性质知道:

(1) $\because$ \quad $1.7 > 1$,$2.5 < 3$,

$\therefore \quad 1.7^{2.5} < 1.7^3$;

(2) $\because$ \quad $0.8 < 1$,$-0.1 > -0.2$,

$\therefore \quad 0.8^{-0.1} < 0.8^{-0.2}$。

\lianxi

\begin{xiaotis}

\xiaoti{在同一坐标系内,画出下列函数的图象:}
\begin{xiaoxiaotis}

    \vspace{0.5em}
    \twoInLine[10em]{\xiaoxiaoti{$y = 3^x$;}}{\xiaoxiaoti{$y = \left( \dfrac 1 3 \right)^x$。}}
    \vspace{0.5em}

\end{xiaoxiaotis}

\xiaoti{一片树林中现有木材 $30000 \text{米}^3$,如果每年增长$5\%$,经过 $x$年,树林中有木材 $y\text{米}^3$,
    写出 $x$,$y$ 间的函数关系式,并且利用图象求约经过多少年,木材可以增加到 $40000 \text{米}^3$ (结果保留一个有效数字)。}

\xiaoti{比较下面各题中两个值的大小:}
\begin{xiaoxiaotis}

    \renewcommand\arraystretch{1.5}
    \begin{tabular}[t]{*{2}{@{}p{16em}}}
        \xiaoxiaoti {$3^{0.8}$,$3^{0.7}$;} & \xiaoxiaoti {$0.75^{-0.1}$,$0.75^{0.1}$;} \\
        \xiaoxiaoti {$1.01^2$,$1.01^{3.5}$;} & \xiaoxiaoti {$0.99^3$,$0.99^{4.5}$。}
    \end{tabular}

\end{xiaoxiaotis}

\end{xiaotis}
