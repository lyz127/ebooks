\subsection{正弦函数、余弦函数的象和性质}\label{subsec:2-8}

我们利用单位圆中的正弦线、余弦线来作正弦函数、余弦函数的图象。

\begin{figure}[htbp]
    \centering
    \begin{tikzpicture}[>=Stealth, scale=0.8]
    % 绘制两组坐标轴
    \foreach \y in {0, 4} {
        \draw [->] (-3.5, \y) -- (7.5, \y) node[anchor=north] {$x$};
        \draw [->] (0, \y-1.5) -- (0, \y+1.9) node[anchor=east] {$y$};
        \node [font=\footnotesize] at (-0.2, \y-0.3) {$O$};
        \node [font=\footnotesize] at (0.2, \y+1.2) {$1$};
        \node [font=\footnotesize] at (0.3, \y-1.2) {$-1$};
    }

    % 绘制左上角的圆 及 圆内的分隔线
    \pgfmathsetmacro{\base}{4}
    \coordinate (O1) at (-2, \base);
    \draw (O1) circle(1);
    \foreach \id in {1, ..., 12} {
        \draw (O1) -- ++(30 *\id:1) coordinate (A\id);
    }
    \draw (O1) +(0.3, -0.3) node [fill=white, inner sep = 1pt, font=\footnotesize] {$O_1$};
    \node at (A12) [anchor=south west, inner sep = 1pt, font=\footnotesize] {$A$};

    %-------------------------------------
    % 绘制 sin 函数 及其相关连线
    \coordinate (sinx) at (0, \base);
    \draw[domain=0:2*pi,samples=100,name path=sin] plot (\x, {\base + sin(\x r)}) +(-1, 1.3) node {$y = \sin x \quad x \in [0, 2\pi]$};
    \foreach \id in {4, 5, 7, 8} {
        \path [name path=pc] (A\id) -- +(10, 0);
        \path [name intersections={of=pc and sin, by={a, b}}];
        \draw (a) -- (a |- sinx); % 第二种绘制方式: |-   (perpendicular)
        \draw (b) -- (b |- sinx);
        \draw [dashed] (A\id) -- (b);
    }
    \draw [dashed] (A3) node [anchor=south, font=\footnotesize] {$B$} -- (pi/2, \base+1);
    \draw (pi/2, \base + 1) node [anchor=south, font=\footnotesize] {$(B)$} -- (pi/2, \base) node [anchor=north, fill=white, inner sep = 1pt] {$\frac \pi 2$} +(0.5, 0.25) node [fill=white, inner sep = 1pt, font=\footnotesize]{$(O_1)$};
    \node [anchor=south] at (pi, \base) {$\pi$};
    \draw (3*pi/2, \base - 1) -- (3*pi/2, \base) node [anchor=south] {$\frac{3\pi}{2}$};
    \node [anchor=south, font=\footnotesize] at (2*pi, \base) {$2\pi$};
    \draw [dashed] (A9) -- (3*pi/2, \base-1);

    %-------------------------------------
    % 绘制 cos 函数 及其相关连线
    \draw [name path=line] (-3.5,-1.5) -- (-0.5,1.5);
    \node [font=\footnotesize] at (-1.8, -0.3) {$O_1$};

    \coordinate (cosx) at (0,0);
    \draw[domain=0:2*pi,samples=100,name path=cos] plot (\x, {cos(\x r)}) +(-1, -2.5) node {$y = \cos x \quad x \in [0, 2\pi]$};

    \foreach \id in {1, 2, 3, 4, 5} {
        \path [name path=pl] (A\id) -- (A\id |- 0,-2);
        \path [name intersections={of=pl and line, by=l1}];
        \path [name path=pc] (l1) -- +(10, 0);
        \draw [name intersections={of=pc and cos, by={c1, c2}}]
            let \p1=(c1), \p2=(c2)
            in (\p1) -- (\x1, 0)   % 第二种绘制方式: let ... in
               (\p2) -- (\x2, 0);
        \draw [dashed] (A\id) -- (l1) -- (c2);
    }

    \foreach \id in {1, 2, 12} {
        \path [name path=pl] (A\id) -- (A\id |- 0,-2);
        \path [name intersections={of=pl and line, by=l1}];
        \draw (l1) -- (l1 |- cosx);
    }

    \foreach \id / \x / \y in { 6/pi/-1, 12/2*pi/1 } {
        \path [name path=pl6] (A\id) -- (A\id |- 0,-2);
        \path [name intersections={of=pl6 and line, by=l6}];
        \draw [dashed] (A\id) -- (l6) -- (\x, \y);
        \draw (\x, \y) -- (\x, 0);
    }

    \draw (-1.30, 0.28) node[fill=white, inner sep=0pt] {$\frac \pi 4$} (-1.6, 0) arc (0:45:0.4);
    \node [anchor=north, font=\footnotesize] at (-0.5, 1.6) {$A'$};
    \node [anchor=north, font=\footnotesize] at (-1, 0) {$A$};
    \node [anchor=south west, fill=white, inner sep=1pt] at (pi/2, 0) {$\frac \pi 2$};
    \node [anchor=south, font=\footnotesize] at (pi, 0) {$\pi$};
    \node [anchor=north] at (3*pi/2, 0) {$\frac{3\pi}{2}$};
    \node [anchor=north, font=\footnotesize] at (2*pi, 0) {$2\pi$};

\end{tikzpicture}

    \caption{}\label{fig:2-19}
\end{figure}

在直角坐标系的 $x$ 轴上任取一点 $O_1$,以 $O_1$ 为圆心作单位圆(见图 \ref{fig:2-19} 的上半部分),
从这个圆与 $x$ 轴的交点 $A$ 起把圆分成 $12$ 等份(等份越多,作出的图象越精确)。过圆上的各分点作
$x$ 轴的垂线,可以得到对应于角 $0$,$\dfrac \pi 6$,$\dfrac \pi 3$,$\dfrac \pi 2$,……,$2\pi$ \vspace{0.5em}
的正弦线及余弦线(例如 $O_1B$对应于角 $\dfrac \pi 2$ 的正弦线)。相应地,再把 $x$ 轴上从 $0$ 到
$2\pi$ 这一段($2\pi \approx 6.28$)分成 $12$ 等份(例如,从原点起向右的第四个点,就是对应于角
$\dfrac \pi 2$ 的点)。把角 $x$ 的正弦线向右平行移动,使得正弦线(是规定了方向的线段)的起点与
$x$ 轴上的点 $x$ 重合(例如,把单位圆中的正弦线 $O_1B$ 向右平行移动,使得 $O_1$ 与 $x$ 轴上的
点 $\dfrac \pi 2$ 重合),再用光滑曲线把这些正弦线的终点连结起来,就得到了正弦函数
$y = \sin x, \, x \in [0, 2\pi)$ 的图象。

为了作出余弦函数 $y = \cos x, \, x \in [0, 2\pi)$ 的图象,我们把坐标系向下平移(见图 \ref{fig:2-19}
的下半部分),过点 $O_1$ 作与 $x$ 轴的正半轴成角 $\dfrac \pi 4$ 的直线,又过余弦线 $O_1A$的
终点 $A$ 作 $x$ 轴的垂线,它与前面所作的直线交于 $A'$。那么,规定了方向的线段 $O_1A$ 与 $AA'$
的长度相等且方向同时为正。这样,我们就把余弦线 $O_1A$ “竖立”起来成为 $AA'$。用同样的方法,将
其他的余弦线也都“竖立”起来。再将它们平移,使起点与 $x$ 轴上的点 $x$ 重合,最后用光滑曲线把这些
竖立起来的线段的终点连结起来,就得到余弦函数 $y = \cos x, \, x \in [0, 2\pi)$ 的图象。

因为终边相同的三角函数值相等,所以正弦函数 $y = \sin x$ 在 $\cdots$,$x \in [-2\pi, 0)$,
$x \in [2\pi, 4\pi)$,$x \in [4\pi, 6\pi)$,$\cdots$ 时的图象,与 $x \in [0, 2\pi)$ 时
的图象的形状完全一样,只是位置不同。余弦函数的情况也相同。我们把 $y = \sin x$,
$y = \cos x$ 在 $x \in [0, 2\pi)$ 时的图象向左和向右平行移动 $2\pi$,$4\pi$,
$\cdots$,就可以得到 $y = \sin x, \, x \in R$ 及 $y = \cos x, \, x \in R$ 的图象(图 \ref{fig:2-20})。

\begin{figure}[htbp]
    \centering
    \begin{tikzpicture}[>=Stealth, scale=0.6]
    % 绘制两组坐标轴
    \pgfmathsetmacro{\base}{4 * pi}
    \foreach \y in {0, 4} {
        \draw [dashed] (-\base-0.3, \y+1) -- (\base+0.3, \y+1);
        \draw [->] (-\base-0.5, \y) -- (\base+0.5, \y) node[anchor=west] {$x$};
        \draw [dashed] (-\base-0.3, \y-1) -- (\base+0.3, \y-1);
        \draw [->] (0, \y-1.5) -- (0, \y+1.9) node[anchor=east] {$y$};
        \node [font=\footnotesize] at (0.3, \y-0.3) {$O$};
        \node [font=\footnotesize] at (0.3, \y+1.3) {$1$};
        \node [font=\footnotesize] at (0.4, \y-1.3) {$-1$};
        \foreach \x / \name in {-4*pi/$-4\pi$, -3*pi/$-3\pi$, -2*pi/$-2\pi$, -1*pi/$-1\pi$, pi/$\pi$, 2*pi/$2\pi$, 3*pi/$3\pi$, 4*pi/$4\pi$} {
            \node [anchor=north, font=\footnotesize] at (\x, \y) {\name};
        }
    }
    \foreach \x in {-3.5, -2.5, ..., 3.5} {
        \draw (\x * pi, 4) -- (\x * pi, 4+0.2);
    }
    \foreach \x in {-4, -3, ..., 4} {
        \draw (\x * pi, 0) -- (\x * pi, 0.2);
    }

    \draw[domain=-4*pi:4*pi,samples=100] plot (\x, {4 + sin(\x r)}) +(-3*pi, 1.6) node {$y = \sin x \quad x \in R$};
    \draw[domain=-4*pi:4*pi,samples=100] plot (\x, {cos(\x r)}) +(-3*pi, 0.6) node {$y = \cos x \quad x \in R$};
\end{tikzpicture}

    \caption{}\label{fig:2-20}
\end{figure}

正弦函数 $y = \sin x, \, x \in R$ 和余弦函数 $y = \cos x, \, x \in R$ 的图象分别叫做\textbf{正弦曲线}和\textbf{余弦曲线}。

\vspace{1em}
\textbf{练 \quad 习}

用描点法作出正弦函数 $y = \sin x, \, x \in [0, 2\pi)$ 和余弦函数 $y = \cos x, \, x \in [0, 2\pi)$ 的图象。
\vspace{2em}

由图 \ref{fig:2-19} 可以看出,下面五个点在确定图象形状时起着关键的作用:
\vspace{0.5em}
$$(0, 0), \, \left(\dfrac \pi 2, 1\right), \, (\pi, 0), \, \left(\dfrac{3\pi}{2}, -1\right), \, (2\pi, 0) \text{。} \vspace{0.5em}$$

这五点描出后,正弦函数 $y = \sin x, \, x \in [0, 2\pi]$ 的图象的形状就基本上确定了;

\vspace{0.5em}
$(0, 1), \, \left(\dfrac \pi 2, 0\right), \, (\pi, -1), \, \left(\dfrac{3\pi}{2}, 0\right), \, (2\pi, 1)$
\vspace{0.5em}这五点描出后,余弦函数 $y = \cos x, \, x \in [0, 2\pi]$ 的图象的形状就基本上确定了。

因此,在精确度要求不太高时,我们常常先描出这五个点,然后用光滑曲线将它们连结起来,
就得到在相应区间内的正弦函数、余弦函数的简图。今后,我们作正、余弦函数的简图,一般
都象这样先找出在确定图象形状时起着关键作用的五个点,然后描点作图。

\liti 作下列函数的简图:
\begin{xiaoxiaotis}

    \xiaoxiaoti{$y = 1 + \sin x, \, x \in [0, 2\pi]$;}

    \xiaoxiaoti{$y = -\cos x, \, x \in [0, 2\pi]$。}

\end{xiaoxiaotis}

\jie (1)列表:

\begin{table}[H]
\renewcommand\arraystretch{2}
\begin{tabular}{|w{c}{5em}|*{5}{w{c}{3em}|}}
    \hline
    $x$ & $0$ & $\dfrac \pi 2$ & $\pi$ & $\dfrac{3\pi}{2}$ & $2\pi$ \\ \hline
    $\sin x$ & $0$ & $1$ & $0$ & $-1$ & $0$ \\ \hline
    $1 + \sin x$ & $1$ & $2$ & $1$ & $0$ & $1$ \\ \hline
\end{tabular}
\end{table}

描点作图(图 \ref{fig:2-21}):

\begin{figure}[H]
    \centering
    \begin{tikzpicture}[>=Stealth]
    \draw [->] (-1.0,0) -- (2*pi +1,0) node[anchor=north] {$x$};
    \draw [->] (0,-1.0) -- (0,2.5) node[anchor=east] {$y$};
    \node at (-0.3,-0.3) {$O$};
    \foreach \x / \name in {
        0.5*pi / $\dfrac \pi 2$,
        pi / $\pi$,
        1.5*pi / $\dfrac {3\pi} 2$,
        2 * pi / $2\pi$
    } {
        \draw (\x,0.2) -- (\x,0) node[anchor=north] {\name};
    }
    \foreach \y in {1,2} {
        \draw (0.2,\y) -- (0,\y) node[anchor=east] {\y};
    }

    \foreach \x / \y in {0 / 1, 0.5*pi / 2, pi / 1, 1.5*pi / 0, 2*pi / 1} {
        \fill (\x, \y) circle[radius=2pt];
    }
    \draw[domain=0:2*pi,samples=100] plot (\x, {1 + sin(\x r)}) +(-1.4, +1.3) node {$y = 1 + \sin x \quad x \in [0, 2\pi]$};
\end{tikzpicture}

    \caption{}\label{fig:2-21}
\end{figure}

(2)列表:

\begin{table}[H]
\renewcommand\arraystretch{2}
\begin{tabular}{|w{c}{5em}|*{5}{w{c}{3em}|}}
    \hline
    $x$ & $0$ & $\dfrac \pi 2$ & $\pi$ & $\dfrac{3\pi}{2}$ & $2\pi$ \\ \hline
    $\cos x$ & $1$ & $0$ & $-1$ & $0$ & $1$ \\ \hline
    $-\cos x$ & $-1$ & $0$ & $1$ & $0$ & $-1$ \\ \hline
\end{tabular}
\end{table}

描点作图(图 \ref{fig:2-22}):

\begin{figure}[H]
    \centering
    \begin{tikzpicture}[>=Stealth]
    \draw [->] (-1.0,0) -- (2*pi +1,0) node[anchor=north] {$x$};
    \draw [->] (0,-1.5) -- (0,1.5) node[anchor=east] {$y$};
    \node at (-0.3,-0.3) {$O$};
    \foreach \x / \name in {
        0.5*pi / $\dfrac \pi 2$,
        pi / $\pi$,
        1.5*pi / $\dfrac {3\pi} 2$,
        2 * pi / $2\pi$
    } {
        \draw (\x,0.2) -- (\x,0) node[anchor=north] {\name};
    }
    \foreach \y in {-1,1} {
        \draw (0.2,\y) -- (0,\y) node[anchor=east] {\y};
    }

    \foreach \x / \y in {0 / -1, 0.5*pi / 0, pi / 1, 1.5*pi / 0, 2*pi / -1} {
        \fill (\x, \y) circle[radius=2pt];
    }
    \draw[domain=0:2*pi,samples=100] plot (\x, {-cos(\x r)}) +(-1.4, 2.5) node {$y = -\cos x \quad x \in [0, 2\pi]$};
\end{tikzpicture}

    \caption{}\label{fig:2-22}
\end{figure}

下面来研究正弦函数 $y = \sin x$ 和余弦函数 $y = \cos x$ 的主要性质。

(1)\textbf{定义域} \mylabel{xingzhi:sincos-1}

\textbf{函数 $y = \sin x$ 及 $y = \cos x$ 的定义域都是 $(-\infty, +\infty)$。}

(2)\textbf{值域} \mylabel{xingzhi:sincos-2}

因为在单位圆中,正弦线、余弦线的长都是等于或小于半径的长 $1$ 的,所以 $|\sin x| \leqslant 1$,
$|\cos x| \leqslant 1$,即 $-1 \leqslant \sin x \leqslant 1$,$-1 \leqslant \cos x \leqslant 1$。
\textbf{函数 $y = \sin x, \, x\in R$ 及 $y = \cos x, \, x\in R$ 的值域都是 $[-1, 1]$。}

\textbf{函数 $y = \sin x$
在 $x = \dfrac \pi 2 + 2k\pi, \, k \in Z$ 时取最大值 $y = 1$;
在 $x = -\dfrac \pi 2 + 2k\pi, \, k \in Z$ 时取最小值 $y = -1$。}

\textbf{函数 $y = \cos x$
在 $x = 2k\pi, \, k \in Z$ 时取最大值 $y = 1$;
在 $x = (2k + 1)\pi, \, k \in Z$ 时取最小值 $y = -1$。}

(3)\textbf{周期性} \mylabel{xingzhi:sincos-3}

由诱导公式 $\sin(x + 2k\pi) = \sin x$,$\cos(x + 2k\pi) = \cos x \, (k \in Z)$ 知道,
正弦函数值、余弦函数值是按照一定的规律不断重复出现的,这是正弦函数和余弦函数的重要性质。

一般地,对于函数 $y = f(x)$,如果存在一个不为零的常数 $T$,使得当 $x$ 取定义域内的每一个值时,
$$f(x + T) = f(x)$$
都成立,那么就把函数 $y = f(x)$ 叫做\textbf{周期函数},不为零的常数 $T$ 叫做这个函数的\textbf{周期}。
例如,对于正弦函数 $\sin x, \, x \in R$ 来说,$2\pi$,$4\pi$,$\cdots$,$-2\pi$,$-4\pi$,$\cdots$
都是它的周期。一般地,$2k\pi \, (k \in Z \text{,且} k \neq 0)$ 都是它的周期。对于一个周期函数来说,
如果在所有的周期中存在着一个最小的正数,就把这个最小的正数叫做\textbf{最小正周期}。例如,$2\pi$ 是
正弦函数 $\sin x, \, x \in R$ 的所有周期中的最小正数,因而 $2\pi$ 是这个函数的最小正周期。

\textbf{正弦函数 $y = \sin x, \, x \in R$ 和余弦函数 $y = \cos x, \, x \in R$ 都是周期函数,
$2k\pi \, (k \in Z \text{,且} k \neq 0)$ 都是它们的周期,最小正周期是 $2\pi$。}\footnote{这个结论可以证明,本书从略。}

今后我们谈到三角函数的周期时,一般指的是三角函数的最小正周期。

(4)\textbf{奇偶性} \mylabel{xingzhi:sincos-4}

由诱导公式 $\sin(-x) = -\sin x$,$\cos(-x) = \cos x$ 可知,\textbf{
正弦函数 $y = \sin x, \, x \in R$ 是奇函数,
余弦函数 $y = \cos x, \, x \in R$ 是偶函数。}

反映在图象上,\textbf{正弦曲线关于坐标系原点 $O$ 对称,余弦曲线关于 $y$ 轴对称。}

(5)\textbf{单调性} \mylabel{xingzhi:sincos-5}

由正弦曲线可以看出:
当 $x$ 由 $-\dfrac \pi 2$ 增大到 $\dfrac \pi 2$ 时,曲线逐渐上升,$\sin x$ 由 $-1$ 增大到 $1$;
当 $x$ 由 $\dfrac \pi 2$ 增大到 $\dfrac{3\pi} 2$ 时,曲线逐渐下降,$\sin x$ 由 $1$ 减小到 $-1$。
这个变化情况如下表所示

\begin{table}[H]
\renewcommand\arraystretch{2}
\begin{tabular}{|w{c}{5em}|*{9}{w{c}{2em}}|}
    \hline
    $x$ & $-\dfrac \pi 2$ & & $0$ & & $\dfrac \pi 2$ & & $\pi$ & & $\dfrac{3\pi}{2}$ \\ \hline
    $\sin x$ & $-1$ & $\nearrow$ & $0$ & $\nearrow$ & $1$ & $\searrow$ & $0$ & $\searrow$ & $-1$ \\ \hline
\end{tabular}
\end{table}

由正弦函数的周期性知道:

\textbf{正弦函数 $y = \sin x$
在每一个闭区间 $\left[ -\dfrac \pi 2 + 2k\pi,  \dfrac \pi 2 + 2k\pi \right]\, (k \in Z)$ \vspace{0.5em} 上,都从 $-1$ 增大到 $1$,是增函数;
在每一个闭区间 $\left[ \dfrac \pi 2 + 2k\pi,  \dfrac{3\pi} 2 + 2k\pi \right]\, (k \in Z)$ \vspace{0.5em} 上,都从 $1$ 减小到 $-1$,是减函数。}
也就是说,正弦函数 $y = \sin x$ 的单调区间是 $\left[ -\dfrac \pi 2 + 2k\pi,  \dfrac \pi 2 + 2k\pi \right]$
及 $\left[ \dfrac \pi 2 + 2k\pi,  \dfrac{3\pi} 2 + 2k\pi \right]\, (k \in Z)$。\vspace{0.5em}

类似地,由余弦曲线可以看出,函数 $y = \cos x$ 在 $[-\pi, \pi]$ 上的变化情况如下表所示:

\begin{table}[H]
\renewcommand\arraystretch{2}
\begin{tabular}{|w{c}{5em}|*{9}{w{c}{2em}}|}
    \hline
    $x$ & $-\pi$ & & $-\dfrac \pi 2$ & & $0$ & & $\dfrac \pi 2$ & & $\pi$ \\ \hline
    $\cos x$ & $-1$ & $\nearrow$ & $0$ & $\nearrow$ & $1$ & $\searrow$ & $0$ & $\searrow$ & $-1$ \\ \hline
\end{tabular}
\end{table}

由余弦函数的周期性知道:

\textbf{余弦函数 $y = \cos x$
在每一个闭区间 $[(2k - 1)\pi, 2k\pi], \, (k \in Z)$ 上,都从 $-1$ 增大到 $1$,是增函数;
在每一个闭区间 $[2k\pi, (2k + 1)\pi], \, (k \in Z)$ 上,都从 $1$ 减小到 $-1$,是减函数。}
也就是说,余弦函数 $y = \cos x$ 的单调区间是 $[(2k - 1)\pi, 2k\pi]$ 及 $[2k\pi, (2k + 1)\pi], \, (k \in Z)$。

\liti 求使下列函数取得最大值的 $x$ 的集合,并说出最大值是多少。
\begin{xiaoxiaotis}

    \twoInLine[11em]{\xiaoxiaoti{$y = \cos x + 1$;}}{\xiaoxiaoti{$y = \sin 2x$。}}

\end{xiaoxiaotis}

\jie (1)使函数 $y = \cos x$ 取得最大值的 $x$,就是使函数 $y = \cos x + 1$ 取得最大值的 $x$,
因而使 $y = \cos x$ 取得最大值的 $x$ 的集合 $\{ x \mid x = 2k\pi, \, k \in Z \}$,就是使
$y = \cos x + 1$ 取得最大值的 $x$ 的集合。

函数 $y = \cos x + 1$  的最大值是 $1 + 1 = 2$。

(2)令 $z = 2x$,那么使函数 $y = \sin z$ 取得最大值的 $z$ 的集合是
$\{ z \mid z = \dfrac \pi 2 + 2k\pi, \, k \in Z \}$。由
$$2x = z = \dfrac \pi 2 + 2k\pi ,$$
得
$$x = \dfrac \pi 4 + k\pi ,$$
就是说,使得 $y = \sin 2x$ 取得最大值的 $x$ 的集合是
$$\{ x \mid x = \dfrac \pi 4 + k\pi, \, k \in Z \} \text{。}$$

函数 $y = \sin 2x$ 的最大值是 $1$。

\liti 求下列函数的周期:
\begin{xiaoxiaotis}

    \threeInLine[11em]{\xiaoxiaoti{$y = 3\cos x$;}}{\xiaoxiaoti{$y = \sin 2x$;}}{\xiaoxiaoti{$y = 2\sin\left( \dfrac 1 2 x - \dfrac \pi 6 \right)$。}}
    \vspace{0.5em}

\end{xiaoxiaotis}

\jie (1)因为 $\cos x$ 的最小正周期是 $2\pi$,所以当自变量 $x \, (x \in R)$ 增加到 $x + 2\pi$
且必须增加到 $x + 2\pi$ 时,函数 $\cos x$ 的值重复出现,函数 $3\cos x$ 的值也重复出现,因此
$y = 3\cos x$ 的周期(即最小正周期,下同)是 $2\pi$。

(2)把 $2x$ 看成是一个新的变量 $z$,那么 $\sin z$ 的最小正周期是 $2\pi$。就是说,当 $z$ 增加到
$z + 2\pi$ 且必须增加到 $z + 2\pi$ 时,函数 $\sin z$ 的值重复出现。而 $z + 2\pi = 2x + 2\pi = 2(x + \pi)$,
所以当自变量 $x$ 增加到 $x + \pi$ 且必须增加到 $x + \pi$ 时,函数值重复出现,因此 $y = \sin 2x$ 的周期是 $\pi$。

(3)把 $\dfrac 1 2 x - \dfrac \pi 6$ 看成是一个新的变量 $z$,那么 $2\sin z$ 的周期是 $2\pi$,由于
$$z + 2\pi = \left( \dfrac 1 2 x - \dfrac \pi 6 \right) + 2\pi = \dfrac 1 2 (x + 4\pi) - \dfrac \pi 6 ,$$
所以当自变量 $x$ 增加到 $x + 4\pi$ 且必须增加到 $x + 4\pi$ 时,函数值重复出现,因此
$y = 2\sin\left( \dfrac 1 2 x - \dfrac \pi 6 \right)$ 的周期是 $4\pi$。

我们看到,例$3$ 中函数周期的变化仅与自变量 $x$ 的系数有关。一般地,\textbf{函数 $y = A \sin(\omega x + \varphi)$
或 $y = A \cos(\omega x + \varphi)$(其中$A$,$\omega$,$\varphi$ 为常数,且 $A \neq 0$,$\omega > 0$,
$x \in R$)的周期 $T = \dfrac{2\pi}{\omega}$。}

事实上,设 $\omega x + \varphi = z$,那么函数 $A \sin z$ 或 $A \cos z$ 的周期是 $2\pi$,但是
$\omega x + \varphi + 2\pi = \omega(x + \dfrac{2\pi}{\omega}) + \varphi$,所以当自变量 $x$
增加到 $x + \dfrac{2\pi}{\omega}$ 且必须增加到 $x + \dfrac{2\pi}{\omega}$ 时,函数值重复出现,因此函数
$$ y = A \sin(\omega x + \varphi) \quad \text{或} \quad y = A \cos(\omega x + \varphi)$$
的周期 $T = \dfrac{2\pi}{\omega}$。根据这个结论,我们可以由正弦函数式或余弦函数式直接写出它的周期。
如在上面的例3中,(1)的周期是 $2\pi$,(2)的周期是$\dfrac{2\pi}{2} = \pi$,
(3)的周期是 $\dfrac{\, 2\pi \,}{\dfrac 1 2} = 4\pi$。

\liti 不通过求值,指出下列各式大于零,还是小于零。
\begin{xiaoxiaotis}

    \xiaoxiaoti{$\sin\left( -\dfrac{\pi}{18} \right) - \sin\left( -\dfrac{\pi}{10} \right)$;}
    \vspace{0.5em}

    \xiaoxiaoti{$\cos\left( -\dfrac{23}{5} \pi \right) - \cos\left( -\dfrac{17}{4} \pi \right)$。}
    \vspace{0.5em}

\end{xiaoxiaotis}

\jie (1)因为 $-\dfrac \pi 2 < -\dfrac{\pi}{10} < -\dfrac{\pi}{18} < \dfrac \pi 2$,\vspace{0.5em}
且正弦函数 $y = \sin x$ 当 $-\dfrac \pi 2 \leqslant x \leqslant \dfrac \pi 2$ 时是增函数,所以
$$ \sin\left( -\dfrac{\pi}{10} \right) < \sin\left( -\dfrac{\pi}{18} \right) ,$$
即
$$\sin\left( -\dfrac{\pi}{18} \right) - \sin\left( -\dfrac{\pi}{10} \right) > 0 \text{。}$$

(2)$\begin{gathered}[t]
    \cos\left( -\dfrac{23}{5} \pi \right) = \cos \dfrac{23}{5} \pi = \cos \dfrac 3 5 \pi, \\
    \cos\left( -\dfrac{17}{4} \pi \right) = \cos \dfrac{17}{4} \pi = \cos \dfrac 1 4 \pi \text{。}
\end{gathered}$

因为 $0 < \dfrac 1 4 \pi < \dfrac 3 5 \pi < \pi$,且余弦函数 $y = \cos x$ 在
$0 \leqslant x \leqslant \pi$ 上是减函数,所以
$$ \cos \dfrac 3 5 \pi < \cos \dfrac 1 4 \pi , $$
即
$$\begin{gathered}
    \cos \dfrac 3 5 \pi - \cos \dfrac 1 4 \pi < 0, \\
    \therefore \quad \cos\left( -\dfrac{23}{5} \pi \right) - \cos\left( -\dfrac{17}{4} \pi \right) < 0 .
\end{gathered}$$

\lianxi
\begin{xiaotis}

\xiaoti{作下列函数的简图$(x \in [0, 2\pi])$:}
\begin{xiaoxiaotis}

    \threeInLine[11em]{\xiaoxiaoti{$y = -\sin x$;}}{\xiaoxiaoti{$y = 1 + \cos x$;}}{\xiaoxiaoti{$y = 2\sin x$。}}

\end{xiaoxiaotis}

\xiaoti{观察正弦曲线和余弦曲线,写出满足下列条件的 $x$ 的区间:}
\begin{xiaoxiaotis}

    \begin{tabular}[t]{*{3}{@{}p{11em}}}
        \xiaoxiaoti {$\sin x > 0$;} & \xiaoxiaoti {$\sin x < 0$;} \\
        \xiaoxiaoti {$\cos x > 0$;} & \xiaoxiaoti {$\cos x < 0$。}
    \end{tabular}

\end{xiaoxiaotis}

\xiaoti{下列各等式能否成立?为什么?}
\begin{xiaoxiaotis}

    \twoInLine[11em]{\xiaoxiaoti{$2\cos x = 3$;}}{\xiaoxiaoti{$\sin^2 x = 0.5$。}}

\end{xiaoxiaotis}

\xiaoti{求使下列函数取得最小值的 $x$ 的合集,并说出函数的最小值是多少。}
\begin{xiaoxiaotis}

    \twoInLine[11em]{\xiaoxiaoti{$y = 2\sin x$;}}{\xiaoxiaoti{$y = 2 - \cos \dfrac x 3$。}}

\end{xiaoxiaotis}

\xiaoti{等式 $\sin(30^\circ + 120^\circ) = \sin 30^\circ$ 是否成立?如果这个等式成立,
    能不能说 $120^\circ$ 是正弦函数 $y = \sin x$ 的周期?为什么?}

\xiaoti{求下列函数的周期:}
\begin{xiaoxiaotis}

    \renewcommand\arraystretch{1.5}
    \begin{tabular}[t]{*{3}{@{}p{13em}}}
        \xiaoxiaoti {$y = \sin 3x$;} & \xiaoxiaoti {$y = \cos \dfrac x 3$;} \\
        \xiaoxiaoti {$y = 3\sin \dfrac x 4$;} & \xiaoxiaoti {$y = \sin \left( x + \dfrac{\pi}{10} \right)$;} \\
        \xiaoxiaoti {$y = \cos \left( 2x + \dfrac{\pi}{3} \right)$;} & \xiaoxiaoti {$y = \sqrt{3} \sin \left( \dfrac 1 2 x - \dfrac{\pi}{4} \right)$。}
    \end{tabular}

\end{xiaoxiaotis}

\xiaoti{不通过求值,比较下列各组中两个三角函数值的大小:}
\begin{xiaoxiaotis}

    \renewcommand\arraystretch{1.5}
    \begin{tabular}[t]{*{3}{@{}p{13em}}}
        \xiaoxiaoti {$\sin 250^\circ$,} & $\sin 260^\circ$; \\
        \xiaoxiaoti {$\cos \dfrac{15}{8} \pi$,} & $\cos \dfrac{14}{9} \pi$; \\
        \xiaoxiaoti {$\cos 515^\circ$,} & $\cos 530^\circ$; \\
        \xiaoxiaoti {$\sin \left( -\dfrac{54}{7} \pi \right)$,} & $\sin \left( -\dfrac{63}{8} \pi \right)$。
    \end{tabular}

\end{xiaoxiaotis}

\end{xiaotis}
