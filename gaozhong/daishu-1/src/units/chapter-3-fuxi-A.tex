{\centering \nonumsubsection{A \hspace{1em} 组}}

\begin{xiaotis}

\xiaoti{以 $\alpha$ 角的顶点 $O$ 作原点,始边作 $x$ 轴的正半轴,建立直角坐标系。在终边上截取 $OP = 1$,写出点 $P$ 的坐标。}

\xiaoti{在直角坐标系中,两点 $A$,$B$ 的坐标分别为 $A(x_1, y_1)$,$B(x_2, y_2)$,写出 $A$,$B$ 间的距离公式。}

\xiaoti{写出同角三角函数的基本关系式系——倒数关系、商数关系、平方关系。}

\xiaoti{写出 $2k\pi + \alpha \; (k \in Z)$,$-\alpha$,$\pi - \alpha$,$\pi + \alpha$,$2\pi - \alpha$ 的诱导公式,并用一句话加以概括。}

\xiaoti{写出函数 $y = \sin\alpha$ 和 $y = \cos\alpha$ 的定又域和值域,说出它们
    在什么时候取得最大值和最小值,并研究它们的单调性、奇偶性和周期性。}

\jiange
\xiaoti{已知 $\cos\alpha = -\dfrac{9}{41}$,$\alpha \in \left( \pi,\, \dfrac{3\pi}{2} \right)$,求 $\tan\left( \dfrac{\pi}{4} - \alpha \right)$。}\jiange

\xiaoti{如果 $\alpha$,$\beta$ 都是锐角,且 $\sin\alpha = \dfrac{\sqrt{5}}{5}$,$\sin\beta = \dfrac{\sqrt{10}}{10}$,求证 $\alpha + \beta = \dfrac{\pi}{4}$。}\jiange

\xiaoti{}
\begin{xiaoxiaotis}

    \vspace{-1.5em} \begin{minipage}{0.9\textwidth}
    \xiaoxiaoti{已知 $A + B = \dfrac{\pi}{4}$,求证 $(1 + \tan A)(1 + \tan B) = 2$;}
    \end{minipage}\jiange

    \xiaoxiaoti{如果 $A$,$B$ 都是锐角,且 $(1 + \tan A)(1 + \tan B) = 2$,求证 $A + B = \dfrac{\pi}{4}$。}\jiange

\end{xiaoxiaotis}

\xiaoti{如图,三个相同的正方形相接,求证 $\alpha + \beta = 45^\circ$。}

\begin{figure}[htbp]
    \centering
    \begin{tikzpicture}[>=Stealth]
    \draw (0, 0) rectangle (2, 2);
    \draw (2, 0) rectangle (4, 2);
    \draw (4, 0) rectangle (6, 2);

    \draw (0, 0) -- (6,2);
    \draw (1.2, 0.2) node {$\beta$} (0.8, 0) arc (0:18:0.8) (0.9, 0) arc (0:18:0.9);

    \draw (2, 0) -- (6,2);
    \draw (3.0, 0.2) node {$\alpha$} (2.8, 0) arc (0:27:0.8);
\end{tikzpicture}

    \caption*{(第9题)}
\end{figure}

\xiaoti{如果 $\alpha$,$\beta$,$\gamma$ 都是锐角,并且它们的正切依次为 $\dfrac{1}{2}$,$\dfrac{1}{5}$,$\dfrac{1}{8}$,求证 $\alpha + \beta + \gamma = 45^\circ$。}\jiange

\xiaoti{写出 $\dfrac{\pi}{2} - \alpha$,$\dfrac{\pi}{2} + \alpha$,$\dfrac{3\pi}{2} - \alpha$,$\dfrac{3\pi}{2} + \alpha$ 的诱导公式,并用一句话加以概括。} \jiange

\xiaoti{已知 $\cos 2\alpha = \dfrac{3}{5}$,求 $\sin^4\alpha - \cos^4\alpha$ 的值。}\jiange

\xiaoti{已知 $\tan x = \dfrac{7}{24}$,求 $\cos 2x$,$\cot\left( 2x - \dfrac{\pi}{4} \right)$ 的值。}\jiange

\xiaoti{已知 $\sin\theta + \cos\theta = \dfrac{2}{3}$,求 $\sin 2\theta$ 的值。}\jiange

\xiaoti{已知 $\sin\varphi \cos\varphi = \dfrac{60}{169}$,且 $\dfrac{\pi}{4} < \varphi < \dfrac{\pi}{2}$,求 $\sin\varphi$,$\cos\varphi$ 的值。}\jiange

\xiaoti{在等腰三角形 $ABC$ 中,腰为底的 $2$ 倍,求顶角 $A$ 的三角函数的值。}

\xiaoti{化下列各式为和差的形式:}
\begin{xiaoxiaotis}

    \renewcommand\arraystretch{1.5}
    \begin{tabular}[t]{*{2}{@{}p{16em}}}
        \xiaoxiaoti{$2\sin\left( \dfrac{\pi}{4} - x \right) \sin\left( \dfrac{\pi}{4} + x \right)$;} & \xiaoxiaoti{$\sin(n - 1)x \cos(n + 1)x$;} \\
        \xiaoxiaoti{$\cos(m - 1)x \cos(m - 3)x$;} & \xiaoxiaoti{$\dfrac{2\sin(30^\circ + \alpha)}{\cos\alpha}$。}
    \end{tabular}

\end{xiaoxiaotis}

\xiaoti{化下列各式为积的形式:}
\begin{xiaoxiaotis}

    \xiaoxiaoti{$1 + \sin 2x - \cos 2x$;}

    \jiange \xiaoxiaoti{$1 + \cos\theta + \cos\dfrac{\theta}{2}$;}\jiange

    \xiaoxiaoti{$\sin\alpha + \sin 2\alpha + \sin 3\alpha$;}

    \jiange \xiaoxiaoti{$1 - \dfrac{1}{4}\sin^2 2\alpha - \sin^2 \beta - \cos^4 \alpha$。}\jiange

\end{xiaoxiaotis}

\xiaoti{化简:}
\begin{xiaoxiaotis}

    \renewcommand\arraystretch{1.5}
    \begin{tabular}[t]{*{2}{@{}p{16em}}}
        \xiaoxiaoti{$\cos52^\circ30' \cos7^\circ30'$;} & \xiaoxiaoti{$\dfrac{\sin2\alpha}{1 + \cos2\alpha} \cdot \dfrac{\cos\alpha}{1 + \cos\alpha}$;} \\
        \xiaoxiaoti{$\tan67^\circ30' - \tan22^\circ30'$;} & \xiaoxiaoti{$\cos20^\circ - \sin10^\circ - \sin50^\circ$;}
    \end{tabular}

    \jiange
    \xiaoxiaoti{$\sin(x + 60^\circ) + 2\sin(x - 60^\circ) - \sqrt{3}\cos(120^\circ - x)$;}

    \jiange
    \xiaoxiaoti{$\cos\alpha \cdot \csc\alpha \cdot \sqrt{\sec^2\alpha - 1} \quad \left( \dfrac{3\pi}{2} < \alpha < 2\pi \right)$。}\jiange

\end{xiaoxiaotis}

\xiaoti{证明下列各式:}
\begin{xiaoxiaotis}

    \xiaoxiaoti{$\tan20^\circ + \tan40^\circ + \sqrt{3}\tan20^\circ \tan40^\circ = \sqrt{3}$;}

    \jiange \xiaoxiaoti{$\sin x\left( 1 + \tan x \tan\dfrac{x}{2} \right) = \tan x$;}\jiange

    \xiaoxiaoti{$\dfrac{\sin(2\alpha + \beta)}{\sin\alpha} - 2\cos(\alpha + \beta) = \dfrac{\sin\beta}{\sin\alpha}$;}\jiange

    \xiaoxiaoti{$2\sin\alpha + \sin2\alpha = \dfrac{2\sin^3\alpha}{1 - \cos\alpha}$;}\jiange

    \xiaoxiaoti{$\sec\theta = \sqrt{\dfrac{\sec^4\theta - \tan^4\theta}{2\sin^2\theta + \cos^2\theta}} \; \left( 0 < \theta < \dfrac{\pi}{2} \right)$;}\jiange

    \xiaoxiaoti{$\dfrac{\cot^2\alpha + 1}{\cot^2\alpha - 1} = \sec2\alpha$;}\jiange

    \xiaoxiaoti{$\sec\alpha - \tan\alpha = \tan\left(\dfrac{\pi}{4} - \dfrac{\alpha}{2}\right)$;}\jiange

    \xiaoxiaoti{$\dfrac{3 - 4\cos2A + \cos4A}{3 + 4\cos2A + \cos4A} = \tan^4A$;}\jiange

    \xiaoxiaoti{$\dfrac{1 + \cos A + \cos2A + \cos3A}{2\cos^2A + \cos A - 1} = 2\cos A$;}\jiange

    \xiaoxiaoti{$\tan3\theta - \tan2\theta - \tan\theta = \tan3\theta \tan2\theta \tan\theta$。}

\end{xiaoxiaotis}

\xiaoti{求下列函数的最大值与最小值:}
\begin{xiaoxiaotis}

    \begin{tabular}[t]{*{2}{@{}p{16em}}}
        \xiaoxiaoti{$y = \sin3x \cos3x$;} & \xiaoxiaoti{$y = \sin(x - 30^\circ) \cos x$;} \\
        \xiaoxiaoti{$y = \sin x - \sqrt{3}\cos x$;} & \xiaoxiaoti{$y = \sin x + \cos x$;} \\
        \xiaoxiaoti{$y = a\sin x + b$。}
    \end{tabular}

\end{xiaoxiaotis}

\xiaoti{在 $\triangle ABC$ 中,求证:}
\begin{xiaoxiaotis}

    \xiaoxiaoti{$\tan A + \tan B + \tan C = \tan A \cdot \tan B \cdot \tan C$;}

    \jiange\xiaoxiaoti{$\dfrac{\cos A}{\sin B \sin C} + \dfrac{\cos B}{\sin C \sin A} + \dfrac{\cos C}{\sin A \sin B} = 2$。}\jiange

\end{xiaoxiaotis}

\xiaoti{在 $\triangle ABC$ 中,求证:}
\begin{xiaoxiaotis}

    \jiange\xiaoxiaoti{$\dfrac{\cos 2A}{a^2} - \dfrac{\cos 2B}{b^2} = \dfrac{1}{a^2} - \dfrac{1}{b^2}$;}\jiange

    \xiaoxiaoti{$(a^2 - b^2 - c^2)\tan A + (a^2 - b^2 + c^2)\tan B = 0$。}

\end{xiaoxiaotis}

\xiaoti{在 $\triangle ABC$ 中,如果 $2\cos B \cdot \sin C = \sin A$,那么 $\triangle ABC$ 为等腰三角形。}

\xiaoti{$\triangle ABC$ 的三个内角 $A$,$B$,$C$ 的对边分别是 $a$,$b$,$c$,如果 $a^2 = b(b + c)$,求证 $A = 2B$。}

\xiaoti{发电厂发出的电是三相交流电,它的三根导线上的电流强度分别是时间 $t$ 的函数:\\
    $I_A = I \sin \omega t, \quad I_B = I \sin(\omega t + 120^\circ), \quad I_C = I \sin(\omega t + 240^\circ)$。\\
    求证: $I_A + I_B + I_C = 0$。
}

\jiange\xiaoti{已知电流 $i = I_m \sin\omega t$,电压 $v = V_m \sin\left( \omega t + \dfrac{\pi}{2} \right)$,求证电功率 $p = iv = \dfrac{1}{2}V_m I_m \sin 2\omega t$。}\jiange

\xiaoti{如图,在直角三角形 $ABC$ 中,$c$ 是斜边,$r$ 是内切圆半径,}
\begin{figure}[H]
    \centering
    \begin{tikzpicture}[>=Stealth]
    \coordinate [label=180:$A$] (A) at (0, 0);
    \coordinate [label=0:$B$] (B) at (5, 0);
    \coordinate [label=90:$C$] (C) at (9/5, 12/5);
    \draw [name path=ab] (A) -- (B);
    \draw [name path=ac] (A) -- (C);
    \draw [name path=bc] (B) -- (C);

    % 求角A的角平分线
    \coordinate (ma) at (2, 0);
    \path [name path=arcA] (ma) arc (0:90:2);
    \path [name intersections={of=arcA and ac, by={na}}];
    \path [name path=arcMA] (ma)+(2,0) arc (0:90:2);
    \path [name path=arcNA] (na)+(30:2) arc (30:-60:2);
    \path [name intersections={of=arcMA and arcNA, by={pa}}];
    \path [name path=splitA] (A) -- (pa);

    % 求角B的角平分线
    \coordinate (mb) at (3, 0);
    \path [name path=arcB] (mb) arc(180:90:2);
    \path [name intersections={of=arcB and bc, by={nb}}];
    \path [name path=arcMB] (mb)+(-2,0) arc(180:90:2);
    \path [name path=arcNB] (nb)+(120:2) arc (120:220:2);
    \path [name intersections={of=arcMB and arcNB, by={pb}}];
    \path [name path=splitB] (B) -- (pb);

    % 根据两角的平分线,确定内切圆的圆心
    \path [name intersections={of=splitA and splitB, by={O}}];
    \node [anchor=south] at (O) {$O$};

    % 绘制内切圆
    \coordinate [label=270:$D$] (D) at (A -| O);
    \draw (O) let
                \p1 = ($ (O) - (D) $)
              in
                circle ({veclen(\x1,\y1)});

    % 其它
    \draw [dashed] (A) -- (O) -- (B) (O) -- (D);
    \node at (2.2, 0.4) {$r$};
    \node at (3.2, -0.2) {$c$};
\end{tikzpicture}
    \caption*{(第28题)}
\end{figure}
\xiaoti*{求证:}
\begin{xiaoxiaotis}

    \xiaoxiaoti{$r = \dfrac{c}{\cot\dfrac{A}{2} + \cot\left(45^\circ - \dfrac{A}{2}\right)}$;}\jiange

    \xiaoxiaoti{$r \leqslant \dfrac{c}{2}(\sqrt{2} - 1)$。}\jiange

\end{xiaoxiaotis}

\end{xiaotis}
