\chapter{换底公式}\label{app:2}

利用常用对数表,可以求得任意一个正数的以 $10$ 为底的对数。现在我们来
说明以其他正数 $a \; (a \neq 1)$ 为底的对数的求法。例如,求 $\log_3 5$。

设 $log_3 5 = x$,写成指数式,得
$$3^x = 5 \text{。}$$

两边取常用对数,得
$$x\lg3 = \lg5 \text{,}$$

$\therefore$
\begin{minipage}{0.9\textwidth}
$$x = \dfrac{\lg5}{\lg3} = \dfrac{0.6990}{0.4771} = 1.465 \text{,}$$
\end{minipage}\jiange\\
就是
$$\log_3 5 = 1.465 \text{。}$$

一般地,我们有下面的换底公式:
$$\log_b N = \dfrac{\log_a N}{\log_a b} \text{。}$$

\zhengming 设 $\log_b N = x$,写成指数式,得
$$b^x = N \text{。}$$

两边取以 $a$ 为底的对数,得
$$x\log_a b = \log_a N \text{。}$$

$\therefore$
\begin{minipage}{0.9\textwidth}
$$x = \dfrac{\log_a N}{\log_a b} \text{。}$$
\end{minipage}\jiange

所以
$$\log_b N = \dfrac{\log_a N}{\log_a b} \text{。}$$

在科学技术中常常使用以无理数 $e = 2.71828\cdots$ 为底的对数,
以 $e$ 为底的对数叫做\textbf{自然对数},$\log_e N$ 通常记作
$\ln N$。根据对数换底公式,可以得到自然对数与常用对数之间的关系:\jiange
$$\ln N = \dfrac{\lg N}{\lg e} = \dfrac{\lg N}{0.4343} \text{,} \jiange$$
就是
$$\ln N = 2.303 \lg N \text{。}$$

\liti 求 $\log_8 9 \cdot \log_{27} 32$ 的值。

\jiange \jie $\begin{aligned}[t]
    \log_8 9 \cdot \log_{27} 32 &= \dfrac{\lg 9}{\lg 8} \cdot \dfrac{\lg 32}{\lg 27} = \dfrac{2 \; \lg 3}{3 \; \lg 2} \cdot \dfrac{5 \; \lg 2}{3 \; \lg 3}\\
    &= \dfrac{2}{3} \times \dfrac{5}{3} = 1\dfrac{1}{9} \text{。}
\end{aligned}$\jiange

\liti 求证 $\log_x y \cdot \log_y Z = \log_x z$。

\zhengming
\begin{minipage}[t]{0.8\textwidth}
把 $\log_y z$ 化成以 $x$ 为底的对数,则\\
$\begin{aligned}[t]
    & \log_x y \cdot \log_y z \\
    = & \log_x y \cdot \dfrac{\log_x z}{\log_x y} = \log_x z \text{。}
\end{aligned}$
\end{minipage}

\lianxi
\begin{xiaotis}

\xiaoti{利用常用对数表,求下列各对数的值:}
\begin{xiaoxiaotis}

    \renewcommand\arraystretch{1.5}
    \begin{tabular}[t]{*{2}{@{}p{16em}}}
        \xiaoxiaoti{$\log_2 1000$;} & \xiaoxiaoti{$\log_5 0.5$;} \\
        \xiaoxiaoti{$\log_3 10$;} & \xiaoxiaoti{$\log_{\frac{1}{2}} \dfrac{1}{3}$。}
    \end{tabular}
    \jiange
\end{xiaoxiaotis}


\xiaoti{利用常用对数表计算:}
\begin{xiaoxiaotis}

    \begin{tabular}[t]{*{2}{@{}p{16em}}}
        \xiaoxiaoti{$\ln \pi$;} & \xiaoxiaoti{$\ln \dfrac{\sqrt{2}}{2}$;} \\
        \xiaoxiaoti{已知 $\ln x = 2.174$,求 $x$;} & \xiaoxiaoti{已知 $\ln x = -0.7103$,求 $x$。}
    \end{tabular}

\end{xiaoxiaotis}

\xiaoti{求下列各式的值:}
\begin{xiaoxiaotis}

    \twoInLineXxt[16em]{$\ln e^2$;}{$e^{\ln x}$。}

\end{xiaoxiaotis}

\xiaoti{不查表计算下列各题:}
\begin{xiaoxiaotis}

    \xiaoxiaoti{$(\lg 5)^2 + \lg 2 \cdot \lg 50$;}

    \xiaoxiaoti{已知 $\lg 2 = 0.3010$,$\lg 7 = 0.8451$,求 $\lg 35$;}

    \jiange\xiaoxiaoti{$\log_2 \dfrac{1}{25} \cdot \log_3 \dfrac{1}{8} \cdot \log_5 \dfrac{1}{9}$。}\jiange

\end{xiaoxiaotis}

\xiaoti{利用换底公式证明:}
\begin{xiaoxiaotis}

    \xiaoxiaoti{$\log_a b = \dfrac{1}{\log_b a}$;}

    \xiaoxiaoti{$(\log_a b)\cdot (\log_b c) \cdot (\log_c a) = 1$。}

\end{xiaoxiaotis}
\end{xiaotis}
