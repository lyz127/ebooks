\subsection{角的概念的推广}\label{subsec:2-1}

% \begin{figure}[htbp]
%     \centering
%     \begin{tikzpicture}[>=Stealth]
    \node at (0,-0.3) {$O$};
    \draw (0,0) -- (3,0) node [anchor=west] {$A$};
    \draw [rotate=135] (0,0) -- (3,0) node [anchor=south] {$B$};
    \draw [->] (0.5,0) arc (0:135:0.5);
    \node at (0.5, 0.5) {$\alpha$};
\end{tikzpicture}

%     \caption{}\label{fig:2-1}
% \end{figure}

\begin{wrapfigure}[8]{r}{4.5cm}
    \centering
    \begin{tikzpicture}[>=Stealth]
    \node at (0,-0.3) {$O$};
    \draw (0,0) -- (3,0) node [anchor=west] {$A$};
    \draw [rotate=135] (0,0) -- (3,0) node [anchor=south] {$B$};
    \draw [->] (0.5,0) arc (0:135:0.5);
    \node at (0.5, 0.5) {$\alpha$};
\end{tikzpicture}

    \vspace{-20pt}
    \caption{}\label{fig:2-1}
\end{wrapfigure}

我们知道,角可以看成是由一条射线绕着它的端点旋转而成的。如图 \ref{fig:2-1},一条射线由原来的位置$OA$
绕着它的端点 $O$ 按逆时针方向旋转到另一位置 $OB$,就形成角 $\alpha$。旋转开始时的射线 $OA$ 叫做角
$\alpha$ 的\textbf{始边},旋转终止时的射线 $OB$ 叫做角 $\alpha$ 的\textbf{终边},射线的端点 $O$
叫做角 $\alpha$ 的顶点。

过去我们所研究的角都是 $0^\circ$ 到 $360^\circ$ 的角。但是,在日常生活中,在生产和科学实验中,还要
经常遇到大于 $360^\circ$ 的角。如图 \ref{fig:2-2}(1)所示,在自行车的车轮按逆时针方向旋转一周的过
程中,$OA$ 形成了 $0^\circ$ 到 $360^\circ$ 的所有的角;在车轮继续旋转第二周的过程中,又形成了
$360^\circ$ 到 $720^\circ$的所有的角;这样下去,可以形成更大的角(如图 \ref{fig:2-2}(2))。

\begin{figure}[htbp]
    \centering
    \begin{minipage}{8cm}
    \centering
    \begin{tikzpicture}[>=Stealth]
    \draw (0, 0) circle [radius = 0.1];
    \draw [name path=c1,line width=3pt] (0, 0) circle [radius = 3];
    
    \path [name path=l1] (0.2,-3) -- (0.2,3);
    \foreach \s in {0,22.5,...,360} {
        \draw [name intersections={of=c1 and l1, by={A, B}}]
            let \p1 = (A), \p2 = (B)
            in [rotate=\s] (\x1, \y1) -- (\x2, \y2);
    }

    \draw (0, 0) -- (4, 0) node[anchor=north] {$A$};
    \draw [rotate=30] (0, 0) -- (4, 0) node[anchor=south] {$B$};
    \draw [->] (3.5, 0) arc (0:30:3.5);
    \node at (3.7, 1) {$\alpha$};
    \node [fill=white,inner sep=0pt] at (-0.5, -0.2) {$O$};
\end{tikzpicture}

    \caption*{(1)}
    \end{minipage}
    \qquad
    \begin{minipage}{8cm}
    \centering
    \begin{tikzpicture}[>=Stealth]
    \draw (0, 0) -- (4, 0) node[anchor=north] {$A$};
    \draw [rotate=30] (0, 0) -- (4, 0) node[anchor=south] {$B$};
    \node [fill=white,inner sep=0pt] at (-0.2, -0.2) {$O$};

    \draw [->] (3.5, 0) arc (0:30:3.5);
    \node at (3.7, 1) {$\alpha$};

    \path [->] (0,0) pic {luoxuan={0.5, 0.5, 0, 750}};
    \node at (0, 1.5) {$750^\circ$};
\end{tikzpicture}

    \caption*{(2)}
    \end{minipage}
    \caption{}\label{fig:2-2}
\end{figure}

在实际生活中,我们还看到角的形成可以按照两种相反的旋转方向:逆时针方向和顺时针方向。为了区别起见,我们
把按逆时针方向旋转所形成的角叫做\textbf{正角},
把按顺时针方向旋转所形成的角叫做\textbf{负角}。如在图 \ref{fig:2-3} 中,以 $OA$ 为始边的角
$\alpha = 210^\circ$,$\beta = -150^\circ$,$\gamma = -660^\circ$。
特别地,当一条射线没有作任何旋转时,我们也认为这时形成了一个角,并把这个角叫 \textbf{零角}。

\begin{figure}[htbp]
    \centering
    \begin{tikzpicture}[>=Stealth]
    \draw (0, 0) -- (4, 0) node[anchor=north] {$A$};
    \draw [rotate=-150] (0, 0) -- (4, 0) node[anchor=north] {$B_1$};
    \draw [rotate=-660] (0, 0) -- (4, 0) node[anchor=west] {$B_2$};
    \node [fill=white,inner sep=0pt] at (0.2, -0.2) {$O$};

    \draw [->] (0.5, 0) arc (0:210:0.5);
    \node at (-0.4, 0.8) {$\alpha = 210^\circ$};
    
    \draw [->] (1.0, 0) arc (0:-150:1.0);
    \node at (0.5, -0.6) {$\beta = -150^\circ$};

    \path [->] (0,0) pic {luoxuan={1.5, -0.5, 0, -660}};
    \node at (0, 2.6) {$\gamma = -660^\circ$};
\end{tikzpicture}

    \caption{}\label{fig:2-3}
\end{figure}

角的概念经这样推广以后,它包括任意大小的正角、负角和零角。

今后我们常在直角坐标系内讨论角,使角的顶点与坐标原点重合,角的始边在 $x$ 轴的正半轴上。
角的终边在第几象限,就说这个角是第几象限的角(或说这个角属于第几象限)。如
图 \ref{fig:2-4}(1)中的 $30^\circ$,$390^\circ$,$-330^\circ$ 的角都是第一象限的角;
图 \ref{fig:2-4}(2)中的 $300^\circ$,$-60^\circ$ 的角都是第四象限的角;
$585^\circ$ 的角是第三象限的角。如果角的终边在坐标轴上,就认为这个角不属于任何象限。

\begin{figure}[htbp]
    \centering
    \begin{minipage}{8cm}
    \centering
    \begin{tikzpicture}[>=Stealth]
    \draw [->] (-2.5,0) -- (2.5,0) node[anchor=north] {$x$};
    \draw [->] (0,-2.5) -- (0,2.5) node[anchor=east] {$y$};
    \node at (0.3,-0.3) {$O$};

    \draw [rotate=30] (0, 0) -- (2.5, 0) node[anchor=south] {$B_1$};

    \path [->] (0,0) pic {luoxuan={0.5, 0.5, 0, 390}};
    \node at (-1.1,0.5) {$390^\circ$};

    \draw [->] (1.3, 0) arc (0:30:1.3);
    \node at (1.6, 0.4) {$30^\circ$};

    \draw [->] (1.8, 0) arc (0:-330:1.8);
    \node at (-2.0, 1.2) {$-330^\circ$};
\end{tikzpicture}

    \caption*{(1)}
    \end{minipage}
    \qquad
    \begin{minipage}{8cm}
    \centering
    \begin{tikzpicture}[>=Stealth]
    \draw [->] (-2.5,0) -- (2.5,0) node[anchor=north] {$x$};
    \draw [->] (0,-2.5) -- (0,2.5) node[anchor=east] {$y$};
    \node at (0.3,-0.2) {$O$};

    \draw [rotate=300] (0, 0) -- (2.5, 0) node[anchor=west] {$B_2$};
    \draw [rotate=225] (0, 0) -- (2.5, 0) node[anchor=east] {$B_3$};

    \path [->] (0,0) pic {luoxuan={0.5, 0.5, 0, 585}};
    \node at (-0.7,1.3) {$585^\circ$};

    \draw [->] (1.3, 0) arc (0:-60:1.3);
    \node at (1.6, -0.6) {$-60^\circ$};

    \draw [->] (1.8, 0) arc (0:300:1.8);
    \node at (-0.5, -2) {$300^\circ$};
\end{tikzpicture}

    \caption*{(2)}
    \end{minipage}
    \caption{}\label{fig:2-4}
\end{figure}

从图 \ref{fig:2-4}(1)中看到,$390^\circ$,$-330^\circ$ 的角都与 $30^\circ$ 的角的
终边相同。$390^\circ$,$-330^\circ$ 可以分别写成下列形式:
$$360^\circ + 30^\circ; \qquad -360^\circ + 30^\circ \text{。}$$

显然,除了这两个角以外,与 $30^\circ$ 的角终边相同的角还有:

\begin{center}
    \begin{tabular}[t]{*{2}{@{}c}}
        $2 \times 360^\circ + 30^\circ$ ; & \qquad $-2 \times 360^\circ + 30^\circ$ ; \\
        $3 \times 360^\circ + 30^\circ$ ; & \qquad $-3 \times 360^\circ + 30^\circ$ ; \\
        $\cdots$ ; & $\cdots$ 。
    \end{tabular}
\end{center}

所有与 $30^\circ$ 的角终边相同的角,连同 $30^\circ$ 的角在内(而且只有这样的角),可以用下式来表示:
$$k \cdot 360^\circ + 30^\circ , \, k \in Z \text{。}$$

当 $k = 0$ 时,它表示 $30^\circ$ 的角;
当 $k = 1$ 时,它表示 $390^\circ$ 的角;
当 $k = -1$ 时,它表示 $-330^\circ$ 的角,等等。

一般地,\textbf{所有与 $\alpha$ 角終边同的角,连同 $\alpha$ 角在内(而且只有这样的角),可以用式子
$k \cdot 360^\circ + \alpha , \, k \in Z$ 来表示。}

由此可见,对于给定的顶点、始边和终边,确定了一个由无限个角组成的集合。
与 $\alpha$ 角终边相同的角的集合可记作:
$$\{ \beta | \beta = k \cdot 360^\circ + \alpha, \, k \in Z \} \text{。}$$

\liti 在 $0^\circ$ \~{} $360^\circ$ 间
\footnote{本书规定,在  $0^\circ$ \~{} $360^\circ$ 间的 $\alpha$ 角,是指 $0^\circ \leqslant \alpha < 360^\circ$。},
找出与下列各角终边相同的角,并判定下列各角是哪个象限的角。
\begin{xiaoxiaotis}

    \threeInLine{\xiaoxiaoti{$-120^\circ$;}}
        {\xiaoxiaoti{$640^\circ$;}}
        {\xiaoxiaoti{$-950^\circ 12'$。}}

\end{xiaoxiaotis}

\jie (1) $\because \quad -120^\circ = -360^\circ + 240^\circ$,

$\therefore$ $-120^\circ$ 的角与 $240^\circ$ 的角的终边相同,它是第三象限的角;

(2) $\because \quad 640^\circ = 360^\circ + 280^\circ$,

$\therefore$ $640^\circ$ 的角与 $280^\circ$ 的角的终边相同,它是第四象限的角;

(3) $\because \quad -950^\circ 12‘ = -3 \times 360^\circ + 129^\circ 48'$,

$\therefore$ $-950^\circ 12‘$ 的角与 $129^\circ 48'$ 的角的终边相同,它是第二象限的角。

\liti 写出下列各角终边相同的角的集合 $S$,并把 $S$ 中在 $-360^\circ$ \~{} $720^\circ$ 间的角写出来:
\begin{xiaoxiaotis} \setcounter{cntxiaoxiaoti}{0}

    \threeInLine{\xiaoxiaoti{$60^\circ$;}}
        {\xiaoxiaoti{$-21^\circ$;}}
        {\xiaoxiaoti{$363^\circ 14'$。}}

\end{xiaoxiaotis}

\jie (1) $S = \{ \beta | \beta = k \cdot 360^\circ + 60^\circ , \, k \in Z \}$。

$S$ 在 $-360^\circ$ \~{} $720^\circ$ 间的角是
\begin{gather*}
    -1 \times 360^\circ + 60^\circ = -300^\circ \text{;} \\
    0 \times 360^\circ + 60^\circ = 60^\circ \text{;} \\
    1 \times 360^\circ + 60^\circ = 420^\circ \text{。}
\end{gather*}

(2) $S = \{ \beta | \beta = k \cdot 360^\circ - 21^\circ , \, k \in Z \}$。

$S$ 在 $-360^\circ$ \~{} $720^\circ$ 间的角是
\begin{gather*}
    0 \times 360^\circ - 21^\circ = -21^\circ \text{;} \\
    1 \times 360^\circ - 21^\circ = 339^\circ \text{;} \\
    2 \times 360^\circ - 21^\circ = 699^\circ \text{。}
\end{gather*}

(3) $S = \{ \beta | \beta = k \cdot 360^\circ + 363^\circ 14', \, k \in Z \}$。

$S$ 在 $-360^\circ$ \~{} $720^\circ$ 间的角是
\begin{gather*}
    -2 \times 360^\circ + 363^\circ 14' = -356^\circ 46' \text{;} \\
    -1 \times 360^\circ + 363^\circ 14' = 3^\circ 14' \text{;} \\
    0 \times 360^\circ + 363^\circ 14' = 363^\circ 14' \text{。}
\end{gather*}

\liti 写出终边在 $y$ 轴上的角的集合。

\begin{figure}[htbp]
    \centering
    \begin{tikzpicture}[>=Stealth]
    \draw [->] (-2.5,0) -- (2.5,0) node[anchor=north] {$x$};
    \draw [->] (0,-2.5) -- (0,2.5) node[anchor=east] {$y$};
    \node at (-0.3,-0.3) {$O$};

    \draw [->] (0.5, 0) arc (0:90:0.5);
    \node at (0.6, 0.6) {$90^\circ$};

    \draw [->] (1.2, 0) arc (0:270:1.2);
    \node at (-1.5, 0.5) {$270^\circ$};
\end{tikzpicture}

    \caption{}\label{fig:2-5}
\end{figure}

\jie 在 $0^\circ$ \~{} $360^\circ$ 间,终边在 $y$ 轴的正半轴上的角为 $90^\circ$,
终边在 $y$ 轴的负半轴上的角为 $270^\circ$(图 \ref{fig:2-5}),因此,终边在 $y$ 轴
的正半轴、负半轴上的所有的角分别是
$$k \cdot 360^\circ + 90^\circ , \qquad k \cdot 360^\circ + 270^\circ , \, k \in Z \text{,}$$
即终边在 $y$ 轴上的所有的角是
$$k \cdot 360^\circ + 90^\circ \text{或} k \cdot 360^\circ + 270^\circ , \, k \in Z \text{。}$$

又 

\begin{align}
    k \cdot 360^\circ + 90^\circ &= 2k \cdot 180^\circ + 90^\circ; \tag{1}\label{eq:2-1} \\
    k \cdot 360^\circ + 270^\circ &= 2k \cdot 180^\circ + 180^\circ + 90^\circ \notag \\
        & = (2k + 1) \cdot 180^\circ + 90^\circ \text{。} \tag{2}\label{eq:2-2}
\end{align}

在 \eqref{eq:2-1} 式等号右边的前一项是 $180^\circ$ 的所有偶数($2k$)倍;
在 \eqref{eq:2-2} 式等号右边的前一项是 $180^\circ$ 的所有奇数($2k + 1$)倍,
因此,它们可以合并为 $180^\circ$ 的所有整数(用 $n$ 来表示)倍。
这样 \eqref{eq:2-1} 式和 \eqref{eq:2-2} 式可以合并写成:
$$n \cdot 180^\circ + 90^\circ , \, n \in Z \text{,}$$
因而终边在 $y$ 轴上的角的集合是
$$S = \{ \beta | \beta = n \cdot 180^\circ + 90^\circ , \, n \in Z \} \text{。}$$

\lianxi
\begin{xiaotis}

\xiaoti{(口答)锐角是第几象限的角?第一象限的角是否都是说角?再就饨角、直角来回答这两个问题。}

\xiaoti{己知角的顶点与直角坐标系的原点重合,始边落在 $x$ 轴的正半轴上,作出下列各角,并指出它们是哪个象限的角:}
\begin{xiaoxiaotis}

    \fourInLine{\xiaoxiaoti{$420^\circ$;}}
        {\xiaoxiaoti{$-75^\circ$;}}
        {\xiaoxiaoti{$855^\circ$;}}
        {\xiaoxiaoti{$-510^\circ$。}}

\end{xiaoxiaotis}

\xiaoti{在 $0^\circ$ \~{} $360^\circ$ 间,找出与下列各角终边相同的角,并指出它们是哪个象限的角:}
\begin{xiaoxiaotis}

    \fourInLine{\xiaoxiaoti{$-54^\circ 18'$;}}
        {\xiaoxiaoti{$395^\circ 8'$;}}
        {\xiaoxiaoti{$-1190^\circ 30'$;}}
        {\xiaoxiaoti{$1563^\circ$。}}

\end{xiaoxiaotis}

\xiaoti{写出与下列各角终边相同的角的集合,并且把集合中在 $-720^\circ$ \~{} $360^\circ$ 间的角写出来:}
\begin{xiaoxiaotis}

    \fourInLine{\xiaoxiaoti{$45^\circ$;}}
        {\xiaoxiaoti{$-30^\circ$;}}
        {\xiaoxiaoti{$1303^\circ 18'$;}}
        {\xiaoxiaoti{$-225^\circ$。}}

\end{xiaoxiaotis}

\end{xiaotis}
