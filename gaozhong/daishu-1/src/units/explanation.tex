
\chapter{说明}

一、本书供六年制中学高中一年级用,每周授课3课时。

二、本书内容包括:幂函数、指数函数和对数函数,三角函数,两角和与差的三角函数。此外,计划安排在初中教学的近似计算法则和換底公式这两项内容,作为本书附录,可以安排在有关章节中进行教学。

三、本书的习题共分三类:练习,习题,复习参考题。

1. 练习\quad 主要供课堂练习用。

2. 习题\quad 主要供课内课外作业用。

3. 复习参考题\quad 在每章之后配备A,B两组复习参考题。
A组题主要供复习本章知识时使用;B组题综合性、灵活性较大,仅供学有余力的学生参考使用。

为了因材施教,使教学更有针对性和灵活性,本书配备的习题和复习参考题A组数量较多,便于教学时根据实际情况选用。

四、本书在编写过程中,曾参考了中小学通用材数学编写组编写的全日制十年制学校高中课本(试用本)《数学》第一册,大部分章、节是以该书为基础编写的。初稿编出后,曾向向省、市、自治区的教研部门、部分师范院校和中学教师征求意见,有的省、市还进行了试教,他们都提出了宝贵的意见。

五、本书由人民教育出版社中小学数学编辑室编写。参加编写工作的有贾云山、蔡上鹤、饶汉昌、李琳等。全书由吕学礼校订。
