\chapter{近似计算的法则}\label{app:1}

在度量的时候,一般只能得到一个近似数。例如,用皮尺量教室一边的长,量得 $6.8$ 米,
实际上这边的长可能比 $6.8$ 略长一些或者略短一些。假定教室一边的长是 $6.82$ 米,
$6.82$ 米就是教室一边长的准确数。这样,由皮尺量得教室一边的长(近似数),就比
教室一边实际长(准确数)少 $0.02$米。

一个近似数和它的准确数的差,叫做这个近似数的\textbf{误差}。

在计算的时候,我们已经知道可以把一个数进行四舍五入得到一个近似数。
由四舍五入得到的一个近似数,它的误差的绝对值不超过这个近似数最末一位的单位的一半。

如果一个近似数的误差的绝对值不超过某一位的单位的一半,从左边第一个不是零的数字起,
到这位数字止,所有的数字都叫做这个近似数的\textbf{有效数字}。如近似数 $56.08$
有四个有效数字 $5$、$6$、$0$、$8$,近似数 $0.0085$ 有两个有效数字 $8$、$5$,
近似数 $0.0390$ 有三个有效数字 $3$、$9$、$0$。

在实际应用中,实数的运算往往取其近似数来进行。近似计算一般采用下面的法则。

\textbf{法则 1. 近似数相加减,所得结果的位数,通常只保留到各个已知数都有的
最后一位为止。已知数中过多的位数,可以先四会五入到这一位的下一位,再进行计算。}

\liti(1)作近似数的加减计算:
$$7.35 - 2.478 - 0.03419 + 18.6 \text{;}\jiange$$

(2)计算:$32 + 14\dfrac{2}{7} + \sqrt{1880}$ (精确到十分位)。\jiange

解:(1)因为 $18.6$ 只精确到十分位,所以结果只要保留到十分位。
可以先把数位过多的各数分别四舍五入到百分位后计算,得出的中间结果
也都保留到百分位。因此,
\begin{align*}
    & 7.35 - 2.478 - 0.03419 + 18.6 \\
    \approx \, & 7.35 - 2.48 - 0.03 + 18.6 \\
    = \, & 23.44 \approx 23.4 \text{。}
\end{align*}

(2)题中各数都是准确数,它们可以精确到任意数位,因为结果只要求精确到十分位,\jiange
所以在计算时,$14\dfrac{2}{7}$ \jiange 和 $\sqrt{1880}$ 只要分别取精确到百分位的
近似值 $14.29$ 和 $43.36$ 就可以了。因此,
\begin{align*}
    32 + 14\dfrac{2}{7} + \sqrt{1880} &\approx 32 + 14.29 + 43.36\\
    &= 89.65 \approx 89.7 \text{。}
\end{align*}

\textbf{法则 2. 近似数相乘除,所得结果的有效数字的个数,通常只保留到
与已知数中有效数字个数最少的一个相同。已知数中过多的有效数字,可以先
四舍五入到比结果应保留的有效数字的个数多一个,再进行计算。}

\liti (1)求近似数 $24.78$ 与 $0.32$ 的积;

(2)求近似数 $7.9$ 除以近似数 $24.78$ 的商;

\jiange(3)作近似数的乘除计算: $\dfrac{80.43 \times 1.05}{24 \times 7.146}$。\jiange


\jie

\twoInLine[16em]{(1)\opmul[voperation=top, voperator=bottom]{24.8}{0.32}}{(2)\longdivision[repeating decimal style = dots,stage=3]{79}{248}}

\twoInLine[16em]{$\therefore \quad 24.8 \times 0.32 \approx 7.9$。}{$\therefore \quad 7.9 \div 24.78 \approx 0.32$。}

(3)因为 $24$ 只有两个有效数字,所以结果只要保留两个有效数字。可以先把有效数字过多的各数分别
四舍五入到有三个有效数字后计算,得出的中间结果也都保留三个有效数字。因此,\jiange
$$ \dfrac{80.43 \times 1.05}{24 \times 7.146} \approx \dfrac{80.4 \times 1.05}{24 \times 7.15} \approx \dfrac{84.4}{172} \approx 0.49 \text{。}\jiange$$

\textbf{法则 3. 近似数平方或开平方,所得结果的有效数字的个数,通常只保留到与底数或被开方数的有效数字的个数相同。}

\liti 作近似数计算:(1)$12.8^2$;(2)$\sqrt{0.049}$。

解:(1) $12.8^2 = 12.8 \times 12.8 \approx 164$。

(2)$\sqrt{0.049} \approx 0.22$。

\textbf{法则 4. 近似数的混合计算,仍按照运算顺序进行计算,计算过程中得出的中间结果,
一般要比按照法则1、2、3进行近似计算应保留的数字多一位。}

\liti 作近似数的计算:
$$3.28 \times 2.15 + 4.8409 \times 2.7$$

\jie $\begin{aligned}[t]
    & 3.28 \times 2.15 + 4.8409 \times 2.7 \\
    \approx \, & 3.28 \times 2.15 + 4.84 \times 2.7 \\
    \approx \, & 7.052 + 13.1 \approx 7.1 + 13.1 \approx 20 \text{。}
\end{aligned}$


注意:1. 在进行计算时,首先要看题中所给的数是近似数还是准确数。
近似数用近似计算法则进行计算,准确数则用一般方法进行计算。

2. 上述法则中所说的近似计算保留数位的方法,只是在一般情况下通常采用的方法,
在实际问题中,也可以根据具体情况,比上述法则所说的多保留或少保留一位数字。

我们知道,在近似计算的计算过程中,由于保留数位的不同或者计算次序的不同,
虽然计算都正确,得数也可能稍有不同,但都应看做是正确的。

3. 有些习题,如果近似计算所涉及的近似数的数字都是很简单的,或者两数相除能除
尽的,开方能开尽的,并且所得结果的位数和用近似计算所得的位数相差不大时,可以
不必用近似计算方法。此外,有些问题的答案,如果用分数或根式来表示比较方便时,
也可用分数或根式来表示,不采用近似计算方法。

\lianxi
\begin{xiaotis}

\xiaoti{计算下列近似数的加、减法:}
\begin{xiaoxiaotis}

    \xiaoxiaoti{$28.5 + 2.974 + 0.06429 + 5.73$;}

    \xiaoxiaoti{$140.0 - 8.3025$;}

    \xiaoxiaoti{$235.0 - 14.012 - 86.1254 + 43.007$;}

    \jiange\xiaoxiaoti{$2 + \sqrt{2} + 3\dfrac{1}{6} + \sqrt{5}$ (精确到百分位)。}\jiange

\end{xiaoxiaotis}

\xiaoti{计算下列近似数的乘、除法:}
\begin{xiaoxiaotis}

    \xiaoxiaoti{$12.7 \times 56.9$;}

    \xiaoxiaoti{$0.078 \times 3.14159265$;}

    \xiaoxiaoti{$7.84 \div 2.46705$;}

    \jiange\xiaoxiaoti{$\dfrac{1.85 \times 64.72 \times 4.0}{17.9 \times 284.3}$。}\jiange

\end{xiaoxiaotis}

\xiaoti{计算下列近似数的平方和平方根:}
\begin{xiaoxiaotis}

    \twoInLineXxt[16em]{$4.87^2$;}{$\sqrt{0.00565}$。}

\end{xiaoxiaotis}

\xiaoti{计算下列近似数的混合运算:}
\begin{xiaoxiaotis}

    \xiaoxiaoti{$3.5 \times 51.2 + 8.25 \times 12.7$;}

    \xiaoxiaoti{$8.64 \div 0.98 - 33.2 \times 0.57$;}

    \xiaoxiaoti{$4.58^2 - \sqrt{165} + 6.72$;}

    \xiaoxiaoti{$(16.7 + 32 - 18.64 + 5.976) \div 0.36$。}

\end{xiaoxiaotis}

\end{xiaotis}
