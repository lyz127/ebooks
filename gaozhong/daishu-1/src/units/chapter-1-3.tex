\subsection{映射}

在初中我们已学习过对应的例子。例如,对于任何一个实数 $a$,数轴上都有唯一的一点 $A$ 和它对应;
坐标平面内的任何一个点 $P$,都有唯一的有序实数对 $(x,y)$ 和它对应。现在我们学习一种特殊的对应——映射。

先看两个集合 $A$,$B$ 的元素之间的一些对应的例子(图\ref{fig:1-5})。为简单起见,这里的 $A$,$B$ 都是有限集。

\begin{figure}[htbp]
    \centering
    \begin{minipage}{7cm}
    \centering
    \begin{tikzpicture}[>=Stealth]
        \draw (0,0) circle [x radius=1cm, y radius=3cm];
        \node at (-0.7,0) {$A$};
        \node at (0,2) {9};
        \node at (0,0) {4};
        \node at (0,-2) {1};
    
        \draw (3,0) circle [x radius=1cm, y radius=3cm];
        \node at (3.7,0) {$B$};
        \node at (3,2.5) {3};
        \node at (3,1.5) {-3};
        \node at (3,0.5) {2};
        \node at (3,-0.5) {-2};
        \node at (3,-1.5) {1};
        \node at (3,-2.5) {-1};
    
        \draw [->] (0.2,2) .. controls(1,2.6) and (1.5, 3) .. (2.8,2.5);
        \draw [->] (0.2,2) .. controls(1,2.1) and (2, 2.4) .. (2.6,1.5);
        \draw [->] (0.2,0) .. controls(1,0.6) and (1.5, 1) .. (2.6,0.5);
        \draw [->] (0.2,0) .. controls(1,0.1) and (2, 0.4) .. (2.6,-0.5);
        \draw [->] (0.2,-2) .. controls(1,-1.4) and (1.5, -1) .. (2.6,-1.5);
        \draw [->] (0.2,-2) .. controls(1,-1.9) and (2, -1.6) .. (2.7,-2.5);
    \end{tikzpicture}
    \caption*{(1)}
    \end{minipage}
    \qquad
    \begin{minipage}{8cm}
    \centering
    \begin{tikzpicture}[>=Stealth]
        \draw (0,0) circle [x radius=1cm, y radius=3cm];
        \node at (-0.7,0) {$A$};
        \node at (0,2) {$30^\circ$};
        \node at (0,0.6) {$60^\circ$};
        \node at (0,-0.6) {$120^\circ$};
        \node at (0,-2) {$150^\circ$};

        \draw (3,0) circle [x radius=1cm, y radius=3cm];
        \node at (3.7,0) {$B$};
        \node [scale=1.5] at (3,2) {$\frac{\sqrt{3}}{2}$};
        \node [scale=1.5] at (3,0.6) {$\frac{1}{2}$};
        \node [scale=1.5] at (3,-0.6) {$-\frac{1}{2}$};
        \node [scale=1.5] at (3,-2) {$-\frac{\sqrt{3}}{2}$};

        \draw [->] (0.4,2) .. controls(1,2.3) and (1.5,2.5) .. (2.5,2);
        \draw [->] (0.4,0.6) .. controls(1,0.9) and (1.5,1.1) .. (2.5,0.6);
        \draw [->] (0.4,-0.6) .. controls(1,-0.3) and (1.5,-0.1) .. (2.5,-0.6);
        \draw [->] (0.4,-2) .. controls(1,-1.7) and (1.5,-1.5) .. (2.5,-1.8);
    \end{tikzpicture}
    \caption*{(2)}
    \end{minipage}

    \begin{minipage}{8cm}
        \centering
        \begin{tikzpicture}[>=Stealth]
            \draw (0,0) circle [x radius=1cm, y radius=3cm];
            \node at (-0.7,0) {$A$};
            \node at (0,2.5) {1};
            \node at (0,1.5) {-1};
            \node at (0,0.5) {2};
            \node at (0,-0.5) {-2};
            \node at (0,-1.5) {3};
            \node at (0,-2.5) {-3};
    
            \draw (3,0) circle [x radius=1cm, y radius=3cm];
            \node at (3.7,0) {$B$};
            \node at (3,2) {1};
            \node at (3,0) {4};
            \node at (3,-2) {9};
    
            \draw [->] (0.2,2.5) .. controls(1,2.6) and (1.5,2.6) .. (2.8,2.1);
            \draw [->] (0.2,1.5) .. controls(1,1.9) and (1.5,1.9) .. (2.8,1.9);
            \draw [->] (0.2,0.5) .. controls(1,0.6) and (1.5,0.6) .. (2.8,0.1);
            \draw [->] (0.2,-0.5) .. controls(1,-0.1) and (1.5,-0.1) .. (2.8,-0.1);
            \draw [->] (0.2,-1.5) .. controls(1,-1.4) and (1.5,-1.4) .. (2.8,-1.9);
            \draw [->] (0.2,-2.5) .. controls(1,-2.1) and (1.5,-2.1) .. (2.8,-2.1);
        \end{tikzpicture}
        \caption*{(3)}
    \end{minipage}
    \caption{}\label{fig:1-5}
\end{figure}

在(1)中,对应法则是“开平方”,即对于集合 $A$ 中的每一个正数 $x$ (如 $x=9$),集合 $B$ 中有两个平方根 $\pm \sqrt{x}$(即 $3$ 与 $-3$)和它对应;
在(2)中,对应法则是“求余弦”,即对于集合 $A$ 中的每一个角 $\alpha$ (如 $\alpha = 120^\circ $),集合 $B$ 中有一个余弦值 $\cos\alpha$(即$-\dfrac{1}{2}$)和它对应;
在(3)中,对应法则是“平方”,即对于集合 $A$ 中的每两个非零整数 $\pm m$(如 $2$ 和 $-2$),集合 $B$ 中有一个平方数 $m^2$ (即 $4$)和它们对应。

一般地,对于一个集合中的一个或几个元素,可以按照某种对应法则,使另一个集合(也可以是原集合)中有一个或几个元素和它对应。

图\ref{fig:1-5}中(2)与(3)这两个对应都有这样的特点:对于第一个集合(即 $A$)中的任何一个元素,第二个集合(即 $B$)中都有唯一的元素和它对应。

一般地,设 $A$,$B$ 是两个集合,如果按照某种对应法则 $f$,对于集合 $A$ 中的任何一个元素,在集合 $B$ 中都有唯一的元素和它对应,这样的对应(包括集合 $A$,$B$ 及从 $A$ 到 $B$ 的对应法则 $f$)叫做从集合 $A$ 到集合 $B$ 的\textbf{映射},记作
$$f: A \to B\text{。}$$

这样,图\ref{fig:1-5}中(2)与(3)这两个对应,都是从集合 $A$ 到集合 $B$ 的映射。又如,设 $A=\{1,2,3,4\}$,$B=\{3,4,5,6,7,8,9\}$,
集合 $A$ 的元素 $x$ 按对应法则“乘2加1”和集合 $B$ 的元素 $2x+1$ 对应,这个对应也是从集合 $A$ 到集合 $B$ 的映射。

对于图\ref{fig:1-5}中(1)这个对应,由于集合 $B$ 中有两个元素 $3$ 与 $-3$ 和集合 $A$ 中的一个元素 $9$ 对应,所以它不是从集合 $A$ 到集合 $B$ 的映射。

如果给定一个从集合 $A$ 到集合 $B$ 的映射,那么,和 $A$ 中的元素 $a$ 对应的 $B$ 中的元素 $b$ 叫做 $a$ 的\textbf{象},$a$ 叫做 $b$ 的 \textbf{原象}。
例如图\ref{fig:1-5}中(2)这个映射,$B$ 中的元素 $\dfrac{1}{2}$ 和 $A$ 中的元素 $60^\circ$ 对应,这里  $\dfrac{1}{2}$ 是 $60^\circ$ 的象, $60^\circ$ 是 $\dfrac{1}{2}$ 的原象。
$A$ 的元素的象的集合是 $\left\{ \dfrac{1}{2}, -\dfrac{1}{2}, \dfrac{\sqrt{3}}{2}, -\dfrac{\sqrt{3}}{2} \right\} ( \subseteq B)$ 。

\lianxi

\begin{xiaotis}

\xiaoti{画图表示从集合 $A$ 到集合 $B$ 的对应(集合 $A$ 各取四个元素),已知:}

\begin{xiaoxiaotis}
    
    \xiaoxiaoti{$A=N$,$B=N$,对应法则是“加倍”(即“乘2”);}

    \xiaoxiaoti{$A=R$,$B=R^+$,对应法则是“取绝对值”;}

    \xiaoxiaoti{$A=\{x \mid x \in R \text{,且} x \neq 0\}$,$B=R$,对应法则是“取倒数”;}

    \xiaoxiaoti{$A=\{x \mid x \in R \text{,且} x<1\}$,$B=\{ \alpha \mid 0^\circ < \alpha < 180^\circ\}$,对应法则是“求正弦值为 $x$ 的三角形内角 $\alpha$”。}

\end{xiaoxiaotis}


\xiaoti{(口答)图\ref{fig:1-5}(1)中,对于正数 $4$,有几个平方根和它对应?
    (2)中,对于 $150^\circ$ 角,有几个余弦值和它对应?
    (3)中,对于整数 $-1$,有几个平方数和它对应?
}

\xiaoti{(口答)在第1题的四个对应中,哪些对应是从集合 $A$ 到集合 $B$ 的映射?}

\xiaoti{在图\ref{fig:1-5}(2)中,元素 $30^\circ$ 的象是什么?元素 $\dfrac 1 2$ 的原象是什么?}

\end{xiaotis}
