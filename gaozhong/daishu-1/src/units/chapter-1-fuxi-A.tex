{\centering \nonumsubsection{A \hspace{1em} 组}}

\begin{xiaotis}

\xiaoti{用列举法写出与下列集合相等的集合:}
\begin{xiaoxiaotis}
    
    \xiaoxiaoti{$A = \{x | x = 9\}$;}

    \xiaoxiaoti{$B = \{x | x \geqslant 1 \text{,且} x \leqslant 2, \, x \in N \}$;}

    \xiaoxiaoti{$C = \{x | x = 1 \text{,或} x = 2 \}$。}

\end{xiaoxiaotis}

\xiaoti{设 $P$ 表示平面内的点,属于下列集合的点组成什么图形?}
\begin{xiaoxiaotis}
    
    \xiaoxiaoti{$\{P | PA = PB\}$($A$,$B$ 是定点);}

    \xiaoxiaoti{$\{P | PO = 3 \text{厘米} \}$($O$ 是定点)。}

\end{xiaoxiaotis}

\xiaoti{设 $A = \{ \text{菱形} \}$,$B = \{ \text{矩形} \}$,求 $A \cap B$。}

\xiaoti{设 $A = \{ \text{过点} M \text{的圆} \}$,$B = \{ \text{过点} P \text{的圆} \}$,求 $A \cap B$。}

\xiaoti{设平面内有三角形 $ABC$,且 $P$ 表示平面内的点,求}
$$\{P | PA = PB\} \cap \{P | PA = PC\} \text{。}$$

\xiaoti{设全集 $I = R$,$A = \{x | x \leqslant 6 \}$,求:}
\begin{xiaoxiaotis}
    
    \begin{tabular}[t]{*{2}{@{}p{16em}}}
        \xiaoxiaoti{$A \cap \kongji$,$A \cup \kongji$;} &  \xiaoxiaoti{$A \cap R$,$A \cup R$;} \\
        \xiaoxiaoti{$\buji{A}$;} & \xiaoxiaoti{$A \cap \buji{A}$,$A \cup \buji{A}$。}
    \end{tabular}

\end{xiaoxiaotis}

\xiaoti{举出符合下列对应法则的例子:}
\begin{xiaoxiaotis}
    
    \xiaoxiaoti{对于一个集合中的几个元素,另一个集合中有一个元素和它们对应;}

    \xiaoxiaoti{对于一个集合中的一个元素,另一个集合中有几个元素和它对应;}

    \xiaoxiaoti{对于一个集合中的一个元素,另一个集合中有且仅有一个元素和它对应。}

\end{xiaoxiaotis}

\xiaoti{举出几个映射的例子,并说明相应于每个映射的象集合及原象集合各是什么。}

\xiaoti{求下列函数的定义域:}
\begin{xiaoxiaotis}
    
    \renewcommand\arraystretch{1.5}
    \begin{tabular}[t]{*{2}{@{}p{16em}}}
        \xiaoxiaoti{$y = \sqrt{4x + 3}$;} &  \xiaoxiaoti{$y = \dfrac{\sqrt{x - 1}}{x + 2}$;} \\
        \xiaoxiaoti{$y = \dfrac{1}{x + 3} + \sqrt{-x} + \sqrt{x + 4}$;} & \xiaoxiaoti{$y = \dfrac{1}{\sqrt{6 - 5x - x^2}}$。}
    \end{tabular}

\end{xiaoxiaotis}

\xiaoti{设 $f(x) = \dfrac{1 + x^2}{1 - x^2}$,求证:}
\begin{xiaoxiaotis}

    \twoInLine[16em]{\xiaoxiaoti{$f(-x) = f(x)$;}}{\xiaoxiaoti{$f\left( \dfrac 1 x \right) = - f(x)$。}}

\end{xiaoxiaotis}

\xiaoti{设 $f(x) = \dfrac{e^x - e^{-x}}{2}$,$g(x) = \dfrac{e^x + e^{-x}}{2}$,求证:}
\vspace{0.5em}
\begin{xiaoxiaotis}

    \xiaoxiaoti{$[g(x)]^2 - [f(x)]^2 = 1$;}

    \xiaoxiaoti{$f(2x) = 2f(x) \cdot g(x)$;}

    \xiaoxiaoti{$g(2x) = [f(x)]^2 + [g(x)]^2$。}

\end{xiaoxiaotis}

\xiaoti{}
\begin{xiaoxiaotis}

    \vspace{-1.7em}
    \begin{minipage}{0.94\textwidth}
        \xiaoxiaoti{当 $n > 0$ 时,幂函数 $y = x^n$ 有哪些共同性质?}

    \end{minipage}

    \xiaoxiaoti{当 $n < 0$ 时,幂函数 $y = x^n$ 有哪些共同性质?}

\end{xiaoxiaotis}

\xiaoti{分下列两种情况写出二次函数 $y = ax^2 + bx + c$ 的单调区间,以及在每一单调区间上,函数是增函数还是减函数。}
\begin{xiaoxiaotis}

    \twoInLine[16em]{\xiaoxiaoti{$a > 0$;}}{\xiaoxiaoti{$a < 0$。}}

\end{xiaoxiaotis}

\xiaoti{求证:在公共的定义域内,}
\begin{xiaoxiaotis}
    
    \xiaoxiaoti{奇函数与奇函数的积是偶函数;}

    \xiaoxiaoti{奇函数与偶函数的积是奇函数;}

    \xiaoxiaoti{偶函数与偶函数的积是偶函数。}

\end{xiaoxiaotis}

\xiaoti{举出几个一一映射的例子,并分别求出它们的逆映射。}

\xiaoti{把下列指数式化为对数式,或对数式化为指数式($a > 0$,且$a \neq 1$):}
\begin{xiaoxiaotis}
    
    \begin{tabular}[t]{*{2}{@{}p{16em}}}
        \xiaoxiaoti{$\log_a N = b$;} &  \xiaoxiaoti{$a^0 = 1$;} \\
        \xiaoxiaoti{$a^1 = a$;} & \xiaoxiaoti{$\log_a \sqrt[3]{a^2} = \dfrac 2 3$。}
    \end{tabular}

\end{xiaoxiaotis}

\xiaoti{写出对数的运算性质:}
\begin{xiaoxiaotis}
    
    \xiaoxiaoti{$\log_a (M \cdot N) =$ \xhx[10em] ;}

    \vspace{0.5em}
    \xiaoxiaoti{$\log_a \dfrac M N =$ \xhx[10em] ;}
    \vspace{0.5em}

    \xiaoxiaoti{$\log_a M^n =$ \xhx[10em] ;}

    \xiaoxiaoti{$\log_a \sqrt[n]{M} =$ \xhx[10em] 。}

\end{xiaoxiaotis}

\xiaoti{写出对数换底公式,并加以证明。}

\xiaoti{求证:}
\begin{xiaoxiaotis}
    
    \xiaoxiaoti{$\log_2 64 = 3 \, \log_8 64$ ;}

    \vspace{0.5em}
    \xiaoxiaoti{$\log_3 81 = \dfrac 4 3 \log_2 8$ ;}
    \vspace{0.5em}

    \xiaoxiaoti{$\dfrac{\log_5 \sqrt{2} \cdot \log_7 9}{\log_5 \dfrac 13 \cdot \log_7 \sqrt[3]{4}} = -\dfrac 3 2$ ;}
    \vspace{0.5em}

    \xiaoxiaoti{$\log_4 8 - \log_{\frac 1 9} 3 - \log_{\sqrt{2}} 4 = -2$ 。}

\end{xiaoxiaotis}

\xiaoti{利用对数计算:}
\begin{xiaoxiaotis}

    \twoInLine[16em]{\xiaoxiaoti{$3.74^{\frac 1 4} \cdot e^{0.24}$;}}{\xiaoxiaoti{$\left( \dfrac{0.034}{127} \right)^2 \times 5^{\ln 3}$。}}
    \vspace{0.5em}

\end{xiaoxiaotis}

\xiaoti{下列函数中哪些互为反函数?在同一坐标系内画出每一对反函数的图象,然后说明各函数的性质:}
\begin{xiaoxiaotis}
    
    \begin{tabular}[t]{*{2}{@{}p{16em}}}
        \xiaoxiaoti{$y = x^4 \, (x \in R)$;} &  \xiaoxiaoti{$y = 4^x \, (x \in R)$;} \\
        \xiaoxiaoti{$y = x^{\frac 1 4} \, (x \in \buji{R^-})$;} & \xiaoxiaoti{$y = \log_4 x \, (x \in R^+)$。}
    \end{tabular}

\end{xiaoxiaotis}

\xiaoti{求下列函数的定义域:}
\begin{xiaoxiaotis}

    \twoInLine[16em]{\xiaoxiaoti{$y = 8^{\frac 1 {2x - 1}}$;}}{\xiaoxiaoti{$y = \sqrt{1 - \left( \dfrac 1 2 \right)^x}$;}}

    \xiaoxiaoti{$y = \log_a(2 - x) \, (a > 0 \text{,且} a \neq 1)$;}
    
    \xiaoxiaoti{$y = \log_a(- x)^2 \, (a > 0 \text{,且} a \neq 1)$。}

\end{xiaoxiaotis}

\xiaoti{设 $x$,$y$ 为非零实数,$a > 0$,且 $a \neq 1$,下列各式哪些成立,哪些不一定成立,为什么?}
\begin{xiaoxiaotis}

    \twoInLine[16em]{\xiaoxiaoti{$\log_a x^2 = 2 \, \log_a x$;}}{\xiaoxiaoti{$\log_a x^2 = 2 \, \log_a |x|$;}}

    \xiaoxiaoti{$\log_a |x \cdot y| = \log_a |x| \cdot \log_a |y|$;}

    \xiaoxiaoti{$\log_a 3 > \log_a 2$。}

\end{xiaoxiaotis}

\xiaoti{解下列方程:}
\begin{xiaoxiaotis}

    \begin{tabular}[t]{*{2}{@{}p{16em}}}
        \xiaoxiaoti{$6^{2x + 4} = 2^{x + 8} \cdot 3^{3x}$;} &  \xiaoxiaoti{$5^x + 5^{x - 1} = 750$;} \\
        \xiaoxiaoti{$9^x = (\sqrt{3})^{x + 2}$;} & \xiaoxiaoti{$4^x - 3 \times 2^x + 2 = 0$;} \\
        \xiaoxiaoti{$4^x - 2 \times 6^x + 9^x = 0$;} & \xiaoxiaoti{$2^x = 3^{x + 1}$ (精确到 $0.01$)。}
    \end{tabular}

\end{xiaoxiaotis}

\xiaoti{解下列方程:}
\begin{xiaoxiaotis}

    \begin{tabular}[t]{*{2}{@{}p{16em}}}
        \xiaoxiaoti{$\log_{\sqrt{x}} 2x = 4$;} &  \xiaoxiaoti{$\log_7 (\log_3 x) = -1$;} \\
        \xiaoxiaoti{$\log_{10} [\log_2(\log_x 25)] = 0$;} & \xiaoxiaoti{$\log_x x^x = 2$;}
    \end{tabular}

    \xiaoxiaoti{$\lg(x-1) + \lg(x-2) = \lg(x+2)$;}
    
    \xiaoxiaoti{$x^{2 \, \lg x} = 10x$。}

\end{xiaoxiaotis}

\xiaoti{设 $1980$ 年底我国人口为 $10$ 亿,查表计算:}
\begin{xiaoxiaotis}
    
    \xiaoxiaoti{如果我国人口每年比上年平均递增 $2\%$,那么到 $2000$ 年底将达到多少(结果保留四个有效数字)。}

    \xiaoxiaoti{要使 $2000$ 年底我国人口不超过 $12$ 亿,那么每年比上一年平均递增率最高是多少(精确到 $0.01\%$)。}

\end{xiaoxiaotis}

\end{xiaotis}
