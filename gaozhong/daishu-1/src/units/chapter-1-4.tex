\subsection{函数}

我们在初中已经学过函数,知道:如果在某变化过程中有两个变量$x$,$y$,并且对于 $x$ 在某个范围内的每一个确定的值,
按照某个对应法则,$y$ 都有唯一确定的值和它对应,那么 $y$ 就是 $x$ 的\textbf{函数},$x$ 叫做\textbf{自变量},
$x$ 的取值范围叫做函数的\textbf{定义域},和 $x$ 的值对应的 $y$ 的值叫做 \textbf{函数值},函数值的集合叫做函数的\textbf{值域}。

从映射的概念可以知道,映射 $f: A \to B$ 包括三个部分:原象集合$A$,象所在的集合 $B$ 以及从$A$到$B$的对应法则$f$。
当集合$A$,$B$都是非空的数的集合,且$B$的每一个元素都有原象时,这样的映射 $f: A \to B$ 就是定义域$A$到值域$B$上的函数。
所以\textbf{函数}是由\textbf{定义域}、\textbf{值域}以及定义域到值域上的\textbf{对应法则}三部分组成的一类特殊的映射。

例如,对于一次函数 $y=3x+2$,函数的定义域是实数集$R$,值域也是$R$,对应法则是“乘3加2”,这个函数是一个$R$到$R$上的映射。

又如,对于二次函数 $y=2x^2+2$,函数的定义域是$R$,值域是 $\{y \mid y \geqslant 2\}$,对应法则是“平方乘2加2”,这个函数是一个$R$到 $\{y \mid y \geqslant 2\}$ 上的映射。

在本书中,将把这类定义域$A$到值域$B$上的特殊的映射 $f : A \to B$ 都叫做函数,并记作
$$y = f(x) \text{。}$$
$x$ 在定义域 $A$ 内取一个确定的值 $a$ 时,对应的函数值记作 $f(a)$。

例如,二次函数 $f(x)=x^2+2x-1$ 在 $x=0$,$x=1$,$x=2$ 时的函数值分别为 $f(0)=-1$,$f(1)=2$,$f(2)=7$。

在同时研究两个或多个函数时,要用不同的符号来表示它们,除 $f(x)$ 外还常用 $F(x)$,$G(x)$,$g(x)$ 等符号。

在研究函数时常常用到区间的概念。

设 $a$,$b$ 是两个实数,而且$a<b$,我们
把满足 $a \leqslant x \leqslant b$ 的实数 $x$ 的集合叫做\textbf{闭区间},表示为$[a,b]$;
把满足 $a < x < b$ 的实数 $x$ 的集合叫做\textbf{开区间},表示为 $(a, b)$;
把满足 $a \leqslant x < b$,$a < x \leqslant b$ 的实数 $x$ 的集合,都叫做\textbf{半开半闭区间},分别表示为$[a,b)$,$(a,b]$。
这里的实数 $a$ 与 $b$ 都叫做相应区间的\textbf{端点}。

实数集 $R$ 也可以用区间表示为 $(-\infty, +\infty)$,“$\infty$”读作“无穷大”,“$-\infty$”读作“负无穷大",
“$+\\infty$"读作“正无穷大"。我们还把满足 $x \geqslant a$,$x > a$,$x \leqslant b$,$x < b$ 的实数 $x$
的集合分别表示为 $[a, +\infty)$,$(a, +\infty)$,$(-\infty, b]$,$(-\infty, b)$。

我们知道,正比例函数和一次函数的图象都是一条直线,二次函数的图象是一条平滑的曲线(抛物线),
反比例函数的图象是两支平滑的曲线(双曲线)。此外,函数的图象也可以是一些点或几条线段等。

\liti 某种茶杯,每个 0.5 元,买 $x$ 个茶杯的钱数(元)
$$f(x) = 0.5x,\, x\in \buji{Z^-} \text{。}$$
画出这个函数的图象。

\jie 这个函数的图象由一些点组成,如图\ref{fig:1-6}所示

\begin{figure}[htbp]
    \centering
    \begin{minipage}{7cm}
    \centering
    \begin{tikzpicture}[>=Stealth]
        \draw [->] (-0.5,0) -- (4.8,0) node[anchor=north] {$x$(个)};
        \draw [->] (0,-0.5) -- (0,3.5) node[anchor=east] {$y$(元)};
        \fill(0,0) circle(3pt) node[anchor=north east] {$O$};
        \foreach \x in {1,...,4} {
            \draw (\x,0.2) -- (\x,0) node[anchor=north] {$\x$};
            \fill (\x,\x /2) circle (3pt);
        }
        \foreach \y in {1,...,3}
            \draw (0.2,\y) -- (0,\y) node[anchor=east] {$\y$};
    \end{tikzpicture}
    \caption{}\label{fig:1-6}
    \end{minipage}
    \qquad
    \begin{minipage}{8cm}
    \centering
    \begin{tikzpicture}[>=Stealth]
        \draw [->] (-0.5,0) -- (4.8,0) node[anchor=north] {$x$(克重)};
        \draw [->] (0,-0.5) -- (0,3.5) node[anchor=east] {$y$(分)};
        \node[anchor=north east] at (0,0) {$O$};
        \foreach \y in {8,16,24}
            \draw (0.2,\y/8) -- (0,\y/8) node[anchor=east] {$\y$};
        \foreach \x in {2,4} {
            \draw (\x*0.75,0.2) -- (\x*0.75,0) node[anchor=north] {$\x0$};
            \filldraw [fill=white] (\x*0.75-1.5, \x/2) circle (0.1) + (0.1,0) -- (\x*0.75,\x/2);
        }
    \end{tikzpicture}
    \caption{}\label{fig:1-7}
    \end{minipage}
\end{figure}

\liti 在国内投寄外埠平信,每封信不超过20克重付邮资8分,超过20克重而不超过40克重付邮资1角6分。那么,每封 $x (0 < x \leqslant 40)$ 克重的信应付邮资(分)
$$
f(x)=
\begin{cases}
    8, &x \in (0,20], \\
    16, &x \in (20,40] \text{。}
\end{cases}
$$
画出这个函数的图象。

\jie 这个函数的图象是两条线段,如图\ref{fig:1-7}所示。

当我们所研究的函数 $y=f(x)$ 是用一个式子表示时,如果不加说明,函数的定义域就是指能使这个式子有意义的实数 $x$ 的集合。

\liti 求函数 $f(x) = \dfrac{1}{x -2}$ 的定义域。

\jie 因为 $x-2=0$ 即 $x=2$ 时,$\dfrac{1}{x - 2}$ 没有意义,而 $x \neq 2$ 时,$\dfrac{1}{x - 2}$ 都有意义,所以这个函数的定义域是
$\{x \mid x \in R \text{,且} x \neq 2\} \text{。}$

\liti 求函数 $f(x) = \sqrt{3x+2}$ 的定义域。

\jie 因为 $3x+2 \geqslant 0$ 即 $x \geqslant -\dfrac 2 3$ 时,$\sqrt{3x+2}$ 有意义,而 $x < -\dfrac 2 3$ 时,$\sqrt{3x+2}$ 没有意义,
所以这个函数的定义域是$\left[ -\dfrac 2 3 , +\infty \right)$。

\liti 求函数 $f(x) = \sqrt{x+1} + \sqrt{1-x} + 2$ 的定义域。

\jie 使 $\sqrt{x+1}$ 有意义的实数 $x$ 的集合是 $[-1, +\infty)$,
使 $\sqrt{1-x}$ 有意义的实数 $x$ 的集合是 $(-\infty, 1]$,所以这个函数的定义域是
$$[-1, +\infty) \cap (-\infty, 1] = [-1, 1] \text{。}$$

\lianxi

\begin{xiaotis}

\vspace{-2em}
\begin{figure}[htbp]
\begin{minipage}[b]{0.65\linewidth}
    \setlength{\parindent}{\defaultParIndent}

    \xiaoti{(口答)如图,已知函数 $y=x^2$,圈内的数都是整数,求:}

    \begin{xiaoxiaotis}

        \xiaoxiaoti{函数的定义域、值域;}

        \xiaoxiaoti{和 $x=-2$ 对应的象;}

        \xiaoxiaoti{$y=9$ 和什么原象对应。}

    \end{xiaoxiaotis}

    \xiaoti{已知函数 $f(x) = 2x - 3,\, x \in \{0,1,2,3,5\}$,求 $f(0)$,$f(2)$,$f(5)$ 及函数的值域。}

    \xiaoti{画出下列函数的图象:}

    \begin{xiaoxiaotis}

        \xiaoxiaoti{$f(x) = 2x,\, x \in Z \text{,且} |x| \leqslant 2$;}

        \xiaoxiaoti{$f(x) =
            \begin{cases}
                1, &x \in (0, +\infty), \\
                -1, &x \in (-\infty, 0] \text{。}
            \end{cases}$
        }

    \end{xiaoxiaotis}
\end{minipage}
\hfill
\begin{minipage}[t]{0.3\linewidth}
    \begin{tikzpicture}[>=Stealth]
        \draw (0,0) circle [x radius=1cm, y radius=3cm];
        \node at (1.5,3) {平方};

        \node at (0,2.5) {$0$};
        \node at (0.3,1.8) {$-3$};
        \node at (-0.2,1.0) {$2$};
        \node at (0.3,0.2) {$3$};
        \node at (-0.5,-0.3) {$-1$};
        \node at (0,-1.0) {$-2$};
        \node at (0,-1.6) {$1$};
        \node at (0,-2.2) {$\vdots$};

        \draw (3,0) circle [x radius=1cm, y radius=3cm];
        \node at (3,2.2) {9};
        \node at (3,1) {0};
        \node at (3,-0.2) {4};
        \node at (3,-1.4) {1};
        \node at (3,-2.2) {$\vdots$};

        \draw [->] (0.2,2.5) .. controls(1,2.6) and (1.8,2.0) .. (2.8,1.1); % 0 -> 0
        \draw [->] (0.6,1.8) .. controls(1,2.1) and (1.8,2.2) .. (2.8,2.2); % -3 -> 9
        \draw [->] (0,1.0) .. controls(1,0.9) and (1.8,0.6) .. (2.8,0);     % 2 -> 4
        \draw [->] (0.5,0.2) .. controls(1,0.8) and (1.8,1.6) .. (2.8,2.0); % 3 -> 9
        \draw [->] (-0.1,-0.3) .. controls(1,-0.2) and (1.8,-0.8) .. (2.8,-1.4); % -1 -> 1
        \draw [->] (0.4,-1.0) .. controls(1,-0.5) and (1.8,-0.2) .. (2.8,-0.2);  % -2 -> 4
        \draw [->] (0.4,-1.6) .. controls(1,-1.4) and (1.8,-1.4) .. (2.8,-1.6);  % 1 -> 1
    \end{tikzpicture}
    \caption*{(第1题)}
\end{minipage}
\end{figure}

\vspace{-2em}
\xiaoti{求下列函数的定义域:}

\begin{xiaoxiaotis}

    \xiaoxiaoti{$f(x) = \dfrac 1 {4x + 7}$;\vspace{1em}}

    \xiaoxiaoti{$f(x) = \sqrt{1 - x} + \sqrt{x + 3} - 1$。}

\end{xiaoxiaotis}

\end{xiaotis}
