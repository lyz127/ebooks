\xiti

\begin{xiaotis}

\xiaoti{根据给定的对应法则,写出和 $x$ 对应的数值:}

\begin{figure}[htbp]
    \centering
    \begin{minipage}{7cm}
    \centering
    \begin{tikzpicture}[>=Stealth]
        \draw (0,0) circle [x radius=1cm, y radius=3cm];
        \node at (0,3.3) {$x$};
        \node at (1.5,3) {平方};

        \node at (0,2.5) {$0$};
        \node at (0.3,1.8) {$-3$};
        \node at (-0.2,1.0) {$2$};
        \node at (0.3,0.2) {$3$};
        \node at (-0.5,-0.3) {$-1$};
        \node at (0,-1.0) {$-2$};
        \node at (0,-1.6) {$1$};
        \node at (0,-2.2) {$\vdots$};

        \draw (3,0) circle [x radius=1cm, y radius=3cm];
        \node at (3,2.2) {9};
        %\node at (3,1) {0};
        %\node at (3,-0.2) {4};
        %\node at (3,-1.4) {1};
        \node at (3,-2.2) {$\vdots$};

        \draw [->] (0.2,2.5) .. controls(1,2.6) and (1.8,2.0) .. (2.8,1.1); % 0 -> 0
        \draw [->] (0.6,1.8) .. controls(1,2.1) and (1.8,2.2) .. (2.8,2.2); % -3 -> 9
        \draw [->] (0,1.0) .. controls(1,0.9) and (1.8,0.6) .. (2.8,0);     % 2 -> 4
        \draw [->] (0.5,0.2) .. controls(1,0.8) and (1.8,1.6) .. (2.8,2.0); % 3 -> 9
        \draw [->] (-0.1,-0.3) .. controls(1,-0.2) and (1.8,-0.8) .. (2.8,-1.4); % -1 -> 1
        \draw [->] (0.4,-1.0) .. controls(1,-0.5) and (1.8,-0.2) .. (2.8,-0.2);  % -2 -> 4
        \draw [->] (0.4,-1.6) .. controls(1,-1.4) and (1.8,-1.4) .. (2.8,-1.6);  % 1 -> 1
    \end{tikzpicture}
    \end{minipage}
    \qquad
    \begin{minipage}{8cm}
    \centering
    \begin{tikzpicture}[>=Stealth]
        \draw (0,0) circle [x radius=1cm, y radius=3cm];
        \node at (0,3.3) {$x$};
        \node at (1.5,3.2) {求倒数};
        \node at (1.5,2.8) {的2倍};

        \node at (0,2.5) {$-3$};
        \node at (-0.2,1.5) {$-2$};
        \node at (0.3,0.5) {$-1$};
        \node at (-0.5,-0.5) {$1$};
        \node at (0.3,-1.5) {$2$};
        \node at (0,-2.5) {$3$};

        \draw (3,0) circle [x radius=1cm, y radius=3cm];

        \draw [->] (0.3,2.5) -- (2.8,2.5); % -3
        \draw [->] (0.2,1.5) -- (2.6,1.5); % -2
        \draw [->] (0.6,0.5) -- (2.6,0.5); % -1
        \draw [->] (-0.1,-0.5) -- (2.6,-0.5); % 1
        \draw [->] (0.6,-1.5) -- (2.6,-1.5); % 2
        \draw [->] (0.3,-2.5) -- (2.8,-2.5); % 3
    \end{tikzpicture}
    \end{minipage}
    \caption*{(第1题)}
\end{figure}

\xiaoti{根据给定的对应法则,写出和 $x$ 对应的数值:}

\begin{figure}[htbp]
    \centering
    \begin{tikzpicture}[>=Stealth]
        \draw (0,0) rectangle (7.4, 1);
        \node at (4.0,0.5) {$3$};

        \draw (0,2) rectangle (7.4, 3);
        \node at (-0.5,2.5) {$x$};
        \node at (0.3,2.5) {$-2$};
        \node at (1.3,2.5) {$-\sqrt{3}$};
        \node at (2.4,2.5) {$-\sqrt{2}$};
        \node at (3.4,2.5) {$-1$};
        \node at (4.1,2.5) {$0$};
        \node at (4.7,2.5) {$1$};
        \node at (5.4,2.5) {$\sqrt{2}$};
        \node at (6.2,2.5) {$\sqrt{3}$};
        \node at (7.0,2.5) {$2$};

        \draw [->] (3.4,2.3) .. controls(3.3,1.6) and (3.4,1.3) .. (3.8,0.75); % -1 -> 3
        \draw [->] (4.7,2.3) .. controls(4.7,1.6) and (4.6,1.3) .. (4.2,0.75); % 1 -> 3

        \node at (3.6,3.5) {平方乘2加1};
    \end{tikzpicture}
    \caption*{(第2题)}
\end{figure}

\xiaoti{下列各图表示的对应是不是从第一个集合到第二个集合的映射,为什么?}

\begin{figure}[htbp]
    \centering
    \begin{minipage}{4cm}
    \centering
    \begin{tikzpicture}[>=Stealth]
        \draw (0,0) circle [x radius=0.6cm, y radius=2cm];
        \node at (0,1.35) {$a_1$};
        \node at (0,0.45) {$a_3$};
        \node at (0,-0.45) {$a_2$};
        \node at (0,-1.35) {$a_4$};

        \draw (2,0) circle [x radius=0.6cm, y radius=2cm];
        \node at (2,1.35) {$b_1$};
        \node at (2,0.45) {$b_2$};
        \node at (2,-0.45) {$b_3$};
        \node at (2,-1.35) {$b_4$};

        \draw [->] (0.2,1.35) .. controls(0.6,1.5) and (1.2,1.5) .. (1.8,1.35);
        \draw [->] (0.2,0.45) .. controls(0.6,-0.5) and (1.2,-1) .. (1.8,-1.35);
        \draw [->] (0.2,-0.45) .. controls(0.6,-0.3) and (1.2,-0.3) .. (1.8,-0.45);
        \draw [->] (0.2,-1.35) .. controls(0.8,-1.0) and (1.4,0.1) .. (1.8,1.25);
    \end{tikzpicture}
    \caption*{(1)}
    \end{minipage}
    \qquad
    \begin{minipage}{4cm}
    \centering
    \begin{tikzpicture}[>=Stealth]
        \draw (0,0) circle [x radius=0.6cm, y radius=2cm];
        \node at (0,1.35) {$a_1$};
        \node at (0,0.45) {$a_3$};
        \node at (0,-0.45) {$a_2$};
        \node at (0,-1.35) {$a_4$};

        \draw (2,0) circle [x radius=0.6cm, y radius=2cm];
        \node at (2,1.35) {$b_1$};
        \node at (2,0.45) {$b_2$};
        \node at (2,-0.45) {$b_3$};
        \node at (2,-1.35) {$b_4$};

        \draw [->] (0.2,1.35) .. controls(0.6,1.5) and (1.2,1.5) .. (1.8,1.35);
        \draw [->] (0.2,0.45) .. controls(0.6,0.6) and (1.2,0.6) .. (1.8,0.45);
        \draw [->] (0.2,-0.45) .. controls(0.6,-1.0) and (1.2,-1.3) .. (1.8,-1.35);
        \draw [->] (0.2,-1.35) .. controls(0.6,-1.3) and (1.2,-0.9) .. (1.8,-0.45);
    \end{tikzpicture}
    \caption*{(2)}
    \end{minipage}
    \qquad
    \begin{minipage}{4cm}
    \centering
    \begin{tikzpicture}[>=Stealth]
        \draw (0,0) circle [x radius=0.6cm, y radius=2cm];
        \node at (0,1.35) {$a_1$};
        \node at (0.2,0.15) {$a_2$};
        \node at (-0.3,0) {$a_3$};
        \node at (0,-1.35) {$a_4$};

        \draw (2,0) circle [x radius=0.6cm, y radius=2cm];
        \node at (2,1.35) {$b_1$};
        \node at (2.3,0.15) {$b_2$};
        \node at (1.8,0) {$b_3$};
        \node at (2,-1.35) {$b_4$};

        \draw [->] (0.2,1.35) .. controls(0.6,1.5) and (1.2,1.5) .. (2.1,0.25);
        \draw [->] (0.4,0.3) .. controls(0.6,0.8) and (1.2,1.2) .. (1.8,1.35);
        \draw [->] (0.4,0.2) .. controls(0.6,0.3) and (1.2,0.2) .. (1.6,0);
        \draw [->] (-0.1,-0.1) .. controls(0.6,-1.0) and (1.2,-1.3) .. (1.8,-1.35);
        \draw [->] (0.2,-1.35) .. controls(0.6,-1.3) and (1.4,-1.0) .. (2.3,-0.1);
    \end{tikzpicture}
    \caption*{(3)}
    \end{minipage}
    \caption*{(第3题)}
\end{figure}

\xiaoti{下列对应是不是从 $A$ 到 $B$ 的映射,为什么?}
\begin{xiaoxiaotis}

    \xiaoxiaoti{$A = R^+$,$B = R$,对应法则是“求常用对数”;}

    \xiaoxiaoti{$A = \{\text{平面} M \text{内的点}\}$,$B = \{\text{平面} M \text{内的圆}\}$,对应法则是“以点 $P$ 为圆心画圆”;}

    \xiaoxiaoti{$A = \{\alpha \mid 0^\circ \leqslant \alpha \leqslant 180^\circ\}$,$B = [0, 1]$,对应法则是“求余弦”;}

    \xiaoxiaoti{$A = \{\text{平面} M \text{内的三角形}\}$,$B = \{\text{平面} M \text{内的圆}\}$,对应法则是“画三角形的外接圆”。}

\end{xiaoxiaotis}

\xiaoti{在从集合 $A$ 到集合 $B$ 的映射中,}
\begin{xiaoxiaotis}

    \xiaoxiaoti{对于集合 $A$ 中的任意一个元素 $a$,在集合 $B$ 中是不是有象?是不是只有一个象?}

    \xiaoxiaoti{对于集合 $B$ 中的任意一个元素 $b$,在集合 $A$ 中是不是有原象?是不是只有一个原象?}

\end{xiaoxiaotis}

\xiaoti{如图,已知从集合 $A$ 到集合 $B$ 的对应法则是“乘2减3”,
    从集合 $B$ 到集合 $C$ 的对应法则是“乘3减5”。
    按对应法则写出集合 $B$,$C$ 中的对应元素。}

\begin{figure}[htbp]
    \center
    \begin{tikzpicture}[>=Stealth]
        \foreach \x in {0, 3, 6}
            \draw (\x,0.2) -- (\x, 2.5) arc [start angle=180, end angle=0, radius=0.5cm]
                     -- (\x+1,0.2) arc [start angle=0, end angle=-180, radius=0.5cm];
        \draw [->] (0.8,2.6) -- (3.3,2.6);
        \draw [->] (3.85,2.6) -- (6.3,2.6);

        \node at (2,3) {乘2减3};
        \node at (5,3) {乘3减5};
        \node at (0.5,0.1) {$6$};
        \node at (0.5,0.6) {$5$};
        \node at (0.5,1.1) {$4$};
        \node at (0.5,1.6) {$3$};
        \node at (0.5,2.1) {$2$};
        \node at (0.5,2.6) {$1$};
        \node at (3.6,2.6) {$-1$};
        \node at (6.6,2.6) {$-8$};
    \end{tikzpicture}
    \caption*{(第6题)}
\end{figure}

\xiaoti{已知函数 $f(x) = 3x+5,\, x\in R$,
    求$f(-3)$,$f(-2)$,$f(0)$,$f(1)$,$f(2)$ 以及函数的值域。}

\xiaoti{画出下列函数的图象,并说出函数的定义域、值域:}
\begin{xiaoxiaotis}

    \xiaoxiaoti{正比例函数 $y = 3x$;}

    \vspace{0.5em}
    \xiaoxiaoti{反比例函数 $y = \dfrac{8}{x}$;}

    \vspace{0.5em}
    \xiaoxiaoti{一次函数 $y = -4x + 5$;}

    \xiaoxiaoti{二次函数 $y = x^2 - 6x + 7$。}

\end{xiaoxiaotis}

\xiaoti{画出下列函数的图象:}
\begin{xiaoxiaotis}

    \xiaoxiaoti{$f(x) = x + 2,\, x \in Z \text{,且} |x| \leqslant 3$;}

    \xiaoxiaoti{$f(x) = 3x - 5,\, x \in (2, 4]$;}

    \xiaoxiaoti{$f(x) = -\sqrt{x},\, x \in [0, +\infty)$;}

    \xiaoxiaoti{$f(x) = 2x^2 - 3x - 2,\, x \in (-\dfrac{1}{2}, 2)$。}

\end{xiaoxiaotis}

\xiaoti{求下列函数的定义域:}
\begin{xiaoxiaotis}

    \xiaoxiaoti{$f(x) = \dfrac{6}{x^2 - 3x + 2}$;}
    \vspace{1em}

    \xiaoxiaoti{$f(x) = \dfrac{\sqrt[3]{4x + 8}}{\sqrt{3x - 2}}$;}
    \vspace{0.5em}

    \xiaoxiaoti{$f(x) = \sqrt{2x - 1} + \sqrt{1 - 2x} + 4$;}

    \xiaoxiaoti{$f(x) = \sqrt{x^2 - 4}$。}

\end{xiaoxiaotis}

\xiaoti{建筑一个容积为 $8000\text{米}^3$,深为 $6$ 米的长方体蓄水池,
    池壁每 $\text{米}^2$ 的造价为 $a$ 元,池底每 $\text{米}^2$ 的造价为 $2a$ 元。
    把总造价 $y$ 元表示为底的一边长 $x$ 米的函数,并指出函数的定义域。}

\xiaoti{如图,灌溉渠的横断面是等腰梯形,底宽 $2$ 米,边坡的倾角为 $45^\circ$,水深 $h$ 米。
    求横断面中有水面积 $A\text{米}^2$ 与水深 $h$ 米的函数关系式。}

\begin{figure}[htbp]
    \center
    \begin{tikzpicture}[>=Stealth]
        \draw (2,-0.1) -- (2,-0.5);
        \draw[<->] (2,-0.3) -- (3.8,-0.3);
        \node [fill=white] at (2.9,-0.3) {2米};
        \draw (3.8,-0.1) -- (3.8,-0.5);

        \node at (4.6,0.3) {$45^\circ$};

        \draw (4.0,0) -- (6.5,0);
        \draw [<->] (6,0) -- (6,1.8);
        \node [fill=white, rotate=90] at (6,0.9) {$h$米};
        \draw (5.8,1.8) -- (6.5,1.8);

        \filldraw[pattern=horizontal lines light gray] (0,2) -- (2,0) -- (3.8,0) -- (5.8,2) -- (5.6,1.8) -- (0.2,1.8);
        \draw (2.9,1.85) -- (3.1,2.2) -- (2.7,2.2) -- cycle;
    \end{tikzpicture}
    \caption*{(第12题)}
\end{figure}
\xiaoti{投寄本埠平信,每 $20$ 克重应贴邮票 $4$ 分,不足 $20$ 克重的以 $20$ 克重计算。
    写出邮资〈分)与信件重量(在60克重以内)的函数关系式,并画出函数的图象。如果要寄
    挂号信,则每封信另加挂号费 1 角 2 分。写出邮资(分)与挂号信件重量(在60克重以内)
    的函数关系式,并画出函数的图象。}

\end{xiaotis}
