\subsection{用单位圆中的线段表示三角函数值}\label{subsec:2-7}

\begin{wrapfigure}[17]{r}{7.4cm}
    \centering
    \begin{tikzpicture}[>=Stealth, scale=0.5]
    \draw [->] (-7,0) -- (7,0) node[anchor=north] {$x$};
    \draw [->] (0,-7) -- (0,7) node[anchor=east] {$y$};
    \node at (0.4,-0.4) {$O$};
    \foreach \x in {-6,-4,-2,1,2,...,6} {
        \draw (\x,0) -- (\x,0.2) node[anchor=south] {$\x$};
    }
    \foreach \x in {-5,-3,-1} {
        \draw (\x,0.2) -- (\x,0) node[anchor=north] {$\x$};
    }
    \foreach \y in {1,...,6} {
        \draw (0.2,\y) -- (0,\y) node[anchor=east] {\y};
    }
    \foreach \y in {-6,...,-1} {
        \draw (0,\y) -- (0.2,\y) node[anchor=west] {\y};
    }

    \draw (-3, 1) node[anchor=north] {$C$} -- (-3, 5) node[anchor=east] {$D$};
    \draw [ultra thick] (2, 0)  node[anchor=north] {$A$} -- (6, 0) node[anchor=north] {$B$};
\end{tikzpicture}
    \caption{}\label{fig:2-17}
\end{wrapfigure}

我们知道,坐标轴是规定了方向的直线。一条与坐标轴平行的线段也可以规定两种相反的方向。
如图 \ref{fig:2-17},$x$ 轴上的线段 $AB$,可以规定从点 $A$ 到点 $B$ 或从点 $B$ 到点 $A$ 这样两种相反的方向;
与 $y$ 轴平行的线段 $CD$,也可以规定从点 $C$ 到点 $D$ 或从点 $D$ 到点 $C$ 这样两种相反的方向。
如果这样的线段的方向与坐标轴的正向一致,就规定这条线段是正的,否则,就规定它是负的。
例如图 \ref{fig:2-17} 中,$AB = 4$(长度单位),$BA = -4$(长度单位)。

如图 \ref{fig:2-18},设任意角 $\alpha$ 的终边与单位圆相交于点 $P(x, y)$,那么,

$$\sin\alpha = \dfrac y r = \dfrac y 1 = y, \quad \cos\alpha = \dfrac x r = \dfrac x 1 = x \text{。}$$
\vspace{0.5em}

过点 $P$ 作 $x$ 轴的垂线,垂足为 $M$。我们把线段 $MP$,$OM$ 都看作是规定了方向的线段,这样,
当 $MP$ 的方向与 $y$ 轴的正向一致时,$MP$ 是正的,相反时,$MP$ 是负的;
当 $OM$ 的方向与 $x$ 轴的正向一致时,$OM$ 是正的,相反时,$OM$ 是负的。因此,
线段 $MP$ 的符号与点 $P$ 的纵坐标 $y$ 的符号相同,且 $MP$ 的长度等于 $|y|$;
线段 $OM$ 的符号与点 $P$ 的横坐标 $x$ 的符号相同,且 $OM$ 的长度等于 $|x|$。
从而,$\sin\alpha = y = MP$,$\cos\alpha = x = OM$。
我们把单位圆中规定了方向的线段 $MP$,$OM$ 分别叫做角 $\alpha$ 的 \textbf{正弦线,余弦线}。

\begin{figure}[H]
    \centering
    \begin{minipage}{8cm}
    \centering
    \begin{tikzpicture}[>=Stealth, scale=2]
    \draw [->] (-1.3,0) -- (1.7,0) node[anchor=north] {$x$};
    \draw [->] (0,-1.3) -- (0,1.3) node[anchor=east] {$y$};
    \node at (-0.15,-0.15) {$O$};
    \draw [name path=c] (0, 0) circle(1.0);

    \draw [name path=a1] (0,0) -- (140:1.3) node[anchor=south] {$\alpha$的终边};
    \draw [name intersections={of=a1 and c, by=A}]
       let \p1=(A)
       in (A) +(-0.05, -0.05) node[anchor=east] {$P$}
          (\x1,\y1) -- (\x1, 0) node [anchor=north] {$M$};

    \path [name path=a2] (1,0) -- (1, -1);
    \path [name path=a3] (0,0) -- (320:1.5);
    \path [name intersections={of=a2 and a3}];
    \coordinate(T) at (intersection-1);
    \draw [dashed] (0, 0) -- (T) node[anchor=north] {$T$};
    \draw (T) -- (1, 0)  +(0.3, 0.15) node {$A(1,0)$};
\end{tikzpicture}

    \end{minipage}
    \qquad
    \begin{minipage}{8cm}
    \centering
    \begin{tikzpicture}[>=Stealth, scale=2]
    \draw [->] (-1.3,0) -- (1.7,0) node[anchor=north] {$x$};
    \draw [->] (0,-1.3) -- (0,1.3) node[anchor=east] {$y$};
    \node at (-0.15,-0.15) {$O$};
    \draw [name path=c] (0, 0) circle(1.0);

    \draw [name path=a1] (0,0) -- (40:1.6) node [anchor=west, align=left] {$\alpha$的\\终边};
    \draw [name intersections={of=a1 and c, by=A}]
       let \p1=(A)
       in (A) +(-0.05, -0.05) node[anchor=east] {$P$}
          (\x1,\y1) -- (\x1, 0) node [anchor=north] {$M$};

    \path [name path=a2] (1,0) -- (1, 1);
    \path [name intersections={of=a1 and a2}];
    \coordinate [label=above:$T$] (T) at (intersection-1);
    \draw (T) -- (1, 0)  +(0.3, 0.15) node {$A(1,0)$};
\end{tikzpicture}

    \end{minipage}
    \begin{minipage}{8cm}
    \centering
    \begin{tikzpicture}[>=Stealth, scale=2]
    \draw [->] (-1.3,0) -- (1.7,0) node[anchor=north] {$x$};
    \draw [->] (0,-1.3) -- (0,1.3) node[anchor=east] {$y$};
    \node at (0.15,-0.15) {$O$};
    \draw [name path=c] (0, 0) circle(1.0);

    \draw [name path=a1] (0,0) -- (220:1.3) node [anchor=north]{$\alpha$的终边};
    \draw [name intersections={of=a1 and c, by=A}]
       let \p1=(A)
       in (A) +(-0.05, -0.05) node[anchor=east] {$P$}
          (\x1,\y1) -- (\x1, 0) node [anchor=south] {$M$};

    \path [name path=a2] (1,0) -- (1, 1);
    \path [name path=a3] (0,0) -- (40:1.5);
    \path [name intersections={of=a2 and a3}];
    \coordinate [label=above:$T$] (T) at (intersection-1);
    \draw [dashed] (0, 0) -- (T);
    \draw (T) -- (1, 0)  +(0.3, 0.15) node {$A(1,0)$};
\end{tikzpicture}

    \end{minipage}
    \qquad
    \begin{minipage}{8cm}
    \centering
    \begin{tikzpicture}[>=Stealth, scale=2]
    \draw [->] (-1.3,0) -- (1.7,0) node[anchor=north] {$x$};
    \draw [->] (0,-1.3) -- (0,1.3) node[anchor=east] {$y$};
    \node at (-0.15,-0.15) {$O$};
    \draw [name path=c] (0, 0) circle(1.0);

    \draw [name path=a1] (0,0) -- (320:1.6) node [anchor=west, align=left] {$\alpha$的\\终边};
    \draw [name intersections={of=a1 and c, by=A}]
       let \p1=(A)
       in (A) +(-0.05, -0.05) node[anchor=north] {$P$}
          (\x1,\y1) -- (\x1, 0) node [anchor=south] {$M$};

    \path [name path=a2] (1,0) -- (1, -1);
    \path [name intersections={of=a1 and a2}];
    \coordinate [label=30:$T$] (T) at (intersection-1);
    \draw (T) -- (1, 0)  +(0.3, 0.15) node {$A(1,0)$};
\end{tikzpicture}

    \end{minipage}
    \caption{}\label{fig:2-18}
\end{figure}


过点 $A(1, 0)$ 作单位圆的切线,那么这条切线平行于 $y$ 轴(为什么?)。设这条切线与角 $\alpha$
的终边(当 $\alpha$ 为第一、四象限的角时)或这条终边的反向延长线(当 $\alpha$ 为第二、三象限
的角时)交于点 $T$。因为 $\triangle OMP \xiangsi \triangle OAT$,并且 $OM$ 与 $MP$ 同号时,
$OA$ 与 $AT$ 也同号,$OM$ 与 $MP$ 异号时,$OA$ 与 $AT$ 也异号,所以
$$\tan\alpha = \dfrac y x = \dfrac{MP}{OM} = \dfrac{AT}{OA} \text{。}$$
但 $OA = 1$,从而
$$\tan\alpha = AT \text{。}$$

我们把规定了方向的线段 $AT$ 叫做角 $\alpha$ 的\textbf{正切线}。

当角 $\alpha$ 的终边在 $x$ 轴上时,点 $T$ 与点 $A$ 重合,这时正切线变成了一个点;
当角 $\alpha$ 的终边在 $y$ 轴上时,点 $T$ 不存在,即正切线不存在。

\lianxi
\begin{xiaotis}

\xiaoti{作出下列各角的正弦线、余弦线、正切钱:}
\begin{xiaoxiaotis}

    \fourInLine[8em]{\xiaoxiaoti{$\dfrac \pi 3$;}}{\xiaoxiaoti{$\dfrac{5\pi}{6}$;}}{\xiaoxiaoti{$-\dfrac{2\pi}{3}$;}}{\xiaoxiaoti{$-\dfrac{13\pi}{6}$。}}
    \vspace{0.5em}

\end{xiaoxiaotis}

\xiaoti{以 $5cm$ 为单位长作单位圆,分别作出$30^\circ$,$225^\circ$,$330^\circ$ 的角的
    正弦线、余弦线、正切线,量出它们的长度,从而写出这些角的正弦值、余弦值、正切值(精确到 $0.01$)。}

\end{xiaotis}
