\xiti
\begin{xiaotis}

\xiaoti{在 $0^\circ$ \~{} $360^\circ$ 之间,找出与下列各角终边相同的角,并判定下列各角是哪个象限的角:}
\begin{xiaoxiaotis}
    
    \begin{tabular}[t]{*{4}{@{}p{8em}}}
        \xiaoxiaoti {$-265^\circ$;} & \xiaoxiaoti {$1185^\circ 14'$;} & \xiaoxiaoti {$-1000^\circ$;} & {\xiaoxiaoti{$-843^\circ 10'$}} \\
        \xiaoxiaoti {$-15^\circ$;} & \xiaoxiaoti {$3900^\circ$;} & \xiaoxiaoti {$560^\circ 24'$;} & \xiaoxiaoti{$2903^\circ 15'$。}
    \end{tabular}

\end{xiaoxiaotis}

\xiaoti{写出与下列各角终边相同的角的集合,并把集合中在 $-360^\circ$ \~{} $360^\circ$ 之间的角写出来:}
\begin{xiaoxiaotis}
    
    \begin{tabular}[t]{*{4}{@{}p{8em}}}
        \xiaoxiaoti {$60^\circ$;} & \xiaoxiaoti {$-75^\circ$;} & \xiaoxiaoti {$-824^\circ 30'$;} & {\xiaoxiaoti{$475^\circ$}} \\
        \xiaoxiaoti {$90^\circ$;} & \xiaoxiaoti {$270^\circ$;} & \xiaoxiaoti {$180^\circ$;} & \xiaoxiaoti{$0^\circ$。}
    \end{tabular}

\end{xiaoxiaotis}

\xiaoti{写出终边在 $x$ 轴上的角的集合。}

\xiaoti{分别写出第一象限的角、第二象限的角、第三象限的角、第四象限的角的集合。}

\xiaoti{一条弦的长等于半径,这条弦所对的圆心角是否为 $1$ 弧度?为什么?}

\xiaoti{把下列各度化成弧度(写成多少 $\pi$ 的形式):}

\hspace{2em}
\begin{tabular}[t]{*{4}{@{}p{8em}}}
    $18^\circ$, & $-120^\circ$, & $735^\circ$, & $-12.5^\circ$, \\
    $10^\circ$, & $1080^\circ$, & $19^\circ 48'$, & $-9^\circ 20'$。
\end{tabular}

\xiaoti{把下列各弧度化成度:}

\vspace{0.5em}
\hspace{2em} $-\dfrac{7\pi}{6}$,$\dfrac \pi {15}$,$\dfrac{5\pi} 8$,
$-\dfrac{8\pi} 3$,$-5$,$1.4$。
\vspace{0.5em}

\xiaoti{填写下表:}

\begin{table}[h]
\hspace{4em}
\begin{tabular}{|*{5}{m{2cm}<{\centering}|}}
    \hline
    \diagbox{角}{函数} & 正弦 & 余弦 & 正切 & 余切 \\ \hline
    $\dfrac \pi 6$ & \rule{0pt}{3em} & & & \\ \hline
    $\dfrac \pi 4$ & \rule{0pt}{3em} & & & \\ \hline
    $\dfrac \pi 3$ & \rule{0pt}{3em} & & & \\ \hline
\end{tabular}
\end{table}

\xiaoti{填写下表:}

\begin{table}[h]
\hspace{4em}
\begin{tabular}{|*{5}{m{2cm}<{\centering}|}}
    \hline
    \diagbox{角}{函数} & 正弦 & 余弦 & 正切 & 余切 \\ \hline
    $\dfrac \pi 2$ & \rule{0pt}{3em} & & & \\ \hline
    $\dfrac {2\pi} 3$ & \rule{0pt}{3em} & & & \\ \hline
    $\dfrac {3\pi} 4$ & \rule{0pt}{3em} & & & \\ \hline
    $\dfrac {5\pi} 6$ & \rule{0pt}{3em} & & & \\ \hline
\end{tabular}
\end{table}

\xiaoti{把下列各角化成 $2k\pi + \alpha \, (0 \leqslant \alpha < 2\pi,\, k \in Z)$ 的形式:}
\begin{xiaoxiaotis}

    \vspace{0.5em}
    \fourInLine[8em]{\xiaoxiaoti{$-\dfrac{25}{6}\pi$;}}{\xiaoxiaoti{$-5\pi$;}}{\xiaoxiaoti{$-45^\circ$;}}{\xiaoxiaoti{$400^\circ$。}}
    \vspace{0.5em}

\end{xiaoxiaotis}

\xiaoti{求下列各三角函数的值:}
\begin{xiaoxiaotis}
    
    \vspace{0.5em}
    \fourInLine[8em]{\xiaoxiaoti{$\tan 1$;}}{\xiaoxiaoti{$\cot \dfrac 1 2$;}}{\xiaoxiaoti{$\cos \dfrac 4 5 \pi$;}}{\xiaoxiaoti{$\sin 2.1$。}}
    \vspace{0.5em}

\end{xiaoxiaotis}

\xiaoti{采用弧度制,重新解答第 $4$ 题。}

\xiaoti{圆的半径等于 $240mm$,求个圆上长 $500mm$ 的弧所对圆心角的度数。}

\xiaoti{直径是 $20cm$ 的滑轮,每秒钟旋转 $45$ 弧度,求轮周上一点经过 $5$ 秒钟所转过的弧长。}

\xiaoti{航海罗盘将圆周分成 $32$ 等份,把每一等份所对的圆心角的大小分别用度与弧度表示出来。}

\xiaoti{某种蒸气机上的飞轮直径为 $1.2m$,每分忡按逆时针方向旋转 $300$ 转,求:}
\begin{xiaoxiaotis}
    
    \xiaoxiaoti{飞轮每秒钟转过的弧度数;}

    \xiaoxiaoti{轮周上的一点每秒钟经过的弧长。}

\end{xiaoxiaotis}

\xiaoti{要在半径 $OA = 100cm$ 的圆形金属板上,截取一块 $\hudu{AB}$ 的长为 $112cm$ 的扇形板,
    应截取的圆心角 $AOB$ 的度数是多少(精确到 $1^\circ$)?}

\xiaoti{已知 $1^\circ$ 的圆心角所对的弧的长为 $1$ 米,这个圆的半径是多少?}

\xiaoti{已知长 $50cm$ 的弧含有 $200^\circ$,求这条弧所在的圆的半径(精确到 $1cm$)。}

\end{xiaotis}
