\xiti
\begin{xiaotis}

\xiaoti{已知角 $\alpha$ 的终边分别经过下列各点,求 $\alpha$ 的六个三角函数值:}
\begin{xiaoxiaotis}

    \twoInLine[8em]{\xiaoxiaoti{$(-8, -6)$;}}{\xiaoxiaoti{$(\sqrt{3}, -1)$。}}

\end{xiaoxiaotis}

\xiaoti{计算:}
\begin{xiaoxiaotis}

    \xiaoxiaoti{$5\sin 90^\circ + 2\cos 0^\circ - 3\sin 270^\circ + 10\cos 180^\circ$;}

    \xiaoxiaoti{$7\cos 270^\circ + 12\sin 0^\circ + 2\cot 90^\circ - 8\sec 180^\circ$;}

    \vspace{0.5em}
    \xiaoxiaoti{$\cos \dfrac \pi 3 - \tan \dfrac \pi 4 + \dfrac 3 4 \tan^2 \dfrac \pi 6 - \sin \dfrac \pi 6 + \cos^2 \dfrac \pi 6 + \sin \dfrac{3\pi}{2}$;}
    \vspace{0.5em}

    \xiaoxiaoti{$\sin^4 \dfrac \pi 4 - \cos^2 \dfrac \pi 2 + 6\tan^3 \dfrac{3\pi}{4}$。}
    \vspace{0.5em}

\end{xiaoxiaotis}

\xiaoti{化简:}
\begin{xiaoxiaotis}

    \xiaoxiaoti{$a\sin 0^\circ + b\cos 90^\circ + c\tan 180^\circ$;}

    \xiaoxiaoti{$-p^2 \sec 180^\circ + q^2 \sin 90^\circ -2pq \cos 0^\circ$;}

    \vspace{0.5em}
    \xiaoxiaoti{$a^2 \cos 2\pi - b^2 \sin \dfrac{3\pi}{2} + ab \cos \pi - ab \csc \dfrac \pi 2$;}
    \vspace{0.5em}

    \xiaoxiaoti{$m\tan 0 + n\cos \dfrac \pi 2 - p\sin \pi - q\cos\dfrac{3}{2}\pi - r\sin 2\pi$。}
    \vspace{0.5em}

\end{xiaoxiaotis}

\xiaoti{根据已知条件计算下式的值:\\
    $\sin \left(\alpha + \dfrac \pi 4 \right) + 2\sin \left( \alpha - \dfrac \pi 4 \right) - 4\cos 2\alpha + 3\cos \left(\alpha + \dfrac 3 4 \pi \right)$。
}
\begin{xiaoxiaotis}

    \twoInLine[10em]{\xiaoxiaoti{$\alpha = \dfrac \pi 4$;}}{\xiaoxiaoti{$\alpha = \dfrac{3\pi}{4}$。}}
    \vspace{0.5em}

\end{xiaoxiaotis}

\xiaoti{确定下列各三角函数值的符号(不求出值):}
\begin{xiaoxiaotis}

    \renewcommand\arraystretch{1.5}
    \begin{tabular}[t]{*{3}{@{}p{10em}}}
        \xiaoxiaoti{$\csc 186^\circ$;} & \xiaoxiaoti{$\cot 505^\circ$;} & \xiaoxiaoti{$\sin 7.6\pi$;} \\
        \xiaoxiaoti{$\tan \left( -\dfrac{23}{4} \pi \right)$;} & \xiaoxiaoti{$\sec 940^\circ$;} & \xiaoxiaoti{$\cos \left( -\dfrac{59}{17} \pi \right)$。}
    \end{tabular}
    \vspace{0.5em}

\end{xiaoxiaotis}

\xiaoti{确定下列各式的符号:}
\begin{xiaoxiaotis}

    \xiaoxiaoti{$\tan 125^\circ \cdot \sin 273^\circ$;}

    \vspace{0.5em}
    \xiaoxiaoti{$\dfrac{\cot 108^\circ}{\cos 305^\circ 12'}$;}
    \vspace{0.5em}

    \vspace{0.5em}
    \xiaoxiaoti{$\sin \dfrac 5 4 \pi \cdot \cos \dfrac 4 5 \pi \cdot \tan \dfrac{11}{6} \pi$;}
    \vspace{0.5em}

    \vspace{0.5em}
    \xiaoxiaoti{$\dfrac{\sec \dfrac 5 6 \pi \cdot \tan \dfrac{11}{6} \pi}{\csc \dfrac 2 3 \pi}$。}
    \vspace{0.5em}

\end{xiaoxiaotis}

\xiaoti{根据下列条件,确定 $\theta$ 是第几象限的角:}
\begin{xiaoxiaotis}

    \renewcommand\arraystretch{1.5}
    \begin{tabular}[t]{*{2}{@{}p{15em}}}
        \xiaoxiaoti{$\sin \theta > 0$ 且 $\cos \theta < 0$;} & \xiaoxiaoti{$\sec \theta < 0$ 且 $\tan \theta > 0$;} \\
        \xiaoxiaoti{$\dfrac{\sin \theta}{\cot \theta} > 0$;} & \xiaoxiaoti{$\sin \theta \cdot \cos \theta$。}
    \end{tabular}
    \vspace{0.5em}

\end{xiaoxiaotis}

\xiaoti{求下列各三角函数值:}
\begin{xiaoxiaotis}

    \renewcommand\arraystretch{1.5}
    \begin{tabular}[t]{*{2}{@{}p{15em}}}
        \xiaoxiaoti{$\cos 840^\circ$;} & \xiaoxiaoti{$\sin \left( -\dfrac{67}{12} \pi \right)$;} \\
        \xiaoxiaoti{$\cot(-1300^\circ)$;} & \xiaoxiaoti{$\tan(-1266^\circ 15')$;} \\
        \xiaoxiaoti{$\sin \dfrac{49}{18} \pi$;} & \xiaoxiaoti{$\cos \left( -\dfrac{11}{3} \pi \right)$;} \\
        \xiaoxiaoti{$\tan \left(-\dfrac{15\pi}{4} \right)$;} & \xiaoxiaoti{$\cos 398^\circ 13'$;} \\
        \xiaoxiaoti{$\cot(-610^\circ 42')$;} & \xiaoxiaoti{$\sin \dfrac{47}{10} \pi$。}
    \end{tabular}
    \vspace{0.5em}

\end{xiaoxiaotis}

\xiaoti{根据下列条件,求角 $\alpha$ 的其他各三角函数值:}
\begin{xiaoxiaotis}

    \vspace{0.5em}
    \xiaoxiaoti{已知 $\sin \alpha = -\dfrac{\sqrt{3}}{2}$,且 $\alpha$ 为第四象限的角;}
    \vspace{0.5em}

    \xiaoxiaoti{已知 $\sec \alpha = -\dfrac 5 4$,且 $\alpha$ 为第三象限的角;}
    \vspace{0.5em}

    \xiaoxiaoti{已知 $\tan \alpha = -\dfrac 3 4$;}
    \vspace{0.5em}

    \xiaoxiaoti{已知 $\cos \alpha = 0.68$(计算结果保留两个有效数字)。}

\end{xiaoxiaotis}

\vspace{0.5em}
\xiaoti{}
\begin{xiaoxiaotis}

    \vspace{-1.7em}
    \begin{minipage}{0.9\textwidth}
    \xiaoxiaoti{已知 $\cos \theta = \dfrac{12}{13}$,并且 $\theta$ 为第四象限的角,求 $\sec \theta$ 及 $\tan \theta$;}
    \end{minipage}
    \vspace{0.5em}

    \xiaoxiaoti{已知 $\sin x = -\dfrac 1 3$,求 $\cos x$ 及 $\tan x$。}
    \vspace{0.5em}

\end{xiaoxiaotis}

\xiaoti{}
\begin{xiaoxiaotis}

    \vspace{-1.7em}
    \begin{minipage}{0.9\textwidth}
    \xiaoxiaoti{已知 $\tan \alpha = \sqrt{3}$,$\pi <\alpha < \dfrac 3 2 \pi$,求 $\cos \alpha - \sin \alpha$;}
    \end{minipage}
    \vspace{0.5em}

    \xiaoxiaoti{已知 $\cos \alpha = \dfrac 4 5$,求 $\sec^2 \alpha + \csc^2 \alpha$。}
    \vspace{0.5em}

\end{xiaoxiaotis}

\xiaoti{已知 $\csc \alpha = t$,求 $\cos \alpha$。}

\xiaoti{}
\begin{xiaoxiaotis}

    \vspace{-1.7em}
    \begin{minipage}{0.9\textwidth}
    \xiaoxiaoti{已知 $\cos \theta \neq 0$,且 $\cos \theta \neq \pm 1$,用 $\cos \theta$ 来表示 $\theta$ 的其他各三角函数;}
    \end{minipage}

    \xiaoxiaoti{用 $\sec \varphi$ 来表示 $\varphi$ 的其他各三角函数;}

    \xiaoxiaoti{已知 $\sin \theta \neq 0$,且 $\sin \theta \neq \pm 1$,用 $\sin \theta$ 来表示 $\theta$ 的其他各三角函数;}

    \xiaoxiaoti{已知 $\cot \alpha \neq 0$,用 $\cot \alpha$ 来表示 $\alpha$ 的其他各三角函数。}

\end{xiaoxiaotis}

\xiaoti{化简:}
\begin{xiaoxiaotis}

    \begin{tabular}[t]{*{2}{@{}p{15em}}}
        \xiaoxiaoti{$\sin^2 190^\circ \cdot \csc^2 190^\circ$;} & \xiaoxiaoti{$(1 + \tan^2 \alpha) \cos^2 \alpha$;} \\
        \xiaoxiaoti{$\csc^2 \theta - \tan \theta \cot \theta$;} & \xiaoxiaoti{$\sec^2 A - \tan^2 A - \sin^2 A$。}
    \end{tabular}

\end{xiaoxiaotis}

\xiaoti{化简:}
\begin{xiaoxiaotis}

    \xiaoxiaoti{$\sec \alpha \sqrt{1 + \tan^2 \alpha} + \tan \alpha \sqrt{\csc^2 \alpha - 1}$ (其中 $\alpha$ 为第四象限的角);}

    \vspace{0.5em}
    \xiaoxiaoti{$\sqrt{\dfrac{1 + \sin\alpha}{1 - \sin\alpha}} - \sqrt{\dfrac{1 - \sin\alpha}{1 + \sin\alpha}}$ (其中 $\alpha$ 为第二象限的角)。}
    \vspace{0.5em}

\end{xiaoxiaotis}

\xiaoti{化简:}
\begin{xiaoxiaotis}

    \vspace{0.5em}
    \xiaoxiaoti{$\dfrac{1 - \cos^2 \alpha}{1 - \sin^2 \alpha} + \cos\alpha \sec\alpha$;}
    \vspace{0.5em}

    \xiaoxiaoti{$(\tan\beta + \cot\beta)^2 - (\tan\beta - \cot\beta)^2$;}

    \vspace{0.5em}
    \xiaoxiaoti{$\dfrac{\sin A + \cos A}{\sec A + \csc A}$;}
    \vspace{0.5em}

    \vspace{0.5em}
    \xiaoxiaoti{$\cos^2 \dfrac \alpha 2 \cdot \csc^2 \dfrac \alpha 2 + \sin^2 \dfrac \alpha 2 + \cos^2 \dfrac \alpha 2$。}
    \vspace{0.5em}

\end{xiaoxiaotis}

\xiaoti{已知 $\tan\alpha = 2$,求 $\dfrac{\sin\alpha + \cos\alpha}{\sin\alpha - \cos\alpha}$ 的值。}

\xiaoti{证明下列恒等式:}
\begin{xiaoxiaotis}

    \vspace{0.5em}
    \xiaoxiaoti{$\dfrac{1 - 2\sin x \cos x}{\cos^2 x - \sin^2 x} = \dfrac{1 - \tan x}{1 + \tan x}$;}
    \vspace{0.5em}

    \xiaoxiaoti{$\tan^2 \theta - \sin^2 \theta = \tan^2 \theta \cdot \sin^2 \theta$;}

    \vspace{0.5em}
    \xiaoxiaoti{$(\sin A - \csc A)(\cos A - \sec A) = \dfrac{1}{\tan A + \cot A}$;}
    \vspace{0.5em}

    \vspace{0.5em}
    \xiaoxiaoti{$\dfrac{1 + \tan^2 A}{1 + \cot^2 A} = \left( \dfrac{1 - \tan A}{1 - \cot A} \right)^2 $。}
    \vspace{0.5em}

\end{xiaoxiaotis}

\xiaoti{证明下列恒等式:}
\begin{xiaoxiaotis}

    \xiaoxiaoti{$(\cos\alpha - 1)^2 + \sin^2 \alpha = 2 - 2\cos\alpha$;}

    \xiaoxiaoti{$(\cos\alpha - \cos\beta)^2 + (\sin\alpha - \sin\beta)^2 = 2 - 2(\cos\alpha \cos\beta + \sin\alpha \sin\beta)$;}

    \xiaoxiaoti{$\sin^4 x + \cos^4 x = 1 - 2\sin^2 x \cos^2 x$;}

    \xiaoxiaoti{$\sin^3 \theta (1 + \cot\theta) + \cos^3 \theta (1 + \tan\theta) = \sin\theta + \cos\theta$;}

    \vspace{0.5em}
    \xiaoxiaoti{$\dfrac{\tan^2 A - \cot^2 A}{\sin^2 A - \cos^2 A} = \sec^2 A + \csc^2 A$;}
    \vspace{0.5em}

    \vspace{0.5em}
    \xiaoxiaoti{$\dfrac{\tan\alpha \cdot \sin\alpha}{\tan\alpha - \sin\alpha} = \dfrac{\tan\alpha + \sin\alpha}{\tan\alpha \cdot \sin\alpha}$;}
    \vspace{0.5em}

    \xiaoxiaoti{$(\sin A + \sec A)^2 + (\cos A + \csc A)^2 = (1 + \sec A \csc A)^2$;}

    \vspace{0.5em}
    \xiaoxiaoti{$\dfrac{\tan A - \tan B}{\cot B - \cot A} = \dfrac{\tan B}{\cot A}$。}
    \vspace{0.5em}

\end{xiaoxiaotis}

\xiaoti{已知 $\alpha$ 是第一象限的角,求证\\
    $$\sqrt{\csc^2 \alpha - 1} - \dfrac{1}{\sqrt{\sec^2 \alpha - 1}} = \sqrt{1 - \cos^2 \alpha} - \tan\alpha \sqrt{1 - \sin^2 \alpha} \text{。}$$
}


\xiaoti{已知 $x = \rho\cos \theta$,$y = \rho\sin \theta$,$x \neq 0$,求证:}
\begin{xiaoxiaotis}

    \vspace{0.5em}
    \twoInLine[15em]{\xiaoxiaoti{$x^2 + y^2 = \rho^2$;}}{\xiaoxiaoti{$\tan\theta = \dfrac y x$。}}
    \vspace{0.5em}

\end{xiaoxiaotis}

\xiaoti{已知 $x \cos\theta = a, \, y \cot\theta = b \, (a \neq 0, b \neq 0)$,求证\\
    $$\dfrac{x^2}{a^2} - \dfrac{y^2}{b^2} = 1 \text{。}$$
}
\vspace{0.5em}

\end{xiaotis}
