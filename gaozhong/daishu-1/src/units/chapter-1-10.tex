\subsection{反函数}\label{subsec:1-10}

函数 $y = f(x) = 2x + 4 \; (x \in R)$ 是由 $f(x)$ 的定义域 $R$ 到值域 $R$ 上的一个一一映射
$f: R \to R$ 确定的,这个一一映射使 $f(x)$ 的值域 $R$ 中的元素 $y = 2x + 4$ 和定义域 $R$
中的元素 $x$ 对应。那么,$f: A \to B$ 的逆映射 $f^{-1}: B \to A$ 就确定了一个函数
$x = \dfrac{1}{2}y - 2 \; (y \in R)$,它使 $f(x)$ 的定义域 $R$ 中的元素 $x = \dfrac{1}{2}y - 2$
和 $f(x)$ 的值域 $R$ 中的元素 $y$ 对应。
\vspace{0.5em}

一般地,如果确定函数 $y = f(x)$ 的映射 $f: A \to B$ 是 $f(x)$ 的定义域 $A$ 到值域 $B$ 上
的一一映射,那么这个映射的逆映射 $f^{-1}: B \to A$ 所确定的函数 $x = f^{-1}(y)$ 叫做函数
$y = f(x)$ 的\textbf{反函数}。函数 $y = f(x)$ 的定义域、值域分别是函数 $x = f^{-1}(y)$ 的值域、定义域。

\vspace{0.5em}
这样,函数 $x = f^{-1}(y) = \dfrac{1}{2}y - 2 \; (y \in R)$ 就是函数 $y = f(x) = 2x + 4 \; (x \in R)$ 的反函数。
\vspace{0.5em}

又如,第 \ref{subsec:1-8} 节例 \hyperref[li:1-8-4]{(4)} 中的映射 $f: \buji{R^-} \to \buji{R^-}$ 确定函数
$y = f(x) = x^2 \; (x \in \buji{R^-})$, 由第 \ref{subsec:1-9} 节知道,这个映射的逆映射 $f^{-1}: \buji{R^-} \to \buji{R^-}$
确定函数 $x = f^{-1}(y) = \sqrt{y} \; (y \in \buji{R^-})$。函数 $x = f^{-1}(y) = \sqrt{y} \; (y \in \buji{R^-})$
就是函数 $y = f(x) = x^2 \; (x \in \buji{R^-})$ 的反函数。

在函数式 $x = f^{-1}(y)$ 中,$y$ 表示自变量,$x$ 表示函数。但在习惯上,我们一般用 $x$ 表示自变量,用 $y$ 表示函数,
为此我们常常对调函数式 $x = f^{-1}(y)$ 中的字母 $x$,$y$,把它改写成 $y = f^{-1}(x)$ (在本书中,今后凡不特别说明,
函数的反函数都是指这种经过改写的反函数)。


\liti 求下列函数的反函数:

\begin{xiaoxiaotis}
    \begin{tabular}[t]{@{}p{14em}@{}p{20em}} 
        \xiaoxiaoti {$y = 3x - 1 \; (x \in R)$;} & \xiaoxiaoti {$y = x^3 + 1 \; (x \in R)$;} \\
        \xiaoxiaoti {$y = \sqrt{x} + 1 \; (x \geqslant 0)$;} & \xiaoxiaoti {$y = \dfrac{2x + 3}{x - 1} \; (x \in R \text{,且} x \neq 1)$。}
    \end{tabular}
\end{xiaoxiaotis}

\jie (1)由于 $y = 3x -1$,可得 $x = \dfrac{y + 1}{3}$,

$\therefore$ 函数 $y = 3x - 1 \; (x \in R)$ 的反函数是 $y = \dfrac{x + 1}{3} \; (x \in R)$;

(2)由 $y = x^3 + 1$,可得 $x = \sqrt[3]{y - 1}$,

$\therefore$ 函数 $y = x^3 + 1 \; (x \in R)$ 的反函数是 $y = \sqrt[3]{x - 1} \; (x \in R)$;

(3)由 $y = \sqrt{x} + 1$,可得 $x = (y - 1)^2$,

$\therefore$ 函数 $y = \sqrt{x} + 1 \; (x \geqslant 0)$ 的反函数是 $y = (x - 1)^2 \; (x \geqslant 1)$;

\vspace{0.5em}
(4)由$y = \dfrac{2x + 3}{x - 1}$,可得 $x = \dfrac{y + 3}{y - 2}$,
\vspace{0.5em}

$\therefore$ 函数 $y = \dfrac{2x + 3}{x - 1} \; (x \in R \text{,且} x \neq 1)$ 的反函数是 $y = \dfrac{x + 3}{x - 2} \; (x \in R \text{,且} x \neq 2)$;
\vspace{0.5em}

求反函数时,由于确定函数 $y = f(x)$ 的映射 $f: A \to B$ 是 $f$ 的定义域 $A$ 到值域 $B$ 上的一一映射,
我们可以先把函数式 $y = f(x)$ 看作以 $x$ 为未知数的方程,从中解出 $x = f^{-1}(y)$,再改写为 $y = f^{-1}(x)$。

如果函数 $y = f(x)$ 的反函数是 $y = f^{-1}(x)$,那么显然函数 $y = f^{-1}(x)$ 的反函数就是 $y = f(x)$。

\lianxi

\begin{xiaotis}

\xiaoti{已知函数 $y = f(x)$,求它的反函数 $y = f^{-1}(x)$:}

\begin{xiaoxiaotis}
    
    \xiaoxiaoti{$y = -2x + 3 \; (x \in R)$;}

    \vspace{0.5em}
    \xiaoxiaoti{$y = -\dfrac 2 x \; (x \in R \text{,且} x \neq 0)$;}
    \vspace{0.5em}

    \xiaoxiaoti{$y = x^4 \; (x \geqslant 0)$;}

    \vspace{0.5em}
    \xiaoxiaoti{$y = \dfrac {x}{3x + 5} \; (x \in R \text{,且} x \neq -\dfrac 5 3)$。}
    \vspace{0.5em}

\end{xiaoxiaotis}

\xiaoti{}

\begin{xiaoxiaotis}
    \vspace{-1.7em}
    \begin{minipage}{0.9\textwidth}
    \xiaoxiaoti{函数 $y = 2x^2 - 3 \; (x \in R)$ 有没有反函数?为什么?}
    \end{minipage}
    
    \xiaoxiaoti{怎样改变定义域,才能使它有反函数?}
\end{xiaoxiaotis}

\end{xiaotis}

