\xiti
\begin{xiaotis}

\xiaoti{已知等腰三角形一个底角的正弦等于 $\dfrac{5}{13}$,求这个三角形的顶角的正弦、余弦及正切。}

\xiaoti{已知 $\cos\phi = -\dfrac{\sqrt 3}{3}$,并且 $180^\circ < \phi < 270^\circ$,求 $\sin2\phi$,$\cos2\phi$,$\tan2\phi$ 的值。}

\xiaoti{证明下列恒等式:}
\begin{xiaoxiaotis}

    \xiaoxiaoti{$\left( \sin\dfrac{\alpha}{2} + \cos\dfrac{\alpha}{2} \right)^2 = 1 + \sin\alpha$;}\jiange

    \xiaoxiaoti{$\tan\theta - \cot\theta = -2\cot2\theta$;}\jiange

    \xiaoxiaoti{$\tan\left( \alpha + \dfrac \pi 4 \right) + \tan\left( \alpha - \dfrac \pi 4 \right) = 2\tan2\alpha$;}\jiange

    \xiaoxiaoti{$\dfrac{1 + \sin2\phi}{\sin\phi + \cos\phi} = \sin\phi + \cos\phi$;}\jiange

    \xiaoxiaoti{$\sin\theta (1 + \cos2\theta) = \sin2\theta \cos\theta$;}\jiange

    \xiaoxiaoti{$2\sin\left( \dfrac \pi 4 + \alpha \right) \sin\left( \dfrac \pi 4 - \alpha \right) = \cos2\alpha$;}\jiange

    \xiaoxiaoti{$\dfrac{1 + 2\sin\alpha \cos\alpha}{\cos^2\alpha - \sin^2\alpha} = \dfrac{1 + \tan\alpha}{1 - \tan\alpha}$;}\jiange

    \xiaoxiaoti{$\dfrac{1 + \sin2\theta - \cos2\theta}{1 + \sin2\theta + \cos2\theta} = \tan\theta$。}\jiange

\end{xiaoxiaotis}

\xiaoti{已知等腰三角形的顶角的余弦等于 $\dfrac{7}{25}$,求这个三角形的一个底角的正弦、余弦及正切。}\jiange

\xiaoti{已知 $\cos\phi = \dfrac 1 3$,并且 $270^\circ < \phi < 360^\circ$,求 $\sin\dfrac{\phi}{2}$,$\cos\dfrac{\phi}{2}$,$\tan\dfrac{\phi}{2}$ 的值。}\jiange

\xiaoti{已知 $2\alpha + \beta = 90^\circ$,且 $\alpha$ 是锐角,求证}\jiange
$$\sin\alpha = \sqrt{\dfrac{1 - \sin\beta}{2}}, \quad \cos\alpha = \sqrt{\dfrac{1 + \sin\beta}{2}} \text{。} \jiange$$

\xiaoti{已知圆心角的正弦等于 $\dfrac 3 5$,求对同弧的圆周角的正弦、余弦及正切。}

\xiaoti{求证:}
\begin{xiaoxiaotis}

    \threeInLineXxt[13em]{$\sin\dfrac{\pi}{8} = \dfrac 1 2 \sqrt{2 - \sqrt{2}}$;}{$\cos\dfrac{\pi}{8} = \dfrac 1 2 \sqrt{2 + \sqrt{2}}$;}{$\tan67^\circ30' = \sqrt{2} + 1$。}
    \jiange

\end{xiaoxiaotis}

\xiaoti{证明下列恒等式:}
\begin{xiaoxiaotis}

    \xiaoxiaoti{$2\sin\theta + \sin2\theta = 4\sin\theta \cos^2\dfrac{\theta}{2}$;}\jiange

    \xiaoxiaoti{$\dfrac{2\sin\alpha - \sin2\alpha}{2\sin\alpha + \sin2\alpha} = \tan^2\dfrac{\alpha}{2}$;}\jiange

    \xiaoxiaoti{$\tan15^\circ + \cot15^\circ = 4$;}\jiange

    \xiaoxiaoti{$\dfrac{\csc^2\alpha - 2}{\csc^2\alpha} = \cos2\alpha$;}\jiange

    \xiaoxiaoti{$\sin(n\pi + \theta) \cos(n\pi - \theta) = \dfrac 1 2 \sin2\theta ,\; (n \in Z)$;}\jiange

    \xiaoxiaoti{$\dfrac{1 + \sin\phi}{\cos\phi} = \dfrac{\cos\phi}{1 - \sin\phi} = \tan\left( \dfrac{\pi}{4} + \dfrac{\phi}{2}\right)$;}\jiange

    \xiaoxiaoti{$\cos\alpha (\cos\alpha - \cos\beta) + \sin\alpha (\sin\alpha - \sin\beta) = 2\sin^2\dfrac{\alpha - \beta}{2}$;}\jiange

    \xiaoxiaoti{$\dfrac{\cos\alpha}{\sec\dfrac{\alpha}{2} + \csc\dfrac{\alpha}{2}} = \dfrac{1}{2} \sin\alpha \left( \cos\dfrac{\alpha}{2} - \sin\dfrac{\alpha}{2} \right)$;}\jiange

    \xiaoxiaoti{$\cos^4\theta = \dfrac{1}{4} + \dfrac{1}{2}\cos2\theta + \dfrac{1}{4}\cos^2 2\theta$;}\jiange

    \xiaoxiaoti{$\sin^4x = \dfrac{3}{8} - \dfrac{1}{2}\cos2x + \dfrac{1}{8}\cos4x$。}\jiange

\end{xiaoxiaotis}

\xiaoti{}
\begin{xiaoxiaotis}

    \vspace{-1.4em} \begin{minipage}{0.9\textwidth}
    \xiaoxiaoti{已知 $\tan\alpha = 2$,求 $\sin2\alpha$,$\cos2\alpha$,$\tan2\alpha$ 的值。}\jiange
    \end{minipage}

    \xiaoxiaoti{已知 $\tan\theta = \dfrac{b}{a}$,求证 $a\cos2\theta + b\sin2\theta = a$;}\jiange

    \xiaoxiaoti{已知 $\tan\dfrac{\alpha}{2} = \dfrac{m}{n}$,求 $m\cos\alpha + n\sin\alpha$ 的值。}\jiange

\end{xiaoxiaotis}

\xiaoti{设 $\sin\alpha$ 与 $\sin\dfrac{\alpha}{2}$ 的比为 $8:5$,求 $\cos\alpha$,$\cot\dfrac{\alpha}{4}$ 的值。}\jiange

\xiaoti{在一块半圆形的铁板中截出一块面积最大的矩形,应该怎样截取?求出这个矩形的面积。}

\xiaoti{已知 $\tan^2\theta = 2\tan^2\phi + 1$,求证 $\cos2\theta + \sin^2\phi = 0$。}

\xiaoti{已知 $x + y = 3 - \cos4\theta$,$x - y = 4\sin2\theta$,求证 $x^{\frac{1}{2}} + y^{\frac{1}{2}} = 2$。}

\end{xiaotis}
