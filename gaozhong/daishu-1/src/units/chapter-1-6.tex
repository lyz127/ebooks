\subsection{函数的单调性}

我们在研究一次函数、二次函数和幂函数时,根据函数的图象研究了函数在某个区间上增大或减小的性质。

一般地,对于给定区间上的函数$f(x)$:

1. 如果对于属于这个区间的任意两个自变量的值 $x_1$,$x_2$,当$x_1 < x_2$时,都有$f(x_1) < f(x_2)$,
那么就说 $f(x)$ 在这个区间上是\textbf{增函数}(图\ref{fig:1-18}(1));

2. 如果对于属于这个区间的任意两个自变量的值 $x_1$,$x_2$,当$x_1 < x_2$时,都有$f(x_1) > f(x_2)$,
那么就说 $f(x)$ 在这个区间上是\textbf{减函数}(图\ref{fig:1-18}(2))。

\begin{figure}[htbp]
    \centering
    \begin{minipage}{7cm}
    \centering
    \begin{tikzpicture}[>=Stealth,scale=0.8]
    \draw [->] (-1,0) -- (5.5,0) node[anchor=north] {$x$};
    \draw [->] (0,-1) -- (0,3.5) node[anchor=east] {$y$};
    \node at (-0.3,-0.3) {$O$};

    \draw [name path=a1] (0.2,1.5) .. controls(1,1.5) and (3.6,1.7) .. (4.8,3.0);
    \node at (1.2,2) {$y = f(x)$};

    \path [name path=a2] (1.0,0) -- (1.0,4);
    \draw [name intersections={of=a1 and a2, by=A}]
        let \p1=(A)
        in  
            (\x1,\y1) -- (\x1, 0)
            node[anchor=north] {$x1$}
            (0.65cm+\x1,0.6) node {$f(x_1)$};

    \path [name path=a3] (3.3,0) -- (3.3,4);
    \draw [name intersections={of=a1 and a3, by=B}]
        let \p1=(B)
        in  
            (\x1,\y1) -- (\x1, 0)
            node[anchor=north] {$x2$}
            (0.65cm+\x1,1.0) node {$f(x_2)$};
\end{tikzpicture}

    \caption*{(1)}
    \end{minipage}
    \qquad
    \begin{minipage}{7cm}
    \centering
    \begin{tikzpicture}[>=Stealth,scale=0.8]
    \draw [->] (-1,0) -- (5.5,0) node[anchor=north] {$x$};
    \draw [->] (0,-1) -- (0,3.5) node[anchor=east] {$y$};
    \node at (-0.3,-0.3) {$O$};

    \draw [name path=a1] (0.2,3.0) .. controls(1,1.7) and (3.6,1.5) .. (4.8,1.5);
    \node at (3.3,2.3) {$y = f(x)$};

    \path [name path=a2] (1.0,0) -- (1.0,4);
    \draw [name intersections={of=a1 and a2, by=A}]
        let \p1=(A)
        in  
            (\x1,\y1) -- (\x1, 0)
            node[anchor=north] {$x1$}
            (0.65cm+\x1,1.0) node {$f(x_1)$};

    \path [name path=a3] (3.3,0) -- (3.3,4);
    \draw [name intersections={of=a1 and a3, by=B}]
        let \p1=(B)
        in  
            (\x1,\y1) -- (\x1, 0)
            node[anchor=north] {$x2$}
            (0.65cm+\x1,0.6) node {$f(x_2)$};
\end{tikzpicture}
    \caption*{(2)}
    \end{minipage}
    \caption{}\label{fig:1-18}
\end{figure}

如果函数 $y = f(x)$ 在某个区间上是增函数或减函数,就说 $f(x)$ 在这一区间上具有(严格的)\textbf{单调性},
这一区间叫做 $f(x)$ 的\textbf{单调区间}。

\liti 图\ref{fig:1-19} 是定义在闭区间 $[-5, 5]$ 上的函数 $f(x)$ 的图象。
根据图象说出 $f(x)$ 的单调区间,以及在每一单调区间上 $f(x)$ 是增函数还是减函数。

\begin{figure}[htbp]
    \centering
    \begin{tikzpicture}[>=Stealth,scale=0.8]
    \draw [->] (-5.5,0) -- (5.5,0) node[anchor=north] {$x$};
    \draw [->] (0,-3) -- (0,3.5) node[anchor=east] {$y$};
    \node at (-0.3,-0.3) {$O$};
    \foreach \x in {-5,-4,-3,-1,1,2,...,5} {
        \draw (\x,0.2) -- (\x,0) node[anchor=north] {$\x$};
    }
    \foreach \x in {-2} {
        \draw (\x,0) -- (\x,0.2) (\x-0.1, 0.2) node[anchor=south] {$\x$};
    }
    \foreach \y in {-2,-1,1,2,3} {
        \draw (0,\y) -- (0.2,\y) node[anchor=west] {$\y$};
    }
    
    \coordinate (A) at (-5, 2.3);
    \coordinate (B) at (-2, -2.5);
    \coordinate (C) at (1, 2.8);
    \coordinate (D) at (3, 1.6);
    \coordinate (E) at (5, 2.3);

    \draw (-5,0) -- (A) .. controls(-3.8,1.7) and (-3.6,-2.5) .. (B);
    \draw (-2,0) -- (B) .. controls(-0.7,-2.4) and (-0.3,2.7) .. (C);
    \draw (1,0) -- (C) .. controls(1.8,2.7) and (2.2,1.6) .. (D);
    \draw (3,0) -- (D) .. controls(4.2,1.7) and (4.6,2.1) .. (E) -- (5,0);
    \node at (3,3) {$y = f(x)$};
\end{tikzpicture}

    \caption{}\label{fig:1-19}
\end{figure}

\jie 函数 $f(x)$ 的单调区间有 $[-5,-2]$,$[-2,1]$,$[1,3]$,$[3,5]$。
其中,$f(x)$ 在区间 $[-5,-2]$,$[1,3]$ 上是减函数,在区间 $[-2,1]$,$[3,5]$ 上是增函数。

\liti 证明函数 $f(x) = 3x + 2$ 在 $(-\infty, +\infty)$ 上是增函数。

\zhengming 设 $x_1$,$x_2$ 是任意两个实数,且 $x_1 < x_2$,则

\begin{minipage}{10cm}
\begin{align*}
    & f(x_1) = 3x_1 + 2, \qquad f(x_2) = 3x_2 + 2 \text{。} \\
    & f(x_2) - f(x_1) = (3x_2 + 2) - (3x_1 + 2) = 3(x_2 - x_1) \text{。} \\
    \because \quad & x_2 > x_1, \qquad x_2 - x_1 > 0, \\
    \therefore \quad & f(x_2) - f(x_1) > 0, \qquad f(x_2) > f(x_1) \text{。}
\end{align*}
\end{minipage}

所以 $f(x) = 3x + 2$ 在 $(-\infty, +\infty)$ 上是增函数。

\vspace{0.5em}
\liti 证明函数 $f(x) = \dfrac 1 x$ 在 $(0, +\infty)$ 上是减函数。
\vspace{0.5em}

\zhengming  设 $x_1 > 0$,$x_2 > 0$,且 $x_1 < x_2$,则

\begin{minipage}{10cm}
    \begin{align*}
        & f(x_1) = \dfrac 1 {x_1}, \qquad f(x_2) = \dfrac 1 {x_2} \text{。} \\
        & f(x_2) - f(x_1) = \dfrac 1 {x_2} - \dfrac 1 {x_1} = \dfrac {x_1 - x_2} {x_1x_2} \text{。}
    \end{align*}
\end{minipage}

由 $x_1 > 0$,$x_2 > 0$,得 $x_1x_2 > 0$;又由 $x_1 < x_2$,得 $x_1 - x_2 < 0$。于是
$$ f(x_2) - f(x_1) < 0, \qquad f(x_2) < f(x_1) \text{。}$$

所以 $f(x) = \dfrac 1 x$ 在 $(0, +\infty)$ 上是减函数。

\lianxi

\begin{xiaotis}

\xiaoti{如图,已知函数 $f(x)$, $g(x)$ 的图象(包括端点),根据图象说出函数的单调区间,
以及在每一单调区间上,函数是增函数还是减函数。}

\begin{figure}[htbp]
    \centering
    \begin{minipage}{7cm}
    \centering
    \begin{tikzpicture}[>=Stealth]
    \draw [->] (-3,0) -- (3,0) node[anchor=north] {$x$};
    \draw [->] (0,-1.5) -- (0,2.5) node[anchor=east] {$y$};
    \node at (0.3,-0.3) {$O$};
    \foreach \x in {1,2} {
        \draw (\x,0.2) -- (\x,0) node[anchor=north] {$\x$};
    }
    \foreach \x in {-2,-1} {
        \draw (\x,0) -- (\x,0.2) (\x-0.1, 0.2) node[anchor=south] {$\x$};
    }
    
    \coordinate (A) at (-2, -0.4);
    \coordinate (B) at (-1, -1.0);
    \coordinate (C) at (0, 1.5);
    \coordinate (D) at (1, 0.9);
    \coordinate (E) at (2, 2);

    \draw (-2,0) -- (A) .. controls(-1.8,-0.6) and (-1.2,-1.0) .. (B);
    \draw (-1,0) -- (B) .. controls(-0.4,-1.0) and (-0.3,1.5) .. (C);
    \draw (0,0) -- (C) .. controls(0.4,1.5) and (0.7,0.9) .. (D);
    \draw (1,0) -- (D) .. controls(1.4,1.5) and (1.7,1.9) .. (E) -- (2,0);
    \node at (0.95,2.1) {$y = f(x)$};
\end{tikzpicture}

    \caption*{(1)}
    \end{minipage}
    \qquad
    \begin{minipage}{8cm}
    \centering
    \begin{tikzpicture}[>=Stealth]
    \draw [->] (-3.5,0) -- (3.5,0) node[anchor=north] {$x$};
    \draw [->] (0,-1.5) -- (0,1.5) node[anchor=east] {$y$};
    \node at (0.3,-0.3) {$O$};
    \draw let \n{pi} = {3.1415}
        in
            (-\n{pi},0) -- (-\n{pi},0.2) node[anchor=south] {$-\pi$}
            (-\n{pi}/2,0) -- (-\n{pi}/2,0.2) node[anchor=south] {$-\frac {\pi} {2}$}
            (\n{pi}/2,0.2) -- (\n{pi}/2,0) node[anchor=north] {$\frac {\pi} {2}$}
            (\n{pi},0.2) -- (\n{pi},0) node[anchor=north] {$\pi$}
            [domain=-\n{pi}:\n{pi},samples=50] plot (\x, {sin(\x r)});
    
    \node at (1.5,1.3) {$y = g(x)$};
\end{tikzpicture}

    \caption*{(2)}
    \end{minipage}
    \caption*{(第1题)}
\end{figure}

\xiaoti{证明函数 $f(x) = -2x + 1$ 在 $(-\infty, +\infty)$ 上是减函数。}

\vspace{0.5em}
\xiaoti{证明函数 $f(x) = \dfrac 3 x$ 在 $(-\infty, 0)$ 上是减函数。}
\vspace{0.5em}

\xiaoti{证明函数 $f(x) = x^2 + 1$ 在 $(0, +\infty)$ 上是增函数。}

\end{xiaotis}
