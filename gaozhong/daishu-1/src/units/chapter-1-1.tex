\subsection{集合}

考察下面几组对象:

\begin{enumerate}[label=(\arabic*), labelindent=\parindent, leftmargin=*]
    \item 1, 2, 3, 4, 5;
    \item 与一个角的两边距离相等的所有的点;
    \item 所有的直角三角形;
    \item $x^2$, $3x + 2$, $5y^3 - x$, $x^2 + y^2$;
    \item 某农场所有的拖拉机。
\end{enumerate}

它们分别是由一些数、一些点、一些图形、一些整式、一些物体组成的。我们说,每一组对象的全体形成一个\textbf{集合}(有时也简称\textbf{集})。集合里的各个对象叫做这个集合的\textbf{元素}。例如,(1)是由数1, 2, 3, 4, 5 组成的集合,其中的对象1, 2, 3, 4, 5都是这个集合的元素。

含有有限个元素的集合叫做\textbf{有限集},上面(1),(4),(5)这三个集合都是有限集;含有无限个元素的集合叫做\textbf{无限集},上面(2),(3)这两个集合都是无限集。

对于一个给定的集合,集合中的元素是确定的,这就是说,任何一个对象或者是这个给定集合的元素,或者不是它的元素。例如,对于由所有的直角三角形组成的集合,内角分别为$30^\circ$,$60^\circ$,$90^\circ$ 的三角形,是这个集合的元素,而内角分别为$50^\circ$,$60^\circ$,$70^\circ$的三角形,就不是这个集合的元素。

对于一个给定的集合,集合中的元素是互异的,这就是说,集合中的任何两个元素都是不同的对象;相同的对象归入任何一个集合时,只能算作这个集合的一个元素。因此,集合中的元素是没有重复现象的。

集合的表示方法,常用的有列举法和描述法。

把集合中的元素一一列举出来,写在大括号内表示集合的方法,叫做\textbf{列举法}。

例如,由数1, 2, 3, 4, 5组成的集合,可以表示为

$$\{1,\ 2,\ 3,\ 4,\ 5\}$$

又如,由整式$x^2$,$3x + 2$, $5y^3 - x$, $x^2 + y^2$ 组成的集合,可以表示为

$$\{x^2,\ 3x+2,\ 5y^3-x,\ x^2+y^2\}$$

用列举法表示集合时,不必考虑元素之间的顺序。例如由四个元素$-3$, 0, 2, 5组成的集合,可以表示为$\{-3,\ 0,\ 2,\ 5\}$,也可以表示为$\{0,\ 2,\ -3,\ 5\}$,等等。

应该注意,$a$与$\{a\}$是不同的:$a$表示一个元素;$\{a\}$表示一个集合,这个集合只有一个元素$a$。

把集合中的元素的公共属性描述出来,写在大括号内表示集合的方法,叫做\textbf{描述法}。这时往往在大括号内先写上这个集合的元素的一般形式,再划一条竖线,在竖线右边写上这个集合的元素的公共属性。

例如:

由不等式$x-3>2$的所有的解组成的集合(即$x-3>2$的解集),可以表示为

$$\{x\ |\ x-3>2\} \qquad \footnote{有的书上用冒号或分号代替竖线,如$\{x\ :\ x-3>2\}$ 或 $\{x\ ;\ x-3>2\}$。} $$

由抛物线$y=x^2+1$上所有的点的坐标组成的集合,可以表示为

$$\{(x, y)\ |\ y = x^2 + 1\}$$

在不引起混淆的情况下,为了简便,有些集合用描述法表示时,可以省去竖线及其左边的部分。例如,由所有的直角三角形组成的集合,可以表示为

$$\{\text{直角三角形}\}$$

由所有的小于6的正整数组成的集合,可以表示为

$$\{\text{小于6的正整数}\}$$

集合通常用大写的拉丁字母表示,集合的元素用小写的拉丁字母表示。如果$a$是集合$A$的元素,就说\textbf{$a$属于集合$A$},记作$a \in A$;如果 $a$ 不是集合 $A$ 的元素,就说\textbf{$a$不属于$A$},记作$a \notin A$(或$a \bar{\in} A$)。例如,设 $B$ 表示集合$\{1,\ 2,\ 3,\ 4,\ 5\}$,则

$$5 \in B, \qquad \dfrac{3}{2} \notin B$$

全体自然数的集合通常简称\textbf{自然数集},记作$N$;

全体整数的集合通常简称\textbf{整数集},记作$Z$;

全体有理数的集合通常简称\textbf{有理数集},记作$Q$;

全体实数的集合通常简称\textbf{实数集},记作$R$。

为了方便起见,有时我们还用 $Q^+$ 表示正有理数集,用 $R^-$ 表示负实数集,等等。

\lianxi

(口答)下面集合里的元素是什么(第1~5题)?

1. $\{\text{大于3小于11的偶数}\}$。

2. $\{\text{平方后等于1的数}\}$。

3. $\{\text{平方后仍等于原数的数}\}$。

4. $\{\text{比2大3的数}\}$。

5. $\{\text{一年中有31天的月份}\}$。

在下列各题中,分别指出了一个集合的所有元素,用适当的方法把这个集合表示出来,然后说出它是有限集还是无限集(第6~10题):

6. 水星、金星、地球、火星、木星、土星、天王星、海王星、冥王星。

7. 周长等于20厘米的三角形。

8. 长江、黄河、珠江、黑龙江。

9. 不等式$x^2 + 5x + 6 >0$ 的解。

10. 大于0的偶数。

下列集合用另一种方法表示出来(第11~13题):

11. $\{2, 4, 6, 8, 10\}$。

12. $\{\text{目前世界乒乓球锦标賽的七个比賽项目\}}$。

13. $\{\text{中国古代四大发明}\}$。

14. 用符号 $\in$ 或 $\notin$ 填空:

\begin{tabular}{*{5}{p{5em}}}
    $1 \xhx N$,& $0 \xhx N$, & $-3 \xhx N$, & $0.5 \xhx N$, & $\sqrt{2} \xhx N$;\\
    $1 \xhx Z$,& $0 \xhx Z$, & $-3 \xhx Z$, & $0.5 \xhx Z$, & $\sqrt{2} \xhx Z$;\\
    $1 \xhx Q$,& $0 \xhx Q$, & $-3 \xhx Q$, & $0.5 \xhx Q$, & $\sqrt{2} \xhx Q$;\\
    $1 \xhx R$,& $0 \xhx R$, & $-3 \xhx R$, & $0.5 \xhx R$, & $\sqrt{2} \xhx R$;\\
\end{tabular}
