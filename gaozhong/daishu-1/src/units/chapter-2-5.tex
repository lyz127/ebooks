\subsection{诱导公式}\label{subsec:2-5}

我们知道,$0^\circ$ \~{} $90^\circ$ 间的角的三角函数值,可以通过查表求得。另外,利用第 \ref{subsec:2-3}
节的 \hyperref[gongshi:1]{公式一},可以把求任意角的三角函数值转化为求 $0^\circ$ \~{} $360^\circ$ 间的角
的三角函数值。因此,如果能把求 $90^\circ$ \~{} $360^\circ$ 间的角的三角函数值转化为求 $0^\circ$ \~{} $90^\circ$
间的角的三角函数值,那么任意角的三角函数值就都能通过查表来求了。

对于 $90^\circ$ \~{} $360^\circ$ 间的角,可用下面的形式来表示:

设 $0^\circ \leqslant \alpha \leqslant 90^\circ$,那么

$90^\circ$ \~{} $180^\circ$ 间的角,可以写成 $180^\circ - \alpha$ ;

$180^\circ$ \~{} $270^\circ$ 间的角,可以写成 $180^\circ + \alpha$ ;

$270^\circ$ \~{} $360^\circ$ 间的角,可以写成 $360^\circ - \alpha$ 。

下面依次讨论 $180^\circ + \alpha$,$-\alpha$,$180^\circ - \alpha$,$360^\circ - \alpha$ 的三角函数值
与 $\alpha$ 的三角函数值之间的关系。为了使讨论更具有一般性,这里假定 $\alpha$ 为任意角。

\begin{figure}[htbp]
    \centering
    \begin{tikzpicture}[>=Stealth, scale=0.8]
    \draw [->] (-3.5,0) -- (3.5,0) node[anchor=north] {$x$};
    \draw [->] (0,-3.5) -- (0,3.5) node[anchor=east] {$y$};
    \node at (-0.3,-0.3) {$O$};
    \draw [name path=c] (0, 0) circle(2.5);

    \draw [name path=a1] (0,0) -- (140:3.5);
    \draw [->] (0.5, 0) arc (0:140:0.5);
    \node at (0.5, 0.5) {$\alpha$};
    \draw [name intersections={of=a1 and c, by=A}]
       let \p1=(A)
       in (A) +(-0.2, 0) node[anchor=east] {$P(x, y)$}
          (\x1,\y1) -- (\x1, 0) node [anchor=north] {$M$}
          (\x1, 0.3) -- +(0.3,0) -- +(0.3, -0.3);

    \draw [name path=a2] (0,0) -- (320:3.5);
    \draw [->] (1.0, 0) arc (0:320:1.0);
    \node at (1.1, 1.2) {$180^\circ + \alpha$};
    \draw [name intersections={of=a2 and c, by=A}]
      let \p1=(A)
      in (A) +(0.2, 0) node[anchor=west] {$P'(-x, -y)$}
        (\x1,\y1) -- (\x1, 0) node [anchor=south] {$N$}
        (\x1, -0.3) -- +(0.3,0) -- +(0.3, 0.3);
\end{tikzpicture}

    \caption{}\label{fig:2-15}
\end{figure}

如图 \ref{fig:2-15},以原点为圆心,等于单位长的线段为半径作一个圆〈这个圆叫做单位圆)。已知任意角 $\alpha$
的终边与这个圆相交于点 $P(x, y)$。由于角 $180^\circ + \alpha$ 的终边就是角 $\alpha$ 的终边的反向延长线,
角 $180^\circ + \alpha$ 的终边与单位圆的交点 $P'$,是与点 $P$ 关于点 $O$ 对称的,因此点 $P'$ 的坐标是
$(-x, -y)$。又因单位圆的半径 $r = 1$,由正弦函数和余弦函数的定义得到 \\
\begin{tabular}{p{5em}p{12em}l}
    & $\sin\alpha = y$, & $\cos\alpha = x$,\\
    & $\sin(180^\circ + \alpha) = -y$,& $\cos(180^\circ + \alpha) = -x$。
\end{tabular} \\
因此 \\
\begin{tabular}{p{5em}p{12em}l}
    & $\sin(180^\circ + \alpha) = -\sin\alpha$,& $\cos(180^\circ + \alpha) = -\cos\alpha$。
\end{tabular}

又根据同角三角函数间的基本关系式,有

\vspace{-1em}\begin{gather*}
    \tan(180^\circ + \alpha) = \dfrac{\sin(180^\circ + \alpha)}{\cos(180^\circ + \alpha)} = \dfrac{-\sin\alpha}{-\cos\alpha} = \tan\alpha \text{,} \\
    \cot(180^\circ + \alpha) = \dfrac{\cos(180^\circ + \alpha)}{\sin(180^\circ + \alpha)} = \dfrac{-\cos\alpha}{-\sin\alpha} = \cot\alpha \text{。}
\end{gather*}
\vspace{0.5em}

于是我们得到一组公式(\textbf{公式二}\mylabel{gongshi:2}):

\begin{center}
    \framebox{\begin{tabular}{p{13em}p{13em}}
            $\sin (180^\circ + \alpha) = -\sin \alpha,$ & $\cos (180^\circ + \alpha) = -\cos \alpha ,$ \\
            $\tan (180^\circ + \alpha) = \tan \alpha,$ & $\cot (180^\circ + \alpha) = \cot \alpha . $
    \end{tabular}}
\end{center}

我们再来研究任意角 $\alpha$ 与 $-\alpha$ 的三角函数值之间的关系。

\begin{wrapfigure}[12]{r}{6.5cm}
    \centering
    \begin{tikzpicture}[>=Stealth, scale=0.8]
    \draw [->] (-3.5,0) -- (3.5,0) node[anchor=north] {$x$};
    \draw [->] (0,-3.5) -- (0,3.5) node[anchor=east] {$y$};
    \node at (0.2,-0.2) {$O$};
    \draw [name path=c] (0, 0) circle(2.5);

    \draw [name path=a1] (0,0) -- (140:3.5);
    \draw (0.6, 0.6) node {$\alpha$} [->] (0.6, 0) arc (0:140:0.6);
    \draw [name intersections={of=a1 and c, by=A}]
       let \p1=(A)
       in (A) +(-0.2, 0) node[anchor=east] {$P(x, y)$}
          (A) -- (\x1, -\y1) +(-0.2, 0) node[anchor=east] {$P'(x, -y)$}
          (\x1, -0.3) -- +(0.3,0) -- +(0.3, 0.3) node [anchor=south] {$M$};
    \draw [name path=a1] (0,0) -- (-140:3.5);
    \draw (0.6, -0.6) node {$-\alpha$} [->] (0.6, 0) arc (0:-140:0.6);
\end{tikzpicture}

    \vspace{-20pt}
    \caption{}\label{fig:2-16}
\end{wrapfigure}


如图 \ref{fig:2-16},任意角 $\alpha$ 的终边与单位圆相交于点 $P(x, y)$,角 $-\alpha$ 的终边与单位圆相交于点 $P'$。
由于角 $\alpha$ 与 $-\alpha$ 是由射线从 $x$ 轴的正半轴开始,按相反的方向绕原点作相同大小的旋转而成的,这两个角的
终边关于 $x$ 轴对称。因此,点 $P'$ 的坐标为 $(x, -y)$。由于 $r = 1$,我们得到
$$\sin(-\alpha) = -y, \qquad \cos(-\alpha) = x \text{,}$$
从而,

\vspace{-1em}\begin{gather*}
    \sin(-\alpha) = -\sin\alpha \text{,} \qquad \cos(-\alpha) = \cos\alpha \text{,} \\
    \tan(-\alpha) = \dfrac{\sin(-\alpha)}{\cos(-\alpha)} = \dfrac{-\sin\alpha}{\cos\alpha} = -\tan\alpha \text{,} \\
    \cot(-\alpha) = \dfrac{\cos(-\alpha)}{\sin(-\alpha)} = \dfrac{\cos\alpha}{-\sin\alpha} = -\cot\alpha \text{。}
\end{gather*}
\vspace{0.5em}

于是得到一组公式(\textbf{公式三}\mylabel{gongshi:3}):

\begin{center}
    \framebox{\begin{tabular}{p{13em}p{13em}}
            $\sin (-\alpha) = -\sin \alpha,$ & $\cos (-\alpha) = \cos \alpha ,$ \\
            $\tan (-\alpha) = -\tan \alpha,$ & $\cot (-\alpha) = -\cot \alpha . $
    \end{tabular}}
\end{center}

\liti 求下列各三角函数值:
\begin{xiaoxiaotis}

    \begin{tabular}[t]{*{2}{@{}p{15em}}}
        \xiaoxiaoti{$\cos 225^\circ$;} & \xiaoxiaoti{$\tan \dfrac 4 3 \pi$;} \\
        \xiaoxiaoti{$\sin \dfrac{11}{10} \pi$;} & \xiaoxiaoti{$\cot 200^\circ 18'$。}
    \end{tabular}

\end{xiaoxiaotis}

\jie (1)$\cos 225^\circ = \cos(180^\circ + 45^\circ) = -\cos 45^\circ = -\dfrac{\sqrt{2}}{2}$;

\vspace{0.5em}
(2)$\tan \dfrac 4 3 \pi = \tan \left( \pi + \dfrac \pi 3 \right) = \tan \dfrac \pi 3 = \sqrt{3}$;
\vspace{0.5em}

(3)$\sin \dfrac{11}{10} \pi = \sin \left( \pi + \dfrac \pi {10} \right) = -\sin \dfrac \pi {10} = -\sin 18^\circ = -0.3090$;
\vspace{0.5em}

(4)$\cot 200^\circ 18' = \cot (180^\circ + 20^\circ 18') = \cot 20^\circ 18' = 2.703$。

\liti 求下列各三角函数值:
\begin{xiaoxiaotis}

    \begin{tabular}[t]{*{2}{@{}p{15em}}}
        \xiaoxiaoti{$\sin \left( -\dfrac \pi 3 \right)$;} & \xiaoxiaoti{$\tan(-210^\circ)$;} \\
        \xiaoxiaoti{$\cos(-240^\circ 12')$;} & \xiaoxiaoti{$\cot(-400^\circ)$。}
    \end{tabular}

\end{xiaoxiaotis}

\jie (1)$\sin \left( -\dfrac \pi 3 \right) = -\sin \dfrac \pi 3 = -\dfrac{\sqrt{3}}{2}$;

(2)$\tan(-210^\circ) = -\tan 210^\circ = -\tan(180^\circ + 30^\circ) = -\tan 30^\circ = -\dfrac{\sqrt{3}}{3}$;
\vspace{0.5em}

(3)$\cos(-240^\circ 12') = \cos 240^\circ 12' = \cos(180^\circ + 60^\circ 12') = -\cos 60^\circ 12' = -0.4970$;

(4)$\cot(-400^\circ) = -\cot 400^\circ = -\cot(360^\circ + 40^\circ) = -\cot 40^\circ = -1.1918$。

\liti 化简
$$\dfrac{\sin(180^\circ + \alpha) \cdot \cos(360^\circ + \alpha)}{\cot(-\alpha - 180^\circ) \cdot \sin(-180^\circ - \alpha)} \text{。}$$

\jie $\begin{aligned}[t]
    \because \quad &\cot(-\alpha - 180^\circ) = \cot[-(180^\circ + \alpha)] = -\cot(180^\circ + \alpha) = -\cot\alpha , \\
    &\sin(-180^\circ - \alpha) = \sin[-(180^\circ + \alpha)] = -\sin(180^\circ + \alpha) = -(-\sin\alpha) = \sin\alpha ,
\end{aligned}$

\vspace{0.5em}
$\therefore \quad \dfrac{\sin(180^\circ + \alpha) \cdot \cos(360^\circ + \alpha)}{\cot(-\alpha - 180^\circ) \cdot \sin(-180^\circ - \alpha)}
= \dfrac{(-\sin\alpha) \cdot \cos\alpha}{(-\cot\alpha) \cdot \sin\alpha} = \dfrac{\cos\alpha}{\cot\alpha} = \sin\alpha
$。
\vspace{0.5em}

\lianxi
\begin{xiaotis}

\xiaoti{求下列各三角函数值:}
\begin{xiaoxiaotis}

    \begin{tabular}[t]{*{2}{@{}p{15em}}}
        \xiaoxiaoti{$\tan 210^\circ$;} & \xiaoxiaoti{$\cos \dfrac{13}{9} \pi$;} \\
        \xiaoxiaoti{$\sin(1 + \pi)$;} & \xiaoxiaoti{$\cot 253^\circ 18'$。}
    \end{tabular}

\end{xiaoxiaotis}

\xiaoti{求下列各三角函数值:}
\begin{xiaoxiaotis}

    \renewcommand\arraystretch{2}
    \begin{tabular}[t]{*{2}{@{}p{15em}}}
        \xiaoxiaoti{$\cot(-45^\circ)$;} & \xiaoxiaoti{$\sin \left( -\dfrac \pi 6 \right)$;} \\
        \xiaoxiaoti{$\cos(-70^\circ 6')$;} & \xiaoxiaoti{$\tan \left( -\dfrac{5}{18} \pi \right)$。}
    \end{tabular}

\end{xiaoxiaotis}

\xiaoti{求下列各三角函数值:}
\begin{xiaoxiaotis}

    \renewcommand\arraystretch{2}
    \begin{tabular}[t]{*{2}{@{}p{15em}}}
        \xiaoxiaoti{$\cos(-420^\circ)$;} & \xiaoxiaoti{$\tan(-800^\circ)$;} \\
        \xiaoxiaoti{$\sin \left(-\dfrac 7 6 \pi \right)$;} & \xiaoxiaoti{$\cot \left( -\dfrac 4 3 \pi \right)$;} \\
        \xiaoxiaoti{$\sin(-1300^\circ)$;} & \xiaoxiaoti{$\cos \left( -\dfrac{79}{6} \pi \right)$。}
    \end{tabular}

\end{xiaoxiaotis}

\xiaoti{化简:}
\begin{xiaoxiaotis}

    \xiaoxiaoti{$\dfrac{\sin(\alpha + 180^\circ) \cos(-\alpha)}{\cot(-\alpha - 180^\circ)}$;}
    \vspace{0.5em}

    \xiaoxiaoti{$\sin^3(-\alpha) \cos(2\pi + \alpha) \tan(-\alpha - \pi)$。}

\end{xiaoxiaotis}

\end{xiaotis}

\vspace{2em}
我们利用 \hyperref[gongshi:2]{公式二} 和 \hyperref[gongshi:3]{公式三},可以推出 $180^\circ - \alpha$ 与 $\alpha$ 的三角函数值之间的关系:

\vspace{-1em}\begin{align*}
    \sin(180^\circ - \alpha) &= \sin[180^\circ + (-\alpha)] = -\sin(-\alpha) = \sin\alpha ; \\
    \cos(180^\circ - \alpha) &= \cos[180^\circ + (-\alpha)] = -\cos(-\alpha) = -\cos\alpha ; \\
    \tan(180^\circ - \alpha) &= \tan[180^\circ + (-\alpha)] = \tan(-\alpha) = -\tan\alpha ; \\
    \cot(180^\circ - \alpha) &= \cot[180^\circ + (-\alpha)] = \cot(-\alpha) = -\cot\alpha \text{。}
\end{align*}

于是又得到一组公式(\textbf{公式四}\mylabel{gongshi:4}):

\begin{center}
    \framebox{\begin{tabular}{p{13em}p{13em}}
            $\sin (180^\circ - \alpha) = \sin \alpha,$ & $\cos (180^\circ - \alpha) = -\cos \alpha ,$ \\
            $\tan (180^\circ - \alpha) = -\tan \alpha,$ & $\cot (180^\circ - \alpha) = -\cot \alpha . $
    \end{tabular}}
\end{center}

同学们还可以利用 \hyperref[gongshi:1]{公式一} 和 \hyperref[gongshi:3]{公式三},自己推证 $360^\circ - \alpha$
与 $\alpha$ 的三角函数值之间的关系(\textbf{公式五}\mylabel{gongshi:5}):

\begin{center}
    \framebox{\begin{tabular}{p{13em}p{13em}}
            $\sin (360^\circ - \alpha) = -\sin \alpha,$ & $\cos (360^\circ - \alpha) = \cos \alpha ,$ \\
            $\tan (360^\circ - \alpha) = -\tan \alpha,$ & $\cot (360^\circ - \alpha) = -\cot \alpha . $
    \end{tabular}}
\end{center}

\hyperref[gongshi:1]{公式一}、\hyperref[gongshi:2]{二}、\hyperref[gongshi:3]{三}、
\hyperref[gongshi:4]{四}、\hyperref[gongshi:5]{五}都叫做 \textbf{诱导公式}。

上面这些诱导公式,可以概括如下:

\textbf{$k \cdot 360^\circ + \alpha \, (k \in Z)$,$-\alpha$,$180^\circ \pm \alpha$,
$360^\circ - \alpha$ 的三角函数值等于 $\alpha$ 的同名函数值,前面加上一个把 $\alpha$ 看成
锐角时原函数值的符号。}

利用诱导公式求任意角的三角函数值,一般可按下面的步骤进行:

\begin{figure}[htbp]
    \centering
    \begin{tikzpicture}[>=Stealth]
    \node[draw, align=left] at (0,2) {任意负角的 \\ 三角函数};
    \node at (2.3,2.4) {用公式三、一};
    \draw[->] (1.2,2) -- (3.7,2);

    \node[draw, align=left] at (4.8,2) {任意正角的 \\ 三角函数};
    \node at (6.8,2.4) {用公式一};
    \draw[->] (6,2) -- (7.8,2);

    \node[draw, align=left] at (0.2,0) {$0^\circ$ \~{} $360^\circ$ 间角 \\ 的三角函数};
    \node at (2.8,0.4) {用公式};
    \draw[->] (1.7,0) -- (3.9,0);
    \node at (2.8,-0.4) {二、四、五};

    \node[draw, align=left] at (5.2,0) {$0^\circ$ \~{} $90^\circ$ 间角 \\ 的三角函数};
    \node at (7.0,0.4) {查表};
    \draw[->] (6.6,0) -- (7.5,0);

    \node[draw, align=left] at (8,0) {求 \\ 值};
\end{tikzpicture}

\end{figure}

\liti 求下列各三角函数值:
\begin{xiaoxiaotis}

    \renewcommand\arraystretch{2}
    \begin{tabular}[t]{*{2}{@{}p{15em}}}
        \xiaoxiaoti{$\tan \dfrac 3 4 \pi$;} & \xiaoxiaoti{$\cos (-150^\circ 15')$;} \\
        \xiaoxiaoti{$\sin \dfrac{11}{6} \pi$;} & \xiaoxiaoti{$\cot 310^\circ 18'$。}
    \end{tabular}

\end{xiaoxiaotis}

\jie (1) $\tan \dfrac 3 4 \pi = \tan \left( \pi - \dfrac \pi 4 \right) = -\tan \dfrac \pi 4 = -1$;
\vspace{0.5em}

(2) $\cos (-150^\circ 15') = \cos 150^\circ 15' = \cos (180^\circ - 29^\circ 45') = -\cos 29^\circ 45' = -0.8682$;

\vspace{0.5em}
(3)$\sin \dfrac{11}{6} \pi = \sin \left( 2\pi - \dfrac \pi 6 \right) = -\sin \dfrac \pi 6 = -\dfrac 1 2$;
\vspace{0.5em}

(4)$\cot 310^\circ 18' = \cot (360^\circ - 49^\circ 42') = -\cot 49^\circ 42' = -0.8481$。

\liti 求下列各三角函数值:
\begin{xiaoxiaotis}

    \renewcommand\arraystretch{2}
    \begin{tabular}[t]{*{2}{@{}p{15em}}}
        \xiaoxiaoti{$\cos 519^\circ$;} & \xiaoxiaoti{$\sin \left( -\dfrac{17}{3} \pi \right)$;} \\
        \xiaoxiaoti{$\cot(-1665^\circ)$;} & \xiaoxiaoti{$\tan(-324^\circ 18')$。}
    \end{tabular}

\end{xiaoxiaotis}

\jie (1) $\cos 519^\circ = \cos(360^\circ + 159^\circ) = \cos 159^\circ = \cos(180^\circ - 21^\circ) = -\cos 21^\circ = -0.9336$;

(2)$\sin \left( -\dfrac{17}{3} \pi \right) = \sin \left( -3 \times 2\pi + \dfrac \pi 3 \right) = \sin \dfrac \pi 3 = \dfrac{\sqrt{3}}{2}$;

(3)$\begin{aligned}[t]
      &\cot(-1665^\circ) = -\cot 1665^\circ \\
    = &-\cot(4 \times 360^\circ + 225^\circ) \\
    = &-\cot 225^\circ = -\cot(180^\circ + 45^\circ) \\
    = &-\cot 45^\circ = -1 \text{;}
\end{aligned}$

(4)$\tan(-324^\circ 18') = \tan(-360^\circ + 35^\circ 42') = \tan 35^\circ 42' = 0.7186$。

\liti 求证
$$\dfrac{\sin(2\pi - \alpha) \tan(\pi + \alpha) \cot(-\alpha - \pi)}{\cos(\pi - \alpha) \tan(3\pi -\alpha)} = 1 \text{。}$$

\zhengming  $\begin{aligned}[t]
      &\dfrac{\sin(2\pi - \alpha) \tan(\pi + \alpha) \cot(-\alpha - \pi)}{\cos(\pi - \alpha) \tan(3\pi -\alpha)} \\
    = &\dfrac{(-\sin\alpha) \tan\alpha [-\cot(\pi + \alpha)]}{(-\cos\alpha) \tan(\pi - \alpha)} \\
    = &\dfrac{(-\sin\alpha) \tan\alpha (-\cot \alpha)}{(-\cos\alpha) (-\tan\alpha)} \\
    = &\dfrac{\sin\alpha}{\cos\alpha} \cdot \dfrac{\cos\alpha}{\sin\alpha} \\
    = & 1 \text{。}
\end{aligned}$

\vspace{2em}
\lianxi
\begin{xiaotis}
\setcounter{cntxiaoti}{0}

\xiaoti{填写下表:}

\begin{table}[H]
    \renewcommand\arraystretch{2}
    \hspace{4em}
    \begin{tabular}{|w{c}{4em}|*{4}{w{c}{6em}|}}
        \hline
        $\alpha$ & $\sin \alpha$ & $\cos \alpha$ & $\tan \alpha$ & $\cot \alpha$ \\ \hline
        $-\dfrac \pi 3$ & \eline{4} \\ \hline
        $\dfrac 2 3 \pi$ & \eline{4} \\ \hline
        $\dfrac 4 3 \pi$ & \eline{4} \\ \hline
        $\dfrac 5 3 \pi$ & \eline{4} \\ \hline
        $\dfrac 7 3 \pi$ & \eline{4} \\ \hline
    \end{tabular}
\end{table}

\xiaoti{求下列各三角函数值:}
\begin{xiaoxiaotis}

    \renewcommand\arraystretch{2}
    \begin{tabular}[t]{*{2}{@{}p{15em}}}
        \xiaoxiaoti{$\sin \dfrac 3 5 \pi$;} & \xiaoxiaoti{$\cos 100^\circ 21'$;} \\
        \xiaoxiaoti{$\cot \left( - \dfrac 3 4 \pi \right)$;} & \xiaoxiaoti{$\tan(-145^\circ 20')$;} \\
        \xiaoxiaoti{$\sin \dfrac{31}{36} \pi$;} & \xiaoxiaoti{$\cos 324^\circ 32'$。}
    \end{tabular}

\end{xiaoxiaotis}

\xiaoti{求下列各三角函数值:}
\begin{xiaoxiaotis}

    \renewcommand\arraystretch{2}
    \begin{tabular}[t]{*{2}{@{}p{15em}}}
        \xiaoxiaoti{$\cos \dfrac{65}{6} \pi$;} & \xiaoxiaoti{$\cot \dfrac{35}{3} \pi$;} \\
        \xiaoxiaoti{$\sin \left( - \dfrac{31}{4} \pi \right)$;} & \xiaoxiaoti{$\tan(-1596^\circ)$;} \\
        \xiaoxiaoti{$\cos(-1182^\circ 13')$;} & \xiaoxiaoti{$\sin 670^\circ 39'$。}
    \end{tabular}

\end{xiaoxiaotis}

\xiaoti{化简:}
\begin{xiaoxiaotis}

    \vspace{0.5em}
    \xiaoxiaoti{$\dfrac{\cos(\alpha - \pi) \cdot \tan(\alpha - 2\pi)}{\sin(\pi - \alpha) \cdot \cot(2\pi - \alpha)}$;}
    \vspace{0.5em}

    \xiaoxiaoti{$\sin^2(-\alpha) - \tan(360^\circ - \alpha) \cot(-\alpha) - \sin(180^\circ - \alpha) \cos(360^\circ - \alpha) \cot(\alpha + 180^\circ)$。}

\end{xiaoxiaotis}

\end{xiaotis}
