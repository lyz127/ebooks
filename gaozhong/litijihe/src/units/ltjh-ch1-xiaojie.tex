\xiaojie

一、本章的主要内容是有关空间的直线与直线、直线与平面以及平面与平面的位置关系和有关图形的画法。
着重研究的是它们之间的平行与垂直关系。


二、本章的四个公理是这一章内容的基础。
此外,平面几何里的定义、定理等,对于空间的任何平面内的平面图形仍然适用;
但对于非平面图形,则需要经过证明才能应用。
在解决立体几何的问题时,常把它转化为平面几何的问题来解决。


三、空间两条直线的位置关系有“平行”、“相交”、“异面”三种;
空间一条直线和一个平面的位置关系有“直线在平面内”、“平行”、“相交”三种;
两个平面的位置关系有“平行”、“相交”两种。


四、关于空间的直线与直线、直线与平面、平面与平面的平行与垂直关系的性质定理与判定定理是本章的中心问题。
应用这些定理时,要弄清定理的题设和结论。
判定定理的题设是结论成立的充分条件;性质定理的结论是题设成立的必要条件。
学完全章后,判定上述的平行和垂直关系的途径就更为广泛。
例如也可以用“垂直于同一个平面的两条直线必平行”去判定两条直线平行;
用“如果两个平面垂直,那么在一个平面内垂直于它们交线的直线垂直于另一个平面”去判定一条直线与一个平面垂直。


五、两条异面直线所成的角、直线与平面所成的角以及二面角都是通过平面几何中的角来定义的,
因而,它们都可以看作是平面几何中的角的概念在空间的拓广。

两条异面直线所成的角和二面角的定义都以定理“两边分别平行且方向相同的两个角相等”为基础。
而斜线和平面所成的角实际是用这条斜线和平面内的直线所成的角中最小的角来定义的。

两条异面直线的距离、直线和平面间的距离以及两个平行平面间的距离,都分别是它们的两点的距离中最小的。

