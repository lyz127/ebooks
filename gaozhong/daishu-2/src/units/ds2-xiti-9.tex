\xiti\mylabel{xiti-9}

\begin{xiaotis}

\xiaoti{用对角线法则计算:}
\begin{xiaoxiaotis}

    \renewcommand\arraystretch{1.2}
    \begin{tabular}[t]{*{2}{@{}p{16em}}}
        \xiaoxiaoti{$\begin{vmatrix*}[r]
                3  & -5 & 1 \\
                2  & 3  & -6 \\
                -7 & 2  & 4
            \end{vmatrix*}$;}
        & \xiaoxiaoti{$\begin{vmatrix*}[r]
                a & b & c \\
                0 & d & e \\
                0 & 0 & f
            \end{vmatrix*}$;} \\
        \xiaoxiaoti{$\begin{vmatrix*}
                0          & -\cos\alpha & -\cos\beta \\
                \cos\alpha & 0           & -\cos\gamma \\
                \cos\beta  & \cos\gamma  & 0
            \end{vmatrix*}$;}
        & \xiaoxiaoti{$\begin{vmatrix*}[r]
                a & h & g \\
                h & b & f \\
                g & f & c
            \end{vmatrix*}$。}
    \end{tabular}

\end{xiaoxiaotis}


\xiaoti{解方程}
\begin{xiaoxiaotis}

    \xiaoxiaoti{$\begin{vmatrix*}
            0   & x-1 & 1 \\
            x-1 & 0   & x-2 \\
            1   & x-2 & 0
        \end{vmatrix*} = 0$;}

    \xiaoxiaoti{$\begin{vmatrix*}
            x-1 & 1   & 1 \\
            1   & x-1 & 1 \\
            1   & 1   & x-1
        \end{vmatrix*} = 0$。}

\end{xiaoxiaotis}


\xiaoti{求证:}
\begin{xiaoxiaotis}

    \xiaoxiaoti{$\begin{vmatrix*}
            1 & \sin3\theta & \cos3\theta \\
            1 & \sin2\theta & \cos2\theta \\
            1 & \sin\theta  & \cos\theta
        \end{vmatrix*} = 2\sin\theta(1 - \cos\theta)$;}

    \xiaoxiaoti{$\begin{vmatrix*}
            2\cos\theta & 1           & 0 \\
            1           & 2\cos\theta & 1 \\
            0           & 1           & 2\cos\theta
        \end{vmatrix*} = \dfrac{\sin4\theta}{\sin\theta} \quad (\theta \neq k\pi,\; k \in Z)$。}

\end{xiaoxiaotis}


\xiaoti{利用行列式的性质计算:}
\begin{xiaoxiaotis}

    \renewcommand\arraystretch{1.2}
    \begin{tabular}[t]{*{2}{@{}p{16em}}}
        \xiaoxiaoti{$\begin{vmatrix*}[r]
                10 & 8  & -2 \\
                15 & 12 & -3 \\
                25 & 32 & 7
            \end{vmatrix*}$;}
        & \xiaoxiaoti{$\begin{vmatrix*}[r]
                \dfrac{1}{2} & \dfrac{1}{3} & \dfrac{1}{4} \\
                12 & 24  & 36 \\
                -5 & -4 & -3
            \end{vmatrix*}$;} \\
        \xiaoxiaoti{$\begin{vmatrix*}[r]
                554 & 427 & 327 \\
                586 & 443 & 343 \\
                711 & 504 & 404
            \end{vmatrix*}$。}
    \end{tabular}

\end{xiaoxiaotis}


\xiaoti{利用行列式的性质计算:}
\begin{xiaoxiaotis}

    \xiaoxiaoti{$\begin{vmatrix*}[r]
            -ab & bd  & bf \\
            ac  & -cd & cf \\
            ae  & de  & -ef
        \end{vmatrix*}$;}

    \xiaoxiaoti{$\begin{vmatrix*}
            a  & b     & c \\
            2a & 3a+2b & 4a+3b+2c \\
            3a & 6a+3b & 10a+9b+3c
        \end{vmatrix*}$。}

\end{xiaoxiaotis}


\xiaoti{下列计算过程中,哪些步骤是对的,哪些不对,应怎样改正?}
\begin{xiaoxiaotis}

    \xiaoxiaoti{$\begin{vmatrix*}
            a_1 & b_1 \\
            a_2 & b_2
        \end{vmatrix*} = \begin{vmatrix*}
            a_1 + ka_2 & b_1 + kb_2 \\
            a_2 - ha_1 & b_2 - hb_1
        \end{vmatrix*}$;}

    \xiaoxiaoti{$\begin{vmatrix*}
            a_1 & b_1 & c_1 \\
            a_2 & b_2 & c_2 \\
            a_3 & b_3 & c_3
        \end{vmatrix*} = \begin{vmatrix*}
            a_1 & b_1 & ka_1 + hc_1 \\
            a_2 & b_2 & ka_2 + hc_2 \\
            a_3 & b_3 & ka_3 + hc_3
        \end{vmatrix*}$。}

\end{xiaoxiaotis}


\xiaoti{不展开行列式,求证:}
\begin{xiaoxiaotis}

    \xiaoxiaoti{$\begin{vmatrix*}
            a    & a+3d & a+6d \\
            a+d  & a+4d & a+7d \\
            a+2d & a+5d & a+8d
        \end{vmatrix*} = 0$;}

    \xiaoxiaoti{$\begin{vmatrix*}
            a_1 & b_1 & c_1 \\
            a_2 & b_2 & c_2 \\
            a_3 & b_3 & c_3
        \end{vmatrix*} = \begin{vmatrix*}
            c_3 & b_3 & a_3 \\
            c_2 & b_2 & a_2 \\
            c_1 & b_1 & a_1
        \end{vmatrix*}$;}

    \xiaoxiaoti{$\begin{vmatrix*}[r]
            0  & am & -abn \\
            -e & 0  & bn \\
            e  & -m & 0
        \end{vmatrix*} = 0$;}

    \xiaoxiaoti{$\begin{vmatrix*}
            a_1 & b_1 & a_1x+b_1y+c_1 \\
            a_2 & b_2 & a_2x+b_2y+c_2 \\
            a_3 & b_3 & a_3x+b_3y+c_3
        \end{vmatrix*} = \begin{vmatrix*}
            a_1 & b_1 & c_1 \\
            a_2 & b_2 & c_2 \\
            a_3 & b_3 & c_3
        \end{vmatrix*}$;}

    \xiaoxiaoti{$\begin{vmatrix*}
            0       & (a-b)^3 & (a-c)^3 \\
            (b-a)^3 & 0       & (b-c)^3 \\
            (c-a)^3 & (c-b)^3 & 0
        \end{vmatrix*} = 0$。}

\end{xiaoxiaotis}


\xiaoti{利用行列式的性质和第 \ref{subsec:4-4} 节中的 \nameref{theorem:sjhlszk-1},计算:}
\begin{xiaoxiaotis}

    \twoInLineXxt[16em]{
        $\begin{vmatrix*}[r]
            6  & -4 & 2 \\
            -3 & 3  & -1 \\
            18 & 7  & 5
        \end{vmatrix*}$;
    }{
        $\begin{vmatrix*}[r]
            8  & 3  & -7 \\
            5  & 0  & -4 \\
            -9 & -2 & 3
        \end{vmatrix*}$。
    }

\end{xiaoxiaotis}


\xiaoti{解关于 $x$ 的方程:}
\begin{xiaoxiaotis}

    \xiaoxiaoti{$\begin{vmatrix*}
            x^2 & x & 1 \\
            a^2 & a & 1 \\
            b^2 & b & 1
        \end{vmatrix*} = 0 \quad (a \neq b)$;}

    \xiaoxiaoti{$\begin{vmatrix*}
            x   & a   & b+c \\
            x   & a+b & c \\
            a+b & b-c & a+c
        \end{vmatrix*} = 0 \quad (b(a+b) \neq 0)$。}

\end{xiaoxiaotis}

\xiaoti{求证:}
\begin{xiaoxiaotis}

    \xiaoxiaoti{$\begin{vmatrix*}
            a & a^2 & 1 \\
            b & b^2 & 1 \\
            c & c^2 & 1
        \end{vmatrix*} = (a - b)(b - c)(c - a)$;}

    \xiaoxiaoti{$\begin{vmatrix*}
            a & a^2 & bc \\
            b & b^2 & ac \\
            c & c^2 & ab
        \end{vmatrix*} = (a - b)(b - c)(c - a)(ab + bc + ca)$;}

    \xiaoxiaoti{$\begin{vmatrix*}
            ax & a^2+x^2 & 1 \\
            ay & a^2+y^2 & 1 \\
            az & a^2+z^2 & 1
        \end{vmatrix*} = a(x - y)(y - z)(z - x)$;}

    \xiaoxiaoti{$\begin{vmatrix*}[r]
            \cos\theta & \cos3\theta & \sin3\theta \\
            \cos\theta & \cos\theta  & \sin\theta \\
            \sin\theta & \sin\theta  & \cos\theta
        \end{vmatrix*} = \sin\theta\sin4\theta$。}

\end{xiaoxiaotis}


\xiaoti{已知直线方程为
    $$\begin{vmatrix*}[r]
        x  & y & 1 \\
        3  & 5 & 1 \\
        -2 & 3 & 1
    \end{vmatrix*} = 0 \text{,}$$
    问点 $P_1 \left( \dfrac{1}{2}, 4 \right)$ 与 $P_2 \left( 4, 7 \right)$ 是否在这条直线上。
}


\xiaoti{利用\nameref{klmfz}解下列关于 $x$,$y$,$z$ 的方程组:}
\begin{xiaoxiaotis}

    \renewcommand\arraystretch{1.2}
    % TODO: 这里使用 longtable 进行表格分页,
    % 但生成的表格与 tabular 生成的表格相比,略有一点缩进。
    \begin{longtable}[t]{*{2}{@{}p{18em}}}
        \xiaoxiaoti{$\begin{cases}
                4x - y - 2z = 4, \\
                2x + y - 4z = 8, \\
                x + 2y + z = 1;
            \end{cases}$}
        & \xiaoxiaoti{$\begin{cases}
                5x - 8y + 3z = 0, \\
                15x + 12y - 15z = 11, \\
                10x - 4y - 6z = 1;
            \end{cases}$} \\
        \xiaoxiaoti{$\begin{cases}
                x - y + z = a, \\
                x + y - z = b, \\
                -x + y + z = c;
            \end{cases}$}
        & \xiaoxiaoti{$\begin{cases}
                bx - ay = -2ab, \\
                -2cy + 3bz = bc, \\
                cx + az = 0 \quad (abc \neq 0)
            \end{cases}$}
    \end{longtable}

\end{xiaoxiaotis}


\xiaoti{求下列关于 $x$,$y$,$z$ 的方程组有唯一解的条件,并把第 (3) 题中的方程组在这个条件下的解求出来:}
\begin{xiaoxiaotis}

    \renewcommand\arraystretch{1.2}
    \begin{tabular}[t]{*{2}{@{}p{18em}}}
        \xiaoxiaoti{$\begin{cases}
                \lambda x + y + z = 1, \\
                x + \lambda y + z = \lambda, \\
                x + y + \lambda z = \lambda^2;
            \end{cases}$}
        & \xiaoxiaoti{$\begin{cases}
                ay + bz = c, \\
                cx + az = b, \\
                bx + cy = a;
            \end{cases}$} \\
        \xiaoxiaoti{$\begin{cases}
                ax + y + z = a - 3, \\
                x + ay + z = 2, \\
                x + y + az = -2 \text{。}
            \end{cases}$}
    \end{tabular}

\end{xiaoxiaotis}

\end{xiaotis}

