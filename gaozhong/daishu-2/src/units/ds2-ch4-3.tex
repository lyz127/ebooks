\subsection{三阶行列式的性质}\label{subsec:4-3}

为了更好地掌握和运用行列式这一工具,简化行列式的计算,我们以三阶行列式为例,来学习行列式的一些性质。

\begin{theorem} \label{theorem:sjhls-1}
    把行列式的各行变为应的列(就是第 $i$ 行变为第 $i$ 列,$i = 1,\; 2,\; 3$), 所得行列式与原行列式等。即
    $$
    \begin{vmatrix*}
        a_1 & b_1 & c_1 \\
        a_2 & b_2 & c_2 \\
        a_3 & b_3 & c_3
    \end{vmatrix*}
    =
    \begin{vmatrix*}
        a_1 & a_2 & a_3 \\
        b_1 & b_2 & b_3 \\
        c_1 & c_2 & c_3
    \end{vmatrix*} \text{。}
    $$
\end{theorem}


\zhengming 按对角线法则分别把上式两边的行列式展开。

$\begin{vmatrix*}
    a_1 & b_1 & c_1 \\
    a_2 & b_2 & c_2 \\
    a_3 & b_3 & c_3
\end{vmatrix*} = a_1b_2c_3 + a_2b_3c_1 + a_3b_1c_2 - a_3b_2c_1 - a_2b_1c_3 - a_1b_3c_2$,

$\begin{vmatrix*}
    a_1 & a_2 & a_3 \\
    b_1 & b_2 & b_3 \\
    c_1 & c_2 & c_3
\end{vmatrix*} = a_1b_2c_3 + a_2b_3c_1 + a_3b_1c_2 - a_3b_2c_1 - a_2b_1c_3 - a_1b_3c_2$

$\therefore \quad
\begin{vmatrix*}
    a_1 & b_1 & c_1 \\
    a_2 & b_2 & c_2 \\
    a_3 & b_3 & c_3
\end{vmatrix*}
=
\begin{vmatrix*}
    a_1 & a_2 & a_3 \\
    b_1 & b_2 & b_3 \\
    c_1 & c_2 & c_3
\end{vmatrix*} \text{。}$

由定理 \ref{theorem:sjhls-1} 可知,对于行列式的行成立的定理对于列也一定成立;反过来也对。


\begin{theorem} \label{theorem:sjhls-2}
    把行列式的两行(或两列)对调, 所得行列式与原行列式绝对值相等,符号相反。
\end{theorem}

\zhengming 我们先证明把行列式的第二行与第三行对调时,结论成立,即
$$
\begin{vmatrix*}
    a_1 & b_1 & c_1 \\
    a_3 & b_3 & c_3 \\
    a_2 & b_2 & c_2
\end{vmatrix*}
=
- \begin{vmatrix*}
    a_1 & b_1 & c_1 \\
    a_2 & b_2 & c_2 \\
    a_3 & b_3 & c_3
\end{vmatrix*} \text{。}
$$

用对角线法则展开上式两边的行列式:

$\begin{vmatrix*}
    a_1 & b_1 & c_1 \\
    a_2 & b_2 & c_2 \\
    a_3 & b_3 & c_3
\end{vmatrix*} = a_1b_2c_3 + a_2b_3c_1 + a_3b_1c_2 - a_3b_2c_1 - a_2b_1c_3 - a_1b_3c_2$,

$\begin{aligned}
\begin{vmatrix*}
    a_1 & b_1 & c_1 \\
    a_3 & b_3 & c_3 \\
    a_2 & b_2 & c_2
\end{vmatrix*}
    &= a_1b_3c_2 + a_3b_2c_1 + a_2b_1c_3 - a_2b_3c_1 - a_3b_1c_2 - a_1b_2c_3 \\
    &= -(a_1b_2c_3 + a_2b_3c_1 + a_3b_1c_2 - a_3b_2c_1 - a_2b_1c_3 - a_1b_3c_2) \text{。}
\end{aligned}$

$\therefore \quad
\begin{vmatrix*}
    a_1 & b_1 & c_1 \\
    a_3 & b_3 & c_3 \\
    a_2 & b_2 & c_2
\end{vmatrix*}
=
- \begin{vmatrix*}
    a_1 & b_1 & c_1 \\
    a_2 & b_2 & c_2 \\
    a_3 & b_3 & c_3
\end{vmatrix*} \text{。}
$

其他情况可类似证明。

\begin{corollary} \label{corollary:sjhls-2-1}
    如果行列式某两行(或两列)的对应元素相同,那么行列式等于零。
\end{corollary}

\zhengming 假设行列式 $D$ 有两行(或两列) 的对应元素相同,把这两行(或两列)对调, 得出的仍是原行列式 $D$。
但根据定理 \ref{theorem:sjhls-2},对调后的行列式应等于 $-D$ 。所以有
$$ D = -D \text{。} $$
由此得出
$$ D = 0 \text{。} $$



\begin{theorem} \label{theorem:sjhls-3}
    把行列式的某一行(或一列) 的所有元素同乘以某个数 $k$,等于用数 $k$ 乘原行列式。
\end{theorem}

\zhengming 我们先证明把行列式的第二行的元素乘以 $k$ 时,结论成立,即
$$
\begin{vmatrix*}
    a_1 & b_1 & c_1 \\
    ka_2 & kb_2 & kc_2 \\
    a_3 & b_3 & c_3
\end{vmatrix*}
=
k \begin{vmatrix*}
    a_1 & b_1 & c_1 \\
    a_2 & b_2 & c_2 \\
    a_3 & b_3 & c_3
\end{vmatrix*} \text{。}
$$

用对角线法则展开上式左边的行列式,得
\begin{align*}
    \begin{vmatrix*}
        a_1 & b_1 & c_1 \\
        ka_2 & kb_2 & kc_2 \\
        a_3 & b_3 & c_3
    \end{vmatrix*}
        &= ka_1b_2c_3 + ka_2b_3c_1 + ka_3b_1c_2 - ka_3b_2c_1 - ka_2b_1c_3 - ka_1b_3c_2 \\
        &= k(a_1b_2c_3 + a_2b_3c_1 + a_3b_1c_2 - a_3b_2c_1 - a_2b_1c_3 - a_1b_3c_2) \\
        &= k \begin{vmatrix*}
                a_1 & b_1 & c_1 \\
                a_2 & b_2 & c_2 \\
                a_3 & b_3 & c_3
            \end{vmatrix*} \text{。}
\end{align*}
因此结论成立。

其他情况可类似证明。


\begin{corollary} \label{corollary:sjhls-3-1}
    行列式的某一行(或一列)有公因子时,可以把公因子提到行列式外面。
\end{corollary}


\liti 计算
$$
\begin{vmatrix*}
    \dfrac{1}{2} & \dfrac{1}{2} & -1 \\[1em]
    \dfrac{1}{3} & \dfrac{2}{3} & -\dfrac{2}{3} \\[1em]
    \dfrac{2}{5} & \dfrac{3}{5} & -\dfrac{1}{5}
\end{vmatrix*} \text{。}
$$

\jie\shangyihang\begin{flalign*}
    \hspace{4em} & \begin{vmatrix*}
            \dfrac{1}{2} & \dfrac{1}{2} & -1 \\[1em]
            \dfrac{1}{3} & \dfrac{2}{3} & -\dfrac{2}{3} \\[1em]
            \dfrac{2}{5} & \dfrac{3}{5} & -\dfrac{1}{5}
          \end{vmatrix*} \\
    ={} & (-1) \times \dfrac{1}{2} \times \dfrac{1}{3} \times \dfrac{1}{5} \times
        \begin{vmatrix*}
            1 & 1 & 2 \\
            1 & 2 & 2 \\
            2 & 3 & 1
        \end{vmatrix*} && \text{\eqnameref{corollary:sjhls-3-1}}\\
    ={} & -\dfrac{1}{30} \times (2 + 6 + 4 - 8 - 1 - 6) \\
    ={} & -\dfrac{1}{30} \times (-3) = \dfrac{1}{10} \text{。}
\end{flalign*}


从例1 可以看出,根据定理 \ref{theorem:sjhls-3} 的推论\ref{corollary:sjhls-3-1},
把行列式中某一行( 或一列) 的公因子提到行列式外面, 往往可以简化行列式的计算。


\begin{corollary} \label{corollary:sjhls-3-2}
    如果行列式某一行(或一列)的所有元素都是零,那么行列式等于零。
\end{corollary}


\begin{theorem} \label{theorem:sjhls-4}
    如果行列式某两行( 或两列) 的对应元素成比例,那么行列式等于零。
\end{theorem}

\zhengming 设行列式的第二列与第一列的对应元素成比例(比例因子为 $k$),即行列式有如下形式
$$
\begin{vmatrix*}
    a_1 & ka_1 & c_1 \\
    a_2 & ka_2 & c_2 \\
    a_3 & ka_3 & c_3
\end{vmatrix*} \text{。}
$$

根据定理 \ref{theorem:sjhls-3} 的推论 \ref{corollary:sjhls-3-1}
和定理 \ref{theorem:sjhls-2} 的 \hyperref[corollary:sjhls-2-1]{推论},我们有

$$
\begin{vmatrix*}
    a_1 & ka_1 & c_1 \\
    a_2 & ka_2 & c_2 \\
    a_3 & ka_3 & c_3
\end{vmatrix*}
= k
\begin{vmatrix*}
    a_1 & a_1 & c_1 \\
    a_2 & a_2 & c_2 \\
    a_3 & a_3 & c_3
\end{vmatrix*}
= 0 \text{。}
$$
因此结论成立。

其他情况可类似证明。


\begin{theorem} \label{theorem:sjhls-5}
    如果行列式的某一行(或一列)的元素都是二项式,那么这个行列式等于把这些二项式
    各取一项作成相应行(或列) 而其余行(或列)不变的两个行列式的和。
\end{theorem}

\zhengming 设行列式的第一行元素都是二项式,即行列式有如下形式:
$$
\begin{vmatrix*}
    a_1 + a_1' & b_1 + b_1' & c_1 + c_1' \\
    a_2 & b_2 & c_2 \\
    a_3 & b_3 & c_3
\end{vmatrix*} \text{。}
$$

把行列式用对角线法则展开,得

\begin{align*}
    & \begin{vmatrix*}
        a_1 + a_1' & b_1 + b_1' & c_1 + c_1' \\
        a_2 & b_2 & c_2 \\
        a_3 & b_3 & c_3
    \end{vmatrix*} \\
    ={} & (a_1 + a_1')b_2c_3 + a_2b_3(c_1 + c_1')
         + a_3(b_1 + b_1')c_2 - a_3b_2(c_1 + c_1')
         - a_2(b_1 + b_1')c_3 - (a_1 + a_1')b_3c_2 \\
    ={} & (a_1b_2c_3 + a_2b_3c_1 + a_3b_1c_2 - a_3b_2c_1 - a_2b_1c_3 - a_1b_3c_2) \\
        & + (a_1'b_2c_3 + a_2b_3c_1' + a_3b_1'c_2 - a_3b_2c_1' - a_2b_1'c_3 - a_1'b_3c_2) \\
    ={} & \begin{vmatrix*}
            a_1 & b_1 & c_1 \\
            a_2 & b_2 & c_2 \\
            a_3 & b_3 & c_3
        \end{vmatrix*}
      + \begin{vmatrix*}
            a_1' & b_1' & c_1' \\
            a_2 & b_2 & c_2 \\
            a_3 & b_3 & c_3
        \end{vmatrix*} \text{。}
\end{align*}
因此结论成立。

其他情况可类似证明。

\liti 求证
$$
\begin{vmatrix*}
    1 & x^2 & a^2 + x^2 \\
    1 & y^2 & a^2 + y^2 \\
    1 & z^2 & a^2 + z^2
\end{vmatrix*} = 0 \text{。}
$$

\zhengming\shangyihang\begin{flalign*}
    \hspace{5em} & \begin{vmatrix*}
                    1 & x^2 & a^2 + x^2 \\
                    1 & y^2 & a^2 + y^2 \\
                    1 & z^2 & a^2 + z^2
                \end{vmatrix*} \\
    ={} & \begin{vmatrix*}
            1 & x^2 & a^2 \\
            1 & y^2 & a^2 \\
            1 & z^2 & a^2
        \end{vmatrix*}+
        \begin{vmatrix*}
            1 & x^2 & x^2 \\
            1 & y^2 & y^2 \\
            1 & z^2 & z^2
        \end{vmatrix*} && \text{\eqnameref{theorem:sjhls-5}} \\
    ={} & 0 \text{。} && \text{(\nameref{theorem:sjhls-4} 和 \nameref{corollary:sjhls-2-1})}
\end{flalign*}



\begin{theorem} \label{theorem:sjhls-6}
    把行列式某一行(或一列)的所有元素同乘以一个数 $k$,加到另一行(或另一列)的
    对应元素上, 所得行列式与原行列式相等。
\end{theorem}

\zhengming 把行列式
$$
\begin{vmatrix*}
    a_1 & b_1 & c_1 \\
    a_2 & b_2 & c_2 \\
    a_3 & b_3 & c_3
\end{vmatrix*}
$$
的第二行的元素乘以 $k$,加到第一行的对应元素上,得
$$
\begin{vmatrix*}
    a_1 + ka_2 & b_1 + kb_2 & c_1 + kc_2 \\
    a_2 & b_2 & c_2 \\
    a_3 & b_3 & c_3
\end{vmatrix*} \text{。}
$$

根据 \nameref{theorem:sjhls-5} 和 \nameref{theorem:sjhls-4},可以推出:
\begin{align*}
      & \begin{vmatrix*}
            a_1 + ka_2 & b_1 + kb_2 & c_1 + kc_2 \\
            a_2 & b_2 & c_2 \\
            a_3 & b_3 & c_3
        \end{vmatrix*} \\
    ={} & \begin{vmatrix*}
            a_1 & b_1 & c_1 \\
            a_2 & b_2 & c_2 \\
            a_3 & b_3 & c_3
          \end{vmatrix*}
        + \begin{vmatrix*}
            ka_2 & kb_2 & kc_2 \\
            a_2 & b_2 & c_2 \\
            a_3 & b_3 & c_3
          \end{vmatrix*} \\
    ={} & \begin{vmatrix*}
            a_1 & b_1 & c_1 \\
            a_2 & b_2 & c_2 \\
            a_3 & b_3 & c_3
          \end{vmatrix*} \text{。}
\end{align*}
因此结论成立。

其他情况可类似证明。

由 \hyperref[xiti-8]{习题八} 的 \hyperref[xiti-8-2]{第 2 题} 可知,三阶行列式的上述性质,对二阶行列式同样成立。


\liti 利用行列式的性质,计算:

\twoInLine[16em]{(1) \quad $\begin{aligned}
    \begin{vmatrix*}[r]
        3 & 2 & 6 \\
        8 & 10 & 9 \\
        6 & -2 & 21
    \end{vmatrix*}
\end{aligned}$;}{(2)\quad $\begin{aligned}
    \begin{vmatrix*}[r]
        10 & -2 & 7 \\
        -15 & 3 & 2 \\
        -5 & 4 & 9
    \end{vmatrix*}
\end{aligned}$。}

\jie

\begin{flalign*}
    \hspace{4em} (1) \begin{vmatrix*}[r]
            3 & 2 & 6 \\
            8 & 10 & 9 \\
            6 & -2 & 21
        \end{vmatrix*}
    &= 3 \times 2 \times \begin{vmatrix*}[r]
            3 &  1 & 2 \\
            8 &  5 & 3 \\
            6 & -1 & 7
        \end{vmatrix*} && \text{\eqnameref{corollary:sjhls-3-1}} \\
    &= 6 \times \begin{vmatrix*}[r]
            3 &  1 + 2 & 2 \\
            8 &  5 + 3 & 3 \\
            6 & -1 + 7 & 7
        \end{vmatrix*} && \text{\eqnameref{theorem:sjhls-6}} \\
    &= 6 \times \begin{vmatrix*}[r]
            3 & 3 & 2 \\
            8 & 8 & 3 \\
            6 & 6 & 7
        \end{vmatrix*} \\
    &= 0 \text{。} && \text{\eqnameref{corollary:sjhls-2-1}}
\end{flalign*}


\begin{flalign*}
    \hspace{4em} (2) \begin{vmatrix*}[r]
            10 & -2 & 7 \\
            -15 & 3 & 2 \\
            -5 & 4 & 9
        \end{vmatrix*}
    &= 5 \times \begin{vmatrix*}[r]
            2 & -2 & 7 \\
            -3 & 3 & 2 \\
            -1 & 4 & 9
        \end{vmatrix*} && \text{\eqnameref{corollary:sjhls-3-1}} \\
    &= 5 \times \begin{vmatrix*}[r]
            2 & -2 + 2 & 7 \\
            -3 & 3 + (-3) & 2 \\
            -1 & 4 + (-1) & 9
        \end{vmatrix*} && \text{\eqnameref{theorem:sjhls-6}} \\
    &= 5 \times \begin{vmatrix*}[r]
            2 & 0 & 7 \\
            -3 & 0 & 2 \\
            -1 & 3 & 9
        \end{vmatrix*} \\
    &= 5 \times (-63 - 12) = -375 \text{。}
\end{flalign*}


从上述例题可以看出,在计算行列式时,如果能直接观察出行列式有两行(或两列)
的对应元素成比例或能化到成比例的形式,那么立即可以判断这个行列式等于零。
一般地,可以先提出行列式中某一行(或一列)的各元素的公因子,
或运用 \nameref{theorem:sjhls-6} 把三阶行列式中某一行(或一列)的两个
元素变为零,从而简化计算。


\liti 利用行列式的性质,证明:

\twoInLine[16em]{(1) \quad $\begin{aligned}
    \begin{vmatrix*}[r]
        0 & a & b \\
        -a & 0 & c \\
        -b & -c & 0
    \end{vmatrix*} = 0
\end{aligned}$;}{(2) \quad $\begin{aligned}
    \begin{vmatrix*}[r]
        a + b & c & -a \\
        a + c & b & -c \\
        b + c & a & -b
    \end{vmatrix*}
    =
    \begin{vmatrix*}[r]
        b & a & c \\
        a & c & b \\
        c & b & a
    \end{vmatrix*}
\end{aligned}$。
}

\zhengming

\begin{flalign*}
    \hspace{4em} (1) \begin{vmatrix*}[r]
        0 & a & b \\
        -a & 0 & c \\
        -b & -c & 0
    \end{vmatrix*} &= \begin{vmatrix*}[r]
            0 & -a & -b \\
            a & 0 & -c \\
            b & c & 0
        \end{vmatrix*} && \text{\eqnameref{theorem:sjhls-1}} \\
    &= - \begin{vmatrix*}[r]
            0 & a & b \\
            -a & 0 & c \\
            -b & -c & 0
        \end{vmatrix*} && \text{\eqnameref{corollary:sjhls-3-1}}
\end{flalign*}

$\therefore \quad
\begin{vmatrix*}[r]
    0 & a & b \\
    -a & 0 & c \\
    -b & -c & 0
\end{vmatrix*} = 0 \text{。}
$

\begin{flalign*}
    \hspace{4em} (2) \begin{vmatrix*}[r]
            a + b & c & -a \\
            a + c & b & -c \\
            b + c & a & -b
        \end{vmatrix*} &= \begin{vmatrix*}[r]
            b & c & -a \\
            a & b & -c \\
            c & a & -b
        \end{vmatrix*} && \text{\eqnameref{theorem:sjhls-6}} \\
    &= - \begin{vmatrix*}[r]
            b & c & a \\
            a & b & c \\
            c & a & b
        \end{vmatrix*} && \text{\eqnameref{corollary:sjhls-3-1}} \\
    &= \begin{vmatrix*}[r]
            b & a & c \\
            a & c & b \\
            c & b & a
        \end{vmatrix*} \text{。} && \text{\eqnameref{theorem:sjhls-2}}
\end{flalign*}



\lianxi
\begin{xiaotis}

\xiaoti{利用行列式的性质,计算:}
\begin{xiaoxiaotis}

    \renewcommand\arraystretch{1.2}
    \begin{tabular}[t]{*{2}{@{}p{16em}}}
        \xiaoxiaoti{$\begin{vmatrix*}[r]
                1 & 3 & 4 \\
                10 & 1 & 11 \\
                7 & 1 & 8
            \end{vmatrix*}$;}
        & \xiaoxiaoti{$\begin{vmatrix*}[r]
                3 & 49 & 4 \\
                2 & 28 & 4 \\
                4 & 35 & 8
            \end{vmatrix*}$;} \\[3em]
        \xiaoxiaoti{$\begin{vmatrix*}[r]
                \dfrac{2}{3} & \dfrac{2}{3} & 3 \\
                7 & 5 & 14 \\
                \dfrac{1}{3} & \dfrac{1}{5} & \dfrac{4}{15}
            \end{vmatrix*}$;}
        & \xiaoxiaoti{$\begin{vmatrix*}[r]
                1 & 4 & 7 \\
                2 & 5 & 8 \\
                3 & 6 & 9
            \end{vmatrix*}$。}
    \end{tabular}

\end{xiaoxiaotis}



\xiaoti{利用行列式的性质,计算:}
\begin{xiaoxiaotis}

    \renewcommand\arraystretch{1.2}
    \begin{tabular}[t]{*{2}{@{}p{16em}}}
        \xiaoxiaoti{$\begin{vmatrix*}[r]
                a & a & a \\
                -a & a & x \\
                -a & -a & x
            \end{vmatrix*}$;}
        & \xiaoxiaoti{$\begin{vmatrix*}[r]
                1 & a & b + c \\
                1 & b & c + a \\
                1 & c & a + b
            \end{vmatrix*}$;} \\
        \xiaoxiaoti{$\begin{vmatrix*}
                1 & 1 & 1 \\
                1 & 1 + b & 1 \\
                1 & 1 & 1 + c
            \end{vmatrix*}$;}
        & \xiaoxiaoti{$\begin{vmatrix*}[r]
                a - b & b - c & c - a \\
                b - c & c - a & a - b \\
                c - a & a - b & b - c
            \end{vmatrix*}$。}
    \end{tabular}

\end{xiaoxiaotis}



\xiaoti{不展开行列式,证明下列等式:}
\begin{xiaoxiaotis}

    \xiaoxiaoti{
        $\begin{vmatrix*}
            1 & 1 & 1 \\
            p & q & p + q \\
            q & p & 0
        \end{vmatrix*} = 0$;
    }

    \xiaoxiaoti{
        $\begin{vmatrix*}[r]
            -a+b+c & a & -b \\
            a-b+c & b & -c \\
            a+b-c & c & -a
        \end{vmatrix*} = \begin{vmatrix*}[r]
            b & a & c \\
            c & b & a \\
            a & c & b
        \end{vmatrix*}$。
    }

\end{xiaoxiaotis}

\end{xiaotis}

