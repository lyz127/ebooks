\subsubsection{二阶行列式}

在初中,我们学过二元一次方程组和三元一次方程组以及用消元法求它们的解。
一次方程又叫做\textbf{线性方程},一次方程组又叫做\textbf{线性方程组}。
在本章中,我们将学习线性方程组的另一种解法,并进一步研究解的情况。
为此,我们先从解二元线性方程组着手来引入一个新的概念 —— 二阶行列式。

一个二元线性方程组,当其中方程的个数与未知数的个数相同时,它的一般形式可以写成

\fangchengzu{eq:fcz-1}(\thefangchengzu)
\begin{minipage}[c]{0.92\textwidth}
    \begin{numcases}{}
        a_1 x + b_1 y = c_1 \text{,} \label{eq:ejhls-1} \\
        a_2 x + b_2 y = c_2 \text{,} \label{eq:ejhls-2}
    \end{numcases}
\end{minipage}
其中 $x$,$y$ 是未知数,$a_1$,$a_2$,$b_1$,$b_2$ 是未知数的系数,$c_1$,$c_2$ 是常数项
(在一般形式中,我们把常数项写在方程的右边)。

如果当 $x = x_1$,$y = y_1$ 时,方程组 \eqref{eq:fcz-1} 中的每个方程左右两边的值相等,也就是说 $x = x_1$,$y = y_1$
适合方程组 \eqref{eq:fcz-1},那么 $x = x_1$,$y = y_1$ 叫做\textbf{方程组 \eqref{eq:fcz-1} 的一个解}, 记为
$$ \begin{cases}
    x = x_1 \text{,} \\
    y = y_1 \text{,} \\
\end{cases}$$
或简记为 $(x_1,\; y_1)$。方程组 \eqref{eq:fcz-1} 的所有的解构成的集合叫做\textbf{方程组 \eqref{eq:fcz-1} 的解集}。

用加减消元法解这个方程组:

$\eqref{eq:ejhls-1} \times b_2 - \eqref{eq:ejhls-2} \times b_1$,得
\begin{equation}
    (a_1b_2 - a_2b_1)x = c_1b_2 - c_2b_1 \text{;} \label{eq:ejhls-3}
\end{equation}

$\eqref{eq:ejhls-2} \times a_1 - \eqref{eq:ejhls-1} \times a_2$,得
\begin{equation}
    (a_1b_2 - a_2b_1)y = a_1c_2 - a_2c_1 \text{。} \label{eq:ejhls-4}
\end{equation}

方程组 \eqref{eq:fcz-1} 的解一定适合方程 \eqref{eq:ejhls-3} 和方程 \eqref{eq:ejhls-4} 。

当 $a_1b_2 - a_2b_1 \neq 0$ 时,可以得出方程组 \eqref{eq:fcz-1} 有唯一解,即
\begin{equation}
    \begin{cases}
        x = \dfrac{c_1b_2 - c_2b_1}{a_1b_2 - a_2b_1} \text{,} \\[1.5em]
        y = \dfrac{a_1c_2 - a_2c_1}{a_1b_2 - a_2b_1} \text{。} \label{eq:ejhls-5}
    \end{cases}
\end{equation}

为了便于记忆这一结果,我们先来对公式 \eqref{eq:ejhls-5} 进行分析。

在公式 \eqref{eq:ejhls-5} 中,两个分母都是 $a_1b_2 - a_2b_1$,并且只含有未知数的系数。
把未知数的系数按照它们在方程组中原来的位置排列成正方形,即
\begin{figure}[H]
    \centering
    \begin{tikzpicture}
    \coordinate [label=180:$a_2$] (a2) at (0, 0);
    \coordinate [label=180:$a_1$] (a1) at (0, 2);
    \coordinate [label=0:$b_1$] (b1) at (2, 2);
    \coordinate [label=0:$b_2$] (b2) at (2, 0);

    \draw (a1) -- (b2);
    \draw [dashed] (a2) -- (b1);
\end{tikzpicture}

\end{figure}
可以看出 $a_1b_2 - a_2b_1$ 是这样的两项的和:
一项是正方形中实线表示的对角线(叫做主对角线)上两数的积,再添上正号;
一项是虚线表示的对角线(叫做副对角线)上两数的积,再添上负号。
我们在这四个数的两旁各加一条竖线,引进符号
\begin{equation}
    \begin{vmatrix}
        a_1 & b_1 \\
        a_2 & b_2
    \end{vmatrix}, \label{eq:ejhls-6}
\end{equation}
并且规定它就表示
\begin{equation}
    a_1b_2 - a_2b_1 \label{eq:ejhls-7}
\end{equation}
这时,符号 \eqref{eq:ejhls-6} 叫做\textbf{二阶行列式},
$a_1$,$a_2$,$b_1$,$b_2$ 叫做行列式 \eqref{eq:ejhls-6} 的\textbf{元素}。
这四个元素排成二行二列(横排叫行,竖排叫列)。例如 $a_2$ 是位于第二行第一列上的元素,
$b_1$ 是位于第一行第二列上的元素。
利用对角线把符号 \eqref{eq:ejhls-6} 表示的二阶行列式展开成 \eqref{eq:ejhls-7} 式,
这种方法叫做二阶行列式展开的\textbf{对角线法则}。

\liti 展开下列行列式,并化简:

\begin{xiaoxiaotis}

    % \renewcommand\arraystretch{1.5}
    \begin{tabular}[t]{*{2}{@{}p{16em}}}
        \xiaoxiaoti{
            $\begin{vmatrix}
                10 & -9 \\
                -3 & 7
            \end{vmatrix}$;
        } & \xiaoxiaoti{
            $\begin{vmatrix}
                m+1 & m+2 \\
                m   & m+1
            \end{vmatrix}$;
        } \\[1.5em]
        \xiaoxiaoti{
            $\begin{vmatrix}
                \sin x & \cos x \\
                \cos x & -\sin x
            \end{vmatrix}$。
        }
    \end{tabular}

\end{xiaoxiaotis}

\jie
(1) $\begin{vmatrix}
        10 & -9 \\
        -3 & 7
    \end{vmatrix} = 10 \times 7 - (-3) \times (-9) = 43$ ;

(2) $\begin{vmatrix}
        m+1 & m+2 \\
        m   & m+1
    \end{vmatrix} = (m + 1)^2 - m(m + 2) = 1$;

(3) $\begin{vmatrix}
        \sin x & \cos x \\
        \cos x & -\sin x
    \end{vmatrix} = -\sin^2x - \cos^2x = -1$。


\lianxi
\begin{xiaotis}

\xiaoti{计算:}
\begin{xiaoxiaotis}

    \twoInLineXxt[16em]{
        $\begin{vmatrix}
            5 & 7 \\
            7 & 9
        \end{vmatrix}$;
    }{
        $\begin{vmatrix}
            -3 & 21 \\
            -1 & 7
        \end{vmatrix}$。
    }

\end{xiaoxiaotis}


\xiaoti{展开下列行列式,并化简:}
\begin{xiaoxiaotis}

    \twoInLineXxt[16em]{
        $\begin{vmatrix}
            6a - b & 2b \\
            3a & b
        \end{vmatrix}$;
    }{
        $\begin{vmatrix}
            \log_a x & \log_a x \\
            m & n
        \end{vmatrix}$。
    }

\end{xiaoxiaotis}

\end{xiaotis}

