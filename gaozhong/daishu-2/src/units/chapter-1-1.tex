\subsection{反正弦函数}\label{subsec:1-1}

我们己经学习了正弦函数 $y = \sin x$ 和它的图象(图\ref{fig:1-1})。从图象可以看到,对于 $x$ 在定义域
$(-\infty, +\infty)$ 上的每一个值,$y$ 都在 $[-1, 1]$ 上有唯一的值和它对应。例如,对于 $x = \dfrac{\pi}{6}$,
有 $y = \sin\dfrac{\pi}{6} = \dfrac{1}{2}$ 和它对应。反过来,对于 $y$ $[-1, 1]$ 上的每一个值,
$x$ 有无穷多个值和它对应。例如,对于 $y = \dfrac{1}{2}$,$x$ 有 $\dfrac{\pi}{6}$,
$\dfrac{5\pi}{6}$,$\cdots$ 等无穷多个值和它对应。由此可见,确定函数 $y = \sin x$ 的映射不是定义域 $(-\infty, +\infty)$
到值域 $[-1, 1]$ 上的一一映射。函数 $y = \sin x \; (x \in (-\infty, +\infty))$ 没有反函数。

\begin{figure}[htbp]
    \centering
    \begin{tikzpicture}[>=Stealth, scale=0.7]
    \draw[domain=-2*pi:4*pi,samples=100] plot (\x, {sin(\x r)});

    \draw [dashed] (-2*pi-0.2, 0.5) -- (4*pi+0.2, 0.5);
    \draw [->] (-2*pi-0.5, 0) -- (4*pi+0.5, 0) node[anchor=west] {$x$};
    \draw [->] (0, -1.5) -- (0, 1.5) node[anchor=east] {$y$};
    \node [font=\footnotesize, fill=white, inner sep=0pt] at (-0.2, -0.3) {$O$};
    \node [font=\footnotesize, fill=white, inner sep=1pt] at (-0.3, 0.5) {$\dfrac{1}{2}$};
    \draw (0,1) -- (0.2, 1) node [font=\footnotesize] at (0.4, 1) {$1$};
    \draw (0,-1) -- (0.2, -1) node [font=\footnotesize] at (-0.4, -1) {$-1$};
    \foreach \x / \name in {
        -2*pi/$-2\pi$,
        -1*pi/$-\pi$,
        0.16667*pi/$\dfrac{\pi}{6}$,
        0.16667*5*pi/$\dfrac{5\pi}{6}$,
        pi/$\pi$,
        2*pi/$2\pi$,
        3*pi/$3\pi$,
        4*pi/$4\pi$} {
        \node [anchor=north, font=\footnotesize] at (\x, 0) {\name};
    }

    \foreach \x in {-2, -1, 1, 2, 3, 4, 1/6, 5/6} {
        \draw (\x * pi, 0) -- (\x * pi, 0.2);
    }
\end{tikzpicture}

    \caption{}\label{fig:1-1}
\end{figure}

但由图 \ref{fig:1-2} 可以看到,在正弦函数的单调区间 $\left[ -\dfrac{\pi}{2}, \dfrac{\pi}{2} \right]$
上,对于 $x$ 的每一个值,$y = \sin x$ 有唯一的值和 $x$ 对应;而对于 $x$ 的不同的值,$y = \sin x$
有不同的值和 $x$ 对应,并且随着 $x$ 由 $-\dfrac{\pi}{2}$ 增大到 $\dfrac{\pi}{2}$,$y = \sin x$
由 $-1$ 增大到 $+1$,取得 $[-1, 1]$ 上的一切值。因此,确定函数
$y = \sin x \; \left( x \in \left[ -\dfrac{\pi}{2}, \dfrac{\pi}{2} \right]\right)$
的映射是区间 $\left[ -\dfrac{\pi}{2}, \dfrac{\pi}{2} \right]$ 到 $[-1, 1]$ 上的一一映射。
所以这个映射有逆映射,函数
$y = \sin x \; \left( x \in \left[ -\dfrac{\pi}{2}, \dfrac{\pi}{2} \right]\right)$
有反函数。

\begin{figure}[htbp]
    \centering
    \begin{tikzpicture}[>=Stealth]
    \draw [dashed] (0, 1) -- (0.5*pi, 1) -- (0.5*pi, 0);
    \draw [dashed] (0, -1) -- (-0.5*pi, -1) -- (-0.5*pi, 0);

    \draw [->] (-0.5*pi-0.5, 0) -- (0.5*pi+0.5, 0) node[anchor=west] {$x$};
    \draw [->] (0, -1.5) -- (0, 1.5) node[anchor=east] {$y$};
    \node [font=\footnotesize, fill=white, inner sep=0pt] at (0.3, -0.3) {$O$};
    \draw (0,1) -- (0.2, 1) node [font=\footnotesize] at (-0.4, 1) {$1$};
    \draw (0,-1) -- (0.2, -1) node [font=\footnotesize] at (0.5, -1) {$-1$};
    \foreach \x / \name in {
        -0.5*pi/$-\dfrac{\pi}{2}$,
        -1/$-1$,
        1/$1$,
        0.5*pi/$\dfrac{\pi}{2}$} {
        \draw (\x, 0) -- (\x, 0.2);
        \node [anchor=north, font=\footnotesize, fill=white, inner sep=1pt] at (\x, 0) {\name};
    }

    \draw[domain=-0.5*pi:0.5*pi,samples=30] plot (\x, {sin(\x r)});
\end{tikzpicture}

    \caption{}\label{fig:1-2}
\end{figure}

函数
$y = \sin x \; \left( x \in \left[ -\dfrac{\pi}{2}, \dfrac{\pi}{2} \right]\right)$
的反函数叫做\textbf{反正弦函函数},记作 $x = \arcsin y$。

\newpage
习惯上用字母 $x$ 表示自变量,用 $y$ 表示函数,所以反正弦函数可以写成 $y = \arcsin x$,
\footnote{有的书上把反正弦函数写作 $y = \sin^{-1}x$。同样,后面讲到的反余弦函数、反正切函数、
反余切函数与写作 $\cos^{-1}x$,$\tan^{-1}x$,$\cot^{-1}x$。}
它的定义域是 $[-1, 1]$,它的值域是 $\left[ -\dfrac{\pi}{2}, \dfrac{\pi}{2} \right]$。

这样,对于属于 $[-1, 1]$ 的每一个 $x$ 值,$\arcsin x$ 就表示属于
$\left[ -\dfrac{\pi}{2}, \dfrac{\pi}{2} \right]$ 的唯一确定的一个值,它的正弦正好等于已知的 $x$。
也可以说,$\arcsin x$ 表示属于的 $\left[ -\dfrac{\pi}{2}, \dfrac{\pi}{2} \right]$ 的唯一确定的
一个角(弧度数),这个角的正弦恰好等于 $x$。例如,对于 $x = \dfrac{1}{2}$,$y = \arcsin \dfrac{1}{2}$
就表示 $\left[ -\dfrac{\pi}{2}, \dfrac{\pi}{2} \right]$ 上使 $\sin y = \dfrac{1}{2}$ 的唯一确定
的一个角,这个角是 $\dfrac{\pi}{6}$,因为根据正弦函数 $y = \sin x$ 在
$\left[ -\dfrac{\pi}{2}, \dfrac{\pi}{2} \right]$ 上的单调性可以知道,在
$\left[ -\dfrac{\pi}{2}, \dfrac{\pi}{2} \right]$ 上,除了 $\dfrac{\pi}{6}$ 以外,其他任何角
的正弦都不等于$\dfrac{1}{2}$。

由此可以得到
$$\sin\left( \arcsin \dfrac{1}{2} \right) = \dfrac{1}{2} \text{。}$$

一般地,根据反正弦函数的定义,可以得到
$$\sin(\arcsin x) = x \text{,}$$
其中 $x \in [-1, 1]$,$\arcsin x \in \left[ -\dfrac{\pi}{2}, \dfrac{\pi}{2} \right]$。

面我们来研究反正弦函数的图象和性质。

根据互为反函数的图象的性质,容易知道,反正弦函数 $y = \arcsin x$ 的图象就是与正弦函数 $y = \sin x$
在 $\left[ -\dfrac{\pi}{2}, \dfrac{\pi}{2} \right]$ 上的一段图象关于直线 $y = x$ 对称的图形(图\ref{fig:1-3})。

\begin{figure}[htbp]
    \centering
    \begin{tikzpicture}[>=Stealth]
    \pgfmathsetmacro{\base}{0.5 * pi}
    \draw [->] (-\base-0.5, 0) -- (\base+0.5, 0) node[anchor=west] {$x$};
    \draw [->] (0, -\base-0.5) -- (0, \base+0.8) node[anchor=east] {$y$};
    \node [font=\footnotesize, fill=white, inner sep=0pt] at (0.3, -0.3) {$O$};
    \foreach \x / \name in {
        -\base/$-\dfrac{\pi}{2}$,
        -1/$-1$,
        1/$1$,
        \base/$\dfrac{\pi}{2}$} {
        \draw (\x, 0.2) -- (\x, 0) node [anchor=north, font=\footnotesize] {\name};
    }

    \foreach \y / \name in {
        -\base/$-\dfrac{\pi}{2}$,
        -1/$-1$,
        1/$1$,
        \base/$\dfrac{\pi}{2}$} {
        \draw (0.2, \y) -- (0, \y) node [anchor=east, font=\footnotesize]{\name};
    }

    \draw (-\base, -\base) -- (\base, \base) node [anchor=west] {$y = x$};
    \draw[dashed, domain=-0.5*pi:0.5*pi,smooth] plot (\x, {sin(\x r)}) node [anchor=west] {$y = \sin x$};
    \draw[domain=-1:1,smooth,samples=50] plot (\x, {rad(asin(\x))}) node at (1.4, 2.0) {$y = \arcsin x$};
    \node[font=\footnotesize] at (2.4, 0.5) {$x \in [-\dfrac{\pi}{2}, \dfrac{\pi}{2}]$};
\end{tikzpicture}

    \caption{}\label{fig:1-3}
\end{figure}

从图象上可以看出,反正弦函数 $y = \arcsin x$ 有以下性质:

\textbf{(1)反正弦函数 $y = \arcsin x$ 在区间 $[-1, 1]$ 上是增函数。}

\textbf{(2)反正弦函数 $y = \arcsin x$ 的图象关于原点对称,这说明它是奇函数。也就是
$$\arcsin(—x) = -\arcsin x, \; x \in [-1, 1] \text{。}$$}

\liti 求下列反正弦函数值:
\begin{xiaoxiaotis}

    \renewcommand\arraystretch{1.8}
    \begin{tabular}[t]{*{2}{@{}p{16em}}}
        \xiaoxiaoti{$\arcsin \dfrac{\sqrt{2}}{2}$;} & \xiaoxiaoti{$\arcsin 0.2672$;} \\
        \xiaoxiaoti{$\arcsin \left( -\dfrac{\sqrt{3}}{2} \right)$;} & \xiaoxiaoti{$\arcsin(-1)$。}
    \end{tabular}

\end{xiaoxiaotis}

解:(1)因为在 $\left[ -\dfrac{\pi}{2}, \dfrac{\pi}{2} \right]$ 上,$\sin\dfrac{\pi}{4} = \dfrac{\sqrt{2}}{2}$,所以
$$\arcsin \dfrac{\sqrt{2}}{2} = \dfrac{\pi}{4} \text{。}$$

注意:虽然 $\sin\dfrac{3\pi}{4} = \dfrac{\sqrt{2}}{2}$,但是
$\dfrac{3\pi}{4} \notin \left[ -\dfrac{\pi}{2}, \dfrac{\pi}{2} \right]$,所以
$\arcsin \dfrac{\sqrt{2}}{2} \neq \dfrac{3\pi}{4}$。同理,
$\arcsin \dfrac{\sqrt{2}}{2}$ 也不等于其他值
$\left( \text{如:}\dfrac{9\pi}{4}, -\dfrac{7\pi}{4} \text{等} \right)$,
只能等于 $\dfrac{\pi}{4}$。

(2)查正弦函数表,得 $\sin15^\circ30' = 0.2672$。又因为 $15^\circ30'$ 的弧度数属于
$\left[ -\dfrac{\pi}{2}, \dfrac{\pi}{2} \right]$,所以
$$\arcsin 0.2672 = 15^\circ30' \text{。}$$

(3)因为在 $\left[ -\dfrac{\pi}{2}, \dfrac{\pi}{2} \right]$ 上,
$\sin\left( -\dfrac{\pi}{3} \right) = -\dfrac{\sqrt{3}}{2}$,所以
$$ \arcsin \left( -\dfrac{\sqrt{3}}{2} \right) = -\dfrac{\pi}{3} \text{。} $$

(4)因为在 $\left[ -\dfrac{\pi}{2}, \dfrac{\pi}{2} \right]$ 上,
$\sin\left( -\dfrac{\pi}{2} \right) = -1$,所以
$$\arcsin(-1) = \dfrac{\pi}{2} \text{。}$$

\liti 求下列各式的值:
\begin{xiaoxiaotis}

    \twoInLineXxt[16em]{$\sin\left( \arcsin \dfrac{2}{3} \right)$;}{$\sin\left[ \arcsin\left( -\dfrac{1}{2} \right) \right]$。}

\end{xiaoxiaotis}

\jie (1)$\because \quad x = \dfrac{2}{3} \in [-1, 1]$,

$\therefore \quad \sin\left( \arcsin \dfrac{2}{3} \right) = \dfrac{2}{3}$。

(2)$\because \quad x = -\dfrac{1}{2} \in [-1, 1]$,

$\therefore \quad \sin\left[ \arcsin\left( -\dfrac{1}{2} \right) \right] = -\dfrac{1}{2}$。

\liti 求下列各式的值:
\begin{xiaoxiaotis}

    \renewcommand\arraystretch{1.5}
    \begin{tabular}[t]{*{2}{@{}p{16em}}}
        \xiaoxiaoti{$\tan\left( \arcsin \dfrac{\sqrt{3}}{2} \right)$;} & \xiaoxiaoti{$\cos\left( \arcsin \dfrac{4}{5} \right)$;} \\
        \xiaoxiaoti{$\cos(\arcsin x), \; x \in [-1, 1]$;} & \xiaoxiaoti{$\sin\left( 2\arcsin \dfrac{3}{5} \right)$。}
    \end{tabular}

\end{xiaoxiaotis}

\jie (1) $\tan\left( \arcsin \dfrac{\sqrt{3}}{2} \right) = \tan\dfrac{\pi}{3} = \sqrt{3}$。

(2)设 $\arcsin \dfrac{4}{5} = \alpha$,则 $\sin\alpha = \dfrac{4}{5}$。

由 $\alpha \in \left[ -\dfrac{\pi}{2}, \dfrac{\pi}{2} \right]$,得 $\cos\alpha \geqslant 0$,可知
$$\cos\alpha = \sqrt{1 - \sin^2\alpha} = \sqrt{1 - \left( \dfrac{4}{5} \right)^2} = \dfrac{3}{5} \text{,}$$

$\therefore \quad \cos\left( \arcsin \dfrac{4}{5} \right) = \dfrac{3}{5}$。

(3)设 $\arcsin x = \alpha$,则 $\sin\alpha = x$,且 $\alpha \in \left[ -\dfrac{\pi}{2}, \dfrac{\pi}{2} \right]$,
$$\cos\alpha = \sqrt{1 - \sin^2\alpha} = \sqrt{1 - x^2} \text{,}$$

$\therefore \quad \cos(\arcsin x) = \sqrt{1 - x^2}$。

或:由 $x \in [-1, 1]$,得 $\arcsin x \in \left[ -\dfrac{\pi}{2}, \dfrac{\pi}{2} \right]$,可知
$$\cos(\arcsin x) \geqslant 0 \text{,}$$

$\therefore \quad \cos(\arcsin x) = \sqrt{1 - [\sin( \arcsin x )]^2} = \sqrt{1 - x^2}$。

(4)利用倍角公式及本例题第(3)题的结果,可知

\qquad $\begin{aligned}[t]
    \sin\left( 2\arcsin \dfrac{3}{5} \right) &= 2\sin\left( \arcsin \dfrac{3}{5} \right) \cos\left( \arcsin \dfrac{3}{5} \right) \\
    &= 2 \times \dfrac{3}{5} \times \sqrt{1 - \left( \dfrac{3}{5} \right)^2} \\
    &= 2 \times \dfrac{3}{5} \times \dfrac{4}{5} = \dfrac{24}{25} \text{。}
\end{aligned}$

\liti 求下列各式的值:
\begin{xiaoxiaotis}

    \twoInLineXxt[16em]{$\arcsin\left( \sin\dfrac{\pi}{4} \right)$;}{$\arcsin\left( \sin\dfrac{2\pi}{3} \right)$。}

\end{xiaoxiaotis}

\jie (1) $\arcsin\left( \sin\dfrac{\pi}{4} \right) = \arcsin\dfrac{\sqrt{2}}{2} = \dfrac{\pi}{4}$。

(2)$\arcsin\left( \sin\dfrac{2\pi}{3} \right) = \arcsin\dfrac{\sqrt{3}}{2} = \dfrac{\pi}{3}$。

由例4 第(2)题可以看出,虽然 $\sin(\arcsin x) = x$,其中 $x \in [-1, 1]$,
但是 $\arcsin(\sin x)$ 不一定等于 $x$,而是等于在闭区间
$\left[ -\dfrac{\pi}{2}, \dfrac{\pi}{2} \right]$
上与 $x$ 有相同正弦的一个值。

\lianxi
\begin{xiaotis}

\xiaoti{用反正弦的形式把下列各式中的 $x \; \left( x \in \left[ -\dfrac{\pi}{2}, \dfrac{\pi}{2} \right] \right)$ 表示出来:}
\begin{xiaoxiaotis}

    \renewcommand\arraystretch{1.8}
    \begin{tabular}[t]{*{2}{@{}p{16em}}}
        \xiaoxiaoti{$\sin x = \dfrac{2}{5}$;} & \xiaoxiaoti{$\sin x = -\dfrac{1}{3}$;} \\
        \xiaoxiaoti{$\sin x = 0.3147$;} & \xiaoxiaoti{$\sin x = -\dfrac{\sqrt{3}}{4}$。}
    \end{tabular}

\end{xiaoxiaotis}

\xiaoti{}
\begin{xiaoxiaotis}

    \vspace{-1.7em} \begin{minipage}{0.9\textwidth}
    \xiaoxiaoti{$\arcsin \sqrt{2}$ 有意义吗,为什么?}
    \end{minipage}

    \xiaoxiaoti{$\sin\left( \arcsin \dfrac{\sqrt{5}}{2} \right) = \dfrac{\sqrt{5}}{2}$ 是否成立,为什么?}

\end{xiaoxiaotis}


\xiaoti{写出下列函数的定义域、值域:}
\begin{xiaoxiaotis}

    \renewcommand\arraystretch{1.5}
    \begin{tabular}[t]{*{2}{@{}p{16em}}}
        \xiaoxiaoti{$y = \arcsin 2x$;} & \xiaoxiaoti{$y = \dfrac{1}{2} \arcsin x$;} \\
        \xiaoxiaoti{$y = 3\arcsin \dfrac{2}{3} x$;} & \xiaoxiaoti{$y = 2\arcsin(1 - x)$。}
    \end{tabular}

\end{xiaoxiaotis}

\xiaoti{求下列反正弦函数值:}
\begin{xiaoxiaotis}

    \renewcommand\arraystretch{1.8}
    \begin{tabular}[t]{*{2}{@{}p{16em}}}
        \xiaoxiaoti{$\arcsin \dfrac{\sqrt{3}}{2}$;} & \xiaoxiaoti{$\arcsin\left( -\dfrac{\sqrt{2}}{2} \right)$;} \\
        \xiaoxiaoti{$\arcsin 0.6959$;} & \xiaoxiaoti{$\arcsin\left( -\dfrac{1}{3} \right)$。}
    \end{tabular}

\end{xiaoxiaotis}

\xiaoti{求下列各式的值:}
\begin{xiaoxiaotis}

    \renewcommand\arraystretch{1.8}
    \begin{tabular}[t]{*{2}{@{}p{16em}}}
        \xiaoxiaoti{$\sin\left( \arcsin \dfrac{4}{5} \right)$;} & \xiaoxiaoti{$\sin\left[ \arcsin\left( -\dfrac{4}{5} \right)\right]$。}
    \end{tabular}

    %\twoInLineXxt[16em]{$\sin\left( \arcsin \dfrac{4}{5} \right)$;}{$\sin\left[ \arcsin\left( -\dfrac{4}{5} \right)\right]$。}

\end{xiaoxiaotis}

\xiaoti{求下列各式的值:}
\begin{xiaoxiaotis}

    \renewcommand\arraystretch{1.8}
    \begin{tabular}[t]{*{2}{@{}p{16em}}}
        \xiaoxiaoti{$\cos\left( \arcsin \dfrac{1}{2} \right)$;} & \xiaoxiaoti{$\tan\left( \arcsin \dfrac{3}{5} \right)$;} \\
        \xiaoxiaoti{$\tan(\arcsin x), \; x \in (-1, 1)$;} & \xiaoxiaoti{$\cos\left( 2\arcsin \dfrac{4}{5} \right)$。}
    \end{tabular}

\end{xiaoxiaotis}

\xiaoti{求下列各式的值:}
\begin{xiaoxiaotis}

    \renewcommand\arraystretch{1.5}
    \begin{tabular}[t]{*{2}{@{}p{16em}}}
        \xiaoxiaoti{$\arcsin\left( \sin\dfrac{3\pi}{4} \right)$;} & \xiaoxiaoti{$\arcsin \left[ \sin\left( -\dfrac{3\pi}{4} \right)\right]$。}
    \end{tabular}

\end{xiaoxiaotis}


\end{xiaotis}


