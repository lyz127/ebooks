\subsection{复数的有关概念}\label{subsec:5-2}

复数 $a + bi$ ($a,\; b \in R$。以后说复数 $a + bi$ 时,都有 $a,\; b \in R$),
当 $b = 0$ 时,就是实数;当 $b \neq 0$ 时,叫做\textbf{虚数},
当 $a = 0$,$b \neq 0$ 时,叫做\textbf{纯虚数};
$a$ 与 $b$ 分别叫做复数 $a + bi$ 的\textbf{实部}与\textbf{虚部}。
例如, $3 + 4i$,$-\dfrac{1}{2} - \sqrt{2}i$,$-0.5i$都是虚数,
它们的实部分别是 $3$,$-\dfrac{1}{2}$,$0$,
虚部分别是 $4$,$-\sqrt{2}$,$-0.5$。

显然,实数集 $R$ 是复数集 $C$ 的真子集,即 $R \subset C$。

如果两个复数 $a + bi$ 与 $c + di$ 的实部与虚部分别相等,我们就说这两个\textbf{复数相等},
记作 $a + bi = c + di$,这就是说,如果 $a,\; b,\; c,\; d \in R$,那么
\begin{align*}
    a + bi = c + di &\iff a = c,\; b = d, \\[-1em]
    a + bi = 0 &\iff a = b = 0 \text{。}
\end{align*}

\textbf{例} \quad 已知 $(2x - 1) + i = y - (3 -- y)i$,其中 $x,\; y \in R$。求 $x$ 与 $y$。

\jie 根据复数相等的定义,得方程组
$$\begin{cases}
    2x - 1 = y, \\
    1 = -(3 - y) \text{。}
\end{cases}$$

$\therefore \quad x = \dfrac{5}{2},\quad y = 4$。


从复数相等的定义,我们知道,任何一个复数 $z = a + bi$,都可以由一个有顺序的实数对 $(a, b)$ 唯一确定。
这就使我们能借用平面直角坐标系来表示复数 $z = a + bi$。如图 \ref{fig:5-1},点 $Z$ 的横坐标是 $a$,
纵坐标是 $b$,复数 $z = a + bi$ 可用点 $Z(a, b)$ 来表示。这个建立了直角坐标系来表示复数的
平面叫做\textbf{复平面},$x$ 轴叫做\textbf{实轴},$y$ 轴除去原点的部分叫做\textbf{虚轴}
(因为原点表示实数 $0$,原点不在虚轴上)。表示实数的点都在实轴上,表示纯虚数的点都在虚轴上。

\begin{figure}[H]
    \centering
    \begin{minipage}{8cm}
        \centering
        \begin{tikzpicture}[>=Stealth, scale=0.8]
    \draw [->] (-1, 0) -- (3, 0) node[anchor=west] {$x$};
    \draw [->] (0, -1) -- (0, 3.5) node[anchor=east] {$y$};
    \node [font=\footnotesize] at (-0.3, -0.3) {$O$};

    \coordinate (Z) at (1.3, 2.6);
    \draw [dashed]
        let
            \p1 = (Z)
        in
            (\x1, 0) -- (\p1) -- (0, \y1);
    \filldraw [fill=white] (Z) circle (0.05);
    \node at ($(Z) + (0.8, 0.3)$) {$Z: a + bi$};
    \node at (0.8, -0.3) {$a$};
    \node at (-0.3, 1.3) {$b$};
\end{tikzpicture}

        \caption{}\label{fig:5-1}
    \end{minipage}
    \quad
    \begin{minipage}{8cm}
        \centering
        \begin{tikzpicture}[>=Stealth, scale=0.8]
    \draw [->] (-1, 0) -- (5, 0) node[anchor=west] {$x$};
    \draw [->] (0, -2.5) -- (0, 2.5) node[anchor=east] {$y$};
    \node [font=\footnotesize] at (-0.3, -0.3) {$O$};

    \coordinate (Z1) at (3, 1.5);
    \coordinate (Z2) at (3, -1.5);
    \draw [dashed] (Z1) -- (Z2);
    \filldraw [fill=white] (Z1) circle (0.05);
    \filldraw [fill=white] (Z2) circle (0.05);
    \node at ($(Z1) + (0.8, 0.3)$) {$Z: a + bi$};
    \node at ($(Z2) + (0.8, -0.4)$) {$\bar{Z}: a - bi$};
    \node at (1.8, 0.3) {$a$};
    \node at (3.3, 0.8) {$b$};
    \node at (3.4, -0.8) {$-b$};
\end{tikzpicture}

        \caption{}\label{fig:5-2}
    \end{minipage}
\end{figure}

很明显,按照这种表示方法,每一个复数,有复平面内唯一的一个点和它对应;反过来,
复平面内的每一个点,有唯一的一个复数和它对应。由此可知,复数集 $C$ 和复平面内所有
的点所成的集合是一一对应的。这是复数的一个几何意义。

当两个复数实部相等,虚部互为相反数时,这两个复数叫做互为\textbf{共轭复数}
(当虚部不等于 $0$ 时也叫做互为\textbf{共轭虚数})。复数 $z$ 的共轭复数可以
用 $\bar{z}$ 来表示,也就是说,复数 $z = a + bi$ 的共轭复数是 $\bar{z} = a - bi$。
显然,复平面内表示两个互为共轭复数的点 $Z$ 与 $\bar{Z}$ 关于实轴对称(图 \ref{fig:5-2}),
而实数 $a$(即虚部为 $0$ 的复数)的共轭复数仍是 $a$ 本身。

两个实数可以比较大小。但两个复数,如果不全是实数,就不能比较它们的大小。
关于这个命题的证明,本书从略。


\lianxi
\begin{xiaotis}

\xiaoti{如果 $a,\; b \in R$,在什么况下,$a + bi$ 是实数?是虚数?是纯虚数?各举一些例子。}

\xiaoti{说出下列数(其中 $i$ 是虚数单位)中,哪些是实数,哪些是纯虚数,哪些是复数:\\
    \phantom{空白} $\begin{aligned}[t]
        &2+\sqrt{7},\quad 0.618,\quad \dfrac{2}{7}i,\quad 0,\quad i,\quad i^2, \\
        &5i+8,\quad 3-9\sqrt{2}i,\quad i(1 - \sqrt{3}),\quad 2-i\sqrt{2} \text{。}
    \end{aligned}$
}

\xiaoti{说出下列复数的实部与虚部:\\
    \phantom{空白} $-5+6i,\quad \dfrac{\sqrt{2}}{2}-i\dfrac{\sqrt{2}}{2},\quad -\sqrt{3},\quad i,\quad 0$。
}

\xiaoti{求适合下列方程的 $x$ 与 $y$ $(x,\; y \in R)$ 的值:}
\begin{xiaoxiaotis}

    \xiaoxiaoti{$(3x + 2y) + (5x - y)i = 17 - 2i$;}

    \xiaoxiaoti{$(3x - 4) + (2y + 3)i = 0$。}

\end{xiaoxiaotis}


\xiaoti{说出图中复平面内各点所表示的复数(每个小正方格子边长为 $1$)}:

\begin{figure}[htbp]
    \centering
    \begin{tikzpicture}[>=Stealth, scale=0.6]
    \draw [thick,->] (-5, 0) -- (5, 0) node[anchor=west] {$x$};
    \draw [thick,->] (0, -4) -- (0, 4) node[anchor=east] {$y$};
    \node [font=\footnotesize] at (-0.3, -0.3) {$O$};

    \foreach \x in {-4, -3.5, -3, ..., 4} {
        \draw (\x, 3) -- (\x, -3);
    }

    \foreach \y in {-3, -2.5, -2, ..., 3} {
        \draw (4, \y) -- (-4, \y);
    }

    \foreach \n/\x/\y in {
            A/4/3, B/3/-3, C/-3/1.5, D/-2.5/-2,
            E/5.5/0, F/-2/0, G/0/5, H/0/-5} {
        \coordinate (P) at (\x/2, \y/2);
        \node [fill=white, inner sep=0pt, font=\footnotesize] at ($(P) + (-0.3, -0.3)$) {$\n$};
        \draw [fill=black] (P) circle (2pt);
    }
\end{tikzpicture}

    \caption*{(第5题)}
\end{figure}


\xiaoti{在复平面内描出表示下列复数的点:}
\begin{xiaoxiaotis}

    \renewcommand\arraystretch{1.2}
    \begin{tabular}[t]{*{2}{@{}p{16em}}}
        \xiaoxiaoti{$2 + 5i$;} & \xiaoxiaoti{$-3 + 2i$;} \\
        \xiaoxiaoti{$\dfrac{1}{2} - 4i$;} & \xiaoxiaoti{$-i - 3$;} \\
        \xiaoxiaoti{$5$;} & \xiaoxiaoti{$-3i$;} \\
        \xiaoxiaoti{$6i$;} & \xiaoxiaoti{$-2$;} \\
        \xiaoxiaoti{$1 - i\sqrt{2}$;} & \xiaoxiaoti{$\sqrt{3}$。}
    \end{tabular}

\end{xiaoxiaotis}


\xiaoti{设复数 $z = a + bi$ 和复平面内的点 $Z(a, b)$ 对应,
    $a$,$b$ 必须满足什么条件,才能使点 $Z$ 位于:}
\begin{xiaoxiaotis}

    \xiaoxiaoti{实轴上?}

    \xiaoxiaoti{虚轴上?}

    \xiaoxiaoti{上半平面(不包括实轴)?}

    \xiaoxiaoti{右半平面(不包括原点和虚轴)?}

\end{xiaoxiaotis}


\xiaoti{说出下列复数的共轭复数,并在复平面内把每一对复数表示出来:\\
    $4-3i,\quad -1+i,\quad -5-12i,\quad 4i+\dfrac{1}{2},\quad 4i,\quad -i\sqrt{5}$。
}


\xiaoti{说出复数 $-\dfrac{1}{3}$,$0$,$\pi$ 的共轭复数。}


\xiaoti{判断下列命题的真假,并说明理由:}
\begin{xiaoxiaotis}

    \xiaoxiaoti{$0i$ 是纯虚数;}

    \xiaoxiaoti{原点是复平面内直角坐标系的实轴与虚轴的公共点;}

    \xiaoxiaoti{实数的共轭复数一定是实数,虚数的共轭复数一定是虚数。}

\end{xiaoxiaotis}

\end{xiaotis}

