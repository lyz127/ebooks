\subsubsection{四阶行列式}

一个三阶行列式可以用三个二阶行列式来表示,如

\begin{equation}
    \begin{vmatrix*}
        a_1 & b_1 & c_1 \\
        a_2 & b_2 & c_2 \\
        a_3 & b_3 & c_3
    \end{vmatrix*}
        = (-1)^{1+1}a_1 \begin{vmatrix*}
                b_2 & c_2 \\
                b_3 & c_3
            \end{vmatrix*}
        + (-1)^{1+2}b_1 \begin{vmatrix*}
                a_2 & c_2 \\
                a_3 & c_3
            \end{vmatrix*}
        + (-1)^{1+3}c_1 \begin{vmatrix*}
                a_2 & b_2 \\
                a_3 & b_3
            \end{vmatrix*} \text{。} \label{eq:sijiehls-1}
\end{equation}
所以,我们可以用二阶行列式来定义三阶行列式。仿此,我们可以把\textbf{四阶行列式}定义为:
\begin{align}
    \begin{vmatrix*}
        a_1 & b_1 & c_1 & d_1 \\
        a_2 & b_2 & c_2 & d_2 \\
        a_3 & b_3 & c_3 & d_3 \\
        a_4 & b_4 & c_4 & d_4
    \end{vmatrix*}
        &= (-1)^{1+1}a_1 \begin{vmatrix*}
                b_2 & c_2 & d_2 \\
                b_3 & c_3 & d_3 \\
                b_4 & c_4 & d_4
            \end{vmatrix*}
        + (-1)^{1+2}b_1 \begin{vmatrix*}
                a_2 & c_2 & d_2 \\
                a_3 & c_3 & d_3 \\
                a_4 & c_4 & d_4
            \end{vmatrix*} \notag \\
        &+ (-1)^{1+3}c_1 \begin{vmatrix*}
                a_2 & b_2 & d_2 \\
                a_3 & b_3 & d_3 \\
                a_4 & b_4 & d_4
            \end{vmatrix*}
        + (-1)^{1+4}d_1 \begin{vmatrix*}
                a_2 & b_2 & c_2 \\
                a_3 & b_3 & c_3 \\
                a_4 & b_4 & c_4
            \end{vmatrix*} \text{。}  \label{eq:sijiehls-2}
\end{align}


对这样定义得出的四阶行列式,第 \ref{subsec:4-3} 节中行列式的性质定理和
第 \ref{subsec:4-4} 节中的两个定理都成立(证明从略)。

类似地,可以用四阶行列式来定义五阶行列式,……,用 $n-1$ 阶行列式来定义 $n$ 阶行列式。
第 \ref{subsec:4-3} 节中行列式的性质定理和第 \ref{subsec:4-4} 节中的两个定理对于任意阶行列式也都成立。

但应注意,用对角线法则展开行列式,仅适用于二阶、三阶行列式,不适用于高于三阶的行列式。


\liti 把行列式
$$\begin{vmatrix*}[r]
	5 & 2  & -3 & 0 \\
	1 & -7 & 2  & 6 \\
	6 & -1 & 1  & -2 \\
	3 & 8  & 4  & 2
\end{vmatrix*}$$
按第二列展开。

\jie
\shangyihang\begin{align*}
    & \begin{vmatrix*}[r]
        5 & 2  & -3 & 0 \\
        1 & -7 & 2  & 6 \\
        6 & -1 & 1  & -2 \\
        3 & 8  & 4  & 2
    \end{vmatrix*} \\
    &= 2 \times (-1)^{1+2} \begin{vmatrix*}[r]
            1 & 2  & 6 \\
            6 & 1  & -2 \\
            3 & 4  & 2
        \end{vmatrix*}
    + (-7) \times (-1)^{2+2} \begin{vmatrix*}[r]
            5 & -3 & 0 \\
            6 & 1  & -2 \\
            3 & 4  & 2
        \end{vmatrix*} \\
    &\qquad + (-1) \times (-1)^{3+2} \begin{vmatrix*}[r]
            5 & -3 & 0 \\
            1 & 2  & 6 \\
            3 & 4  & 2
        \end{vmatrix*}
    + 8 \times (-1)^{4+2} \begin{vmatrix*}[r]
            5 & -3 & 0 \\
            1 & 2  & 6 \\
            6 & 1  & -2
        \end{vmatrix*} \\
    &= -2 \begin{vmatrix*}[r]
            1 & 2  & 6 \\
            6 & 1  & -2 \\
            3 & 4  & 2
        \end{vmatrix*}
    - 7 \begin{vmatrix*}[r]
            5 & -3 & 0 \\
            6 & 1  & -2 \\
            3 & 4  & 2
        \end{vmatrix*}
    +   \begin{vmatrix*}[r]
            5 & -3 & 0 \\
            1 & 2  & 6 \\
            3 & 4  & 2
        \end{vmatrix*}
    + 8 \begin{vmatrix*}[r]
            5 & -3 & 0 \\
            1 & 2  & 6 \\
            6 & 1  & -2
        \end{vmatrix*} \text{。}
\end{align*}


\liti 计算
$$\begin{vmatrix*}[r]
	1  & 3  & 7 & 2 \\
	2  & 1  & 0 & -2 \\
	7  & 4  & 1 & -6 \\
	-3 & -2 & 4 & 5
\end{vmatrix*} \text{。}$$

\jie

\begin{align*}
    \begin{vmatrix*}[r]
        1  & 3  & 7 & 2 \\
        2  & 1  & 0 & -2 \\
        7  & 4  & 1 & -6 \\
        -3 & -2 & 4 & 5
    \end{vmatrix*}  & \xlongequal[\text{第二列乘以 (-2) 加到第一列}]{\text{第一列加到第四列}} \begin{vmatrix*}[r]
            -5 & 3  & 7 & 3 \\
            0  & 1  & 0 & 0 \\
            -1 & 4  & 1 & 1 \\
            1  & -2 & 4 & 2
        \end{vmatrix*} \\
    & \xlongequal{\text{按第二行展开}} (-1)^{2+2} \begin{vmatrix*}[r]
            -5 & 7 & 3 \\
            -1 & 1 & 1 \\
            1  & 4 & 2
        \end{vmatrix*} \\
    &= \begin{vmatrix*}[r]
            2 & 4 & 3 \\
            0 & 0 & 1 \\
            5 & 2 & 2
        \end{vmatrix*} = -\begin{vmatrix*}[r]
                2 & 4 \\
                5 & 2
            \end{vmatrix*} = 16 \text{。}
\end{align*}


\liti 求证
$$
\begin{vmatrix*}[r]
    a_0 & -1 & 0  & 0 \\
    a_1 & x  & -1 & 0 \\
    a_2 & 0  & x  & -1 \\
    a_3 & 0  & 0  & x
\end{vmatrix*} = a_0x^3 + a_1x^2 + a_2x + a_3 \text{。}
$$

\zhengming
\shangyihang\begin{flalign*}
    \hspace{5em}\begin{vmatrix*}[r]
        a_0 & -1 & 0  & 0 \\
        a_1 & x  & -1 & 0 \\
        a_2 & 0  & x  & -1 \\
        a_3 & 0  & 0  & x
    \end{vmatrix*} &= a_0 \begin{vmatrix*}[r]
            x  & -1 & 0 \\
            0  & x  & -1 \\
            0  & 0  & x
        \end{vmatrix*} + \begin{vmatrix*}[r]
            a_1 & -1 & 0 \\
            a_2 & x  & -1 \\
            a_3 & 0  & x
        \end{vmatrix*} && \\
    &= a_0x^3 + a_1\begin{vmatrix*}[r]
            x  & -1 \\
            0  & x
        \end{vmatrix*} + \begin{vmatrix*}[r]
            a_2 & -1 \\
            a_3 & x
        \end{vmatrix*} \\
    &= a_0x^3 + a_1x^2 + a_2x + a_3 \text{。}
\end{flalign*}


\liti 利用行列式的性质计算
$$\begin{vmatrix*}
    a & 1 & 1 & 1 \\
    1 & a & 1 & 1 \\
    1 & 1 & a & 1 \\
    1 & 1 & 1 & a
\end{vmatrix*} \text{。}$$

\jie
\shangyihang\begin{flalign*}
    \hspace{4em}\begin{vmatrix*}
        a & 1 & 1 & 1 \\
        1 & a & 1 & 1 \\
        1 & 1 & a & 1 \\
        1 & 1 & 1 & a
    \end{vmatrix*} &= \begin{vmatrix*}
            a + 3 & 1 & 1 & 1 \\
            a + 3 & a & 1 & 1 \\
            a + 3 & 1 & a & 1 \\
            a + 3 & 1 & 1 & a
        \end{vmatrix*} && \\
    &= (a + 3) \begin{vmatrix*}
            1 & 1 & 1 & 1 \\
            1 & a & 1 & 1 \\
            1 & 1 & a & 1 \\
            1 & 1 & 1 & a
        \end{vmatrix*} \\
    &= (a + 3) \begin{vmatrix*}
            1 & 1 & 1 & 1 \\
            0 & a-1 & 0 & 0 \\
            0 & 0 & a-1 & 0 \\
            0 & 0 & 0 & a-1
        \end{vmatrix*} \\
    &= (a + 3)(a -  1)^3 \text{。}
\end{flalign*}


\lianxi
\begin{xiaotis}

\xiaoti{已知行列式}
$$\begin{vmatrix*}[r]
    1  & 2 & 5  & 7 \\
    1  & 0 & -2 & 0 \\
    -1 & 1 & 4  & 5 \\
    3  & 2 & 9  & 9
\end{vmatrix*} \text{,}$$

\begin{xiaoxiaotis}

    \xiaoxiaoti{写出行列式中第三行第二列的元素的余子式及代数余子式;}

    \xiaoxiaoti{把行列式按第二行展开,并进行计算;}

    \xiaoxiaoti{把行列式按第一列展开,并进行计算。}

\end{xiaoxiaotis}

\xiaoti*{比较以上两种计算结果是否相同。}

\xiaoti{利用行列式的性质和展开定理,计算}
$$\begin{vmatrix*}
    0 & q & r & s \\
    p & 0 & r & s \\
    p & q & 0 & s \\
    p & q & r & 0
\end{vmatrix*} \text{。}$$

\end{xiaotis}








