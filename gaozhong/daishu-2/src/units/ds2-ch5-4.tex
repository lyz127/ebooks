\subsection{复数的加法与减法}\label{subsec:5-4}

复数的加法规定按照以下的法则进行:设 $z_1 = a + b\,i$,$z_2 = c + d\,i$
是任意两个复数,那么它们的\textbf{和}
$$(a + b\,i) + (c + d\,i) = (a + c) + (b + d)\,i \text{。}$$

很明显,两个复数的和仍然是一个复数。

容易验证,复数的加法满足交换律、结合律,即对任何 $z_1$,$z_2$,$z_3 \in C$,有
\begin{align*}
    z_1 + z_2 &= z_2 + z_1 \text{,} \\[-1em]
    (z_1 + z_2) + z_3 &= z_1 + (z_2 + z_3) \text{。}
\end{align*}

现在我们来看复数加法的几何意义。

从物理学知道,要求出作用于同一点 $O$、但不在同一直线上的两个力
$\overrightarrow{F1}$ 与 $\overrightarrow{F2}$ 的合力,只要用表示
$\overrightarrow{F1}$ 与 $\overrightarrow{F2}$ 的向量为相邻的两边
画一个平行四边形, 那么,平行四边形中,以力的作用点 $O$ 为起点的那条
对角线所表示的向量就是合力 $\overrightarrow{F}$ (图 \ref{fig:5-5}(1))。
这个法则通常叫做向量加法的\textbf{平行四边形法则}。

\begin{figure}[H]
    \centering
    \begin{minipage}{8cm}
        \centering
        \begin{tikzpicture}[>=Stealth, scale=0.8]
    \draw [->] (-1, 0) -- (4, 0) node[anchor=west] {$x$};
    \draw [->] (0, -1) -- (0, 4) node[anchor=east] {$y$};
    \node at (-0.3, -0.3) {$O$};

    \coordinate (O) at (0, 0);
    \coordinate [label=0:$\overrightarrow{F_1}$] (F1) at (2.2, 0.5);
    \coordinate [label=90:$\overrightarrow{F_2}$] (F2) at (1, 2);
    \coordinate [label=45:$\overrightarrow{F}$] (F) at (3.5, 2.8);
    \draw[->, thick] (O) -- (F1);
    \draw[->, thick] (O) -- (F2);
    \draw[->, thick] (O) -- (F);
    \draw (F1) -- (F) -- (F2);
\end{tikzpicture}

        \caption*{(1)}
    \end{minipage}
    \quad
    \begin{minipage}{8cm}
        \centering
        \begin{tikzpicture}[>=Stealth, scale=0.8]
    \draw [->] (-1, 0) -- (4.5, 0) node[anchor=west] {$x$};
    \draw [->] (0, -1) -- (0, 4) node[anchor=east] {$y$};
    \node at (-0.3, -0.3) {$O$};

    \coordinate (O) at (0, 0);
    \coordinate [label=100:$Z_1$] (Z1) at (2.2, 0.5);
    \coordinate [label=90:$Z_2$] (Z2) at (1, 2);
    \coordinate [label=45:$Z$] (Z) at (3.5, 2.8);
    \draw[->, thick] (O) -- (Z1);
    \draw[->, thick] (O) -- (Z2);
    \draw[->, thick] (O) -- (Z);
    \draw (Z1) -- (Z) -- (Z2);

    \coordinate [label=270:$Q$] (Q) at (1, 0);
    \coordinate [label=270:$P$] (P) at (2.2, 0);
    \coordinate [label=270:$R$] (R) at (3.5, 0);
    \coordinate [label=0:$S$] (S) at (3.5, 0.5);
    \draw[dashed] (Q) -- (Z2);
    \draw[dashed] (P) -- (Z1) -- (S);
    \draw[dashed] (R) -- (Z);
\end{tikzpicture}

        \caption*{(2)}
    \end{minipage}
    \caption{}\label{fig:5-5}
\end{figure}


复数用向量来表示,如果与这些复数对应的向量不在同一直线上,那么这些复数的加法就可以按照向量
加法的平行四边形法则来进行。下面我们来证明这个事实。

设 $\overrightarrow{OZ_1}$ 及 $\overrightarrow{OZ_2}$ 分别与复数 $a + b\,i$ 及 $c + d\,i$ 对应,
且 $\overrightarrow{OZ_1}$ ,$\overrightarrow{OZ_2}$ 不在同一直线上(图 \ref{fig:5-5}(2))。
以 $\overrightarrow{OZ_1}$ 及 $\overrightarrow{OZ_2}$ 为两条邻边画平行四边形 $OZ_1ZZ_2$,
画 $x$ 轴的垂线 $PZ_1$,$QZ_2$ 及 $RZ$,并且画 $Z_1S \perp RZ$,容易证明
$$ \triangle ZZ_1S \quandeng \triangle Z_2OQ \text{,}$$ %$$ \triangle ZZ_1S \cong \triangle Z_2OQ \text{,}$$
并且四边形 $Z_1PRS$ 是矩形,因此
\begin{align*}
    OR &= OP + PR = OP + Z_1S = OP + OQ = a + c, \\[-1em]
    RZ &= RS + SZ = PZ_1 + QZ_2 = b + d \text{。}
\end{align*}

于是,点 $Z$ 的坐标是 $(a + c,\; b + d)$,这说明 $\overrightarrow{OZ}$ 就是与
复数 $(a + c) + (b + d)\,i$ 对应的向量。

由此可知,求两个复数的和,可以先画出与这两个复数对应的
向量 $\overrightarrow{OZ_1}$,$\overrightarrow{OZ_2}$,
如果 $\overrightarrow{OZ_1}$,$\overrightarrow{OZ_2}$ 不在同一直线上,
再以这两个向量为两条邻边画平行四边形,那么与这个平行四边形的对角线 $OZ$
所表示的向量 $\overrightarrow{OZ}$ 对应的复数,就是所求两个复数的和。

如果 $\overrightarrow{OZ_1}$,$\overrightarrow{OZ_2}$ 在同一直线上,
我们可以画出一个“压扁”了的平行四边形,并据此画出它的对角线来表示
$\overrightarrow{OZ_1}$,$\overrightarrow{OZ_2}$ 的和。

总之,复数的加法可以按照向量的加法法则来进行,这是复数加法的几何意义。

下面再来看复数的减法。

复数的减法规定是加法的逆运算,即把满足
$$ (c + d\,i) + (x + y\,i) = a + b\,i $$
的复数 $x + y\,i$,叫做复数 $a + b\,i$ 减去 $c + d\,i$
的\textbf{差},记作 $(a + b\,i) - (c + d\,i)$。根据复数相等的定义,有
$$ c + x = a,\quad d + y = b, $$
由此
$$ x = a - c,\quad y = b - d, $$
所以
$$ x + y\,i = (a - c) + (b - d)\,i, $$
即
$$ (a + b\,i) - (c + d\,i) = (a - c) + (b - d)\,i, $$
这就是复数的减法法则。由此可见,两个复数的差是一个唯一确定的复数。

现设 $\overrightarrow{OZ}$ 与复数 $a + b\,i$ 对应,
$\overrightarrow{OZ_1}$ 与复数 $c + d\,i$ 对应(图 \ref{fig:5-6})。
以 $\overrightarrow{OZ}$ 为一条对角线, $\overrightarrow{OZ_1}$ 为一条边
画平行四边形, 那么这个平行四边形的另一边 $OZ_2$ 所表示的向量 $\overrightarrow{OZ_2}$
就与复数 $(a - c) + (b - d)\,i$ 对应。 因为 $Z_1Z \pxdy OZ_2$,
所以向量 $\overrightarrow{Z_1Z}$ 也与这个差对应。

\begin{figure}[htbp]
    \centering
    \begin{tikzpicture}[>=Stealth, scale=0.8]
    \draw [->] (-1, 0) -- (4, 0) node[anchor=west] {$x$};
    \draw [->] (0, -1) -- (0, 4) node[anchor=east] {$y$};
    \node at (-0.3, -0.3) {$O$};

    \coordinate (O) at (0, 0);
    \coordinate [label=0:$Z_1$] (Z1) at (2.2, 0.5);
    \coordinate [label=90:$Z_2$] (Z2) at (1, 2);
    \coordinate [label=45:$Z$] (Z) at (3.5, 2.8);
    \draw[->, thick] (O) -- (Z1);
    \draw[->, thick] (O) -- (Z2);
    \draw[->, thick] (O) -- (Z);
    \draw (Z1) -- (Z) -- (Z2);
\end{tikzpicture}

    \caption{}\label{fig:5-6}
\end{figure}


这就是说,两个复数的差 $z - z_1$(即 $\overrightarrow{OZ} - \overrightarrow{OZ_1}$)
与连结两个向量终点并指向被减数的向量对应。这是复数减法的几何意义。

由上所述,我们可以看出, 复数的加(减)法与多项式的加(减)法是类似的,
就是把复数的实部与实部、虚部与虚部分别相加(减),即

\begin{center}
    \framebox{\begin{minipage}{20em}
        \begin{gather*}
            (a + b\,i) \pm (c + d\,i) = (a \pm c) + (b \pm d)\,i \text{。}
        \end{gather*}
    \end{minipage}}
\end{center}


\liti 计算 $(5-6\,i) + (-2-i) - (3+4\,i)$。

\jie \quad $\begin{aligned}[t]
       & (5-6\,i) + (-2-i) - (3+4\,i) \\[-1em]
    ={}& (5 - 2 - 3) + (-6 - 1 - 4)\,i \\[-1em]
    ={}& -11\,i \text{。}
\end{aligned}$



\begin{minipage}{\textwidth} % 本页有两个 wrapfigure ,直接使用会报错,所以将其中一个封装在 minipage 中
\begin{wrapfigure}[22]{r}{5cm}
    \centering
    \begin{tikzpicture}[>=Stealth, scale=0.8]
    \draw [->] (-1, 0) -- (4, 0) node[anchor=west] {$x$};
    \draw [->] (0, -1) -- (0, 4) node[anchor=east] {$y$};
    \node at (-0.3, -0.3) {$O$};

    \coordinate (O) at (0, 0);
    \coordinate [label=0:$Z_1$] (Z1) at (3, 1);
    \coordinate [label=90:$Z_2$] (Z2) at (1, 2);
    \draw[->, thick] (O) -- (Z1);
    \draw[->, thick] (O) -- (Z2);
    \draw[->, thick] (Z1) -- (Z2);
\end{tikzpicture}

    \caption{}\label{fig:5-7}
\end{wrapfigure}
\liti 根据复数的几何意义及向量表示,求复平面内两点间的距离公式。

\jie 如图 \ref{fig:5-7} ,设复平面内的任意两点 $Z_1$,$Z_2$ 分别表示复数
$z_1 = x_1 + y_1\,i$,$z_2 = x_2 + y_2\,i$ 出那么 $\overrightarrow{Z_1Z_2}$ 就是与复数 $z_2 - z_1$
对应的向量。如果用 $d$ 表示点 $Z_1$,$Z_2$ 之间的距离, 那么 $d$ 就是向量
$\overrightarrow{Z_1Z_2}$ 的模,即复数 $z_2 - z_1$ 的模,所以
$$ d = |z_2 - z_1| \text{。} $$
这就是复平面内两点间的距离公式。而
\begin{align*}
    d = |z_2 - z_1| &= |(x_2 + y_2\,i) - (x_1 + y_1\,i)| \\
        &= |(x_2 - x_1) + (y_2 - y_1)\,i| \\
        &= \sqrt{(x_2 - x_1)^2 + (y_2 - y_1)^2} \text{。}
\end{align*}

这与我们以前导出的两点间的距离公式一致。
\end{minipage}


\begin{wrapfigure}[22]{r}{5cm}
    \centering
    \begin{tikzpicture}[>=Stealth, scale=0.8]
    \draw [->] (-1, 0) -- (5, 0) node[anchor=west] {$x$};
    \draw [->] (0, -1) -- (0, 5) node[anchor=east] {$y$};
    \node at (-0.3, -0.3) {$O$};

    \coordinate (O) at (0, 0);
    \coordinate [label=0:$P$] (P) at (2, 2.5);
    \draw [name path=c] (P) circle [radius=1.5];
    \path [name path=l] (O) -- (1.5, 5);
    \draw [name intersections={of=c and l, by={Z, Y}}];
    \node [label=90:$Z$] at (Z) {};
    \draw [arrows = {-Stealth[length=10pt,width=5pt]}] (O) -- (Z);
    \draw [arrows = {-Stealth[length=10pt,width=5pt]}] (O) -- (P) -- (Z);
\end{tikzpicture}

    \caption{}\label{fig:5-8}
\end{wrapfigure}
\liti 根据复数的几何意义及向量表示,求复平面内的圆的方程。

\jie 如图 \ref{fig:5-8},设圆心为 $P$,点 $P$ 与复数 $p = a + b\,i$ 对应,
圆的半径为 $r$,圆上任意一点 $Z$ 与复数 $z = x + y\,i$ 对应,那么
$$ |z - p| = r \text{。} $$

这就是复平面内的圆的方程。特别地,当点 $P$ 在原点时,圆的方程就成了 $|z| = r$。

请同学们利用复数的减法法则,把圆的方程 $|z - p| = r$ 化成用实数表示的一般形式
$$ (x - a)^2 + (y - b)^2 = r^2 \text{。} $$


\lianxi
\begin{xiaotis}

\xiaoti{证明复数的加法满足交换律与结合律。}

\xiaoti{分别用代数及几何方法计算:}
\begin{xiaoxiaotis}

    \renewcommand\arraystretch{1.2}
    \begin{tabular}[t]{*{2}{@{}p{16em}}}
        \xiaoxiaoti{$(4+5\,i) + (2+3\,i)$;} & \xiaoxiaoti{$(2+4\,i) + (3-4\,i)$;} \\
        \xiaoxiaoti{$(-3-4\,i) + (-2+i)$;} & \xiaoxiaoti{$-5\,i + (-i-1)$。}
    \end{tabular}

\end{xiaoxiaotis}



\xiaoti{分别用代数及几何方法计算:}
\begin{xiaoxiaotis}

    \renewcommand\arraystretch{1.2}
    \begin{tabular}[t]{*{2}{@{}p{16em}}}
        \xiaoxiaoti{$(4+5\,i) - (3+2\,i)$;} & \xiaoxiaoti{$(-3+2\,i) - (4-5\,i)$;} \\
        \xiaoxiaoti{$(6-3\,i) - (-3\,i-2)$;} & \xiaoxiaoti{$5 - (3+2\,i)$。}
    \end{tabular}

\end{xiaoxiaotis}


\xiaoti{设 $z = a + b\,i$,求 $z + \bar{z}$ 与 $z - \bar{z}$。}


\xiaoti{\begin{minipage}[t]{0.9\textwidth}
    设复平面内的定点 $P$ 与复数 $p = a + b\,i$ 对应,动点 $Z$ 与复数 $z = x + y\,i$ 对应,
    $\varepsilon \in R^+$,满足不等式
        $$ |z - p| < \varepsilon $$
    的点 $Z$ 的集合是什么图形?
\end{minipage}}

\end{xiaotis}

