\subsection{数列}\label{subsec:2-1}

我们看下面的例子:

图 \ref{fig:2-1} 表示堆放的钢管,共堆放了 7 层。自上而下各层的钢管数排列成一列数:
\begin{gather}
    4,\; 5,\; 6,\; 7,\; 8,\; 9,\; 10 \text{。} \tag{$1$}\label{eq:shulie-1}
\end{gather}

\begin{figure}[htbp]
    \centering
    \subsection{数列}\label{subsec:2-1}

我们看下面的例子:

图 \ref{fig:2-1} 表示堆放的钢管,共堆放了 7 层。自上而下各层的钢管数排列成一列数:
\begin{gather}
    4,\; 5,\; 6,\; 7,\; 8,\; 9,\; 10 \text{。} \tag{$1$}\label{eq:shulie-1}
\end{gather}

\begin{figure}[htbp]
    \centering
    \subsection{数列}\label{subsec:2-1}

我们看下面的例子:

图 \ref{fig:2-1} 表示堆放的钢管,共堆放了 7 层。自上而下各层的钢管数排列成一列数:
\begin{gather}
    4,\; 5,\; 6,\; 7,\; 8,\; 9,\; 10 \text{。} \tag{$1$}\label{eq:shulie-1}
\end{gather}

\begin{figure}[htbp]
    \centering
    \subsection{数列}\label{subsec:2-1}

我们看下面的例子:

图 \ref{fig:2-1} 表示堆放的钢管,共堆放了 7 层。自上而下各层的钢管数排列成一列数:
\begin{gather}
    4,\; 5,\; 6,\; 7,\; 8,\; 9,\; 10 \text{。} \tag{$1$}\label{eq:shulie-1}
\end{gather}

\begin{figure}[htbp]
    \centering
    \input{../pic/ds2-ch2-1}
    \caption{}\label{fig:2-1}
\end{figure}

自然数 $1$, $2$, $3$, $4$, $5$, $\cdots$ 的倒数排列成一列数:
\begin{gather}
    1,\; \dfrac{1}{2},\; \dfrac{1}{3},\; \dfrac{1}{4},\; \dfrac{1}{5},\; \cdots, \text{。}  \tag{$2$}\label{eq:shulie-2}
\end{gather}

$\sqrt{2}$ 的精确到 $1$,$0.1$,$0.01$,$0.001$,$\cdots$ 的不足近似值排列成一列数:
\begin{gather}
    1,\; 1.4,\; 1.41,\; 1.414,\; \cdots \text{。} \tag{$3$}\label{eq:shulie-3}
\end{gather}

$-1$ 的 $1$ 次幂,$2$ 次幂,$3$ 次幂,$4$ 次幂,……排列成一列数:
\begin{gather}
    -1,\; 1,\; -1,\; 1,\; -1,\; 1,\; \cdots \text{。} \tag{$4$}\label{eq:shulie-4}
\end{gather}

无穷多个 $1$ 排列成一列数:
\begin{gather}
    1,\; 1,\; 1,\; 1,\; \cdots \text{。}  \tag{$5$}\label{eq:shulie-5}
\end{gather}

像上面的例子中,按一定次序排列的一列数叫做\textbf{数列}。数列中的每一个数都叫做这个数列的\textbf{项},
各项依次叫做这个数列的第 $1$ 项(或首项),第 $2$ 项,……,第 $n$ 项,……。
对于上面的数列(\ref{eq:shulie-1}),每一项与它的序号有下面的对应关系:

\vspace{-1em}
\begin{table}[H]
    \centering
    \begin{tabular}{w{c}{5em}*{7}{w{c}{2em}}}
        项 & 4 & 5 & 6 & 7 & 8 & 9 & 10 \\
         & $\uparrow$ & $\uparrow$ & $\uparrow$ & $\uparrow$ & $\uparrow$ & $\uparrow$ & $\uparrow$ \\
        序号 & 1 & 2 & 3 & 4 & 5 & 6 & 7 \\
    \end{tabular}
\end{table}

这告诉我们:数列可以看作一个定义域为自然数集$N$(或它的有限子集$\{1, 2, \cdots, n\}$)的函数
当自变量从小到大依次取值时对应的一列函数值。

数列的一般形式可以写成
$$ a_1,\; a_2,\; a_3,\; \cdots,\; a_n,\; \cdots, $$
其中 $a_n$ 是数列的第 $n$ 项。有时我们把上面的数列简记作 $\{a_n\}$。例如,把数列
$$ 1,\; \dfrac{1}{2},\; \dfrac{1}{3},\; \cdots,\; \dfrac{1}{n},\; \cdots $$
简记作 $\left\{ \dfrac{1}{n} \right\}$。如果数列 $\{a_n\}$ 的第 $n$ 项 $a_n$ 与 $n$
之间的函数关系可以用一个公式来表示,这个公式就叫做这个数列的\textbf{通项公式}。例如,
数列 (\ref{eq:shulie-1}) 的通项公式是 $a_n = n + 3 \;(n \leqslant 7)$;
数列 (\ref{eq:shulie-2}) 的通项公式是 $a_n = \dfrac{1}{n}$。
如果已知一个数列的通项公式,那么只要依次用 $1$, $2$, $3$, $\cdots$ 去代替公式中的 $n$,
就可以求出这个数列的各项。

数列可以用图形来表示。在画图时,为方便起见,在平面直角坐标系的两个坐标轴上所取的单位长度可以不同。
图 \ref{fig:2-2}(1),(2) 分别是数列 (\ref{eq:shulie-1}), (\ref{eq:shulie-2}) 的图形表示。
从图上看,数列可用一群孤立的点来表示。

\begin{figure}[htbp]
    \centering
    \begin{minipage}{8cm}
    \centering
    \input{../pic/ds2-ch2-2-1}
    \caption*{(1)}
    \end{minipage}
    \qquad
    \begin{minipage}{8cm}
    \centering
    \input{../pic/ds2-ch2-2-2}
    \caption*{(2)}
    \end{minipage}
    \caption{}\label{fig:2-2}
\end{figure}

项数有限的数列叫做\textbf{有穷数列},
项数无限的数列叫做\textbf{无穷数列}。
上面的数列 (\ref{eq:shulie-1}) 是有穷数列, 数列 (\ref{eq:shulie-2}), (\ref{eq:shulie-3}),
(\ref{eq:shulie-4}),(\ref{eq:shulie-5}) 都是无穷数列。


\liti 根据下面数列 $\{a_n\}$ 的通项公式,写出它的前 $5$ 项:
\begin{xiaoxiaotis}

    \twoInLineXxt{$a_n = \dfrac{n}{n+1}$;}{$a_n = (-1)^n \cdot n$。}

\end{xiaoxiaotis}

\jie (1)在通项公式中依次取 $n = 1, 2, 3, 4, 5$,得到数列 $\{a_n\}$ 的前 $5$ 项为
$$ \dfrac{1}{2},\quad \dfrac{2}{3},\quad \dfrac{3}{4},\quad \dfrac{4}{5},\quad \dfrac{5}{6} ; $$

(2)在通项公式中依次取 $n = 1, 2, 3, 4, 5$,得到数列 $\{a_n\}$ 的前 $5$ 项为
$$ -1,\quad 2,\quad -3,\quad 4,\quad -5 \text{。} $$

\liti 写出数列的一个通项公式,使它的前 $4$ 项分别是下列各数:
\begin{xiaoxiaotis}

    \xiaoxiaoti{$1$,$3$,$5$,$7$;}

    \xiaoxiaoti{$\dfrac{2^2 - 1}{2}$,$\dfrac{3^2 - 1}{3}$,$\dfrac{4^2 - 1}{4}$,$\dfrac{5^2 - 1}{5}$;}

    \xiaoxiaoti{$-\dfrac{1}{1 \cdot 2}$,$\dfrac{1}{2 \cdot 3}$,$-\dfrac{1}{3 \cdot 4}$,$\dfrac{1}{4 \cdot 5}$。}

\end{xiaoxiaotis}

\jie (1)数列的前 $4$ 项 $1$,$3$,$5$,$7$ 都是序号的 $2$ 倍减去 $1$,所以通项公式是 $a_n = 2n - 1$;

(2)数列的前 $4$ 项 $\dfrac{2^2 - 1}{2}$,$\dfrac{3^2 - 1}{3}$,$\dfrac{4^2 - 1}{4}$,$\dfrac{5^2 - 1}{5}$
的分母都是序号加上 $1$,分子都是分母的平方减去 $1$,所以通项公式是:
$$ a_n = \dfrac{(n+1)^2 - 1}{n + 1} = \dfrac{n(n + 2)}{n + 1} \text{;}$$

(3)数列的前 $4$ 项 $-\dfrac{1}{1 \cdot 2}$,$\dfrac{1}{2 \cdot 3}$,$-\dfrac{1}{3 \cdot 4}$,$\dfrac{1}{4 \cdot 5}$
的绝对值都等于序号与序号加上 $1$ 的积的倒数,且奇数项为负,偶数项为正,所以通项公式是
$$ a_n = \dfrac{(-1)^n}{n(n + 1)} \text{。}$$


\lianxi

\begin{xiaotis}

\xiaoti{根据下面数列 $\{a_n\}$ 的通项公式,写出它的前 $5$ 项:}
\begin{xiaoxiaotis}

    \renewcommand\arraystretch{1.5}
    \begin{tabular}[t]{*{2}{@{}p{16em}}}
        \xiaoxiaoti{$a_n = n^2$;} & \xiaoxiaoti{$a_n = 10n$;} \\
        \xiaoxiaoti{$a_n = 5(-1)^{n + 1}$;} & \xiaoxiaoti{$a_n = \dfrac{2n + 1}{n^2 + 1}$。}
    \end{tabular}

\end{xiaoxiaotis}


\xiaoti{根据下面数列 $\{a_n\}$ 的通项公式,写出它的第 $7$ 项与第 $10$ 项:}
\begin{xiaoxiaotis}

    \renewcommand\arraystretch{1.5}
    \begin{tabular}[t]{*{2}{@{}p{16em}}}
        \xiaoxiaoti{$a_n = \dfrac{1}{n^3}$;} & \xiaoxiaoti{$a_n = n(n + 2)$;} \\
        \xiaoxiaoti{$a_n = \dfrac{(-1)^{n + 1}}{n}$;} & \xiaoxiaoti{$a_n = -2^n + 3$。}
    \end{tabular}

\end{xiaoxiaotis}

\xiaoti{(口答)说出数列的一个通项公式,使它的前 $4$ 项分别是下列各数:}
\begin{xiaoxiaotis}

    \renewcommand\arraystretch{1.5}
    \begin{tabular}[t]{*{2}{@{}p{16em}}}
        \xiaoxiaoti{$2$,$4$,$6$,$8$;} & \xiaoxiaoti{$15$,$25$,$35$,$45$;} \\
        \xiaoxiaoti{$-\dfrac{1}{2}$,$\dfrac{1}{4}$,$-\dfrac{1}{8}$,$\dfrac{1}{16}$;} & \xiaoxiaoti{$1 - \dfrac{1}{2}$,$\dfrac{1}{2} - \dfrac{1}{3}$,$\dfrac{1}{3} - \dfrac{1}{4}$,$\dfrac{1}{4} - \dfrac{1}{5}$。}
    \end{tabular}

\end{xiaoxiaotis}


\xiaoti{观察下面数列的特点,用适当的数填空,并对每一个数列各写出一个通项公式:}
\begin{xiaoxiaotis}

    \xiaoxiaoti{$2$,$4$,$(\qquad)$,$8$,$10$,$(\qquad)$,$14$;}

    \xiaoxiaoti{$2$,$4$,$(\qquad)$,$16$,$32$,$(\qquad)$,$128$,$(\qquad)$;}

    \xiaoxiaoti{$(\qquad)$,$4$,$9$,$16$,$25$,$(\qquad)$,$49$;}

    \xiaoxiaoti{$(\qquad)$,$4$,$3$,$2$,$1$,$(\qquad)$,$-1$,$(\qquad)$;}

    \xiaoxiaoti{$1$,$\sqrt{2}$,$(\qquad)$,$2$,$\sqrt{5}$,$(\qquad)$,$\sqrt{7}$。}

\end{xiaoxiaotis}

\end{xiaotis}


\liti 已知数列 $\{a_n\}$ 的第 $1$ 项是 $1$,以后各项由公式 $a_n = 1 + \dfrac{1}{a_{n-1}}$
给出,写出这个数列的前 $5$ 项。

\jie $\begin{aligned}[t]
    a_1 &= 1, \\
    a_2 &= 1 + \dfrac{1}{a_1} = 1 + \dfrac{1}{1} =  2, \\
    a_3 &= 1 + \dfrac{1}{a_2} = 1 + \dfrac{1}{2} = \dfrac{3}{2}, \\
    a_4 &= 1 + \dfrac{1}{a_3} = 1 + \dfrac{\;1\;}{\dfrac{3}{2}} = \dfrac{5}{3}, \\
    a_5 &= 1 + \dfrac{1}{a_4} = 1 + \dfrac{\;1\;}{\dfrac{5}{3}} = \dfrac{8}{5} \text{。}
\end{aligned}$

\lianxi

写出下面数列 $\{a_n\}$ 的前 $5$ 项:

\begin{xiaotis}
\setcounter{cntxiaoti}{0}

\xiaoti{$a_1 = 5$,$a_{n+1} = a_n + 3$。}

\xiaoti{$a_1 = 2$,$a_{n+1} = 2a_n$。}

\xiaoti{$a_1 = 3$,$a_2 = 6$,$a_{n+2} = a_{n+1} - a_n$。}

\xiaoti{$a_1 = 1$,$a_{n+1} = a_n + \dfrac{1}{a_n}$。}

\end{xiaotis}


    \caption{}\label{fig:2-1}
\end{figure}

自然数 $1$, $2$, $3$, $4$, $5$, $\cdots$ 的倒数排列成一列数:
\begin{gather}
    1,\; \dfrac{1}{2},\; \dfrac{1}{3},\; \dfrac{1}{4},\; \dfrac{1}{5},\; \cdots, \text{。}  \tag{$2$}\label{eq:shulie-2}
\end{gather}

$\sqrt{2}$ 的精确到 $1$,$0.1$,$0.01$,$0.001$,$\cdots$ 的不足近似值排列成一列数:
\begin{gather}
    1,\; 1.4,\; 1.41,\; 1.414,\; \cdots \text{。} \tag{$3$}\label{eq:shulie-3}
\end{gather}

$-1$ 的 $1$ 次幂,$2$ 次幂,$3$ 次幂,$4$ 次幂,……排列成一列数:
\begin{gather}
    -1,\; 1,\; -1,\; 1,\; -1,\; 1,\; \cdots \text{。} \tag{$4$}\label{eq:shulie-4}
\end{gather}

无穷多个 $1$ 排列成一列数:
\begin{gather}
    1,\; 1,\; 1,\; 1,\; \cdots \text{。}  \tag{$5$}\label{eq:shulie-5}
\end{gather}

像上面的例子中,按一定次序排列的一列数叫做\textbf{数列}。数列中的每一个数都叫做这个数列的\textbf{项},
各项依次叫做这个数列的第 $1$ 项(或首项),第 $2$ 项,……,第 $n$ 项,……。
对于上面的数列(\ref{eq:shulie-1}),每一项与它的序号有下面的对应关系:

\vspace{-1em}
\begin{table}[H]
    \centering
    \begin{tabular}{w{c}{5em}*{7}{w{c}{2em}}}
        项 & 4 & 5 & 6 & 7 & 8 & 9 & 10 \\
         & $\uparrow$ & $\uparrow$ & $\uparrow$ & $\uparrow$ & $\uparrow$ & $\uparrow$ & $\uparrow$ \\
        序号 & 1 & 2 & 3 & 4 & 5 & 6 & 7 \\
    \end{tabular}
\end{table}

这告诉我们:数列可以看作一个定义域为自然数集$N$(或它的有限子集$\{1, 2, \cdots, n\}$)的函数
当自变量从小到大依次取值时对应的一列函数值。

数列的一般形式可以写成
$$ a_1,\; a_2,\; a_3,\; \cdots,\; a_n,\; \cdots, $$
其中 $a_n$ 是数列的第 $n$ 项。有时我们把上面的数列简记作 $\{a_n\}$。例如,把数列
$$ 1,\; \dfrac{1}{2},\; \dfrac{1}{3},\; \cdots,\; \dfrac{1}{n},\; \cdots $$
简记作 $\left\{ \dfrac{1}{n} \right\}$。如果数列 $\{a_n\}$ 的第 $n$ 项 $a_n$ 与 $n$
之间的函数关系可以用一个公式来表示,这个公式就叫做这个数列的\textbf{通项公式}。例如,
数列 (\ref{eq:shulie-1}) 的通项公式是 $a_n = n + 3 \;(n \leqslant 7)$;
数列 (\ref{eq:shulie-2}) 的通项公式是 $a_n = \dfrac{1}{n}$。
如果已知一个数列的通项公式,那么只要依次用 $1$, $2$, $3$, $\cdots$ 去代替公式中的 $n$,
就可以求出这个数列的各项。

数列可以用图形来表示。在画图时,为方便起见,在平面直角坐标系的两个坐标轴上所取的单位长度可以不同。
图 \ref{fig:2-2}(1),(2) 分别是数列 (\ref{eq:shulie-1}), (\ref{eq:shulie-2}) 的图形表示。
从图上看,数列可用一群孤立的点来表示。

\begin{figure}[htbp]
    \centering
    \begin{minipage}{8cm}
    \centering
    \begin{tikzpicture}[>=Stealth]
    \draw [->] (-1,0) -- (4,0) node[anchor=north] {$n$};
    \draw [->] (0,-1) -- (0,5.5) node[anchor=east] {$a_n$};
    \node at (-0.2,-0.2) {$O$};
    \foreach \x in {1,...,7} {
        \draw (\x/2,0.2) -- (\x/2,0) node[anchor=north] {$\x$};
    }
    \foreach \y in {1,...,10} {
        \draw (0.2,\y/2) -- (0,\y/2) node[anchor=east] {$\y$};
    }

    \foreach \n in {4,...,10} {
        \filldraw [fill=black] (\n/2 - 1.5, \n/2) circle (0.1);
    }
\end{tikzpicture}
    \caption*{(1)}
    \end{minipage}
    \qquad
    \begin{minipage}{8cm}
    \centering
    \begin{tikzpicture}[>=Stealth]
    \draw [->] (-1,0) -- (4,0) node[anchor=north] {$n$};
    \draw [->] (0,-1) -- (0,5.5) node[anchor=east] {$a_n$};
    \node at (-0.2,-0.2) {$O$};
    \foreach \x in {1,...,7} {
        \draw (\x/2,0.2) -- (\x/2,0) node[anchor=north] {$\x$};
    }
    \draw (0.2, 5) -- (0,5) node[anchor=east] {$1$};
    \foreach \y in {2, 4, 8} {
        \draw (0.2, 5/\y) -- (0,5/\y) node[anchor=east] {$\frac{1}{\y}$};
    }

    \foreach \n in {1,...,7} {
        \filldraw [fill=black] (\n/2, 5/\n) circle (0.1);
    }
\end{tikzpicture}


    \caption*{(2)}
    \end{minipage}
    \caption{}\label{fig:2-2}
\end{figure}

项数有限的数列叫做\textbf{有穷数列},
项数无限的数列叫做\textbf{无穷数列}。
上面的数列 (\ref{eq:shulie-1}) 是有穷数列, 数列 (\ref{eq:shulie-2}), (\ref{eq:shulie-3}),
(\ref{eq:shulie-4}),(\ref{eq:shulie-5}) 都是无穷数列。


\liti 根据下面数列 $\{a_n\}$ 的通项公式,写出它的前 $5$ 项:
\begin{xiaoxiaotis}

    \twoInLineXxt{$a_n = \dfrac{n}{n+1}$;}{$a_n = (-1)^n \cdot n$。}

\end{xiaoxiaotis}

\jie (1)在通项公式中依次取 $n = 1, 2, 3, 4, 5$,得到数列 $\{a_n\}$ 的前 $5$ 项为
$$ \dfrac{1}{2},\quad \dfrac{2}{3},\quad \dfrac{3}{4},\quad \dfrac{4}{5},\quad \dfrac{5}{6} ; $$

(2)在通项公式中依次取 $n = 1, 2, 3, 4, 5$,得到数列 $\{a_n\}$ 的前 $5$ 项为
$$ -1,\quad 2,\quad -3,\quad 4,\quad -5 \text{。} $$

\liti 写出数列的一个通项公式,使它的前 $4$ 项分别是下列各数:
\begin{xiaoxiaotis}

    \xiaoxiaoti{$1$,$3$,$5$,$7$;}

    \xiaoxiaoti{$\dfrac{2^2 - 1}{2}$,$\dfrac{3^2 - 1}{3}$,$\dfrac{4^2 - 1}{4}$,$\dfrac{5^2 - 1}{5}$;}

    \xiaoxiaoti{$-\dfrac{1}{1 \cdot 2}$,$\dfrac{1}{2 \cdot 3}$,$-\dfrac{1}{3 \cdot 4}$,$\dfrac{1}{4 \cdot 5}$。}

\end{xiaoxiaotis}

\jie (1)数列的前 $4$ 项 $1$,$3$,$5$,$7$ 都是序号的 $2$ 倍减去 $1$,所以通项公式是 $a_n = 2n - 1$;

(2)数列的前 $4$ 项 $\dfrac{2^2 - 1}{2}$,$\dfrac{3^2 - 1}{3}$,$\dfrac{4^2 - 1}{4}$,$\dfrac{5^2 - 1}{5}$
的分母都是序号加上 $1$,分子都是分母的平方减去 $1$,所以通项公式是:
$$ a_n = \dfrac{(n+1)^2 - 1}{n + 1} = \dfrac{n(n + 2)}{n + 1} \text{;}$$

(3)数列的前 $4$ 项 $-\dfrac{1}{1 \cdot 2}$,$\dfrac{1}{2 \cdot 3}$,$-\dfrac{1}{3 \cdot 4}$,$\dfrac{1}{4 \cdot 5}$
的绝对值都等于序号与序号加上 $1$ 的积的倒数,且奇数项为负,偶数项为正,所以通项公式是
$$ a_n = \dfrac{(-1)^n}{n(n + 1)} \text{。}$$


\lianxi

\begin{xiaotis}

\xiaoti{根据下面数列 $\{a_n\}$ 的通项公式,写出它的前 $5$ 项:}
\begin{xiaoxiaotis}

    \renewcommand\arraystretch{1.5}
    \begin{tabular}[t]{*{2}{@{}p{16em}}}
        \xiaoxiaoti{$a_n = n^2$;} & \xiaoxiaoti{$a_n = 10n$;} \\
        \xiaoxiaoti{$a_n = 5(-1)^{n + 1}$;} & \xiaoxiaoti{$a_n = \dfrac{2n + 1}{n^2 + 1}$。}
    \end{tabular}

\end{xiaoxiaotis}


\xiaoti{根据下面数列 $\{a_n\}$ 的通项公式,写出它的第 $7$ 项与第 $10$ 项:}
\begin{xiaoxiaotis}

    \renewcommand\arraystretch{1.5}
    \begin{tabular}[t]{*{2}{@{}p{16em}}}
        \xiaoxiaoti{$a_n = \dfrac{1}{n^3}$;} & \xiaoxiaoti{$a_n = n(n + 2)$;} \\
        \xiaoxiaoti{$a_n = \dfrac{(-1)^{n + 1}}{n}$;} & \xiaoxiaoti{$a_n = -2^n + 3$。}
    \end{tabular}

\end{xiaoxiaotis}

\xiaoti{(口答)说出数列的一个通项公式,使它的前 $4$ 项分别是下列各数:}
\begin{xiaoxiaotis}

    \renewcommand\arraystretch{1.5}
    \begin{tabular}[t]{*{2}{@{}p{16em}}}
        \xiaoxiaoti{$2$,$4$,$6$,$8$;} & \xiaoxiaoti{$15$,$25$,$35$,$45$;} \\
        \xiaoxiaoti{$-\dfrac{1}{2}$,$\dfrac{1}{4}$,$-\dfrac{1}{8}$,$\dfrac{1}{16}$;} & \xiaoxiaoti{$1 - \dfrac{1}{2}$,$\dfrac{1}{2} - \dfrac{1}{3}$,$\dfrac{1}{3} - \dfrac{1}{4}$,$\dfrac{1}{4} - \dfrac{1}{5}$。}
    \end{tabular}

\end{xiaoxiaotis}


\xiaoti{观察下面数列的特点,用适当的数填空,并对每一个数列各写出一个通项公式:}
\begin{xiaoxiaotis}

    \xiaoxiaoti{$2$,$4$,$(\qquad)$,$8$,$10$,$(\qquad)$,$14$;}

    \xiaoxiaoti{$2$,$4$,$(\qquad)$,$16$,$32$,$(\qquad)$,$128$,$(\qquad)$;}

    \xiaoxiaoti{$(\qquad)$,$4$,$9$,$16$,$25$,$(\qquad)$,$49$;}

    \xiaoxiaoti{$(\qquad)$,$4$,$3$,$2$,$1$,$(\qquad)$,$-1$,$(\qquad)$;}

    \xiaoxiaoti{$1$,$\sqrt{2}$,$(\qquad)$,$2$,$\sqrt{5}$,$(\qquad)$,$\sqrt{7}$。}

\end{xiaoxiaotis}

\end{xiaotis}


\liti 已知数列 $\{a_n\}$ 的第 $1$ 项是 $1$,以后各项由公式 $a_n = 1 + \dfrac{1}{a_{n-1}}$
给出,写出这个数列的前 $5$ 项。

\jie $\begin{aligned}[t]
    a_1 &= 1, \\
    a_2 &= 1 + \dfrac{1}{a_1} = 1 + \dfrac{1}{1} =  2, \\
    a_3 &= 1 + \dfrac{1}{a_2} = 1 + \dfrac{1}{2} = \dfrac{3}{2}, \\
    a_4 &= 1 + \dfrac{1}{a_3} = 1 + \dfrac{\;1\;}{\dfrac{3}{2}} = \dfrac{5}{3}, \\
    a_5 &= 1 + \dfrac{1}{a_4} = 1 + \dfrac{\;1\;}{\dfrac{5}{3}} = \dfrac{8}{5} \text{。}
\end{aligned}$

\lianxi

写出下面数列 $\{a_n\}$ 的前 $5$ 项:

\begin{xiaotis}
\setcounter{cntxiaoti}{0}

\xiaoti{$a_1 = 5$,$a_{n+1} = a_n + 3$。}

\xiaoti{$a_1 = 2$,$a_{n+1} = 2a_n$。}

\xiaoti{$a_1 = 3$,$a_2 = 6$,$a_{n+2} = a_{n+1} - a_n$。}

\xiaoti{$a_1 = 1$,$a_{n+1} = a_n + \dfrac{1}{a_n}$。}

\end{xiaotis}


    \caption{}\label{fig:2-1}
\end{figure}

自然数 $1$, $2$, $3$, $4$, $5$, $\cdots$ 的倒数排列成一列数:
\begin{gather}
    1,\; \dfrac{1}{2},\; \dfrac{1}{3},\; \dfrac{1}{4},\; \dfrac{1}{5},\; \cdots, \text{。}  \tag{$2$}\label{eq:shulie-2}
\end{gather}

$\sqrt{2}$ 的精确到 $1$,$0.1$,$0.01$,$0.001$,$\cdots$ 的不足近似值排列成一列数:
\begin{gather}
    1,\; 1.4,\; 1.41,\; 1.414,\; \cdots \text{。} \tag{$3$}\label{eq:shulie-3}
\end{gather}

$-1$ 的 $1$ 次幂,$2$ 次幂,$3$ 次幂,$4$ 次幂,……排列成一列数:
\begin{gather}
    -1,\; 1,\; -1,\; 1,\; -1,\; 1,\; \cdots \text{。} \tag{$4$}\label{eq:shulie-4}
\end{gather}

无穷多个 $1$ 排列成一列数:
\begin{gather}
    1,\; 1,\; 1,\; 1,\; \cdots \text{。}  \tag{$5$}\label{eq:shulie-5}
\end{gather}

像上面的例子中,按一定次序排列的一列数叫做\textbf{数列}。数列中的每一个数都叫做这个数列的\textbf{项},
各项依次叫做这个数列的第 $1$ 项(或首项),第 $2$ 项,……,第 $n$ 项,……。
对于上面的数列(\ref{eq:shulie-1}),每一项与它的序号有下面的对应关系:

\vspace{-1em}
\begin{table}[H]
    \centering
    \begin{tabular}{w{c}{5em}*{7}{w{c}{2em}}}
        项 & 4 & 5 & 6 & 7 & 8 & 9 & 10 \\
         & $\uparrow$ & $\uparrow$ & $\uparrow$ & $\uparrow$ & $\uparrow$ & $\uparrow$ & $\uparrow$ \\
        序号 & 1 & 2 & 3 & 4 & 5 & 6 & 7 \\
    \end{tabular}
\end{table}

这告诉我们:数列可以看作一个定义域为自然数集$N$(或它的有限子集$\{1, 2, \cdots, n\}$)的函数
当自变量从小到大依次取值时对应的一列函数值。

数列的一般形式可以写成
$$ a_1,\; a_2,\; a_3,\; \cdots,\; a_n,\; \cdots, $$
其中 $a_n$ 是数列的第 $n$ 项。有时我们把上面的数列简记作 $\{a_n\}$。例如,把数列
$$ 1,\; \dfrac{1}{2},\; \dfrac{1}{3},\; \cdots,\; \dfrac{1}{n},\; \cdots $$
简记作 $\left\{ \dfrac{1}{n} \right\}$。如果数列 $\{a_n\}$ 的第 $n$ 项 $a_n$ 与 $n$
之间的函数关系可以用一个公式来表示,这个公式就叫做这个数列的\textbf{通项公式}。例如,
数列 (\ref{eq:shulie-1}) 的通项公式是 $a_n = n + 3 \;(n \leqslant 7)$;
数列 (\ref{eq:shulie-2}) 的通项公式是 $a_n = \dfrac{1}{n}$。
如果已知一个数列的通项公式,那么只要依次用 $1$, $2$, $3$, $\cdots$ 去代替公式中的 $n$,
就可以求出这个数列的各项。

数列可以用图形来表示。在画图时,为方便起见,在平面直角坐标系的两个坐标轴上所取的单位长度可以不同。
图 \ref{fig:2-2}(1),(2) 分别是数列 (\ref{eq:shulie-1}), (\ref{eq:shulie-2}) 的图形表示。
从图上看,数列可用一群孤立的点来表示。

\begin{figure}[htbp]
    \centering
    \begin{minipage}{8cm}
    \centering
    \begin{tikzpicture}[>=Stealth]
    \draw [->] (-1,0) -- (4,0) node[anchor=north] {$n$};
    \draw [->] (0,-1) -- (0,5.5) node[anchor=east] {$a_n$};
    \node at (-0.2,-0.2) {$O$};
    \foreach \x in {1,...,7} {
        \draw (\x/2,0.2) -- (\x/2,0) node[anchor=north] {$\x$};
    }
    \foreach \y in {1,...,10} {
        \draw (0.2,\y/2) -- (0,\y/2) node[anchor=east] {$\y$};
    }

    \foreach \n in {4,...,10} {
        \filldraw [fill=black] (\n/2 - 1.5, \n/2) circle (0.1);
    }
\end{tikzpicture}
    \caption*{(1)}
    \end{minipage}
    \qquad
    \begin{minipage}{8cm}
    \centering
    \begin{tikzpicture}[>=Stealth]
    \draw [->] (-1,0) -- (4,0) node[anchor=north] {$n$};
    \draw [->] (0,-1) -- (0,5.5) node[anchor=east] {$a_n$};
    \node at (-0.2,-0.2) {$O$};
    \foreach \x in {1,...,7} {
        \draw (\x/2,0.2) -- (\x/2,0) node[anchor=north] {$\x$};
    }
    \draw (0.2, 5) -- (0,5) node[anchor=east] {$1$};
    \foreach \y in {2, 4, 8} {
        \draw (0.2, 5/\y) -- (0,5/\y) node[anchor=east] {$\frac{1}{\y}$};
    }

    \foreach \n in {1,...,7} {
        \filldraw [fill=black] (\n/2, 5/\n) circle (0.1);
    }
\end{tikzpicture}


    \caption*{(2)}
    \end{minipage}
    \caption{}\label{fig:2-2}
\end{figure}

项数有限的数列叫做\textbf{有穷数列},
项数无限的数列叫做\textbf{无穷数列}。
上面的数列 (\ref{eq:shulie-1}) 是有穷数列, 数列 (\ref{eq:shulie-2}), (\ref{eq:shulie-3}),
(\ref{eq:shulie-4}),(\ref{eq:shulie-5}) 都是无穷数列。


\liti 根据下面数列 $\{a_n\}$ 的通项公式,写出它的前 $5$ 项:
\begin{xiaoxiaotis}

    \twoInLineXxt{$a_n = \dfrac{n}{n+1}$;}{$a_n = (-1)^n \cdot n$。}

\end{xiaoxiaotis}

\jie (1)在通项公式中依次取 $n = 1, 2, 3, 4, 5$,得到数列 $\{a_n\}$ 的前 $5$ 项为
$$ \dfrac{1}{2},\quad \dfrac{2}{3},\quad \dfrac{3}{4},\quad \dfrac{4}{5},\quad \dfrac{5}{6} ; $$

(2)在通项公式中依次取 $n = 1, 2, 3, 4, 5$,得到数列 $\{a_n\}$ 的前 $5$ 项为
$$ -1,\quad 2,\quad -3,\quad 4,\quad -5 \text{。} $$

\liti 写出数列的一个通项公式,使它的前 $4$ 项分别是下列各数:
\begin{xiaoxiaotis}

    \xiaoxiaoti{$1$,$3$,$5$,$7$;}

    \xiaoxiaoti{$\dfrac{2^2 - 1}{2}$,$\dfrac{3^2 - 1}{3}$,$\dfrac{4^2 - 1}{4}$,$\dfrac{5^2 - 1}{5}$;}

    \xiaoxiaoti{$-\dfrac{1}{1 \cdot 2}$,$\dfrac{1}{2 \cdot 3}$,$-\dfrac{1}{3 \cdot 4}$,$\dfrac{1}{4 \cdot 5}$。}

\end{xiaoxiaotis}

\jie (1)数列的前 $4$ 项 $1$,$3$,$5$,$7$ 都是序号的 $2$ 倍减去 $1$,所以通项公式是 $a_n = 2n - 1$;

(2)数列的前 $4$ 项 $\dfrac{2^2 - 1}{2}$,$\dfrac{3^2 - 1}{3}$,$\dfrac{4^2 - 1}{4}$,$\dfrac{5^2 - 1}{5}$
的分母都是序号加上 $1$,分子都是分母的平方减去 $1$,所以通项公式是:
$$ a_n = \dfrac{(n+1)^2 - 1}{n + 1} = \dfrac{n(n + 2)}{n + 1} \text{;}$$

(3)数列的前 $4$ 项 $-\dfrac{1}{1 \cdot 2}$,$\dfrac{1}{2 \cdot 3}$,$-\dfrac{1}{3 \cdot 4}$,$\dfrac{1}{4 \cdot 5}$
的绝对值都等于序号与序号加上 $1$ 的积的倒数,且奇数项为负,偶数项为正,所以通项公式是
$$ a_n = \dfrac{(-1)^n}{n(n + 1)} \text{。}$$


\lianxi

\begin{xiaotis}

\xiaoti{根据下面数列 $\{a_n\}$ 的通项公式,写出它的前 $5$ 项:}
\begin{xiaoxiaotis}

    \renewcommand\arraystretch{1.5}
    \begin{tabular}[t]{*{2}{@{}p{16em}}}
        \xiaoxiaoti{$a_n = n^2$;} & \xiaoxiaoti{$a_n = 10n$;} \\
        \xiaoxiaoti{$a_n = 5(-1)^{n + 1}$;} & \xiaoxiaoti{$a_n = \dfrac{2n + 1}{n^2 + 1}$。}
    \end{tabular}

\end{xiaoxiaotis}


\xiaoti{根据下面数列 $\{a_n\}$ 的通项公式,写出它的第 $7$ 项与第 $10$ 项:}
\begin{xiaoxiaotis}

    \renewcommand\arraystretch{1.5}
    \begin{tabular}[t]{*{2}{@{}p{16em}}}
        \xiaoxiaoti{$a_n = \dfrac{1}{n^3}$;} & \xiaoxiaoti{$a_n = n(n + 2)$;} \\
        \xiaoxiaoti{$a_n = \dfrac{(-1)^{n + 1}}{n}$;} & \xiaoxiaoti{$a_n = -2^n + 3$。}
    \end{tabular}

\end{xiaoxiaotis}

\xiaoti{(口答)说出数列的一个通项公式,使它的前 $4$ 项分别是下列各数:}
\begin{xiaoxiaotis}

    \renewcommand\arraystretch{1.5}
    \begin{tabular}[t]{*{2}{@{}p{16em}}}
        \xiaoxiaoti{$2$,$4$,$6$,$8$;} & \xiaoxiaoti{$15$,$25$,$35$,$45$;} \\
        \xiaoxiaoti{$-\dfrac{1}{2}$,$\dfrac{1}{4}$,$-\dfrac{1}{8}$,$\dfrac{1}{16}$;} & \xiaoxiaoti{$1 - \dfrac{1}{2}$,$\dfrac{1}{2} - \dfrac{1}{3}$,$\dfrac{1}{3} - \dfrac{1}{4}$,$\dfrac{1}{4} - \dfrac{1}{5}$。}
    \end{tabular}

\end{xiaoxiaotis}


\xiaoti{观察下面数列的特点,用适当的数填空,并对每一个数列各写出一个通项公式:}
\begin{xiaoxiaotis}

    \xiaoxiaoti{$2$,$4$,$(\qquad)$,$8$,$10$,$(\qquad)$,$14$;}

    \xiaoxiaoti{$2$,$4$,$(\qquad)$,$16$,$32$,$(\qquad)$,$128$,$(\qquad)$;}

    \xiaoxiaoti{$(\qquad)$,$4$,$9$,$16$,$25$,$(\qquad)$,$49$;}

    \xiaoxiaoti{$(\qquad)$,$4$,$3$,$2$,$1$,$(\qquad)$,$-1$,$(\qquad)$;}

    \xiaoxiaoti{$1$,$\sqrt{2}$,$(\qquad)$,$2$,$\sqrt{5}$,$(\qquad)$,$\sqrt{7}$。}

\end{xiaoxiaotis}

\end{xiaotis}


\liti 已知数列 $\{a_n\}$ 的第 $1$ 项是 $1$,以后各项由公式 $a_n = 1 + \dfrac{1}{a_{n-1}}$
给出,写出这个数列的前 $5$ 项。

\jie $\begin{aligned}[t]
    a_1 &= 1, \\
    a_2 &= 1 + \dfrac{1}{a_1} = 1 + \dfrac{1}{1} =  2, \\
    a_3 &= 1 + \dfrac{1}{a_2} = 1 + \dfrac{1}{2} = \dfrac{3}{2}, \\
    a_4 &= 1 + \dfrac{1}{a_3} = 1 + \dfrac{\;1\;}{\dfrac{3}{2}} = \dfrac{5}{3}, \\
    a_5 &= 1 + \dfrac{1}{a_4} = 1 + \dfrac{\;1\;}{\dfrac{5}{3}} = \dfrac{8}{5} \text{。}
\end{aligned}$

\lianxi

写出下面数列 $\{a_n\}$ 的前 $5$ 项:

\begin{xiaotis}
\setcounter{cntxiaoti}{0}

\xiaoti{$a_1 = 5$,$a_{n+1} = a_n + 3$。}

\xiaoti{$a_1 = 2$,$a_{n+1} = 2a_n$。}

\xiaoti{$a_1 = 3$,$a_2 = 6$,$a_{n+2} = a_{n+1} - a_n$。}

\xiaoti{$a_1 = 1$,$a_{n+1} = a_n + \dfrac{1}{a_n}$。}

\end{xiaotis}


    \caption{}\label{fig:2-1}
\end{figure}

自然数 $1$, $2$, $3$, $4$, $5$, $\cdots$ 的倒数排列成一列数:
\begin{gather}
    1,\; \dfrac{1}{2},\; \dfrac{1}{3},\; \dfrac{1}{4},\; \dfrac{1}{5},\; \cdots, \text{。}  \tag{$2$}\label{eq:shulie-2}
\end{gather}

$\sqrt{2}$ 的精确到 $1$,$0.1$,$0.01$,$0.001$,$\cdots$ 的不足近似值排列成一列数:
\begin{gather}
    1,\; 1.4,\; 1.41,\; 1.414,\; \cdots \text{。} \tag{$3$}\label{eq:shulie-3}
\end{gather}

$-1$ 的 $1$ 次幂,$2$ 次幂,$3$ 次幂,$4$ 次幂,……排列成一列数:
\begin{gather}
    -1,\; 1,\; -1,\; 1,\; -1,\; 1,\; \cdots \text{。} \tag{$4$}\label{eq:shulie-4}
\end{gather}

无穷多个 $1$ 排列成一列数:
\begin{gather}
    1,\; 1,\; 1,\; 1,\; \cdots \text{。}  \tag{$5$}\label{eq:shulie-5}
\end{gather}

像上面的例子中,按一定次序排列的一列数叫做\textbf{数列}。数列中的每一个数都叫做这个数列的\textbf{项},
各项依次叫做这个数列的第 $1$ 项(或首项),第 $2$ 项,……,第 $n$ 项,……。
对于上面的数列(\ref{eq:shulie-1}),每一项与它的序号有下面的对应关系:

\vspace{-1em}
\begin{table}[H]
    \centering
    \begin{tabular}{w{c}{5em}*{7}{w{c}{2em}}}
        项 & 4 & 5 & 6 & 7 & 8 & 9 & 10 \\
         & $\uparrow$ & $\uparrow$ & $\uparrow$ & $\uparrow$ & $\uparrow$ & $\uparrow$ & $\uparrow$ \\
        序号 & 1 & 2 & 3 & 4 & 5 & 6 & 7 \\
    \end{tabular}
\end{table}

这告诉我们:数列可以看作一个定义域为自然数集$N$(或它的有限子集$\{1, 2, \cdots, n\}$)的函数
当自变量从小到大依次取值时对应的一列函数值。

数列的一般形式可以写成
$$ a_1,\; a_2,\; a_3,\; \cdots,\; a_n,\; \cdots, $$
其中 $a_n$ 是数列的第 $n$ 项。有时我们把上面的数列简记作 $\{a_n\}$。例如,把数列
$$ 1,\; \dfrac{1}{2},\; \dfrac{1}{3},\; \cdots,\; \dfrac{1}{n},\; \cdots $$
简记作 $\left\{ \dfrac{1}{n} \right\}$。如果数列 $\{a_n\}$ 的第 $n$ 项 $a_n$ 与 $n$
之间的函数关系可以用一个公式来表示,这个公式就叫做这个数列的\textbf{通项公式}。例如,
数列 (\ref{eq:shulie-1}) 的通项公式是 $a_n = n + 3 \;(n \leqslant 7)$;
数列 (\ref{eq:shulie-2}) 的通项公式是 $a_n = \dfrac{1}{n}$。
如果已知一个数列的通项公式,那么只要依次用 $1$, $2$, $3$, $\cdots$ 去代替公式中的 $n$,
就可以求出这个数列的各项。

数列可以用图形来表示。在画图时,为方便起见,在平面直角坐标系的两个坐标轴上所取的单位长度可以不同。
图 \ref{fig:2-2}(1),(2) 分别是数列 (\ref{eq:shulie-1}), (\ref{eq:shulie-2}) 的图形表示。
从图上看,数列可用一群孤立的点来表示。

\begin{figure}[htbp]
    \centering
    \begin{minipage}{8cm}
    \centering
    \begin{tikzpicture}[>=Stealth]
    \draw [->] (-1,0) -- (4,0) node[anchor=north] {$n$};
    \draw [->] (0,-1) -- (0,5.5) node[anchor=east] {$a_n$};
    \node at (-0.2,-0.2) {$O$};
    \foreach \x in {1,...,7} {
        \draw (\x/2,0.2) -- (\x/2,0) node[anchor=north] {$\x$};
    }
    \foreach \y in {1,...,10} {
        \draw (0.2,\y/2) -- (0,\y/2) node[anchor=east] {$\y$};
    }

    \foreach \n in {4,...,10} {
        \filldraw [fill=black] (\n/2 - 1.5, \n/2) circle (0.1);
    }
\end{tikzpicture}
    \caption*{(1)}
    \end{minipage}
    \qquad
    \begin{minipage}{8cm}
    \centering
    \begin{tikzpicture}[>=Stealth]
    \draw [->] (-1,0) -- (4,0) node[anchor=north] {$n$};
    \draw [->] (0,-1) -- (0,5.5) node[anchor=east] {$a_n$};
    \node at (-0.2,-0.2) {$O$};
    \foreach \x in {1,...,7} {
        \draw (\x/2,0.2) -- (\x/2,0) node[anchor=north] {$\x$};
    }
    \draw (0.2, 5) -- (0,5) node[anchor=east] {$1$};
    \foreach \y in {2, 4, 8} {
        \draw (0.2, 5/\y) -- (0,5/\y) node[anchor=east] {$\frac{1}{\y}$};
    }

    \foreach \n in {1,...,7} {
        \filldraw [fill=black] (\n/2, 5/\n) circle (0.1);
    }
\end{tikzpicture}


    \caption*{(2)}
    \end{minipage}
    \caption{}\label{fig:2-2}
\end{figure}

项数有限的数列叫做\textbf{有穷数列},
项数无限的数列叫做\textbf{无穷数列}。
上面的数列 (\ref{eq:shulie-1}) 是有穷数列, 数列 (\ref{eq:shulie-2}), (\ref{eq:shulie-3}),
(\ref{eq:shulie-4}),(\ref{eq:shulie-5}) 都是无穷数列。


\liti 根据下面数列 $\{a_n\}$ 的通项公式,写出它的前 $5$ 项:
\begin{xiaoxiaotis}

    \twoInLineXxt{$a_n = \dfrac{n}{n+1}$;}{$a_n = (-1)^n \cdot n$。}

\end{xiaoxiaotis}

\jie (1)在通项公式中依次取 $n = 1, 2, 3, 4, 5$,得到数列 $\{a_n\}$ 的前 $5$ 项为
$$ \dfrac{1}{2},\quad \dfrac{2}{3},\quad \dfrac{3}{4},\quad \dfrac{4}{5},\quad \dfrac{5}{6} ; $$

(2)在通项公式中依次取 $n = 1, 2, 3, 4, 5$,得到数列 $\{a_n\}$ 的前 $5$ 项为
$$ -1,\quad 2,\quad -3,\quad 4,\quad -5 \text{。} $$

\liti 写出数列的一个通项公式,使它的前 $4$ 项分别是下列各数:
\begin{xiaoxiaotis}

    \xiaoxiaoti{$1$,$3$,$5$,$7$;}

    \xiaoxiaoti{$\dfrac{2^2 - 1}{2}$,$\dfrac{3^2 - 1}{3}$,$\dfrac{4^2 - 1}{4}$,$\dfrac{5^2 - 1}{5}$;}

    \xiaoxiaoti{$-\dfrac{1}{1 \cdot 2}$,$\dfrac{1}{2 \cdot 3}$,$-\dfrac{1}{3 \cdot 4}$,$\dfrac{1}{4 \cdot 5}$。}

\end{xiaoxiaotis}

\jie (1)数列的前 $4$ 项 $1$,$3$,$5$,$7$ 都是序号的 $2$ 倍减去 $1$,所以通项公式是 $a_n = 2n - 1$;

(2)数列的前 $4$ 项 $\dfrac{2^2 - 1}{2}$,$\dfrac{3^2 - 1}{3}$,$\dfrac{4^2 - 1}{4}$,$\dfrac{5^2 - 1}{5}$
的分母都是序号加上 $1$,分子都是分母的平方减去 $1$,所以通项公式是:
$$ a_n = \dfrac{(n+1)^2 - 1}{n + 1} = \dfrac{n(n + 2)}{n + 1} \text{;}$$

(3)数列的前 $4$ 项 $-\dfrac{1}{1 \cdot 2}$,$\dfrac{1}{2 \cdot 3}$,$-\dfrac{1}{3 \cdot 4}$,$\dfrac{1}{4 \cdot 5}$
的绝对值都等于序号与序号加上 $1$ 的积的倒数,且奇数项为负,偶数项为正,所以通项公式是
$$ a_n = \dfrac{(-1)^n}{n(n + 1)} \text{。}$$


\lianxi

\begin{xiaotis}

\xiaoti{根据下面数列 $\{a_n\}$ 的通项公式,写出它的前 $5$ 项:}
\begin{xiaoxiaotis}

    \renewcommand\arraystretch{1.5}
    \begin{tabular}[t]{*{2}{@{}p{16em}}}
        \xiaoxiaoti{$a_n = n^2$;} & \xiaoxiaoti{$a_n = 10n$;} \\
        \xiaoxiaoti{$a_n = 5(-1)^{n + 1}$;} & \xiaoxiaoti{$a_n = \dfrac{2n + 1}{n^2 + 1}$。}
    \end{tabular}

\end{xiaoxiaotis}


\xiaoti{根据下面数列 $\{a_n\}$ 的通项公式,写出它的第 $7$ 项与第 $10$ 项:}
\begin{xiaoxiaotis}

    \renewcommand\arraystretch{1.5}
    \begin{tabular}[t]{*{2}{@{}p{16em}}}
        \xiaoxiaoti{$a_n = \dfrac{1}{n^3}$;} & \xiaoxiaoti{$a_n = n(n + 2)$;} \\
        \xiaoxiaoti{$a_n = \dfrac{(-1)^{n + 1}}{n}$;} & \xiaoxiaoti{$a_n = -2^n + 3$。}
    \end{tabular}

\end{xiaoxiaotis}

\xiaoti{(口答)说出数列的一个通项公式,使它的前 $4$ 项分别是下列各数:}
\begin{xiaoxiaotis}

    \renewcommand\arraystretch{1.5}
    \begin{tabular}[t]{*{2}{@{}p{16em}}}
        \xiaoxiaoti{$2$,$4$,$6$,$8$;} & \xiaoxiaoti{$15$,$25$,$35$,$45$;} \\
        \xiaoxiaoti{$-\dfrac{1}{2}$,$\dfrac{1}{4}$,$-\dfrac{1}{8}$,$\dfrac{1}{16}$;} & \xiaoxiaoti{$1 - \dfrac{1}{2}$,$\dfrac{1}{2} - \dfrac{1}{3}$,$\dfrac{1}{3} - \dfrac{1}{4}$,$\dfrac{1}{4} - \dfrac{1}{5}$。}
    \end{tabular}

\end{xiaoxiaotis}


\xiaoti{观察下面数列的特点,用适当的数填空,并对每一个数列各写出一个通项公式:}
\begin{xiaoxiaotis}

    \xiaoxiaoti{$2$,$4$,$(\qquad)$,$8$,$10$,$(\qquad)$,$14$;}

    \xiaoxiaoti{$2$,$4$,$(\qquad)$,$16$,$32$,$(\qquad)$,$128$,$(\qquad)$;}

    \xiaoxiaoti{$(\qquad)$,$4$,$9$,$16$,$25$,$(\qquad)$,$49$;}

    \xiaoxiaoti{$(\qquad)$,$4$,$3$,$2$,$1$,$(\qquad)$,$-1$,$(\qquad)$;}

    \xiaoxiaoti{$1$,$\sqrt{2}$,$(\qquad)$,$2$,$\sqrt{5}$,$(\qquad)$,$\sqrt{7}$。}

\end{xiaoxiaotis}

\end{xiaotis}


\liti 已知数列 $\{a_n\}$ 的第 $1$ 项是 $1$,以后各项由公式 $a_n = 1 + \dfrac{1}{a_{n-1}}$
给出,写出这个数列的前 $5$ 项。

\jie $\begin{aligned}[t]
    a_1 &= 1, \\
    a_2 &= 1 + \dfrac{1}{a_1} = 1 + \dfrac{1}{1} =  2, \\
    a_3 &= 1 + \dfrac{1}{a_2} = 1 + \dfrac{1}{2} = \dfrac{3}{2}, \\
    a_4 &= 1 + \dfrac{1}{a_3} = 1 + \dfrac{\;1\;}{\dfrac{3}{2}} = \dfrac{5}{3}, \\
    a_5 &= 1 + \dfrac{1}{a_4} = 1 + \dfrac{\;1\;}{\dfrac{5}{3}} = \dfrac{8}{5} \text{。}
\end{aligned}$

\lianxi

写出下面数列 $\{a_n\}$ 的前 $5$ 项:

\begin{xiaotis}
\setcounter{cntxiaoti}{0}

\xiaoti{$a_1 = 5$,$a_{n+1} = a_n + 3$。}

\xiaoti{$a_1 = 2$,$a_{n+1} = 2a_n$。}

\xiaoti{$a_1 = 3$,$a_2 = 6$,$a_{n+2} = a_{n+1} - a_n$。}

\xiaoti{$a_1 = 1$,$a_{n+1} = a_n + \dfrac{1}{a_n}$。}

\end{xiaotis}

