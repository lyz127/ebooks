\subsection{复数的向量表示}\label{subsec:5-3}

在物理学中,我们经常遇到力、速度、加速度、电场强度等,这些量,除了要考虑它们的绝对值大小以外,
还要考虑它们的方向。我们把这种既有绝对值大小又有方向的量叫做\textbf{向量}。向量可以用有向
线段来表示,线段的长度就是这个向量的绝对值(叫做这个\textbf{向量的模}),线段的方向(用箭头表示)
就是这个向量的方向。模相等且方向相同的向量,不管它们的起点在哪里,都认为是\textbf{相等的向量}。
在这一规定下,向量可以根据需要进行平移。模为零的向量(它的方向是任意的)叫做\textbf{零向量}。
规定所有零向量相等。

复数可以用向量来表示。如图 \ref{fig:5-3},设复平面内的点 $Z$ 表示复数 $z = a + bi$,
\begin{figure}[htbp]
    \centering
    \begin{tikzpicture}[>=Stealth, scale=0.8]
    \draw [->] (-1, 0) -- (3, 0) node[anchor=west] {$x$};
    \draw [->] (0, -1) -- (0, 3) node[anchor=east] {$y$};
    \node [font=\footnotesize] at (-0.3, -0.3) {$O$};

    \coordinate (Z) at (1.8, 2.2);
    \draw [dashed]
        let
            \p1 = (Z)
        in
            (\x1, 0) -- (\p1);
    \draw [->] (0, 0) -- (Z);
    \node at ($(Z) + (1.2, 0)$) {$Z: a + bi$};
    \node at (0.8, -0.3) {$a$};
    \node at (2.0, 1.1) {$b$};
    \node at (0.6, 1.1) {$r$};
\end{tikzpicture}

    \caption{}\label{fig:5-3}
\end{figure}
连结 $OZ$,如果我们把有向线段 $OZ$(方向是从点 $O$ 指向点 $Z$)看成向量,
记作 $\overrightarrow{OZ}$, 就把复数同向量联系起来了。
很明显,向量 $\overrightarrow{OZ}$ 是由点 $Z$ 唯一确定的;
反过来,点 $Z$ 也可由向量 $\overrightarrow{OZ}$ 唯一确定。
因此,复数集 $C$ 与复平面内所有以原点 $O$ 为起点的向量所成的集合也是一一对应的。
为方便起见, 我们常把复数 $z = a + bi$ 说成点 $Z$ 或说成向量 $\overrightarrow{OZ}$。
此外, 我们还规定,相等的向量表示同一个复数。


图 \ref{fig:5-3} 中的向量 $\overrightarrow{OZ}$ 的模(即有向线段 $OZ$ 的长度)
$r$ 叫做\textbf{复数 $z = a + bi$ 的模(或绝对值)},记作 $|z|$ 或 $|a + bi|$。
如果 $b = 0$,那么 $z = a + bi$ 是一个实数 $a$,它的模就等于 $|a|$
(即 $a$ 在实数意义上的绝对值)。容易看出,
$$ |z| = |a + bi| = r = \sqrt{a^2 + b^2} \text{。}$$

\liti 求复数 $z_1 = 3 + 4i$ 及 $z_2 = -\dfrac{1}{2} - \sqrt{2}i$ 的模,
并且比较它们的模的大小。

\jie $\begin{aligned}[t]
    &|z_1| = \sqrt{3^2 + 4^2} = 5, \\
    &|z_2| = \sqrt{\left( -\dfrac{1}{2} \right)^2 + \left( -\sqrt{2} \right)^2} = \dfrac{3}{2} \text{。}
\end{aligned}$

$\because \quad 5 > \dfrac{3}{2},$

$\therefore \quad |z_1| > |z_2| \text{。}$



\liti 设 $z \in C$,满足下列条件的点 $Z$ 的集合是什么图形?

\twoInLine[16em]{(1) $|z| = 4$;}{(2) $2 < |z| < 4$。}

\jie (1) 复数 $z$ 的模等于 $4$,就是说,向量 $\overrightarrow{OZ}$ 的模
(即点 $Z$ 与原点 $O$ 的距离)等于 $4$ ,所以满足条件 $|z| = 4$ 的
点 $Z$ 的集合是以原点 $O$ 为圆心,以 $4$ 为半径的圆。

(2) 不等式 $2 < |z| < 4$ 可化为不等式组
$$
\begin{cases}
    |z| < 4, \\
    |z| > 2 \text{。}
\end{cases}
$$

不等式 $|z| < 4$ 的解集是圆 $|z| = 4$ 内部所有的点组成的集合,
不等式 $|z| > 2$ 的解集是圆 $|z| = 2$ 外部所有的点组成的集合,
这两个集合的交集,就是上述不等式组的解集,也就是满足条件 $2 < |z| < 4$ 的点 $Z$ 的集合。
容易看出,所求的集合是以原点 $O$ 为圆心,以 $2$ 及 $4$ 为半径的圆所夹的圆环,但不包括圆环的边界(图 \ref{fig:5-4})。

\begin{figure}[htbp]
    \centering
    \begin{tikzpicture}[>=Stealth, scale=0.8]
    \filldraw [lightgray] (0, 0) circle (2);
    \draw [dashed, thick] (0, 0) circle(2);

    \filldraw [white] (0, 0) circle (1);
    \draw [dashed, thick] (0, 0) circle(1);

    \draw [->] (-3, 0) -- (3, 0) node[anchor=west] {$x$};
    \draw [->] (0, -3) -- (0, 3) node[anchor=east] {$y$};
    \node at (-0.3, -0.3) {$O$};
    \node [fill=white, inner sep=0pt] at (1.2, -0.3) {$2$};
    \node at (2.2, -0.3) {$4$};
\end{tikzpicture}

    \caption{}\label{fig:5-4}
\end{figure}


\lianxi
\begin{xiaotis}

\xiaoti{已知复数 $\sqrt{3} + i$,$-2 + 4i$, $-2i$,$4$。}
\begin{xiaoxiaotis}

    \xiaoxiaoti{在复平面内描出表示这些复数的点;}

    \xiaoxiaoti{在复平面内画出表示这些复数的向量;}

    \xiaoxiaoti{求各复数的模。}

\end{xiaoxiaotis}


\xiaoti{求证复平面内分别和复数 $z_1 = 1 + 2i$,$z_2 = \sqrt{2} + \sqrt{3}i$,
    $z_3 = \sqrt{3} - \sqrt{2}i$,$z_4 = -2 + i$ 对应的四点 $Z_1$,$Z_2$,$Z_3$,$Z_4$ 共圆。}


\end{xiaotis}