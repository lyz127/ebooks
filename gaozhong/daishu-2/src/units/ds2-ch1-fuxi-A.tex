{\centering \nonumsubsection{A \hspace{1em} 组}}

\begin{xiaotis}

\xiaoti{}
\begin{xiaoxiaotis}

    \vspace{-1.6em} \begin{minipage}{0.9\textwidth}
    \xiaoxiaoti{怎样的函数可以有反函数? 举出函数和它的反函数的例子。}
    \end{minipage}

    \xiaoxiaoti{函数和它的反函数的图象之间有什么关系?}

\end{xiaoxiaotis}

\xiaoti{}
\begin{xiaoxiaotis}

    \vspace{-1.6em} \begin{minipage}{0.9\textwidth}
    \xiaoxiaoti{写出三角函数的诱导公式;}
    \end{minipage}

    \xiaoxiaoti{写出同角三角函数的基本关系式;}

    \xiaoxiaoti{写出和角、差角、倍角、半角的三角函数的公式;}

    \xiaoxiaoti{写出三角函数的和差化积及积化和差公式。}

\end{xiaoxiaotis}

\xiaoti{画出 $y = x$ 及 $y = \sin(\arcsin x)$ 的图象,并比较两个图象的相同点及不同点。}

\xiaoti{求下列函数的反函数, 并写出反函数的定义域、值域:}
\begin{xiaoxiaotis}

    \renewcommand\arraystretch{1.5}
    \begin{tabular}[t]{*{2}{@{}p{16em}}}
        \xiaoxiaoti{$y = \dfrac{1}{2}\arcsin 3x$;} & \xiaoxiaoti{$y = 2\arccos \dfrac{x}{4}$;} \\
        \xiaoxiaoti{$y = \dfrac{\pi}{2} + \arctan 2x$。} &
    \end{tabular}

\end{xiaoxiaotis}

\xiaoti{求下列各式的值:}
\begin{xiaoxiaotis}

    \xiaoxiaoti{$\sin\left( 2\arcsin \dfrac{1}{4} \right)$;}

    \xiaoxiaoti{$\cos \left[ \dfrac{1}{2} \arcsin \left( -\dfrac{4}{5} \right) \right]$;}

    \xiaoxiaoti{$\sin \left( \arcsin\dfrac{3}{5} - \arccos\dfrac{1}{2} \right)$;}

    \xiaoxiaoti{$\cos \left( \arccos\dfrac{3}{5} - \arcsin\dfrac{5}{13} \right)$。}

\end{xiaoxiaotis}

\xiaoti{求出下列各式里的 $x$:}
\begin{xiaoxiaotis}

    \xiaoxiaoti{$\arcsin\dfrac{20}{29} = \arccos x$;}

    \xiaoxiaoti{$\arcsin x = \arccos\dfrac{5}{12}$;}

    \xiaoxiaoti{$\arccot\dfrac{11}{60} = -\arctan x$。}

\end{xiaoxiaotis}

\xiaoti{当 $\alpha$ 取什么值时,下列三角方程的解集是空集?}
\begin{xiaoxiaotis}

    \renewcommand\arraystretch{1.5}
    \begin{tabular}[t]{*{2}{@{}p{16em}}}
        \xiaoxiaoti{$\sin x = \dfrac{1 + \alpha}{2}$;} & \xiaoxiaoti{$\cos x = \dfrac{1 - \alpha}{2}$。}
    \end{tabular}

\end{xiaoxiaotis}

\xiaoti{解下列方程:}
\begin{xiaoxiaotis}

    \xiaoxiaoti{$4\sin^2 x + (2\sqrt{3}-2)\cos x - (4 - \sqrt{3}) = 0$;}

    \xiaoxiaoti{$\sec^2 x = 1 + \tan x$;}

    \xiaoxiaoti{$\cos 3\theta + 2\cos\theta = 0$;}

    \xiaoxiaoti{$\cos 2\theta + \sin 3\theta = 0$;}

    \xiaoxiaoti{$\tan 3x = \tan 4x$;}

    \xiaoxiaoti{$\dfrac{\sin 2x}{\cos x} = \dfrac{\cos 2x}{\sin x}$;}

    \xiaoxiaoti{$5\cos x + 12\sin x = 13$;}

    \xiaoxiaoti{$4\sin x - 3\cos x = \dfrac{5\sqrt{2}}{2}$。}

\end{xiaoxiaotis}


\xiaoti{解下列方程:}
\begin{xiaoxiaotis}

    \xiaoxiaoti{$\sin^4 x - \cos^4 x = \cos x + \sin x$;}

    \xiaoxiaoti{$\cos\left( x + \dfrac{\pi}{4} \right) + \sec\left( x + \dfrac{\pi}{4} \right) + 2 = 0$;}

    \xiaoxiaoti{$\sin\left( x + \dfrac{\pi}{4} \right) \sin\left( x - \dfrac{\pi}{12} \right) = \dfrac{1}{2}$;}

    \xiaoxiaoti{$\cos 2\theta = \cos\theta + \sin\theta$;}

    \xiaoxiaoti{$5\sin^2 x + 7\sin x \cos x + 4\cos^2 x = 1$;}

    \xiaoxiaoti{$6\sin^2 x + 3\sin x \cos x - 5\cos^2 x = 2$。}

\end{xiaoxiaotis}

\begin{minipage}{0.6\textwidth}

\xiaoti{圆的半径为 $R$,弦长为 $a$,用反正弦表示这条弦所对的圆周角,并确定 $R$ 与 $a$ 的取值范围。}

\xiaoti{如图, 已知正三棱锥 $S-ABC$ 的各侧棱长都等于底面边长 $a$, 又 $E$,$F$ 分别是 $AB$,
$CS$ 的中点, 求 $EF$ 和平面 $ABC$ 所成的角(用反三角函数表示)。}

\xiaoti{炮弹以初速度 $v_0$(米/秒)沿与水平方向成 $\theta$ 角的方向向上射出,它的射程 $s$(米)可用下式表示:
    $$s = \dfrac{v_0^2 \sin 2\theta}{9.8}$$
    已知 $v_0 = 630$(米/秒),要使射程为 $20$ 公里,$\theta$ 角应取多大?}

\end{minipage}
\begin{minipage}{0.3\textwidth}
    \centering
    \begin{tikzpicture}[>=Stealth]
    \coordinate [label=180:$A$] (A) at (0, 0);
    \coordinate [label=270:$B$] (B) at (1.2, -1.8);
    \coordinate [label=0:$C$] (C) at (5, 0);
    \coordinate [label=90:$S$] (S) at (2, 3);
    \path
        let
            \p{ab} = ($ (A) !.5! (B) $),
            \p{cs} = ($ (C) !.5! (S) $)
        in
            coordinate [label=200:$E$] (E) at (\p{ab})
            coordinate [label=50:$F$] (F) at (\p{cs});

    \draw (S) -- (A) -- (B) -- (S) -- (C) -- (B);
    \draw [dashed] (A) -- (C);
    \draw [dashed] (E) -- (F);
\end{tikzpicture}

    (第 11 题)
\end{minipage}

\end{xiaotis}
