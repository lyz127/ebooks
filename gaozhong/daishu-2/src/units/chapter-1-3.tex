\subsection{反正切函数与反余切函数}\label{subsec:1-3}

正切函数 $y = \tan x \; \left( x \in \left( -\dfrac{\pi}{2}, \dfrac{\pi}{2} \right)\right)$
的反函数叫\textbf{反正切函数},记作 $y = \arctan x$,它的定义域是 $(-\infty, +\infty)$,
值域是 $\left( -\dfrac{\pi}{2}, \dfrac{\pi}{2} \right)$。

余切函数 $y = \cot x \; (x \in (0, \pi)$
的反函数叫\textbf{反余切函数},记作 $y = \arccot x$,它的定义域是 $(-\infty, +\infty)$,
值域是 $(0, \pi)$。

由反正切函数与反余切函数的定义,我们得到:
\textbf{$$\tan(\arctan x) = x \text{,}$$
其中 $x \in (-\infty, +\infty), \; \arctan x \in \left( -\dfrac{\pi}{2}, \dfrac{\pi}{2} \right)$;
$$\cot(\arccot x) = x \text{,}$$
其中 $x \in (-\infty, +\infty), \; \arccot x \in (0, \pi)$。}

\begin{figure}[htbp]
    \centering
    \begin{minipage}{8cm}
    \centering
    \begin{tikzpicture}[>=Stealth, scale=0.8]
    \pgfmathsetmacro{\half}{0.5 * pi}
    \draw[dashed] (-2.6, \half) -- (2.6, \half);
    \draw[dashed] (-2.6, -\half) -- (2.6, -\half);
    \draw [->] (-\half-1.0, 0) -- (\half+1.0, 0) node[anchor=west] {$x$};
    \draw [->] (0, -\half-1.0) -- (0, \half+1.0) node[anchor=east] {$y$};
    \node [font=\footnotesize, fill=white, inner sep=0pt] at (0.3, -0.3) {$O$};
    \foreach \x / \name in {
        -\half/$-\dfrac{\pi}{2}$,
        \half/$\dfrac{\pi}{2}$} {
        \draw (\x, 0.2) -- (\x, 0) node [anchor=north, font=\footnotesize] {\name};
    }

    \foreach \y / \name in {
        -\half/$-\dfrac{\pi}{2}$,
        \half/$\dfrac{\pi}{2}$} {
        \draw (0.2, \y) -- (0, \y) node [anchor=east, font=\footnotesize, fill=white, inner sep=1pt]{\name};
    }

    \draw (-2.2, -2.2) -- (2.2, 2.2) node [anchor=north west] {$y = x$};
    \draw[dashed, domain=-1.2:1.2,smooth] plot (\x, {tan(\x r)}) +(1.5, -0.2) node [above] {$y = \tan x \quad x \in \left( -\dfrac{\pi}{2}, \dfrac{\pi}{2} \right)$};
    \draw[domain=-2.572:2.572,smooth] plot (\x, {rad(atan(\x))}) node [anchor=north west] {$y = \arctan x$};
    % tan(1.2) = 2.572
\end{tikzpicture}

    \caption{}\label{fig:1-6}
    \end{minipage}
    \qquad
    \begin{minipage}{8cm}
    \centering
    \begin{tikzpicture}[>=Stealth, scale=0.8]
    \pgfmathsetmacro{\half}{0.5 * pi}

    \draw[dashed] (-2.5, pi) -- (3.6, pi);
    \draw [->] (-2.5, 0) -- (3.8, 0) node[anchor=west] {$x$};
    \draw [->] (0, -3) -- (0, pi+0.8) node[anchor=east] {$y$};
    \node [font=\footnotesize, fill=white, inner sep=0pt] at (0.3, -0.3) {$O$};

    \node[below, font=\footnotesize] at (\half, 0) {$\frac{\pi}{2}$};
    \node[below, font=\footnotesize] at (pi, 0) {$\pi$};

    \node[left, font=\footnotesize] at (0, \half) {$\frac{\pi}{2}$};
    \node[left, font=\footnotesize, fill=white, inner sep=1pt] at (0, pi) {$\pi$};

    \draw (-1.5, -1.5) -- (pi+0.2, pi+0.2) node [above] {$y = x$};
    \draw[dashed, domain=0.28:pi-0.4,smooth] plot (\x, {cot(\x r)}) node [below] {$y = \cot x \quad x \in [0, \pi]$};
    \draw[domain=3.477:-2.365,smooth,samples=50] plot (\x, {rad(90 - atan(\x))}) node at (-2.0, 1.9) {$y = \mathrm{arccot}\,x$};
\end{tikzpicture}

    \caption{}\label{fig:1-7}
    \end{minipage}
\end{figure}

图\ref{fig:1-6}与图\ref{fig:1-7}分别是反正切函数与反余切函数的图象。

从图象上可以看出:

\textbf{(1)反正切函数 $y = \arctan x$ 在区间 $(-\infty, +\infty)$ 上是增函数;
反余切函数 $y = \arccot x$ 在区间 $(-\infty, +\infty)$上是减函数。}

\textbf{(2)反正切函数 $y = \arctan x$ 是奇函数,即}
$$\arctan(-x) = -\arctan x, \; x \in (-\infty, +\infty) \text{。}$$

\textbf{(3)反余切函数有下述关系:}
$$\arccot(-x) = \pi - \arccot x, \; x \in (-\infty, +\infty) \text{。}$$

这个性质与反余弦函数是类似的。

反正弦函数、反余弦函数、反正切函数、反余切函数,都叫做\textbf{反三角函数}。
\footnote{反三角函数还有反正割函数和反余割函数两种。这两种反三角函数在本书中不研究。}

\liti 求下列各式的值:
\begin{xiaoxiaotis}

    %\renewcommand\arraystretch{1.5}
    \begin{tabular}[t]{*{2}{@{}p{16em}}}
        \xiaoxiaoti{$\arctan 0$;} & \xiaoxiaoti{$\arctan(-2)$;} \\
        \xiaoxiaoti{$\arccot 1$;} & \xiaoxiaoti{$\arccot(-\sqrt{3})$。}
    \end{tabular}

\end{xiaoxiaotis}

\jie (1)$\arctan 0 = 0$;

(2)$\arctan(-2) = -\arctan 2 = -63^\circ 26'$;

(3)$\arccot 1 = \dfrac{\pi}{4}$;

(4)$\arccot(-\sqrt{3}) = \pi - \arccot\sqrt{3} = \pi - \dfrac{\pi}{6} = \dfrac{5\pi}{6}$。

\liti 求证 $\arctan x + \arccot x = \dfrac{\pi}{2}$。

\zhengming 根据诱导公式与反余切函数的定义,得
$$\tan\left( \dfrac{\pi}{2} - \arccot x \right) = \cot(\arccot x) = x \text{,}$$

因此,$\dfrac{\pi}{2} - \arccot x$ 是正切等于 $x$ 的一个值。

又因为 $0 < \arccot x < \pi$,所以 $0 > -\arccot x > -\pi$,由此可得
$\dfrac{\pi}{2} > \dfrac{\pi}{2} - \arccot x > -\dfrac{\pi}{2}$,即
$\dfrac{\pi}{2} - \arccot x \in \left( -\dfrac{\pi}{2}, \dfrac{\pi}{2} \right)$。

因此,$\dfrac{\pi}{2} - \arccot x$ 是属于 $\left( -\dfrac{\pi}{2}, \dfrac{\pi}{2} \right)$
且它的正切等于 $x$ 的一个值。于是
$$\arctan x = \dfrac{\pi}{2} - \arccot x \text{,}$$

$\therefore \quad \arctan x + \arccot x = \dfrac{\pi}{2}$。

\lianxi
\begin{xiaotis}

\xiaoti{用反正切或反余切的形式把下列各式中的 $x$ 表示出来:}
\begin{xiaoxiaotis}

    \renewcommand\arraystretch{1.5}
    \begin{tabular}[t]{*{2}{@{}p{18em}}}
        \xiaoxiaoti{$\tan x = 0.6 \; \left( -\dfrac{\pi}{2} < x < \dfrac{\pi}{2} \right)$;} & \xiaoxiaoti{$\tan x - \sqrt{5} = 0 \; \left( -\dfrac{\pi}{2} < x < \dfrac{\pi}{2} \right)$;} \\
        \xiaoxiaoti{$\cot x = 3 \; (0 < x < \pi)$;} & \xiaoxiaoti{$3\cot x + 1 = 0 \; (0 < x < \pi)$。}
    \end{tabular}

\end{xiaoxiaotis}

\xiaoti{写出下列函数的定义域、值域:}
\begin{xiaoxiaotis}

    \twoInLineXxt[16em]{$y = \arctan\dfrac{x}{2}$;}{$y = 3\arccot(1 - x)$。}

\end{xiaoxiaotis}

\xiaoti{求下列各式的值:}
\begin{xiaoxiaotis}

    \renewcommand\arraystretch{1.5}
    \begin{tabular}[t]{*{2}{@{}p{16em}}}
        \xiaoxiaoti{$\arctan\dfrac{\sqrt{3}}{3}$;} & \xiaoxiaoti{$\arctan(-2.689)$;} \\
        \xiaoxiaoti{$\arccot 0$;} & \xiaoxiaoti{$\arccot(-1)$。}
    \end{tabular}

\end{xiaoxiaotis}

\xiaoti{求下列各式的值:}
\begin{xiaoxiaotis}

    \renewcommand\arraystretch{1.5}
    \begin{tabular}[t]{*{2}{@{}p{16em}}}
        \xiaoxiaoti{$\tan(\arctan 2.84)$;} & \xiaoxiaoti{$\arctan\left( \tan\dfrac{4\pi}{5} \right)$;} \\
        \xiaoxiaoti{$\cot\left[ \arccot\left( -\dfrac{1}{2} \right)\right]$;} & \xiaoxiaoti{$\arccot\left[ \cot\left( -\dfrac{1}{2} \right)\right]$。}
    \end{tabular}

\end{xiaoxiaotis}

\xiaoti{求下列各式的值:}
\begin{xiaoxiaotis}

    \renewcommand\arraystretch{1.5}
    \begin{tabular}[t]{*{2}{@{}p{16em}}}
        \xiaoxiaoti{$\cot(\arctan\sqrt{3})$;} & \xiaoxiaoti{$\sin(\arccot 2)$;} \\
        \xiaoxiaoti{$\tan\left( \arctan\dfrac{1}{4} + \arctan\dfrac{2}{5} \right)$;} & \xiaoxiaoti{$\cos(2\arctan 5)$。}
    \end{tabular}

\end{xiaoxiaotis}

\end{xiaotis}


