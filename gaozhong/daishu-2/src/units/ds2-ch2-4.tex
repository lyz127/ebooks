\subsection{数学归纳法}\label{subsec:2-4}

在第 \ref{subsec:2-2} 节中,我们是这样推导首项为 $a_1$,公差为 $d$ 的等差数列 $\{a_n\}$ 的通项公式的:

\begin{align*}
    &a_1 = a_1 = a_1 + 0d, \\
    &a_2 = a_1 + d = a_1 + 1d, \\
    &a_3 = a_2 + d = a_1 + 2d, \\
    &a_4 = a_3 + d = a_1 + 3d, \\
    &\cdots\cdots\cdots \qquad \cdots\cdots\cdots
\end{align*}

由此得到,等差数列 $\{a_n\}$ 的通项公式是
$$ a_n = a_1 + (n - 1)d \text{。} $$

象这种由一系列有限的特殊事例得出一般结论的推理方法,通常叫做\textbf{归纳法}。用归纳法可以帮助我们从具体事例中发现一般规律。
但是应该注意,仅根据一系列有限的特殊事例所得出的一般结论有时是不正确的。例如一个数列的通项公式是
$$ a_n = (n^2 - 5n + 5)^2 \text{,} $$
容易验证
$$ a_1 = 1,\quad a_2 = 1,\quad a_3 = 1,\quad a_4 = 1, $$
如果我们由此作出结论——对于任何 $n \in N$,$a_n = (n^2 - 5n + 5)^2 = 1$ 都成立,那就是错误的。
事实上, $a_5 = 25 \neq 1$。

对于由归纳法得到的某些与自然数有关的数学命题,我们常常采用下面的方法来证明它们的正确性:
先证明当 $n$ 取第一个值 $n_0$(例如 $n_0 = 1$)时命题成立,然后假设当 $n = k$ 时命题成立,
证明当 $n = k + 1$ 时命题也成立(因为证明了这一点,就可以断定这个命题对于 $n$ 取第一个值
后面的所有自然数也都成立)。这种证明方法叫做\textbf{数学归纳法}。

例如,我们用数学归纳法来证明:如果 $\{a_n\}$ 是一个等差数列,那么
$$ a_n = a_1 + (n - 1)d $$
对一切 $n \in N$ 都成立。

\zhengming (1) 当 $n = 1$ 时,左边是 $a_1$,右边是 $a_1 + 0d = a_1$ ,等式是成立的。

(2) 假设当 $n = k$ 时等式成立,就是
$$ a_k = a_1 + (k - 1)d \text{,} $$
那么,
\begin{align*}
    a_{k+1} &= a_k + d \\
            &= a_1 + (k - 1)d + d \\
            &= a_1 + [(k + 1) - 1]d \text{。}
\end{align*}

这就是说, 当 $n = k + 1$ 时,等式也成立。

根据(1),$n = 1$ 时等式成立, 再根据(2), $n = 1 + 1 = 2$ 时等式也成立。
由于 $n = 2$ 时等式成立,再根据(2),$n = 2 + 1 = 3$ 时等式也成立。
这样递推下去, 就知道 $n = 4$,$5$,$6$,$\cdots$ 时等式都成立。
因此根据(1) 和 (2) 可以断定,等式对任何 $n \in N$ 都成立。

从上面的例子看到,用数学归纳法证明一个与自然数有关的命题的步骤是:

\textbf{(1)证明当 $n$ 取第一个值 $n_0$( 例如 $n_0 = 1$ 或 $2$ 等) 时结论正确;}

\textbf{(2)假设当 $n = k \; (k \in N \text{,且} k \geqslant n_0)$ 时结论正确,证明当 $n = k + 1$ 时结论也正确。}

在完成了这两个步骤以后,就可以断定命题对于从 $n_0$ 开始的所有自然数 $n$ 都正确。

\textbf{例} \quad 用数学归纳法证明
$$ 1 + 3 + 5 + \cdots + (2n - 1) = n^2 \text{。} $$

\zhengming (1) 当 $n = 1$ 时, 左边$=1$, 右边$= 1$,等式成立。

(2) 假设当 $n = k$ 时等式成立,就是
$$ 1 + 3 + 5 + \cdots + (2k - 1) = k^2 \text{,} $$
那么,
\begin{align*}
      & 1 + 3 + 5 + \cdots + (2k - 1) + [2(k + 1) - 1] \\
    = & k^2 + [2(k + 1) - 1] \\
    = & k^2 + 2k + 1 \\
    = & (k + 1)^2 \text{。}
\end{align*}

这就是说,当 $n = k + 1$ 时等式也成立。

根据(1)和(2),可知等式对任何 $n \in N$ 都成立。

\begin{wrapfigure}[17]{r}{5cm}
    \centering
    \begin{tikzpicture}[>=Stealth]
    \foreach \a / \b in {1/2, 3/4} {
        \draw[pattern={Lines[angle=35]}] (0, \a) --  (\a, \a) -- (\a, 0) -- (\b, 0) -- (\b, \b) -- (0, \b) -- (0, \a);
    }
    \foreach \x in {0,...,5} {
        \draw (\x,0) -- (\x,5);
    }
    \foreach \y in {0,...,5} {
        \draw (0,\y) -- (5,\y);
    }
    \foreach \x / \n in {0.5/1, 1.5/3, 2.5/5, 3.5/7}
        \node [fill=white] at (\x, \x) {$\n$};

    \node [below] at (2.5, 0) {$n$};
    \node [right] at (5, 2.5) {$n$};
\end{tikzpicture}

    \caption{}\label{fig:2-7}
\end{wrapfigure}


本例所证明的等式可以用图 \ref{fig:2-7} 表示出来。

用数学归纳法证明命题的这两个步骤,是缺一不可的。从上面计算数列 $\{a_n\}$(其中 $a_n = (n^2 - 5n + 5)^2$)
各项的值可以看到,只完步骤(1) 而缺少步骤(2),就可能得出不正确的结论,因为单靠步骤(1),我们无法递推下去,
所以,对于取 $2$,$3$,$4$,$5$,$\cdots$ 时命题是否正确,我们无法判定。同样,只有步骤(2)而缺少步骤(1),
也可能得出不正确结论。例如,假设 $n = k$ 时,等式
$$ 2 + 4 + 6 + \cdots + 2n = n^2 + n + 1 $$
成立,就是
$$ 2 + 4 + 6 + \cdots + 2k = k^2 + k + 1 \text{,} $$
那么,
\begin{align*}
      & 2 + 4 + 6 + \cdots + 2k + 2(k + 1) \\
    = & k^2 + k + 1 + 2(k + 1) \\
    = & (k + 1)^2 + (k + 1) + 1 \text{。}
\end{align*}

这就是说,如果 $n = k$ 时等式成立,那么 $n = k + 1$ 时等式也成立。但如果仅根据这一步就得出等式对于任何 $n \in N$
都成立的结论,那就错了。事实上,当 $n = 1$ 时,上式左边 $= 2$,右边 $= 1^2 + 1 + 1 = 3$, 左边 $\neq$ 右边。
这也说明,如果缺少步骤(1)这个基础, 步骤(2)就没有意义了。

\lianxi

用数学归纳法证明:

\begin{xiaotis}

\xiaoti{$1 + 2 + 3 + \cdots + n = \dfrac{1}{2}n(n + 1)$。}

\xiaoti{$1 + 2 + 2^2 + \cdots + 2^{n -1} = 2^n - 1$。}

\xiaoti{首项是 $a_1$,公比是 $q$ 的等比数列的通项公式是
    $$ a_n = a_1 q^{n - 1} \text{。} $$
}

\end{xiaotis}

