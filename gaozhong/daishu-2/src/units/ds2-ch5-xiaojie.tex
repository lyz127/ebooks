\xiaojie

一、本章主要内容是复数的概念,复数的代数、几何、三角表示方法及复数的代数运算法则。

二、要注意实数、虚数、纯虚数、复数之问的区别与联系。
复数 $a + b\,i$ 当 $b = 0$ 时为实数,当 $b \neq 0$ 时为虚数,当 $b \neq 0$ 且 $a = 0$ 时为纯虚数。
实数集与虚数集的交集是空集,它们都是复数集的真子集,它们的并集就是复数集;
纯虚数集是虚数集的真子集,它可以与非零实数所组成的集合一一对应。
这些集合之间的关系可以用下图表示。

\begin{figure}[htbp]
    \centering
    \begin{tikzpicture}[>=Stealth, scale=0.8]
    \draw [thick] (0, 0) rectangle (10, 4);
    \draw [thick] (7, 0) -- (7, 4);

    \node at (2, 1) {\Large{虚数集}};

    \draw [thick] (4, 0.5) rectangle (6, 2.3);
    \node at (5, 1.8) {\Large{纯虚}};
    \node at (5, 1) {\Large{数集}};

    \node [fill=white, inner sep=0] at (7, 3.3) {\huge{复数集}};

    \node at (8.5, 1) {\Large{实数集}};
\end{tikzpicture}


\end{figure}


复数的分类表如下:
\vspace{4em}% 被忽略的高度
$$
\text{复数}
\begin{cases}
    \text{实数} \smash[t]{\left\{
    \begin{aligned}
        &\begin{aligned}
            \text{有理数} \\
            \text{(分数)}
        \end{aligned}
        \smash{\left\{
            \begin{aligned}
                &\text{正有理数} \\
                &\text{零} \\
                &\text{负有理数}
            \end{aligned}
        \right\}}
        \begin{aligned}
            &\text{循环小数(包括} \\
            &\text{整数、有限小数)}
        \end{aligned}
        \\
        \, \\
        &\text{无理数}
        \smash[b]{\left\{
            \begin{aligned}
                \text{正无理数} \\
                \text{负无理数}
            \end{aligned}
        \right\}}
        \text{无限不循环小数}
    \end{aligned}
    \right\}} \text{小数} \\

    \, \\

    \text{虚数}
\end{cases}
$$


三、任一复数 $z = a + b\,i$ 和复平面内的一点 $Z(a, b)$ 对应,也可以和以原点为起点、
点 $Z(a, b)$ 为终点的向量 $\overrightarrow{OZ}$ 对应。这些对应都是一一对应,即
\begin{figure}[htbp]
    \centering
    \begin{tikzpicture}[>=Stealth, scale=0.8]
    \node at (0, 0) {点 $Z(a, b)$};
    \node at (6, 0) {向量 $\overrightarrow{OZ}$};
    \node at (3, 3) {复数 $Z = a + b\,i$};

    \draw [thick, <->] (1.2, 0) -- (5, 0);
    \node at (3, 0.5) {一一对应};

    \draw [thick, <->] (0.5, 0.4) -- (2.5, 2.5);
    \node [rotate=45] at (1, 1.5) {一一对应};

    \draw [thick, <->] (3.5, 2.5) -- (5.5, 0.4);
    \node [rotate=-45] at (5, 1.5) {一一对应};
\end{tikzpicture}

\end{figure}

\hspace{-2em}在这些一一对应下, 复数的各种运算,都有特定的几何意义。

四、实数集 $R$ 中的加、乘运算律, 在复数集 $C$ 中仍然成立。同实数加、减、乘、除、乘方的结果仍是实数一样,
复数加、减、乘、除、乘方的结果仍是复数。除此以外,复数开 $n \; (n \in N)$ 次方的结果是 $n$ 个复数,这却
是实数集 $R$ 所没有的性质(在实数集 $R$ 中,负数不能开偶次方,或者说,负数没有偶次方根)。

五、复数 $z$ 的三角形式是 $z = r(\cos\theta + i\,\sin\theta)$。把复数表示成三角形式,可以给复数的乘、
除、乘方及开方运算带来很大方便。至于复数的加、减运算,还是用代数形式 $z = a + b\,i$ 来进行比较方便。



