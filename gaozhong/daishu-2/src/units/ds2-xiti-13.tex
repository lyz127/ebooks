\xiti\mylabel{xiti-13}

\begin{xiaotis}

\xiaoti{计算:}
\begin{xiaoxiaotis}

    \xiaoxiaoti{$\left( \dfrac{2}{3} + i \right) + \left( 1 - \dfrac{2}{3}\,i \right) - \left( \dfrac{1}{2} + \dfrac{3}{4}\,i \right)$;}

    \xiaoxiaoti{$(-\sqrt{2} + \sqrt{3}\,i) - [(\sqrt{3} - \sqrt{2}) + (\sqrt{3} + \sqrt{2})\,i] + (-\sqrt{2}\,i + \sqrt{3})$;}

    \xiaoxiaoti{$[(a + b) + (a - b)\,i] - [(a - b) - (a + b)\,i]$。}

\end{xiaoxiaotis}


\xiaoti{复数 $6 + 5\,i$ 与 $-3 + 4\,i$ 分别表示向量 $\overrightarrow{OA}$ 与 $\overrightarrow{OB}$,求表示向量 $\overrightarrow{BA}$ 与 $\overrightarrow{AB}$ 的复数。}


\xiaoti{求复平面内和下列各题中两个复数对应的两点之间的距离:}
\begin{xiaoxiaotis}

    \twoInLineXxt[16em]{$2 + i$,$3 - i$;}{$8 + 5\,i$,$4 - 2\,i$。}

\end{xiaoxiaotis}


\xiaoti{求证一个复数与它的共轭复数的和,等于这个复数的实部的 $2$ 倍。用图把这一结论表示出来。}


\xiaoti{已知 $z = a + b\,i \; (a,\,b \in R)$,$|z - \bar{z}|$ 等于什么?用图把结论表示出来。}


\xiaoti{设 $z_1$,$z_2$ 是不等于零的复数,用几何方法证明}
$$ \big| \, |z_1| - |z_2| \, \big| \leqslant |z_1 \pm z_2| \leqslant |z_1| + |z_2| \text{。} $$

\xiaoti{求证 $|z_1 + z_2|^2 + |z_1 - z_2|^2 = 2|z_1|^2 + 2|z_2|^2$ 。}


\xiaoti{设 $Z_1$,$Z_2$ 是复平面内两点,写出线段 $Z_1Z_2$ 的垂直平分线的方程。}


\xiaoti{已知复平面内一椭圆的两个焦点的坐标为 $(-\sqrt{5}, 0)$,$(\sqrt{5}, 0)$,椭圆上的点到两焦点的距离之和为 $6$,写出这个椭圆的方程。}


\xiaoti{计算:}
\begin{xiaoxiaotis}

    \xiaoxiaoti{$(-0.2 + 0.3\,i) (0.5 - 0.4\,i)$;}

    \xiaoxiaoti{$(1 - 2\,i) (2 + i) (3 - 4\,i)$;}

    \xiaoxiaoti{$(\sqrt{a} + \sqrt{b}\,i) (\sqrt{a} - \sqrt{b}\,i) \quad (a,\, b \in R^+)$;}

    \xiaoxiaoti{$(a + b\,i) (a - b\,i) (-a + b\,i) (-a - b\,i)$。}

\end{xiaoxiaotis}


\xiaoti{利用公式 $a^2 + b^2 = (a + b\,i) (a - b\,i)$,把下列各式分解成一次因式的积:}
\begin{xiaoxiaotis}

    \renewcommand\arraystretch{1.2}
    \begin{tabular}[t]{*{2}{@{}p{16em}}}
        \xiaoxiaoti{$x^2 + 4$;} & \xiaoxiaoti{$a^4 - b^4$;} \\
        \xiaoxiaoti{$a^2 + 2ab + b^2 + c^2$;} & \xiaoxiaoti{$x^2 + 2x + 3$。}
    \end{tabular}

\end{xiaoxiaotis}


\xiaoti{计算:}
\begin{xiaoxiaotis}

    \xiaoxiaoti{$(1 - i) + (2 - i^3) + (3 - i^5) + (4 - i^7)$;}

    \twoInLineXxt[16em]{$\left( \dfrac{\sqrt{2}}{2} - \dfrac{\sqrt{2}}{2}\,i \right)^2$;}{$(a + b\,i)^2$。}

\end{xiaoxiaotis}


\xiaoti{设 $\omega = -\dfrac{1}{2} + \dfrac{\sqrt{3}}{2}\,i$,求证}
\begin{xiaoxiaotis}

    \twoInLineXxt[16em]{$1 + \omega + \omega^2 = 0$;}{$\omega^3 = 1$。}

\end{xiaoxiaotis}


\xiaoti{计算:}
\begin{xiaoxiaotis}

    \renewcommand\arraystretch{1.8}
    \begin{tabular}[t]{*{2}{@{}p{16em}}}
        \xiaoxiaoti{$\dfrac{1}{11 - 5\,i}$;} & \xiaoxiaoti{$\dfrac{7 - 9\,i}{1 + i}$;} \\
        \xiaoxiaoti{$\dfrac{1 - 2\,i}{3 + 4\,i}$;} & \xiaoxiaoti{$\dfrac{1 + 2\,i}{2 - 4\,i^3}$;} \\
        \xiaoxiaoti{$\dfrac{(1 - 2\,i)^2}{3 - 4\,i} - \dfrac{(2 + i)^2}{4 - 3\,i}$;} & \xiaoxiaoti{$\dfrac{\sqrt{5} + \sqrt{3}\,i}{\sqrt{5} - \sqrt{3}\,i} - \dfrac{\sqrt{3} + \sqrt{5}\,i}{\sqrt{3} - \sqrt{5}\,i}$。}
    \end{tabular}

\end{xiaoxiaotis}


\xiaoti{设 $z_1,\, z_2 \in C$,求证:}
\begin{xiaoxiaotis}

    \renewcommand\arraystretch{1.2}
    \begin{tabular}[t]{*{2}{@{}p{16em}}}
        \xiaoxiaoti{$\overline{z_1 + z_2} = \overline{z_1} + \overline{z_2}$;} & \xiaoxiaoti{$\overline{z_1 - z_2} = \overline{z_1} - \overline{z_2}$;} \\
        \xiaoxiaoti{$\overline{z_1 \cdot z_2} = \overline{z_1} \cdot \overline{z_2}$;} & \xiaoxiaoti{$\overline{\left( \dfrac{z_1}{z_2} \right)} = \dfrac{\overline{z_1}}{\overline{z_2}} \quad (z_2 \neq 0)$。}
    \end{tabular}

\end{xiaoxiaotis}


\xiaoti{已知 $z_1,\, z_2 \in C$,$z_1z_2 = 0$,求证 $z_1$,$z_2$ 中至少有一个是 $0$。}


\xiaoti{已知 $z_1 = 5 + 10\,i$,$z_2 = 3 - 4\,i$,$\dfrac{1}{z} = \dfrac{1}{z_1} + \dfrac{1}{z_2}$,求 $z$ 。}


\xiaoti{设 $z = x + y\,i \; (x,\, y \in R)$ 的平方等于 $5 - 12\,i$,求 $z$。}

\xiaoti{设 $f(z) = \dfrac{z^2 - z + 1}{z^2 + z + 1}$,求:}
\begin{xiaoxiaotis}

    \twoInLineXxt[16em]{$f(2 + 3\,i)$;}{$f(1 - i)$。}

\end{xiaoxiaotis}


\xiaoti{规定 $i^0$ 的意义是 $1$,$i^{-m}$ 的意义是 $\dfrac{1}{i^m} \; (m \in N)$,求证
    $$ i^{4n} = 1,\quad i^{4n+1} = i,\quad i^{4n+2} = -1,\quad i^{4n+3} = -i $$
    对一切 $n \in Z$ 都能成立。
}

\end{xiaotis}

