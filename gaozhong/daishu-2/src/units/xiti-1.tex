\xiti

\begin{xiaotis}

\xiaoti{求下列反正弦函数的值:}
\begin{xiaoxiaotis}

    \renewcommand\arraystretch{1.5}
    \begin{tabular}[t]{*{2}{@{}p{18em}}}
        \xiaoxiaoti{$\arcsin 0$;} & \xiaoxiaoti{$\arcsin 0.7841$;} \\
        \xiaoxiaoti{$\arcsin\left( -\dfrac{1}{4} \right)$; } & \xiaoxiaoti{$\arcsin\dfrac{1 + \sqrt{5}}{4}$。}
    \end{tabular}

\end{xiaoxiaotis}

\xiaoti{用反正弦的形式把下列各式中的 $x$ 表示出来:}
\begin{xiaoxiaotis}

    \renewcommand\arraystretch{2.0}
    \begin{tabular}[t]{*{2}{@{}p{18em}}}
        \xiaoxiaoti{$\sin x = \dfrac{\sqrt{3}}{5} \; \left(0 < \pi < \dfrac{\pi}{2} \right)$;} & \xiaoxiaoti{$\sin x = -\dfrac{1}{4} \; \left( -\dfrac{\pi}{2} < x < \dfrac{\pi}{2} \right)$;} \\
        \xiaoxiaoti{$\sin x = \dfrac{\sqrt{3}}{5} \; \left( \dfrac{\pi}{2} < x < \pi \right)$;} & \xiaoxiaoti{$\sin x = -\dfrac{1}{4} \; \left( \pi < x < \dfrac{3\pi}{2} \right)$。}
    \end{tabular}

\end{xiaoxiaotis}

\xiaoti{求下列各式的值:}
\begin{xiaoxiaotis}

    \renewcommand\arraystretch{2.0}
    \begin{tabular}[t]{*{2}{@{}p{18em}}}
        \xiaoxiaoti{$\sin\left[ \arcsin\left( -\dfrac{4}{7} \right)\right]$;} & \xiaoxiaoti{$\cos\left( \arcsin\dfrac{\sqrt{5}}{4} \right)$;} \\
        \xiaoxiaoti{$\tan(\arcsin 0.8)$;} & \xiaoxiaoti{$\sin\left( 2\arcsin \dfrac{1}{6} \right)$;} \\
        \xiaoxiaoti{$\cos(2\arcsin 0.5)$;} & \xiaoxiaoti{$\sin\left[ \dfrac{\pi}{3} + \arcsin\left( -\dfrac{\sqrt{3}}{2} \right)\right]$;} \\
        \xiaoxiaoti{$\arcsin\left( \sin\dfrac{15\pi}{4} \right)$;} & \xiaoxiaoti{$\arcsin\left( \dfrac{1}{3} + \sin\dfrac{\pi}{6} \right)$。}
    \end{tabular}

\end{xiaoxiaotis}

\xiaoti{求下列函数的定义域、值域:}
\begin{xiaoxiaotis}

    \renewcommand\arraystretch{1.5}
    \begin{tabular}[t]{*{2}{@{}p{18em}}}
        \xiaoxiaoti{$y = \arcsin 3x$;} & \xiaoxiaoti{$y = \dfrac{1}{3}\arcsin(x - 1)$;} \\
        \xiaoxiaoti{$y = \dfrac{3}{5} \arcsin(2 - x)$;} & \xiaoxiaoti{$y = \dfrac{\pi}{2} + \arcsin \dfrac{x}{2}$。}
    \end{tabular}

\end{xiaoxiaotis}

\xiaoti{求下列各式的值:}
\begin{xiaoxiaotis}

    \renewcommand\arraystretch{1.5}
    \begin{tabular}[t]{*{2}{@{}p{18em}}}
        \xiaoxiaoti{$\arccos 1$;} & \xiaoxiaoti{$\arccos\left( -\dfrac{1}{2} \right)$;} \\
        \xiaoxiaoti{$\arccos\left( -\dfrac{\sqrt{3}}{2} \right)$;} & \xiaoxiaoti{$\arccos 0.6943$;} \\
        \xiaoxiaoti{$\arccos(-0.9178)$;} & \xiaoxiaoti{$\arccos\dfrac{\sqrt{5} - 1}{4}$。}
    \end{tabular}

\end{xiaoxiaotis}

\xiaoti{用反余弦的形式把下列各式中的 $x$ 表示出来:}
\begin{xiaoxiaotis}

    \xiaoxiaoti{$\cos x = \dfrac{1}{3} \quad \left( 0 < x < \dfrac{\pi}{2} \right)$;}

    \xiaoxiaoti{$\cos x - \dfrac{3}{7} = 0 \quad \left( -\dfrac{\pi}{2} < x < 0 \right)$。}

\end{xiaoxiaotis}

\xiaoti{求下列各式的值:}
\begin{xiaoxiaotis}

    \renewcommand\arraystretch{1.5}
    \begin{tabular}[t]{*{2}{@{}p{18em}}}
        \xiaoxiaoti{$\cos(\arccos 0.2571)$;} & \xiaoxiaoti{$\cos\left[ \arccos\left( -\dfrac{\sqrt{5}}{13} \right)\right]$;} \\
        \xiaoxiaoti{$\sin\left( 2\arccos \dfrac{2}{3} \right)$;} & \xiaoxiaoti{$\arccos\left[ \cos\left( -\dfrac{\pi}{3} \right) \right]$。}
    \end{tabular}

\end{xiaoxiaotis}

\xiaoti{求下列函数的定义域、值域:}
\begin{xiaoxiaotis}

    %\renewcommand\arraystretch{1.5}
    %\begin{tabular}[t]{*{2}{@{}p{18em}}}
    %    \xiaoxiaoti{$y = \arccos\left( \dfrac{1}{2} - x \right)$;} & \xiaoxiaoti{$y = \dfrac{1}{\arccos x} $。}
    %\end{tabular}

    \twoInLineXxt[18em]{$y = \arccos\left( \dfrac{1}{2} - x \right)$;}{$y = \dfrac{1}{\arccos x} $。}

\end{xiaoxiaotis}

\xiaoti{求下列各式的值:}
\begin{xiaoxiaotis}

    \renewcommand\arraystretch{1.5}
    \begin{tabular}[t]{*{2}{@{}p{18em}}}
        \xiaoxiaoti{$\arctan(-\sqrt{3})$;} & \xiaoxiaoti{$\arctan 2.747$;} \\
        \xiaoxiaoti{$\arccot\left( -\dfrac{1}{4} \right)$;} & \xiaoxiaoti{$\arccot(-7.238)$;}
    \end{tabular}

    \xiaoxiaoti{$\arctan\sqrt{3} + \arctan(\sqrt{2} + 1)$;}

    \xiaoxiaoti{$\arccot(-6.460) + \arctan(-6.460)$。}

\end{xiaoxiaotis}

\xiaoti{求下列各式的值:}
\begin{xiaoxiaotis}

    \renewcommand\arraystretch{1.5}
    \begin{tabular}[t]{*{2}{@{}p{18em}}}
        \xiaoxiaoti{$\cot(\arctan 0.4)$;} & \xiaoxiaoti{$\tan\left( \arctan\dfrac{\sqrt{2}}{2} + \arctan\dfrac{\sqrt{3}}{3} \right)$;} \\
        \xiaoxiaoti{$\tan(2\arccot x)$;} & \xiaoxiaoti{$\cos[\arccot(-5)]$;} \\
        \xiaoxiaoti{$\arctan\left( \tan\dfrac{5\pi}{6} \right)$;} & \xiaoxiaoti{$\arccot\left( \tan\dfrac{\pi}{3} \right)$。}
    \end{tabular}

\end{xiaoxiaotis}

\xiaoti{求下列函数的定义域、值域:}
\begin{xiaoxiaotis}

    \twoInLineXxt[18em]{$y = \arctan\sqrt{x}$;}{$y = \sqrt{\arccot x}$。}

\end{xiaoxiaotis}

\xiaoti{已知等腰三角形的高与底的比为 $4:3$,用反三角函数把它的三个内角表示出来。}

\xiaoti{求证:}
\begin{xiaoxiaotis}

    \xiaoxiaoti{$\cos(\arctan x) = \dfrac{1}{\sqrt{1 + x^2}}$;}

    \xiaoxiaoti{$\sin(\arctan x) = \dfrac{x}{\sqrt{1 + x^2}}$。}

\end{xiaoxiaotis}

\end{xiaotis}
