\subsection{数的概念的发展}\label{subsec:5-1}

数的概念是从实践中产生和发展起来的。早在原始社会末期,由于计数的需要,人们就建立起自然数的溉念。
自然数的全体构成自然数集 $N$。

随着生产和科学的发展,数的概念也得到发展。

为了表示各种具有相反意义的量以及满足记数法的要求,人们引进了零及负数,把自然数看作正整数,
把正整数、零、负整数合并在一起,构成整数集 $Z$。

为了解决测量、分配中遇到的将某些量进行等分的问题,人们又引进了有理数,规定它们就是一切形如 $\dfrac{m}{n}$ 的数,
其中 $m \in Z$,$n \in N$。这样,就把整数集 $Z$ 扩大为有理数集 $Q$ 。显然,$Z \subset Q$。
如果把整数看作分母为 $1$ 的分数,那么有理数集实际上就是分数集。

每一个有理数都可以表示成整数、有限小数或循环节不为 $0$ 的循环小数;
反过来, 整数、有限小数或循环节不为 $0$ 的循环小数也都是有理数。
如果把整数、有限小数都看作循环节为 $0$ 的循环小数,那么有理数集实际上也就是循环小数的集合。

为了解决有些量与量之间的比值(例如用正方形的边长去度量它的对角线所得结果)不能用有理数表示的矛盾,
人们又引进了无理数。所谓无理数,就是无限不循环小数。有理数集与无理数集合并在一起,构成实数集 $R$。
因为有理数都可看作循环小数(包括整数、有限小数),无理数都是无限不循环小数,所以实数集就是小数集。

从解方程来看,方程 $x + 5 = 3$ 在自然数集 $N$ 中无解,在整数集 $Z$ 中就有一个解 $x = -2$;
方程 $3x = 5$ 在整数集 $Z$ 中无解,在有理数集 $Q$ 中就有一个解 $x = \dfrac{5}{3}$;
方程 $x^2 = 2$ 在有理数集 $Q$ 中无解,在实数集 $R$ 中就有两个解 $x = \pm\sqrt{2}$。
但是,数的范围扩充到实数集 $R$ 以后,象 $x^2 = -1$ 这样的方程还是无解,因为没有一个实数的平方等于 $-1$。
在十六世纪,由于解方程的需要,人们开始引进一个新数 $i$,叫做\textbf{虚数单位},并规定:

\begin{enumerate}[(1), nosep, left=\parindent]
    \item \textbf{它的平方等于 $-1$ ,即 $$i^2 = -1 \text{;}$$}
    \item \textbf{实数与它进行四则运算时,原有的加、乘运算律仍然成立。}
\end{enumerate}

在这种规定下,$i$ 可以与实数 $b$ 相乘,再同实数 $a$ 相加,由于满足乘法交换律及加法交换律,
从而可以把结果写成 $a + bi$。这样,数的范围又扩充了,出现了形如 $a + bi \; (a,\; b \in R)$
的数,人们把它们叫做\textbf{复数}。全体复数所成的集合,一般用字母 $C$ 来表示。\footnote{$C$ 是英文词组 Complex numbers (复数)的第一个字母。}

在这种规定下,$i$ 就是 $-1$ 的一个平方根。因此,方程 $x^2 = -1$ 在复数集 $C$ 中就至少有一个解 $x = i$。

十八世纪以后,复数在数学、力学和电学中得到了应用。从此对它的研究日益展开。现在复数已成为科学技术中普遍使用的一种数学工具。

