\subsubsection{乘法与乘方}\label{subsec:5-7-1}

如果把复数 $z_1$,$z_2$ 分别写成三角形式
\begin{gather*}
    z_1 = r_1 (\cos\theta_1 + i\,\sin\theta_1), \\
    z_2 = r_2 (\cos\theta_2 + i\,\sin\theta_2),
\end{gather*}
就有
\begin{align*}
    z_1 \cdot z_2 &= r_1 (\cos\theta_1 + i\,\sin\theta_1) \cdot r_2 (\cos\theta_2 + i\,\sin\theta_2) \\
        &= r_1 r_2[(\cos\theta_1 \cos\theta_2 - \sin\theta_1 \sin\theta_2) + i\,(\sin\theta_1 \cos\theta_2 + \cos\theta_1 \sin\theta_2)] \\
        &= r_1 r_2[\cos(\theta_1 + \theta_2) + i\,\sin(\theta_1 + \theta_2)],
\end{align*}
即
\begin{center}
    \framebox{\begin{minipage}{23em}
        \begin{gather*}
            r_1 (\cos\theta_1 + i\,\sin\theta_1) \cdot r_2 (\cos\theta_2 + i\,\sin\theta_2) \\
            = r_1 r_2[\cos(\theta_1 + \theta_2) + i\,\sin(\theta_1 + \theta_2)]
        \end{gather*}
    \end{minipage}}
\end{center}

这就是说,\textbf{两个复数相乘,积的模等于各复数的模的积,积的辐角等于各复数的辐角的和。}

据此,两个复数 $z_1$,$z_2$ 相乘时,可以先画出分别与 $z_1$,$z_2$ 对应的向量
$\overrightarrow{OP_1}$,$\overrightarrow{OP_2}$, 然后把向量 $\overrightarrow{OP_1}$
按逆时针方向旋转一个角 $\theta_2$ (如果 $\theta_2 < 0$ ,就要把 $\overrightarrow{OP_1}$
按顺针方向旋转一个角 $|\theta_2|$ ), 再把它的模变为原来的 $r_2$ 倍,所得的向量
$\overrightarrow{OP}$, 就表示积 $z_1 \cdot z_2$ ( 图 \ref{fig:5-10} ) 。
这是复数乘法的几何意义。

\begin{figure}[htbp]
    \centering
    \begin{tikzpicture}[>=Stealth, scale=0.8]
    \draw [->] (-1, 0) -- (4.5, 0) node[anchor=west] {$x$};
    \draw [->] (0, -1) -- (0, 6) node[anchor=east] {$y$};
    \node at (-0.3, -0.3) {$O$};

    \coordinate (O) at (0, 0);
    \draw[rotate=25,->,thick] (O) -- (3.8, 0) node[anchor=west] {$P2$};
    \draw [->] (1.0, 0) arc (0:25:1.0);
    \node at (1.4, 0.3) {$\theta_2$};
    \node at (2.9, 1.0) {$r_2$};

    \draw[rotate=40,->,thick] (O) -- (3.8, 0) node[anchor=west] {$P1$};
    \draw [->] (1.8, 0) arc (0:40:1.8);
    \node at (2.1, 0.5) {$\theta_1$};
    \node at (2.7, 1.8) {$r_1$};

    \draw[rotate=65,->,thick] (O) -- (6, 0) node[anchor=west] {$P$}; % 原本长度应该是 3.8*3.8 ,但为了图形更好看,取了一个较小的值 6
    \draw [->] (2.5, 0) arc (0:65:2.5);
    \node at (3.3, 0.3) {$\theta_1 + \theta_2$};
    \node at (2.7, 4.5) {$r_1 r_2$};
\end{tikzpicture}

    \caption{}\label{fig:5-10}
\end{figure}


用数学归纳法容易证明(请同学们自己证明),上面的结论可以推广到 $n$ 个复数相乘的情况,就是:

\begin{align*}
    z_1 \cdot z_2 \cdot \cdots \cdot z_n &= r_1 (\cos\theta_1 + i \, \sin\theta_1) \cdot r_2 (\cos\theta_2 + i \, \sin\theta_2) \cdot \; \cdots \; \cdot r_n (\cos\theta_n + i \, \sin\theta_n) \\
    &= r_1 r_2 \cdots r_n [\cos(\theta_1 + \theta_2 + \cdots + \theta_n) + i \, \sin(\theta_1 + \theta_2 + \cdots + \theta_n) ] \text{。}
\end{align*}

因此,如果
$$ r_1 = r_2 = \cdots = r_n = r , \qquad \theta_1 = \theta_2 = \cdots = \theta_n = \theta $$
时,就有
\begin{center}
    \framebox{\begin{minipage}{26em}
        \begin{gather*}
            [r(\cos\theta + i\, \sin\theta)]^n = r^n (\cos n\theta + i\, \sin n\theta) \quad (n \in N) \text{。}
        \end{gather*}
    \end{minipage}}
\end{center}

这就是说,\textbf{复数的 $n \; (n \in N)$ 次幂的模等于这个复数的模的 $n$ 次幂,它的辐角等于这个复数的辐角的 $n$ 倍。}
这个定理叫做 \textbf{棣莫佛\footnote{棣莫佛(Abraham de Moivre, 1667 —— 1754 年),法国数学家。} 定理。}


\liti 计算
$$ \sqrt{2} \left( \cos\dfrac{\pi}{12} + i\, \sin\dfrac{\pi}{12} \right) \cdot \sqrt{3} \left( \cos\dfrac{\pi}{6} + i\, \sin\dfrac{\pi}{6} \right) \text{。} $$

\jie $\begin{aligned}[t]
        & \sqrt{2} \left( \cos\dfrac{\pi}{12} + i\, \sin\dfrac{\pi}{12} \right) \cdot \sqrt{3} \left( \cos\dfrac{\pi}{6} + i\, \sin\dfrac{\pi}{6} \right) \\
    ={} & \sqrt{2} \cdot \sqrt{3} \left[ \cos\left( \dfrac{\pi}{12} + \dfrac{\pi}{6} \right) + i\, \sin\left( \dfrac{\pi}{12} + \dfrac{\pi}{6} \right) \right] \\
    ={} & \sqrt{2} \cdot \sqrt{3} \left( \cos\dfrac{\pi}{4} + i\, \sin\dfrac{\pi}{4} \right) \\
    ={} & \sqrt{2} \cdot \sqrt{3} \left( \dfrac{\sqrt{2}}{2} + \dfrac{\sqrt{2}}{2} \, i \right) \\
    ={} & \sqrt{3} + \sqrt{3} \, i \text{。}
\end{aligned}$



\liti 计算 $(\sqrt{3} - i)^6$ 。

\jie 因为 $\sqrt{3} - i = 2 \left( \cos\dfrac{11\pi}{6} + i\, \sin\dfrac{11\pi}{6} \right)$,所以

$\begin{aligned}
    (\sqrt{3} - i)^6 &= \left[ 2 \left( \cos\dfrac{11\pi}{6} + i\, \sin\dfrac{11\pi}{6} \right)\right]^6 \\
    &= 2^6 (\cos 11\pi + i\, \sin 11\pi) \\
    &= 64 (\cos\pi + i\, \sin\pi) \\
    &= 64 \cdot (-1) = -64 \text{。}
\end{aligned}$


\begin{wrapfigure}[22]{r}{4cm}
    \centering
    \begin{tikzpicture}[>=Stealth, scale=0.8]
    \draw [->] (-2.5, 0) -- (2.5, 0) node[anchor=west] {$x$};
    \draw [->] (0, -2.5) -- (0, 2.5) node[anchor=east] {$y$};
    \node at (0.3, -0.3) {$O$};

    \draw[->, thick] (0, 0) -- (-1.5, 1.5) node[anchor=east] {$Z$};
    \draw [dashed] (-1.5, 0) -- (-1.5, 1.5) -- (0, 1.5);
    \node at (-1.7, -0.3) {$-1$};
    \node at (0.3, 1.5) {$1$};

    \draw[rotate=120, ->, thick] (0, 0) -- (-1.5, 1.5) node[anchor=east] {$Z'$};
    \draw [rotate=135,->] (1, 0) arc (0:120:1);
    \node [rotate=90, fill=white, inner sep=0] at (-1, 0) {$120^\circ$};
\end{tikzpicture}

    \caption{}\label{fig:5-11}
\end{wrapfigure}

\liti 如图 \ref{fig:5-11},向量 $\overrightarrow{OZ}$ 与复数 $-1 + i$ 对应,把 $\overrightarrow{OZ}$
按逆时针方向旋转 $120^\circ$,得到 $\overrightarrow{OZ'}$。求与向量 $\overrightarrow{OZ'}$
对应的复数(用代数形式表示)。

\jie 所求的复数就是 $-1 + i$ 乘以一个复数 $z_0$ 的积,这个复数 $z_0$ 的模是 $1$,
辐角的主值是 $120^\circ$ 。 所以所求的复数是

$\begin{aligned}
        & (-1 + i) \cdot 1 (\cos 120^\circ + i\, \sin 120^\circ) \\
    ={} & (-1 + i)\left( -\dfrac{1}{2} + \dfrac{\sqrt{3}}{2} \, i \right) \\
    ={} & \dfrac{1 - \sqrt{3}}{2} - \dfrac{1 + \sqrt{3}}{2} \, i \, \text{。}
\end{aligned}$


\liti 如图 \ref{fig:5-12},已知平面内并列的三个相等的正方形,利用复数证明
$$ \angle 1 + \angle 2 + \angle 3 = \dfrac{\pi}{2} \text{。} $$

\begin{figure}[htbp]
    \centering
    \begin{tikzpicture}[>=Stealth, scale=1.5]
    \draw [->] (-0.5, 0) -- (3.8, 0) node[anchor=west] {$x$};
    \draw [->] (0, -0.5) -- (0, 1.8) node[anchor=east] {$y$};
    \node at (-0.2, -0.2) {$O$};

    \draw [thick] (3, 1) -- (0, 1) node [left] {$1$};
    \draw [thick] (1, 1) -- (1, 0) node [below] {$1$};
    \draw [thick] (2, 1) -- (2, 0) node [below] {$2$};
    \draw [thick] (3, 1) -- (3, 0) node [below] {$3$};

    \draw (0, 0) -- (1, 1);
    \draw (0.4, 0) arc (0:45:0.4);
    \draw (0.7,1) arc (180:225:0.3);
    \node at (0.6, 0.9) {\footnotesize$1$};

    \draw (0, 0) -- (2, 1);
    \draw (0.6, 0) arc (0:27:0.6);
    \draw (1.7,1) arc (180:207:0.3);
    \node at (1.6, 0.9) {\footnotesize$2$};

    \draw (0, 0) -- (3, 1);
    \draw (0.8, 0) arc (0:19:0.8);
    \draw (2.5,1) arc (180:199:0.5);
    \node at (2.4, 0.9) {\footnotesize$3$};
\end{tikzpicture}

    \caption{}\label{fig:5-12}
\end{figure}

\zhengming 如图建立坐标系(确定复平面),由于平行线的内错角相等,$\angle 1$,$\angle 2$,$\angle 3$
分别等于复数 $1 + i$,$2 + i$,$3 + i$ 的辐角的主值,这样 $\angle 1 + \angle 2 + \angle 3$
就是积 $(1 + i)(2 + i)(3 + i)$ 的辐角,而
$$ (1 + i)(2 + i)(3 + i) = (1 + 3i)(3 + i) = 10i \text{,}$$
其辐角的主值是 $\dfrac{\pi}{2}$,并且 $\angle 1$,$\angle 2$,$\angle 3$ 都是锐角,于是
$$ 0 < \angle 1 + \angle 2 + \angle 3 < \dfrac{3\pi}{2} \text{,} $$
所以
$$ \angle 1 + \angle 2 + \angle 3 = \dfrac{\pi}{2} \text{。} $$



\lianxi
\begin{xiaotis}

\xiaoti{计算:}
\begin{xiaoxiaotis}

    \xiaoxiaoti{$8\left( \cos\dfrac{\pi}{6} + i\, \sin\dfrac{\pi}{6} \right) \cdot 2\left( \cos\dfrac{\pi}{4} + i\, \sin\dfrac{\pi}{4} \right)$;}

    \xiaoxiaoti{$2\left( \cos\dfrac{4\pi}{3} + i\, \sin\dfrac{4\pi}{3} \right) \cdot 4\left( \cos\dfrac{5\pi}{6} + i\, \sin\dfrac{5\pi}{6} \right)$;}

    \xiaoxiaoti{$\sqrt{2} (\cos 240^\circ + i\, \sin 240^\circ) \cdot \dfrac{\sqrt{3}}{2} (\cos 60^\circ + i\, \sin 60^\circ)$;}

    \xiaoxiaoti{$3(\cos 18^\circ + i\, \sin 18^\circ) \cdot 2(\cos 54^\circ + i\, \sin 54^\circ) \cdot 5(\cos 108^\circ + i\, \sin 108^\circ)$。}

\end{xiaoxiaotis}



\xiaoti{用棣莫佛定理计算:}
\begin{xiaoxiaotis}

    \xiaoxiaoti{$[3(\cos 18^\circ + i\, \sin 18^\circ)]^5$;}

    \xiaoxiaoti{$\left[ \sqrt{2} \left( \cos\dfrac{\pi}{4} + i\, \sin\dfrac{\pi}{4} \right) \right]^6$;}

    \xiaoxiaoti{$(1 - i) \left( -\dfrac{1}{2} + \dfrac{\sqrt{3}}{2}\, i \right)^7$;}

    \xiaoxiaoti{$(-1 - i)^6$。}

\end{xiaoxiaotis}


\xiaoti{直角三角形 $ABC$ 中, $\angle C = \dfrac{\pi}{2}$, $BC = \dfrac{1}{3} AC$,
    点 $E$ 在 $AC$ 上, 且 $EC = 2AE$。利用复数证明
}
$$\angle CBE + \angle CBA = \dfrac{3\pi}{4} \text{。} $$

\end{xiaotis}


