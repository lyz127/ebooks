\subsection{三元线性方程组}\label{subsec:4-5}

一个三元线性方程组,当其中方程的个数与未知数的个数相同时,它的一般形式是

\fangchengzu{eq:fcz-2}(\thefangchengzu)
\begin{minipage}[c]{0.90\textwidth}
    \begin{numcases}{}
        a_1 x + b_1 y + c_1 z = d_1 \text{,} \label{eq:syxxfcz-1} \\
        a_2 x + b_2 y + c_2 z = d_2 \text{,} \label{eq:syxxfcz-2} \\
        a_3 x + b_3 y + c_3 z = d_3 \text{。} \label{eq:syxxfcz-3}
    \end{numcases}
\end{minipage}

如果当 $x = x_1$,$y = y_1$,$z = z_1$ 时,方程组 \eqref{eq:fcz-2} 中的每个方程
左右两边的值相等,那么 $x = x_1$,$y = y_1$,$z = z_1$ 叫做
\textbf{方程组 \eqref{eq:fcz-2} 的一个解},简记为 $(x_1, y_1, z_1)$。
方程组 \eqref{eq:fcz-2} 的所有的解构成的集合叫做
\textbf{方程组 \eqref{eq:fcz-2} 的解集}。
\footnote{对一般 $n$ 元线性方程组的解与解集,也可作相应定义。}

我们现在利用第 \ref{subsec:4-4} 节的两个定理来导出方程组 \eqref{eq:fcz-2} 的解。

用 $D$ 表示方程组 \eqref{eq:fcz-2} 的系数行列式,即
$$
D =
\begin{vmatrix*}
	a_1 & b_1 & c_1 \\
	a_2 & b_2 & c_2 \\
	a_3 & b_3 & c_3
\end{vmatrix*} \text{。}
$$
用元素 $a_1$,$a_2$,$a_3$ 对应的代数余子式 $A_1$,$A_2$,$A_3$ 分别乘方程
\eqref{eq:syxxfcz-1},\eqref{eq:syxxfcz-2},\eqref{eq:syxxfcz-3} 的两边,得
\begin{gather*}
    a_1A_1x + b_1A_1y + c_1A_1z = d_1A_1 \text{,} \\
    a_2A_2x + b_2A_2y + c_2A_2z = d_2A_2 \text{,} \\
    a_3A_3x + b_3A_3y + c_3A_3z = d_3A_3 \text{。}
\end{gather*}
把上面三式的等号两边分别相加,得
\begin{align}
    &(a_1A_1 + a_2A_2 + a_3A_3)x
        + (b_1A_1 + b_2A_2 + b_3A_3)y
        + (c_1A_1 + c_2A_2 + c_3A_3)z \notag \\
    &= d_1A_1 + d_2A_2 + d_3A_3 \text{。} \label{eq:syxxfcz-4}
\end{align}
根据第 \ref{subsec:4-4} 节的 \nameref{theorem:sjhlszk-1} 和 \nameref{theorem:sjhlszk-2},
\eqref{eq:syxxfcz-4} 中 $x$ 的系数是 $D$,而 $y$,$z$ 的系数都是零,
所以 \eqref{eq:syxxfcz-4} 式成为
\begin{equation}
    D \cdot x = d_1A_1 + d_2A_2 + d_3A_3 \text{。} \label{eq:syxxfcz-5}
\end{equation}

用类似的方法,从方程组 \eqref{eq:fcz-2} 中消去 $x$,$z$,或者 $x$,$y$,分别得到
\begin{align}
    D \cdot y &= d_1B_1 + d_2B_2 + d_3B_3 \text{,} \label{eq:syxxfcz-6} \\
    D \cdot z &= d_1C_1 + d_2C_2 + d_3C_3 \text{。} \label{eq:syxxfcz-7}
\end{align}
令
\begin{align*}
    D_x &= d_1A_1 + d_2A_2 + d_3A_3
        =   \begin{vmatrix*}
                d_1 & b_1 & c_1 \\
                d_2 & b_2 & c_2 \\
                d_3 & b_3 & c_3
            \end{vmatrix*} \text{,} \\
    D_y &= d_1B_1 + d_2B_2 + d_3B_3
        =   \begin{vmatrix*}
                a_1 & d_1 & c_1 \\
                a_2 & d_2 & c_2 \\
                a_3 & d_3 & c_3
            \end{vmatrix*} \text{,} \\
    D_z &= d_1C_1 + d_2C_2 + d_3C_3
        =   \begin{vmatrix*}
                a_1 & b_1 & d_1 \\
                a_2 & b_2 & d_2 \\
                a_3 & b_3 & d_3
            \end{vmatrix*} \text{,}
\end{align*}
$D_x$,$D_y$,$D_z$ 是把 $D$ 中第一、二、三列分别换成方程组 \eqref{eq:fcz-2} 的常数
项列而得出的。这时 \eqref{eq:syxxfcz-5},\eqref{eq:syxxfcz-6},\eqref{eq:syxxfcz-7}
式就可写成
\begin{numcases}{}
    D \cdot x = D_x \text{,} \label{eq:syxxfcz-8} \\
    D \cdot y = D_y \text{,} \label{eq:syxxfcz-9} \\
    D \cdot z = D_z \text{。} \label{eq:syxxfcz-10}
\end{numcases}

从上面的推导过程可知,如果方程组 \eqref{eq:fcz-2} 有解,这个解一定适合方程
\eqref{eq:syxxfcz-8},\eqref{eq:syxxfcz-9},\eqref{eq:syxxfcz-10}。
当 $D \neq 0$ 时,方程 \eqref{eq:syxxfcz-8},\eqref{eq:syxxfcz-9},\eqref{eq:syxxfcz-10}
组成的方程组的唯一解是
\begin{equation}
    \begin{cases}
        x = \dfrac{D_x}{D} \text{,} \\[1em]
        y = \dfrac{D_y}{D} \text{,} \\[1em]
        z = \dfrac{D_z}{D} \text{。}
    \end{cases} \label{eq:syxxfcz-11}
\end{equation}
因此,当系数行列式 $D \neq 0$ 时,方程组 \eqref{eq:fcz-2} 如果有解,
解只能有一个,并且可以写成 \eqref{eq:syxxfcz-11} 式的形式。

现在来验证 \eqref{eq:syxxfcz-11} 式确是方程组 \eqref{eq:fcz-2} 的解。
把 \eqref{eq:syxxfcz-11} 代入方程 \eqref{eq:syxxfcz-1} 的左边,我们有
\begin{align*}
    \text{左边} &= a_1\dfrac{D_x}{D} + b_1\dfrac{D_y}{D} + c_1\dfrac{D_z}{D} \\
        &= \dfrac{a_1}{D}(d_1A_1 + d_2A_2 + d_3A_3)
            + \dfrac{b_1}{D}(d_1B_1 + d_2B_2 + d_3B_3)
            + \dfrac{c_1}{D}(d_1C_1 + d_2C_2 + d_3C_3) \\
        &= \dfrac{1}{D}[(a_1A_1 + b_1B_1 + c_1C_1)d_1
                        + (a_1A_2 + b_1B_2 + c_1C_2)d_2
                        + (a_1A_3 + b_1B_3 + c_1C_3)d_3] \\
        &= \dfrac{1}{D}[D \cdot d_1 + 0 \cdot d_2 + 0 \cdot d_3] \\
        &= d_1 = \text{右边。}
\end{align*}
即 \eqref{eq:syxxfcz-11} 式适合方程 \eqref{eq:syxxfcz-1}。
同样可以验证 \eqref{eq:syxxfcz-11} 分别适合方程 \eqref{eq:syxxfcz-2} 和方程 \eqref{eq:syxxfcz-3}。
因此,\eqref{eq:syxxfcz-11} 式是方程组 \eqref{eq:fcz-2} 的解。

综上所述,可得以下结论:

\textbf{三元线性方程组 \eqref{eq:fcz-2},当它的系数行列式 $D$ 不等于零时,
有唯一解 $\left( \dfrac{D_x}{D},\; \dfrac{D_y}{D},\; \dfrac{D_z}{D} \right)$,
其中 $D_x$,$D_y$,$D_z$ 是把系数行列式 $D$ 中第一、二、三列分别换成
方程组 \eqref{eq:fcz-2} 的常数项列而得出的三个三阶行列式。}

我们已经知道, 对二元线性方程组 \eqref{eq:fcz-1} 已有类似的结论。
事实上,对 $n$ 元线性方程组都有类似的结论。
这一结论称为\textbf{克莱姆法则}\mylabel{klmfz}[克莱姆法则]\footnote{克莱姆 (Gabriel Cramer, 1704 — 1752 年),瑞士数学家。}
上面只是对 $n = 3$ 的情况进行了证明。

当方程组 \eqref{eq:fcz-2} 的系数行列式 $D = 0$ 时,方程组 \eqref{eq:fcz-2}
或者无解,或者有无穷多解(证明从略)。例如方程组
$$
\begin{cases}
    x + y + z = 1, \\
    x + y + 2z = 2, \\
    2x + 2y + 3z = 5,
\end{cases} \qquad
\begin{cases}
    x + y + z = 1, \\
    x + y + z = 2, \\
    x + y + z = 3
\end{cases}
$$
都没有解,而方程组
$$
\begin{cases}
    x + y + z = 1, \\
    x + 2y + 2z = 1, \\
    y + z = 0,
\end{cases} \qquad
\begin{cases}
    x + y + z = 1, \\
    2x + 2y + 2z = 2, \\
    4x + 4y + 4z = 4
\end{cases}
$$
都有无穷多解。

\liti 判断下列方程组是否有唯一解;如果有唯一解,根据克莱姆法则把解求出来。
$$
(1) \; \begin{cases}
    2x + 3y - 5z = 3, \\
    x - 2y + z = 0, \\
    3x + y + 3z = 7,
\end{cases} \qquad
(2) \; \begin{cases}
    x - 3y + z = 1, \\
    2x + y - z = 0, \\
    4x - 5y + z = 2 \text{。}
\end{cases}
$$

\jie

$
    (1)\; D = \begin{vmatrix*}[r]
        2 & 3  & -5 \\
        1 & -2 & 1 \\
        3 & 1  & 3
    \end{vmatrix*} = \begin{vmatrix*}[r]
        2 & 7 & -7 \\
        1 & 0 & 0 \\
        3 & 7 & 0
    \end{vmatrix*} = - \begin{vmatrix*}[r]
        7 & -7 \\
        7 & 0
    \end{vmatrix*} = -49 \neq 0 \text{,}
$

所以方程组有唯一解。由
\begin{align*}
    D_x &= \begin{vmatrix*}[r]
            3 & 3  & -5 \\
            0 & -2 & 1 \\
            7 & 1  & 3
        \end{vmatrix*} = \begin{vmatrix*}[r]
            3 & -7 & -5 \\
            0 & 0  & 1 \\
            7 & 7  & 3
        \end{vmatrix*} = - \begin{vmatrix*}[r]
            3 & -7 \\
            7 & 7
        \end{vmatrix*} = -70 \text{,} \\
    D_y &= \begin{vmatrix*}[r]
            2 & 3 & -5 \\
            1 & 0 & 1 \\
            3 & 7 & 3
        \end{vmatrix*} = \begin{vmatrix*}[r]
            7 & 3 & -5 \\
            0 & 0 & 1 \\
            0 & 7 & 3
        \end{vmatrix*} = 7 \begin{vmatrix*}[r]
            0 & 1 \\
            7 & 3
        \end{vmatrix*} = -49 \text{,} \\
    D_z &= \begin{vmatrix*}[r]
            2 & 3  & 3 \\
            1 & -2 & 0 \\
            3 & 1  & 7
        \end{vmatrix*} = \begin{vmatrix*}[r]
            2 & 7 & 3 \\
            1 & 0 & 0 \\
            3 & 7 & 7
        \end{vmatrix*} = - \begin{vmatrix*}[r]
            7 & 3 \\
            7 & 7
        \end{vmatrix*} = -28 \text{,}
\end{align*}
得
$$
    \dfrac{D_x}{D} = \dfrac{-70}{-49} = \dfrac{10}{7},\quad
    \dfrac{D_y}{D} = \dfrac{-49}{-49} = 1,\quad
    \dfrac{D_z}{D} = \dfrac{-28}{-49} = \dfrac{4}{7} \text{。}
$$

方程组的解集是 $\left\{\, \left( \dfrac{10}{7},\; 1,\; \dfrac{4}{7} \right) \,\right\}$。

$
    (2)\; D = \begin{vmatrix*}[r]
        1 & -3 & 1 \\
        2 & 1  & -1 \\
        4 & -5 & 1
    \end{vmatrix*} = \begin{vmatrix*}[r]
        1 & -3 & 1 \\
        3 & 2  & 0 \\
        3 & -2 & 0
    \end{vmatrix*} = 0 \text{,}
$
方程组或者无解,或者有无穷多解。因此,方程组不可能是有唯一解。


\lianxi

判断下列方程组是否有唯一解;如果有唯一解,根据克莱姆法则把解求出来。

\begin{xiaoxiaotis}

    \twoInLineXxt[16em]{
        $\begin{cases}
            x - 2y + z = 0, \\
            3x + y - 2z = 0, \\
            7x + 6y + 7z = 100;
        \end{cases}$
    }{
        $\begin{cases}
            3x - 2y + 3z = 11, \\
            4x - 3y + 2z = 9, \\
            5x - 4y + z = 7;
        \end{cases} $
    }

    \twoInLineXxt[16em]{
        $\begin{cases}
            2x + 3y + 4z = 2, \\
            3x + 5y + 7z = -3, \\
            x + 2y + 3z = 4;
        \end{cases}$
    }{
        $\begin{cases}
            x - 3y + z = 6, \\
            2x + y + 2z = -2, \\
            4x - 5y + 6z = 10 \text{。}
        \end{cases} $
    }

\end{xiaoxiaotis}

