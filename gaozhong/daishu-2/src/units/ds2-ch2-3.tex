\subsection{等比数列}\label{subsec:2-3}

看下面的数列:
$$ 1,\; 2,\; 4,\; 8,\; \cdots \text{。}$$
这个数列有这样的特点: 从第 $2$ 项起, 每一项与它前一项的比都等于常数 $2$ 。

一般地,如果一个数列从第 $2$ 项起,每一项与它前一项的比等于同一个常数,
这个数列就叫做\textbf{等比数列}, 这个常数叫做等比数列的\textbf{公比},
公比通常用字母 $q$ 表示。例如,数列
$$ 5,\quad 25,\quad 125,\quad 625,\quad \cdots $$
与
$$ 1,\quad -\dfrac{1}{2},\quad \dfrac{1}{4},\quad -\dfrac{1}{8},\quad \cdots $$
都是等比数列,它们的公比分别是 $5$ 与 $-\dfrac{1}{2}$。

因为在一个等比数列里,从第 $2$ 项起,每一项与它的前一项的比都等于公比,
所以每一项都等于它的前一项乘以公比。这就是说,如果等比数列
$a_1$,$a_2$,$a_3$,$a_4$,$\cdots$ 的公比是 $q$,那么

$\qquad \begin{aligned}
    a_2 &= a_1 q, \\
    a_3 &= a_2 q = (a_1 q)q = a_1 q^2, \\
    a_4 &= a_3 q = (a_1 q^2)q = a_1 q^3, \\
    &\cdots\cdots \qquad\qquad \cdots\cdots \text{。}
\end{aligned}$ \\
由此可知,等比数列 $\{a_n\}$ 的通项公式是
\begin{center}
    \framebox{\begin{minipage}{12em}
        \begin{gather*}
            a_n = a_1  q^{n - 1} \text{。}
        \end{gather*}
    \end{minipage}}
\end{center}

上面的公式还可以改写成
$$ a_n = \dfrac{a_1}{q} q^n = c q^n \text{,}$$
这里 $c = \dfrac{a_1}{q}$ ,它是一个不为零的常数。当 $q$ 是不等于 $1$ 的正数时,
$y = q^x$ 是一个指数函数,而函数 $y = c q^x$ 是一个不为零的常数与指数函数的积。
因此, 从图上看,表示数列 $\{c q^n\}$ 各项的点都在函数 $y = c q^x$ 的图像上。
例如,当 $a_1 = 1$,$q = 2$ 时,
$$ a_n = \dfrac{1}{2} \cdot 2^n \text{,}$$
表示数列各项的点都在函数 $y = \dfrac{1}{2} \cdot 2^x$ 的图像上(图 \ref{fig:2-6})。

\begin{wrapfigure}[22]{r}{5cm}
    \centering
    \begin{tikzpicture}[>=Stealth]
    \draw [->] (-1,0) -- (4.8,0) node[anchor=north] {$n$};
    \draw [->] (0,-1) -- (0,8.5) node[anchor=east] {$a_n$};
    \node at (-0.3,-0.3) {$O$};
    \foreach \x in {1,...,4} {
        \draw (\x,0.2) -- (\x,0) node[anchor=north] {$\x$};
    }
    \foreach \y in {1,...,8} {
        \draw (0.2,\y) -- (0,\y) node[anchor=east] {$\y$};
    }

    \draw[domain=-0.2:4.1,smooth,samples=50] plot (\x, {0.5 * 2^\x});
\end{tikzpicture}
    \caption{}\label{fig:2-6}
\end{wrapfigure}

\liti 培育水稻新品种,如果第 $1$ 代得到 $120$ 粒种子,并且从第 $1$ 代起,
以后各代的每一粒种子都可以得到下一代的 $120$ 粒种子,到第 $5$ 代大约可以
得到这种新品种的种子多少粒(保留两个有效数字)?

\jie 由于每代的种子数是它的前一代种子数的 $120$ 倍,逐代的种子数组成等比数列,
记为 $\{a_n\}$ ,其中 $a_1 = 120$,$q = 120$,因此,
$$ a_5 = 120 \times 120^{5 - 1} \approx 2.5 \times 10^{10} \text{。} $$

答:到第 $5$ 代大约可以得到种子 $2.5 \times 10^{10}$ 粒。


\liti 一个等比数列的第 $3$ 项与第 $4$ 项分别是 $12$ 与 $18$,求它的第 $1$ 项与第 $2$ 项。

\jie 设这个等比数列的第 $1$ 项是 $a_1$,公比是 $q$,那么
\begin{align*}
    a_1 q^2 &= 12, \tag{$1$} \\
    a_1 q^3 &= 18  \tag{$2$}
\end{align*}
解 (1),(2) 所组成的方程组,得
$$ q = \dfrac{3}{2}, \quad a_1 = \dfrac{16}{3} \text{,} $$
因此,
$$ a_1 q = \dfrac{16}{3} \times \dfrac{3}{2} = 8 \text{。} $$

答: 这个数列的 $1$ 项与第 $2$ 项分别是与 $\dfrac{16}{3}$ 与 $8$。


\liti 某种电讯产品自投放市场以来,经过三次降价,单价由原来的 $174$ 元降到 $58$ 元。
这种电讯产品平均每次降价的百分率大约是多少( 精确到 $1\%$)?

\jie 设平均每次降价的百分率是 $x$,那么每次降价后的单价应是降价前的 $(1 - x)$ 倍。
这样,将原单价与三次降价后的单价依次排列,就组成一个等比数列,记为 $\{a_n\}$ ,其中
$$ a_1 = 174,\quad a_4 = 58,\quad n = 4 \text{。} $$
由等比数列的通项公式,得
$$ 58 = 174 \times (1 - x)^{4 - 1} \text{。}$$
整理后,得
\begin{align*}
    (1 - x)^3 &= \dfrac{1}{3} , \\
    1 - x &= \sqrt[3]{\dfrac{1}{3}} \\
    &= 0.693 \text{,}
\end{align*}
因此,
\begin{align*}
    x &= 1 - 0.693 \\
      &\approx 31\% \text{。}
\end{align*}

答:上述电讯产品平均每次降价的百分率大约是 $31\%$。

如果在 $a$ 与 $b$ 中间插入一个数 $G$,使 $a$,$G$,$b$ 成等比数列,那么 $G$ 叫 $a$ 与 $b$ 的\textbf{等比中项}。

如果 $G$ 是 $a$ 与 $b$ 的等比中项,那么 $\dfrac{G}{a} = \dfrac{b}{G}$,即 $G^2 = ab$,因此,
$$ G = \pm \sqrt{ab} \text{。}$$

容易看出,一个等比数列从第 $2$ 项起, 每一项(有穷等比数列的末项除外)是它的前一项与后一项的等比中项。

\lianxi

\begin{xiaotis}


\xiaoti{已知等比数列 $\{a_n\}$ ,问:}
\begin{xiaoxiaotis}

    \renewcommand\arraystretch{1.5}
    \begin{tabular}[t]{*{2}{@{}p{16em}}}
        \xiaoxiaoti{$a_1$ 能不能是零?} & \xiaoxiaoti{公比 $q$ 能不能是零?}
    \end{tabular}

\end{xiaoxiaotis}

\xiaoti{求下面等比数列的第 $4$ 项与笫 $5$ 项:}
\begin{xiaoxiaotis}

    \renewcommand\arraystretch{1.5}
    \begin{tabular}[t]{*{2}{@{}p{16em}}}
        \xiaoxiaoti{$5$,$-15$,$45$,$\cdots$;} & \xiaoxiaoti{$1.2$,$2.4$,$4.8$,$\cdots$;} \\
        \xiaoxiaoti{$\dfrac{2}{3}$,$\dfrac{1}{2}$,$\dfrac{3}{8}$,$\cdots$;} & \xiaoxiaoti{$\sqrt{2}$,$1$,$\dfrac{\sqrt{2}}{2}$,$\cdots$。}
    \end{tabular}

\end{xiaoxiaotis}


\xiaoti{}
\begin{xiaoxiaotis}

    \vspace{-1.6em} \begin{minipage}{0.9\textwidth}
    \xiaoxiaoti{一个等比数列的第 $9$ 项是 $\dfrac{4}{9}$,公比是 $-\dfrac{1}{3}$,求它的第 $1$ 项;}
    \end{minipage}

    \xiaoxiaoti{一个等比数列的第 $2$ 项是 $10$,第 $3$ 项是 $20$,求它的第 $1$ 项与第 $4$ 项。}

\end{xiaoxiaotis}

\xiaoti{}
\begin{xiaoxiaotis}

    \vspace{-1.6em} \begin{minipage}{0.9\textwidth}
    \xiaoxiaoti{已知等比数列 $\{a_n\}$ 的 $a_2 = 2$,$a_5 = 54$,求 $q$;}
    \end{minipage}

    \xiaoxiaoti{已知等比数列 $\{a_n\}$ 的 $a_1 = 1$,$a_n = 256$,$q = 2$,求 $n$;}

\end{xiaoxiaotis}

\end{xiaotis}

\,

下面我们来求等比数列
$$ a_1,\; a_2,\; a_3,\; \cdots,\; a_n,\;  \cdots $$
前 $n$ 项的和 $S_n$。

根据等比数列 $\{a_n\}$ 的通项公式,等比数列 $\{a_n\}$ 前 $n$ 项的和可以写成:
\begin{gather*}
    S_n = a_1 + a_1 q + \cdots + a_1 q^{n-2} + a_1 q^{n-1} \text{。} \tag{$1$}
\end{gather*}

我们知道,将等比数列的每一项乘以公比,就得到它后面邻的一项。现将 (1) 的两边分别乘以公比 $q$ ,得到
\begin{gather*}
    q S_n = a_1 q + a_1 q^2 + \cdots + a_1 q^{n-1} + a_1 q^n \text{。} \tag{$2$}
\end{gather*}

比较 (1) 和 (2) , 我们看到, (1) 的右边从第 $2$ 项到最后一项, 与 (2) 的右边从第 $1$ 项到倒数第 $2$ 项完全相同。
于是,从 (1) 的两边分别减去 (2) 的两边,可以消去这些相同项,得到
$$ (1 - q)S_n = a_1 - a_1 q^n \text{。} $$

由此得到 $q \neq 1$ 时等比数列 $\{a_n\}$ 的前项的和的公式
\begin{center}
    \framebox{\begin{minipage}{12em}
        \begin{gather*}
            S_n = \dfrac{a_1(1 - q^n)}{1 - q} \text{。}
        \end{gather*}
    \end{minipage}}
\end{center}

因为
$$ a_1 q^n = (a_1 q^{n - 1})q = a_n q \text{,}$$
所以上面的公式还可以写成
\begin{center}
    \framebox{\begin{minipage}{12em}
        \begin{gather*}
            S_n = \dfrac{a_1 - a_n q}{1 - q} \text{。}
        \end{gather*}
    \end{minipage}}
\end{center}

很明显,当 $q = 1$ 时,$S_n = n a_1$ 。

\liti 求等比数列 $\dfrac{1}{2}$,$\dfrac{1}{4}$,$\dfrac{1}{8}$,$\cdots$ 的前 $8$ 项的和。

\jie $\because \quad a_1 = \dfrac{1}{2},\; q = \dfrac{1}{2},\; n = 8,$

$\therefore \quad  S_8 = \dfrac{\dfrac{1}{2} \left[ 1 - \left( \dfrac{1}{2} \right)^8 \right]}{1 - \dfrac{1}{2}} = \dfrac{255}{256} \text{。} $


\liti 某制糖厂今年制糖 $5$ 万吨,如果平均每年的产量比上一年增加 $10\%$,那么从今年起,几年内可以使总产量达到 $30$ 万吨(保留到个位)?

\jie 由题意可知,这个糖厂从今年起,平均每年的产量(万吨)组成一个等比数列,记为 $\{a_n\}$ ,其中
$$ a_1 = 5, \quad q = 1 + 10\%  = 1.1, \quad S_n = 30 \text{,} $$
于是得到
$$ \dfrac{5(1 - 1.1^n)}{1 - 1.1} = 30 \text{。} $$

整理后,得
$$ 1.1^n = 1.6 \text{。} $$
两边取对数,得
$$ n\lg 1.1 = \lg 1.6 \text{。} $$
$\therefore \quad n = \dfrac{\lg 1.6}{\lg 1.1} = \dfrac{0.2041}{0.0414} \approx 5 \text{。}$

答:$5$ 年内可以使总产量达到 $30$ 万吨。


\liti 已知无穷数列
$$ 10^\frac{0}{5},\; 10^\frac{1}{5},\; 10^\frac{2}{5},\; \cdots,\; 10^\frac{n-1}{5},\; \cdots, $$
求证:

(1)这个数列是等比数列;

(2)这个数列中的任意一项是它后面第 $5$ 项的 $\dfrac{1}{10}$;

(3)这个数列中任意两项的积仍然在这个数列中。

\zhengming (1)这个数列中的第 $n$ 项与第 $n + 1$ 项分别是 $10^\frac{n-1}{5}$ 与 $10^\frac{n}{5}$ $(n \geqslant 1)$,
于是第 $n + 1$ 项与第 $n$ 项的比为
$$ \dfrac{10^\frac{n}{5}}{10^\frac{n-1}{5}} = 10^{\frac{n}{5} - \frac{n-1}{5}} = 10^\frac{1}{5} \text{,}$$
即它们的比值是常数 $10^\frac{1}{5}$ 。因此这个数列是以 $10^\frac{1}{5}$ 为公比的等比数列。

(2)这个数列第 $n$ 项与第 $n + 5$ 项分别是 $10^\frac{n-1}{5}$ 与 $10^\frac{n+4}{5}$ $(n \geqslant 1)$,于是
$$ \dfrac{10^\frac{n-1}{5}}{10^\frac{n+1}{5}} = 10^{\frac{n-1}{5} - \frac{n+4}{5}} = 10^{-\frac{5}{5}} = \dfrac{1}{10} \text{,}$$
这说明,这个数列中的任意一项经过 $5$ 次等比的递增以后,变大到它本身的 $10$ 倍。
例如,数列中第 $3$ 项是 $10^\frac{2}{5}$ ,第 $8$ 项就变大到 $10 \times 10^\frac{2}{5}$ 。

(3) 从这个数列中任意取出两项,假定它们分别是第 $n_1$ 项与第 $n_2$ 项, 即 $10^\frac{n_1 - 1}{5}$ 与 $10^\frac{n_2 - 1}{5}$,这里,$n_1,\; n_2 \in N$ ,于是
$$ 10^\frac{n_1 - 1}{5} \times 10^\frac{n_2 - 1}{5} = 10^{\frac{n_1 - 1}{5} + \frac{n_2 - 1}{5}} = 10^\frac{(n_1 + n_2 -1) - 1}{5} \text{。}$$

因为 $n_1 \geqslant 1$,$n_2 \geqslant 1$ ,所以 $n_1 + n_2 \geqslant 2$ ,即
$$ n_1 + n_2 - 1 \geqslant 1 \text{。}$$
又因为 $n_1,\; n_2 \in N$ ,所以 $n_1 + n_2 - 1 \in N$ 。这就证明 $10^\frac{(n_1 + n_2 -1) - 1}{5}$ 属于
上述数列,而且是数列的第 $n_1 + n_2 - 1$ 项。


\lianxi
\begin{xiaotis}
\setcounter{cntxiaoti}{0}

\xiaoti{根据下列各题中的条件,求相应的等比数列 $\{a_n\}$ 的 $S_n$ :}
\begin{xiaoxiaotis}

    \xiaoxiaoti{$a_1 = 3$,$q = 2$,$n = 6$;}

    \xiaoxiaoti{$a_1 = 2.4$,$q = -1.5$,$n = 5$;}

    \xiaoxiaoti{$a_1 = 8$,$q = \dfrac{1}{2}$,$n = 5$;}

    \xiaoxiaoti{$a_1 = -2.7$,$q = -\dfrac{1}{3}$,$n = 6$。}

\end{xiaoxiaotis}


\xiaoti{}
\begin{xiaoxiaotis}

    \vspace{-1.6em} \begin{minipage}{0.9\textwidth}
    \xiaoxiaoti{求等比数列 $1$,$2$,$4$,$\cdots$ 从第 $5$ 项到第 $10$ 项的和;}
    \end{minipage}

    \xiaoxiaoti{求等比数列 $\dfrac{3}{2}$,$\dfrac{3}{4}$,$\dfrac{3}{8}$,$\cdots$ 从第 $3$ 项到第 $7$ 项的和。}

\end{xiaoxiaotis}

\end{xiaotis}

