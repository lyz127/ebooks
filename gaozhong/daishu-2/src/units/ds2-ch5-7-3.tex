\subsubsection{开方}\label{subsec:5-7-3}

设 $\rho (\cos\phi + i\,\sin\phi)$ 是复数 $r (\cos\theta + i\,\sin\theta)$ 的 $n \; (n \in N)$ 次方根,那么
\begin{align*}
    r (\cos\theta + i\,\sin\theta) &= [\rho (\cos\phi + i\,\sin\phi)]^n \\
        &= \rho^n (\cos n\phi + i\,\sin n\phi) \text{。}
\end{align*}

因为相等的复数,它们的模相等,辐角可以相差 $2\pi$ 的整数倍,所以
$$\begin{cases}
    \rho^n = r, \\
    n\phi = \theta + 2k\pi \quad (k \in Z) \text{。}
\end{cases}$$

由此可知,
$$ \rho = \sqrt[n]{r},\quad \phi = \dfrac{\theta + 2k\pi}{n} \text{,}$$
因此 $r (\cos\theta + i\,\sin\theta)$ 的 $n$ 次方根是
$$ \sqrt[n]{r} \left( \cos\dfrac{\theta + 2k\pi}{n} + i\,\sin\dfrac{\theta + 2k\pi}{n} \right) \text{。}$$
当 $k$ 取 $0$,$1$,$\cdots$,$n - 1$ 各值时,就可以得到上式的 $n$ 个值。由于正弦、余弦函数
的周期都是 $2\pi$,当 $k$ 取 $n$,$n + 1$ 以及其他各个整数值时,又重复出现
$k$ 取 $0$,$1$,$\cdots$,$n - 1$ 时的结果。所以
\begin{center}
    \framebox{\begin{minipage}{18em}
        复数 $r (\cos\theta + i\,\sin\theta)$ 的 $n$ 次方根 \footnotemark 是 \\
        $\sqrt[n]{r} \left( \cos\dfrac{\theta + 2k\pi}{n} + i\,\sin\dfrac{\theta + 2k\pi}{n} \right)$  \\
        $(k = 0,\, 1, \, \cdots, \, n-1)$ 。
    \end{minipage}}
\end{center}
\footnotetext{有的书上用 $\sqrt[n]{z}$ 表示复数 $z$ 的 $n$ 次方根。采用这个符号时,一定要记住 $\sqrt[n]{z}$ 表示 $n$ 个复数。}
这就是说,\textbf{复数的 $n \; (n \in N)$ 次方根是 $n$ 个复数,它们的模都等于这个复数的模的 $n$ 次算术根,它们的辐角分别等于这个复数的辐角
与 $2\pi$ 的 $0$,$1$,$\cdots$,$n - 1$ 倍的和的 $n$ 分之一。}



\setcounter{cntliti}{5}
\liti 求 $1 - i$ 的立方根。

\jie 因为 $1 - i = \sqrt{2} \left( \cos\dfrac{7\pi}{4} + i \sin\dfrac{7\pi}{4} \right)$,
所以 $1 - i$ 的立方根是
\begin{align*}
    & \sqrt[6]{2} \left( \cos\dfrac{\dfrac{7\pi}{4} + 2k\pi}{3} + i\,\sin\dfrac{\dfrac{7\pi}{4} + 2k\pi}{3} \right) \\
    ={} & \sqrt[6]{2} \left( \cos\dfrac{7\pi + 8k\pi}{12} + i\,\sin\dfrac{7\pi + 8k\pi}{12} \right) \quad (k = 0,\, 1, \, 2) \text{,}
\end{align*}
即 $1 - i$ 的立方根是下面的三个复数:
\begin{align*}
    & \sqrt[6]{2} \left( \cos\dfrac{7\pi}{12} + i\,\sin\dfrac{7\pi}{12} \right) \text{,} \\
    & \sqrt[6]{2} \left( \cos\dfrac{5\pi}{4} + i\,\sin\dfrac{5\pi}{4} \right) \text{,} \\
    & \sqrt[6]{2} \left( \cos\dfrac{23\pi}{12} + i\,\sin\dfrac{23\pi}{12} \right) \text{。}
\end{align*}


\liti 设 $a \in R^+$,求 $-a$ 的平方根。

\jie 因为 $-a = a(\cos\pi + i\,\sin\pi)$,所以 $-a$ 的平方根是
$$ \sqrt{a} \left( \cos\dfrac{\pi + 2k\pi}{2} + i\,\sin\dfrac{\pi + 2k\pi}{2} \right) \quad (k = 0,\, 1) \text{,}$$
即 $-a$ 的平方根是下面两个复数:
$$
    \sqrt{a} \left( \cos\dfrac{\pi}{2} + i\,\sin\dfrac{\pi}{2} \right),\quad
    \sqrt{a} \left( \cos\dfrac{3\pi}{2} + i\,\sin\dfrac{3\pi}{2} \right) \text{,}
$$
或
$$ \sqrt{a}\,i, \quad -\sqrt{a}\,i \text{。} $$

从例 7 可以看到,\textbf{$a \in R^+$ 时 $-a$ 的平方根是 $\pm\sqrt{a}\,i$。}

我们知道,对于实系数一元二次方程 $ax^2 + bx + c = 0$,如果 $b^2 - 4ac < 0$,
那么它在实数集 $R$ 中没有根。现在我们在复数集 $C$ 中考察这种情况。经过变形,原方程可化为
$$ x^2 + \dfrac{b}{a} x = - \dfrac{c}{a} \text{,} $$

$\therefore \quad \begin{gathered}[t]
    x^2 + 2 \cdot x \cdot \dfrac{b}{2a} + \left( \dfrac{b}{2a} \right)^2 = \left( \dfrac{b}{2a} \right)^2 - \dfrac{c}{a}, \\
    \left( x + \dfrac{b}{2a} \right)^2 = \dfrac{b^2 - 4ac}{(2a)^2} , \\
    \left( x + \dfrac{b}{2a} \right)^2 = - \left[ \dfrac{-(b^2 - 4ac)}{(2a)^2} \right] \text{。}
\end{gathered}$

由于 $\dfrac{-(b^2 - 4ac)}{(2a)^2} \in R^+$,根据例 7,我们得到
$$ x + \dfrac{b}{2a} = \dfrac{\pm \sqrt{-(b^2 - 4ac)}\,i}{2a} \text{,} $$
\textbf{所以方程 $ax^2 + bx + c = 0$ 在复数集 $C$ 中有两个根
    $$ x = \dfrac{-b \pm \sqrt{-(b^2 - 4ac)}\,i}{2a} \quad (b^2 - 4ac < 0) \text{。} $$
}
显然,它们是一对共轭复数。



\liti 在复数集 $C$ 中解方程 $x^2 - 4x + 5 = 0$。

\jie 因为 $b^2 - 4ac = 16 - 20 = -4 < 0$,所以
$$ x = \dfrac{4 \pm \sqrt{4}\,i}{2} = \dfrac{4 \pm 2\,i}{2} = 2 \pm i \text{,} $$

根据以前学过的一元二次方程的有关知识,我们知道,例 8 中方程左边的二次三项式
$x^2 - 4x + 5$ 在复数集 $C$ 中就可以通过求根的方法分解成两个一次因式的积,即
\begin{align*}
    x^2 - 4x + 5 &= [x - (2 + i)] [x - (2 - i)] \\
        &= (x - 2 - i)(x - 2 + i) \text{。}
\end{align*}

形如 $a_n x^n + a_0 = 0 \; (a_0, \, a_n \in C \text{,且} a_n \neq 0)$ 的方程
叫做\textbf{二项方程}。任何一个二项方程都可以化成 $x^n = b \; (b \in C)$ 的形式,
因此,都可以通过复数开方来求根。




\liti 在复数集 $C$ 中解方程 $x^5 = 32$ 。

\jie 原方程就是
$$ x^5 = 32(\cos 0 + i\,\sin 0) \text{。}$$

$\therefore \quad \begin{aligned}[t]
    x &= \sqrt[5]{32} \left( \cos\dfrac{0 + 2k\pi}{5} + i\,\sin\dfrac{0 + 2k\pi}{5} \right) \\
      & = 2 \left[ \cos\left( k \cdot \dfrac{2\pi}{5} \right) + i\,\sin\left( k \cdot \dfrac{2\pi}{5} \right) \right] \quad (k = 0,\, 1,\, 2,\, 3,\, 4) \text{,}
\end{aligned}$\\
就是
\begin{align*}
    x_1 &= 2 (\cos 0 + i \sin 0) = 2 , \\
    x_2 &= 2 \left( \cos\dfrac{2\pi}{5} + i\,\sin\dfrac{2\pi}{5} \right), \\
    x_3 &= 2 \left( \cos\dfrac{4\pi}{5} + i\,\sin\dfrac{4\pi}{5} \right), \\
    x_4 &= 2 \left( \cos\dfrac{6\pi}{5} + i\,\sin\dfrac{6\pi}{5} \right), \\
    x_5 &= 2 \left( \cos\dfrac{8\pi}{5} + i\,\sin\dfrac{8\pi}{5} \right) \text{。}
\end{align*}

这个方程的根的几何意义是复平面内的五个点,这些点均匀分布在以原点为圆心、以 $2$ 为半径的圆上(图 \ref{fig:5-13}).

\begin{figure}[htbp]
    \centering
    \begin{tikzpicture}[>=Stealth, scale=0.8]
    \draw [->] (-3, 0) -- (3, 0) node[anchor=west] {$x$};
    \draw [->] (0, -3) -- (0, 3) node[anchor=east] {$y$};
    \node at (-0.2, -0.5) {$O$};

    \draw (1, 0.2) -- (1, 0) node[below] {$1$};
    \draw [thick] (0, 0) circle [radius = 2];
    \draw [fill] (2, 0) circle(2pt) node[anchor=north west] {$2$} node [anchor=south west] {$x_1$};
    \draw [thick, fill, rotate=75] (0, 0) -- (2, 0) circle(2pt) node [anchor=south] {$x_2$};
    \draw [thick, fill, rotate=150] (0, 0) -- (2, 0) circle(2pt) node [anchor=south east] {$x_3$};
    \draw [thick, fill, rotate=225] (0, 0) -- (2, 0) circle(2pt) node [anchor=north east] {$x_4$};
    \draw [thick, fill, rotate=290] (0, 0) -- (2, 0) circle(2pt) node [anchor=north west] {$x_5$};
\end{tikzpicture}

    \caption{}\label{fig:5-13}
\end{figure}


一般地,方程 $x^n = b \; (b \in C)$ 的根的几何意义是复平面内的 $n$ 个点,这些点均匀分布在以原点
为圆心、以 $\sqrt[n]{|b|}$ 为半径的圆上。


\lianxi
\begin{xiaotis}

\xiaoti{(口答)求下列各数的平方根:\\
    $-9$,$-2.89$,$-5$,$-t \; (t \in R^+)$,$-m^2 \; (m \in R)$,$a - b \; (a,\, b \in R\text{,且} a < b)$。
}

\xiaoti{在复数集 $C$ 中解下列方程:}
\begin{xiaoxiaotis}

    \renewcommand\arraystretch{1.2}
    \begin{tabular}[t]{*{2}{@{}p{16em}}}
        \xiaoxiaoti{$9x^2 + 16 = 0$;} & \xiaoxiaoti{$-3x^2 = 5$;} \\
        \xiaoxiaoti{$x^2 + x + 6 = 0$;} & \xiaoxiaoti{$18x^2 - 42x + 29 = 0$。}
    \end{tabular}

\end{xiaoxiaotis}

\xiaoti{求:}
\begin{xiaoxiaotis}

    \xiaoxiaoti{$-i$ 的平方根;}

    \xiaoxiaoti{$-\dfrac{1}{2} + \dfrac{\sqrt{3}}{2}\,i$ 的平方根;}

    \xiaoxiaoti{$1$ 的平方根;}

    \xiaoxiaoti{$-16$ 的四次方根。}

\end{xiaoxiaotis}


\xiaoti{在复数集 $C$ 中解下列方程,并在复平面内把方程的根表示出来:}
\begin{xiaoxiaotis}

    \renewcommand\arraystretch{1.2}
    \begin{tabular}[t]{*{2}{@{}p{16em}}}
        \xiaoxiaoti{$x^3 - 27 = 0$;} & \xiaoxiaoti{$x^3 + 1 = 0$;} \\
        \xiaoxiaoti{$x^4 - 16 = 0$;} & \xiaoxiaoti{$x^4 + 1 = 0$。}
    \end{tabular}

\end{xiaoxiaotis}

\end{xiaotis}

