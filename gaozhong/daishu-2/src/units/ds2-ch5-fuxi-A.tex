{\centering \nonumsubsection{A \hspace{1em} 组}}

\begin{xiaotis}

\xiaoti{求适合下列方程的 $x$ 与 $y \; (x,\, y \in R)$ 的值:}
\begin{xiaoxiaotis}

    \xiaoxiaoti{$(1 + 2\,i)x + (3 - 10\,i)y = 5 - 6i$;}

    \xiaoxiaoti{$x^2 + x\,i + 2 - 3\,i = y^2 + y\,i + 9 -2i$;}

    \xiaoxiaoti{$2x^2 - 5x + 3 + (y^2 + y - 6)\,i = 0$;}

    \xiaoxiaoti{$\dfrac{x}{1 - i} + \dfrac{y}{1 - 2\,i} = \dfrac{5}{1 - 3\,i}$。}

\end{xiaoxiaotis}



\xiaoti{判断下列各命题的真假,并说明理由:}
\begin{xiaoxiaotis}

    \xiaoxiaoti{如果让实数 $a$ 与线虚数 $a\,i$ 对应,那么实数集 $R$ 与纯虚数集一一对应;}

    \xiaoxiaoti{复数集 $C$ 与复平面内所有向量的集合一一对应。}

\end{xiaoxiaotis}



\xiaoti{计算:}
\begin{xiaoxiaotis}

    \xiaoxiaoti{$\dfrac{69 - 7\sqrt{15} + (\sqrt{3} - 6\sqrt{5})\,i}{3 - (\sqrt{3} - 3\sqrt{5})\,i}$;}

    \xiaoxiaoti{$[(\sqrt{3} + 1) + (\sqrt{3} - 1)\,i]^3$;}

    \xiaoxiaoti{$(x - 1 - \sqrt{2}\,i) (x - 1 + \sqrt{2}\,i) \cdot (x - 2 + \sqrt{3}\,i) (x - 2 - \sqrt{3}\,i)$。}

\end{xiaoxiaotis}



\xiaoti{已知复数 $z = x + y\,i \; (x,\, y \in R)$, 求下列各式的实部与虚部:}
\begin{xiaoxiaotis}

    \renewcommand\arraystretch{1.2}
    \begin{tabular}[t]{*{2}{@{}p{16em}}}
        \xiaoxiaoti{$z^2$;} & \xiaoxiaoti{$z^3$;} \\
        \xiaoxiaoti{$\dfrac{1}{z}$;} & \xiaoxiaoti{$V_0 z + \dfrac{M}{2\pi} \cdot \dfrac{1}{z} \; (V_0, \, M \in R)$。}
    \end{tabular}

\end{xiaoxiaotis}



\xiaoti{已知 $(x + y\,i)^3 = a + b\,i$ ,这里 $a,\, b,\, x,\, y \in R$,求证
    $$ \dfrac{a}{x} + \dfrac{b}{y} = 4(x^2 - y^2) \text{。} $$
}



\xiaoti{求证:}
\begin{xiaoxiaotis}

    \xiaoxiaoti{$(1 + i) (1 + \sqrt{3}\,i) (\cos\theta + i\, \sin\theta) = 2\sqrt{2} \left[ \cos\left( \dfrac{7\pi}{12} + \theta \right) + i\,\sin\left( \dfrac{7\pi}{12} + \theta \right) \right]$;}

    \xiaoxiaoti{$\dfrac{(1 - \sqrt{3}\,i) (\cos\theta + i\,\sin\theta)}{(1 - i) (\cos\theta - i\,\sin\theta)} = \sqrt{2} \left[ \cos\left( 2\theta - \dfrac{\pi}{12} \right) + i\,\sin\left( 2\theta - \dfrac{\pi}{12} \right) \right]$。}

\end{xiaoxiaotis}



\xiaoti{化简 $\dfrac{(\cos2\theta - i\,\sin2\theta) (\cos\phi + i\,\sin\phi)^2}{\cos(\theta + \phi) + i\,\sin(\theta + \phi)} \times \dfrac{(\cos2\theta + i\,\sin2\theta)^2 (\cos2\phi - i\,\sin2\phi)}{\cos(\theta - \phi) + i\,\sin(\theta - \phi)}$。}


\xiaoti{要把复数 $a (\cos\alpha + i\,\sin\alpha)$,$b (\cos\beta + i\,\sin\beta)$ 的和写成复数
$r(\cos\theta + i\,\sin\theta)$,应该怎样用 $a$,$b$,$\alpha$,$\beta$ 来表示 $r$,$\theta$?}



\xiaoti{设点 $Z$ 表示复数 $z$,在复平面内如何通过画图的方法,找出表示下列复数的点?}
\begin{xiaoxiaotis}

    \renewcommand\arraystretch{1.2}
    \begin{tabular}[t]{*{2}{@{}p{16em}}}
        \xiaoxiaoti{$z + (3 + 4\,i)$;} & \xiaoxiaoti{$0.2z$;} \\
        \xiaoxiaoti{$-\sqrt{2}z$;} & \xiaoxiaoti{$z(\cos 60^\circ + i\,\sin 60^\circ)$;} \\
        \xiaoxiaoti{$-i\,z$;} & \xiaoxiaoti{$\dfrac{a^2}{z} \quad (a \in R^+)$。}
    \end{tabular}

\end{xiaoxiaotis}


\xiaoti{已知 $n \in N$,求证:}
\begin{xiaoxiaotis}

    \xiaoxiaoti{$i^n + i^{n+1} + i^{n+2} + i^{n+3} = 0$;}

    \xiaoxiaoti{$\left( \dfrac{1 + i}{1 - i} \right)^{2n}$
        当 $n$ 是偶数时为 $1$,当 $n$ 是奇数时为 $-1$;}

    \xiaoxiaoti{$\left( -\dfrac{1}{2} + \dfrac{\sqrt{3}}{2}\,i \right)^n + \left( -\dfrac{1}{2} - \dfrac{\sqrt{3}}{2}\,i \right)^n$
        当 $n$ 是 $3$ 的倍数时为 $2$,当 $n$ 不是 $3$ 的倍数时为 $-1$。}

\end{xiaoxiaotis}



\xiaoti{在复数集 $C$ 中分解因式:}
\begin{xiaoxiaotis}

    \renewcommand\arraystretch{1.2}
    \begin{tabular}[t]{*{2}{@{}p{16em}}}
        \xiaoxiaoti{$x^2 + 5$;} & \xiaoxiaoti{$2x^2 - 6x + 5$;} \\
        \xiaoxiaoti{$x^2 - 2x\cos\alpha + 1$;} & \xiaoxiaoti{$x^6 - 1$。}
    \end{tabular}

\end{xiaoxiaotis}


\xiaoti{解下列方程:}
\begin{xiaoxiaotis}

    \renewcommand\arraystretch{1.2}
    \begin{tabular}[t]{*{2}{@{}p{16em}}}
        \xiaoxiaoti{$x^4 + 24\,i = 0$;} & \xiaoxiaoti{$(x + 1)^9 = (1 + i)^9$。}
    \end{tabular}

\end{xiaoxiaotis}


\xiaoti{设 $a + b\,i,\, c + d\,i \in C$,下列命题成立的充要条件是什么?}
\begin{xiaoxiaotis}

    \xiaoxiaoti{$(a + b\,i) + (c + d\,i) \in R$;}

    \xiaoxiaoti{$(a + b\,i) + (c + d\,i)$ 是纯虚数;}

    \xiaoxiaoti{$(a + b\,i) (c + d\,i) \in R$;}

    \xiaoxiaoti{$(a + b\,i) (c + d\,i)$ 是纯虚数;}

    \xiaoxiaoti{$\dfrac{a + b\,i}{c + d\,i} \in R$;}

    \xiaoxiaoti{$\dfrac{a + b\,i}{c + d\,i}$ 是纯虚数。}

\end{xiaoxiaotis}



\xiaoti{已知 $z$ 是虚数,解下列方程:}
\begin{xiaoxiaotis}

    \renewcommand\arraystretch{1.2}
    \begin{tabular}[t]{*{2}{@{}p{16em}}}
        \xiaoxiaoti{$z + |\overline{z}| = 2 + i$;} & \xiaoxiaoti{$z^2 = \overline{z}$。}
    \end{tabular}

\end{xiaoxiaotis}


\xiaoti{求证 $|z| = 1 \; (z \in C)$ 的充要条件是 $\dfrac{1}{z} = \overline{z}$。}


\xiaoti{求证:}
\begin{xiaoxiaotis}

    \xiaoxiaoti{共轭复数的 $n \; (n \in N)$ 次幂仍是共轭复数;}

    \xiaoxiaoti{虚数的平方根仍是虚数。}

\end{xiaoxiaotis}

\end{xiaotis}


