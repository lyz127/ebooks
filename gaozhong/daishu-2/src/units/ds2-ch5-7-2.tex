\subsubsection{除法}\label{subsec:5-7-2}

设 $z_1 = r_1 (\cos\theta_1 + i\,\sin\theta_1)$,$z_2 = r_2 (\cos\theta_2 + i\,\sin\theta_2)$,且 $z_2 \neq 0$。因为
$$ r_2 (\cos\theta_2 + i\,\sin\theta_2) \cdot \dfrac{r_1}{r_2} [\cos(\theta_1 - \theta_2) + i\,\sin(\theta_1 - \theta_2)] = r_1(\cos\theta_1 + i\,\sin\theta_1) \text{,} $$
所以根据复数的除法的定义,有
\begin{center}
    \framebox{\begin{minipage}{28em}
        \begin{gather*}
            \dfrac{r_1 (\cos\theta_1 + i\,\sin\theta_1)}{r_2 (\cos\theta_2 + i\,\sin\theta_2)} = \dfrac{r_1}{r_2} [\cos(\theta_1 - \theta_2) + i\,\sin(\theta_1 - \theta_2)] \text{。}
        \end{gather*}
    \end{minipage}}
\end{center}
这就是说,\textbf{两个复数相除,商的模等于被除数的模除以除数的模所得的商,商的辐角等于被除数的辐角减去除数的辐角所得的差。}


\setcounter{cntliti}{4}
\liti 计算
$$ 4 \left( \cos\dfrac{4\pi}{3} + i\,\sin\dfrac{4\pi}{3} \right) \div 2 \left( \cos\dfrac{5\pi}{6} + i\,\sin\dfrac{5\pi}{6} \right) \text{。}$$

\jie $\begin{aligned}[t]
    \text{原式} &= \dfrac{4 \left( \cos\dfrac{4\pi}{3} + i\,\sin\dfrac{4\pi}{3} \right)}{2 \left( \cos\dfrac{5\pi}{6} + i\,\sin\dfrac{5\pi}{6} \right)} \\
        &= 2 \left[ \cos\left( \dfrac{4\pi}{3} - \dfrac{5\pi}{6} \right) + i\,\sin\left( \dfrac{4\pi}{3} - \dfrac{5\pi}{6} \right) \right] \\
        &= 2 \left[ \cos\dfrac{\pi}{2} + i\,\sin\dfrac{\pi}{2} \right] = 2(0 + i) = 2\,i \text{。}
\end{aligned}$



\lianxi
\begin{xiaotis}


\xiaoti{计算:}
\begin{xiaoxiaotis}

    \xiaoxiaoti{$12 \left( \cos\dfrac{7\pi}{4} + i\,\sin\dfrac{7\pi}{4} \right) \div 6 \left( \cos\dfrac{2\pi}{3} + i\,\sin\dfrac{2\pi}{3} \right)$;}

    \xiaoxiaoti{$\sqrt{3}(\cos 150^\circ + i\,\sin 150^\circ) \div \sqrt{2}(\cos 225^\circ + i\,\sin 225^\circ)$;}

    \xiaoxiaoti{$2 \div \left( \cos\dfrac{\pi}{4} + i\,\sin\dfrac{\pi}{4} \right)$;}

    \xiaoxiaoti{$-i \div 2 (\cos 120^\circ + i\,\sin 120^\circ)$。}

\end{xiaoxiaotis}


\xiaoti{复数除法的几何意义是什么?}

\end{xiaotis}

